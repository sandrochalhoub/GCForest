\documentclass{article}
\usepackage{neurips_2021}

\usepackage[usenames,dvipsnames,svgnames,table]{xcolor}%% http://ctan.org/pkg/xcolor
\usepackage[utf8]{inputenc}
\usepackage[T1]{fontenc}

\usepackage{hyperref}       % hyperlinks
\usepackage{url}            % simple URL typesetting
\usepackage{booktabs}       % professional-quality tables
\usepackage{amsfonts}       % blackboard math symbols
\usepackage{nicefrac}       % compact symbols for 1/2, etc.
\usepackage{microtype}      % microtypography
\usepackage{xcolor}         % colors

\usepackage{xspace}
\usepackage{array}
%\usepackage{amsthm}
\usepackage{amsmath} 
\usepackage{amssymb} 
\usepackage[ruled,vlined]{algorithm2e}
\usepackage{multirow}
\usepackage{tikz}
\usetikzlibrary{arrows,shadows,fit,calc,positioning,decorations.pathreplacing,matrix,shapes,petri,topaths,fadings,mindmap,backgrounds,shapes.geometric}

\usepackage{pgfplots}
\usepackage{fp}
\usepackage{subfig}

% \usepackage{geometry}p
\usepackage{xifthen}
\usepackage{rotating}
\usepackage{forest}
\usepackage{relsize}

\newtheorem{theorem}{Theorem}
\newtheorem{example}{Example}
\newenvironment{proof}{\paragraph{Proof:}}{\hfill$\square$}


\input{src/macros.tex}
	

\DontPrintSemicolon

\title{A Simple and Efficient Anytime Algorithm for Computing Optimal Decision Trees}


\author{%
  Emir Demirovi\'c \\
  TU DELFT \\
	The Netherlands \\
  \texttt{e.demirovic@tudelft.nl} \\
	\And
	Emmanuel Hebrard \\
	LAAS-CNRS \\
	Universit\'e de Toulouse, CNRS \\
	France \\
	\texttt{hebrard@laas.fr} \\
	\And
	Louis Jean \\
	LAAS-CNRS \\
	Universit\'e de Toulouse, CNRS \\
	France \\
	\texttt{ljean@laas.fr} \\
}


\begin{document}


\maketitle






\begin{abstract}
	In this paper we introduce a relatively {simple} algorithm to learn optimal decision trees of bounded depth. This algorithm, \budalg, is as memory and time efficient as heuristics, and yet more efficient than most exact methods on most data sets. State-of-the-art exact methods often have poor anytime behavior, and hardly scale to deep trees. Experiments show that they are typically orders of magnitude slower than the proposed algorithm to compute optimally accurate classifiers of a given depth.
On the other hand, \budalg\ finds, without significant computational overhead, solutions comparable to those returned by standard greedy heuristics, and can quickly improve their accuracy when given more computation time.
% the first solution found by \budalg\ is comparable to those found by standard greedy heuristics and that significantly improve upon greedy heuristics. On the 
\end{abstract}



\section{Introduction}

In conclusion of their short paper showing that computing decision trees of maximum accuracy is NP-complete, Hyafil and Rivest stated: ``Accordingly, it is to be expected that that good heuristics for constructing near-optimal binary decision trees will be the best solution to this problem in the near future.''~\cite{NPhardTrees}. Indeed, heuristic approaches such as \cart\cite{breiman1984classification}, \idthree~\cite{10.1023/A:1022643204877} or \cfour~\cite{c4-5} have been prevalent long afterward, and are still vastly more commonly used in practice than exact approaches.

%\medskip

It is well established, however, that optimal trees (for some combination of accuracy, depth and size) generalize better to unseen data.
% than heuristic trees. 
This experiment has been confirmed in several publications\footnote{Hence we shall not reproduce once again such experiments in this paper.}, in particular for the objective criterion considered in this paper: maximizing the accuracy given an upper bound on the depth~\cite{avellanedaefficient,bertsimas2017optimal,bertsimas2007classification,DBLP:conf/ijcai/Hu0HH20,DBLP:journals/corr/abs-2007-12652,dl8}. 
Another valuable feature of smaller and/or shallower trees is that interpreting or explaining their prediction is comparatively easier, which is often valuable. 

%\medskip

Despite these desirable features, exact methods have not been widely adopted yet for a simple reason: they do not scale. There has been a significant progress lately, and the most recent approaches show very promising results. However, no exact method can be considered consistantly better than heuristics. 
For SAT~\cite{avellanedaefficient,narodytska2018learning} and Integer Programming approaches~\cite{aghaei2020learning,bertsimas2017optimal,bertsimas2007classification,verwer2019learning}, the size of the encoding is a first hurdle. All these models require a number of variables at least proportional to the size of the tree and to the number of datapoints. As a result, scaling beyond a few thousands datapoints is difficult. 
On the other hand, dynamic programming algorithms \olddleight~\cite{dl8} and \dleight~\cite{dl85} scale very well to large data sets. Moreover, these algorithms leverage branch independence: sibling subtrees can be optimized independently, which as a great impact on computational complexity. However, \dleight tends to be memory hungry and furthermore, is not anytime.
The constraint programming approach of Verhaeghe \textit{et al.} emulates these positive features using dedicated propagation algorithms and search strategies~\cite{verhaeghe2019learning}, while being potentially anytime, although it does not quite match \dleight's efficiency.
Finally, a recently introduced algorithm, \murtree~\cite{DBLP:journals/corr/abs-2007-12652}, improves on earlier dynamic programming in several ways. As a result, it clearly dominates previous exact methods. It is more memory efficient, orders of magnitude faster than \dleight, and has a better anytime behavior. However, our experimental results show that for deeper trees, none of these methods can reliably outperform heuristics.
 

% \medskip

In this paper we introduce a relatively \emph{simple} algorithm (\budalg), that is as memory and time efficient as heuristics, and yet more efficient than most exact methods on most data sets. 
This algorithm can be seen as an instance of the more general framework introduced in \cite{DBLP:journals/corr/abs-2007-12652}, however tuned to have the best scalability to large trees and the best anytime behavior as possible. 
%As a result, it is comparable to \murtree on shallow trees, while clearly outperforming the state of the art on deep trees.

In a nutshell, \budalg emulates the dynamic programming algorithm \dleight~\cite{dl8}, while always expanding non-terminal branches (a.k.a ``buds'') before optimizing grown branches. As a result, this algorithm is in a sense strictly better than both the standard dynamic programming approach (because it is anytime and at least as fast) and than classic heuristics (because it emulates them during search, without significant overhead).
%but explores the search space so as to improve its anytime behaviour.
Our experimental results show that it outperforms the state of the art, to the exception of \murtree on relatively shallow trees (typically for maximum depth up to 4), for which its more sophisticated (albeit more complex) algorithmic features can pay off.
In particular, on data sets that \dleight can tackle, \budalg can always find classifiers at least as accurate faster, and when the former can prove optimality, the latter does it orders of magnitude faster.




% Therefore, we shall not reproduce once again such experiments and only consider efficiency with respect to the primary objective.
% Other criteria have been used, for instance a number of approaches have considered the problem of computing a perfectly accurate tree of minimal size and/or depth~\cite{DBLP:conf/cp/BessiereHO09,narodytska2018learning}. This criterion is usually considered less useful because it is not robust to noisy data sets and is prone to overfitting. It is possible to adapt these approaches to accuracy maximization under a fixed depth~\cite{DBLP:conf/ijcai/Hu0HH20}, although the resulting method is not extremely efficient.
% Alternatively, Hu \textit{et al.} advocate a combination of accuracy and tree size~\cite{hu2019optimal}. We leave more complex criteria for future work as the main focus of this paper is to improve the state of the art for the most commonly used objective.
% %Alternatively, a number of approaches have considered the problem of computing a perfectly accurate tree of minimal size and/or depth
%
%
% %The idea of computing optimal decision tree classifiers is very old. The problem being
% %Ever since the pionneering work on decision tree classifiers~\cite{breiman1984classification,10.1023/A:1022643204877}, the question of computing \emph{optimal} decision trees has been alive, and recently a number of algorithms have been introduced for that purpose.
% Several variants and criteria have been proposed. %, e.g. computing the perfect classifier of minimum size~\cite{},
% Among those, computing a depth-bounded decision tree of maximum accuracy is often the preferred criterion~\cite{bertsimas2017optimal,hu2019optimal,dl8,verhaeghe2019learning}, because while being straightforward, it captures many desired features: it is resilient to noisy data, and shallow trees are both easier to explain and less prone to overfitting.
%
% \medskip
%
% Despite the vast offer of exact methods that can potentially provide optimal decision trees, or at least should, given enough time, improve on heuristics, the latter are still vastly more commonly used in practice. Some methods are memory-hungry, some are not anytime, and as far as we know there is not a single algorithm that can provide optimal classifiers while scaling to large data sets, feature space, or tree depth as heuristics do.
% % \begin{itemize}
% % 	\item MinDT -> does not scale, problem with noise
% % 	\item other SAT approaches -> do not scale in some way or another
% % 	\item DL8 -> does not scale in memory, slower
% % 	\item BinOCT -> ? (probably much slower)
% %   \item Murtree -> by far the most efficient, however, does not scale as well on deep trees and not as anytime
% % \end{itemize}
%
%
%
% \medskip
%
%
%
% %
% %
% % \begin{itemize}
% % 	\item Same worst-case complexity than DL8
% % 	\item No memory usage
% % 	\item Better anytime behaviour than DL8 (in fact as good as the state of the art heuristics)
% % 	\item Therefore, strictly better than greedy heuristics and almost always better than DL8
% % \end{itemize}
% %
% %
% % We consider the problem of finding the bounded-depth decision tree of maximum accuracy.
% % The state of the art includes MIP approaches (BinOCT), a MaxSAT approach based on the SAT encoding proposed by Narodytska et al, and DL8.5.
% % The latter algorithm is by far the most efficient, however, it is not \emph{anytime}: the left branch must be optimally solved before a solution of the right branch can be found. Moreover the use of a cache structure means that it uses a lot of memory. This algorithm is practical for a maximum depth of 4 (although using gigabytes of memory) but often not much beyond. Therefore, in a number of cases, a greedy heuristic (such as CART) is still the best method in practice.
% %
% % \medskip
%
% % In this note we introduce what is essentially an anytime version of DL8.5, without cache.
% % This algorithm therefore uses linear (in the size of the tree) memory and anytime, hence in principle strictly better than CART. Moreover, on instance where DL8.5 can find a solution, the algorithm described in this note is significantly faster (by about a factor 10).

\section{Preliminaries}

A data set on a binary feature set \features is a pair $\langle \negex,\posex \rangle$ where $\negex$ and $\posex$ are subsets of the feature space $\prod_{\afeat \in \features}\{\afeat,\bar{\afeat}\}$, and are standing, respectively, for negative and positive datapoints.
% $2^{\features}$.
% It is associated a label function $\classlabel : 2^{\features} \mapsto \{\posclass,\negclass\}$ such that:
% $$
% \forall \aclass \in \{\posclass,\negclass\}, \forall \ex \in \setex{\aclass}, \classlabel[\ex] = \aclass
% $$
%We denote $\allex$ the 
We denote the union of all datapoints by $\allex = \negex \cup \posex$ and $\bar{\features} = \{\bar{\afeat} \mid \afeat \in \features\}$ the set of negated features.
% %$\posclass$ and $\negclass$ are class labels and
% A datapoint $\ex$ can equivalently be seen as a subset of $\features$, or as the conjunction:
% $$
% \bigwedge_{\afeat \in \ex}\afeat \wedge \bigwedge_{\afeat \in \features \setminus \ex}\bar{\afeat}
% $$


A \emph{binary decision tree} is a tree whose 
%leaves are labelled with either $\posclass$ or $\negclass$, 
internal vertices are labelled with features and the two edges exiting a node labelled with $\afeat$ are respectively labelled with the feature $\afeat$ and its negation $\bar{\afeat}$.
To a \emph{branch} of a decision tree we associate the ordered set of labels on its edges, from root to leaf.

%$\abranch \in 2^{\features \cup \bar{\features}}$ of labels on its edges. Moreover, \abranch\ is naturally ordered from root to leaf.

%can therefore be identified to the ordered set of labels on the corresponding edges.
%To a \emph{branch} of a decision tree we can associate the set of labels on the corresponding edges.
%conjunction of features which are on the path from the root to a leaf. If the feature vertex is exited by a $1$-edge, the feature is positive in the conjunction, otherwise it is negative.

 
%Moreover, if we also consider data points as conjunctions of features (where every feature appears either positively or negatively),
%given a branch $\abranch \subseteq \features$ 
Given a data set $\langle \negex,\posex \rangle$, we can associate a data set $\langle \negex[\abranch],\posex[\abranch] \rangle$ to a branch $\abranch$ where $\negex[\abranch] = \{\ex \mid \ex \in \negex, \abranch \subseteq \ex\}$ and $\posex[\abranch] = \{\ex \mid \ex \in \posex, \abranch \subseteq \ex\}$.
% \begin{eqnarray*}
% \negex[\abranch] = \{\ex \mid \ex \in \negex, \abranch \subseteq \ex\}\\
% \posex[\abranch] = \{\ex \mid \ex \in \posex, \abranch \subseteq \ex\}
% \end{eqnarray*}
%
% For instance, the branch $\abranch = \{\afeat_i, \bar{\afeat_j}, \bar{\afeat_k}, \afeat_l\}$ has length 4 and
%
We write $\grow{\abranch}{\afeat}$ as a shortcut for $\abranch \cup \{\afeat\}$.
The classification error for branch $\abranch$ is $\error[\abranch]=\min(|\negex[\abranch]|, |\posex[\abranch]|)$.
% %Let $\error[\abranch]$ be $\min(|\negex[\abranch]|, |\posex[\abranch]|)$,
% and we write $\error[\abranch,\afeat]$ for $\error[\grow{\abranch}{\afeat}] + \error[\grow{\abranch}{\bar{\afeat}}]$.
% %A branch \abranch\ is said \emph{pure} iff $\error[\abranch]=0$.

% \medskip

Given a binary data set $\langle \negex,\posex \rangle$ on the features \features, %with label function $\classlabel$,
the \emph{minimum error bounded depth decision tree problem} consists in finding a binary decision tree with vertex labels in \features\ whose branches have cardinality at most $\mdepth$ and the sum of the classification error %($\error[\abranch]$) 
of its maximal branches is minimum.
%of depth at most $\maxd$ whose sum of error $\error[\abranch]$ for all branches $\abranch$ from the root to the leaves is equal to $\epsilon$.




\subsection{Dynamic Programming Algorithm}

The solver DL8.5 is a dynamic programming algorithm for the minimum error bounded depth decision tree problem. It relies on the observation that given a feature test, the two resulting branches are independent subproblems. Algorithm~\ref{alg:dynprog} gives a high level view of DL8.5.


	% \begin{algorithm}
	% 	\caption{Dynamic Programming Algorithm\label{alg:dynprog}}
	% 	\TitleOfAlgo{\dynprog}
	% 	  \KwData{$\negex,\posex,\maxd,\abranch[=(\tru)]$}
	% 	  \KwResult{The minimum error on $\negex,\posex$ for decision trees of depth at most $\maxd$}
	% 		\lIf{$\maxd = 0$ or $\error[\abranch] = 0$} {
	% 		\Return $\error[\abranch]$
	% 		}
	% 		$\best \gets \error[\abranch]$\;
	% 		\ForEach{$\afeat \in \features \setminus \abranch$} {
	% 				$\best \gets \min(\best, \dynprog(\negex,\posex, \grow{\abranch}{\afeat}, \maxd-1) + \dynprog(\negex,\posex, \grow{\abranch}{\bar{\afeat}}, \maxd-1))$\;
	% 		}
	% 		\Return $\best$\;
	% \end{algorithm}
	
	\begin{algorithm}
		\begin{footnotesize}
		\caption{Dynamic Programming Algorithm\label{alg:dynprog}}
		\TitleOfAlgo{\dynprog}
		  \KwData{$\negex,\posex,\features,\maxd$}
		  \KwResult{The minimum error on $\negex,\posex,\features$ for decision trees of depth $\maxd$}
			$\error \gets \min(|\negex|,|\posex|)$\;
			
			\If{$\maxd > 0$ and $\error > 0$} {
			\ForEach{$\afeat \in \features$} {
					$\error \gets \min(\error, \dynprog(\negex(\{\afeat\}),\posex(\{\afeat\}),\features \setminus \{\afeat\},\maxd-1)$\;
					\ \ \ \ \ \ \ \ \ \ \ \ \ \ \ \ \ \ \ \ \ \ \ \ \ \ \ \ \ \ \ \ \  $ + \dynprog(\negex(\{\bar{\afeat}\}),\posex(\{\bar{\afeat}\}),\features \setminus \{\afeat\},\maxd-1))$\;
			}
			}
			\Return $\error$\;
			\end{footnotesize}
	\end{algorithm}
	
	
	% \end{document}
	
	
	Let $\numex = |\posex| + |\negex|$, $\numfeat = |\features|$, and let $\mdepth$ be the maximum depth.
	We can safely assume $\mdepth \leq \numfeat$ (otherwise all features can be tested on every branch) and $\mdepth \leq \log \numex$ (otherwise we could have one distinct branch per datapoint), and it is often assumed that
	 $\mdepth \ll \numfeat$ and $\mdepth \ll \log \numex$. 
	 
	 
	 Algorithm~\ref{alg:dynprog} explores, in the worst case, $2^{\mdepth}$ branches for each of the $\Perm{\numfeat}{\mdepth}$ permutations of $\mdepth$ features in $\features$ for a total of $\Theta(\Perm{\numfeat}{\mdepth}2^{\mdepth})$ recursive calls.\footnote{More than half of the calls are at depth $\mdepth$ so counting leaves is sufficient.} 
	% 
	 Note that this is a significant improvement with respect to the $\Theta(\numfeat^{2^{\mdepth}})$ trees (with redundant branches) explored by a brute-force algorithm.
	 
	 Moreover, at each call, the data set must be split into two subsets. This takes time linear in the size of the data set. However,  %it can be amortized over the $2^{\mdepth-1}$ branches of depth $\mdepth$.
	 Consider
	 an ordered set of $\mdepth$ features and the $2^{\mdepth-1}$ branches testing these features in that order. Their corresponding data sets form a partition of the original data set. Therefore, the $\mdepth$-th split can done in $\Theta(\numex)$ time amortized over these $2^{\mdepth-1}$ branches.
	 It follows that the overall time complexity for splitting the data set is in $\Theta(\Perm{\numfeat}{\mdepth}\numex)$.\footnote{Again, more than half of the splits are at depth $\mdepth$.}
	 
	 
		 %
	 % for a ``level'' $l \in [1,\mdepth]$, the $2^l$ branches of length $l$ and testing the same feature in the same order (e.g., the branches $\{a,b\}$ Figure~\ref{fig:searchtree}).
	 % The data sets associated to these branches form a partition of the original data set. Therefore, the $l$-th split is done in $\Theta(\numex)$ time amortized over these $2^l$ branches.
	 % It follows that the overall time complexity for splitting the data set is in $\Theta(\Perm{\numfeat}{\mdepth}\numex)$.
	 
	 Algorithm~\ref{alg:dynprog} therefore runs in $\Theta((\numex+2^\mdepth){\Perm{\numfeat}{\mdepth}}) \subset O((\numex + 2^{\mdepth})\numfeat^\mdepth)$ time. Moreover, with the above assumptions on $\numex,\numfeat$ and $\mdepth$, $O(\numex\numfeat^\mdepth)$ is a good approximation of its worst-case time complexity.
	 
	 
	 % \medskip
	 
	 Notice that computing an optimal decision tree of bounded depth (for a polynomially computable definition of ``optimal'') is therefore polynomial unless the maximum depth is an input.
	 
	 
	 
	%  However, for a given level $l \in [1,\mdepth]$, this splitting procedure can be done in time $\Theta(\numex)$ amortised over the $2^l$ branches
	%
	%  %a given level of a particular decision tree,
	%  since the data sets associated to branches up to a given level form a partition of the original data set.
	%
	%
	%
	% %The number of recursive calls for Algorithm~\ref{alg:dynprog} is $\Theta(2^{\mdepth-1}{\numfeat \choose \mdepth})$, to explore at most ${\numfeat \choose \mdepth}$ combinations of features for at most $2^{\mdepth-1}$ branches.
	% %The number of recursive calls for Algorithm~\ref{alg:dynprog} is $\Theta(2^{\mdepth}{\numfeat \choose \mdepth})$, to explore $\Theta({\numfeat \choose \mdepth})$ combinations of features for $\Theta(2^{\mdepth})$ branches.
	% The number of recursive calls for Algorithm~\ref{alg:dynprog} is $\Theta(2^{\mdepth}\numfeat^{\mdepth})$.
	% Moreover, at each call, the data set must be split into two subsets. However, this splitting procedure can be done in time $\Theta(\numex)$ amortised over a given level of a particular decision tree, since the data sets associated to branches up to a given level form a partition of the original data set.
	% Therefore, since Algorithm~\ref{alg:dynprog} independently explores $\Theta({\numfeat \choose \mdepth})$ sets of branches of depth $\mdepth$, it runs in
	% $\Theta((\numex+2^\mdepth){\numfeat \choose \mdepth})$ time (hence $O(\numex\numfeat^\mdepth)$ time, since we suppose $\mdepth \ll \numfeat$ and $\mdepth \ll \log \numex$).
	%
	% Note that this is a significant improvement with respect to the $\Theta(\numfeat^{2^{\mdepth}})$ trees (with redundant branches) explored by a brute-force algorithm.
	%  %
	%  % a brute-force algorithm, since the total number of decision trees of depth $\mdepth$ with $\numfeat$ attributes is $\prod_{x=1}^{\mdepth}(1-x+\numfeat)^{2^{\mdepth}}$. If we assume $\mdepth \in O(1)$, this is $\Theta(\numfeat^{2^{\mdepth}})$ distinct trees.
	%  %




\section{An Anytime Algorithm}

Algorithm~\ref{alg:bud} shows the pseudo-code of an iterative, anytime, version of Algorithm~\ref{alg:dynprog} (highlighted code can be ignored for now). This algorithm
%In a nutshell, Algorithm~\ref{alg:bud} 
explores the same search space as Algorithm~\ref{alg:dynprog}: the same branch is never explored twice. However, incomplete branches are expanded before trying alternative features for already explored branches. In other words, instead of optimizing the left subtree before exploring the right subtree as in Algorithm~\ref{alg:dynprog}, Algorithm~\ref{alg:bud} first fully expands a decision tree before exploring alternatives for any branch, see Figure~\ref{fig:searchtree} for an illustration of the branch exploration order.

%For instance, consider a data set with three binary features $\features = \{a,b,c\}$. Figure~\ref{fig:searchtree} shows the branches explored by both Algorithm~\ref{dynprog} and Algorithm~\ref{alg:bud}. Both algorithms explore first the branches $\{a,b\}$ and $\{a,\bar{b}\}$. However, whereas Algorithm~\ref{dynprog} explores next the branches $\{a,c\}$ and $\{a,\bar{c}\}$, Algorithm~\ref{alg:bud} explores next the branches $\{\bar{a},b\}$ and $\{\bar{a},\bar{b}\}$, hence immediately 


%For $\mdepth=2$ Algorithm~\ref{dynprog} explores the branches 


	\begin{figure}
	\begin{center}
		\tabcolsep=0pt
		\scalebox{1}{
			\begin{forest}
				for tree={%
					l sep=20pt,
					s sep=3.5pt,
					node options={shape=rectangle, minimum width=10pt, inner sep=0pt, font=\footnotesize},
		  		edge={thick, -latex, shorten >=1pt, shorten <=1pt},
				}
				[{$\emptyset$}
					[{$a$}
						[{\begin{tabular}{c}$b$\\1\end{tabular}}]
						[{\begin{tabular}{c}$\bar{b}$\\2\end{tabular}}]
						[{\begin{tabular}{c}$c$\\7\end{tabular}}]
						[{\begin{tabular}{c}$\bar{c}$\\8\end{tabular}}]
					]
					[{$\bar{a}$}
						[{\begin{tabular}{c}$b$\\3\end{tabular}}]
						[{\begin{tabular}{c}$\bar{b}$\\4\end{tabular}}]
						[{\begin{tabular}{c}$c$\\5\end{tabular}}]
						[{\begin{tabular}{c}$\bar{c}$\\6\end{tabular}}]
					]
					[{$b$}
						[{\begin{tabular}{c}$a$\\9\end{tabular}}]
						[{\begin{tabular}{c}$\bar{a}$\\10\end{tabular}}]
						[{\begin{tabular}{c}$c$\\15\end{tabular}}]
						[{\begin{tabular}{c}$\bar{c}$\\16\end{tabular}}]
					]
					[{$\bar{b}$}
						[{\begin{tabular}{c}$a$\\11\end{tabular}}]
						[{\begin{tabular}{c}$\bar{a}$\\12\end{tabular}}]
						[{\begin{tabular}{c}$c$\\13\end{tabular}}]
						[{\begin{tabular}{c}$\bar{c}$\\14\end{tabular}}]
					]
					[{$c$}
						[{\begin{tabular}{c}$a$\\17\end{tabular}}]
						[{\begin{tabular}{c}$\bar{a}$\\18\end{tabular}}]
						[{\begin{tabular}{c}$b$\\23\end{tabular}}]
						[{\begin{tabular}{c}$\bar{b}$\\24\end{tabular}}]
					]
					[{$\bar{c}$}
						[{\begin{tabular}{c}$a$\\19\end{tabular}}]
						[{\begin{tabular}{c}$\bar{a}$\\20\end{tabular}}]
						[{\begin{tabular}{c}$b$\\21\end{tabular}}]
						[{\begin{tabular}{c}$\bar{b}$\\22\end{tabular}}]
					]
				]
			\end{forest}
		}
	\end{center}
	\caption{\label{fig:searchtree} The search tree for decision trees. \texttt{DynProg} explores it depth first, whereas \texttt{Bud-first-search} explores the leaves in the order given below.}
	%\caption{\label{fig:searchtree} The search tree for decision trees. \dynprog explores it depth first, whereas \budalg explores branches in the order given below the leaves.}
	\end{figure}


% \medskip

Let $\afeat_i <_{\abranch} \afeat_j$ if and only if feature $\afeat_i$ is selected before feature $\afeat_j$ when expanding branch $\abranch$ at Line~\ref{line:assignment}. Algorithm~\ref{alg:bud} has the following invariants, from which a formal proof of correctness easily follows:

\begin{itemize}
	\item \sequence\ represents the current decision tree: if $(\abranch,\afeat) \in \sequence$, then the current tree tests feature $\afeat$ at the extremity of branch $\abranch$. We say that the branch $\abranch$ is in the current tree, and that feature $\afeat$ is tested on branch $\abranch$.
	
	\item If $(\abranch,\afeat) \in \sequence$, then every subtree of $\abranch$ starting with a feature test $\aofeat <_{\abranch} \afeat$ has already been explored and $\best[\abranch]$ contains the minimum of their errors. The set $\dom[\abranch]$ contains all \emph{untried} feature tests for branch $\abranch$ ($\dom[\abranch] = \{\aofeat \mid \aofeat \in \features ~\wedge~ \afeat <_{\abranch} \aofeat \}$).
	
	\item If $(\abranch,\afeat) \in \sequence$ but one of its children $\grow{\abranch}{\afeat}$ or $\grow{\abranch}{\afeat}$ (call it $\aobranch$) is not in the current tree, then:

	\begin{itemize}
		\item it is \emph{terminal} ($|\aobranch|=k$ or $\error[\aobranch]=0$), or
		\item it is a \emph{bud} yet to be expanded ($\aobranch \in \bud$), or
		\item it is \emph{optimal}: all possible feature tests have been tried for $\aobranch$.
	\end{itemize}
\end{itemize}


\begin{algorithm}[t]
\begin{footnotesize}
		\caption{Blossom Algorithm\label{alg:bud}}
		\TitleOfAlgo{\budalg}
		  \KwData{$\negex,\posex, \maxd$}
		  \KwResult{The minimum error on $\negex,\posex$ for decision trees of depth $\maxd$}
		$\sequence \gets []$\;
		% $\bud \gets \emptyset$\;
		$\bud \gets \newbud(\emptyset,\emptyset)$\;
		
		% $\bud \gets \{\emptyset\}$\;
		% $\dom[\emptyset] \gets \features$\;
		% $\best[\emptyset] \gets \min(\negex, \posex)$\;
		% \HiLi $\opt[\emptyset] \gets \texttt{false}$\;
		
		
		\While{$|\sequence| + |\bud| > 0$}{
		\lnl{line:dive}\If{$\bud \neq \emptyset$}{
			%$\abranch \gets \select{\bud}$\;
			\lnl{line:budchoice}pick and remove $\abranch$ from $\bud$\;
			
			% \lnl{line:leaves}\eIf{$|\abranch| = \maxd$ or $\error[\abranch] = 0$} {
			% 	% $\error \gets \error + \error[\abranch,\afeat]$\;
			% 	\lnl{line:best}$\best[\abranch] \gets \error[\abranch]$\;
			% }{
			\lnl{line:notterminal}\If{$|\abranch| < \maxd$ and $\best[\abranch] > 0$}{
			\lnl{line:assignment} pick and remove $\afeat$ from $\dom[\abranch]$\;
			% $\dom[\abranch] \gets \dom[\abranch] \setminus \{\afeat\}$\;
			push $(\abranch,\afeat)$ on $\sequence$\;
			% split $\negex[\abranch]$ and $\posex[\abranch]$ w.r.t. $\afeat$\;
			% \lnl{line:branching}\ForEach{$v \in \{\afeat, \bar{\afeat}\}$}{
			% 			\lnl{line:newbud}$\bud \gets \bud \cup \{\grow{\abranch}{v}\}$\;
			% 			\lnl{line:domain}$\dom(\grow{\abranch}{v}) \gets \features \setminus \{\afeat \mid \afeat \in \abranch ~\vee~ \bar{\afeat} \in \abranch\}$\;
			% 			$\best(\grow{\abranch}{v}) \gets \min(\negex[\grow{\abranch}{v}], \posex[\grow{\abranch}{v}])$\;
			% 			\HiLi $\opt(\grow{\abranch}{v}) \gets \texttt{false}$\;
			% 		}
			% }
			\lnl{line:branching} \lForEach{$v \in \{\afeat, \bar{\afeat}\}$}{
				$\bud \gets \newbud{\bud,\grow{\abranch}{v}}$ %\newbud{$\grow{\abranch}{v}$}
				}
			}
		}
		\lnl{line:else}\Else {
			% $\best[\emptyset] \gets \min(\best[\emptyset], \error)$\;
			\lnl{line:backtrack}\While{$|\sequence| > 0$}{
				\lnl{line:pop}pop $(\abranch,\afeat)$ from $\sequence$\;
				\lnl{line:storebest}$\best[\abranch] \gets \min(\best[\abranch], \best[\grow{\abranch}{\afeat}] +  \best[\grow{\abranch}{\bar{\afeat}}])$\;
				% $\error \gets \error - \best[\grow{\abranch}{\afeat}] -  \best[\grow{\abranch}{\bar{\afeat}}]$\;
				\lnl{line:optimal}\If{$\dom[\abranch] \neq \emptyset$ 
				\colorbox{yellow!50}{and $|\abranch|<\mdepth$}
				} {
					\lnl{line:fail} \HiLi \If{$\forall \abranch' \in \ancestors[\abranch], \lb{\abranch',\abranch} < \best[\abranch']$} {
					\lnl{line:right}$\bud \gets \bud \cup \{\abranch\}$\;
					\Break\;
					}
					% \lIf{$\opt[\abranch]$}{$\error \gets \error - \best[\abranch]$}
				} 
				% \HiLi \lnl{line:markoptimal} \lElse{
				% 	$\opt[\abranch] \gets \texttt{true}$
				% }
				% \lElse {
				% 	$\error \gets \error + \best[\abranch]$%$\error[\abranch,\afeat]$
				% }
				% {
				% 	$\opt[\abranch] \gets \tru$\;
				% }
			}
			
			
			
			% \lnl{line:backtrack}\Repeat{($\dom[\abranch] \neq \emptyset$ or $|\sequence|=0$) and $\opt[\abranch]=0$}{
			% 	pop $(\abranch,\afeat)$ from $\sequence$\;
			% 	\lnl{line:storebest}$\best[\abranch] \gets \min(\best[\abranch], \best[\grow{\abranch}{\afeat}] +  \best[\grow{\abranch}{\bar{\afeat}}])$\;
			% 	% $\error \gets \error - \best[\grow{\abranch}{\afeat}] -  \best[\grow{\abranch}{\bar{\afeat}}]$\;
			% 	\lIf{$\dom[\abranch] \neq \emptyset$} {
			% 		$\bud \gets \bud \cup \{\abranch\}$
			% 		% \lIf{$\opt[\abranch]$}{$\error \gets \error - \best[\abranch]$}
			% 	}
			% 	\lElse {
			% 		$\opt[\abranch] \gets 1$\;
			% 	}
			% 	% \lElse {
			% 	% 	$\error \gets \error + \best[\abranch]$%$\error[\abranch,\afeat]$
			% 	% }
			% 	% {
			% 	% 	$\opt[\abranch] \gets \tru$\;
			% 	% }
			% }
		}
		}
		\Return $\best[\emptyset]$\;
	  % \setcounter{AlgoLine}{0}
	   \SetKwProg{myproc}{Procedure}{}{}
	   \myproc{\newbud{$\bud, \abranch$}}{
		% \lnl{line:newbud}$\bud \gets \bud \cup \{\abranch\}$\;
		\lnl{line:splitting}compute $\negex[\abranch]$ and $\posex[\abranch]$ \colorbox{yellow!50}{and $p(\afeat,\negex[\abranch])$ and $p(\afeat,\posex[\abranch]), \forall \afeat \in \features$}\;
		\lnl{line:domain}$\dom(\abranch) \gets \features \setminus \{\afeat \mid \afeat \in \abranch ~\vee~ \bar{\afeat} \in \abranch\}$ \colorbox{yellow!50}{sorted by increasing Gini score}\;
		$\best(\abranch) \gets \min(\negex[\abranch], \posex[\abranch])$\;
		% \colorbox{yellow!50}{$\opt(\abranch) \gets \texttt{false}$}\;
		\Return{$\bud \cup \{\abranch\}$}\;
		}
	\end{footnotesize}
	\end{algorithm}


As long as there is a bud ($\bud \neq \emptyset$), we pick any one $\abranch \in \bud$ at Line~\ref{line:budchoice} and check if it can or need to be expanded in Line~\ref{line:notterminal}. %If its length is $\mdepth$ the error at this leaf is recorded in $\best[\abranch]$. 
If so, we pick a feature $\afeat$ marked as \emph{untried} for \abranch, unmark it, 
expand the tree with the test $\afeat$ at branch $\abranch$. The two children $\grow{\abranch}{\afeat}$ and $\grow{\abranch}{\bar{\afeat}}$ can then be added to $\bud$.
%add the pair $(\abranch,\afeat)$ to \sequence\ and expand the tree with the two branches $\grow{\abranch}{\afeat}$ and $\grow{\abranch}{\bar{\afeat}}$. 

If there is no bud ($\bud = \emptyset$), then the current tree is complete: every branch $\abranch$ is either terminal or optimal. In that case we pop the last assignment $(\abranch,\afeat)$ from \sequence\ 
%, mark the feature $\afeat$ as tried for branch $\abranch$ 
and update the best error of its subtrees. If there is at least one untried feature for branch $\abranch$, we add $\abranch$ to $\bud$.
Otherwise, it is optimal since all features have been tried, and $\best[\abranch]$ contains the minimum error for any subtree of branch $\abranch$. 
%and its error is the sum of the errors of its best subtrees. 
This branch will never be expanded anymore since it is not added to $\bud$.
%
When the algorithm ends, $\best[\emptyset]$ contains the minimum error of any decision tree of depth $\mdepth$. % on the data set.


% Algorithm~\ref{alg:bud} starts from a singleton set \bud\ of open branches or \emph{buds},
% % (open branches are branches of length strictly less than $\maxd$ that are not pure). % The set \nodes\ contains all the nodes of the current tree, open or closed, it is initially equal to \bud. Finally,
% and an initially empty stack of decisions $\sequence$.
%
% \begin{itemize}
% 	\item As long as there is a bud ($\bud \neq \emptyset$), we pick any one $\abranch \in \bud$ and check if it can be expanded in Line~\ref{line:leaves}. If its length is $\mdepth$ the error at this leaf is recorded in $\best[\abranch]$. Otherwise, we pick a feature $\afeat$ marked as \emph{untried} for \abranch, add the pair $(\abranch,\afeat)$ to \sequence\ and expand the tree with the two branches $\grow{\abranch}{\afeat}$ and $\grow{\abranch}{\bar{\afeat}}$.
% 	%
% 	%
% 	%
% 	%
% 	%  a feature $\afeat$ marked as \emph{available} for \abranch, add the pair $(\abranch,\afeat)$ to \sequence\ and expand the tree with the two branches $\grow{\abranch}{\afeat}$ and $\grow{\abranch}{\bar{\afeat}}$.
% 	% %These new nodes are added to $\nodes$.
% 	% They are added to $\bud$ if their depth is strictly less than $\maxd-1$ (the last feature test is chosen according to minimum error) and if they are not pure, otherwise they are terminal tests and we record the corresponding error.
%
% \item If there is no bud ($\bud = \emptyset$), then the tree is complete: every branch $\abranch$ is such that $|\abranch| = \mdepth$ or $\error[\abranch]=0$. In that case we pop the last assignment $(\abranch,\afeat)$ from \sequence, mark the feature $\afeat$ as tried for branch $\abranch$ and update the recorded best error of its subtrees. If there is at least one untried feature for branch $\abranch$, we add $\abranch$ to $\bud$.
% Otherwise, we consider it \emph{terminal} and its error is the sum of the error of its best subtrees. This branch will never be expanded anymore since it is not added to $\bud$.
% % and we
% %store the minimum error recorded for any of the possible features.
%
% \end{itemize}


% \medskip

To simplify the pseudo-code, we use branches to index array-like data structures in Algorithm~\ref{alg:bud} (e.g. $\dom[\abranch]$). Actually, a set of \emph{indices} (at most $2^{\mdepth}$ in the worst case) are used as proxy for branches in all contexts, since the current tree cannot have more than $2^{\mdepth}$ branches. At Line~\ref{line:storebest}, the indices for $\grow{\abranch}{\afeat}$, $\grow{\abranch}{\bar{\afeat}}$ are released, and a free index is marked as used when expanding a branch at Line~\ref{line:branching}. Moreover, the pseudo-code in Algorithm~\ref{alg:bud} does not show how the best subtrees of optimal branches are recorded, nor how the overall best error is updated when completing a new decision tree at Line~\ref{line:else}.
%The worst case space complexity of the algorithm is therefore in $\Theta(2^{\mdepth}\numfeat)$. Under the standard assumption that $2^{\mdepth} \leq \numex$, this is less than the size of the input.

% where $\sizetree \leq 2^{\mdepth}$ is the maximum size of the explored tree, that is the maximum length of $\sequence$.


%The classification error of a tree is equal to the sum of the error of its terminal banches, the algorithm returns the minimum value encountered when exploring the possible decision trees.
%In other words, this algorithm will first build a complete tree by expanding non-pure, non-maximal branches in any order. When all leaves are pure of a maximal depth, the last test of the last expanded branch w



% \medskip





		

		\begin{theorem}
			The worst case time complexity of Algorithm~\ref{alg:bud} is $\Theta((\numex+2^\mdepth){\Perm{\numfeat}{\mdepth}}) \subset O((\numex + 2^{\mdepth})\numfeat^\mdepth)$ and its worst case space complexity is in $\Theta(2^{\mdepth}\numfeat)$.
			\end{theorem}
			
			
			\begin{proof}
				From the invariants, we can see that Algorithm~\ref{alg:bud} explores the same set of $\Perm{\numfeat}{\mdepth}2^\mdepth$ branches (i.e., the $2^{\mdepth}$ outcomes of each permutation of ${\mdepth}$ features).
				%
				Moreover, the ``yes'' branch of Condition~\ref{line:dive} dominates the time complexity since at most one element is added to $\sequence$, whereas  Loop~\ref{line:backtrack} suppresses exactly one element of $\sequence$ at every iteration (and each of its iterations is in constant time).
				
				The time complexity is therefore dominated by the splitting procedure whereby $\negex[\grow{\abranch}{\afeat}]$, $\negex[\grow{\abranch}{\bar{\afeat}}]$, $\posex[\grow{\abranch}{\afeat}]$ and $\posex[\grow{\abranch}{\bar{\afeat}}]$ are computed from $\negex[\abranch]$ and $\posex[\abranch]$. As discussed earlier, this takes linear time amortized over the $2^{\mdepth}$ branches sharing the same set of $\mdepth$ features. Therefore, the overall time complexity for the splitting operations is in $\Theta(\Perm{\numfeat}{\mdepth}\numex)$.
				
				
				Since branches can be stored in constant space (an index, the parent branch and the two children),
				the worst case space complexity $\Theta(2^{\mdepth}\numfeat)$ to record which feature have been tried (the sets $\dom$).
				\end{proof}
				
				
				%
				%
				% We say that a branch is \emph{explored} if it is picked and removed from \bud\ at Line~\ref{line:budchoice}, or, equivalently, if it is added \bud\, because since Loop~\ref{line:backtrack} terminates, every added branch will eventually be picked.
				%
				%
				% % We show that every terminal branch is explored exactly once, by recursion of the maximum depth $\mdepth$.
				% % For $\mdepth=0$, the unique branch is $\emptyset$, it is explored and the algorithm returns immediatly.
				% %
				% % Now suppose that for $\mdepth=d$ every terminal branch is explored exactly once. Now, let $\mdepth=d+1$ and consider a terminal branch $\grow{\abranch}{v}$.
				%
				% A branch added at Line~\ref{line:right} cannot be terminal, since the pair $(\abranch,\afeat)$ has been popped out of \sequence, witnessing the branches $\grow{\abranch}{\afeat}$ and $\grow{\abranch}{\bar{\afeat}}$.
				%
				% Therefore, if a branch $\grow{\abranch}{v}$ is explored more than once, it must be added twice at Line~\ref{line:newbud}. However, $\dom[\abranch]$ forbids that unless $\branch$ was itself added twice at Line~\ref{line:newbud} because $\dom[\abranch]$ is only reset at Line~\ref{line:domain}. This argument can be repeated for the ancestors of $\abranch$ until reaching $\emptyset$ which is explored only once.
				%
				% Now we need to show that every terminal branch is explored at least once.
				% We show that by recursion of the maximum depth $\mdepth$.
				% For $\mdepth=0$, the unique branch is $\emptyset$, and it is explored.
				%
				% Now suppose that for $\mdepth=d$ every terminal branch is explored, let $\mdepth=d+1$ and consider a terminal branch $\grow{\abranch}{v}$. Since increasing $k$ can only increase the number of explored branches, by the recursion hypothesis, $\abranch$ is explored. Moreover, since $|\abranch|<\mdepth$, a feature $\afeat$ will be selected and $\grow{\abranch}{\afeat}$ and $\grow{\abranch}{\bar{\afeat}}$ added to $\bud$. Therefore, if $v=\afeat$ or $v=\bar{afeat}$, then the claim holds.
				% Otherwise, the pair $(\abranch,\afeat)$ is added to $\sequence$ and it will eventually [TODO, NEED AN ARGUMENT HERE?] be popped out at Line~\ref{line:pop}.
				% Then, $\afeat$ will be removed from $\dom[\abranch]$ and $\abranch$ reinserted into $\bud$.
				%
				%
				%
				%
				% We first show that every terminal branch is explored exactly once, by recursion of the maximum depth $\mdepth$.
				% For $\mdepth=0$, the unique branch is $\emptyset$, it is explored and the algorithm returns immediatly.
				%
				% Now suppose that for $\mdepth=d$ every terminal branch is explored exactly once. Now, let $\mdepth=d+1$ and consider a terminal branch $\grow{\abranch}{v}$.
				%
				%
				%
				% We first show that a terminal branch is explored
				%
				%
				%
				%
				%
				% The proof follows from the fact that Algorithm~\ref{alg:bud} explores exactly once every permutation of (negated) features of size $\mdepth$.
				%
				% Therefore, any branch added to $\bud$ will eventually be picked at Line~\ref{line:budchoice}.
				% We say that a branch is explored if it is added (and therefore picked from) $\bud$.
				%
				% We prove the claim by recursion on $\mdepth$.
				% For $\mdepth = 0$, the empty branch $\emptyset$ is explored and the algorithm terminates by returning $\error[\emptyset] = \min(|\negex|, |\posex|)$.
				%
				% Now suppose that, for $\mdepth \leq d$, every branch of length $\mdepth$ is explored exactly once, and consider the case $\mdepth=d+1$.
				% Let $\grow{\abranch}{v}$ be a branch of length $d+1$. By the recursion hypothesis, and since the value of $\mdepth$ is only tested at Line~\ref{line:leaves}, $\abranch$ is also explored when $\mdepth=d+1$.
				% Moreover, since $|\abranch|<\mdepth$, a feature $\afeat$ will be selected and $\grow{\abranch}{\afeat}$ and $\grow{\abranch}{\bar{\afeat}}$ added to $\bud$. Therefore, if $v=\afeat$ or $v=\bar{afeat}$, then the claim holds.
				% Otherwise, the pair $(\abranch,\afeat)$ is added to $\sequence$ and it will eventually [TODO, NEED AN ARGUMENT HERE?] be popped out at Line~\ref{line:pop}.
				% Then, $\afeat$ will be removed from $\dom[\abranch]$ and $\abranch$ reinserted into $\bud$.
				%
				% Moreover, the algorithm will not stop until $\sequence$ is not empty, hence $(\abranch, \afeat)$ will eventually be popped out of $\sequence$, $\afeat$ removed from $\dom[\abranch]$ and $\abranch$ reinserted into $\bud$. Therefore, for every $\afeat \in \features$, such that neither $\afeat \in \abranch$ nor $\bar{\afeat}\in \abranch$,
				%  $\grow{\abranch}{\afeat}$ and $\grow{\abranch}{\bar{\afeat}}$ will eventually be explored exactly once.
				%
				%
				% Let $\abranch$ be a branch of length $d$.
				% By the recursion hypothesis, and since the value of $\mdepth$ is only tested at Line~\ref{line:leaves}, $\abranch$ is also explored when $\mdepth=d+1$. We ignore the case where $\error[\abranch]=0$, since no extension of $\abranch$ need be explored.
				%
				% Moreover, since $|\abranch|<\mdepth$, a feature $\afeat$ will be selected and $\grow{\abranch}{\afeat}$ and $\grow{\abranch}{\bar{\afeat}}$ added to $\bud$, and the pair $(\abranch,\afeat)$ added to $\sequence$.
				%
				% \medskip
				%
				%
				%  both branches $\grow{\abranch}{\afeat}$
				%
				%
				% Without loss of generality, let $\abranch$ of length $d$ and show that the same extentions of $\abranch$ are explored by both algorithms.
				%
				%
				% both algorithms explore exactly the same set of branches, albeit not in the same order:
				% they explore every permutation of (negated) feature of size $\mdepth$.
				%
				% We show that this is true for Algorithm~\ref{alg:bud} by recursion on the maximum depth $\mdepth$.
				% For $\mdepth = 0$, both algorithm return $\error[\emptyset]=\min(|\negex|, |\posex|)$.
				%
				% Now suppose that, $\mdepth \leq d$, \dynprog and \budalg explore exactly the same set of branches: a recursive call of \dynprog ends on the branch $\abranch$ if and only if $\best[\abranch]$ is set in Line~\ref{line:best} of \budalg.
				% %every branch $\abranch$ explored by \dynprog (a recursive call ends on this branch) is also explored by \budalg ($\best[\abranch]$ is set in Line~\ref{line:best}), and
				% Now, let $\mdepth=d+1$. Without loss of generality, we can take an arbitrary branch $\abranch$ of length $d$ and show that the same extentions of $\abranch$ are explored by both algorithms.
				% If $\error[\abranch]=0$, then no extension of $\abranch$ is explored by either algorithm.
				% Otherwise, if $\afeat \not\in \abranch$ and $\bar{\afeat}\not\in \abranch$, then
				% \dynprog explores the branches $\grow{\abranch}{\afeat}$ and $\grow{\abranch}{\bar{\afeat}}$.
				% %every branch $\grow{\abranch}{\afeat}$ and $\grow{\abranch}{\bar{\afeat}}$ for $\afeat \in \features \setminus \abranch$.
				% Since $\abranch$ is explored when $\mdepth = d$ and since $\error[\abranch] \neq 0$, then $\abranch$ will fail the test on Line~\ref{line:leaves} and a feature $\afeat$ such that $\afeat \not\in \abranch$ and $\bar{\afeat}\not\in \abranch$ will be selected,
				% %for a feature $\afeat \in \features \setminus \abranch$,
				% the branches $\grow{\abranch}{\afeat}$ and $\grow{\abranch}{\bar{\afeat}}$ will be explored, and the pair $(\abranch, \afeat)$ will be added to $\sequence$.
				% Moreover, the algorithm will not stop until $\sequence$ is not empty, hence $(\abranch, \afeat)$ will eventually be popped out of $\sequence$, $\afeat$ removed from $\dom[\abranch]$ and $\abranch$ reinserted into $\bud$. Therefore, for every $\afeat \in \features$, such that neither $\afeat \in \abranch$ nor $\bar{\afeat}\in \abranch$,
				%  $\grow{\abranch}{\afeat}$ and $\grow{\abranch}{\bar{\afeat}}$ will eventually be explored exactly once.
			% 	\hfill$\square$
			% \end{proof}
			
			% \begin{proof}[sketch]
			% 	The proof follows from the fact that both algorithms explore exactly the same set of branches, albeit not in the same order:
			% 	they explore every permutation of (negated) feature of size $\mdepth$.
			%
			% 	We show that this is true for Algorithm~\ref{alg:bud} by recursion on the maximum depth $\mdepth$.
			% 	For $\mdepth = 0$, both algorithm return $\error[\emptyset]=\min(|\negex|, |\posex|)$.
			%
			% 	Now suppose that, $\mdepth \leq d$, \dynprog and \budalg explore exactly the same set of branches: a recursive call of \dynprog ends on the branch $\abranch$ if and only if $\best[\abranch]$ is set in Line~\ref{line:best} of \budalg.
			% 	%every branch $\abranch$ explored by \dynprog (a recursive call ends on this branch) is also explored by \budalg ($\best[\abranch]$ is set in Line~\ref{line:best}), and
			% 	Now, let $\mdepth=d+1$. Without loss of generality, we can take an arbitrary branch $\abranch$ of length $d$ and show that the same extentions of $\abranch$ are explored by both algorithms.
			% 	If $\error[\abranch]=0$, then no extension of $\abranch$ is explored by either algorithm.
			% 	Otherwise, if $\afeat \not\in \abranch$ and $\bar{\afeat}\not\in \abranch$, then
			% 	\dynprog explores the branches $\grow{\abranch}{\afeat}$ and $\grow{\abranch}{\bar{\afeat}}$.
			% 	%every branch $\grow{\abranch}{\afeat}$ and $\grow{\abranch}{\bar{\afeat}}$ for $\afeat \in \features \setminus \abranch$.
			% 	Since $\abranch$ is explored when $\mdepth = d$ and since $\error[\abranch] \neq 0$, then $\abranch$ will fail the test on Line~\ref{line:leaves} and a feature $\afeat$ such that $\afeat \not\in \abranch$ and $\bar{\afeat}\not\in \abranch$ will be selected,
			% 	%for a feature $\afeat \in \features \setminus \abranch$,
			% 	the branches $\grow{\abranch}{\afeat}$ and $\grow{\abranch}{\bar{\afeat}}$ will be explored, and the pair $(\abranch, \afeat)$ will be added to $\sequence$.
			% 	Moreover, the algorithm will not stop until $\sequence$ is not empty, hence $(\abranch, \afeat)$ will eventually be popped out of $\sequence$, $\afeat$ removed from $\dom[\abranch]$ and $\abranch$ reinserted into $\bud$. Therefore, for every $\afeat \in \features$, such that neither $\afeat \in \abranch$ nor $\bar{\afeat}\in \abranch$,
			% 	 $\grow{\abranch}{\afeat}$ and $\grow{\abranch}{\bar{\afeat}}$ will eventually be explored exactly once.
			% 	\hfill$\square$
			% \end{proof}
			
			\medskip
			
			The key difference between Algorithms~\ref{alg:dynprog} and \ref{alg:bud} is the order in which branches are explored (see Figure~\ref{fig:searchtree}). In particular, \dynprog must complete the first recursive call before outputing a full tree. Therefore, the computation time for finding a first complete tree is $\Theta((\numex+2^{\mdepth})\Perm{\numfeat-1}{\mdepth-1})$, that is $O(\numex(\numfeat-1)^{\mdepth-1})$ time. On the other hand, \budalg finds a first tree in linear time: $\Theta(2^{\mdepth}+\numex\mdepth) = \Theta(\numex\mdepth)$.
			Another difference with actual implementations of Algorithm~\ref{alg:dynprog} (\olddleight\ and \dleight) is that the latter methods use a cache structure in order to reduce the number of branches that need to be explored. Indeed, by using memory, it is sufficient to explore every \emph{combination}\footnote{Actually, some combination may be completely avoided using bounds reasoning and subset lookup.} (instead of every permutation) of $\mdepth$ features since the order of the tests does not matter, given a single branch. Our experimental evaluations, however, show that the overhead of cache lookups may not be always beneficial. Moreover, the space complexity of managing the cache may be prohibitive. On the other hand, Algorithms~\ref{alg:dynprog} and \ref{alg:bud} are essentially memoryless, since, under the standard assumption that $2^{\mdepth} \leq \numex$, their worst-case space complexity is less than the size of the input.
			
			
		
			
			
			
% 			% First, notice that the task of splitting the data set on eevry branch of the search tree can be done exactly as in DL8, that is, in $\Theta(\numex)$ amortised time for each of the $\mdepth$ tree levels.
%
% 			% \begin{proof}
%
% 			First, all branches eventually reached. Consider an arbitrary branch $\abranch$.
%
%
% 			\medskip
%
%
% 			$\best[\abranch]$ is the minimal error of any subtree rooted at $\abranch$ with a feature in $\features \setminus \dom[\abranch]$ is $\best[\abranch]$.
% 			This is true for pure branches
%
%
% 			 and maximum-depth branches (code after condition in Line~\ref{line:leaves}). Now consider a branch $\abranch$ such that $|\abranch|<\mdepth$ and $\forall \afeat \in \features, \error[\abranch,\afeat] > 0$.
%
% 			\medskip
%
%
%
%
% 			Let a branch $\abranch$ be \emph{explored} iff it was put in the stack $\sequence$ and $\dom[\abranch] = \emptyset$.
% 			If a branch $\abranch$ is explored, then $\best[\abranch]$ is the minimal error of any subtree rooted at $\abranch$.
% 			This is true for pure branches and maximum-depth branches (code after condition in Line~\ref{line:leaves}). Now consider a branch $\abranch$ such that $|\abranch|<\mdepth$ and $\forall \afeat \in \features, \error[\abranch,\afeat] > 0$.
%
% 			\medskip
%
%
%
% 				First, we show that the algorithm is correct. In particular the following property holds:
% 				after Line~\ref{line:storebest}, the minimum error of any subtree rooted at $\abranch$ with a feature in $\features \setminus \dom[\abranch]$ is $\best[\abranch]$.
%
%
% 				Observe that $|\abranch| \leq \mdepth-2$. Indeed, no pair $(\abranch,\afeat)$ is put on $\sequence$ unless $|\abranch| \leq \mdepth-2$. Now suppose that $|\abranch|=\mdepth-2$. Then $\best[\grow{\abranch}{\afeat}]$ is by definition the minimal error of any single-node tree rooted at $\grow{\abranch}{\afeat}$ and likewise, $\best[\grow{\abranch}{\bar{\afeat}}]$ is the minimal error of any single-node tree rooted at $\grow{\abranch}{\bar{\afeat}}$. Therefore, $\best[\grow{\abranch}{\afeat}] + \best[\grow{\abranch}{\bar{\afeat}}]$ is the minimal error of a depth 2 tree rooted at $\abranch$ with a test on $\afeat$. Since all features in $\features \setminus \dom[\abranch]$ have been tried (or belong to $\abranch$) and the minimum was kept, the property holds.
%
% 				Suppose now that the property holds for $|\abranch|=\mdepth-x$ with $x>2$. Then by the same reasoning as above, the property will also hold for $|\abranch|=\mdepth-x-1$. Therefore is always hold.
%
%
%
% 			\medskip
%
%
%
%
%
%
% 			The key is to observe that given a branch $\abranch$ and a feature $\afeat$, just as in DL8, the complexity of computing
% 			$\error[\abranch,\afeat]$ is equal to the complexity of computing $\error[\grow{\abranch}{\afeat}]$ plus the complexity of computing $\error[\grow{\abranch}{\bar{\afeat}}]$.
% 			Indeed, wlog, let the branch $\grow{\abranch}{\afeat}$ be chosen before $\grow{\abranch}{\bar{\afeat}}$.
% 				Both branches will be completed up to the maximal depth, however,
% 			the test appended to $\grow{\abranch}{\afeat}$ will no change until
%
%
%
%
% 			the best possible errors for $\grow{\abranch}{\afeat}$ and $\grow{\abranch}{\bar{\afeat}}$
%
%
% 			% \end{proof}
%
% 			We first show that the number of assignment of tests (Line~\ref{line:assignment}) is $2^{\mdepth-1}\numfeat^{\mdepth-1}$.
% 			For $\mdepth=1$ this is true since there is a single assigned test (with the feature $\argmin_{\afeat \in \features}(\error[\emptyset,\afeat])$).
%
% 			Now suppose that the property holds for depth $\mdepth-1$ and consider depth $\mdepth$.
%
%
%
%
% 			No proof is given, but independent subtrees are not explored in Algorithm~\ref{alg:bud} hence both algorithms do the same computation, except not in the same order. Notice that Algorithm~\ref{alg:bud} eagerly compute the conditional error $\error[\abranch,\afeat]$ for every feature $\afeat \in \features$, which incurs an extra factor $\numfeat$ for the data set partitionning task. However, in return the last test of each branch is chosen in $O(1)$ so we gain the same factor $\numfeat$.
%
% 			The big difference is that whereas the cost of finding a first solution is $\Theta((\numex + 2^{\mdepth-1})\numfeat^{\mdepth-1})$ for Algorithm~\ref{alg:dynprog}, it is equal to $\Theta(\mdepth\numex + 2^\mdepth)$ for Algorithm~\ref{alg:bud}, which in practice is very important, as shown in the experimental section.
%
% 	%T(m,k) = 2mT(m-1, k-1)
%
%
%
%
% % \clearpage






	
	
	% \begin{algorithm}
	% 	\caption{Anytime Algorithm\label{alg:bud}}
	% 	\TitleOfAlgo{\budalg}
	% 	  \KwData{$\negex,\posex, \maxd$}
	% 	  \KwResult{The minimum error on $\negex,\posex$ for decision trees of depth at most $\maxd$}
	% 	$\sequence \gets []$\;
	% 	$\bud \gets \{\emptyset\}$\;
	% 	$\error \gets \min(|\negex|,|\posex|)$\;
	% 	$\dom \gets (\lambda : {2^{\features}} \mapsto \features)$\;
	% 	$\best \gets (\lambda : {2^{\features}} \mapsto \infty)$\;
	% 	% $\opt \gets (\lambda : {2^{\features}} \mapsto \fal)$\;
	%
	% 	\While{$|\sequence| + |\bud| > 0$}{
	% 	\eIf{$\bud \neq \emptyset$}{
	% 		%$\abranch \gets \select{\bud}$\;
	% 		pick and remove $\abranch$ from $\bud$\;
	% 		\lnl{line:assignment}$\afeat \gets \argmin_{\afeat \in \dom[\abranch]}(\error[\abranch,\afeat])$\;
	% 		\lnl{line:leaves}\eIf{$\error[\abranch,\afeat] = 0$ or $|\abranch| = \maxd-1$} {
	% 			$\error \gets \error + \error[\abranch,\afeat]$\;
	% 			$\best[\abranch] \gets \error[\abranch,\afeat]$\;
	% 			$\dom[\abranch] \gets \emptyset$\;
	% 			% $\opt[\abranch] \gets \tru$\;
	% 		}{
	% 		$\dom[\abranch] \gets \dom[\abranch] \setminus \{\afeat\}$\;
	% 		push $(\abranch,\afeat)$ on $\sequence$\; % $ \gets \sequence \oplus (\abranch,\afeat)$\;
	% 		\ForEach{$v \in \{\afeat, \bar{\afeat}\}$}{
	% 				\lIf{$\error[\abranch,v] > 0$}{
	% 					$\bud \gets \bud \cup \{\abranch \wedge v\}$
	% 				}
	% 		}
	% 		}
	% 	}{
	% 		$\best[\emptyset] \gets \min(\best[\emptyset], \error)$\;
	% 		\Repeat{$\dom[\abranch] \neq \emptyset$ or $|\sequence|=0$}{
	% 			pop $(\abranch,\afeat)$ from $\sequence$\;
	% 			\lnl{line:storebest}$\best[\abranch] \gets \min(\best[\abranch], \best[\grow{\abranch}{\afeat}] +  \best[\grow{\abranch}{\bar{\afeat}}])$\;
	% 			$\error \gets \error - \best[\grow{\abranch}{\afeat}] -  \best[\grow{\abranch}{\bar{\afeat}}]$\;
	% 			\lIf{$\dom[\abranch] \neq \emptyset$} {
	% 				$\bud \gets \bud \cup \{\abranch\}$
	% 				% \lIf{$\opt[\abranch]$}{$\error \gets \error - \best[\abranch]$}
	% 			}
	% 			\lElse {
	% 				$\error \gets \error + \best[\abranch]$%$\error[\abranch,\afeat]$
	% 			}
	% 			% {
	% 			% 	$\opt[\abranch] \gets \tru$\;
	% 			% }
	% 		}
	% 	}
	% 	}
	% 	\Return $\best[\emptyset]$\;
	% \end{algorithm}
	
	
	

	
	
	
	
	% \begin{figure}
	% \begin{center}
	% 	\tabcolsep=0pt
	% 	\scalebox{1}{
	% 		\begin{forest}
	% 			for tree={%
	% 				l sep=20pt,
	% 				s sep=3.5pt,
	% 				node options={shape=rectangle, minimum width=10pt, inner sep=0pt, font=\footnotesize},
	% 	  		edge={thick, -latex, shorten >=1pt, shorten <=1pt},
	% 			}
	% 			[{$\emptyset$}
	% 			 [{$a,\bar{a}$}
	% 			  [{$a \wedge b$,$a \wedge \bar{b}$}
	% 					[{$a \wedge b \wedge c$,$a \wedge b \wedge \bar{c}$}]
	% 					[{$a \wedge \bar{b} \wedge c$,$a \wedge \bar{b} \wedge \bar{c}$}]
	% 				]
	% 			  [{$\bar{a} \wedge b$,$\bar{a} \wedge \bar{b}$}]
	% 			 ]
	% 			 [{$b$}]
	% 			 [{$c$}]
	% 			]
	% 		\end{forest}
	% 	}
	% \end{center}
	% \caption{\label{fig:searchtree} The search tree for decision trees. \texttt{DynProg} explores it depth first, whereas \texttt{Bud-first-search} explores branches in the order given below the leaves.}
	% %\caption{\label{fig:searchtree} The search tree for decision trees. \dynprog explores it depth first, whereas \budalg explores branches in the order given below the leaves.}
	% \end{figure}
	
	
	% \begin{tabular}{c|ll}
	% 	\# & \bud & \sequence \\
	% 	\hline
	% 	1 & $\{\emptyset\}$ & $[]$ \\
	% 	2 & $\{a,\bar{a}\}$ & $[(\emptyset,a)]$ \\
	% 	3 & $\{a \wedge b,a \wedge \bar{b},\bar{a}\}$ & $[(\emptyset,a),(a,b)]$ \\
	% 	4 & $\{a \wedge \bar{b},\bar{a}\}$ & $[(\emptyset,a),(a,b)]$ \\
	% 	5 & $\{\bar{a}\}$ & $[(\emptyset,a),(a,b)]$ \\
	% 	6 & $\{\bar{a} \wedge b, \bar{a} \wedge \bar{b}\}$ & $[(\emptyset,a),(a,b),(\bar{a},b)]$ \\
	% 	7 & $\{\bar{a} \wedge \bar{b}\}$ & $[(\emptyset,a),(a,b),(\bar{a},b)]$ \\
	% 	8 & $\{\}$ & $[(\emptyset,a),(a,b),(\bar{a},b)]$ \\
	% 	9 & $\{\bar{a} \wedge c, \bar{a} \wedge \bar{c}\}$ & $[(\emptyset,a),(a,b),(\bar{a},c)]$ \\
	% 	10 & $\{\bar{a} \wedge \bar{c}\}$ & $[(\emptyset,a),(a,b),(\bar{a},c)]$ \\
	% 	11 & $\{\}$ & $[(\emptyset,a),(a,b),(\bar{a},c)]$ \\
	% 	12 & $\{a \wedge c, a \wedge \bar{c}\}$ & $[(\emptyset,a),(a,c)]$ \\
	% 	13 & $\{a \wedge \bar{c}\}$ & $[(\emptyset,a),(a,c)]$ \\
	% 	14 & $\{\}$ & $[(\emptyset,a),(a,c)]$ \\
	% \end{tabular}


% For readability, we cut the algorithm into four blocks. The initialisation procedure (Algorithm~\ref{alg:init}) set up the data structures used in all other procedures:
% \begin{itemize}
% 	\item \sequence\ is simply the list of nodes in the current tree, ordered as they are explored.
% 	\item \nodes\ is the set of integers used to index a node of the current tree
% 	\item \bud\ is the set of nodes which do no have an assigned test yet
% 	\item \mdepth\ stores the depth of a node
% 	\item \test\ stores the feature tested at a node
% 	\item \dom\ stores the set of possible features which have no yet been tried for this node
% 	\item \best\ stores the error of the best subtree rooted at a node
% 	\item \opt\ indicates whether the best subtree of a given node is optimal
% 	\item \child\ stores the children of a node (children can be nodes or $\{\posclass, \negclass\}$)
% 	\item $\error{\anode}$ $\min(|\posex(\anode)|,|\negex(\anode)|)$
% 	\item $\error{\anode,\afeat}$ $\min(|\posex(\anode=\afeat)|,|\negex(\anode=\afeat)|)$
% \end{itemize}
%
% Algorithm~\ref{alg:search} is a bactracking procedure which expands a current decision tree
%
% 	\begin{algorithm}
% 		\caption{Data Structures\label{alg:init}}
% 		\TitleOfAlgo{Initialise}
% 		$\sequence \gets []$\;
% 		$\bud \gets \emptyset$\;
% 		$\nodes \gets \emptyset$\;
% 		$\ub \gets \min(|\negex|,|\posex|)$\;
% 		$\error \gets ub$\;
%
% 		$\child \gets (\lambda : \mathbb{N} \times \{\fal, \tru\} \mapsto \emptyset)$\;
% 		$\mdepth \gets (\lambda : \mathbb{N} \mapsto 0)$\;
%
% 		$\test \gets (\lambda : \mathbb{N} \mapsto \emptyset)$\;
% 		$\dom \gets (\lambda : \mathbb{N} \mapsto \features)$\;
%
% 		$\best \gets (\lambda : \mathbb{N} \mapsto \infty)$\;
% 		$\opt \gets (\lambda : \mathbb{N} \mapsto \fal)$\;
% 	\end{algorithm}
%
%
%
% 	\begin{algorithm}
% 		\caption{Create a new node after branching\label{alg:alloc}}
% 		\TitleOfAlgo{\grow}
% 	  \KwData{integer \anode}
%
% 		$\nodes \gets \nodes \cup \{\anode\}$\;
% 		$\dom[\anode] \gets \features$ sorted by decreasing conditional error $\min(|\posex(\anode=\afeat)|,|\negex(\anode=\afeat)|)$\;
% 		$\test[\anode] \gets \pop(\dom[\anode])$\;
%
%
% 		\eIf{$\mdepth[\anode]=k-1$ or $\error{\anode,\test[\anode]}$}
% 		{
% 			$\best[\anode] = \error{\anode,\test[\anode]}$\; %\min(|\posex(\anode=\test[\anode])|,|\negex(\anode=\test[\anode])|)$\;
% 			$\opt[\anode] = \tru$\;
% 			\ForEach{$branch \in \{\tru, \fal\}$}
% 			{
% 				% $\child[\anode,branch] \gets (|\posex(\anode=\test[\anode])| > |\negex(\anode=\test[\anode])|)$\;
% 				\lIf{$|\posex(\anode=\test[\anode])| > |\negex(\anode=\test[\anode])|$}{$\child[\anode,branch] \gets \posclass$}
% 				\lElse{$\child[\anode,branch] \gets \negclass$}
% 			}
% 		}
% 		{
% 			$\bud \gets \bud \cup \{n\}$\;
% 			$\best[\anode] = \min(|\posex(\anode)|, |\negex(\anode)|)$\;
% 			$\opt[\anode] \gets \fal$\;
% 		}
%
%
% 	\end{algorithm}
%
%
% 	\begin{algorithm}
% 		\caption{Suppress a node and all its descendants\label{alg:free}}
% 		\TitleOfAlgo{\prune}
% 	  \KwData{integer \anode}
%
% 		$\bud \gets \bud \setminus \{\anode\}$\;
% 		$\nodes \gets \nodes \setminus \{\anode\}$\;
%
% 		\ForEach{$branch \in \{\tru, \fal\}$}
% 		{
% 		\lIf{$\child[\anode,branch] \not\in \{\posclass, \negclass\}$}
% 		{
% 			$\prune{\child[\anode,branch]}$
% 		}
% 		}
%
% 		\lIf{$\mdepth[\anode] = k-1$ or $\opt[\anode]$}{$error \gets error - \best[\anode]$}
%
% 	\end{algorithm}
%
%
% \begin{algorithm}
% 	\caption{Search loop\label{alg:search}}
%   \TitleOfAlgo{\dt}
%   \KwData{$\negex,\posex, k$}
%   \KwResult{A decision tree}
%
% 	$\bnode \gets 0$\;
% 	$\posex(1),\negex(1) \gets \posex, \negex$\;
% 	$\grow{\bud, \sequence, 1}$\;
%
% 	\While{\textbf{true}}{
% 		\eIf{$\bud = \emptyset$} {
% 			$\ub \gets \min(\ub,\error)$\;
% 			$deadend \gets \fal$\;
% 			\Repeat{$deadend$}{
% 				\lIf{$\bnode > 0$}{$\opt[\bnode] \gets \tru$}
% 				\lIf{$\bnode = 1$}{\Return}
% 				$\bnode \gets \pop{\sequence}$\;
% 				$\best[\bnode] \gets \min(\best[\bnode], \best(\child[\bnode,\tru]) + \best(\child[\bnode,\fal]))$\;
% 				$\test[\bnode] \gets \pop{\dom[\bnode]}$\;
% 				$\prune(\child[\bnode,\tru])$\;
% 				$\prune(\child[\bnode,\fal])$\;
%
% 				$deadend \gets \best[\bnode] = 0 ~\vee~ \dom[\bnode] = \emptyset$\;
% 				\If{$deadend$}
% 				{
% 				$\opt[\bnode] \gets \tru$\;
% 				$\error \gets \error + \error{\bnode}$\; %$\best[\bnode]$\;
% 				}
% 			}
% 			$\bud \gets \bud \cup \{b\}$\;
% 			$\error \gets \error + \min(|\posex(\bnode)|, |\negex(\bnode)|)$\;
% 		}
% 		{
% 			\If{$b = 0$}{
% 				$b=\select{\bud}$\;
% 				% $\bud \gets \bud \setminus \{b\}$\;
% 				$\push(\bnode,\sequence)$\;
% 			}
% 			$c_{\tru}, c_{\fal} = \argmin_{x,y}(\mathbb{N} \setminus \nodes)$\;
% 			$\posex(c_{\tru}),\negex(c_{\tru}),\posex(c_{\fal}),\negex(c_{\fal}) \gets \branch(\posex(\bnode),\negex(\bnode),\test[\bnode])$\;
% 			\ForEach{$branch \in \{\tru, \fal\}$}{
% 				\eIf{$\min(|\posex(c_{branch})|,|\negex(c_{branch})|) = 0$}
% 				{
% 					\lIf{$|\posex(c_{branch})|>|\negex(c_{branch})|$}{$\child[\bnode,branch] \gets \posclass$}
% 					\lElse{$\child[\bnode,branch] \gets \negclass$}
% 				}{
% 					$\child[\bnode,branch] \gets c_{branch}$\;
% 					$\mdepth[c_{branch}] \gets \mdepth[\bnode]+1$\;
% 					$\grow(\bud, \sequence, c_{branch})$\;
% 				}
% 			}
% 			$\bnode \gets 0$\;
% 		}
% 	}
%
% \end{algorithm}

%\clearpage

\section{Extensions of the Algorithm}
\label{sec:ext}

The pseudo-code given in Algorithm~\ref{alg:bud} shows the basic structure of the algorithm. 
As discussed earlier, some important (but rather tedious) parts of the algorithms have been omitted, such as how the best subtrees are stored in Line~\ref{line:storebest} when the best classification score is updated.
We discuss here only improvements that have an impact on the efficiency of the algorithm.




\subsection{Heuristic Ordering}
\label{sec:heuristic}

In order to quickly find accurate trees, it is important to select first the most promising features. We tried three heuristics based on scores to minimize: The \emph{classification error}, the \emph{entropy}~\cite{10.1023/A:1022643204877}, and the \emph{Gini impurity}~\cite{breiman1984classification}. 
Each of these heuristics associates a score to a feature $\afeat$ at a branch $\abranch$:
\begin{eqnarray}
	\textrm{classification error}: & \error[\grow{\abranch}{\afeat}] + \error[\grow{\abranch}{\bar{\afeat}}] \\
	%\textrm{minimum entropy:} & \sum_{v \in \{\afeat,\bar{\afeat}\}} \frac{|\allex(\abranch \wedge v)|}{|\allex[\abranch]|} \cdot -\sum_{\aclass \in \{\negclass,\posclass\}} \frac{|\setex{\aclass}(\abranch \wedge v)|}{|\setex{\aclass}(\abranch)} \log_{2} \frac{|\setex{\aclass}(\abranch \wedge v)|}{|\setex{\aclass}(\abranch)|} \\
	\textrm{entropy:} & \sum\limits_{v \in \{\afeat,\bar{\afeat}\}} -p(v,\allex) \sum\limits_{\aclass \in \{\negclass,\posclass\}} p(v,\setex{\aclass}) \log_{2} p(v,\setex{\aclass}) \\
	\textrm{Gini impurity:} &  \sum\limits_{v \in \{\afeat,\bar{\afeat}\}} p(v,\allex)(1 - \sum\limits_{\aclass \in \{\negclass,\posclass\}} p(v,\setex{\aclass})^2)
\end{eqnarray}
With $p(v,{\cal S}) = \frac{|{\cal S}(\grow{\abranch}{v})|}{|{\cal S}(\abranch)|}$ the ratio of datapoints with feature $v$ in the set ${\cal S}$.
% With $p(v,{\cal S}) = \frac{|{\cal S}(\abranch \wedge v)|}{|{\cal S}(\abranch)|}$.
%The minimum error is simply defined as 

The feature tests at Line~\ref{line:assignment} of Algorithm~\ref{alg:bud} are explored in non-decreasing order with respect to one of the scores above.


In the data sets we used, the Gini impurity was significantly better, and hence all reported experiment results are using Gini impurity unless stated otherwise. For branches of length $\mdepth-1$, however, we use the error instead. Indeed, the optimal feature $\afeat$ for a branch $\abranch$ that cannot be extended further is the one minimizing 
% $\error[\abranch,\afeat]$. 
$\error[\grow{\abranch}{\afeat}] + \error[\grow{\abranch}{\bar{\afeat}}]$.
This means that we actually do not have to try other features for that node. This is implemented by the highlighted code at Line~\ref{line:optimal}: since one cannot improve on the first feature for test at depth $\mdepth$, branches of length $\mdepth-1$ do not have to be put back into \bud, and can be backtracked upon.

 % which means that we effectively restrict search to branches of length $\mdepth-1$.


%We order the possible features for branch $\abranch$ in non-decreasing order with respect to a score above and 
%explore the features in that order in Line~\ref{line:assignment}.
Computing the frequencies $p(\afeat,{\negex[\abranch]})$ and $p(\afeat,{\posex[\abranch]})$ of every feature $\afeat$ can be done in  $\Theta(\numfeat\numex)$ time where 
$\numex = |\negex[\abranch]|+|\posex[\abranch]|$.\footnote{$p(\bar{\afeat},{\negex[\abranch]}) = |\negex[\abranch]| - p({\afeat},{\negex[\abranch]})$ and $p(\bar{\afeat},{\posex[\abranch]}) = |\posex[\abranch]| - p({\afeat},{\posex[\abranch]})$ can then be queried in constant time} In other words this is more expensive than the splitting procedure by a factor $\numfeat$, but can be similarly amortized. However, since the depth of the branches is effectively reduced by one, the number of terminal branches is reduced by the same factor $\numfeat$, hence this incurs no asymptotic increase in complexity.
Furthermore, ordering the features (at Line~\ref{line:domain})
%Computing this order 
costs $\Theta(\numfeat \log \numfeat)$ for each of the $2^{\mdepth-1}\numfeat^{\mdepth-1}$ branches added to $\bud$ at Line~\ref{line:branching}. Again, since the depth of the branches is effectively reduced by one, the resulting complexity 
%(excluding the time for splitting the data set) 
is $O((\numex + 2^{\mdepth} \log \numfeat) \numfeat^{\mdepth})$. This very slight increase is often inconsequencial, as 
$\numex$ is still often the dominating term.
% long as we have $\numex \geq 2^{\mdepth} \log \numfeat$.

The feature ordering has a very significant impact on how quickly the algorithm can improve the accuracy of the classifier. Moreover, it also has an impact (though indirect and much less significant) on the computational time necessary to explore the whole search space and prove optimality, because of the lower bound technique detailed in the next section.


\subsection{Lower Bound}
\label{sec:lb}

It is possible to fail early using a lower bound on the error given prior decisions, similarly as \dleight\ does~\cite{dl8}.
%, following the idea introduced in \cite{dl8}. 
% The idea is that once
When some subtrees along a branch $\abranch$ are optimal and the sum of their errors is larger than the current upper bound, 
%(the best solution found so far) 
then there is no need to continue exploring branch $\abranch$. 


%Line~\ref{line:leaves} can be changed to ``\textbf{If} $\bud \neq \emptyset ~\& \not\exists \abranch \in \bud, \dominated{\abranch}$ \textbf{then}''. %Notice that when a branch is ``pruned'' in this way, its 


%In this case, we can fail by forcing 


First, observe that $\best[\abranch]$ is an upper bound on the classification error for any subtree rooted at $\abranch$, since this value comes from an actual tree (of depth $\mdepth - |\abranch|$ for the data set $\langle \negex[\abranch],\posex[\abranch] \rangle$). It is possible to propagate this upper bound to parent nodes efficiently (in $O(|\abranch|)$ time). Here we assume that this is done recursively for the parent branch, every time the value $\best[\abranch]$ is  updated. %, by recursively applying the same update procedure to the parent.


Now, when the condition in Line~\ref{line:optimal} fails for a branch $\abranch$, it means that $\best[\abranch]$ is \emph{optimal}, there is no subtree rooted at $\abranch$ of maximum depth $\mdepth - |\abranch|$ whose classification error is lower than $\best[\abranch]$. This is true either because every subtree has been explored, or, with the changes described in Section~\ref{sec:heuristic}, because $\mdepth - |\abranch| = 1$ and the feature $\afeat$ with least 
%$\error[\abranch,\afeat]$ 
$\error[\grow{\abranch}{\afeat}] + \error[\grow{\abranch}{\bar{\afeat}}]$
has been chosen. 
Let $\opt[\abranch]=1$ if the branch is optimal and $\opt[\abranch]=0$ otherwise. Notice that this is equivalent, for a branch $\grow{\abranch}{v}$ ending on test $v \in \{\afeat,\bar{\afeat}\}$, to checking if $|\abranch|=\mdepth-1$, or if $(\abranch,\afeat) \in \sequence$ but there is no pair $(\grow{\abranch}{v}, g)$ in sequence.\footnote{Alternatively, and indeed in our implementation, this information can be stored in an array.}
Moreover, let $\ancestors[\abranch]$ denote the ancestors of branch $\abranch$ in the current tree, i.e., $\ancestors[\abranch] = \{\aobranch \mid (\aobranch,v) \in \sequence ~\wedge~ \aobranch \subset \abranch\}$.

% Either way, a branch $\grow{\abranch}{v}$ ending on test $v \in \{\afeat,\bar{\afeat}\}$
% is optimal if and only if
% The extra Line~\ref{line:markoptimal} simply stores this information in the array $\opt$ which shall be used to compute a lower bound.

%loop in Line~\ref{line:backtrack} makes two or more iterations, it means that for the penultimate branch $\abranch$ popped out of \sequence, all possible subtrees have been explored, and therefore, $\best[\abranch]$ is also a lower bound on the classification error for any subtree rooted at $\abranch$. Let $\opt[\abranch]$ be 1 if $\abranch$ has been ``backtracked over'' in this way and 0 otherwise.



%Let $\abranch$ be a bud in $\bud$. Trivially, if $|\abranch|=\mdepth-1$ then this branch will entail $\min\{\error[\afeat] \mid \afeat \in \dom[\abranch]\}$ misclassifications.
Now, consider the highlighted code in Line~\ref{line:fail}.
For any ancestor $\abranch'$ of $\abranch$, we define a lower bound $\lb{\abranch',\abranch}$, \emph{given the feature tests} $\abranch \setminus \abranch'$ as follows:
$$
\lb{\abranch',\abranch} = \sum\limits_{\abranch' \subset \grow{\abranch''}{\afeat} \subseteq \abranch}\opt[\grow{\abranch''}{\bar{\afeat}}] \cdot \best[\grow{\abranch''}{\bar{\afeat}}]
$$
In plain words, $\lb{\abranch',\abranch}$ is the sum the errors of optimal ``sibling'' branches between $\abranch'$ and $\abranch$. %We illustrate this bound in Example~\ref{ex:lb}.
As long as these choices of feature tests stand (i.e., as long as $\abranch$ belongs to the current tree), these subtrees cannot be improved, hence this lower bound is correct.


%
% This lower bound is correct as long as the branch $\abranch$ belongs to the decision tree.
% The procedure $\dominated{\abranch}$ can therefore simply check, for all parent $\abranch'$ of $\abranch$ up until the root ($\emptyset$), whether $\lb{\abranch',\abranch} \geq \best[\abranch']$. As a result, a branch $\abranch$ which is guaranteed, by this reasoning, to never belong to a non-dominated tree will not be explored further.


% \begin{example}[Lower bound reasoning]
% 	\label{ex:lb}
%
%
% 	Figure~\ref{fig:lowerbound} shows a snapshot of the excution of \budalg. Every node is labelled with the feature test on that node, and with the values of $\best[\abranch]$ for the branch $\abranch$ ending on that node. When all subtrees of a branch $\abranch$ have been explored (hence $\opt[\abranch]=1$), this is marked by a ``$^*$''. We assume that the branch considered at Line~\ref{line:fail} is $\abranch = \{r, \bar{a}, \bar{c}, g\}$. For instance, we can suppose that a tree rooted at $\abranch$ with feature $e$ has been found (misclassifying 2 data points). Then, search moved to the sibling branch $\{r, \bar{a}, \bar{c}, \bar{g}\}$, which was then optimized for a total error of $4$, and now the pair $(\abranch,e)$ is popped out of \sequence. For all branches $\abranch'$ of $\abranch$, we give the values of $\lb{\abranch',\abranch}$ and $\best[\abranch']$ between brackets. Since there exists $\abranch'$ such that $\lb{\abranch',\abranch} \geq \best[\abranch']$ (e.g., $\emptyset$ and $\{r, \bar{a}\}$), we know that $\abranch$ cannot belong to an improving solution, and hence there is no need to try to extend it further.
%
% 	 % the current best classifier cannot be improved as long as
%
%
% 	\begin{figure}
% 	\begin{center}
% % \subfloat[upper bounds] {
% 		\scalebox{1}{
% 			\begin{forest}
% 				for tree={%
% 					l sep=25pt,
% 					s sep=10pt,
% 					node options={shape=rectangle, minimum width=10pt, inner sep=1pt, font=\footnotesize},
% 		  		edge={-latex, shorten >=1pt, shorten <=1pt},
% 				}
% 				[{$r:[50,50]$}
% 					[{$a:[19,22]$}, edge={very thick}, edge label={node[midway,fill=white,inner sep=2pt,font=\scriptsize]{$b$}}
% 					 [{$h:15$}, edge label={node[midway,fill=white,inner sep=2pt,font=\scriptsize]{$a$}}
% 					 	 [{$e:10^*$}, edge label={node[midway,fill=white,inner sep=2pt,font=\scriptsize]{$\bar{h}$}}
% 					 	 ]
% 						 [{$c:3$}, edge label={node[midway,fill=white,inner sep=2pt,font=\scriptsize]{$h$}}
% 						 	[{$d:0^*$}, edge label={node[midway,fill=white,inner sep=2pt,font=\scriptsize]{$c$}}]
% 							[{$f:3^*$}, edge label={node[midway,fill=white,inner sep=2pt,font=\scriptsize]{$\bar{c}$}}]
% 						 ]
% 					 ]
% 					 [{$c:[19,17]$}, edge={very thick}, edge label={node[midway,fill=white,inner sep=2pt,font=\scriptsize]{$\bar{a}$}}
% 					 	[{$f:15^*$}, edge label={node[midway,fill=white,inner sep=2pt,font=\scriptsize]{$c$}}
% 							% [$\posclass$, edge label={node[midway,fill=white,inner sep=2pt,font=\scriptsize]{$f$}}]
% 							% [$\negclass$, edge label={node[midway,fill=white,inner sep=2pt,font=\scriptsize]{$\bar{f}$}}]
% 						]
% 						[{$g:[4,6]$}, edge={very thick}, edge label={node[midway,fill=white,inner sep=2pt,font=\scriptsize]{$\bar{c}$}}
% 							[{$e:2$}, edge={very thick}, edge label={node[midway,fill=white,inner sep=2pt,font=\scriptsize]{$g$}}
% 							]
% 							[{$h:4^*$}, edge label={node[midway,fill=white,inner sep=2pt,font=\scriptsize]{$\bar{g}$}}]
% 					 	]
% 					 ]
% 					]
% 					[{$d:31^*$}, edge label={node[midway,fill=white,inner sep=2pt,font=\scriptsize]{$\bar{b}$}}
% 					]
% 				]
% 			\end{forest}
% 		}
% 		% }
% 		% \subfloat[lower bounds w.r.t. $\{b,\bar{a},c,\bar{g}\}$] {
% 		% \scalebox{1}{
% 		% 	\begin{forest}
% 		% 		for tree={%
% 		% 			l sep=25pt,
% 		% 			s sep=10pt,
% 		% 			node options={shape=rectangle, minimum width=10pt, inner sep=1pt, font=\footnotesize},
% 		%   		edge={-latex, shorten >=1pt, shorten <=1pt},
% 		% 		}
% 		% 		[{$b,[49,50]$}
% 		% 			[{$a,[18,22]$}, edge={very thick}, edge label={node[midway,fill=white,inner sep=2pt,font=\scriptsize]{$b$}}
% 		% 			 [{$.$}, edge label={node[midway,fill=white,inner sep=2pt,font=\scriptsize]{$a$}}
% 		% 			 ]
% 		% 			 [{$c,[18,17]$}, edge={very thick}, edge label={node[midway,fill=white,inner sep=2pt,font=\scriptsize]{$\bar{a}$}}
% 		% 			 	[{$f,15^*$}, edge label={node[midway,fill=white,inner sep=2pt,font=\scriptsize]{$c$}}
% 		% 					% [$\posclass$, edge label={node[midway,fill=white,inner sep=2pt,font=\scriptsize]{$f$}}]
% 		% 					% [$\negclass$, edge label={node[midway,fill=white,inner sep=2pt,font=\scriptsize]{$\bar{f}$}}]
% 		% 				]
% 		% 				[{$g,[3,\infty]$}, edge={very thick}, edge label={node[midway,fill=white,inner sep=2pt,font=\scriptsize]{$\bar{c}$}}
% 		% 					[., edge={very thick}, edge label={node[midway,fill=white,inner sep=2pt,font=\scriptsize]{$g$}}
% 		% 					]
% 		% 					[{$h,3^*$}, edge label={node[midway,fill=white,inner sep=2pt,font=\scriptsize]{$\bar{g}$}}]
% 		% 			 	]
% 		% 			 ]
% 		% 			]
% 		% 			[{$d:31^*$}, edge label={node[midway,fill=white,inner sep=2pt,font=\scriptsize]{$\bar{b}$}}
% 		% 			]
% 		% 		]
% 		% 	\end{forest}
% 		% }
% 		% }
% 	\end{center}
% 	\caption{\label{fig:lowerbound} Example of lower bound computation w.r.t. the branch }
% 	%\caption{\label{fig:searchtree} The search tree for decision trees. \dynprog explores it depth first, whereas \budalg explores branches in the order given below the leaves.}
% 	\end{figure}
%
% \end{example}


% \medskip

This reasoning is more effective when good upper bounds are found early, hence the feature ordering  discussed in the previous section has an impact. Moreover, the choice of branch in Line~\ref{line:budchoice}) has an impact as well. We found that the simplest branch selection strategy was also the one giving the best results: we expand first the branch that was inserted into \bud\ first (i.e., \bud\ is \emph{FIFO}). One possible explanation is that by avoiding to unnecessarily ``jump'' to different parts of the decision tree, this strategy promotes optimizing sibling subtrees first, and therefore, deeper tree earlier.

% intuitivelly, one want to optimize the branches of the decision trees with the largest error first, in order to benefit from larger lower bounds earlier. To this end, it



 %for any feature test $\afeat \in \abranch \setminus \abranch'$, if 








%is $O((\numex + 2^{\mdepth}\numfeat \log \numfeat) \numfeat^{\mdepth-1})$, which is often better than 




% $$
% \sum_{v \in \{\afeat,\bar{\afeat\}} \frac{|\allex(\abranch \wedge v)|}{|\allex[\abranch]|} \cdot -\sum_{c \in \{\negclass,\posclass\}} \frac{|\setex{c}(\abranch \wedge v)}{|\setex{c}(\abranch)} \log_{2} \frac{|\setex{c}(\abranch \wedge v)}{|\setex{c}(\abranch)}
% $$


% $$
% 2 - \sum\limits_{v \in \{\afeat,\bar{\afeat}\}}\sum\limits_{c \in \{\negclass,\posclass\}} p(v,\setex{c})^2
% $$



\subsection{Preprocessing}
\label{sec:preprocessing}

Finally, we use two preprocessing techniques, one on the data set and one on the features. Although extremely straightforward (and probably not novel), they both have a significant impact.

\paragraph{Dataset reduction.}
It is easy to adapt \budalg (or most decision tree classifiers, actually) to handle weighted data sets by redefining the error as follows, given a weight function $\weight$ on $\allex$:
$$
\error[\abranch] = \min\left( \sum\limits_{x \in \negex[\abranch]}\weight[x], \sum\limits_{x \in \posex[\abranch]}\weight[x] \right)
$$
We can use the weighted version to handle noisy data, by merging duplicated datapoints and suppressing inconsistent datapoints.

Let $\weight^{\negclass}$ (resp. $\weight^{\posclass}$) denote the number of occurrences of $x$ in $\setex{\negclass}$ (resp. $\setex{\posclass}$). We use the weight function $\weight[x] = |\weight^{\negclass}(x) - \weight^{\posclass}(x)|$. Then, for any datapoint $x$, we remove all but one of its occurrences, in $\setex{\negclass}$ if $\weight^{\negclass}>\weight^{\posclass}$, in $\setex{\posclass}$ if $\weight^{\posclass}>\weight^{\negclass}$, and suppress it completely if $\weight^{\posclass}=\weight^{\negclass}$.
The reported error will then need to be offset by the number of pairs of suppressed inconsistent datapoints, that is:
$
\sum_{x \in \allex}\min(\weight^{\negclass}(x), \weight^{\posclass}(x))
$.
Reducing the number of datapoints in the data set has a non-null, although tiny impact on efficiency. However, suppressing inconsistent datapoints is very important. In particular, proving optimality when the minimum error is positive basically requires to exhaust the search space and is therefore extremely costly. On the other hand, when there exists a perfect tree within the maximum depth, we can stop as soon as we find it. This preprocessing allows to benefit from that when we find a tree whose error is equal to the number of pair of inconsistent datapoints in the original data set.
This preprocessing can be done in $O(\numfeat \numex \log \numex)$ by ordering the datapoints in lexicographic order and then processing them in sequence.

\paragraph{Feature reduction.}

A feature $\afeat$ is redundant if there exists another feature $\afeat'$ such that either: $\forall x \in \allex, \afeat \in x \iff \afeat' \in x$, or $\forall x \in \allex, \afeat \in x \iff \afeat' \not\in x$. 
%We simply remove such redundant features. 
They can be found by comparing pairs of rows of the data set via bitset operations, and therefore in time $O(\numex\numfeat^2)$.

Removing redundant features may appear very naive, however, it turns out that the binarization techniques (one-hot encoding) used to turn general data sets into binary data sets are often not optimized and many redundant features do exist. The number of features ($\numfeat$) has a huge impact on the complexity:
 %of the algorithm since 
 the branching factor of the algorithm is indeed
 %in the tree representing the search space is 
 $2\numfeat$ (see Figure~\ref{fig:searchtree}).

Moreover, at every branch, ``informationless'' features (i.e., features $\afeat$ such that $(\forall x \in \posex[\abranch] \afeat \in x) \iff (\forall x \in \negex[\abranch] \afeat \in x)$) can be suppressed at no additional cost since this can be detected when computing the feature ordering criterion.




\section{Experimental Results}
\label{sec:exp}

All experiments were run
on 4 cluster nodes, each with 36 Intel Xeon CPU E5-2695 v4 2.10GHz cores
running Linux Ubuntu 16.04.4. Sources were compiled using g++8. 
Every algorithm was run until completion or until reaching a time limit of one hour, and within a memory limit of 50GB.


We used a collection of 58 data sets formed by the union of the data sets from related work~\cite{narodytska2018learning,dl85,verwer2019learning}, to which we added extra data sets (\texttt{bank}, \texttt{titanic}, \texttt{surgical-deepnet} and \texttt{weather-aus}, as well as \texttt{mnist\_0}, \texttt{adult\_discretized}, \texttt{compas\_discretized} and \texttt{taiwan\_binarised}). Further description of the data sets as well as the raw data from our experimental results are given in appendix. For reason of space, we present aggregated results in this section.



\subsection{Computing optimal classifiers}

We first compare \budalg to state-of-the-art algorithms, \murtree~\cite{DBLP:journals/corr/abs-2007-12652} and \dleight~\cite{dl85}, as well as the best MIP (\binoct)~\cite{verwer2019learning} and CP (\cp)~\cite{verhaeghe2019learning} models, for computing and proving optimal trees. 

The data sets in Table~\ref{tab:summaryaccsmall} are organized 
in two classes according to
the size \numfeat\ of their feature set.
Every method is run with an upper bound $\mdepth$ on the tree depth shown in the first column. 
We report for both classes and for every depth: the ratio of optimality proofs (opt.); the average classification error (error); and %the average accuracy (acc.), 
 %as well as 
 the average cpu time (cpu) to prove optimality.
Since \dleight and \binoct exceed the memory limit of 50GB in some cases, we also provide, for those two methods, the ratio of runs where at least one solution was found (sol.). For the same reason, we give their classification error, marked by a ``$^*$'', as the average marginal increase over \budalg's on these ``successful'' data sets. 
Similarly, the CPU time for all other methods is given as the average marginal increase over \budalg's on data sets for which both methods prove optimality.

\budalg is comparable to \murtree for the number of optimality proofs.
% The number of optimality proofs is similar for \budalg and \murtree.
It is slightly less efficient for  $\mdepth \leq 7$, but slightly more for $\mdepth = 10$. The gap on shallow trees can be explained by \murtree's caching, and because it puts less emphasis on finding good trees faster, but rather tries to exhaust the search space faster. The gap on deep trees can be partly explained by the removal of inconsistent datapoints: whereas \budalg can stop searching when the overall classification error reaches the number of inconsistent datapoints, \murtree must exhaust the search space. Despite what a quick look at Table~\ref{tab:summaryaccsmall} may suggest, both methods have similar speed. The large gaps are either due to the same phenomenon described above (when in favor or \budalg), or due to a few data sets, e.g. \texttt{mnist\_0}, where caching is probably helpful (when in favor or \murtree).

When not proving optimality, however, \budalg is significantly better than \murtree, especially as the depth and the number of features grow.
 % both algorithms find trees of similar qualities for $\mdepth \leq 5$ and $\numfeat < 100$, however, \budalg is significantly better as these parameters grow.
All other methods are systematically outperformed. \cp has good results on very shallow trees ($\mdepth \leq 4$) but is ineffective for deeper tree. Indeed, the quality of the tree actually \emph{decreases} when $\mdepth$ increases! \dleight can also find optimal trees in most cases 
for low values of \numfeat\ and $\mdepth$.
% \numfeat) is low, and for small values of $\mdepth$.
When, $\numfeat$ grows, however, it often exceeds the memory limit of 50GB (whereas \budalg does not require more memory than the size of the data set). Finally, \binoct does not produce a single optimality proof and very often exceeds the memory limit.%\footnote{In the experiments in \cite{verwer2019learning} not all datapoints were used.}
 

% As \dleight does not provide a solution for every data set (on some instance it goes over the memory limit of 50GB), we provide the number of data sets for which a solution was returned (sol.). Moreover,
% instead of absolute values, we provide the average relative difference in error and accuracy w.r.t. \budalg, however, and only for the data sets where a decision tree was found. Similarly, we report the average cpu time ratio w.r.t. \budalg, however, only for instances which were proven optimal by both algorithms\footnote{every instance proven optimal by \dleight is also proven optimal by \budalg and \murtree}.


\begin{table}[htbp]
\begin{center}
\begin{footnotesize}
\tabcolsep=3.75pt
\begin{tabular}{lrrrrrrrrrrrrrrr}
\toprule
\multirow{2}{*}{$\mdepth$}&  \multicolumn{3}{c}{\budalg} & \multicolumn{3}{c}{\murtree} & \multicolumn{3}{c}{\cp} & \multicolumn{4}{c}{\dleight} & \multicolumn{2}{c}{\binoct}\\
\cmidrule(rr){2-4}\cmidrule(rr){5-7}\cmidrule(rr){8-10}\cmidrule(rr){11-14}\cmidrule(rr){15-16}
& \multicolumn{1}{c}{opt.} & \multicolumn{1}{c}{error} & \multicolumn{1}{c}{cpu} & \multicolumn{1}{c}{opt.} & \multicolumn{1}{c}{error} & \multicolumn{1}{c}{cpu$^*$} & \multicolumn{1}{c}{opt.} & \multicolumn{1}{c}{error} & \multicolumn{1}{c}{cpu$^*$} & \multicolumn{1}{c}{sol.} & \multicolumn{1}{c}{opt.} & \multicolumn{1}{c}{error$^*$} & \multicolumn{1}{c}{cpu$^*$} & \multicolumn{1}{c}{sol.} & \multicolumn{1}{c}{error$^*$} \\
\midrule

&\multicolumn{15}{c}{$\numfeat < 100$ (29 data sets)}\\
\midrule
\texttt{3} & 1.00 & 458 & 0.23 & 1.00 & 458 & $\mathsmaller{+}$0.31 & 1.00 & 458 & $\mathsmaller{+}$3.2 & 1.00 & 1.00 & 0 & $\mathsmaller{+}$2.5 & 0.52 & $\mathsmaller{+}$57\\
\texttt{4} & 1.00 & 412 & 14 & 1.00 & 412 & $\mathsmaller{+}$8.0 & 1.00 & 412 & $\mathsmaller{+}$115 & 1.00 & 1.00 & 0 & $\mathsmaller{+}$105 & 0.52 & $\mathsmaller{+}$89\\
\texttt{5} & 0.93 & 379 & 187 & 0.97 & 379 & -12 & 0.62 & 380 & $\mathsmaller{+}$121 & 0.76 & 0.66 & $\mathsmaller{+}$2.5 & $\mathsmaller{+}$2.0 & 0.52 & $\mathsmaller{+}$211\\
\texttt{7} & 0.66 & 329 & 81 & 0.69 & 348 & $\mathsmaller{+}$90 & 0.45 & 682 & $\mathsmaller{+}$193 & 0.66 & 0.55 & $\mathsmaller{+}$93 & $\mathsmaller{+}$6.7 & 0.52 & $\mathsmaller{+}$350\\
\texttt{10} & 0.79 & 270 & 85 & 0.52 & 320 & $\mathsmaller{+}$83 & 0.45 & 766 & $\mathsmaller{+}$2.6 & 0.62 & 0.52 & $\mathsmaller{+}$278 & $\mathsmaller{+}$49 & 0.41 & $\mathsmaller{+}$292\\
\midrule
&\multicolumn{15}{c}{$\numfeat \geq 100$ (29 data sets)}\\
\midrule
\texttt{3} & 0.86 & 1127 & 100 & 0.86 & 1156 & -42 & 0.72 & 1155 & $\mathsmaller{+}$256 & 0.76 & 0.66 & $\mathsmaller{+}$165 & $\mathsmaller{+}$247 & 0.62 & $\mathsmaller{+}$148\\
\texttt{4} & 0.55 & 979 & 662 & 0.72 & 1023 & $\mathsmaller{+}$64 & 0.28 & 1585 & $\mathsmaller{+}$576 & 0.48 & 0.24 & $\mathsmaller{+}$565 & $\mathsmaller{+}$258 & 0.62 & $\mathsmaller{+}$189\\
\texttt{5} & 0.34 & 870 & 452 & 0.34 & 947 & $\mathsmaller{+}$99 & 0.14 & 1870 & $\mathsmaller{+}$11 & 0.34 & 0.10 & $\mathsmaller{+}$1136 & $\mathsmaller{+}$12 & 0.62 & $\mathsmaller{+}$329\\
\texttt{7} & 0.31 & 688 & 11 & 0.31 & 791 & $\mathsmaller{+}$7.4 & 0.28 & 1857 & $\mathsmaller{+}$571 & 0.34 & 0.14 & $\mathsmaller{+}$1805 & $\mathsmaller{+}$793 & 0.55 & $\mathsmaller{+}$510\\
\texttt{10} & 0.45 & 550 & 101 & 0.41 & 659 & $\mathsmaller{+}$19 & 0.38 & 1827 & $\mathsmaller{+}$85 & 0.45 & 0.28 & $\mathsmaller{+}$1612 & $\mathsmaller{+}$183 & 0.21 & $\mathsmaller{+}$375\\
\bottomrule
\end{tabular}

\end{footnotesize}
\end{center}
\caption{\label{tab:summaryaccsmall} Comparison with the state of the art}
\end{table}



\subsection{Computing accurate classifiers efficiently}

Next, we shift our focus to how fast can we obtain accurate trees and how fast can we improve the accuracy over basic solutions found by heuristics.
We use a well known heuristic as baseline: \cart (we ran its implementation in scikit-learn).
Here we report the average error after a given period of time (3 seconds, 10 seconds, 1 minute or 5 minutes), both for \murtree and \budalg in Table~\ref{tab:summaryspeed}.


% \medskip

We can see that the first solution found by \budalg has comparable accuracy to the one found by \cart. The implementation of \cart in scikit-learn does not seem to be very efficient computationally. However, this is not so relevant as it is clear that one greedy run of the heuristic can be implemented to be as fast as the first dive of \budalg. The point of this experiment is threefold. Firstly, it shows that the first solution is very similar to that found by \cart. There is actually a slight advantage for \budalg, which can
be explained by the small difference in the heuristic selection of features: whereas \cart systematically selects the feature with minimum Gini impurity, \budalg does so for all \emph{but the deepest feature test}, for which it selects the feature with least classification error. 
Secondly, this first tree
is found extremely quickly, and there is no scaling issue with respect to the depth of the tree or with respect to the size of the data set. Thirdly, even for large data sets and deep trees, the accuracy of the initial classifier can be significantly improved given a reasonable computation time.

%Moreover, although we would need larger data sets to be confident about that, it seems that our algorithm is faster than \cart to find this first decision tree. %One can conjecture that \cart uses more sophisticated heuristic choices to explain these two observations.

%Then, in most cases, it is possible to improve the first solution significantly within a few seconds. Notice that for larger depth, improving the initial solution is harder and the 3s time limit is comparatively tighter than for smaller trees, so the gain of \budalg over \cart is more sensible for small trees.


\begin{table}[htbp]
\begin{center}
\begin{footnotesize}
\tabcolsep=5pt
\begin{tabular}{lrrrrrrrrrrrrrr}
\toprule
&  \multicolumn{6}{c}{\budalg} & \multicolumn{6}{c}{\murtree} & \multicolumn{2}{c}{\cart}\\
\cmidrule(rr){2-7}\cmidrule(rr){8-13}\cmidrule(rr){14-15}
& \multicolumn{1}{c}{cpu} & \multicolumn{1}{c}{first} & \multicolumn{1}{c}{$\leq$3s} & \multicolumn{1}{c}{$\leq$10s} & \multicolumn{1}{c}{$\leq$1m} & \multicolumn{1}{c}{$\leq$5m} & \multicolumn{1}{c}{cpu} & \multicolumn{1}{c}{first} & \multicolumn{1}{c}{$\leq$3s} & \multicolumn{1}{c}{$\leq$10s} & \multicolumn{1}{c}{$\leq$1m} & \multicolumn{1}{c}{$\leq$5m} & \multicolumn{1}{c}{cpu} & \multicolumn{1}{c}{first} \\
\midrule

\texttt{D = 3} & \textbf{0.04} & \textbf{1446} & \textbf{1378} & \textbf{1348} & \textbf{1330} & \textbf{1329} & 0.05 & 2860 & 1643 & 1378 & 1362 & 1360 & 1.83 & 1504\\
\texttt{D = 4} & \textbf{0.04} & \textbf{1266} & \textbf{1202} & \textbf{1186} & \textbf{1167} & \textbf{1156} & 0.05 & 2860 & 2064 & 1248 & 1244 & 1217 & 2.11 & 1284\\
\texttt{D = 5} & \textbf{0.04} & \textbf{1126} & \textbf{1070} & \textbf{1063} & \textbf{1051} & \textbf{1028} & 0.05 & 2860 & 1962 & 1194 & 1186 & 1144 & 2.93 & 1152\\
\texttt{D = 6} & \textbf{0.04} & \textbf{993} & \textbf{949} & \textbf{944} & \textbf{936} & \textbf{902} & 0.05 & 2860 & 1168 & 1108 & 1098 & 1067 & 2.87 & 1011\\
\texttt{D = 7} & \textbf{0.04} & \textbf{882} & \textbf{848} & \textbf{841} & \textbf{833} & \textbf{812} & 0.05 & 2860 & 1728 & 1025 & 1014 & 991 & 3.84 & 901\\
\texttt{D = 8} & \textbf{0.05} & \textbf{782} & \textbf{758} & \textbf{749} & \textbf{735} & \textbf{720} & 0.05 & 2860 & 1616 & 917 & 907 & 893 & 3.66 & 795\\
\texttt{D = 9} & 0.05 & \textbf{706} & \textbf{688} & \textbf{683} & \textbf{671} & \textbf{654} & \textbf{0.05} & 2860 & 875 & 833 & 818 & 810 & 4.10 & 720\\
\texttt{D = 10} & 0.05 & \textbf{636} & \textbf{618} & \textbf{617} & \textbf{600} & \textbf{587} & \textbf{0.05} & 2860 & 790 & 749 & 741 & 728 & 5.02 & 652\\
\bottomrule
\end{tabular}

\end{footnotesize}
\end{center}
\caption{\label{tab:summaryspeed} Comparison with state the of the art: computing accurate trees}
\end{table}


Figure~\ref{fig:cactus} reports the evolution of the average accuracy (across all 58 data sets) over time, giving a good view of the difference between \murtree and \budalg during search. The accuracy of the tree returned by \cart is given for reference. 
We can see in those graphs that \murtree finds an initial tree extremely quickly, although its accuracy is very low. This is because \murtree shows progress even when the tree is not complete, e.g., the first solution is always a single node with the most promising feature. We can see in Table~\ref{tab:summaryspeed} that this is indeed always the same first tree, irrespective of the depth.



\begin{figure}
	\subfloat[depth=3]{\cactus{Average Accuracy}{CPU time}{\budalg, \murtree, \cart}{{{(0.8662064701436945, 0) [a] 
(0.8700043377060156, 0.001) [a] 
(0.8702714609936868, 0.002) [a] 
(0.8765154118753802, 0.003) [a] 
(0.8767551379027775, 0.006) [a] 
(0.8767811653000378, 0.008) [a] 
(0.8767976036562022, 0.009) [a] 
(0.8831415798503431, 0.01) [a] 
(0.8831813058777404, 0.011) [a] 
(0.8837758264256855, 0.012) [a] 
(0.8867170325349597, 0.015) [a] 
(0.8867184023979734, 0.017) [a] 
(0.8867197722609871, 0.019) [a] 
(0.8879462562792519, 0.02) [a] 
(0.8879489960052793, 0.021) [a] 
(0.8879613247724026, 0.025) [a] 
(0.8880092699778821, 0.026) [a] 
(0.8880311877861012, 0.027) [a] 
(0.8880846124436355, 0.028) [a] 
(0.8884740644984299, 0.03) [a] 
(0.8885261192929504, 0.033) [a] 
(0.8885617357313066, 0.034) [a] 
(0.8885740644984299, 0.035) [a] 
(0.8887761192929505, 0.04) [a] 
(0.8888939275121286, 0.043) [a] 
(0.8889720097039094, 0.044) [a] 
(0.8889774891559642, 0.048) [a] 
(0.8890822836765121, 0.05) [a] 
(0.8890850234025395, 0.058) [a] 
(0.8891639275121286, 0.06) [a] 
(0.8892317357313066, 0.08) [a] 
(0.889264475457334, 0.09) [a] 
(0.8892740644984299, 0.096) [a] 
(0.8892768042244573, 0.098) [a] 
(0.8893884480600738, 0.11) [a] 
(0.8894048864162382, 0.138) [a] 
(0.8894499549093888, 0.14) [a] 
(0.8894526946354162, 0.155) [a] 
(0.8894787220326765, 0.157) [a] 
(0.8895557083340464, 0.16) [a] 
(0.8895598179230875, 0.167) [a] 
(0.8895625576491148, 0.168) [a] 
(0.8895639275121285, 0.174) [a] 
(0.8895652973751422, 0.176) [a] 
(0.8895689960052792, 0.2) [a] 
(0.8904115342533334, 0.22) [a] 
(0.8904653698697718, 0.23) [a] 
(0.8904749589108677, 0.239) [a] 
(0.89049619178758, 0.24) [a] 
(0.8905057808286759, 0.248) [a] 
(0.8905113972670321, 0.25) [a] 
(0.8905144109656622, 0.26) [a] 
(0.890519890417717, 0.305) [a] 
(0.8919044380401684, 0.311) [a] 
(0.8919427942045519, 0.32) [a] 
(0.8919681366703054, 0.33) [a] 
(0.8920037531086615, 0.339) [a] 
(0.8920112873552368, 0.34) [a] 
(0.8920126572182505, 0.359) [a] 
(0.892029506533319, 0.36) [a] 
(0.8920541640675655, 0.403) [a] 
(0.8920671777661957, 0.41) [a] 
(0.8920740270812642, 0.492) [a] 
(0.8920767668072915, 0.493) [a] 
(0.8921537531086614, 0.51) [a] 
(0.8922153969442779, 0.539) [a] 
(0.8951170574714661, 0.54) [a] 
(0.8951204821290003, 0.55) [a] 
(0.8951276054166716, 0.56) [a] 
(0.8951307561016031, 0.57) [a] 
(0.8951362355536578, 0.62) [a] 
(0.8952143177454387, 0.64) [a] 
(0.8953047287043429, 0.739) [a] 
(0.895557123287671, 0.83) [a] 
(0.8955573972602737, 0.86) [a] 
(0.8955628767123285, 0.907) [a] 
(0.8955630136986299, 0.96) [a] 
(0.8955709589041093, 1.17) [a] 
(0.8955750684931504, 1.213) [a] 
(0.8955764383561641, 1.219) [a] 
(0.8955765753424655, 1.24) [a] 
(0.8955771232876709, 1.25) [a] 
(0.8955780821917805, 1.26) [a] 
(0.8955808219178079, 1.39) [a] 
(0.8956130136986298, 1.43) [a] 
(0.8956131506849312, 1.44) [a] 
(0.8956143835616435, 1.65) [a] 
(0.8956145205479449, 1.78) [a] 
(0.8956150684931503, 1.8) [a] 
(0.895615342465753, 2.07) [a] 
(0.895617534246575, 2.37) [a] 
(0.8956631506849312, 2.39) [a] 
(0.8956868493150681, 2.72) [a] 
(0.8957173972602737, 2.73) [a] 
(0.8957254794520545, 2.76) [a] 
(0.8957268493150682, 2.81) [a] 
(0.8957753424657532, 2.83) [a] 
(0.8957761643835614, 2.84) [a] 
(0.8957912328767121, 2.85) [a] 
(0.8957931506849314, 3.35) [a] 
(0.8957932876712328, 3.52) [a] 
(0.8957946575342465, 3.54) [a] 
(0.8957960273972602, 3.541) [a] 
(0.8957965753424656, 3.65) [a] 
(0.895796712328767, 3.66) [a] 
(0.8957999999999999, 4.07) [a] 
(0.8959006849315068, 5.84) [a] 
(0.8959253424657534, 5.926) [a] 
(0.8959561643835616, 5.96) [a] 
(0.8959639726027397, 6.09) [a] 
(0.8959742465753425, 6.1) [a] 
(0.8959761643835618, 6.47) [a] 
(0.8959808219178084, 6.71) [a] 
(0.8959849315068494, 6.714) [a] 
(0.8959920547945207, 6.74) [a] 
(0.8959967123287673, 6.98) [a] 
(0.8959995890410961, 7.17) [a] 
(0.8960178082191783, 7.21) [a] 
(0.896032876712329, 7.22) [a] 
(0.8960349315068495, 7.27) [a] 
(0.8960395890410962, 7.39) [a] 
(0.8960442465753428, 7.73) [a] 
(0.8960757534246578, 7.769) [a] 
(0.8960880821917812, 7.772) [a] 
(0.89609095890411, 7.82) [a] 
(0.8960956164383566, 8.06) [a] 
(0.8960993150684935, 8.18) [a] 
(0.8961031506849318, 8.38) [a] 
(0.8961095890410963, 8.71) [a] 
(0.8961105479452058, 8.74) [a] 
(0.8961115068493154, 8.81) [a] 
(0.8961152054794523, 9) [a] 
(0.8961190410958907, 9.17) [a] 
(0.8961217808219181, 9.32) [a] 
(0.8961271232876715, 9.54) [a] 
(0.8961291780821921, 9.55) [a] 
(0.8961305479452057, 9.56) [a] 
(0.8961375342465756, 9.57) [a] 
(0.8961504109589044, 9.58) [a] 
(0.896159178082192, 9.61) [a] 
(0.8961630136986304, 9.86) [a] 
(0.896167671232877, 10.04) [a] 
(0.8961723287671236, 10.32) [a] 
(0.8961769863013702, 10.56) [a] 
(0.896179863013699, 10.77) [a] 
(0.896183561643836, 10.93) [a] 
(0.8962205479452058, 11.06) [a] 
(0.8962209589041099, 11.35) [a] 
(0.8962212328767126, 11.38) [a] 
(0.8962357534246578, 13.79) [a] 
(0.8962360273972605, 13.8) [a] 
(0.8962401369863016, 13.83) [a] 
(0.896240273972603, 14.27) [a] 
(0.8962512328767126, 14.87) [a] 
(0.8962963013698633, 16.51) [a] 
(0.896338767123288, 16.52) [a] 
(0.8963652054794523, 17.05) [a] 
(0.8963653424657537, 17.57) [a] 
(0.8963673972602743, 17.6) [a] 
(0.896380273972603, 17.61) [a] 
(0.8963816438356167, 17.65) [a] 
(0.896382465753425, 17.68) [a] 
(0.89641397260274, 19.18) [a] 
(0.8964268493150688, 22.11) [a] 
(0.8964405479452058, 22.31) [a] 
(0.8964426027397263, 24.38) [a] 
(0.8964430136986304, 28.2) [a] 
(0.896449178082192, 28.55) [a] 
(0.8964645205479455, 36.23) [a] 
(0.8964867123287674, 36.84) [a] 
(0.8964980821917811, 37.77) [a] 
(0.8965009589041099, 39.14) [a] 
(0.8965243835616441, 40.23) [a] 
(0.8965247945205482, 40.39) [a] 
(0.896535068493151, 41.84) [a] 
(0.8965378082191784, 42.39) [a] 
(0.8965397260273976, 42.4) [a] 
(0.8965406849315072, 42.41) [a] 
(0.8965426027397264, 42.42) [a] 
(0.8965480821917812, 48.04) [a] 
(0.8965504109589044, 48.05) [a] 
(0.8965541095890414, 48.51) [a] 
(0.8965550684931509, 48.62) [a] 
(0.8965560273972605, 48.93) [a] 
(0.8965597260273974, 48.99) [a] 
(0.896564383561644, 49.44) [a] 
(0.8965672602739728, 49.95) [a] 
(0.8965887671232879, 50.22) [a] 
(0.8965891780821921, 72.07) [a] 
(0.8965895890410962, 72.18) [a] 
(0.8965952054794524, 102.49) [a] 
(0.8965971232876716, 108.77) [a] 
(0.896597260273973, 108.8) [a] 
(0.8966043835616443, 146.63) [a] 
(0.896605753424658, 180.9) [a] 
(0.896609863013699, 198.87) [a] 
(0.8966123287671237, 235.62) [a] 
(0.8966132876712333, 259.14) [a] 
(0.8966217808219182, 570.34) [a] 
(0.896628356164384, 953.22) [a] 
(0.8966320547945209, 953.29) [a] 
(0.8966330136986305, 953.31) [a] 
},{(0.817408444061846, 0) [b] 
(0.8384010790637917, 0.001) [b] 
(0.8391064481252993, 0.002) [b] 
(0.8417932760083486, 0.003) [b] 
(0.8458168057858171, 0.004) [b] 
(0.8479885373510532, 0.005) [b] 
(0.8481468078130099, 0.006) [b] 
(0.8491866560216892, 0.007) [b] 
(0.8503493854895878, 0.008) [b] 
(0.8520215428122333, 0.009) [b] 
(0.8544867603200413, 0.01) [b] 
(0.8546307386305436, 0.011) [b] 
(0.8592859266156418, 0.012) [b] 
(0.8610235662789468, 0.013) [b] 
(0.8612124198702003, 0.015) [b] 
(0.8615598602431475, 0.016) [b] 
(0.8616379151157229, 0.017) [b] 
(0.8616980847942162, 0.019) [b] 
(0.8617980747952161, 0.02) [b] 
(0.8618349983265826, 0.022) [b] 
(0.8648811823884713, 0.023) [b] 
(0.8651471494591998, 0.024) [b] 
(0.8661520019520069, 0.028) [b] 
(0.8661857424203246, 0.032) [b] 
(0.8662194828886424, 0.033) [b] 
(0.8667971139432578, 0.039) [b] 
(0.8673560180528468, 0.046) [b] 
(0.8676655431245913, 0.047) [b] 
(0.8681532136417611, 0.049) [b] 
(0.8685642373148075, 0.05) [b] 
(0.8686131054618829, 0.051) [b] 
(0.8688525212211616, 0.052) [b] 
(0.8701710338855647, 0.053) [b] 
(0.8701910904450625, 0.06) [b] 
(0.870353525207397, 0.064) [b] 
(0.8710689275549766, 0.066) [b] 
(0.8712207596624062, 0.067) [b] 
(0.8712260609898586, 0.071) [b] 
(0.871273667002498, 0.075) [b] 
(0.8723982050290713, 0.077) [b] 
(0.872496972581783, 0.082) [b] 
(0.872710457726775, 0.084) [b] 
(0.873065664596167, 0.085) [b] 
(0.8730866425902511, 0.097) [b] 
(0.8730874079623674, 0.101) [b] 
(0.873703076732569, 0.108) [b] 
(0.8743957040990458, 0.109) [b] 
(0.8763582296855675, 0.11) [b] 
(0.87643919085659, 0.112) [b] 
(0.8774396526081676, 0.114) [b] 
(0.8774449539356199, 0.116) [b] 
(0.8774457193077362, 0.121) [b] 
(0.8778871917481699, 0.122) [b] 
(0.8782269548406317, 0.123) [b] 
(0.8782659822769194, 0.13) [b] 
(0.8782956329915017, 0.132) [b] 
(0.8784016185244764, 0.133) [b] 
(0.8784225965185606, 0.134) [b] 
(0.878825336244588, 0.141) [b] 
(0.8788295551020371, 0.148) [b] 
(0.8789761304445028, 0.16) [b] 
(0.8790877807214814, 0.165) [b] 
(0.8793060081100991, 0.167) [b] 
(0.8798513564188701, 0.186) [b] 
(0.8799937664202943, 0.221) [b] 
(0.8800567004025469, 0.231) [b] 
(0.8801869904038498, 0.243) [b] 
(0.8804392470036503, 0.259) [b] 
(0.8809428249342071, 0.282) [b] 
(0.880984871622307, 0.297) [b] 
(0.8810618302185822, 0.326) [b] 
(0.8810636581649576, 0.334) [b] 
(0.8810901545482013, 0.335) [b] 
(0.8812440717407517, 0.344) [b] 
(0.8812540215782632, 0.383) [b] 
(0.8812731996604549, 0.455) [b] 
(0.8812822896605458, 0.474) [b] 
(0.8813019243637421, 0.523) [b] 
(0.8813788829600173, 0.543) [b] 
(0.8813829925490584, 0.727) [b] 
(0.8817224902659534, 0.818) [b] 
(0.8824706637819352, 0.822) [b] 
(0.8824966327490195, 0.907) [b] 
(0.882713984347193, 0.936) [b] 
(0.8851077697713563, 0.98) [b] 
(0.8851847283676315, 1.027) [b] 
(0.885602079965805, 1.072) [b] 
(0.8856025365868095, 1.095) [b] 
(0.8856131392417144, 1.106) [b] 
(0.8857181569248389, 1.167) [b] 
(0.8857951155211141, 1.184) [b] 
(0.8862000304740875, 1.251) [b] 
(0.8862301674603889, 1.253) [b] 
(0.886251788748091, 1.354) [b] 
(0.8863298436206664, 1.408) [b] 
(0.8863545011549129, 1.459) [b] 
(0.886388477464159, 1.759) [b] 
(0.8863925870532001, 1.792) [b] 
(0.8864877925326522, 1.876) [b] 
(0.8869433218547471, 1.887) [b] 
(0.8898446457258623, 1.936) [b] 
(0.8900375681002916, 2.06) [b] 
(0.8902814037167299, 2.109) [b] 
(0.890286887555856, 2.154) [b] 
(0.8906576638115639, 2.247) [b] 
(0.8906622300216095, 2.263) [b] 
(0.8910229606152168, 2.279) [b] 
(0.8912784400672716, 2.28) [b] 
(0.8912969332179566, 2.301) [b] 
(0.8916309514827967, 2.313) [b] 
(0.891631179793299, 2.32) [b] 
(0.8917688510261756, 2.353) [b] 
(0.8917709058206962, 2.407) [b] 
(0.8918921386974085, 2.471) [b] 
(0.8920572071905591, 2.482) [b] 
(0.8921679377841665, 2.497) [b] 
(0.8923040108435272, 2.603) [b] 
(0.8923346044508331, 2.611) [b] 
(0.8923476181494633, 2.726) [b] 
(0.8923777551357647, 2.732) [b] 
(0.8924028692910159, 2.823) [b] 
(0.8924259286517464, 2.871) [b] 
(0.8924752437202396, 2.892) [b] 
(0.8925074355010615, 2.944) [b] 
(0.8925104128168921, 3.062) [b] 
(0.8925279927255679, 3.103) [b] 
(0.8925318740041067, 3.17) [b] 
(0.8925366685246546, 3.237) [b] 
(0.892551508707303, 3.319) [b] 
(0.8925521936388099, 3.96) [b] 
(0.892643085023605, 4.039) [b] 
(0.8926622631057968, 4.282) [b] 
(0.8926976512336506, 4.557) [b] 
(0.892712491416299, 4.645) [b] 
(0.8927321261194954, 4.753) [b] 
(0.8927638612793127, 5.057) [b] 
(0.8927668385951434, 5.623) [b] 
(0.893709052419738, 5.752) [b] 
(0.8937218378078658, 7.42) [b] 
(0.8937225227393727, 7.815) [b] 
(0.8937490191226164, 8.91) [b] 
(0.8937737291657045, 9.007) [b] 
(0.8938046167195647, 9.064) [b] 
(0.8938482240255008, 10.628) [b] 
(0.893854845030067, 11.149) [b] 
(0.8938735664912542, 11.604) [b] 
(0.8939168090666584, 12.525) [b] 
(0.8939202337241926, 13.263) [b] 
(0.8939222885187131, 15.528) [b] 
(0.8939593535833453, 18.716) [b] 
(0.8940600385148522, 21.544) [b] 
(0.8940604951358567, 22.093) [b] 
(0.8941042677700876, 22.653) [b] 
(0.8941145417426903, 23.043) [b] 
(0.894148103386526, 23.828) [b] 
(0.894153141963128, 24.777) [b] 
(0.8941537717852033, 25.594) [b] 
(0.8941690685888563, 27.185) [b] 
(0.8941720366253859, 27.745) [b] 
(0.8942152792007901, 31.283) [b] 
(0.8942230417578677, 32.109) [b] 
(0.8942233566689053, 37.731) [b] 
(0.8942542442227654, 42.9) [b] 
(0.8942551574647746, 50.679) [b] 
(0.8942583065751509, 58.189) [b] 
(0.8942653842007217, 73.439) [b] 
(0.8942674389952422, 76.434) [b] 
(0.8942699504107673, 98.767) [b] 
(0.8942740599998084, 99.756) [b] 
(0.8942825074883929, 311.363) [b] 
(0.8946803153331636, 489.37) [b] 
(0.896160949650166, 507.799) [b] 
(0.8963453452864479, 564.835) [b] 
(0.8963509613971976, 577.042) [b] 
(0.8963514430886359, 611.124) [b] 
(0.896382315217506, 1117.87) [b] 
(0.8963828414469753, 1133.81) [b] 
(0.8963940736684748, 3522.86) [b] 
},{(0.8813585753424655, 0.001) [c] 
(0.8813585753424655, 1.4935057671232883) [c] 
(0.8813585753424655, 3600) [c] 
}}}{legend pos=north west}}
	\subfloat[depth=7]{\cactus{Average Accuracy}{CPU time}{\budalg, \murtree, \cart}{{{(0.9037554038841665, 0) [a] 
(0.9041101984047145, 0.001) [a] 
(0.9094849275422694, 0.002) [a] 
(0.9096041056244613, 0.003) [a] 
(0.9098328727477489, 0.004) [a] 
(0.914078691554262, 0.005) [a] 
(0.920268328606179, 0.006) [a] 
(0.9203765477842611, 0.007) [a] 
(0.9209059998390556, 0.008) [a] 
(0.920936136825357, 0.009) [a] 
(0.9262416255124435, 0.01) [a] 
(0.926252584416553, 0.011) [a] 
(0.9262553241425804, 0.012) [a] 
(0.9263018994850462, 0.013) [a] 
(0.9263429953754572, 0.014) [a] 
(0.9263621734576489, 0.015) [a] 
(0.9264279268823064, 0.016) [a] 
(0.9293869412107586, 0.017) [a] 
(0.92941707819706, 0.018) [a] 
(0.9294732425806216, 0.019) [a] 
(0.9335256170098453, 0.02) [a] 
(0.9335379457769686, 0.021) [a] 
(0.9335393156399823, 0.022) [a] 
(0.9335900005714892, 0.023) [a] 
(0.9336064389276536, 0.024) [a] 
(0.9336105485166947, 0.026) [a] 
(0.9336324663249138, 0.027) [a] 
(0.9336352060509412, 0.028) [a] 
(0.9336502745440919, 0.029) [a] 
(0.9365284937221742, 0.03) [a] 
(0.9365298635851879, 0.033) [a] 
(0.9365723293386125, 0.034) [a] 
(0.9365736992016261, 0.038) [a] 
(0.9365805485166946, 0.039) [a] 
(0.9378299549093887, 0.04) [a] 
(0.9378395439504845, 0.042) [a] 
(0.9378628316217175, 0.043) [a] 
(0.9378751603888408, 0.044) [a] 
(0.9378970781970599, 0.046) [a] 
(0.9378998179230873, 0.049) [a] 
(0.938016804224457, 0.05) [a] 
(0.9380318727176077, 0.052) [a] 
(0.9380346124436351, 0.055) [a] 
(0.9380483110737721, 0.058) [a] 
(0.9380743384710324, 0.059) [a] 
(0.9381155713477446, 0.06) [a] 
(0.9381292699778816, 0.061) [a] 
(0.9381429686080186, 0.062) [a] 
(0.9381580371011693, 0.064) [a] 
(0.9381703658682926, 0.066) [a] 
(0.9381717357313063, 0.067) [a] 
(0.93817310559432, 0.069) [a] 
(0.9383406398408953, 0.07) [a] 
(0.9384474891559638, 0.071) [a] 
(0.9384584480600734, 0.072) [a] 
(0.9384981740874706, 0.073) [a] 
(0.9385242014847309, 0.074) [a] 
(0.9385379001148679, 0.075) [a] 
(0.9385735165532241, 0.077) [a] 
(0.9385913247724021, 0.078) [a] 
(0.9385926946354158, 0.079) [a] 
(0.938823516553224, 0.08) [a] 
(0.9388454343614432, 0.081) [a] 
(0.9388481740874706, 0.083) [a] 
(0.9388618727176076, 0.084) [a] 
(0.9388837905258267, 0.085) [a] 
(0.9388879001148678, 0.087) [a] 
(0.9389318727176075, 0.09) [a] 
(0.9389346124436349, 0.091) [a] 
(0.9389373521696622, 0.092) [a] 
(0.9389442014847307, 0.095) [a] 
(0.9389661192929498, 0.096) [a] 
(0.9389674891559635, 0.097) [a] 
(0.9389888590189772, 0.1) [a] 
(0.938998448060073, 0.101) [a] 
(0.9391903658682924, 0.11) [a] 
(0.9391958453203472, 0.115) [a] 
(0.9391972151833609, 0.116) [a] 
(0.9391992699778814, 0.12) [a] 
(0.9392362562792512, 0.124) [a] 
(0.9392376261422649, 0.128) [a] 
(0.9392395439504841, 0.13) [a] 
(0.9392429686080183, 0.14) [a] 
(0.93926214669021, 0.141) [a] 
(0.9393606398408949, 0.15) [a] 
(0.9393620097039086, 0.156) [a] 
(0.9394181740874703, 0.16) [a] 
(0.9394400918956894, 0.166) [a] 
(0.9394455713477442, 0.169) [a] 
(0.9394488590189771, 0.17) [a] 
(0.9394502288819908, 0.171) [a] 
(0.9394817357313059, 0.174) [a] 
(0.9394940644984292, 0.179) [a] 
(0.9395076261422648, 0.18) [a] 
(0.9395309138134976, 0.181) [a] 
(0.9395325576491141, 0.19) [a] 
(0.9395339275121278, 0.193) [a] 
(0.9395613247724017, 0.195) [a] 
(0.939573653539525, 0.196) [a] 
(0.9395791329915798, 0.197) [a] 
(0.9396585850463742, 0.2) [a] 
(0.9396818727176071, 0.21) [a] 
(0.9396832425806207, 0.212) [a] 
(0.9396955713477441, 0.216) [a] 
(0.9396969412107578, 0.217) [a] 
(0.9397024206628125, 0.218) [a] 
(0.9397100918956893, 0.22) [a] 
(0.939711461758703, 0.229) [a] 
(0.9397162562792509, 0.23) [a] 
(0.9397217357313057, 0.238) [a] 
(0.9397306398408947, 0.24) [a] 
(0.9397320097039084, 0.244) [a] 
(0.9414771506916886, 0.25) [a] 
(0.9415149589108667, 0.26) [a] 
(0.9415334520615516, 0.27) [a] 
(0.9415350958971681, 0.28) [a] 
(0.9415363287738804, 0.29) [a] 
(0.9415364657601818, 0.3) [a] 
(0.9415652328834695, 0.307) [a] 
(0.9415775616505928, 0.308) [a] 
(0.9416364657601819, 0.309) [a] 
(0.9416367397327846, 0.31) [a] 
(0.9416545479519627, 0.311) [a] 
(0.9416548219245654, 0.32) [a] 
(0.9416756438423736, 0.33) [a] 
(0.9417118082259353, 0.34) [a] 
(0.9417131780889489, 0.341) [a] 
(0.9417176986368941, 0.35) [a] 
(0.9417187945273051, 0.36) [a] 
(0.9417624931574421, 0.38) [a] 
(0.9418403013766202, 0.39) [a] 
(0.9445248489990716, 0.397) [a] 
(0.9445608763963319, 0.4) [a] 
(0.9445651229716743, 0.41) [a] 
(0.944565396944277, 0.42) [a] 
(0.944607588725099, 0.43) [a] 
(0.9446089585881127, 0.432) [a] 
(0.9446103284511264, 0.435) [a] 
(0.9446166298209894, 0.44) [a] 
(0.9446179996840031, 0.444) [a] 
(0.9446488216018113, 0.45) [a] 
(0.944650191464825, 0.452) [a] 
(0.9446515613278387, 0.455) [a] 
(0.9446751229716742, 0.46) [a] 
(0.9446764928346879, 0.467) [a] 
(0.9475833588413282, 0.47) [a] 
(0.9475878793892734, 0.48) [a] 
(0.9475881533618761, 0.49) [a] 
(0.9476102081563966, 0.5) [a] 
(0.9476115780194103, 0.503) [a] 
(0.9476252766495473, 0.509) [a] 
(0.947626646512561, 0.51) [a] 
(0.9476389752796843, 0.516) [a] 
(0.9476397971974926, 0.52) [a] 
(0.9476452766495473, 0.529) [a] 
(0.9476470574714652, 0.53) [a] 
(0.9476508930879035, 0.54) [a] 
(0.9476511670605062, 0.55) [a] 
(0.9476521259646159, 0.56) [a] 
(0.9476576054166707, 0.567) [a] 
(0.9476582903481775, 0.58) [a] 
(0.9476589752796843, 0.59) [a] 
(0.9476633588413281, 0.6) [a] 
(0.9476647287043418, 0.605) [a] 
(0.9477519889783145, 0.63) [a] 
(0.9477611670605063, 0.64) [a] 
(0.9477629478824241, 0.65) [a] 
(0.9477643177454378, 0.651) [a] 
(0.9477780163755748, 0.653) [a] 
(0.9477913040468077, 0.67) [a] 
(0.947794043772835, 0.68) [a] 
(0.9477943177454378, 0.69) [a] 
(0.9478132218550268, 0.7) [a] 
(0.9478225369235199, 0.71) [a] 
(0.9478606191153007, 0.72) [a] 
(0.9478984273344789, 0.73) [a] 
(0.947900345142698, 0.74) [a] 
(0.9479004821289995, 0.75) [a] 
(0.9479039067865338, 0.77) [a] 
(0.9479081533618763, 0.78) [a] 
(0.9479439067865338, 0.8) [a] 
(0.9479452766495475, 0.809) [a] 
(0.9479629478824243, 0.81) [a] 
(0.947964317745438, 0.812) [a] 
(0.9479656876084517, 0.819) [a] 
(0.9479678793892736, 0.82) [a] 
(0.9479692492522873, 0.828) [a] 
(0.94798952322489, 0.83) [a] 
(0.9479908930879037, 0.832) [a] 
(0.9479922629509174, 0.833) [a] 
(0.9480230848687257, 0.84) [a] 
(0.948035413635849, 0.842) [a] 
(0.9480381533618764, 0.847) [a] 
(0.94803952322489, 0.849) [a] 
(0.9480550026769449, 0.85) [a] 
(0.948055413635849, 0.87) [a] 
(0.9480559615810544, 0.88) [a] 
(0.9480903451426982, 0.91) [a] 
(0.9481254136358489, 0.92) [a] 
(0.9481262355536572, 0.93) [a] 
(0.9481276054166708, 0.939) [a] 
(0.9481281533618763, 0.94) [a] 
(0.9481308930879037, 0.964) [a] 
(0.9481322629509173, 0.965) [a] 
(0.948133632813931, 0.969) [a] 
(0.9485330136986291, 0.97) [a] 
(0.9485343835616428, 0.979) [a] 
(0.9485361643835606, 0.98) [a] 
(0.9485387671232866, 0.99) [a] 
(0.9485826027397248, 1.018) [a] 
(0.9485879452054783, 1.02) [a] 
(0.9485961643835605, 1.024) [a] 
(0.9486057534246565, 1.028) [a] 
(0.9486061643835606, 1.03) [a] 
(0.948606301369862, 1.04) [a] 
(0.9486065753424647, 1.05) [a] 
(0.9486071232876702, 1.06) [a] 
(0.9486084931506839, 1.07) [a] 
(0.9486135616438345, 1.08) [a] 
(0.9486139726027386, 1.11) [a] 
(0.9486263013698619, 1.119) [a] 
(0.9486390410958894, 1.12) [a] 
(0.9486393150684921, 1.14) [a] 
(0.9486395890410948, 1.16) [a] 
(0.9486401369863002, 1.17) [a] 
(0.948641917808218, 1.18) [a] 
(0.9486420547945195, 1.19) [a] 
(0.9486427397260263, 1.2) [a] 
(0.948643013698629, 1.21) [a] 
(0.9486457534246564, 1.223) [a] 
(0.9486484931506838, 1.233) [a] 
(0.9486553424657522, 1.234) [a] 
(0.9486635616438345, 1.236) [a] 
(0.9486650684931496, 1.24) [a] 
(0.9486828767123276, 1.241) [a] 
(0.9486897260273961, 1.244) [a] 
(0.9486924657534235, 1.246) [a] 
(0.9486936986301358, 1.25) [a] 
(0.9486938356164372, 1.26) [a] 
(0.9487061643835604, 1.262) [a] 
(0.9487143835616425, 1.3) [a] 
(0.9487321917808206, 1.308) [a] 
(0.9487417808219164, 1.31) [a] 
(0.9487423287671218, 1.33) [a] 
(0.9487443835616424, 1.37) [a] 
(0.9487841095890396, 1.376) [a] 
(0.9487949315068477, 1.38) [a] 
(0.9488042465753408, 1.41) [a] 
(0.9488145205479435, 1.42) [a] 
(0.948815479452053, 1.45) [a] 
(0.9488161643835599, 1.46) [a] 
(0.9488164383561626, 1.49) [a] 
(0.9488178082191763, 1.501) [a] 
(0.94881917808219, 1.519) [a] 
(0.9488205479452037, 1.52) [a] 
(0.9488215068493132, 1.54) [a] 
(0.9488228767123269, 1.559) [a] 
(0.9488256164383543, 1.564) [a] 
(0.9488267123287653, 1.58) [a] 
(0.9488272602739707, 1.59) [a] 
(0.9488273972602721, 1.61) [a] 
(0.9488279452054775, 1.63) [a] 
(0.9488293150684912, 1.642) [a] 
(0.9488301369862995, 1.65) [a] 
(0.9488302739726009, 1.66) [a] 
(0.9488504109589022, 1.67) [a] 
(0.9488598630136967, 1.68) [a] 
(0.9488626027397241, 1.688) [a] 
(0.9488643835616419, 1.69) [a] 
(0.9488671232876693, 1.706) [a] 
(0.9488698630136967, 1.707) [a] 
(0.9489054794520528, 1.73) [a] 
(0.9489068493150665, 1.77) [a] 
(0.9489113698630118, 1.79) [a] 
(0.9489497260273954, 1.792) [a] 
(0.9489706849315049, 1.8) [a] 
(0.9489726027397241, 1.81) [a] 
(0.9489990410958884, 1.83) [a] 
(0.949019999999998, 1.87) [a] 
(0.9490213698630117, 1.873) [a] 
(0.9490843835616418, 1.88) [a] 
(0.9490857534246555, 1.881) [a] 
(0.9490884931506829, 1.883) [a] 
(0.949088904109587, 1.89) [a] 
(0.9490980821917788, 1.9) [a] 
(0.9490989041095871, 1.92) [a] 
(0.9491004109589021, 1.93) [a] 
(0.9491069863013678, 1.94) [a] 
(0.9491971232876691, 1.96) [a] 
(0.9492506849315048, 1.97) [a] 
(0.9492684931506828, 1.98) [a] 
(0.949452602739724, 2) [a] 
(0.9494861643835596, 2.01) [a] 
(0.9495039726027377, 2.02) [a] 
(0.9495053424657514, 2.029) [a] 
(0.9495232876712308, 2.04) [a] 
(0.9495234246575323, 2.05) [a] 
(0.949538493150683, 2.052) [a] 
(0.9495398630136966, 2.059) [a] 
(0.9495412328767103, 2.063) [a] 
(0.949542602739724, 2.067) [a] 
(0.9495968493150665, 2.08) [a] 
(0.9495972602739706, 2.12) [a] 
(0.9495975342465733, 2.13) [a] 
(0.9495976712328748, 2.16) [a] 
(0.9496004109589021, 2.162) [a] 
(0.9496041095890391, 2.17) [a] 
(0.9496042465753405, 2.18) [a] 
(0.9496056164383542, 2.204) [a] 
(0.9496234246575322, 2.21) [a] 
(0.9496247945205459, 2.224) [a] 
(0.9496252054794501, 2.24) [a] 
(0.9496279452054774, 2.248) [a] 
(0.9496297260273953, 2.25) [a] 
(0.9496330136986282, 2.26) [a] 
(0.9496332876712309, 2.28) [a] 
(0.9496334246575323, 2.31) [a] 
(0.9496335616438337, 2.32) [a] 
(0.9496472602739707, 2.33) [a] 
(0.9496482191780803, 2.36) [a] 
(0.9496499999999981, 2.37) [a] 
(0.9496513698630118, 2.39) [a] 
(0.9498052054794501, 2.4) [a] 
(0.9499571232876693, 2.43) [a] 
(0.9499836986301351, 2.45) [a] 
(0.9499973972602721, 2.48) [a] 
(0.9499976712328748, 2.49) [a] 
(0.9500153424657516, 2.51) [a] 
(0.950015890410957, 2.53) [a] 
(0.9500505479452036, 2.54) [a] 
(0.9500772602739707, 2.58) [a] 
(0.9500782191780802, 2.64) [a] 
(0.9500791780821898, 2.68) [a] 
(0.9500798630136966, 2.92) [a] 
(0.9500805479452035, 2.93) [a] 
(0.9500808219178062, 2.94) [a] 
(0.9500809589041076, 2.96) [a] 
(0.9500823287671213, 2.971) [a] 
(0.950083698630135, 2.974) [a] 
(0.9500850684931487, 2.998) [a] 
(0.950108904109587, 3) [a] 
(0.950132602739724, 3.01) [a] 
(0.9501417808219158, 3.02) [a] 
(0.9501420547945185, 3.05) [a] 
(0.9501430136986281, 3.1) [a] 
(0.9501447945205459, 3.11) [a] 
(0.9501471232876691, 3.15) [a] 
(0.950147808219176, 3.16) [a] 
(0.9501479452054774, 3.21) [a] 
(0.9501482191780801, 3.22) [a] 
(0.9501524657534226, 3.31) [a] 
(0.9501538356164363, 3.377) [a] 
(0.9501542465753404, 3.39) [a] 
(0.95015520547945, 3.41) [a] 
(0.9501553424657514, 3.42) [a] 
(0.9501567123287651, 3.456) [a] 
(0.9501594520547925, 3.459) [a] 
(0.9501608219178062, 3.475) [a] 
(0.9501621917808198, 3.483) [a] 
(0.9501635616438335, 3.494) [a] 
(0.9501663013698609, 3.503) [a] 
(0.9502198630136965, 3.52) [a] 
(0.9502209589041075, 3.53) [a] 
(0.9502236986301349, 3.535) [a] 
(0.9502250684931486, 3.543) [a] 
(0.9502253424657513, 3.56) [a] 
(0.9502254794520527, 3.57) [a] 
(0.9502432876712308, 3.6) [a] 
(0.9502435616438335, 3.65) [a] 
(0.9502613698630116, 3.68) [a] 
(0.9502617808219157, 3.74) [a] 
(0.9502624657534225, 3.79) [a] 
(0.9502720547945185, 3.915) [a] 
(0.9502775342465732, 3.92) [a] 
(0.9502984931506828, 3.94) [a] 
(0.9502987671232855, 3.95) [a] 
(0.9502990410958883, 3.96) [a] 
(0.950304520547943, 3.976) [a] 
(0.9503230136986279, 4) [a] 
(0.9503339726027374, 4.039) [a] 
(0.9503394520547922, 4.04) [a] 
(0.9503436986301347, 4.08) [a] 
(0.9503646575342443, 4.09) [a] 
(0.9503831506849292, 4.17) [a] 
(0.9503968493150662, 4.21) [a] 
(0.9504023287671209, 4.347) [a] 
(0.9504115068493127, 4.48) [a] 
(0.9504120547945182, 4.57) [a] 
(0.9504124657534223, 4.58) [a] 
(0.9504128767123264, 4.59) [a] 
(0.9504252054794498, 4.617) [a] 
(0.9504261643835593, 4.62) [a] 
(0.950427534246573, 4.64) [a] 
(0.9504282191780798, 4.65) [a] 
(0.950436438356162, 4.655) [a] 
(0.9504371232876688, 4.69) [a] 
(0.9504456164383538, 4.73) [a] 
(0.9504465753424634, 4.98) [a] 
(0.9504468493150661, 5.01) [a] 
(0.9504472602739702, 5.02) [a] 
(0.9504475342465729, 5.05) [a] 
(0.9504653424657511, 5.08) [a] 
(0.9504721917808195, 5.099) [a] 
(0.9504776712328743, 5.115) [a] 
(0.9504780821917784, 5.13) [a] 
(0.9504804109589017, 5.16) [a] 
(0.9504808219178058, 5.34) [a] 
(0.9504835616438332, 5.357) [a] 
(0.9504904109589016, 5.36) [a] 
(0.950493150684929, 5.361) [a] 
(0.9505141095890386, 5.39) [a] 
(0.9505143835616413, 5.41) [a] 
(0.9505161643835591, 5.45) [a] 
(0.9505339726027372, 5.536) [a] 
(0.9505463013698605, 5.631) [a] 
(0.9505465753424632, 5.64) [a] 
(0.9505489041095865, 5.65) [a] 
(0.9505845205479426, 5.73) [a] 
(0.9506023287671207, 5.752) [a] 
(0.9506026027397234, 5.81) [a] 
(0.9506119178082165, 5.89) [a] 
(0.9506297260273946, 6.006) [a] 
(0.9506310958904083, 6.139) [a] 
(0.9506338356164357, 6.168) [a] 
(0.950636575342463, 6.2) [a] 
(0.9506420547945178, 6.205) [a] 
(0.9506443835616412, 6.25) [a] 
(0.9506471232876685, 6.313) [a] 
(0.9506698630136959, 6.38) [a] 
(0.9506767123287644, 6.388) [a] 
(0.9506794520547918, 6.391) [a] 
(0.9506849315068465, 6.392) [a] 
(0.950691780821915, 6.394) [a] 
(0.9506945205479423, 6.397) [a] 
(0.9506972602739697, 6.399) [a] 
(0.9507041095890382, 6.4) [a] 
(0.9507068493150655, 6.405) [a] 
(0.9507123287671203, 6.41) [a] 
(0.9507164383561614, 6.411) [a] 
(0.9507191780821888, 6.415) [a] 
(0.9507232876712298, 6.512) [a] 
(0.9507260273972572, 6.54) [a] 
(0.9507287671232846, 6.597) [a] 
(0.950731506849312, 6.601) [a] 
(0.950735616438353, 6.613) [a] 
(0.9507383561643804, 6.615) [a] 
(0.9507410958904078, 6.62) [a] 
(0.9507412328767092, 6.7) [a] 
(0.9507590410958873, 6.737) [a] 
(0.950760410958901, 6.74) [a] 
(0.950778219178079, 6.743) [a] 
(0.9507791780821886, 6.75) [a] 
(0.9507795890410927, 6.82) [a] 
(0.9507823287671201, 6.865) [a] 
(0.9507991780821886, 6.9) [a] 
(0.9508013698630106, 6.93) [a] 
(0.9508191780821886, 7) [a] 
(0.9508360273972571, 7.09) [a] 
(0.9508634246575312, 7.21) [a] 
(0.9508661643835585, 7.263) [a] 
(0.9508839726027366, 7.275) [a] 
(0.9509050684931476, 7.28) [a] 
(0.9509187671232846, 7.36) [a] 
(0.9509365753424627, 7.389) [a] 
(0.9509434246575311, 7.397) [a] 
(0.9509571232876681, 7.4) [a] 
(0.951025616438353, 7.41) [a] 
(0.951053013698627, 7.42) [a] 
(0.9510557534246544, 7.421) [a] 
(0.9510831506849284, 7.43) [a] 
(0.9511009589041065, 7.479) [a] 
(0.9511119178082161, 7.487) [a] 
(0.9511804109589012, 7.49) [a] 
(0.9511927397260245, 7.495) [a] 
(0.9512050684931478, 7.497) [a] 
(0.9512187671232848, 7.5) [a] 
(0.9512310958904081, 7.502) [a] 
(0.9512420547945177, 7.516) [a] 
(0.951254383561641, 7.517) [a] 
(0.951268082191778, 7.54) [a] 
(0.9512817808219151, 7.614) [a] 
(0.9512823287671205, 7.81) [a] 
(0.9512831506849287, 7.82) [a] 
(0.9512836986301342, 7.85) [a] 
(0.9512838356164356, 7.86) [a] 
(0.9512841095890383, 7.89) [a] 
(0.9512845205479424, 7.92) [a] 
(0.9512847945205452, 7.96) [a] 
(0.9512850684931479, 7.99) [a] 
(0.9512861643835588, 8) [a] 
(0.9512904109589013, 8.02) [a] 
(0.951290684931504, 8.05) [a] 
(0.9512913698630109, 8.07) [a] 
(0.9513016438356137, 8.08) [a] 
(0.9513427397260247, 8.09) [a] 
(0.9513569863013671, 8.1) [a] 
(0.9513843835616411, 8.12) [a] 
(0.9513846575342438, 8.13) [a] 
(0.9513849315068466, 8.14) [a] 
(0.951385068493148, 8.15) [a] 
(0.9513853424657507, 8.16) [a] 
(0.9513856164383534, 8.17) [a] 
(0.9513869863013671, 8.19) [a] 
(0.9513880821917781, 8.2) [a] 
(0.9513887671232849, 8.23) [a] 
(0.9513890410958876, 8.26) [a] 
(0.9513935616438328, 8.53) [a] 
(0.951393972602737, 8.64) [a] 
(0.951398082191778, 8.719) [a] 
(0.9514008219178054, 8.763) [a] 
(0.9514035616438328, 8.764) [a] 
(0.9514063013698602, 8.779) [a] 
(0.9514186301369835, 8.9) [a] 
(0.9514295890410931, 8.906) [a] 
(0.9514419178082164, 8.908) [a] 
(0.9514542465753397, 8.92) [a] 
(0.9514652054794493, 8.925) [a] 
(0.9514656164383535, 9.02) [a] 
(0.9514661643835589, 9.03) [a] 
(0.951466575342463, 9.09) [a] 
(0.9514834246575316, 9.15) [a] 
(0.9514861643835589, 9.218) [a] 
(0.9514902739726, 9.219) [a] 
(0.9514906849315041, 9.22) [a] 
(0.9514934246575315, 9.23) [a] 
(0.9514961643835589, 9.244) [a] 
(0.9514989041095863, 9.353) [a] 
(0.9515084931506821, 9.36) [a] 
(0.9515112328767095, 9.361) [a] 
(0.9515139726027368, 9.362) [a] 
(0.9515308219178054, 9.45) [a] 
(0.9515363013698601, 9.463) [a] 
(0.951545890410956, 9.464) [a] 
(0.951564383561641, 9.47) [a] 
(0.9515698630136957, 9.493) [a] 
(0.9515726027397231, 9.541) [a] 
(0.9516094520547916, 9.63) [a] 
(0.9516098630136958, 9.86) [a] 
(0.9516139726027368, 9.93) [a] 
(0.9516413698630108, 10.08) [a] 
(0.9516468493150656, 10.09) [a] 
(0.951653698630134, 10.1) [a] 
(0.9516564383561614, 10.27) [a] 
(0.9516619178082162, 10.32) [a] 
(0.9516623287671203, 10.53) [a] 
(0.951663698630134, 10.72) [a] 
(0.9516679452054765, 10.91) [a] 
(0.9516747945205449, 10.97) [a] 
(0.9516753424657504, 10.99) [a] 
(0.9516754794520518, 11.01) [a] 
(0.9516760273972572, 11.05) [a] 
(0.9516802739725997, 11.06) [a] 
(0.9516806849315038, 11.28) [a] 
(0.951681095890408, 11.3) [a] 
(0.9516815068493121, 11.31) [a] 
(0.9516993150684901, 11.88) [a] 
(0.9517006849315038, 12.55) [a] 
(0.9517020547945175, 12.61) [a] 
(0.9517230136986271, 12.67) [a] 
(0.9517284931506819, 12.95) [a] 
(0.9517494520547914, 13.06) [a] 
(0.9517497260273942, 13.22) [a] 
(0.9517572602739695, 13.43) [a] 
(0.9517599999999968, 13.81) [a] 
(0.9517771232876681, 13.85) [a] 
(0.9517773972602708, 13.97) [a] 
(0.9517775342465722, 13.98) [a] 
(0.9517776712328736, 14.22) [a] 
(0.951780410958901, 14.38) [a] 
(0.9519343835616407, 14.41) [a] 
(0.9520113698630106, 14.67) [a] 
(0.952014109589038, 14.72) [a] 
(0.952027808219175, 16.06) [a] 
(0.9520332876712297, 16.17) [a] 
(0.9520334246575312, 16.23) [a] 
(0.9520353424657503, 16.26) [a] 
(0.9520354794520517, 16.32) [a] 
(0.9520357534246544, 16.33) [a] 
(0.9520384931506818, 16.34) [a] 
(0.9520389041095859, 16.35) [a] 
(0.9520436986301338, 16.36) [a] 
(0.9520464383561612, 16.39) [a] 
(0.9520467123287639, 16.4) [a] 
(0.9520469863013666, 16.41) [a] 
(0.9520473972602708, 16.43) [a] 
(0.9520478082191749, 16.44) [a] 
(0.9520480821917776, 16.46) [a] 
(0.9520487671232845, 16.47) [a] 
(0.9520490410958872, 16.5) [a] 
(0.9520491780821886, 16.54) [a] 
(0.9520494520547913, 16.57) [a] 
(0.9520512328767091, 16.59) [a] 
(0.952051917808216, 16.6) [a] 
(0.9520538356164351, 16.63) [a] 
(0.9520552054794488, 16.64) [a] 
(0.9520553424657502, 16.98) [a] 
(0.9520563013698597, 17.01) [a] 
(0.9520568493150652, 17.09) [a] 
(0.9520569863013666, 17.1) [a] 
(0.952057534246572, 17.11) [a] 
(0.9520576712328734, 17.12) [a] 
(0.9520579452054762, 17.13) [a] 
(0.9520582191780789, 17.16) [a] 
(0.9520583561643803, 17.17) [a] 
(0.9520591780821885, 17.19) [a] 
(0.9520594520547913, 17.22) [a] 
(0.9520598630136954, 17.23) [a] 
(0.9520601369862981, 17.26) [a] 
(0.9520620547945173, 17.82) [a] 
(0.9521390410958872, 18.59) [a] 
(0.9521394520547913, 19.08) [a] 
(0.952139726027394, 19.1) [a] 
(0.9521399999999968, 19.19) [a] 
(0.9521401369862982, 19.26) [a] 
(0.952161917808216, 19.33) [a] 
(0.9521621917808187, 19.45) [a] 
(0.9521658904109557, 19.47) [a] 
(0.9521664383561611, 19.52) [a] 
(0.9521665753424625, 19.62) [a] 
(0.9521675342465721, 19.71) [a] 
(0.9521678082191748, 19.77) [a] 
(0.9521682191780789, 21.23) [a] 
(0.9521683561643803, 21.46) [a] 
(0.952177123287668, 21.5) [a] 
(0.9521773972602707, 22.24) [a] 
(0.9521983561643803, 22.34) [a] 
(0.9521989041095857, 23.31) [a] 
(0.9521993150684899, 23.44) [a] 
(0.9522010958904077, 23.45) [a] 
(0.9522028767123255, 23.63) [a] 
(0.952215616438353, 23.75) [a] 
(0.95222931506849, 24.04) [a] 
(0.9522436986301338, 24.08) [a] 
(0.9522441095890379, 24.24) [a] 
(0.9522446575342434, 24.25) [a] 
(0.9522452054794488, 24.27) [a] 
(0.9522456164383529, 24.28) [a] 
(0.9522458904109556, 24.3) [a] 
(0.9522463013698598, 24.32) [a] 
(0.9522465753424625, 25.05) [a] 
(0.9522469863013666, 26.11) [a] 
(0.9522479452054762, 26.23) [a] 
(0.9522657534246542, 26.27) [a] 
(0.9522835616438323, 26.61) [a] 
(0.9523013698630104, 26.84) [a] 
(0.9523191780821885, 27.1) [a] 
(0.9523960273972569, 27.98) [a] 
(0.9524730136986268, 28.17) [a] 
(0.9524909589041063, 29.55) [a] 
(0.9524921917808186, 32.06) [a] 
(0.952506438356161, 32.1) [a] 
(0.952520136986298, 32.96) [a] 
(0.9525557534246542, 32.98) [a] 
(0.9525735616438322, 33.04) [a] 
(0.9525749315068459, 33.43) [a] 
(0.9525758904109555, 34.2) [a] 
(0.952576849315065, 34.27) [a] 
(0.9525861643835581, 34.53) [a] 
(0.9525865753424623, 34.64) [a] 
(0.9525887671232842, 34.75) [a] 
(0.9526024657534212, 36.2) [a] 
(0.9526183561643802, 36.54) [a] 
(0.9526186301369829, 36.93) [a] 
(0.9526187671232843, 37.33) [a] 
(0.952619041095887, 37.43) [a] 
(0.9526191780821884, 37.44) [a] 
(0.9526194520547911, 37.84) [a] 
(0.9526323287671199, 38.31) [a] 
(0.9526336986301336, 38.44) [a] 
(0.9526343835616404, 39.21) [a] 
(0.9526357534246541, 39.69) [a] 
(0.9526367123287637, 39.9) [a] 
(0.9526380821917774, 40.26) [a] 
(0.9526421917808184, 40.27) [a] 
(0.9526558904109554, 41.85) [a] 
(0.9526560273972569, 41.97) [a] 
(0.9526697260273939, 42.09) [a] 
(0.9526706849315034, 42.21) [a] 
(0.9526843835616404, 42.28) [a] 
(0.95268534246575, 42.41) [a] 
(0.9526965753424623, 42.81) [a] 
(0.9526973972602706, 42.82) [a] 
(0.9527002739725994, 42.86) [a] 
(0.9527012328767089, 42.89) [a] 
(0.9527149315068459, 43.49) [a] 
(0.952728630136983, 43.65) [a] 
(0.9527299999999966, 45.12) [a] 
(0.9527313698630103, 45.21) [a] 
(0.9527323287671199, 45.61) [a] 
(0.9527331506849281, 45.65) [a] 
(0.9527341095890377, 45.73) [a] 
(0.9527350684931473, 46.4) [a] 
(0.9527369863013665, 46.67) [a] 
(0.952754931506846, 46.77) [a] 
(0.9527553424657501, 47.63) [a] 
(0.9527556164383528, 47.64) [a] 
(0.9527558904109555, 47.65) [a] 
(0.9527568493150651, 48.49) [a] 
(0.9527586301369829, 48.5) [a] 
(0.9527709589041062, 48.65) [a] 
(0.9527846575342432, 48.87) [a] 
(0.9527856164383528, 49.11) [a] 
(0.9527858904109555, 49.44) [a] 
(0.9527861643835582, 49.97) [a] 
(0.9527871232876678, 50.27) [a] 
(0.9528008219178048, 50.32) [a] 
(0.9528010958904075, 50.43) [a] 
(0.9528134246575308, 50.7) [a] 
(0.9528135616438322, 51.24) [a] 
(0.9528336986301336, 51.9) [a] 
(0.9528339726027363, 51.92) [a] 
(0.9528367123287637, 51.93) [a] 
(0.9528376712328732, 52.1) [a] 
(0.9528379452054759, 52.9) [a] 
(0.9528383561643801, 53) [a] 
(0.9528389041095855, 53.32) [a] 
(0.9528419178082157, 53.34) [a] 
(0.9528428767123253, 53.91) [a] 
(0.95284835616438, 53.96) [a] 
(0.9528484931506814, 53.97) [a] 
(0.9528489041095856, 54.51) [a] 
(0.952849041095887, 54.71) [a] 
(0.9528493150684897, 54.73) [a] 
(0.9528497260273938, 54.74) [a] 
(0.9528499999999965, 54.76) [a] 
(0.9528636986301335, 55.66) [a] 
(0.9528776712328733, 55.67) [a] 
(0.9528915068493117, 55.68) [a] 
(0.9528917808219144, 55.69) [a] 
(0.9529054794520514, 55.92) [a] 
(0.9529191780821884, 55.93) [a] 
(0.9529271232876678, 56.92) [a] 
(0.9529273972602705, 57.12) [a] 
(0.952927534246572, 57.13) [a] 
(0.9529289041095856, 57.86) [a] 
(0.9529298630136952, 57.9) [a] 
(0.9529304109589006, 57.91) [a] 
(0.9529310958904075, 59.31) [a] 
(0.9529447945205445, 60.06) [a] 
(0.9529450684931472, 60.15) [a] 
(0.9529453424657499, 60.17) [a] 
(0.9529454794520513, 60.19) [a] 
(0.9529458904109555, 61.57) [a] 
(0.9529463013698596, 61.61) [a] 
(0.9529465753424623, 61.66) [a] 
(0.9529469863013664, 61.75) [a] 
(0.9529472602739691, 61.84) [a] 
(0.9529475342465719, 61.9) [a] 
(0.9529567123287637, 62.24) [a] 
(0.9529658904109555, 62.27) [a] 
(0.9529752054794486, 62.32) [a] 
(0.9529935616438322, 62.49) [a] 
(0.952994246575339, 62.56) [a] 
(0.9529949315068459, 62.57) [a] 
(0.953004246575339, 62.74) [a] 
(0.9530134246575308, 62.75) [a] 
(0.9530141095890377, 62.77) [a] 
(0.9530326027397226, 62.78) [a] 
(0.9530328767123253, 62.97) [a] 
(0.953060547945202, 63.3) [a] 
(0.9530698630136951, 63.31) [a] 
(0.9531067123287636, 63.4) [a] 
(0.9531068493150651, 63.58) [a] 
(0.9531078082191746, 63.64) [a] 
(0.9531091780821883, 63.9) [a] 
(0.9531269863013664, 64.3) [a] 
(0.9531272602739691, 64.49) [a] 
(0.9531450684931472, 65.15) [a] 
(0.953145753424654, 66.23) [a] 
(0.9531567123287636, 66.6) [a] 
(0.953169041095887, 66.61) [a] 
(0.9531693150684897, 66.66) [a] 
(0.9531694520547911, 67.08) [a] 
(0.9531968493150651, 67.54) [a] 
(0.9531971232876678, 67.7) [a] 
(0.9531973972602705, 67.71) [a] 
(0.9531997260273938, 68.16) [a] 
(0.9531999999999965, 69.29) [a] 
(0.9532005479452019, 70.29) [a] 
(0.9532008219178046, 70.33) [a] 
(0.9532015068493115, 70.42) [a] 
(0.9532138356164348, 72.62) [a] 
(0.9532158904109553, 72.82) [a] 
(0.9532165753424622, 72.86) [a] 
(0.9532178082191745, 72.87) [a] 
(0.9532179452054759, 72.88) [a] 
(0.9532189041095854, 73.36) [a] 
(0.9532199999999964, 73.37) [a] 
(0.9532213698630101, 73.4) [a] 
(0.9532254794520512, 73.61) [a] 
(0.953226164383558, 73.62) [a] 
(0.9532267123287634, 73.63) [a] 
(0.9533036986301333, 73.81) [a] 
(0.953303972602736, 73.97) [a] 
(0.953307260273969, 74.32) [a] 
(0.9533086301369826, 74.34) [a] 
(0.95331136986301, 74.35) [a] 
(0.9533115068493114, 74.58) [a] 
(0.9533117808219141, 75.34) [a] 
(0.9533121917808183, 75.96) [a] 
(0.9533163013698593, 76.23) [a] 
(0.9533178082191744, 76.58) [a] 
(0.9533180821917772, 78.29) [a] 
(0.9533182191780786, 78.34) [a] 
(0.9533184931506813, 78.38) [a] 
(0.953318767123284, 78.39) [a] 
(0.9533189041095854, 78.41) [a] 
(0.9533312328767087, 78.59) [a] 
(0.953343561643832, 78.6) [a] 
(0.9533438356164348, 78.89) [a] 
(0.9533441095890375, 78.92) [a] 
(0.9533650684931471, 79.19) [a] 
(0.9533654794520512, 79.41) [a] 
(0.9533657534246539, 79.43) [a] 
(0.9533664383561608, 79.46) [a] 
(0.9533901369862977, 79.72) [a] 
(0.9533942465753388, 79.73) [a] 
(0.9533945205479415, 79.74) [a] 
(0.9533975342465716, 80.21) [a] 
(0.953400273972599, 80.71) [a] 
(0.9534004109589004, 81.36) [a] 
(0.95341136986301, 82.49) [a] 
(0.9534291780821881, 82.56) [a] 
(0.9534297260273935, 82.84) [a] 
(0.9534301369862976, 83.69) [a] 
(0.9534306849315031, 83.9) [a] 
(0.9534308219178045, 83.98) [a] 
(0.9534310958904072, 83.99) [a] 
(0.9534313698630099, 84) [a] 
(0.953431780821914, 84.01) [a] 
(0.9534441095890374, 85.06) [a] 
(0.953445479452051, 85.8) [a] 
(0.9534578082191744, 85.92) [a] 
(0.953459178082188, 86.68) [a] 
(0.9534595890410922, 87.18) [a] 
(0.9534616438356127, 87.45) [a] 
(0.9534617808219141, 88.73) [a] 
(0.9534658904109552, 88.74) [a] 
(0.9534663013698593, 89.07) [a] 
(0.953467671232873, 89.09) [a] 
(0.9534680821917771, 89.12) [a] 
(0.9534708219178045, 89.18) [a] 
(0.9534721917808182, 89.21) [a] 
(0.9534735616438319, 89.56) [a] 
(0.95349136986301, 89.65) [a] 
(0.9534993150684894, 89.78) [a] 
(0.9535072602739688, 89.81) [a] 
(0.9535152054794482, 90.4) [a] 
(0.9535310958904071, 90.55) [a] 
(0.9535390410958865, 90.56) [a] 
(0.9535393150684892, 91.92) [a] 
(0.9535395890410919, 92.22) [a] 
(0.95355739726027, 92.39) [a] 
(0.953558493150681, 92.45) [a] 
(0.9535587671232837, 93.27) [a] 
(0.9536006849315029, 93.29) [a] 
(0.9536427397260234, 93.38) [a] 
(0.9536430136986261, 93.4) [a] 
(0.953643698630133, 93.44) [a] 
(0.9536439726027357, 93.47) [a] 
(0.9536649315068453, 93.9) [a] 
(0.953665205479448, 93.94) [a] 
(0.9536654794520507, 94.06) [a] 
(0.953666301369859, 94.08) [a] 
(0.9536831506849275, 94.67) [a] 
(0.9537506849315028, 94.77) [a] 
(0.9537520547945165, 94.8) [a] 
(0.9537547945205439, 94.91) [a] 
(0.9537550684931466, 95.45) [a] 
(0.9537553424657493, 95.46) [a] 
(0.9537554794520507, 95.47) [a] 
(0.9537595890410918, 96.81) [a] 
(0.9537597260273932, 96.82) [a] 
(0.9537604109589001, 96.88) [a] 
(0.9537613698630096, 96.92) [a] 
(0.9537627397260233, 97.47) [a] 
(0.953764109589037, 97.8) [a] 
(0.9537671232876671, 97.82) [a] 
(0.9537850684931466, 99.78) [a] 
(0.9537864383561603, 101.1) [a] 
(0.9538073972602699, 101.32) [a] 
(0.9538242465753384, 101.36) [a] 
(0.9538917808219137, 101.46) [a] 
(0.9538919178082151, 101.65) [a] 
(0.9538946575342425, 102.7) [a] 
(0.9538960273972562, 102.8) [a] 
(0.9538982191780782, 102.91) [a] 
(0.9538984931506809, 103.59) [a] 
(0.9539153424657494, 104.2) [a] 
(0.9539321917808179, 104.21) [a] 
(0.9539331506849275, 104.98) [a] 
(0.9539332876712289, 104.99) [a] 
(0.9539335616438316, 107.47) [a] 
(0.9539338356164343, 108.56) [a] 
(0.9539342465753384, 108.59) [a] 
(0.9539347945205439, 108.6) [a] 
(0.9539354794520507, 108.7) [a] 
(0.9539368493150644, 108.71) [a] 
(0.9539376712328727, 109.1) [a] 
(0.9539379452054754, 109.11) [a] 
(0.9539383561643795, 109.16) [a] 
(0.9539386301369822, 110.59) [a] 
(0.9539408219178042, 110.92) [a] 
(0.9539430136986261, 111.01) [a] 
(0.9539767123287631, 111.55) [a] 
(0.9539931506849275, 111.79) [a] 
(0.9539938356164344, 111.87) [a] 
(0.9539941095890371, 112.13) [a] 
(0.9540078082191741, 113.7) [a] 
(0.9540352054794481, 114.4) [a] 
(0.9540489041095851, 114.42) [a] 
(0.9540498630136947, 114.44) [a] 
(0.9540505479452015, 114.46) [a] 
(0.9540508219178042, 114.49) [a] 
(0.954052602739722, 114.5) [a] 
(0.954066301369859, 114.92) [a] 
(0.9540690410958864, 115.2) [a] 
(0.9540732876712289, 116.43) [a] 
(0.954073698630133, 116.73) [a] 
(0.9540739726027357, 116.74) [a] 
(0.9540742465753385, 116.77) [a] 
(0.9540746575342426, 116.89) [a] 
(0.954074794520544, 117.49) [a] 
(0.9540749315068454, 117.93) [a] 
(0.9540804109589002, 121.5) [a] 
(0.9540882191780783, 121.64) [a] 
(0.954089589041092, 123.7) [a] 
(0.9540897260273934, 123.77) [a] 
(0.9540899999999961, 124.07) [a] 
(0.9540931506849276, 124.32) [a] 
(0.954094794520544, 124.33) [a] 
(0.9540991780821878, 124.63) [a] 
(0.9540994520547905, 125.06) [a] 
(0.9542319178082151, 125.07) [a] 
(0.9542849315068452, 125.08) [a] 
(0.9543379452054753, 125.09) [a] 
(0.9543380821917767, 125.32) [a] 
(0.9543383561643795, 125.52) [a] 
(0.9543387671232836, 125.55) [a] 
(0.9543401369862973, 126.3) [a] 
(0.9543476712328726, 131.94) [a] 
(0.9543599999999959, 136.4) [a] 
(0.9543602739725986, 136.61) [a] 
(0.9543606849315027, 136.62) [a] 
(0.9543620547945164, 136.8) [a] 
(0.9543623287671191, 136.96) [a] 
(0.9543626027397218, 139.07) [a] 
(0.954363013698626, 139.85) [a] 
(0.9543631506849274, 141.39) [a] 
(0.9543636986301328, 141.41) [a] 
(0.9543815068493109, 142.36) [a] 
(0.9543816438356123, 142.6) [a] 
(0.954381917808215, 142.62) [a] 
(0.9543821917808177, 142.65) [a] 
(0.9543823287671191, 143.05) [a] 
(0.9543826027397219, 143.68) [a] 
(0.9543869863013656, 144.55) [a] 
(0.9543954794520506, 144.58) [a] 
(0.954395616438352, 144.62) [a] 
(0.954409315068489, 146) [a] 
(0.9544094520547904, 146.31) [a] 
(0.9544097260273932, 149.77) [a] 
(0.9544234246575302, 150.29) [a] 
(0.9544371232876672, 150.67) [a] 
(0.9544373972602699, 152.96) [a] 
(0.9544376712328726, 160.87) [a] 
(0.9544468493150644, 162.16) [a] 
(0.9544475342465712, 163.84) [a] 
(0.9544489041095849, 164) [a] 
(0.9544491780821877, 164.03) [a] 
(0.9544493150684891, 166.69) [a] 
(0.9544494520547905, 166.71) [a] 
(0.9544501369862973, 167.27) [a] 
(0.9544502739725987, 168.86) [a] 
(0.9544657534246536, 169.9) [a] 
(0.954481095890407, 169.99) [a] 
(0.9545428767123247, 170.48) [a] 
(0.9545442465753384, 172.5) [a] 
(0.9545445205479411, 174.82) [a] 
(0.954559999999996, 174.87) [a] 
(0.9545602739725987, 175.28) [a] 
(0.9545627397260233, 175.45) [a] 
(0.9545631506849275, 175.46) [a] 
(0.9545647945205439, 175.47) [a] 
(0.954565205479448, 175.58) [a] 
(0.9545658904109549, 176.97) [a] 
(0.9545827397260234, 177.18) [a] 
(0.9545854794520507, 177.4) [a] 
(0.9545868493150644, 178.6) [a] 
(0.9545882191780781, 178.8) [a] 
(0.9545884931506808, 180.28) [a] 
(0.9545912328767082, 181) [a] 
(0.9545926027397219, 181.8) [a] 
(0.9545939726027356, 181.9) [a] 
(0.954596712328763, 182.2) [a] 
(0.9545994520547904, 185.1) [a] 
(0.9546021917808177, 185.2) [a] 
(0.9546024657534204, 187.01) [a] 
(0.9546026027397219, 187.72) [a] 
(0.9546028767123246, 187.97) [a] 
(0.9546035616438314, 188.63) [a] 
(0.9546204109588999, 188.66) [a] 
(0.9546217808219136, 190) [a] 
(0.954624520547941, 190.8) [a] 
(0.9546249315068451, 191.62) [a] 
(0.9546254794520506, 192.86) [a] 
(0.9546258904109547, 192.87) [a] 
(0.9546272602739684, 197) [a] 
(0.9546286301369821, 197.4) [a] 
(0.9546289041095848, 198.43) [a] 
(0.9546293150684889, 198.46) [a] 
(0.9546294520547903, 198.98) [a] 
(0.954630821917804, 208.9) [a] 
(0.9546335616438314, 215.3) [a] 
(0.9546349315068451, 215.4) [a] 
(0.9546486301369821, 215.48) [a] 
(0.9546623287671191, 216.05) [a] 
(0.9546626027397218, 218.18) [a] 
(0.9546627397260232, 218.44) [a] 
(0.9546631506849274, 218.86) [a] 
(0.9546641095890369, 218.97) [a] 
(0.9546649315068452, 218.98) [a] 
(0.9546676712328726, 219.3) [a] 
(0.9546679452054753, 220.19) [a] 
(0.9546720547945163, 220.6) [a] 
(0.954672328767119, 221.77) [a] 
(0.9546736986301327, 221.8) [a] 
(0.9546739726027355, 222.08) [a] 
(0.9546742465753382, 222.33) [a] 
(0.9546743835616396, 222.85) [a] 
(0.9546746575342423, 223.44) [a] 
(0.954674931506845, 223.94) [a] 
(0.9546753424657491, 224.4) [a] 
(0.9546767123287628, 226.7) [a] 
(0.9546773972602697, 232.29) [a] 
(0.9546778082191738, 234.63) [a] 
(0.9546805479452012, 234.64) [a] 
(0.9546810958904066, 235.03) [a] 
(0.9546817808219135, 235.04) [a] 
(0.9546819178082149, 235.16) [a] 
(0.9546826027397217, 239.46) [a] 
(0.9546836986301327, 242.18) [a] 
(0.9546843835616395, 243.27) [a] 
(0.9546846575342423, 243.98) [a] 
(0.954684931506845, 244.27) [a] 
(0.9546856164383518, 245.74) [a] 
(0.9546858904109545, 248.61) [a] 
(0.9546861643835572, 249.15) [a] 
(0.9546868493150641, 252) [a] 
(0.9546871232876668, 252.11) [a] 
(0.9546873972602695, 254.47) [a] 
(0.9546876712328722, 254.51) [a] 
(0.954687945205475, 254.85) [a] 
(0.9546886301369818, 255.7) [a] 
(0.95468945205479, 256.4) [a] 
(0.9546898630136942, 256.5) [a] 
(0.9547021917808175, 257.2) [a] 
(0.9547026027397216, 259.31) [a] 
(0.9547028767123243, 259.32) [a] 
(0.9547030136986258, 259.47) [a] 
(0.9547035616438312, 259.48) [a] 
(0.9547036986301326, 259.67) [a] 
(0.954704246575338, 259.7) [a] 
(0.9547045205479407, 259.72) [a] 
(0.954706849315064, 260.78) [a] 
(0.9547072602739681, 260.79) [a] 
(0.9547075342465708, 261.64) [a] 
(0.9547139726027353, 261.81) [a] 
(0.9547157534246531, 262.15) [a] 
(0.9547169863013654, 262.27) [a] 
(0.9547173972602695, 262.28) [a] 
(0.9547269863013653, 264) [a] 
(0.9547434246575296, 264.1) [a] 
(0.9547436986301323, 265.47) [a] 
(0.9547491780821871, 266.2) [a] 
(0.9547495890410912, 268.54) [a] 
(0.9547498630136939, 268.6) [a] 
(0.9547512328767076, 269.56) [a] 
(0.9547549315068445, 269.71) [a] 
(0.9547556164383514, 269.88) [a] 
(0.9547560273972555, 272.1) [a] 
(0.9547615068493103, 272.9) [a] 
(0.9547656164383513, 273.8) [a] 
(0.954765890410954, 275.18) [a] 
(0.9547686301369814, 275.31) [a] 
(0.9547691780821869, 277.81) [a] 
(0.954770684931502, 280.25) [a] 
(0.9547713698630088, 280.27) [a] 
(0.9547721917808171, 280.28) [a] 
(0.9547731506849266, 280.41) [a] 
(0.954791780821913, 283.7) [a] 
(0.9548194520547897, 283.71) [a] 
(0.9548197260273924, 285.42) [a] 
(0.9548215068493102, 285.45) [a] 
(0.9548308219178033, 286.43) [a] 
(0.9548335616438307, 287.6) [a] 
(0.954835890410954, 293) [a] 
(0.9548413698630087, 299.1) [a] 
(0.9548468493150635, 303.7) [a] 
(0.9548469863013649, 305.78) [a] 
(0.9548472602739676, 306.11) [a] 
(0.9548475342465703, 306.14) [a] 
(0.9548476712328717, 307.14) [a] 
(0.954854794520543, 314.04) [a] 
(0.9548643835616389, 314.05) [a] 
(0.9548675342465704, 314.06) [a] 
(0.9548701369862964, 322.22) [a] 
(0.954872191780817, 322.32) [a] 
(0.954889999999995, 322.6) [a] 
(0.9549078082191731, 322.7) [a] 
(0.9549084931506799, 323.4) [a] 
(0.9549091780821868, 327.98) [a] 
(0.9549094520547895, 329.99) [a] 
(0.9549097260273922, 333.97) [a] 
(0.9549104109588991, 334.34) [a] 
(0.9549106849315018, 336.4) [a] 
(0.9549113698630086, 339.77) [a] 
(0.9549116438356113, 341.09) [a] 
(0.9549128767123236, 343.95) [a] 
(0.9549131506849263, 343.96) [a] 
(0.954913424657529, 344.04) [a] 
(0.9549141095890359, 344.24) [a] 
(0.9549143835616386, 346.45) [a] 
(0.9549156164383509, 347.92) [a] 
(0.9549165753424604, 348.97) [a] 
(0.9549475342465701, 350.1) [a] 
(0.9549628767123235, 350.19) [a] 
(0.9549783561643783, 350.21) [a] 
(0.954978630136981, 351.05) [a] 
(0.9549789041095837, 351.06) [a] 
(0.9549791780821865, 352.89) [a] 
(0.9549794520547892, 352.91) [a] 
(0.954994931506844, 353.68) [a] 
(0.9550104109588988, 356.77) [a] 
(0.9550257534246522, 357.34) [a] 
(0.9550267123287618, 357.44) [a] 
(0.9550280821917755, 359.9) [a] 
(0.9550283561643782, 361.88) [a] 
(0.9550286301369809, 362.42) [a] 
(0.9550295890410905, 362.48) [a] 
(0.955031643835611, 362.53) [a] 
(0.9550317808219124, 362.54) [a] 
(0.9550324657534193, 363.89) [a] 
(0.955032739726022, 366.6) [a] 
(0.9550330136986247, 367.18) [a] 
(0.9550334246575288, 367.63) [a] 
(0.9550343835616384, 368.52) [a] 
(0.9550347945205425, 368.53) [a] 
(0.9550350684931452, 369.51) [a] 
(0.9550360273972548, 370.89) [a] 
(0.9550363013698575, 371.14) [a] 
(0.9550517808219123, 371.45) [a] 
(0.9550521917808165, 372.7) [a] 
(0.9550524657534192, 372.78) [a] 
(0.9550528767123233, 372.79) [a] 
(0.9550534246575287, 372.97) [a] 
(0.9550539726027342, 373.69) [a] 
(0.9550541095890356, 374.78) [a] 
(0.9550547945205424, 374.79) [a] 
(0.9550550684931451, 374.87) [a] 
(0.9550553424657479, 374.9) [a] 
(0.9550582191780767, 375.26) [a] 
(0.9550595890410903, 375.27) [a] 
(0.9550602739725972, 375.29) [a] 
(0.9550605479451999, 375.35) [a] 
(0.955060958904104, 375.36) [a] 
(0.9550616438356109, 375.37) [a] 
(0.9550771232876657, 378.65) [a] 
(0.9550773972602684, 379.48) [a] 
(0.9550779452054738, 379.79) [a] 
(0.955078356164378, 380.01) [a] 
(0.9550789041095834, 381.1) [a] 
(0.9550794520547888, 384.15) [a] 
(0.955079863013693, 384.86) [a] 
(0.9550802739725971, 392.12) [a] 
(0.9550805479451998, 392.13) [a] 
(0.9551175342465696, 396.3) [a] 
(0.9551189041095833, 406.45) [a] 
(0.9551194520547888, 406.47) [a] 
(0.9551208219178025, 406.48) [a] 
(0.9551212328767066, 406.49) [a] 
(0.9551230136986244, 406.52) [a] 
(0.9551232876712271, 409.01) [a] 
(0.9551252054794463, 409.77) [a] 
(0.9551256164383505, 409.79) [a] 
(0.9551260273972546, 415.15) [a] 
(0.95512657534246, 415.16) [a] 
(0.9551269863013642, 415.17) [a] 
(0.9551271232876656, 419.67) [a] 
(0.9551275342465697, 421.03) [a] 
(0.9551280821917751, 421.27) [a] 
(0.9551335616438299, 422) [a] 
(0.9551336986301313, 424.3) [a] 
(0.9551349315068436, 424.94) [a] 
(0.955137671232871, 424.95) [a] 
(0.9551380821917751, 430.61) [a] 
(0.9551387671232819, 432.38) [a] 
(0.9551398630136929, 438.17) [a] 
(0.9551412328767066, 438.18) [a] 
(0.9551430136986244, 438.21) [a] 
(0.9552202739725971, 442.14) [a] 
(0.9552356164383505, 442.15) [a] 
(0.9552665753424601, 442.32) [a] 
(0.9552669863013642, 448.56) [a] 
(0.9552672602739669, 448.61) [a] 
(0.955267671232871, 450.35) [a] 
(0.9552693150684874, 453.99) [a] 
(0.9552695890410902, 462.02) [a] 
(0.9552697260273916, 462.49) [a] 
(0.9552699999999943, 470.16) [a] 
(0.9552706849315011, 471.17) [a] 
(0.9552709589041039, 471.18) [a] 
(0.9552710958904053, 471.27) [a] 
(0.9552717808219121, 471.45) [a] 
(0.9552731506849258, 471.46) [a] 
(0.9552734246575285, 479.53) [a] 
(0.9552735616438299, 486.2) [a] 
(0.9553504109588984, 497.55) [a] 
(0.9553508219178025, 498.2) [a] 
(0.9553517808219121, 498.23) [a] 
(0.9553526027397203, 498.3) [a] 
(0.9553531506849258, 498.31) [a] 
(0.9553541095890353, 502.03) [a] 
(0.9553550684931449, 502.09) [a] 
(0.955355479452049, 504.78) [a] 
(0.9553557534246517, 504.79) [a] 
(0.9553560273972544, 506.06) [a] 
(0.9553564383561586, 510.34) [a] 
(0.95535657534246, 511.49) [a] 
(0.9553568493150627, 511.5) [a] 
(0.9553571232876654, 512) [a] 
(0.9553572602739668, 512.22) [a] 
(0.9553590410958847, 514.44) [a] 
(0.9553631506849258, 514.45) [a] 
(0.9553653424657478, 514.48) [a] 
(0.9553667123287615, 514.49) [a] 
(0.9553721917808162, 514.51) [a] 
(0.9553726027397204, 514.66) [a] 
(0.9553734246575286, 514.68) [a] 
(0.955373972602734, 514.73) [a] 
(0.9553784931506792, 515.07) [a] 
(0.9553883561643779, 515.08) [a] 
(0.9553906849315013, 517.07) [a] 
(0.9553910958904054, 517.48) [a] 
(0.9553924657534191, 522.95) [a] 
(0.9553928767123232, 522.96) [a] 
(0.9553931506849259, 523.04) [a] 
(0.9553938356164328, 523.33) [a] 
(0.9553939726027342, 523.91) [a] 
(0.9553953424657479, 524.13) [a] 
(0.9554723287671177, 524.24) [a] 
(0.9554750684931451, 524.52) [a] 
(0.9554772602739671, 524.54) [a] 
(0.9554826027397205, 524.61) [a] 
(0.9554830136986247, 524.62) [a] 
(0.9554832876712274, 524.63) [a] 
(0.9554834246575288, 524.66) [a] 
(0.9554836986301315, 525.87) [a] 
(0.9554861643835562, 526.03) [a] 
(0.9554880821917753, 533.35) [a] 
(0.9554884931506794, 534.2) [a] 
(0.9555063013698575, 546.08) [a] 
(0.9555241095890356, 553.75) [a] 
(0.955524657534241, 558.61) [a] 
(0.9555249315068437, 568.09) [a] 
(0.9555254794520491, 569.23) [a] 
(0.9555256164383505, 569.36) [a] 
(0.9555258904109533, 586.31) [a] 
(0.9555260273972547, 586.36) [a] 
(0.9555263013698574, 586.4) [a] 
(0.9555265753424601, 586.51) [a] 
(0.9555279452054738, 594.56) [a] 
(0.9555280821917752, 597.48) [a] 
(0.9555293150684875, 597.54) [a] 
(0.9555347945205422, 608.3) [a] 
(0.955540273972597, 612.5) [a] 
(0.9555457534246518, 612.8) [a] 
(0.9555512328767065, 612.9) [a] 
(0.9555608219178023, 613) [a] 
(0.9555619178082133, 619.5) [a] 
(0.9555642465753366, 619.58) [a] 
(0.9555646575342407, 645.95) [a] 
(0.9555650684931448, 663.51) [a] 
(0.9555664383561585, 667.44) [a] 
(0.9555673972602681, 667.74) [a] 
(0.9555680821917749, 667.75) [a] 
(0.955568493150679, 667.76) [a] 
(0.9555687671232818, 667.81) [a] 
(0.9555694520547886, 667.82) [a] 
(0.9555701369862954, 668.21) [a] 
(0.955571095890405, 668.22) [a] 
(0.9555712328767064, 668.23) [a] 
(0.9555719178082133, 668.27) [a] 
(0.9555738356164325, 669.93) [a] 
(0.9555743835616379, 670.02) [a] 
(0.9555745205479393, 670.04) [a] 
(0.9555750684931448, 670.06) [a] 
(0.9555757534246516, 692.32) [a] 
(0.955575890410953, 692.37) [a] 
(0.9555761643835557, 692.39) [a] 
(0.9555775342465694, 692.5) [a] 
(0.9555830136986242, 692.6) [a] 
(0.9555834246575283, 692.75) [a] 
(0.955583698630131, 692.76) [a] 
(0.9555839726027338, 693.32) [a] 
(0.9555842465753365, 694.78) [a] 
(0.9555853424657474, 695.47) [a] 
(0.9555857534246516, 695.5) [a] 
(0.9555860273972543, 695.51) [a] 
(0.955586301369857, 698.01) [a] 
(0.9555876712328707, 704.3) [a] 
(0.9555880821917748, 704.35) [a] 
(0.9555894520547885, 704.5) [a] 
(0.9555921917808159, 724.5) [a] 
(0.9556031506849255, 730.2) [a] 
(0.9556032876712269, 740.48) [a] 
(0.9556035616438296, 740.5) [a] 
(0.9556038356164324, 740.51) [a] 
(0.9556039726027338, 748.26) [a] 
(0.9556094520547885, 761.3) [a] 
(0.9556097260273912, 764.95) [a] 
(0.9556098630136927, 765.03) [a] 
(0.9556101369862954, 765.17) [a] 
(0.9556104109588981, 765.31) [a] 
(0.9556105479451995, 765.42) [a] 
(0.9556108219178022, 777.79) [a] 
(0.9556113698630077, 777.91) [a] 
(0.9556168493150624, 781.4) [a] 
(0.9556171232876651, 791.36) [a] 
(0.9556172602739665, 791.39) [a] 
(0.9556175342465693, 791.44) [a] 
(0.9556197260273912, 793.1) [a] 
(0.9556202739725966, 793.13) [a] 
(0.9556216438356103, 826.1) [a] 
(0.9556394520547884, 827.05) [a] 
(0.9556421917808158, 827.2) [a] 
(0.9556601369862953, 827.23) [a] 
(0.9556605479451994, 840.9) [a] 
(0.9556608219178021, 849.56) [a] 
(0.9556609589041035, 850.78) [a] 
(0.9556612328767062, 850.79) [a] 
(0.955661506849309, 850.81) [a] 
(0.9556619178082131, 850.82) [a] 
(0.9556621917808158, 856.21) [a] 
(0.9556623287671172, 862.77) [a] 
(0.9556626027397199, 862.78) [a] 
(0.955663013698624, 868.84) [a] 
(0.9556635616438295, 868.89) [a] 
(0.9556639726027336, 869.04) [a] 
(0.9556653424657473, 869.3) [a] 
(0.95566561643835, 870.49) [a] 
(0.9556660273972541, 876.66) [a] 
(0.9556665753424596, 876.67) [a] 
(0.9557435616438295, 891.68) [a] 
(0.9557490410958842, 913.5) [a] 
(0.9557497260273911, 916.14) [a] 
(0.9557552054794458, 917.3) [a] 
(0.9557558904109527, 936.23) [a] 
(0.9557561643835554, 936.24) [a] 
(0.9557563013698568, 936.26) [a] 
(0.9557565753424595, 936.8) [a] 
(0.9557569863013636, 960.36) [a] 
(0.9557576712328705, 960.52) [a] 
(0.9557579452054732, 960.56) [a] 
(0.9557582191780759, 960.57) [a] 
(0.9557583561643773, 961.61) [a] 
(0.95575863013698, 968.52) [a] 
(0.9557594520547883, 983.32) [a] 
(0.9557794520547883, 1006.6) [a] 
(0.9557849315068431, 1007) [a] 
(0.9557853424657472, 1008.3) [a] 
(0.9557856164383499, 1008.4) [a] 
(0.9557869863013636, 1008.5) [a] 
(0.9558020547945142, 1079) [a] 
(0.9558047945205416, 1085) [a] 
(0.9558089041095826, 1086) [a] 
(0.95581164383561, 1104) [a] 
(0.9558124657534183, 1106.2) [a] 
(0.9558134246575278, 1106.5) [a] 
(0.9558157534246511, 1107) [a] 
(0.9558161643835552, 1113.8) [a] 
(0.9558171232876648, 1116.7) [a] 
(0.9558308219178018, 1126) [a] 
(0.9558310958904045, 1138.5) [a] 
(0.9558319178082127, 1140.8) [a] 
(0.9558447945205415, 1170.7) [a] 
(0.9558576712328702, 1172) [a] 
(0.9558661643835552, 1178.5) [a] 
(0.9558834246575277, 1178.8) [a] 
(0.9558876712328702, 1180.2) [a] 
(0.9558904109588976, 1181) [a] 
(0.9559041095890346, 1191.7) [a] 
(0.955906849315062, 1215) [a] 
(0.9559071232876647, 1217.5) [a] 
(0.955933150684925, 1233) [a] 
(0.9560015068493086, 1236.4) [a] 
(0.956018356164377, 1237) [a] 
(0.9560202739725961, 1245.8) [a] 
(0.956022054794514, 1245.9) [a] 
(0.9560343835616373, 1247) [a] 
(0.9560353424657468, 1247.9) [a] 
(0.9560476712328702, 1272) [a] 
(0.9560599999999935, 1294) [a] 
(0.9560602739725962, 1360.5) [a] 
(0.9560606849315003, 1368.7) [a] 
(0.9560675342465688, 1400) [a] 
(0.9560757534246509, 1401) [a] 
(0.9560826027397193, 1413) [a] 
(0.956083972602733, 1427.7) [a] 
(0.9560845205479385, 1429.3) [a] 
(0.9560872602739658, 1437) [a] 
(0.9560899999999932, 1481) [a] 
(0.9560902739725959, 1485.5) [a] 
(0.9560930136986233, 1533) [a] 
(0.9560957534246507, 1536) [a] 
(0.9560972602739658, 1571.6) [a] 
(0.9560986301369795, 1578.7) [a] 
(0.9560993150684863, 1584.4) [a] 
(0.9561034246575274, 1596) [a] 
(0.956104383561637, 1600.5) [a] 
(0.9561045205479384, 1601.2) [a] 
(0.956109178082185, 1617.5) [a] 
(0.956110273972596, 1629.2) [a] 
(0.9561130136986233, 1642) [a] 
(0.9561267123287603, 1650) [a] 
(0.9561390410958837, 1665) [a] 
(0.9561445205479384, 1669) [a] 
(0.9561513698630069, 1670) [a] 
(0.9561568493150616, 1671) [a] 
(0.956159589041089, 1723) [a] 
(0.9561597260273904, 1744.6) [a] 
(0.9561599999999931, 1745.6) [a] 
(0.9561602739725958, 1745.9) [a] 
(0.9561605479451986, 1800.1) [a] 
(0.9561608219178013, 1806.5) [a] 
(0.9561615068493081, 1806.6) [a] 
(0.9561619178082122, 1807.2) [a] 
(0.9561623287671164, 1811) [a] 
(0.9561801369862943, 1833) [a] 
(0.9561897260273902, 1834) [a] 
(0.9561901369862943, 1838.7) [a] 
(0.956190410958897, 1839.9) [a] 
(0.9561917808219107, 1841.4) [a] 
(0.9561945205479381, 1854) [a] 
(0.9561947945205408, 1854.3) [a] 
(0.9561975342465682, 1855) [a] 
(0.9562002739725955, 1856) [a] 
(0.9562009589041024, 1856.5) [a] 
(0.9562013698630065, 1858.3) [a] 
(0.9562017808219107, 1862.6) [a] 
(0.9562021917808148, 1872.7) [a] 
(0.9562027397260202, 1875.5) [a] 
(0.9562031506849243, 1875.6) [a] 
(0.9562034246575271, 1896.8) [a] 
(0.9562036986301298, 1898.5) [a] 
(0.9562038356164312, 1899.9) [a] 
(0.9562065753424586, 1902) [a] 
(0.9562068493150613, 1916) [a] 
(0.9562073972602667, 1919.8) [a] 
(0.9562076712328694, 1927.3) [a] 
(0.9562080821917736, 1955.6) [a] 
(0.9562094520547872, 1956) [a] 
(0.9562104109588968, 1957.6) [a] 
(0.9562145205479379, 1971) [a] 
(0.9562610958904035, 1976) [a] 
(0.9562613698630062, 1976.1) [a] 
(0.9562617808219104, 1978.6) [a] 
(0.9562620547945131, 1982.3) [a] 
(0.9562621917808145, 1982.4) [a] 
(0.9562624657534172, 1990.2) [a] 
(0.9562627397260199, 1994.1) [a] 
(0.9563463013698558, 2002) [a] 
(0.9563464383561572, 2047.6) [a] 
(0.956350821917801, 2050.3) [a] 
(0.9563510958904037, 2067) [a] 
(0.9563553424657462, 2091.8) [a] 
(0.9563595890410886, 2101) [a] 
(0.9563983561643762, 2101.1) [a] 
(0.9564026027397187, 2101.4) [a] 
(0.9564027397260201, 2105.5) [a] 
(0.9564113698630063, 2110.3) [a] 
(0.956422328767116, 2142) [a] 
(0.9564228767123214, 2144.5) [a] 
(0.956447534246568, 2167) [a] 
(0.9564543835616365, 2174) [a] 
(0.9564546575342392, 2266) [a] 
(0.9564547945205406, 2267.2) [a] 
(0.9564550684931433, 2269.7) [a] 
(0.9564554794520475, 2269.9) [a] 
(0.9564557534246502, 2281.9) [a] 
(0.9564584931506775, 2300) [a] 
(0.9564589041095817, 2368.7) [a] 
(0.9564630136986227, 2389) [a] 
(0.9564671232876638, 2392) [a] 
(0.9564672602739652, 2397.2) [a] 
(0.9564699999999926, 2428) [a] 
(0.9564809589041022, 2449) [a] 
(0.9565056164383489, 2451) [a] 
(0.9565165753424585, 2456) [a] 
(0.956520821917801, 2479.9) [a] 
(0.9565215068493078, 2530.9) [a] 
(0.9565216438356092, 2548.9) [a] 
(0.9565302739725955, 2549.9) [a] 
(0.9565304109588969, 2556.2) [a] 
(0.9565306849314996, 2575.3) [a] 
(0.9565320547945133, 2665) [a] 
(0.956560821917801, 2666) [a] 
(0.9565621917808147, 2678) [a] 
(0.9565649315068421, 2684) [a] 
(0.956566027397253, 2741) [a] 
(0.9565663013698558, 2756.3) [a] 
(0.9565799999999928, 2760) [a] 
(0.9565813698630065, 2768) [a] 
(0.9565854794520475, 2773) [a] 
(0.9565978082191708, 2775) [a] 
(0.9565980821917736, 2835.1) [a] 
(0.9565995890410887, 2842.9) [a] 
(0.9565997260273901, 2852.9) [a] 
(0.9565999999999928, 2853) [a] 
(0.9566002739725955, 2894.7) [a] 
(0.9566005479451982, 2901.6) [a] 
(0.9566019178082119, 2930) [a] 
(0.9566189041095817, 2957.7) [a] 
(0.9566526027397187, 2957.8) [a] 
(0.9567031506849243, 2958.8) [a] 
(0.9567369863013626, 2959) [a] 
(0.9567538356164311, 2960) [a] 
(0.9567894520547873, 3000) [a] 
(0.9568064383561571, 3033.6) [a] 
(0.9568401369862941, 3033.8) [a] 
(0.9568906849314996, 3034.7) [a] 
(0.956924520547938, 3034.9) [a] 
(0.9569413698630065, 3035.9) [a] 
(0.9569421917808147, 3188.4) [a] 
(0.9569458904109517, 3229.8) [a] 
(0.9569472602739654, 3231) [a] 
(0.9569482191780749, 3233.7) [a] 
(0.9569495890410886, 3242) [a] 
(0.9569509589041023, 3334) [a] 
(0.9569687671232804, 3343) [a] 
(0.9569728767123215, 3478.5) [a] 
(0.9569771232876639, 3483.8) [a] 
(0.9570047945205407, 3514.7) [a] 
(0.9570139726027325, 3516.5) [a] 
(0.9570232876712256, 3531.9) [a] 
(0.9570416438356092, 3540.2) [a] 
(0.9570443835616366, 3542) [a] 
(0.9570498630136913, 3548) [a] 
(0.9570526027397187, 3552) [a] 
(0.9570710958904036, 3556.3) [a] 
(0.9570987671232803, 3556.4) [a] 
(0.9571265753424584, 3556.8) [a] 
(0.9571819178082118, 3557.8) [a] 
(0.9571846575342392, 3562) [a] 
},{(0.8409422201844767, 0) [b] 
(0.8785613862012297, 0.001) [b] 
(0.8842536400495972, 0.002) [b] 
(0.8866360940434371, 0.003) [b] 
(0.8894388030721719, 0.004) [b] 
(0.8920894459338549, 0.005) [b] 
(0.8930915423785478, 0.006) [b] 
(0.8937002138683207, 0.007) [b] 
(0.8951468637628993, 0.008) [b] 
(0.8978414544934821, 0.009) [b] 
(0.898924935721479, 0.01) [b] 
(0.8993738114642417, 0.011) [b] 
(0.8998480624184614, 0.012) [b] 
(0.9003716935058265, 0.013) [b] 
(0.9005937095936516, 0.014) [b] 
(0.9015209225304416, 0.015) [b] 
(0.9028208070289961, 0.016) [b] 
(0.9053475728020446, 0.017) [b] 
(0.9062294410128422, 0.018) [b] 
(0.9075139021383288, 0.019) [b] 
(0.9076141726820337, 0.02) [b] 
(0.9080027033420485, 0.021) [b] 
(0.9085352324700763, 0.022) [b] 
(0.9086684200495498, 0.023) [b] 
(0.9090056557119436, 0.024) [b] 
(0.909166458326771, 0.025) [b] 
(0.9091923863333394, 0.026) [b] 
(0.9094007179777633, 0.027) [b] 
(0.909819443443096, 0.028) [b] 
(0.9100084682586714, 0.029) [b] 
(0.9100389542743315, 0.03) [b] 
(0.9103332566217355, 0.031) [b] 
(0.9107510902330811, 0.032) [b] 
(0.9108627400583889, 0.033) [b] 
(0.9110614387721153, 0.034) [b] 
(0.9111726147615028, 0.037) [b] 
(0.9111779160889552, 0.038) [b] 
(0.9112024584314415, 0.041) [b] 
(0.9112132617675117, 0.043) [b] 
(0.9117721658771007, 0.046) [b] 
(0.9120802106087824, 0.047) [b] 
(0.9122768036398837, 0.049) [b] 
(0.9124742936867157, 0.05) [b] 
(0.9125713256501861, 0.051) [b] 
(0.9125726955131997, 0.052) [b] 
(0.9128242374293434, 0.053) [b] 
(0.9128398127293985, 0.054) [b] 
(0.9128821705734721, 0.055) [b] 
(0.9129557736979051, 0.056) [b] 
(0.9130426512535481, 0.057) [b] 
(0.9130660481931889, 0.058) [b] 
(0.9130824865493533, 0.06) [b] 
(0.9131181600653351, 0.061) [b] 
(0.9131486448717636, 0.063) [b] 
(0.9131491014927682, 0.064) [b] 
(0.9131812780523871, 0.065) [b] 
(0.9131853876414282, 0.066) [b] 
(0.9131858442624328, 0.067) [b] 
(0.9131872141254465, 0.068) [b] 
(0.9132072706849442, 0.069) [b] 
(0.9132494385984603, 0.07) [b] 
(0.913264718836998, 0.071) [b] 
(0.9136940809262909, 0.072) [b] 
(0.9137002453098525, 0.073) [b] 
(0.913740358428848, 0.074) [b] 
(0.9141321392507659, 0.075) [b] 
(0.9142032812604697, 0.076) [b] 
(0.9142067059180039, 0.077) [b] 
(0.914319719616634, 0.078) [b] 
(0.9143217744111546, 0.079) [b] 
(0.9144551362833007, 0.08) [b] 
(0.9145882413061318, 0.081) [b] 
(0.9145924601635809, 0.082) [b] 
(0.9146234290546662, 0.083) [b] 
(0.9146316482327483, 0.084) [b] 
(0.9146394107898259, 0.085) [b] 
(0.9146468406529003, 0.087) [b] 
(0.9146509502419413, 0.088) [b] 
(0.9147886660868044, 0.089) [b] 
(0.9148065028447953, 0.09) [b] 
(0.9148630431187982, 0.095) [b] 
(0.9149020705550859, 0.102) [b] 
(0.9150170200061423, 0.104) [b] 
(0.9150886563596075, 0.106) [b] 
(0.9152046795322131, 0.107) [b] 
(0.9152408618599818, 0.109) [b] 
(0.9153093550106667, 0.116) [b] 
(0.9159325534039353, 0.118) [b] 
(0.9160153359612592, 0.119) [b] 
(0.918249546279099, 0.12) [b] 
(0.9182654011750909, 0.121) [b] 
(0.9183125176216955, 0.122) [b] 
(0.9188535109006448, 0.123) [b] 
(0.9189035146065095, 0.124) [b] 
(0.9189041995380164, 0.126) [b] 
(0.9189556336937247, 0.127) [b] 
(0.9190396948203613, 0.128) [b] 
(0.9190654118982154, 0.129) [b] 
(0.9191508000260693, 0.13) [b] 
(0.9191650284280101, 0.131) [b] 
(0.9191742593108517, 0.132) [b] 
(0.9191806520049156, 0.133) [b] 
(0.9192297251033038, 0.134) [b] 
(0.9192326313142232, 0.136) [b] 
(0.9192526878737209, 0.137) [b] 
(0.9202737610259714, 0.139) [b] 
(0.9206585540073474, 0.142) [b] 
(0.9206714378582256, 0.143) [b] 
(0.9206720676803009, 0.144) [b] 
(0.9206734375433145, 0.145) [b] 
(0.9206953553515337, 0.146) [b] 
(0.9207364512419446, 0.147) [b] 
(0.9207542594611227, 0.149) [b] 
(0.9207563142556432, 0.15) [b] 
(0.9207574558081546, 0.155) [b] 
(0.9208113370866934, 0.156) [b] 
(0.920917501470255, 0.157) [b] 
(0.9209590539816706, 0.158) [b] 
(0.921081200100392, 0.16) [b] 
(0.9211330265844102, 0.161) [b] 
(0.921250606493086, 0.162) [b] 
(0.9212725243013051, 0.163) [b] 
(0.9213169604839917, 0.164) [b] 
(0.9215658189314803, 0.165) [b] 
(0.9219822572876446, 0.166) [b] 
(0.9222308874246309, 0.167) [b] 
(0.922248695643809, 0.168) [b] 
(0.9222591979269139, 0.169) [b] 
(0.9223119376529413, 0.17) [b] 
(0.9223966135300828, 0.171) [b] 
(0.9228236346099616, 0.172) [b] 
(0.9228555980802813, 0.173) [b] 
(0.9229393880346192, 0.174) [b] 
(0.9229695250209206, 0.175) [b] 
(0.9230578811853042, 0.176) [b] 
(0.9230661003633863, 0.177) [b] 
(0.9230667852948932, 0.178) [b] 
(0.923511762463843, 0.179) [b] 
(0.9238804839250302, 0.18) [b] 
(0.9238809405460348, 0.181) [b] 
(0.9243371049295963, 0.182) [b] 
(0.9244151871213772, 0.183) [b] 
(0.9246696456518337, 0.184) [b] 
(0.9247703305833406, 0.185) [b] 
(0.9248963579806008, 0.186) [b] 
(0.9249933899440711, 0.187) [b] 
(0.9251415634600528, 0.188) [b] 
(0.9251493260171304, 0.189) [b] 
(0.9251703305833404, 0.19) [b] 
(0.9251865406290025, 0.191) [b] 
(0.9251874538710116, 0.192) [b] 
(0.9252091167966027, 0.193) [b] 
(0.9252720507788553, 0.194) [b] 
(0.9254140205473957, 0.195) [b] 
(0.92541744520493, 0.197) [b] 
(0.9254704132414596, 0.198) [b] 
(0.9254911894971674, 0.2) [b] 
(0.9259179252835358, 0.202) [b] 
(0.926069751767554, 0.205) [b] 
(0.926072948114586, 0.207) [b] 
(0.926204708004361, 0.208) [b] 
(0.9265704614290186, 0.21) [b] 
(0.9265711463605254, 0.212) [b] 
(0.9265839317486533, 0.215) [b] 
(0.9266051811501661, 0.217) [b] 
(0.9268397978828182, 0.218) [b] 
(0.9268619440015397, 0.219) [b] 
(0.9268904828143251, 0.222) [b] 
(0.926928382357704, 0.223) [b] 
(0.9269411271911231, 0.225) [b] 
(0.9269464285185754, 0.228) [b] 
(0.9269902641350138, 0.234) [b] 
(0.9269953027116159, 0.241) [b] 
(0.9270044351317072, 0.242) [b] 
(0.9270263529399263, 0.245) [b] 
(0.9270393374234684, 0.246) [b] 
(0.9270783648597561, 0.249) [b] 
(0.9270984214192539, 0.261) [b] 
(0.9270991063507608, 0.266) [b] 
(0.9271344525230206, 0.271) [b] 
(0.9271498962999507, 0.281) [b] 
(0.9271795766652475, 0.286) [b] 
(0.9276831545958043, 0.289) [b] 
(0.9278735786463619, 0.291) [b] 
(0.9278749485093756, 0.292) [b] 
(0.9278868500125355, 0.293) [b] 
(0.9279939635409741, 0.294) [b] 
(0.9280032298071321, 0.297) [b] 
(0.9280495611379224, 0.328) [b] 
(0.9280504971563807, 0.333) [b] 
(0.9280570492855886, 0.335) [b] 
(0.9280620878621907, 0.338) [b] 
(0.9280768886809594, 0.339) [b] 
(0.9280778334140722, 0.34) [b] 
(0.9281508927748028, 0.341) [b] 
(0.928330116519095, 0.344) [b] 
(0.9284143630944375, 0.346) [b] 
(0.9284248653775425, 0.352) [b] 
(0.9285325075002452, 0.356) [b] 
(0.9285343795371618, 0.36) [b] 
(0.9285449821920666, 0.363) [b] 
(0.928618927650271, 0.371) [b] 
(0.9286242289777233, 0.377) [b] 
(0.9286404212119278, 0.379) [b] 
(0.9286728056803367, 0.383) [b] 
(0.9287129187993323, 0.386) [b] 
(0.9288156585253596, 0.389) [b] 
(0.9288318507595641, 0.394) [b] 
(0.9288480429937686, 0.399) [b] 
(0.9288735014941311, 0.404) [b] 
(0.9288910635913492, 0.409) [b] 
(0.9288931183858697, 0.424) [b] 
(0.9289068170160067, 0.44) [b] 
(0.9289334652449799, 0.443) [b] 
(0.9289341854673846, 0.454) [b] 
(0.9289460576135034, 0.457) [b] 
(0.9289629764324047, 0.463) [b] 
(0.9291591807703041, 0.47) [b] 
(0.929177017528295, 0.471) [b] 
(0.929185902645945, 0.475) [b] 
(0.9292215761619268, 0.479) [b] 
(0.9293754933544772, 0.481) [b] 
(0.929393330112468, 0.484) [b] 
(0.9293940150439749, 0.486) [b] 
(0.9294118518019657, 0.487) [b] 
(0.9294296885599566, 0.488) [b] 
(0.9294317433544771, 0.502) [b] 
(0.9294337981489976, 0.507) [b] 
(0.9294480973453828, 0.521) [b] 
(0.9294487822768897, 0.544) [b] 
(0.929451522002917, 0.549) [b] 
(0.92968591220549, 0.557) [b] 
(0.9296863688264946, 0.564) [b] 
(0.9296884236210151, 0.569) [b] 
(0.9296897934840288, 0.573) [b] 
(0.9297076302420196, 0.577) [b] 
(0.9297083151735265, 0.58) [b] 
(0.9297281781872251, 0.589) [b] 
(0.9297374090700667, 0.593) [b] 
(0.9297463131796558, 0.596) [b] 
(0.9297490529056832, 0.629) [b] 
(0.9300017778894198, 0.633) [b] 
(0.9301322410335816, 0.638) [b] 
(0.9301757287483022, 0.639) [b] 
(0.9301810300757546, 0.641) [b] 
(0.9302245177904752, 0.645) [b] 
(0.9303022073225128, 0.673) [b] 
(0.9303042621170333, 0.692) [b] 
(0.9303084809744824, 0.698) [b] 
(0.9303829101982267, 0.721) [b] 
(0.9304916294850282, 0.723) [b] 
(0.9304984112089153, 0.733) [b] 
(0.930499315438767, 0.734) [b] 
(0.9304997675536928, 0.735) [b] 
(0.9305029323581734, 0.746) [b] 
(0.9305033844730992, 0.75) [b] 
(0.9305060971626541, 0.769) [b] 
(0.930508905218029, 0.774) [b] 
(0.9305116449440564, 0.797) [b] 
(0.9305886035403316, 0.803) [b] 
(0.9306831414046183, 0.814) [b] 
(0.9306999897368674, 0.852) [b] 
(0.9307052910643198, 0.89) [b] 
(0.9307105923917721, 0.896) [b] 
(0.9307502296317518, 0.907) [b] 
(0.930831769096853, 0.911) [b] 
(0.9309540782945046, 0.912) [b] 
(0.9309930317451311, 0.965) [b] 
(0.9310060162286733, 0.979) [b] 
(0.9310104587874983, 0.99) [b] 
(0.9310322026448586, 0.998) [b] 
(0.9310344239242712, 1.029) [b] 
(0.9310602261059221, 1.054) [b] 
(0.9310624473853346, 1.055) [b] 
(0.9310793176194935, 1.062) [b] 
(0.9310923021030356, 1.076) [b] 
(0.9311133352080584, 1.129) [b] 
(0.9311902938043336, 1.147) [b] 
(0.9312299310443133, 1.231) [b] 
(0.9312360540212434, 1.243) [b] 
(0.9312444731145224, 1.244) [b] 
(0.9312451580460293, 1.259) [b] 
(0.9312889936624675, 1.262) [b] 
(0.9312910484569881, 1.268) [b] 
(0.9312918138291043, 1.279) [b] 
(0.9312925792012205, 1.327) [b] 
(0.9312933445733368, 1.329) [b] 
(0.9312964060618019, 1.334) [b] 
(0.9312970909933087, 1.347) [b] 
(0.9313361184295964, 1.354) [b] 
(0.9313751458658841, 1.356) [b] 
(0.9314532007384595, 1.358) [b] 
(0.9314611281864554, 1.368) [b] 
(0.9314690556344514, 1.369) [b] 
(0.9314715670499765, 1.424) [b] 
(0.9314874219459683, 1.536) [b] 
(0.9314953493939643, 1.538) [b] 
(0.9314958060149688, 1.541) [b] 
(0.9315011073424212, 1.603) [b] 
(0.9316070928753959, 1.61) [b] 
(0.9316078582475121, 1.663) [b] 
(0.9316085784699168, 1.667) [b] 
(0.9316107391371309, 1.675) [b] 
(0.9316444796054486, 1.678) [b] 
(0.9316782200737663, 1.682) [b] 
(0.9316796605185756, 1.699) [b] 
(0.9316814870025939, 1.706) [b] 
(0.9316957861989791, 1.709) [b] 
(0.9317000050564281, 1.74) [b] 
(0.931717841814419, 1.75) [b] 
(0.931752545010766, 1.885) [b] 
(0.9318404130680911, 1.9) [b] 
(0.9318531387867381, 1.943) [b] 
(0.9318855588780623, 1.948) [b] 
(0.9347868827491775, 1.978) [b] 
(0.9347871857424787, 1.986) [b] 
(0.9348529391671362, 2.013) [b] 
(0.9348967747835746, 2.019) [b] 
(0.9349317062904239, 2.042) [b] 
(0.9349589756875247, 2.066) [b] 
(0.9349808934957439, 2.078) [b] 
(0.9350247291121823, 2.091) [b] 
(0.9350466469204014, 2.098) [b] 
(0.9351080624455156, 2.122) [b] 
(0.9351500715779357, 2.141) [b] 
(0.935251898061954, 2.169) [b] 
(0.9352587473770224, 2.177) [b] 
(0.9352593533636246, 2.2) [b] 
(0.9352599593502269, 2.203) [b] 
(0.9353117858342451, 2.205) [b] 
(0.9353140689392679, 2.275) [b] 
(0.9353325307049513, 2.279) [b] 
(0.9353417615877929, 2.288) [b] 
(0.9353602233534762, 2.308) [b] 
(0.9353809542809877, 2.311) [b] 
(0.9354028720892068, 2.335) [b] 
(0.935424789897426, 2.336) [b] 
(0.9354250789122889, 2.351) [b] 
(0.9354269053963071, 2.383) [b] 
(0.9354577273141154, 2.393) [b] 
(0.9354886148679755, 2.417) [b] 
(0.9354892997994824, 2.424) [b] 
(0.9354940943200303, 2.447) [b] 
(0.9355002718308023, 2.689) [b] 
(0.9355244783863936, 2.753) [b] 
(0.9355258482494073, 2.766) [b] 
(0.9355260236592304, 2.775) [b] 
(0.9355267085907373, 2.796) [b] 
(0.9357338675918485, 2.812) [b] 
(0.9357868603583359, 2.888) [b] 
(0.9358543412949714, 2.926) [b] 
(0.9358880817632891, 2.927) [b] 
(0.9359555626999245, 2.935) [b] 
(0.9359724329340834, 2.936) [b] 
(0.9359893031682422, 2.938) [b] 
(0.9360061734024011, 2.94) [b] 
(0.93602304363656, 2.949) [b] 
(0.9360399138707188, 3.006) [b] 
(0.9360567841048777, 3.017) [b] 
(0.9361516210081287, 3.019) [b] 
(0.9368979898055276, 3.072) [b] 
(0.9368984464265322, 3.081) [b] 
(0.936915316660691, 3.104) [b] 
(0.9369321868948499, 3.106) [b] 
(0.9369490571290088, 3.113) [b] 
(0.9369659273631676, 3.115) [b] 
(0.9369662781828139, 3.118) [b] 
(0.936966453592637, 3.164) [b] 
(0.936980152222774, 3.186) [b] 
(0.936993850852911, 3.197) [b] 
(0.9369981450943332, 3.241) [b] 
(0.937057691410947, 3.247) [b] 
(0.9370755553059311, 3.266) [b] 
(0.9370798932054745, 3.339) [b] 
(0.9370841120629235, 3.348) [b] 
(0.9377204989012229, 3.4) [b] 
(0.937751386455083, 3.457) [b] 
(0.9377515618649062, 3.47) [b] 
(0.9377575164965676, 3.513) [b] 
(0.9377605464965979, 3.522) [b] 
(0.9377635764966282, 3.526) [b] 
(0.9378405350929034, 3.64) [b] 
(0.9378464897245649, 3.667) [b] 
(0.9382225683853459, 3.668) [b] 
(0.9382427405150053, 3.693) [b] 
(0.9382438820675167, 3.77) [b] 
(0.9384933148360012, 3.837) [b] 
(0.938500283681927, 4.382) [b] 
(0.938504502539376, 4.505) [b] 
(0.9385087213968251, 4.612) [b] 
(0.9385129402542741, 4.628) [b] 
(0.938530777012265, 4.784) [b] 
(0.938538082948338, 4.942) [b] 
(0.9385559197063288, 5.337) [b] 
(0.9386094299803014, 5.338) [b] 
(0.9386272667382922, 5.356) [b] 
(0.9386628072088925, 5.415) [b] 
(0.9386783561647801, 5.416) [b] 
(0.9386939051206677, 5.424) [b] 
(0.9387027902383178, 5.425) [b] 
(0.9387433913406681, 5.431) [b] 
(0.9387491482133894, 5.457) [b] 
(0.9387497541999916, 5.469) [b] 
(0.9387536931129062, 5.483) [b] 
(0.9388066858793936, 5.548) [b] 
(0.9388423593953754, 5.557) [b] 
(0.9388601961533662, 5.574) [b] 
(0.9388692861534571, 5.629) [b] 
(0.9388753461535178, 5.649) [b] 
(0.9390260310850246, 5.726) [b] 
(0.9390534283452986, 5.728) [b] 
(0.9390808256055726, 5.729) [b] 
(0.9390945242357096, 5.732) [b] 
(0.9391314477670761, 5.739) [b] 
(0.9391406786499178, 5.74) [b] 
(0.9391591404156011, 5.743) [b] 
(0.9391683712984428, 5.749) [b] 
(0.9391820699285798, 5.753) [b] 
(0.9392005316942631, 5.755) [b] 
(0.9392007600047654, 5.777) [b] 
(0.9392829517855873, 5.785) [b] 
(0.9393240476759983, 5.786) [b] 
(0.9393377463061353, 5.814) [b] 
(0.93942082425171, 5.815) [b] 
(0.9394485169002349, 5.816) [b] 
(0.9394663536582257, 5.963) [b] 
(0.9394755845410674, 6.016) [b] 
(0.9394822483793049, 6.071) [b] 
(0.9394844696587175, 6.077) [b] 
(0.9394889122175425, 6.088) [b] 
(0.9395541437896234, 6.274) [b] 
(0.9395719805476143, 6.578) [b] 
(0.9395742018270268, 6.837) [b] 
(0.9395841483553945, 7.049) [b] 
(0.9395846004703203, 7.052) [b] 
(0.9395864089300235, 7.06) [b] 
(0.9395891216195784, 7.078) [b] 
(0.9395898869916947, 7.106) [b] 
(0.9395907912215463, 7.245) [b] 
(0.9395912433364721, 7.248) [b] 
(0.9396071473188293, 7.314) [b] 
(0.9396124486462817, 7.316) [b] 
(0.9396334266403659, 7.489) [b] 
(0.93965440463445, 7.49) [b] 
(0.9396548567493759, 8.051) [b] 
(0.9396564549228918, 8.281) [b] 
(0.9396573591527435, 8.424) [b] 
(0.9396716583491287, 8.43) [b] 
(0.9396846428326708, 8.931) [b] 
(0.9397000866096009, 8.995) [b] 
(0.9397265829928446, 9.1) [b] 
(0.939730920892388, 9.301) [b] 
(0.9398848380849384, 9.691) [b] 
(0.9398976966238655, 9.965) [b] 
(0.9399155333818563, 9.99) [b] 
(0.9399924919781315, 10.107) [b] 
(0.9400134699722157, 10.611) [b] 
(0.9400344479662999, 10.612) [b] 
(0.9400387341459422, 10.634) [b] 
(0.9400391907669468, 10.643) [b] 
(0.9400396428818726, 11.111) [b] 
(0.9401211823469736, 11.139) [b] 
(0.9401619520795241, 11.142) [b] 
(0.9402027218120746, 11.143) [b] 
(0.9402434915446252, 11.162) [b] 
(0.9402842612771757, 11.201) [b] 
(0.9403250310097262, 11.453) [b] 
(0.9403335980079357, 11.573) [b] 
(0.9403514347659265, 11.595) [b] 
(0.940392204498477, 12.082) [b] 
(0.9403952344985074, 12.328) [b] 
(0.9403982644985377, 12.34) [b] 
(0.9405648855427756, 12.56) [b] 
(0.9405767870459355, 12.561) [b] 
(0.9406243930585749, 12.563) [b] 
(0.9406362945617348, 12.565) [b] 
(0.9406481960648947, 12.57) [b] 
(0.9406524149223437, 12.608) [b] 
(0.9406643164255036, 12.62) [b] 
(0.9406881194318234, 12.621) [b] 
(0.9407000209349833, 12.627) [b] 
(0.9407217647923436, 12.632) [b] 
(0.9407336662955035, 12.691) [b] 
(0.940769370804983, 12.894) [b] 
(0.9407812723081429, 12.898) [b] 
(0.9408022503022271, 13.373) [b] 
(0.9408287466854708, 13.395) [b] 
(0.9408379775683124, 13.457) [b] 
(0.9408564393339958, 13.46) [b] 
(0.9408656702168374, 13.681) [b] 
(0.9408841319825207, 13.686) [b] 
(0.9409019687405116, 13.837) [b] 
(0.9409427384730621, 14.988) [b] 
(0.9409581822499922, 15.43) [b] 
(0.9409736260269222, 15.438) [b] 
(0.9409766033427529, 15.583) [b] 
(0.9410030997259966, 15.956) [b] 
(0.9410110271739925, 16.18) [b] 
(0.9410467006899743, 16.508) [b] 
(0.9410823742059561, 16.644) [b] 
(0.9411002109639469, 16.771) [b] 
(0.9411055122913993, 17.011) [b] 
(0.9411108136188516, 17.072) [b] 
(0.9411286503768425, 17.096) [b] 
(0.9411339517042948, 17.132) [b] 
(0.9411392530317472, 17.209) [b] 
(0.9411445543591995, 17.21) [b] 
(0.9411498556866519, 17.218) [b] 
(0.9411514538601679, 17.332) [b] 
(0.9411692906181587, 17.407) [b] 
(0.941174591945611, 17.481) [b] 
(0.9411798932730634, 17.487) [b] 
(0.9411999498325612, 17.679) [b] 
(0.9412209278266453, 17.816) [b] 
(0.9412419058207295, 17.82) [b] 
(0.9412441101979929, 18.132) [b] 
(0.9412450549311058, 18.201) [b] 
(0.9412529823791017, 18.251) [b] 
(0.9412609098270976, 18.258) [b] 
(0.9412688372750936, 18.941) [b] 
(0.9412767647230895, 19.045) [b] 
(0.9412846921710855, 19.125) [b] 
(0.9412926196190814, 19.191) [b] 
(0.9412933045505882, 19.337) [b] 
(0.9412960442766156, 19.44) [b] 
(0.9413075678350905, 20.175) [b] 
(0.941321972283184, 20.18) [b] 
(0.9413234127279934, 20.201) [b] 
(0.9413241329503981, 20.218) [b] 
(0.9413270138400168, 20.237) [b] 
(0.9413313351744449, 20.294) [b] 
(0.9413738282963209, 20.732) [b] 
(0.9413882327444144, 20.738) [b] 
(0.9413889529668191, 20.758) [b] 
(0.94140678972481, 20.91) [b] 
(0.941419144746354, 21.065) [b] 
(0.9414314997678981, 21.093) [b] 
(0.9414345885232841, 21.141) [b] 
(0.9414524252812749, 21.312) [b] 
(0.941455514036661, 21.317) [b] 
(0.9414733507946518, 21.354) [b] 
(0.9414902210288106, 22.041) [b] 
(0.9415070912629695, 22.1) [b] 
(0.9415114291625128, 22.246) [b] 
(0.941512342404522, 22.409) [b] 
(0.9415202698525179, 23.616) [b] 
(0.9415381066105087, 23.818) [b] 
(0.9415387915420156, 23.836) [b] 
(0.9415566283000064, 23.893) [b] 
(0.9417105454925568, 25.078) [b] 
(0.941721961017671, 26.167) [b] 
(0.9417274404697258, 26.218) [b] 
(0.9417329199217805, 26.332) [b] 
(0.9417336401441853, 26.466) [b] 
(0.9417352383177012, 26.602) [b] 
(0.94173811920732, 26.889) [b] 
(0.9417590972014042, 27.685) [b] 
(0.9417604670644178, 27.92) [b] 
(0.9417609236854224, 28.485) [b] 
(0.9417652615849658, 28.612) [b] 
(0.9417675221595948, 28.967) [b] 
(0.9417679742745206, 28.968) [b] 
(0.9417715911939271, 28.969) [b] 
(0.9417741026094522, 29.154) [b] 
(0.9417782121984932, 30.06) [b] 
(0.9417859747555708, 30.069) [b] 
(0.9417864313765754, 30.118) [b] 
(0.9418033016107342, 30.302) [b] 
(0.9418201718448931, 30.483) [b] 
(0.9418971304411683, 31.848) [b] 
(0.9419014246825905, 32.117) [b] 
(0.9419186016482796, 32.124) [b] 
(0.9419228958897018, 32.131) [b] 
(0.941927190131124, 32.144) [b] 
(0.9419615440625021, 32.151) [b] 
(0.9419658383039243, 32.211) [b] 
(0.9419688997923894, 32.834) [b] 
(0.9419696651645056, 32.857) [b] 
(0.9419711959087382, 32.875) [b] 
(0.9419734920250871, 32.876) [b] 
(0.9419742573972033, 32.877) [b] 
(0.9419785516386255, 32.994) [b] 
(0.9419800823828581, 33.825) [b] 
(0.9419808477549744, 33.827) [b] 
(0.9419816131270906, 33.836) [b] 
(0.9419823784992069, 34.407) [b] 
(0.9419831438713231, 34.466) [b] 
(0.9419839092434393, 34.934) [b] 
(0.9420311458990841, 36.583) [b] 
(0.9420569113476176, 36.591) [b] 
(0.9420688128507775, 38.291) [b] 
(0.9420842566277076, 39.138) [b] 
(0.9420872866277379, 39.802) [b] 
(0.9420957243426359, 39.841) [b] 
(0.9420999432000849, 39.852) [b] 
(0.9421105458549898, 40.075) [b] 
(0.9421169385490537, 40.819) [b] 
(0.9421171668595559, 40.897) [b] 
(0.9421173951700582, 41.934) [b] 
(0.9421189933435742, 42.001) [b] 
(0.9421326919737112, 42.192) [b] 
(0.9421329202842135, 42.443) [b] 
(0.942135141563626, 48.423) [b] 
(0.9421550045773246, 48.487) [b] 
(0.9421728413353154, 49.076) [b] 
(0.942258001152667, 53.525) [b] 
(0.9422618824312058, 53.597) [b] 
(0.9422621107417081, 53.709) [b] 
(0.942268503435772, 53.876) [b] 
(0.9422703299197902, 54.112) [b] 
(0.9423048048056349, 55.738) [b] 
(0.9423100559471874, 55.823) [b] 
(0.9423870145434626, 56.229) [b] 
(0.9423999990270048, 56.976) [b] 
(0.9424154428039349, 57.016) [b] 
(0.9424414117710191, 57.377) [b] 
(0.9424543962545613, 57.385) [b] 
(0.9424673807381034, 57.445) [b] 
(0.9424803652216456, 57.497) [b] 
(0.9424805935321479, 58.013) [b] 
(0.94249357801569, 58.069) [b] 
(0.942496774362722, 58.163) [b] 
(0.9425136445968808, 58.183) [b] 
(0.9425642552993574, 58.184) [b] 
(0.9425811255335163, 58.223) [b] 
(0.9425979957676751, 60.119) [b] 
(0.9426486064701517, 60.12) [b] 
(0.9426654767043106, 60.158) [b] 
(0.9427169108600189, 60.615) [b] 
(0.9427211970396612, 60.627) [b] 
(0.9427383417582306, 61.818) [b] 
(0.9427552119923894, 61.904) [b] 
(0.9427720822265483, 61.909) [b] 
(0.9427850667100904, 62.14) [b] 
(0.9427980511936326, 63.252) [b] 
(0.9428149214277914, 63.854) [b] 
(0.9428317916619503, 63.857) [b] 
(0.9428406767796004, 63.889) [b] 
(0.9428495618972504, 63.901) [b] 
(0.942851783176663, 63.905) [b] 
(0.9428540044560755, 63.921) [b] 
(0.942856225735488, 64.058) [b] 
(0.9428584470149005, 64.104) [b] 
(0.942860668294313, 64.105) [b] 
(0.9428717746913756, 64.109) [b] 
(0.9428762172502007, 64.332) [b] 
(0.9428785003552235, 64.365) [b] 
(0.942880721634636, 64.372) [b] 
(0.9428827764291565, 65.468) [b] 
(0.9428836896711656, 65.573) [b] 
(0.9428859109505782, 66.242) [b] 
(0.9428881322299907, 66.247) [b] 
(0.9428925747888157, 66.309) [b] 
(0.9428959994463499, 68.017) [b] 
(0.9429014788984047, 68.022) [b] 
(0.9429021638299115, 68.03) [b] 
(0.9429055884874458, 68.037) [b] 
(0.9429069583504595, 68.196) [b] 
(0.94290901314498, 68.275) [b] 
(0.9429096980764868, 68.838) [b] 
(0.9429106023063385, 69.203) [b] 
(0.942912657100859, 69.35) [b] 
(0.9429133420323659, 69.414) [b] 
(0.9429140269638727, 69.492) [b] 
(0.9429147118953796, 70.956) [b] 
(0.9429222461419549, 70.961) [b] 
(0.9429400828999457, 71.332) [b] 
(0.9429410462828223, 71.761) [b] 
(0.9429415029038268, 72.332) [b] 
(0.9429437241832394, 72.878) [b] 
(0.9429479430406884, 74.445) [b] 
(0.9429502261457112, 74.948) [b] 
(0.9429621276488711, 75.15) [b] 
(0.9429628020168848, 75.492) [b] 
(0.9429670208743338, 75.529) [b] 
(0.9429677862464501, 75.579) [b] 
(0.9429685516185663, 75.613) [b] 
(0.9429693169906825, 75.771) [b] 
(0.9429700823627988, 75.773) [b] 
(0.9429918262201591, 75.819) [b] 
(0.9429960450776081, 80.695) [b] 
(0.9429982663570207, 81.114) [b] 
(0.9429987229780252, 84.491) [b] 
(0.9429994079095321, 85.618) [b] 
(0.9430079802688168, 85.73) [b] 
(0.9430086652003237, 85.74) [b] 
(0.9430296431944079, 86.235) [b] 
(0.9431345331648289, 86.236) [b] 
(0.943155511158913, 86.238) [b] 
(0.9431764891529972, 86.242) [b] 
(0.9431974671470814, 86.248) [b] 
(0.9432184451411656, 86.344) [b] 
(0.943260401129334, 86.377) [b] 
(0.9432813791234181, 86.378) [b] 
(0.9433023571175023, 86.413) [b] 
(0.9433028092324282, 86.533) [b] 
(0.943303261347354, 86.534) [b] 
(0.9433037134622798, 87.044) [b] 
(0.9433057682568003, 88.714) [b] 
(0.9433064531883072, 89.043) [b] 
(0.9433484091764756, 89.67) [b] 
(0.9433693871705597, 89.676) [b] 
(0.9433903651646439, 89.759) [b] 
(0.9433908217856485, 98.162) [b] 
(0.943394474753685, 99.358) [b] 
(0.9433951596851918, 99.891) [b] 
(0.943416137679276, 102.988) [b] 
(0.9434790716615287, 102.99) [b] 
(0.9435000496556128, 102.991) [b] 
(0.943521027649697, 103.003) [b] 
(0.9435629836378654, 103.055) [b] 
(0.9435839616319496, 103.058) [b] 
(0.9436077646382692, 115.197) [b] 
(0.9436091345012829, 115.969) [b] 
(0.9437416164175013, 116.338) [b] 
(0.9437448127645333, 116.496) [b] 
(0.9437455781366495, 116.688) [b] 
(0.9437875341248179, 116.991) [b] 
(0.943808512118902, 116.994) [b] 
(0.9438294901129862, 116.996) [b] 
(0.9438559864962299, 117.016) [b] 
(0.9438824828794736, 117.494) [b] 
(0.9441321974913457, 117.599) [b] 
(0.9441857077653183, 117.6) [b] 
(0.9442392180392909, 117.626) [b] 
(0.9442570547972817, 117.63) [b] 
(0.9443105650712543, 117.886) [b] 
(0.9443284018292452, 117.887) [b] 
(0.9443997488612086, 117.892) [b] 
(0.9444262452444523, 118.005) [b] 
(0.9444440820024431, 118.475) [b] 
(0.944461918760434, 118.5) [b] 
(0.9444797555184248, 118.586) [b] 
(0.9444818103129453, 119.987) [b] 
(0.9444996470709361, 121.76) [b] 
(0.9446245043768722, 122.166) [b] 
(0.9446958514088356, 122.167) [b] 
(0.9447136881668264, 122.17) [b] 
(0.9447315249248173, 122.174) [b] 
(0.9447493616828081, 122.179) [b] 
(0.9447671984407989, 122.18) [b] 
(0.944782642217729, 122.189) [b] 
(0.9447980859946591, 122.2) [b] 
(0.9448135297715892, 122.229) [b] 
(0.9448670400455618, 122.272) [b] 
(0.9448723413730141, 122.387) [b] 
(0.9449080148889959, 122.406) [b] 
(0.9449084715100005, 122.7) [b] 
(0.9449263082679913, 122.711) [b] 
(0.9449417520449214, 122.944) [b] 
(0.9449422086659259, 123.456) [b] 
(0.9449424369764282, 124.488) [b] 
(0.9449467231560705, 124.591) [b] 
(0.9449552955153553, 124.592) [b] 
(0.9449707392922854, 125.212) [b] 
(0.9449917172863695, 125.691) [b] 
(0.9450071610632996, 125.853) [b] 
(0.9450249978212905, 129.492) [b] 
(0.9450428345792813, 129.523) [b] 
(0.9450547360824412, 130.868) [b] 
(0.945066637585601, 130.873) [b] 
(0.9450668658961033, 130.969) [b] 
(0.9450787673992632, 132.864) [b] 
(0.9450789600758386, 134.523) [b] 
(0.9451027630821582, 134.628) [b] 
(0.9451057930821886, 148.228) [b] 
(0.9451212368591186, 148.907) [b] 
(0.9451624216712832, 149.183) [b] 
(0.9451716525541248, 149.281) [b] 
(0.9451917091136226, 150.031) [b] 
(0.9452318222326181, 150.032) [b] 
(0.9452530275424277, 150.464) [b] 
(0.9452636301973325, 150.466) [b] 
(0.9452689315247849, 150.47) [b] 
(0.9452742328522372, 150.509) [b] 
(0.9452795341796896, 150.55) [b] 
(0.9452848355071419, 150.852) [b] 
(0.9452954381620468, 150.857) [b] 
(0.9453007394894991, 150.968) [b] 
(0.9453185762474899, 152.466) [b] 
(0.9453364130054808, 152.481) [b] 
(0.9453417143329331, 152.913) [b] 
(0.945352316987838, 153.96) [b] 
(0.9453576183152903, 153.962) [b] 
(0.9453629196427427, 153.963) [b] 
(0.945368220970195, 153.964) [b] 
(0.9453735222976474, 156.439) [b] 
(0.9453819464637719, 157.748) [b] 
(0.9453872477912243, 157.972) [b] 
(0.9453925491186767, 157.984) [b] 
(0.945397850446129, 158.021) [b] 
(0.9454031517735814, 158.145) [b] 
(0.9454084531010337, 158.154) [b] 
(0.9454262898590245, 177.647) [b] 
(0.9454266752121752, 179.828) [b] 
(0.9454291866277003, 184.655) [b] 
(0.9454312414222208, 186.086) [b] 
(0.9454365427496731, 192.598) [b] 
(0.9454395727497035, 194.104) [b] 
(0.9454426615050895, 204.908) [b] 
(0.9454431136200153, 206.022) [b] 
(0.9454435657349411, 206.027) [b] 
(0.945444017849867, 206.047) [b] 
(0.9454444699647928, 206.072) [b] 
(0.9454449220797186, 206.397) [b] 
(0.9454499449107688, 214.466) [b] 
(0.945450858152778, 214.485) [b] 
(0.9454540544998099, 214.624) [b] 
(0.9455636435409058, 215.224) [b] 
(0.9458239175135086, 215.226) [b] 
(0.9458376161436456, 215.343) [b] 
(0.9459061092943305, 215.399) [b] 
(0.9459335065546045, 215.401) [b] 
(0.9459472051847415, 215.403) [b] 
(0.9459609038148785, 215.407) [b] 
(0.9460019997052895, 216.188) [b] 
(0.9460156983354265, 216.203) [b] 
(0.9460430955957005, 216.213) [b] 
(0.9460966058696731, 216.242) [b] 
(0.946114442627664, 216.244) [b] 
(0.9461322793856548, 216.245) [b] 
(0.9461679529016366, 216.249) [b] 
(0.9461857896596274, 216.261) [b] 
(0.9462214631756092, 216.511) [b] 
(0.9462392999336, 217.242) [b] 
(0.9462571366915908, 217.25) [b] 
(0.9462708353217278, 218.539) [b] 
(0.9462845339518648, 218.546) [b] 
(0.9462849905728694, 219.91) [b] 
(0.946285447193874, 221.989) [b] 
(0.9462868170568877, 222.964) [b] 
(0.9463005156870247, 223.987) [b] 
(0.9463006120253123, 226.533) [b] 
(0.9463143106554494, 226.987) [b] 
(0.9463165712300784, 227.735) [b] 
(0.9463192839196333, 228.606) [b] 
(0.9463511085472149, 237.288) [b] 
(0.946378505807489, 238.066) [b] 
(0.946392204437626, 238.067) [b] 
(0.9463944257170385, 244.535) [b] 
(0.946402188274116, 246.309) [b] 
(0.9464092658996868, 246.36) [b] 
(0.9464097225206913, 247.067) [b] 
(0.9464371197809653, 247.604) [b] 
(0.9464645170412394, 247.605) [b] 
(0.946465886904253, 247.761) [b] 
(0.946493284164527, 247.77) [b] 
(0.946506982794664, 248.174) [b] 
(0.9465206814248011, 250.449) [b] 
(0.9465211380458056, 253.946) [b] 
(0.9466319086399052, 263.81) [b] 
(0.946687293936955, 263.812) [b] 
(0.9466965248197967, 263.816) [b] 
(0.94671498658548, 263.936) [b] 
(0.9467242174683217, 263.938) [b] 
(0.9467334483511634, 263.956) [b] 
(0.946742679234005, 264.246) [b] 
(0.9467519101168467, 264.289) [b] 
(0.9467611409996883, 266.805) [b] 
(0.9467689035567659, 267.795) [b] 
(0.946783515428912, 267.863) [b] 
(0.9468019771945954, 268.489) [b] 
(0.9468050659499814, 268.638) [b] 
(0.9468327585985062, 271.255) [b] 
(0.9468419894813479, 271.256) [b] 
(0.9468604512470312, 271.257) [b] 
(0.9468696821298729, 271.289) [b] 
(0.946870595371882, 275.849) [b] 
(0.946909300477681, 279.066) [b] 
(0.9469122777935116, 279.202) [b] 
(0.9469212097410037, 279.308) [b] 
(0.9469241870568343, 282.322) [b] 
(0.9469378856869713, 283.215) [b] 
(0.9469515843171084, 283.221) [b] 
(0.9469652829472454, 283.385) [b] 
(0.9470206682442952, 283.607) [b] 
(0.9470298991271369, 283.61) [b] 
(0.9470391300099785, 283.619) [b] 
(0.9470421073258092, 283.939) [b] 
(0.9470558059559462, 284.128) [b] 
(0.9470650368387878, 284.242) [b] 
(0.9470680141546185, 286.713) [b] 
(0.9470709914704492, 286.955) [b] 
(0.9470802223532908, 293.088) [b] 
(0.9470894532361325, 293.093) [b] 
(0.9470986841189741, 293.153) [b] 
(0.9471171458846575, 293.686) [b] 
(0.9471263767674991, 295.335) [b] 
(0.9471448385331824, 295.338) [b] 
(0.9471478158490131, 296.532) [b] 
(0.9471507931648437, 296.596) [b] 
(0.947154217822378, 301.377) [b] 
(0.947181615082652, 304.759) [b] 
(0.9472585736789272, 305.546) [b] 
(0.9475681720167437, 321.276) [b] 
(0.9476451306130189, 324.961) [b] 
(0.9476633106132007, 325.24) [b] 
(0.9476814906133826, 325.245) [b] 
(0.9476845206134129, 325.354) [b] 
(0.9476875506134432, 325.43) [b] 
(0.9476905806134736, 327.853) [b] 
(0.9477042792436106, 337.254) [b] 
(0.9477224592437924, 337.294) [b] 
(0.947728519243853, 337.299) [b] 
(0.9477315492438834, 337.534) [b] 
(0.9477452478740204, 340.653) [b] 
(0.9477464757427823, 347.841) [b] 
(0.9477759445930674, 365.822) [b] 
(0.9477781658724799, 370.717) [b] 
(0.947849574891439, 370.988) [b] 
(0.9478614763945988, 370.989) [b] 
(0.9479090824072383, 370.991) [b] 
(0.9479209839103981, 371.028) [b] 
(0.947932885413558, 371.383) [b] 
(0.9479447869167179, 371.386) [b] 
(0.9479566884198778, 371.393) [b] 
(0.9479685899230377, 371.413) [b] 
(0.9479804914261976, 371.416) [b] 
(0.9480161959356771, 372.129) [b] 
(0.948028097438837, 372.13) [b] 
(0.9480399989419969, 372.133) [b] 
(0.9480638019483165, 372.149) [b] 
(0.9480876049546362, 372.153) [b] 
(0.9480995064577961, 372.274) [b] 
(0.9481233094641157, 372.324) [b] 
(0.9481352109672756, 372.325) [b] 
(0.9481471124704355, 372.347) [b] 
(0.9481590139735954, 372.352) [b] 
(0.9481777343427611, 381.469) [b] 
(0.9481807643427914, 393.253) [b] 
(0.9481812164577172, 398.891) [b] 
(0.9481816685726431, 398.906) [b] 
(0.9481821206875689, 399.121) [b] 
(0.9481828056190758, 408.31) [b] 
(0.9482006423770666, 409.995) [b] 
(0.9482015466069182, 414.971) [b] 
(0.9482024508367699, 414.972) [b] 
(0.9482202875947607, 416.04) [b] 
(0.9482245737744031, 420.171) [b] 
(0.9482331461336877, 420.266) [b] 
(0.9482353674131002, 424.933) [b] 
(0.9482375886925127, 424.941) [b] 
(0.9482398099719253, 427.626) [b] 
(0.9482431427585646, 461.057) [b] 
(0.9482454258635874, 540.451) [b] 
(0.9482481655896148, 540.654) [b] 
(0.9482486222106193, 544.056) [b] 
(0.9482568413887016, 550.379) [b] 
(0.9482620925302541, 550.497) [b] 
(0.9482630057722632, 553.135) [b] 
(0.9482632340827655, 563.664) [b] 
(0.9483646039457791, 563.925) [b] 
(0.9483657454982906, 564.01) [b] 
(0.9483664304297974, 564.187) [b] 
(0.9483680286033134, 567.463) [b] 
(0.9483858653613042, 576.952) [b] 
(0.948403702119295, 583.525) [b] 
(0.9484039304297973, 584.901) [b] 
(0.9484107797448658, 589.266) [b] 
(0.9484130628498886, 589.33) [b] 
(0.9484172817073376, 589.516) [b] 
(0.9484179666388445, 589.943) [b] 
(0.9486435470872909, 591.278) [b] 
(0.9486456018818115, 604.282) [b] 
(0.9486458301923137, 604.389) [b] 
(0.9486488601923441, 605.296) [b] 
(0.9486490885028463, 608.639) [b] 
(0.9486703213795586, 610.652) [b] 
(0.9486705496900609, 612.116) [b] 
(0.9486707780005632, 614.97) [b] 
(0.9486876482347221, 619.164) [b] 
(0.9487045184688809, 619.191) [b] 
(0.9487213887030398, 620.329) [b] 
(0.9487382589371987, 623.204) [b] 
(0.9487551291713575, 624.627) [b] 
(0.9487719994055164, 624.954) [b] 
(0.9487888696396752, 625.896) [b] 
(0.9487889659779629, 627.419) [b] 
(0.9488058362121218, 629.093) [b] 
(0.9488227064462806, 630.595) [b] 
(0.9488395766804395, 632.168) [b] 
(0.9488434579589783, 635.52) [b] 
(0.9488436862694806, 636.459) [b] 
(0.9488605565036394, 650.694) [b] 
(0.9488774267377983, 651.801) [b] 
(0.9488860152206428, 654.657) [b] 
(0.948890309462065, 654.663) [b] 
(0.9489031921863318, 654.669) [b] 
(0.948907486427754, 654.676) [b] 
(0.948914038556962, 655.348) [b] 
(0.9489142668674643, 658.798) [b] 
(0.9489311371016231, 666.866) [b] 
(0.948948007335782, 666.897) [b] 
(0.9489658440937728, 676.797) [b] 
(0.9490015176097546, 676.798) [b] 
(0.9490193543677454, 680.884) [b] 
(0.9490371911257363, 685.202) [b] 
(0.9490374194362385, 694.989) [b] 
(0.9490404494362689, 715.736) [b] 
(0.9490409015511947, 738.715) [b] 
(0.9490413536661205, 738.718) [b] 
(0.9490532551692804, 771.271) [b] 
(0.9490651566724403, 826.176) [b] 
(0.9490770581756002, 826.501) [b] 
(0.9490988736932809, 834.976) [b] 
(0.9491157439274398, 836.208) [b] 
(0.9491326141615987, 853.043) [b] 
(0.9491342123351146, 854.662) [b] 
(0.9491348972666215, 860.578) [b] 
(0.9491358014964731, 861.282) [b] 
(0.9491394184158796, 861.283) [b] 
(0.9491398705308054, 861.289) [b] 
(0.9491412268755828, 861.317) [b] 
(0.9491416789905086, 861.318) [b] 
(0.9491421311054344, 861.423) [b] 
(0.9491425832203603, 861.528) [b] 
(0.9491555677039024, 875.761) [b] 
(0.9491942158767027, 881.198) [b] 
(0.9492070986009695, 881.205) [b] 
(0.9492113928423918, 881.452) [b] 
(0.9492262395141466, 887.875) [b] 
(0.9492440762721375, 934.178) [b] 
(0.9492454461351512, 963.102) [b] 
(0.9492459027561557, 963.907) [b] 
(0.9492472726191694, 964.143) [b] 
(0.9492479575506763, 967.422) [b] 
(0.9492484141716808, 982.917) [b] 
(0.9492500123451968, 991.784) [b] 
(0.9492592432280385, 1022.08) [b] 
(0.9492777049937218, 1022.09) [b] 
(0.9492869358765634, 1022.13) [b] 
(0.9492961667594051, 1025.94) [b] 
(0.9493027877639713, 1065.73) [b] 
(0.9493084955265283, 1065.95) [b] 
(0.9493126051155694, 1066.61) [b] 
(0.949321835998411, 1088.16) [b] 
(0.9493405617809326, 1154.6) [b] 
(0.949344162892956, 1154.62) [b] 
(0.9493448831153607, 1154.65) [b] 
(0.9493693706771198, 1154.99) [b] 
(0.9493708111219291, 1155.11) [b] 
(0.9493722515667384, 1161.85) [b] 
(0.9493729717891431, 1162.14) [b] 
(0.9493734284101477, 1167.58) [b] 
(0.949374868854957, 1170.84) [b] 
(0.9493791901893851, 1171.46) [b] 
(0.9493799104117898, 1171.51) [b] 
(0.9493806306341945, 1171.57) [b] 
(0.9493813508565992, 1171.62) [b] 
(0.949382071079004, 1172.61) [b] 
(0.9493851010790343, 1174.09) [b] 
(0.9493983430881667, 1175.75) [b] 
(0.9494035942297192, 1175.82) [b] 
(0.9494047357822306, 1175.91) [b] 
(0.9494051924032352, 1175.98) [b] 
(0.949405877334742, 1176.14) [b] 
(0.9494061056452443, 1176.3) [b] 
(0.9494075460900536, 1176.6) [b] 
(0.9494082663124583, 1183.94) [b] 
(0.9494091795544675, 1186.52) [b] 
(0.9494235631161113, 1186.7) [b] 
(0.9494603211069789, 1186.8) [b] 
(0.9494637457645131, 1187.48) [b] 
(0.9494644659869178, 1187.87) [b] 
(0.9494649226079224, 1194.12) [b] 
(0.9494662924709361, 1202.38) [b] 
(0.9494691733605548, 1283.15) [b] 
(0.9494727744725782, 1283.16) [b] 
(0.9494736877145873, 1298.02) [b] 
(0.9494748996877918, 1345.73) [b] 
(0.949475127998294, 1418.89) [b] 
(0.9494769544823123, 1418.94) [b] 
(0.9494774111033168, 1419.57) [b] 
(0.9495304038698043, 1425.41) [b] 
(0.9495833966362917, 1425.82) [b] 
(0.949584837081101, 1512.47) [b] 
(0.9495884900491375, 1519.57) [b] 
(0.9495907731541603, 1601.73) [b] 
(0.9495923038983928, 1726.65) [b] 
(0.949593069270509, 1726.66) [b] 
(0.9495938346426253, 1726.7) [b] 
(0.9495946000147415, 1726.81) [b] 
(0.9495953653868577, 1726.87) [b] 
(0.949596130758974, 1727.77) [b] 
(0.9496022537359041, 1742.05) [b] 
(0.9496037844801367, 1742.06) [b] 
(0.949604549852253, 1742.08) [b] 
(0.9496076113407181, 1742.1) [b] 
(0.9496083767128344, 1742.38) [b] 
(0.949632179719154, 1840.12) [b] 
(0.9496559827254738, 1840.14) [b] 
(0.9496797857317935, 1840.15) [b] 
(0.9496916872349533, 1840.18) [b] 
(0.9497392932475928, 1840.24) [b] 
(0.9497511947507526, 1840.65) [b] 
(0.9497630962539125, 1840.66) [b] 
(0.9497749977570724, 1840.71) [b] 
(0.9497868992602323, 1840.87) [b] 
(0.9497988007633922, 1840.88) [b] 
(0.9498107022665521, 1841.5) [b] 
(0.9498114676386683, 1885.33) [b] 
(0.9498122330107845, 1885.79) [b] 
(0.9498479065267663, 2000.36) [b] 
(0.9498482095200674, 2002.58) [b] 
(0.9498518264394739, 2014.59) [b] 
(0.9498522785543997, 2015.84) [b] 
(0.9498554433588804, 2015.88) [b] 
(0.9498768742570922, 2050.98) [b] 
(0.9498940189756616, 2051) [b] 
(0.949911163694231, 2051.25) [b] 
(0.9499119290663472, 2155.18) [b] 
(0.9499126944384635, 2166.14) [b] 
(0.9499134598105797, 2214.55) [b] 
(0.949914225182696, 2214.85) [b] 
(0.9499149905548122, 2215.28) [b] 
(0.9499157559269285, 2223.13) [b] 
(0.9499422523101722, 2250.87) [b] 
(0.9499687486934159, 2254.19) [b] 
(0.9499717260092465, 2264.44) [b] 
(0.9499747033250772, 2264.68) [b] 
(0.9499751599460817, 2382.42) [b] 
(0.9499763014985931, 2387.8) [b] 
(0.9499767581195977, 2392.11) [b] 
(0.9499772147406023, 2393.49) [b] 
(0.9499895697621463, 2539.95) [b] 
(0.9499988360283044, 2539.97) [b] 
(0.9500019247836904, 2540.01) [b] 
(0.9500020211219781, 2824.02) [b] 
(0.9500789797182533, 2925.31) [b] 
(0.9500853724123172, 2940.16) [b] 
(0.9500881121383445, 2940.17) [b] 
(0.9500890253803537, 2940.22) [b] 
(0.9500903952433674, 2940.36) [b] 
(0.9500908518643719, 2940.53) [b] 
(0.9500922217273856, 2941.97) [b] 
(0.9500940482114039, 2941.99) [b] 
(0.9500954180744176, 2942.05) [b] 
(0.9500963313164267, 2955.85) [b] 
(0.9500967879374312, 2956) [b] 
(0.9500972445584358, 2956.11) [b] 
(0.950098009930552, 2991.97) [b] 
(0.9500984665515566, 3077.74) [b] 
(0.9500989231725612, 3123.55) [b] 
(0.9500993797935657, 3123.71) [b] 
(0.9500998364145703, 3123.86) [b] 
(0.9501002930355749, 3124.17) [b] 
(0.9501007496565794, 3124.7) [b] 
(0.9501008459948671, 3256.95) [b] 
},{(0.9366412054794517, 0.001) [c] 
(0.9366412054794517, 3.5640519178082193) [c] 
(0.9366412054794517, 3600) [c] 
}}}{legend pos=north west}}
	\subfloat[depth=10]{\cactus{Average Accuracy}{CPU time}{\budalg, \murtree, \cart}{{{(0.9139377867048767, 0) [a] 
(0.918538426817828, 0.001) [a] 
(0.9186932213383758, 0.002) [a] 
(0.9240326836595572, 0.003) [a] 
(0.9241710398239407, 0.004) [a] 
(0.9252761696540128, 0.005) [a] 
(0.9297644542139507, 0.006) [a] 
(0.9298302076386084, 0.007) [a] 
(0.9298726733920331, 0.008) [a] 
(0.9336982027274873, 0.009) [a] 
(0.937360942453515, 0.01) [a] 
(0.9373965588918711, 0.011) [a] 
(0.9373979287548848, 0.012) [a] 
(0.937410257522008, 0.013) [a] 
(0.9407229953754571, 0.014) [a] 
(0.943683379566923, 0.015) [a] 
(0.9437039275121285, 0.018) [a] 
(0.943714886416238, 0.019) [a] 
(0.9452148864162379, 0.02) [a] 
(0.9452203658682926, 0.021) [a] 
(0.9452655713477447, 0.022) [a] 
(0.9452874891559638, 0.023) [a] 
(0.9453107768271967, 0.024) [a] 
(0.945313516553224, 0.025) [a] 
(0.9453162562792514, 0.029) [a] 
(0.9470646124436348, 0.03) [a] 
(0.9470659823066485, 0.031) [a] 
(0.9509610507997992, 0.04) [a] 
(0.9509966672381553, 0.047) [a] 
(0.9510322836765115, 0.048) [a] 
(0.9523473065075618, 0.05) [a] 
(0.9523651147267399, 0.052) [a] 
(0.952400731165096, 0.053) [a] 
(0.9532815530829043, 0.058) [a] 
(0.954538722032676, 0.06) [a] 
(0.954542831621717, 0.062) [a] 
(0.9545551603888404, 0.067) [a] 
(0.9545811877861007, 0.068) [a] 
(0.954588722032676, 0.07) [a] 
(0.9545955713477444, 0.073) [a] 
(0.9546092699778814, 0.077) [a] 
(0.9546325576491144, 0.078) [a] 
(0.9546448864162377, 0.079) [a] 
(0.9546788590189775, 0.08) [a] 
(0.9546898179230872, 0.085) [a] 
(0.9546911877861008, 0.087) [a] 
(0.9547899549093884, 0.09) [a] 
(0.9547913247724021, 0.093) [a] 
(0.9548036535395255, 0.095) [a] 
(0.9548050234025391, 0.096) [a] 
(0.9548406398408953, 0.098) [a] 
(0.954842009703909, 0.099) [a] 
(0.9548810507997993, 0.1) [a] 
(0.9548920097039089, 0.106) [a] 
(0.9549043384710322, 0.107) [a] 
(0.9549057083340459, 0.109) [a] 
(0.9549213247724022, 0.11) [a] 
(0.9549692699778817, 0.119) [a] 
(0.9549814617587036, 0.12) [a] 
(0.9549828316217173, 0.121) [a] 
(0.954984201484731, 0.122) [a] 
(0.9549965302518543, 0.123) [a] 
(0.954997900114868, 0.128) [a] 
(0.9550013247724022, 0.13) [a] 
(0.9550054343614433, 0.131) [a] 
(0.9550068042244569, 0.134) [a] 
(0.9550081740874706, 0.137) [a] 
(0.9550232425806213, 0.138) [a] 
(0.9550570781970597, 0.14) [a] 
(0.9550584480600733, 0.143) [a] 
(0.955059817923087, 0.146) [a] 
(0.9550776261422651, 0.147) [a] 
(0.9550981740874706, 0.149) [a] 
(0.9551003658682925, 0.15) [a] 
(0.9551044754573336, 0.151) [a] 
(0.955107215183361, 0.152) [a] 
(0.9551085850463746, 0.153) [a] 
(0.9551099549093883, 0.156) [a] 
(0.9551455713477446, 0.157) [a] 
(0.9551484480600734, 0.16) [a] 
(0.9551498179230871, 0.162) [a] 
(0.9551607768271967, 0.163) [a] 
(0.9551621466902104, 0.164) [a] 
(0.9551635165532241, 0.167) [a] 
(0.9551658453203474, 0.17) [a] 
(0.9551685850463748, 0.173) [a] 
(0.9552288590189777, 0.174) [a] 
(0.9552466672381558, 0.175) [a] 
(0.9552480371011695, 0.176) [a] 
(0.9552494069641831, 0.177) [a] 
(0.9552503658682927, 0.18) [a] 
(0.9552517357313064, 0.181) [a] 
(0.9552531055943201, 0.187) [a] 
(0.9552539275121282, 0.19) [a] 
(0.9552552973751419, 0.192) [a] 
(0.9552566672381556, 0.198) [a] 
(0.9552676261422652, 0.199) [a] 
(0.9552692699778816, 0.2) [a] 
(0.9552706398408953, 0.204) [a] 
(0.955272009703909, 0.207) [a] 
(0.9552761192929501, 0.209) [a] 
(0.9552774891559638, 0.21) [a] 
(0.9552802288819912, 0.213) [a] 
(0.9553144754573337, 0.22) [a] 
(0.9553158453203474, 0.222) [a] 
(0.9553172151833611, 0.225) [a] 
(0.9553185850463748, 0.229) [a] 
(0.9553524206628132, 0.23) [a] 
(0.9553537905258269, 0.236) [a] 
(0.9553626946354159, 0.24) [a] 
(0.9553640644984296, 0.249) [a] 
(0.9553715987450049, 0.25) [a] 
(0.9553784480600733, 0.256) [a] 
(0.955379817923087, 0.257) [a] 
(0.9553963932655527, 0.26) [a] 
(0.9553977631285664, 0.268) [a] 
(0.9554500918956896, 0.27) [a] 
(0.9554514617587033, 0.274) [a] 
(0.955502831621717, 0.28) [a] 
(0.9555206398408951, 0.3) [a] 
(0.9555233795669225, 0.303) [a] 
(0.9555247494299361, 0.306) [a] 
(0.957987150691689, 0.31) [a] 
(0.9579885205547027, 0.312) [a] 
(0.9579918082259355, 0.32) [a] 
(0.9579931780889492, 0.329) [a] 
(0.9580086575410041, 0.33) [a] 
(0.9580233150752506, 0.35) [a] 
(0.9580246849382643, 0.352) [a] 
(0.9580424931574424, 0.358) [a] 
(0.9580438630204561, 0.359) [a] 
(0.9580459178149766, 0.36) [a] 
(0.9580472876779903, 0.365) [a] 
(0.9580479726094971, 0.37) [a] 
(0.9580493424725108, 0.372) [a] 
(0.9580507123355245, 0.375) [a] 
(0.9580696164451136, 0.38) [a] 
(0.9580709863081273, 0.381) [a] 
(0.958072356171141, 0.384) [a] 
(0.9581304383629218, 0.39) [a] 
(0.9581318082259355, 0.394) [a] 
(0.9581441369930588, 0.397) [a] 
(0.9581763287738806, 0.4) [a] 
(0.9582009863081272, 0.402) [a] 
(0.9582242739793602, 0.403) [a] 
(0.9582256438423739, 0.404) [a] 
(0.9582283835684012, 0.406) [a] 
(0.9582297534314149, 0.408) [a] 
(0.958277561650593, 0.41) [a] 
(0.9582789315136067, 0.411) [a] 
(0.958281671239634, 0.412) [a] 
(0.9582830411026477, 0.418) [a] 
(0.9583798904177163, 0.42) [a] 
(0.95838126028073, 0.424) [a] 
(0.9584356438423738, 0.43) [a] 
(0.9584370137053875, 0.438) [a] 
(0.9584856438423739, 0.44) [a] 
(0.9585361917875794, 0.45) [a] 
(0.9612960818757843, 0.465) [a] 
(0.9613052599579761, 0.47) [a] 
(0.9613312873552364, 0.476) [a] 
(0.9613323832456474, 0.48) [a] 
(0.9613337531086611, 0.487) [a] 
(0.9642625369235205, 0.49) [a] 
(0.9642803451426986, 0.491) [a] 
(0.9642811670605068, 0.5) [a] 
(0.9643300711700957, 0.51) [a] 
(0.9644039067865341, 0.52) [a] 
(0.9644340437728355, 0.525) [a] 
(0.9644477424029725, 0.536) [a] 
(0.9644759615810546, 0.55) [a] 
(0.9644766465125615, 0.57) [a] 
(0.9644780163755752, 0.575) [a] 
(0.9645137698002327, 0.58) [a] 
(0.9645334958276299, 0.59) [a] 
(0.964551304046808, 0.596) [a] 
(0.9645526739098217, 0.598) [a] 
(0.9645885643207807, 0.6) [a] 
(0.9646063725399587, 0.601) [a] 
(0.964607194457767, 0.61) [a] 
(0.9646085643207807, 0.614) [a] 
(0.9646099341837944, 0.62) [a] 
(0.9646103451426985, 0.63) [a] 
(0.9646108930879039, 0.64) [a] 
(0.9646117150057121, 0.65) [a] 
(0.9646888382933834, 0.66) [a] 
(0.9647070574714656, 0.67) [a] 
(0.9647248656906436, 0.678) [a] 
(0.9647262355536573, 0.705) [a] 
(0.964727605416671, 0.71) [a] 
(0.9647280163755751, 0.72) [a] 
(0.9648073314440683, 0.73) [a] 
(0.964807605416671, 0.81) [a] 
(0.9648080163755751, 0.82) [a] 
(0.9648085643207805, 0.84) [a] 
(0.9648177424029724, 0.85) [a] 
(0.9648178793892738, 0.88) [a] 
(0.9648534958276299, 0.885) [a] 
(0.9648536328139313, 0.9) [a] 
(0.9648814410331095, 0.91) [a] 
(0.9648817150057122, 0.93) [a] 
(0.964882399937219, 0.95) [a] 
(0.9648947287043423, 0.959) [a] 
(0.9649080163755752, 0.96) [a] 
(0.9649082903481779, 0.98) [a] 
(0.9649093862385889, 0.99) [a] 
(0.9649103451426985, 1) [a] 
(0.9649156876084518, 1.01) [a] 
(0.9649170574714655, 1.063) [a] 
(0.9649184273344792, 1.08) [a] 
(0.9655319178082182, 1.13) [a] 
(0.9657178082191771, 1.14) [a] 
(0.9657180821917798, 1.16) [a] 
(0.9657379452054784, 1.17) [a] 
(0.9657434246575332, 1.205) [a] 
(0.9657557534246564, 1.206) [a] 
(0.9657626027397248, 1.207) [a] 
(0.9657680821917796, 1.22) [a] 
(0.9657772602739714, 1.24) [a] 
(0.9658142465753412, 1.28) [a] 
(0.965823424657533, 1.29) [a] 
(0.9658247945205467, 1.344) [a] 
(0.9658275342465741, 1.376) [a] 
(0.9658357534246562, 1.377) [a] 
(0.9658426027397247, 1.378) [a] 
(0.965845342465752, 1.384) [a] 
(0.9658547945205466, 1.39) [a] 
(0.9658731506849302, 1.4) [a] 
(0.9658854794520535, 1.401) [a] 
(0.9658991780821905, 1.402) [a] 
(0.9659115068493138, 1.404) [a] 
(0.9659208219178069, 1.41) [a] 
(0.9659299999999987, 1.42) [a] 
(0.9659342465753412, 1.46) [a] 
(0.9659393150684918, 1.47) [a] 
(0.9659484931506837, 1.48) [a] 
(0.9659498630136973, 1.497) [a] 
(0.965951232876711, 1.506) [a] 
(0.9659526027397247, 1.519) [a] 
(0.9659568493150672, 1.52) [a] 
(0.965957534246574, 1.72) [a] 
(0.9659589041095877, 1.738) [a] 
(0.9659602739726014, 1.741) [a] 
(0.9659616438356151, 1.747) [a] 
(0.9659630136986288, 1.762) [a] 
(0.9659635616438342, 1.77) [a] 
(0.9659639726027384, 1.78) [a] 
(0.965965342465752, 1.797) [a] 
(0.9659673972602726, 1.81) [a] 
(0.9659687671232863, 1.816) [a] 
(0.9659730136986288, 1.82) [a] 
(0.9659743835616424, 1.879) [a] 
(0.9659757534246561, 1.892) [a] 
(0.9659894520547931, 1.907) [a] 
(0.9660017808219165, 1.91) [a] 
(0.9660031506849301, 1.911) [a] 
(0.9660045205479438, 1.913) [a] 
(0.9660058904109575, 1.916) [a] 
(0.9660072602739712, 1.92) [a] 
(0.9660086301369849, 1.93) [a] 
(0.9660117808219164, 1.94) [a] 
(0.9660131506849301, 1.948) [a] 
(0.9660135616438342, 1.95) [a] 
(0.9660149315068479, 1.954) [a] 
(0.9660163013698616, 1.956) [a] 
(0.9660176712328753, 1.961) [a] 
(0.966019041095889, 1.966) [a] 
(0.9660217808219164, 1.967) [a] 
(0.96602315068493, 1.968) [a] 
(0.9660258904109574, 1.974) [a] 
(0.9660275342465738, 1.98) [a] 
(0.9660289041095875, 1.982) [a] 
(0.9660302739726012, 1.987) [a] 
(0.9660330136986286, 1.992) [a] 
(0.9660343835616423, 1.995) [a] 
(0.9660526027397245, 2) [a] 
(0.9660539726027382, 2.006) [a] 
(0.9660567123287656, 2.013) [a] 
(0.9660580821917792, 2.056) [a] 
(0.9660594520547929, 2.091) [a] 
(0.9660608219178066, 2.122) [a] 
(0.9660621917808203, 2.13) [a] 
(0.966063561643834, 2.132) [a] 
(0.9660643835616423, 2.23) [a] 
(0.9660812328767108, 2.28) [a] 
(0.9660980821917793, 2.51) [a] 
(0.966099452054793, 2.565) [a] 
(0.9661104109589026, 2.572) [a] 
(0.9661117808219163, 2.583) [a] 
(0.9661145205479437, 2.591) [a] 
(0.9661199999999984, 2.592) [a] 
(0.9661241095890395, 2.593) [a] 
(0.9661843835616423, 2.668) [a] 
(0.9661953424657519, 2.671) [a] 
(0.9662076712328752, 2.673) [a] 
(0.9662199999999985, 2.674) [a] 
(0.966236849315067, 2.69) [a] 
(0.9662491780821904, 2.702) [a] 
(0.9662601369863, 2.704) [a] 
(0.9662615068493137, 2.714) [a] 
(0.9662628767123274, 2.717) [a] 
(0.966264246575341, 2.721) [a] 
(0.9662656164383547, 2.725) [a] 
(0.9662669863013684, 2.729) [a] 
(0.9662697260273958, 2.732) [a] 
(0.9662699999999985, 2.82) [a] 
(0.9662878082191766, 2.88) [a] 
(0.9662880821917793, 2.89) [a] 
(0.9663154794520533, 2.93) [a] 
(0.9663428767123273, 2.94) [a] 
(0.9663565753424643, 2.96) [a] 
(0.9663567123287657, 2.98) [a] 
(0.9663842465753412, 2.99) [a] 
(0.9663979452054782, 3) [a] 
(0.9664116438356152, 3.02) [a] 
(0.9664119178082179, 3.11) [a] 
(0.9664132876712316, 3.147) [a] 
(0.9664242465753411, 3.148) [a] 
(0.9664352054794506, 3.149) [a] 
(0.9664406849315054, 3.15) [a] 
(0.9664447945205464, 3.162) [a] 
(0.9664502739726012, 3.17) [a] 
(0.966455753424656, 3.179) [a] 
(0.9664776712328751, 3.18) [a] 
(0.9664872602739709, 3.181) [a] 
(0.9664927397260257, 3.188) [a] 
(0.9664982191780804, 3.198) [a] 
(0.9665036986301352, 3.228) [a] 
(0.9665050684931489, 3.254) [a] 
(0.9665064383561626, 3.257) [a] 
(0.9665219178082174, 3.4) [a] 
(0.9665287671232858, 3.401) [a] 
(0.9665315068493132, 3.409) [a] 
(0.9665342465753406, 3.41) [a] 
(0.966536986301368, 3.411) [a] 
(0.9665465753424639, 3.412) [a] 
(0.9665493150684913, 3.413) [a] 
(0.966554794520546, 3.415) [a] 
(0.9665553424657515, 3.57) [a] 
(0.9665560273972583, 3.59) [a] 
(0.9665568493150666, 3.6) [a] 
(0.9665705479452036, 3.75) [a] 
(0.9665715068493131, 3.8) [a] 
(0.9665728767123268, 3.838) [a] 
(0.9665742465753405, 3.85) [a] 
(0.9665756164383542, 3.864) [a] 
(0.9665769863013679, 3.904) [a] 
(0.9665783561643816, 3.914) [a] 
(0.9665797260273953, 3.917) [a] 
(0.966581095890409, 3.919) [a] 
(0.9665824657534227, 3.948) [a] 
(0.9665827397260254, 4.01) [a] 
(0.9665828767123268, 4.02) [a] 
(0.9665842465753405, 4.054) [a] 
(0.9665856164383542, 4.096) [a] 
(0.9665857534246556, 4.24) [a] 
(0.9665871232876693, 4.272) [a] 
(0.966588493150683, 4.343) [a] 
(0.9665898630136966, 4.512) [a] 
(0.9665904109589021, 4.53) [a] 
(0.9665917808219158, 4.544) [a] 
(0.9665931506849295, 4.574) [a] 
(0.9665945205479431, 4.624) [a] 
(0.9666123287671212, 4.645) [a] 
(0.9666127397260253, 4.67) [a] 
(0.9666305479452034, 4.686) [a] 
(0.9666319178082171, 4.849) [a] 
(0.9666320547945185, 4.85) [a] 
(0.9666443835616418, 4.874) [a] 
(0.9666487671232856, 5.16) [a] 
(0.9666542465753404, 5.467) [a] 
(0.9666569863013678, 5.489) [a] 
(0.9666583561643814, 5.498) [a] 
(0.9666597260273951, 5.518) [a] 
(0.9666610958904088, 5.532) [a] 
(0.9666624657534225, 5.544) [a] 
(0.9666638356164362, 5.556) [a] 
(0.9666652054794499, 5.57) [a] 
(0.9666665753424636, 5.577) [a] 
(0.9666679452054773, 5.582) [a] 
(0.966669315068491, 5.607) [a] 
(0.966673424657532, 5.615) [a] 
(0.9666747945205457, 5.657) [a] 
(0.9666761643835594, 5.66) [a] 
(0.9666775342465731, 5.665) [a] 
(0.9666789041095868, 5.685) [a] 
(0.9666802739726005, 5.707) [a] 
(0.9666816438356142, 5.716) [a] 
(0.9666830136986279, 5.737) [a] 
(0.9666843835616415, 5.746) [a] 
(0.9666857534246552, 5.793) [a] 
(0.9666871232876689, 5.801) [a] 
(0.9666884931506826, 5.803) [a] 
(0.9666898630136963, 5.843) [a] 
(0.96669123287671, 5.855) [a] 
(0.9667035616438333, 5.865) [a] 
(0.9667268493150663, 5.866) [a] 
(0.9667282191780799, 5.877) [a] 
(0.9667295890410936, 5.888) [a] 
(0.9667309589041073, 5.91) [a] 
(0.966732328767121, 5.92) [a] 
(0.9667336986301347, 5.934) [a] 
(0.9667350684931484, 5.939) [a] 
(0.9667364383561621, 5.979) [a] 
(0.9667378082191758, 6.026) [a] 
(0.9667391780821895, 6.029) [a] 
(0.9667405479452031, 6.047) [a] 
(0.9667460273972579, 6.102) [a] 
(0.9667693150684908, 6.181) [a] 
(0.9667939726027375, 6.182) [a] 
(0.9668049315068471, 6.187) [a] 
(0.9668172602739704, 6.222) [a] 
(0.9668227397260252, 6.243) [a] 
(0.9668460273972581, 6.246) [a] 
(0.9668630136986279, 6.25) [a] 
(0.9668672602739704, 6.35) [a] 
(0.9668795890410937, 6.453) [a] 
(0.9668809589041074, 6.625) [a] 
(0.9668979452054772, 6.8) [a] 
(0.9669020547945183, 6.809) [a] 
(0.9669047945205457, 6.814) [a] 
(0.966907534246573, 6.86) [a] 
(0.9669102739726004, 6.914) [a] 
(0.9669143835616415, 6.916) [a] 
(0.9669171232876689, 6.927) [a] 
(0.9669184931506826, 7.024) [a] 
(0.9669198630136963, 7.032) [a] 
(0.96692123287671, 7.035) [a] 
(0.9669226027397236, 7.049) [a] 
(0.9669394520547921, 7.19) [a] 
(0.9669408219178058, 7.294) [a] 
(0.9669531506849292, 7.328) [a] 
(0.9669654794520525, 7.343) [a] 
(0.9669668493150662, 7.385) [a] 
(0.9669682191780798, 7.417) [a] 
(0.9669691780821894, 7.42) [a] 
(0.9669706849315045, 7.44) [a] 
(0.9669734246575319, 7.543) [a] 
(0.9669747945205456, 7.648) [a] 
(0.9669790410958881, 7.7) [a] 
(0.9669791780821895, 7.76) [a] 
(0.9669795890410936, 7.8) [a] 
(0.9669802739726004, 7.85) [a] 
(0.966997123287669, 7.86) [a] 
(0.9669975342465731, 7.87) [a] 
(0.9669978082191758, 7.89) [a] 
(0.9669980821917785, 7.91) [a] 
(0.9669982191780799, 7.92) [a] 
(0.9669995890410936, 7.99) [a] 
(0.9670009589041073, 8.02) [a] 
(0.967002328767121, 8.028) [a] 
(0.9670036986301347, 8.058) [a] 
(0.9670050684931484, 8.083) [a] 
(0.9670064383561621, 8.09) [a] 
(0.9670065753424635, 8.15) [a] 
(0.9670079452054772, 8.164) [a] 
(0.9670093150684909, 8.179) [a] 
(0.9670094520547923, 8.18) [a] 
(0.967010821917806, 8.204) [a] 
(0.9670109589041074, 8.25) [a] 
(0.9670123287671211, 8.295) [a] 
(0.9670136986301348, 8.39) [a] 
(0.9670139726027375, 8.43) [a] 
(0.9670142465753402, 8.45) [a] 
(0.9670156164383539, 8.664) [a] 
(0.9670183561643813, 8.893) [a] 
(0.9670594520547923, 9.04) [a] 
(0.9670868493150663, 9.25) [a] 
(0.9671005479452033, 9.28) [a] 
(0.967100821917806, 9.65) [a] 
(0.9671035616438334, 9.733) [a] 
(0.9671039726027375, 9.97) [a] 
(0.9671041095890389, 9.98) [a] 
(0.9671042465753403, 10.05) [a] 
(0.967105616438354, 10.23) [a] 
(0.9671069863013677, 10.29) [a] 
(0.9671124657534225, 10.44) [a] 
(0.9671217808219156, 10.51) [a] 
(0.9671221917808197, 10.56) [a] 
(0.9671238356164361, 10.59) [a] 
(0.9671252054794498, 10.76) [a] 
(0.9671254794520525, 11.34) [a] 
(0.9671256164383539, 11.63) [a] 
(0.9671258904109566, 11.64) [a] 
(0.9671299999999977, 12.1) [a] 
(0.9671391780821895, 12.25) [a] 
(0.9671405479452032, 12.5) [a] 
(0.9671427397260252, 12.55) [a] 
(0.9671430136986279, 12.56) [a] 
(0.967143424657532, 12.58) [a] 
(0.9671436986301347, 12.59) [a] 
(0.9671439726027374, 12.67) [a] 
(0.9671442465753401, 12.74) [a] 
(0.9671443835616416, 12.84) [a] 
(0.9671457534246553, 12.96) [a] 
(0.9671467123287648, 13.64) [a] 
(0.9671473972602717, 13.68) [a] 
(0.9671475342465731, 13.69) [a] 
(0.9671478082191758, 13.7) [a] 
(0.9671480821917785, 13.71) [a] 
(0.9671482191780799, 13.73) [a] 
(0.9671489041095868, 13.75) [a] 
(0.9671495890410936, 13.76) [a] 
(0.967150136986299, 13.79) [a] 
(0.9671508219178059, 13.84) [a] 
(0.9671513698630113, 14.01) [a] 
(0.9671563013698606, 14.02) [a] 
(0.9672332876712305, 14.57) [a] 
(0.9672339726027374, 14.96) [a] 
(0.9672494520547922, 14.98) [a] 
(0.9672504109589017, 14.99) [a] 
(0.9672505479452032, 15.02) [a] 
(0.9672508219178059, 15.03) [a] 
(0.9672510958904086, 15.05) [a] 
(0.96725123287671, 15.07) [a] 
(0.9672515068493127, 15.27) [a] 
(0.9672516438356141, 15.6) [a] 
(0.9672521917808196, 15.61) [a] 
(0.967252328767121, 15.63) [a] 
(0.9672526027397237, 15.64) [a] 
(0.9672528767123264, 15.66) [a] 
(0.9672530136986278, 15.69) [a] 
(0.9672532876712305, 15.78) [a] 
(0.9672535616438332, 15.83) [a] 
(0.967253835616436, 15.84) [a] 
(0.9672542465753401, 15.9) [a] 
(0.9672545205479428, 15.93) [a] 
(0.9672546575342442, 16.32) [a] 
(0.9672561643835593, 16.42) [a] 
(0.967256438356162, 17.08) [a] 
(0.9672657534246552, 17.27) [a] 
(0.9672663013698606, 17.49) [a] 
(0.9672799999999976, 17.54) [a] 
(0.9672936986301346, 17.95) [a] 
(0.9673073972602716, 18.01) [a] 
(0.9673210958904086, 18.38) [a] 
(0.9673347945205456, 18.49) [a] 
(0.9673350684931483, 18.92) [a] 
(0.9673354794520524, 18.93) [a] 
(0.9673509589041073, 18.94) [a] 
(0.9673601369862991, 19.11) [a] 
(0.9673602739726005, 19.12) [a] 
(0.9673694520547923, 19.14) [a] 
(0.9673913698630114, 19.31) [a] 
(0.9673927397260251, 19.39) [a] 
(0.9673941095890388, 19.5) [a] 
(0.9674119178082168, 19.79) [a] 
(0.9674187671232852, 19.8) [a] 
(0.9674269863013674, 19.84) [a] 
(0.9674297260273947, 19.85) [a] 
(0.9674390410958879, 19.98) [a] 
(0.9674404109589015, 20.04) [a] 
(0.9674431506849289, 20.07) [a] 
(0.9674445205479426, 20.08) [a] 
(0.967444657534244, 20.12) [a] 
(0.9674460273972577, 20.14) [a] 
(0.9674528767123262, 20.49) [a] 
(0.9674556164383535, 20.5) [a] 
(0.9674583561643809, 20.51) [a] 
(0.9674675342465727, 20.57) [a] 
(0.9674730136986275, 20.94) [a] 
(0.9674883561643809, 21.17) [a] 
(0.9675020547945179, 21.72) [a] 
(0.9675157534246549, 21.74) [a] 
(0.9675294520547919, 21.75) [a] 
(0.967529863013696, 21.87) [a] 
(0.9675302739726002, 21.96) [a] 
(0.9675313698630111, 22.08) [a] 
(0.9675316438356139, 22.09) [a] 
(0.9675326027397234, 22.14) [a] 
(0.9675327397260248, 22.22) [a] 
(0.9675330136986275, 22.26) [a] 
(0.967549863013696, 22.62) [a] 
(0.9675567123287645, 23.36) [a] 
(0.9675569863013672, 23.51) [a] 
(0.9675573972602713, 23.52) [a] 
(0.9675742465753399, 25.06) [a] 
(0.9675747945205453, 26.38) [a] 
(0.9675790410958878, 26.52) [a] 
(0.9675794520547919, 26.68) [a] 
(0.967579863013696, 26.7) [a] 
(0.9675804109589015, 26.83) [a] 
(0.9675941095890385, 26.94) [a] 
(0.9675949315068467, 27.04) [a] 
(0.9676086301369837, 27.29) [a] 
(0.9676087671232851, 27.34) [a] 
(0.9676224657534221, 28.85) [a] 
(0.9676267123287646, 29.25) [a] 
(0.9676310958904084, 30.15) [a] 
(0.9676313698630111, 30.6) [a] 
(0.9676316438356138, 31.29) [a] 
(0.9676317808219153, 31.3) [a] 
(0.9676327397260248, 31.31) [a] 
(0.9676338356164358, 31.32) [a] 
(0.9676341095890385, 31.33) [a] 
(0.9676343835616412, 31.81) [a] 
(0.9676357534246549, 32.11) [a] 
(0.9676364383561618, 32.47) [a] 
(0.9676367123287645, 33.02) [a] 
(0.9676373972602713, 33.04) [a] 
(0.9676378082191754, 33.05) [a] 
(0.9676383561643809, 33.11) [a] 
(0.9676384931506823, 34.13) [a] 
(0.967638767123285, 34.14) [a] 
(0.9676394520547918, 34.16) [a] 
(0.9676397260273946, 34.17) [a] 
(0.9676402739726, 34.68) [a] 
(0.9676406849315041, 34.69) [a] 
(0.9676512328767096, 34.82) [a] 
(0.9676520547945179, 34.85) [a] 
(0.967653972602737, 34.86) [a] 
(0.9676547945205453, 35.15) [a] 
(0.9676571232876685, 35.2) [a] 
(0.9676620547945178, 35.41) [a] 
(0.9676630136986274, 35.43) [a] 
(0.9676634246575315, 35.44) [a] 
(0.9676867123287645, 35.63) [a] 
(0.9676990410958878, 35.66) [a] 
(0.9676997260273946, 36.31) [a] 
(0.9677001369862988, 36.32) [a] 
(0.9677771232876686, 36.51) [a] 
(0.9677773972602713, 36.85) [a] 
(0.9677783561643809, 36.92) [a] 
(0.9677938356164357, 38.44) [a] 
(0.9678093150684905, 38.47) [a] 
(0.9678120547945179, 38.64) [a] 
(0.9678257534246549, 38.72) [a] 
(0.9678394520547919, 38.77) [a] 
(0.9678408219178056, 38.93) [a] 
(0.9678613698630111, 39.36) [a] 
(0.9678639726027372, 39.37) [a] 
(0.9678667123287645, 39.67) [a] 
(0.967873561643833, 39.68) [a] 
(0.9678763013698604, 40.91) [a] 
(0.9678764383561618, 41.26) [a] 
(0.9678767123287645, 41.36) [a] 
(0.9678769863013672, 41.37) [a] 
(0.9678771232876686, 41.4) [a] 
(0.9678773972602713, 41.42) [a] 
(0.9678775342465727, 41.8) [a] 
(0.9678778082191755, 41.81) [a] 
(0.9678780821917782, 41.82) [a] 
(0.9678782191780796, 41.84) [a] 
(0.9678784931506823, 41.86) [a] 
(0.9678939726027371, 42.01) [a] 
(0.9678942465753398, 42.07) [a] 
(0.9678943835616413, 42.13) [a] 
(0.967894657534244, 42.24) [a] 
(0.9678949315068467, 42.29) [a] 
(0.9678952054794494, 43.36) [a] 
(0.9679089041095864, 44.78) [a] 
(0.9679102739726001, 44.84) [a] 
(0.9679116438356138, 44.85) [a] 
(0.9679130136986275, 44.87) [a] 
(0.9679143835616412, 44.95) [a] 
(0.9679157534246549, 45.35) [a] 
(0.9679171232876685, 46.2) [a] 
(0.9679184931506822, 46.44) [a] 
(0.9679198630136959, 46.76) [a] 
(0.9679202739726, 46.77) [a] 
(0.9679220547945179, 46.78) [a] 
(0.9679230136986274, 46.79) [a] 
(0.967923972602737, 46.8) [a] 
(0.9679243835616411, 46.82) [a] 
(0.9679247945205453, 46.83) [a] 
(0.9679253424657507, 46.88) [a] 
(0.9679261643835589, 46.9) [a] 
(0.9679275342465726, 46.93) [a] 
(0.9679298630136959, 46.95) [a] 
(0.9679302739726, 46.96) [a] 
(0.9679316438356137, 46.98) [a] 
(0.9679317808219151, 47.03) [a] 
(0.9679323287671205, 47.04) [a] 
(0.9679327397260247, 47.07) [a] 
(0.9679336986301342, 47.1) [a] 
(0.9679387671232849, 47.11) [a] 
(0.9679419178082164, 47.12) [a] 
(0.9679436986301342, 47.13) [a] 
(0.9679442465753396, 47.14) [a] 
(0.9679450684931479, 47.15) [a] 
(0.967946986301367, 47.16) [a] 
(0.9679491780821889, 47.17) [a] 
(0.9679510958904081, 47.18) [a] 
(0.9679524657534218, 47.19) [a] 
(0.96795328767123, 47.23) [a] 
(0.9679546575342437, 47.24) [a] 
(0.9679573972602712, 47.25) [a] 
(0.967960684931504, 47.26) [a] 
(0.9679610958904081, 47.29) [a] 
(0.9679615068493123, 47.3) [a] 
(0.9679620547945177, 47.44) [a] 
(0.9679631506849287, 47.46) [a] 
(0.9679636986301341, 47.48) [a] 
(0.9679650684931478, 47.49) [a] 
(0.9679664383561615, 47.5) [a] 
(0.9679682191780793, 47.51) [a] 
(0.9679699999999971, 47.52) [a] 
(0.9679719178082162, 47.53) [a] 
(0.9679723287671204, 47.54) [a] 
(0.9679727397260245, 47.55) [a] 
(0.9679732876712299, 47.58) [a] 
(0.9679750684931477, 47.7) [a] 
(0.9679779452054765, 47.71) [a] 
(0.9679786301369834, 47.72) [a] 
(0.9679791780821888, 47.78) [a] 
(0.9679795890410929, 47.79) [a] 
(0.9679797260273943, 47.95) [a] 
(0.967981095890408, 48.03) [a] 
(0.9679824657534217, 48.14) [a] 
(0.9679838356164354, 48.15) [a] 
(0.9679852054794491, 48.16) [a] 
(0.9679871232876682, 48.17) [a] 
(0.9679875342465724, 48.18) [a] 
(0.9679884931506819, 48.2) [a] 
(0.967988904109586, 48.21) [a] 
(0.9679912328767093, 48.22) [a] 
(0.967992602739723, 48.23) [a] 
(0.9679930136986271, 48.24) [a] 
(0.9679934246575312, 48.25) [a] 
(0.9679943835616408, 48.26) [a] 
(0.9679947945205449, 48.27) [a] 
(0.9679953424657504, 48.5) [a] 
(0.967996712328764, 49.16) [a] 
(0.9679971232876682, 49.17) [a] 
(0.9679975342465723, 49.18) [a] 
(0.9679984931506819, 49.19) [a] 
(0.9679998630136956, 49.24) [a] 
(0.9680002739725997, 49.38) [a] 
(0.9680016438356134, 49.79) [a] 
(0.9680030136986271, 49.8) [a] 
(0.9680049315068462, 49.81) [a] 
(0.968006712328764, 49.82) [a] 
(0.9680084931506818, 49.83) [a] 
(0.9680090410958873, 49.85) [a] 
(0.9680094520547914, 49.91) [a] 
(0.9680098630136955, 49.94) [a] 
(0.9680108219178051, 50.2) [a] 
(0.9680112328767092, 50.21) [a] 
(0.9680126027397229, 50.37) [a] 
(0.9680139726027366, 50.44) [a] 
(0.9680153424657503, 50.49) [a] 
(0.968016712328764, 50.51) [a] 
(0.9680172602739694, 50.55) [a] 
(0.9680180821917777, 50.56) [a] 
(0.9680194520547913, 50.6) [a] 
(0.9680199999999968, 50.61) [a] 
(0.9680204109589009, 50.75) [a] 
(0.9680213698630105, 50.92) [a] 
(0.9680249315068461, 50.93) [a] 
(0.9680258904109557, 50.94) [a] 
(0.968028219178079, 50.95) [a] 
(0.9680295890410927, 50.96) [a] 
(0.9680299999999968, 50.97) [a] 
(0.9680313698630105, 50.98) [a] 
(0.9680327397260242, 50.99) [a] 
(0.9680331506849283, 51) [a] 
(0.9680336986301338, 51.11) [a] 
(0.968034520547942, 51.22) [a] 
(0.9680350684931475, 52.17) [a] 
(0.9680354794520516, 52.18) [a] 
(0.9680358904109557, 52.23) [a] 
(0.9680363013698599, 52.45) [a] 
(0.9680365753424626, 52.46) [a] 
(0.968036712328764, 52.47) [a] 
(0.9680383561643804, 52.5) [a] 
(0.9680387671232845, 52.56) [a] 
(0.9680390410958872, 52.57) [a] 
(0.96803931506849, 52.58) [a] 
(0.9680394520547914, 52.61) [a] 
(0.9680399999999968, 52.62) [a] 
(0.9680401369862982, 52.63) [a] 
(0.9680404109589009, 52.64) [a] 
(0.9680406849315036, 52.7) [a] 
(0.9680410958904078, 52.73) [a] 
(0.9680420547945173, 52.86) [a] 
(0.9680443835616407, 52.88) [a] 
(0.9680447945205448, 52.89) [a] 
(0.9680461643835585, 52.9) [a] 
(0.9680463013698599, 53.03) [a] 
(0.9680476712328736, 53.34) [a] 
(0.9680480821917777, 53.41) [a] 
(0.9680494520547914, 53.47) [a] 
(0.9680498630136956, 53.62) [a] 
(0.9680501369862983, 53.65) [a] 
(0.968050410958901, 53.72) [a] 
(0.9680509589041064, 53.86) [a] 
(0.9680512328767091, 53.87) [a] 
(0.9680513698630105, 54.22) [a] 
(0.9680516438356133, 54.23) [a] 
(0.9680520547945174, 54.24) [a] 
(0.9680523287671201, 54.61) [a] 
(0.9680527397260242, 56.38) [a] 
(0.968053013698627, 56.41) [a] 
(0.9680536986301338, 56.48) [a] 
(0.9680539726027365, 56.5) [a] 
(0.9680541095890379, 56.52) [a] 
(0.9680543835616406, 56.55) [a] 
(0.9680546575342434, 56.59) [a] 
(0.9680547945205448, 56.6) [a] 
(0.9680552054794489, 56.61) [a] 
(0.9680563013698598, 56.63) [a] 
(0.9680564383561612, 56.66) [a] 
(0.9680573972602707, 56.72) [a] 
(0.9680576712328735, 56.75) [a] 
(0.9680578082191749, 56.76) [a] 
(0.9680580821917776, 56.79) [a] 
(0.9680583561643803, 56.81) [a] 
(0.9680584931506817, 56.87) [a] 
(0.9680587671232844, 56.88) [a] 
(0.9680590410958871, 56.9) [a] 
(0.9680591780821886, 56.92) [a] 
(0.9680594520547913, 56.94) [a] 
(0.968059726027394, 56.96) [a] 
(0.9680598630136954, 57.06) [a] 
(0.9680601369862981, 57.07) [a] 
(0.9680604109589008, 57.11) [a] 
(0.968060821917805, 57.47) [a] 
(0.9680613698630104, 57.59) [a] 
(0.9680621917808186, 57.69) [a] 
(0.9680626027397228, 57.83) [a] 
(0.9680653424657502, 58.85) [a] 
(0.9680654794520516, 58.96) [a] 
(0.9680747945205447, 59.49) [a] 
(0.9680901369862981, 59.52) [a] 
(0.9680993150684899, 60.17) [a] 
(0.9681763013698598, 60.25) [a] 
(0.9681776712328735, 60.33) [a] 
(0.9681790410958871, 60.34) [a] 
(0.9681798630136954, 61.12) [a] 
(0.9681804109589008, 61.17) [a] 
(0.9681812328767091, 61.26) [a] 
(0.9681824657534214, 61.38) [a] 
(0.9681838356164351, 61.92) [a] 
(0.9681879452054761, 62.78) [a] 
(0.9681884931506816, 64.28) [a] 
(0.9681894520547911, 64.4) [a] 
(0.9681898630136953, 64.65) [a] 
(0.968190136986298, 64.87) [a] 
(0.9681905479452021, 65.25) [a] 
(0.9681909589041062, 65.47) [a] 
(0.9681939726027363, 65.71) [a] 
(0.9681941095890377, 65.77) [a] 
(0.9681943835616404, 67.87) [a] 
(0.9681947945205446, 67.92) [a] 
(0.9681952054794487, 67.95) [a] 
(0.9681954794520514, 67.98) [a] 
(0.9681957534246541, 68.06) [a] 
(0.9681958904109556, 68.07) [a] 
(0.9681961643835583, 68.09) [a] 
(0.9681965753424624, 68.1) [a] 
(0.9682102739725994, 69.19) [a] 
(0.9682104109589008, 70.34) [a] 
(0.9682106849315035, 70.35) [a] 
(0.9682109589041062, 70.49) [a] 
(0.9682110958904077, 70.51) [a] 
(0.9682113698630104, 70.61) [a] 
(0.9682116438356131, 70.62) [a] 
(0.9682117808219145, 70.69) [a] 
(0.9682131506849282, 70.72) [a] 
(0.9682147945205446, 70.73) [a] 
(0.9682161643835583, 70.79) [a] 
(0.968216438356161, 70.8) [a] 
(0.9682178082191747, 70.83) [a] 
(0.9682179452054761, 70.95) [a] 
(0.9682182191780788, 70.96) [a] 
(0.9682213698630103, 72.14) [a] 
(0.9682223287671199, 72.15) [a] 
(0.968222739726024, 72.16) [a] 
(0.9682230136986267, 72.28) [a] 
(0.9682234246575309, 72.29) [a] 
(0.9682236986301336, 72.37) [a] 
(0.9682241095890377, 72.41) [a] 
(0.968226438356161, 72.84) [a] 
(0.9682286301369829, 72.86) [a] 
(0.9682326027397227, 73.37) [a] 
(0.9682343835616405, 73.5) [a] 
(0.9682354794520515, 73.6) [a] 
(0.9682361643835583, 73.61) [a] 
(0.968236438356161, 73.62) [a] 
(0.9682367123287637, 73.9) [a] 
(0.9682369863013665, 73.91) [a] 
(0.9682383561643801, 74.69) [a] 
(0.9682397260273938, 75.08) [a] 
(0.9682410958904075, 75.2) [a] 
(0.9682438356164349, 75.5) [a] 
(0.9682530136986267, 75.71) [a] 
(0.9682623287671198, 75.74) [a] 
(0.9682636986301335, 76.03) [a] 
(0.9682650684931472, 76.61) [a] 
(0.9682664383561609, 76.9) [a] 
(0.9682691780821883, 77.33) [a] 
(0.968270547945202, 77.42) [a] 
(0.9682712328767088, 79.22) [a] 
(0.9682726027397225, 80.42) [a] 
(0.9682739726027362, 80.52) [a] 
(0.9682753424657499, 81.21) [a] 
(0.9682767123287636, 81.66) [a] 
(0.9682808219178046, 81.99) [a] 
(0.968283561643832, 82) [a] 
(0.9682849315068457, 83.19) [a] 
(0.9682863013698594, 83.51) [a] 
(0.9682876712328731, 83.53) [a] 
(0.9682890410958868, 84.14) [a] 
(0.9682904109589004, 84.17) [a] 
(0.9682919178082156, 84.2) [a] 
(0.9682932876712292, 84.22) [a] 
(0.9682946575342429, 84.24) [a] 
(0.9682949315068456, 84.41) [a] 
(0.9682952054794484, 84.53) [a] 
(0.9682953424657498, 84.6) [a] 
(0.9682956164383525, 84.65) [a] 
(0.9682960273972566, 84.69) [a] 
(0.9683115068493114, 84.99) [a] 
(0.9683117808219142, 85.18) [a] 
(0.9683426027397224, 85.29) [a] 
(0.9683517808219142, 85.81) [a] 
(0.9683531506849279, 86.66) [a] 
(0.9683545205479416, 86.67) [a] 
(0.968357260273969, 86.68) [a] 
(0.9683586301369826, 86.69) [a] 
(0.9683599999999963, 87.66) [a] 
(0.96836136986301, 88.37) [a] 
(0.9683627397260237, 88.45) [a] 
(0.9683641095890374, 90.03) [a] 
(0.9683654794520511, 90.04) [a] 
(0.9683668493150648, 90.15) [a] 
(0.9683682191780785, 90.23) [a] 
(0.9683695890410922, 90.43) [a] 
(0.9683697260273936, 90.67) [a] 
(0.9683699999999963, 92.4) [a] 
(0.9683919178082153, 92.85) [a] 
(0.9683973972602701, 92.86) [a] 
(0.9684028767123248, 92.87) [a] 
(0.9684083561643796, 92.89) [a] 
(0.9684124657534207, 92.9) [a] 
(0.9684179452054754, 93.2) [a] 
(0.9684180821917768, 93.33) [a] 
(0.9684183561643795, 93.34) [a] 
(0.9684187671232837, 93.35) [a] 
(0.9684190410958864, 93.38) [a] 
(0.9684193150684891, 93.43) [a] 
(0.9684194520547905, 93.44) [a] 
(0.9684197260273932, 93.46) [a] 
(0.968419999999996, 93.5) [a] 
(0.9684204109589001, 93.53) [a] 
(0.9684210958904069, 93.55) [a] 
(0.9684213698630096, 93.59) [a] 
(0.9684227397260233, 93.63) [a] 
(0.9684228767123247, 93.74) [a] 
(0.9684242465753384, 95.24) [a] 
(0.9684247945205439, 95.65) [a] 
(0.968425205479448, 95.68) [a] 
(0.9684254794520507, 95.82) [a] 
(0.9684256164383521, 95.93) [a] 
(0.9684258904109548, 95.96) [a] 
(0.9684261643835576, 95.97) [a] 
(0.9684275342465712, 96) [a] 
(0.9684283561643795, 96.08) [a] 
(0.9684286301369822, 96.1) [a] 
(0.9684293150684891, 96.22) [a] 
(0.9684295890410918, 96.23) [a] 
(0.9684306849315027, 96.36) [a] 
(0.9684309589041055, 96.37) [a] 
(0.9684310958904069, 96.45) [a] 
(0.9684324657534206, 97.64) [a] 
(0.9684338356164343, 98.11) [a] 
(0.968439315068489, 99.24) [a] 
(0.9684394520547904, 102.04) [a] 
(0.9684397260273931, 102.71) [a] 
(0.9684399999999959, 103.28) [a] 
(0.9684402739725986, 103.29) [a] 
(0.9684406849315027, 103.45) [a] 
(0.9684409589041054, 103.46) [a] 
(0.9684410958904068, 103.48) [a] 
(0.968441506849311, 103.92) [a] 
(0.9684428767123247, 106.4) [a] 
(0.9684431506849274, 106.42) [a] 
(0.9684434246575301, 107.7) [a] 
(0.9684436986301328, 107.71) [a] 
(0.9684450684931465, 111.04) [a] 
(0.9684476712328725, 111.05) [a] 
(0.9684479452054752, 111.12) [a] 
(0.9684482191780779, 111.13) [a] 
(0.968448630136982, 111.15) [a] 
(0.9684490410958861, 111.3) [a] 
(0.9684494520547903, 118.7) [a] 
(0.968449726027393, 118.74) [a] 
(0.9684524657534204, 120.2) [a] 
(0.9684616438356122, 122.51) [a] 
(0.9684709589041053, 124.02) [a] 
(0.9684801369862971, 124.03) [a] 
(0.9684804109588998, 124.34) [a] 
(0.9684810958904067, 124.38) [a] 
(0.9684841095890367, 124.41) [a] 
(0.9684842465753382, 124.76) [a] 
(0.9684852054794477, 124.81) [a] 
(0.9684861643835573, 125.57) [a] 
(0.968487534246571, 125.8) [a] 
(0.9684879452054751, 127.32) [a] 
(0.9684882191780778, 128.91) [a] 
(0.9685060273972558, 133) [a] 
(0.9685087671232832, 133.1) [a] 
(0.9685142465753379, 133.2) [a] 
(0.9685297260273927, 134.14) [a] 
(0.9685338356164338, 135.6) [a] 
(0.9685352054794475, 144.24) [a] 
(0.9685354794520502, 145.38) [a] 
(0.9685358904109543, 145.4) [a] 
(0.968536164383557, 145.41) [a] 
(0.9685363013698585, 146.63) [a] 
(0.9685365753424612, 146.64) [a] 
(0.9685369863013653, 146.65) [a] 
(0.9685524657534201, 147.1) [a] 
(0.9685535616438311, 147.5) [a] 
(0.9685539726027352, 147.52) [a] 
(0.968554246575338, 147.85) [a] 
(0.9685545205479407, 147.91) [a] 
(0.9685549315068448, 147.94) [a] 
(0.9685552054794475, 147.95) [a] 
(0.9685553424657489, 147.96) [a] 
(0.9685558904109544, 148.11) [a] 
(0.9685580821917763, 148.15) [a] 
(0.968558356164379, 148.24) [a] 
(0.9685586301369817, 153.85) [a] 
(0.9685587671232831, 162.61) [a] 
(0.9685590410958859, 162.62) [a] 
(0.9685593150684886, 162.68) [a] 
(0.96855945205479, 162.69) [a] 
(0.9685597260273927, 162.96) [a] 
(0.9685599999999954, 163.66) [a] 
(0.9685602739725981, 163.68) [a] 
(0.9685605479452009, 163.9) [a] 
(0.9685606849315023, 164.71) [a] 
(0.968560958904105, 165.34) [a] 
(0.9685613698630091, 165.37) [a] 
(0.9685616438356118, 165.4) [a] 
(0.9685619178082145, 166.08) [a] 
(0.968562054794516, 168.94) [a] 
(0.9685638356164338, 182.65) [a] 
(0.9685663013698584, 182.66) [a] 
(0.9685695890410914, 182.67) [a] 
(0.9685706849315023, 182.68) [a] 
(0.968570958904105, 182.69) [a] 
(0.968572054794516, 182.7) [a] 
(0.9685724657534202, 182.72) [a] 
(0.9685730136986256, 183.69) [a] 
(0.968573150684927, 186.85) [a] 
(0.9685734246575297, 187.14) [a] 
(0.9685741095890366, 187.15) [a] 
(0.9685743835616393, 187.16) [a] 
(0.9685754794520502, 187.17) [a] 
(0.968575753424653, 187.51) [a] 
(0.9685758904109544, 187.87) [a] 
(0.9685895890410914, 187.99) [a] 
(0.9685980821917763, 188) [a] 
(0.9686024657534201, 188.01) [a] 
(0.968604246575338, 188.15) [a] 
(0.9686046575342421, 188.17) [a] 
(0.9686050684931462, 190.45) [a] 
(0.9686053424657489, 190.49) [a] 
(0.9686054794520503, 190.56) [a] 
(0.9686071232876667, 190.8) [a] 
(0.9686073972602695, 190.88) [a] 
(0.9686075342465709, 192.99) [a] 
(0.9686078082191736, 193.02) [a] 
(0.9686080821917763, 193.06) [a] 
(0.9686082191780777, 193.2) [a] 
(0.9686084931506804, 193.34) [a] 
(0.9686087671232831, 193.35) [a] 
(0.9686089041095846, 193.47) [a] 
(0.9686091780821873, 193.57) [a] 
(0.968610547945201, 195.9) [a] 
(0.9686149315068447, 198.1) [a] 
(0.9686152054794475, 203.63) [a] 
(0.9686153424657489, 204.45) [a] 
(0.9686167123287626, 212.1) [a] 
(0.9686169863013653, 215.76) [a] 
(0.968617260273968, 216.91) [a] 
(0.9686175342465707, 218.38) [a] 
(0.9686216438356118, 218.8) [a] 
(0.9686223287671186, 220.84) [a] 
(0.9686332876712281, 221.7) [a] 
(0.9686360273972555, 231.9) [a] 
(0.9686363013698582, 238.25) [a] 
(0.9686369863013651, 238.31) [a] 
(0.9686372602739678, 238.33) [a] 
(0.9686373972602692, 238.51) [a] 
(0.9686376712328719, 238.52) [a] 
(0.9686431506849267, 240.6) [a] 
(0.9686486301369814, 243) [a] 
(0.9686513698630088, 244.1) [a] 
(0.9687282191780773, 245.69) [a] 
(0.9687286301369814, 246.26) [a] 
(0.9687327397260225, 248.4) [a] 
(0.9687328767123239, 253.36) [a] 
(0.9687334246575293, 253.37) [a] 
(0.9687335616438307, 253.41) [a] 
(0.9687345205479403, 253.43) [a] 
(0.9687372602739677, 253.44) [a] 
(0.9687380821917759, 253.45) [a] 
(0.9687390410958855, 253.48) [a] 
(0.9687393150684882, 253.52) [a] 
(0.9687397260273923, 254.1) [a] 
(0.9687416438356115, 254.15) [a] 
(0.9687420547945156, 254.21) [a] 
(0.9687680821917759, 256) [a] 
(0.9687941095890362, 256.1) [a] 
(0.9687943835616389, 256.28) [a] 
(0.9687950684931458, 256.43) [a] 
(0.9687952054794472, 256.44) [a] 
(0.9687954794520499, 257.02) [a] 
(0.968795890410954, 257.66) [a] 
(0.9687964383561595, 257.75) [a] 
(0.9688228767123238, 258.45) [a] 
(0.9688242465753375, 258.9) [a] 
(0.9688352054794471, 259.7) [a] 
(0.9688354794520498, 268.48) [a] 
(0.968835890410954, 268.94) [a] 
(0.9688361643835567, 268.97) [a] 
(0.9689131506849266, 271.69) [a] 
(0.9689135616438307, 287.34) [a] 
(0.968925890410954, 287.8) [a] 
(0.9689260273972554, 287.81) [a] 
(0.9689265753424608, 287.89) [a] 
(0.9689267123287623, 287.9) [a] 
(0.968926986301365, 287.95) [a] 
(0.9689626027397212, 288.9) [a] 
(0.9689627397260226, 289.27) [a] 
(0.9689630136986254, 289.39) [a] 
(0.9689632876712281, 289.43) [a] 
(0.9689635616438308, 289.44) [a] 
(0.9689636986301322, 289.45) [a] 
(0.9689639726027349, 289.47) [a] 
(0.9689646575342418, 289.48) [a] 
(0.9689649315068445, 289.51) [a] 
(0.9689650684931459, 289.52) [a] 
(0.9689653424657486, 289.57) [a] 
(0.9689656164383513, 289.61) [a] 
(0.9689660273972555, 289.62) [a] 
(0.9689663013698582, 289.64) [a] 
(0.9689664383561596, 289.89) [a] 
(0.9689667123287623, 289.93) [a] 
(0.9689790410958856, 290.1) [a] 
(0.9689793150684883, 291.17) [a] 
(0.9689797260273925, 291.18) [a] 
(0.9689920547945158, 291.7) [a] 
(0.9689924657534199, 292.53) [a] 
(0.968992876712324, 298.6) [a] 
(0.9689938356164336, 298.67) [a] 
(0.9689942465753377, 299.05) [a] 
(0.9689947945205432, 299.21) [a] 
(0.9689975342465705, 299.22) [a] 
(0.9689979452054747, 299.23) [a] 
(0.9690002739725979, 299.24) [a] 
(0.969000684931502, 299.26) [a] 
(0.9690020547945157, 299.35) [a] 
(0.9690038356164336, 299.84) [a] 
(0.9690047945205431, 299.85) [a] 
(0.9690052054794472, 299.86) [a] 
(0.9690057534246527, 299.87) [a] 
(0.9690061643835568, 299.93) [a] 
(0.9690075342465705, 299.99) [a] 
(0.9690079452054746, 300.02) [a] 
(0.96900849315068, 300.22) [a] 
(0.9690098630136937, 300.23) [a] 
(0.9690102739725979, 300.24) [a] 
(0.9690116438356116, 300.71) [a] 
(0.9690226027397212, 300.8) [a] 
(0.9690235616438307, 300.82) [a] 
(0.9690239726027349, 300.96) [a] 
(0.969024383561639, 301.05) [a] 
(0.9690249315068444, 301.06) [a] 
(0.9690253424657486, 301.18) [a] 
(0.9690263013698581, 302.08) [a] 
(0.9690267123287623, 302.1) [a] 
(0.9690271232876664, 302.11) [a] 
(0.9690276712328718, 302.41) [a] 
(0.9690280821917759, 302.42) [a] 
(0.9690284931506801, 302.6) [a] 
(0.9690290410958855, 302.67) [a] 
(0.9690294520547896, 302.7) [a] 
(0.9690298630136938, 302.71) [a] 
(0.9690304109588992, 302.99) [a] 
(0.9690308219178033, 303.03) [a] 
(0.9690312328767074, 303.04) [a] 
(0.9690317808219129, 304.08) [a] 
(0.9690320547945156, 306.49) [a] 
(0.9690328767123239, 308.79) [a] 
(0.9690334246575293, 308.81) [a] 
(0.9690338356164334, 308.96) [a] 
(0.9690342465753375, 310.51) [a] 
(0.969034794520543, 311.33) [a] 
(0.9690352054794471, 311.36) [a] 
(0.9690356164383512, 311.41) [a] 
(0.9690369863013649, 311.87) [a] 
(0.9690379452054745, 313.07) [a] 
(0.9690383561643786, 313.37) [a] 
(0.9690393150684882, 313.4) [a] 
(0.9690406849315019, 313.41) [a] 
(0.969041095890406, 313.47) [a] 
(0.9690416438356114, 313.63) [a] 
(0.9690420547945156, 313.81) [a] 
(0.9690424657534197, 313.86) [a] 
(0.9690430136986251, 314.04) [a] 
(0.9690434246575292, 314.62) [a] 
(0.9690438356164334, 314.63) [a] 
(0.9690443835616388, 314.8) [a] 
(0.9690452054794471, 315.47) [a] 
(0.9690461643835566, 315.65) [a] 
(0.9690471232876662, 316.04) [a] 
(0.9690475342465703, 316.05) [a] 
(0.9690479452054744, 316.72) [a] 
(0.9690484931506799, 316.73) [a] 
(0.9690493150684881, 316.93) [a] 
(0.9690498630136936, 317.17) [a] 
(0.9690502739725977, 319.44) [a] 
(0.9690516438356114, 319.45) [a] 
(0.9690520547945155, 319.54) [a] 
(0.9690526027397209, 319.61) [a] 
(0.9690534246575292, 320.42) [a] 
(0.9690536986301319, 322.01) [a] 
(0.969055205479447, 322.02) [a] 
(0.9690554794520497, 322.04) [a] 
(0.9690557534246524, 322.06) [a] 
(0.9690558904109539, 322.08) [a] 
(0.9690561643835566, 322.27) [a] 
(0.9690568493150634, 322.28) [a] 
(0.9690571232876661, 322.29) [a] 
(0.9690572602739675, 322.32) [a] 
(0.9690575342465703, 322.33) [a] 
(0.969057808219173, 322.34) [a] 
(0.9690584931506798, 322.36) [a] 
(0.9690586301369812, 322.37) [a] 
(0.969058904109584, 322.39) [a] 
(0.9690591780821867, 322.4) [a] 
(0.9690593150684881, 322.42) [a] 
(0.9690598630136935, 322.43) [a] 
(0.9690612328767072, 322.49) [a] 
(0.9690613698630086, 322.79) [a] 
(0.9690616438356113, 324.12) [a] 
(0.9690620547945155, 324.14) [a] 
(0.9690623287671182, 324.38) [a] 
(0.9690626027397209, 324.4) [a] 
(0.9690631506849263, 325.33) [a] 
(0.9690635616438305, 325.46) [a] 
(0.9690758904109538, 327.2) [a] 
(0.9690768493150633, 327.24) [a] 
(0.9690891780821866, 327.6) [a] 
(0.96910150684931, 327.7) [a] 
(0.9691124657534196, 327.8) [a] 
(0.9691261643835566, 328.7) [a] 
(0.9691265753424607, 328.73) [a] 
(0.969138904109584, 328.8) [a] 
(0.9691393150684882, 329.12) [a] 
(0.9691398630136936, 329.8) [a] 
(0.969140410958899, 333.57) [a] 
(0.9691412328767073, 334.12) [a] 
(0.9691421917808168, 334.13) [a] 
(0.9691430136986251, 334.15) [a] 
(0.9691435616438305, 334.29) [a] 
(0.9691443835616388, 334.82) [a] 
(0.9691449315068442, 334.88) [a] 
(0.9691456164383511, 335.18) [a] 
(0.9691460273972552, 335.19) [a] 
(0.9691463013698579, 335.2) [a] 
(0.969146712328762, 337.25) [a] 
(0.9691468493150635, 337.64) [a] 
(0.9691471232876662, 353.92) [a] 
(0.9691473972602689, 353.98) [a] 
(0.9691482191780771, 354.01) [a] 
(0.9691484931506799, 354.04) [a] 
(0.9691487671232826, 354.08) [a] 
(0.969148904109584, 354.09) [a] 
(0.9691495890410908, 354.11) [a] 
(0.9691498630136935, 354.12) [a] 
(0.9691502739725977, 354.13) [a] 
(0.9691505479452004, 354.33) [a] 
(0.9691506849315018, 359.8) [a] 
(0.9691509589041045, 363.09) [a] 
(0.9691512328767072, 363.29) [a] 
(0.9691513698630086, 363.82) [a] 
(0.9691516438356114, 363.92) [a] 
(0.9691519178082141, 364.03) [a] 
(0.9691520547945155, 364.04) [a] 
(0.9691523287671182, 364.06) [a] 
(0.9691526027397209, 365.3) [a] 
(0.9691527397260223, 369.2) [a] 
(0.969153013698625, 369.22) [a] 
(0.9691532876712278, 370.28) [a] 
(0.9691535616438305, 376.1) [a] 
(0.96915452054794, 377.48) [a] 
(0.9691549315068442, 377.49) [a] 
(0.9691553424657483, 377.5) [a] 
(0.969155616438351, 377.53) [a] 
(0.9691560273972551, 380.97) [a] 
(0.9691564383561593, 381.97) [a] 
(0.9691568493150634, 397.8) [a] 
(0.9691571232876661, 397.89) [a] 
(0.969234109589036, 398.81) [a] 
(0.9692464383561593, 412) [a] 
(0.9692465753424607, 412.81) [a] 
(0.9692471232876662, 413.3) [a] 
(0.9692476712328716, 416.44) [a] 
(0.9692599999999949, 418) [a] 
(0.9692709589041045, 418.1) [a] 
(0.9692832876712278, 418.2) [a] 
(0.9693065753424608, 419.2) [a] 
(0.969342191780817, 419.3) [a] 
(0.9693545205479404, 419.4) [a] 
(0.9693549315068445, 420.96) [a] 
(0.9693553424657486, 436.26) [a] 
(0.9693556164383513, 441.77) [a] 
(0.969356986301365, 445.5) [a] 
(0.9693583561643787, 450.3) [a] 
(0.9693597260273924, 453.7) [a] 
(0.969360684931502, 455.72) [a] 
(0.9693616438356115, 455.73) [a] 
(0.9693626027397211, 456.85) [a] 
(0.9693634246575293, 458.28) [a] 
(0.9693643835616389, 458.32) [a] 
(0.9693646575342416, 461.81) [a] 
(0.969364794520543, 461.85) [a] 
(0.9693650684931457, 461.89) [a] 
(0.9693653424657485, 461.9) [a] 
(0.9693654794520499, 461.95) [a] 
(0.9693657534246526, 461.96) [a] 
(0.9693660273972553, 461.98) [a] 
(0.9693661643835567, 461.99) [a] 
(0.9693664383561594, 462.32) [a] 
(0.9693667123287621, 462.66) [a] 
(0.9693676712328717, 463.58) [a] 
(0.9693686301369813, 464.13) [a] 
(0.9693713698630086, 464.54) [a] 
(0.9693723287671182, 464.71) [a] 
(0.9693732876712278, 465.73) [a] 
(0.9693742465753373, 466.54) [a] 
(0.9693752054794469, 466.55) [a] 
(0.9693761643835564, 466.59) [a] 
(0.9693789041095838, 466.7) [a] 
(0.9693806849315016, 467.8) [a] 
(0.9693816438356112, 468.64) [a] 
(0.9693826027397208, 470.11) [a] 
(0.9693835616438303, 470.12) [a] 
(0.9693845205479399, 470.13) [a] 
(0.9693853424657481, 470.14) [a] 
(0.9693863013698577, 470.15) [a] 
(0.9693872602739673, 470.17) [a] 
(0.9693882191780768, 470.18) [a] 
(0.9693891780821864, 470.2) [a] 
(0.9693909589041042, 470.22) [a] 
(0.9693919178082138, 470.23) [a] 
(0.9693928767123233, 470.24) [a] 
(0.9693938356164329, 470.25) [a] 
(0.9693965753424603, 470.28) [a] 
(0.9693975342465698, 470.29) [a] 
(0.969399452054789, 470.31) [a] 
(0.9694004109588985, 470.32) [a] 
(0.9694013698630081, 470.33) [a] 
(0.9694031506849259, 470.34) [a] 
(0.9694041095890354, 470.36) [a] 
(0.9694060273972547, 470.4) [a] 
(0.9694069863013642, 470.46) [a] 
(0.9694078082191725, 470.51) [a] 
(0.9694097260273917, 470.66) [a] 
(0.969411643835611, 470.8) [a] 
(0.9694126027397205, 470.87) [a] 
(0.9694134246575288, 470.98) [a] 
(0.969415342465748, 471.05) [a] 
(0.9694163013698576, 471.09) [a] 
(0.9694172602739671, 471.14) [a] 
(0.9694182191780767, 471.2) [a] 
(0.969419041095885, 471.27) [a] 
(0.9694199999999945, 471.34) [a] 
(0.9694209589041041, 471.36) [a] 
(0.9694219178082136, 471.41) [a] 
(0.9694228767123232, 471.47) [a] 
(0.9694238356164327, 471.54) [a] 
(0.969424657534241, 471.65) [a] 
(0.9694256164383506, 472.31) [a] 
(0.969432465753419, 472.8) [a] 
(0.9694334246575286, 472.88) [a] 
(0.96943356164383, 473.36) [a] 
(0.9694354794520492, 474.71) [a] 
(0.969440958904104, 476.3) [a] 
(0.9694412328767067, 477.21) [a] 
(0.9694415068493094, 477.24) [a] 
(0.9694416438356108, 477.25) [a] 
(0.9694419178082135, 477.27) [a] 
(0.9694421917808163, 477.36) [a] 
(0.9694423287671177, 477.49) [a] 
(0.9694426027397204, 477.51) [a] 
(0.9694428767123231, 477.54) [a] 
(0.9694431506849258, 478.1) [a] 
(0.9694432876712272, 478.16) [a] 
(0.9694460273972546, 480.1) [a] 
(0.9694463013698573, 480.51) [a] 
(0.9694472602739669, 484.91) [a] 
(0.969447671232871, 488.04) [a] 
(0.9694480821917751, 488.05) [a] 
(0.9694482191780766, 490.31) [a] 
(0.9694484931506793, 490.32) [a] 
(0.9694489041095834, 490.33) [a] 
(0.969449863013693, 490.34) [a] 
(0.9694501369862957, 490.41) [a] 
(0.9694502739725971, 490.43) [a] 
(0.9694505479451998, 490.51) [a] 
(0.9694508219178025, 490.52) [a] 
(0.9694510958904052, 492.2) [a] 
(0.9694512328767066, 492.28) [a] 
(0.9694515068493094, 492.3) [a] 
(0.9694638356164327, 493.7) [a] 
(0.9694994520547889, 494) [a] 
(0.9695227397260219, 494.1) [a] 
(0.9695350684931452, 494.6) [a] 
(0.9695939726027344, 495.8) [a] 
(0.9696063013698577, 496) [a] 
(0.9696068493150631, 498.63) [a] 
(0.9696071232876659, 499.39) [a] 
(0.9696072602739673, 499.45) [a] 
(0.96960753424657, 501.95) [a] 
(0.9696079452054741, 503.61) [a] 
(0.9696082191780768, 514.3) [a] 
(0.9696083561643782, 526.3) [a] 
(0.9696097260273919, 531) [a] 
(0.9696110958904056, 533) [a] 
(0.9696879452054741, 544.89) [a] 
(0.9696882191780768, 545.89) [a] 
(0.9696936986301316, 546.2) [a] 
(0.9696950684931452, 549.7) [a] 
(0.9696978082191726, 550.4) [a] 
(0.9697101369862959, 553.3) [a] 
(0.9697105479452001, 554.4) [a] 
(0.9697110958904055, 589.75) [a] 
(0.9697119178082138, 589.82) [a] 
(0.969714246575337, 590.21) [a] 
(0.9697146575342411, 590.22) [a] 
(0.9697152054794466, 590.23) [a] 
(0.9697160273972548, 590.24) [a] 
(0.9697269863013644, 604) [a] 
(0.9697275342465699, 604.06) [a] 
(0.9697398630136932, 604.2) [a] 
(0.9697402739725973, 612.53) [a] 
(0.969741643835611, 612.54) [a] 
(0.9697430136986247, 612.59) [a] 
(0.9697439726027343, 612.6) [a] 
(0.969749452054789, 620.6) [a] 
(0.9697502739725973, 625.04) [a] 
(0.9697504109588987, 664.06) [a] 
(0.9697506849315014, 664.08) [a] 
(0.9697508219178028, 666.34) [a] 
(0.969751232876707, 673.72) [a] 
(0.9697515068493097, 698) [a] 
(0.9697516438356111, 712.43) [a] 
(0.9698576712328713, 736.92) [a] 
(0.9698841095890356, 737.11) [a] 
(0.9699106849315013, 737.13) [a] 
(0.9699901369862959, 743.38) [a] 
(0.9699905479452, 752.89) [a] 
(0.9699910958904054, 753.54) [a] 
(0.9699912328767069, 763.64) [a] 
(0.9699915068493096, 763.65) [a] 
(0.9699923287671178, 772.94) [a] 
(0.9699924657534192, 777.33) [a] 
(0.969992739726022, 777.35) [a] 
(0.9699930136986247, 778.04) [a] 
(0.9699931506849261, 779.08) [a] 
(0.9699934246575288, 779.11) [a] 
(0.9699936986301315, 779.86) [a] 
(0.9700030136986246, 783.42) [a] 
(0.9700213698630082, 783.43) [a] 
(0.9700306849315014, 789.3) [a] 
(0.9700398630136932, 789.31) [a] 
(0.9700401369862959, 789.82) [a] 
(0.9700402739725973, 789.83) [a] 
(0.9700495890410904, 798.12) [a] 
(0.9700587671232822, 798.17) [a] 
(0.9700772602739671, 798.22) [a] 
(0.9700956164383507, 798.35) [a] 
(0.9701049315068438, 798.78) [a] 
(0.9701052054794466, 800.34) [a] 
(0.9701143835616384, 800.69) [a] 
(0.9701236986301315, 800.7) [a] 
(0.9701373972602685, 805.5) [a] 
(0.9701376712328712, 813.88) [a] 
(0.9701379452054739, 813.92) [a] 
(0.9701380821917753, 813.95) [a] 
(0.970138356164378, 814.26) [a] 
(0.9701475342465699, 819.01) [a] 
(0.9701567123287617, 819.45) [a] 
(0.9701660273972548, 819.47) [a] 
(0.9701669863013643, 820.74) [a] 
(0.9701793150684876, 835.3) [a] 
(0.9701795890410904, 890.61) [a] 
(0.9701798630136931, 898.53) [a] 
(0.9701801369862958, 898.54) [a] 
(0.9701802739725972, 898.58) [a] 
(0.9701808219178026, 898.59) [a] 
(0.970180958904104, 905.31) [a] 
(0.9701810958904055, 915.99) [a] 
(0.970182054794515, 916.06) [a] 
(0.9701824657534192, 916.07) [a] 
(0.9701834246575287, 918.56) [a] 
(0.9701847945205424, 918.61) [a] 
(0.9701852054794465, 919.04) [a] 
(0.9701856164383507, 919.08) [a] 
(0.9701869863013644, 919.4) [a] 
(0.9701961643835562, 922.3) [a] 
(0.9702054794520493, 922.32) [a] 
(0.9702146575342411, 923.26) [a] 
(0.9702238356164329, 923.28) [a] 
(0.970233150684926, 923.33) [a] 
(0.9702423287671178, 923.4) [a] 
(0.9702516438356109, 923.5) [a] 
(0.9702521917808163, 925.87) [a] 
(0.9702534246575286, 926.24) [a] 
(0.9702565753424601, 926.29) [a] 
(0.9702830136986245, 938.66) [a] 
(0.9703095890410902, 938.67) [a] 
(0.9703105479451998, 939.27) [a] 
(0.9703126027397203, 939.28) [a] 
(0.9703130136986244, 939.8) [a] 
(0.9703134246575286, 939.9) [a] 
(0.9703398630136929, 941.79) [a] 
(0.9703408219178025, 944.32) [a] 
(0.9703499999999943, 948.9) [a] 
(0.9703591780821861, 949.34) [a] 
(0.9703684931506792, 950.38) [a] 
(0.970377671232871, 952.38) [a] 
(0.9703790410958847, 953) [a] 
(0.9703882191780765, 953.4) [a] 
(0.9703886301369806, 988.15) [a] 
(0.970389178082186, 988.2) [a] 
(0.9703895890410902, 988.29) [a] 
(0.9703898630136929, 989.8) [a] 
(0.970390273972597, 1001.1) [a] 
(0.9703995890410901, 1007.7) [a] 
(0.970408767123282, 1008.5) [a] 
(0.9704089041095834, 1022.4) [a] 
(0.970410273972597, 1088) [a] 
(0.9704116438356107, 1089) [a] 
(0.9704226027397203, 1090) [a] 
(0.9704239726027339, 1091) [a] 
(0.9704294520547887, 1092) [a] 
(0.9704295890410901, 1120) [a] 
(0.9704298630136928, 1142.2) [a] 
(0.970430273972597, 1142.3) [a] 
(0.9704387671232819, 1159.5) [a] 
(0.9704524657534189, 1168) [a] 
(0.9704647945205422, 1169) [a] 
(0.970465068493145, 1276.8) [a] 
(0.9704653424657477, 1742) [a] 
(0.9704658904109531, 1880.8) [a] 
(0.9704701369862956, 1896.7) [a] 
(0.9704705479451997, 1932.8) [a] 
(0.9704719178082134, 2019) [a] 
(0.9704746575342408, 2095) [a] 
(0.9704760273972545, 2096) [a] 
(0.9704773972602682, 2106) [a] 
(0.9704787671232818, 2108) [a] 
(0.9704789041095833, 2124.1) [a] 
(0.970479178082186, 2141.2) [a] 
(0.9704805479451997, 2185) [a] 
(0.9704819178082134, 2205) [a] 
(0.9704846575342407, 2211) [a] 
(0.9704873972602681, 2233) [a] 
(0.9704901369862955, 2235) [a] 
(0.9704915068493092, 2248) [a] 
(0.9705024657534187, 2315) [a] 
(0.9705065753424598, 2350) [a] 
(0.9705120547945145, 2466) [a] 
(0.9705134246575282, 2480) [a] 
(0.970518904109583, 2509) [a] 
(0.9705191780821857, 2519.5) [a] 
(0.9705193150684871, 2519.6) [a] 
(0.9705247945205419, 2584) [a] 
(0.9705302739725966, 2590) [a] 
(0.970530410958898, 2590.7) [a] 
(0.9705345205479391, 2591) [a] 
(0.9705349315068432, 2601.5) [a] 
(0.9705384931506789, 2601.7) [a] 
(0.9705430136986241, 2602.1) [a] 
(0.9705584931506789, 2607.5) [a] 
(0.9705621917808158, 2610) [a] 
(0.970593013698624, 2616.4) [a] 
(0.9706084931506789, 2616.5) [a] 
(0.9706239726027337, 2618.6) [a] 
(0.9706393150684871, 2625.8) [a] 
(0.9706410958904049, 2631.3) [a] 
(0.9706424657534186, 2631.4) [a] 
(0.970645205479446, 2632) [a] 
(0.9706465753424597, 2632.1) [a] 
(0.9706493150684871, 2650) [a] 
(0.9706506849315008, 2746) [a] 
(0.9706563013698569, 2746.4) [a] 
(0.9706571232876652, 2748.9) [a] 
(0.9706598630136926, 2758) [a] 
(0.9706636986301308, 2789.5) [a] 
(0.9706646575342404, 2789.8) [a] 
(0.9706654794520486, 2790.1) [a] 
(0.9706664383561582, 2790.2) [a] 
(0.9706702739725965, 2791.4) [a] 
(0.9706710958904048, 2793.7) [a] 
(0.9706752054794459, 2794) [a] 
(0.9706761643835554, 2865) [a] 
(0.9706916438356102, 3059) [a] 
},{(0.8376900139820977, 0) [b] 
(0.8767567398860433, 0.001) [b] 
(0.8886131303190717, 0.002) [b] 
(0.8916529000219553, 0.003) [b] 
(0.8946961797557202, 0.004) [b] 
(0.8976931120676619, 0.005) [b] 
(0.9005804475751058, 0.006) [b] 
(0.9024372274318829, 0.007) [b] 
(0.9044364170752185, 0.008) [b] 
(0.9058465305280985, 0.009) [b] 
(0.9097762597397734, 0.01) [b] 
(0.9106073847412837, 0.011) [b] 
(0.9118857143922223, 0.012) [b] 
(0.9121496708401005, 0.013) [b] 
(0.9132834383273725, 0.014) [b] 
(0.9138242152627452, 0.015) [b] 
(0.9146955191684331, 0.016) [b] 
(0.9154868223657393, 0.017) [b] 
(0.9164138986517395, 0.018) [b] 
(0.9180236492053733, 0.019) [b] 
(0.9192933292202428, 0.02) [b] 
(0.9196272152794488, 0.021) [b] 
(0.9203628330376294, 0.022) [b] 
(0.9204634756730997, 0.023) [b] 
(0.9206894680490959, 0.024) [b] 
(0.9210499000285238, 0.025) [b] 
(0.9211290222429078, 0.026) [b] 
(0.9212664646908405, 0.027) [b] 
(0.9217743796577806, 0.028) [b] 
(0.9219448778850695, 0.029) [b] 
(0.9220875662641906, 0.03) [b] 
(0.9221446740900906, 0.031) [b] 
(0.9223120886712041, 0.032) [b] 
(0.9224253046638345, 0.033) [b] 
(0.9225036265151643, 0.034) [b] 
(0.9226361057295169, 0.035) [b] 
(0.9226905266937181, 0.036) [b] 
(0.9226948480281462, 0.037) [b] 
(0.9227169816724856, 0.038) [b] 
(0.9228238482260045, 0.039) [b] 
(0.9231466781065716, 0.04) [b] 
(0.9233630930396741, 0.041) [b] 
(0.923472567127163, 0.042) [b] 
(0.9240577316363359, 0.043) [b] 
(0.9240721137461707, 0.044) [b] 
(0.9243131165036171, 0.045) [b] 
(0.9243951386657229, 0.046) [b] 
(0.9244253736783743, 0.047) [b] 
(0.9244275949577868, 0.048) [b] 
(0.924445365193087, 0.049) [b] 
(0.9245433935469244, 0.05) [b] 
(0.9248149482280588, 0.051) [b] 
(0.9249857219265589, 0.052) [b] 
(0.9250858779905208, 0.053) [b] 
(0.9251987085892817, 0.054) [b] 
(0.9252424695774072, 0.055) [b] 
(0.9252535263780999, 0.056) [b] 
(0.9252592881573373, 0.057) [b] 
(0.9252711790292384, 0.058) [b] 
(0.9252742529384615, 0.059) [b] 
(0.9252920896964524, 0.06) [b] 
(0.9253099264544432, 0.061) [b] 
(0.9253378739646343, 0.062) [b] 
(0.9253784398671685, 0.063) [b] 
(0.9254012709173968, 0.064) [b] 
(0.9254547567431163, 0.065) [b] 
(0.9254574964691437, 0.066) [b] 
(0.9254821825422029, 0.067) [b] 
(0.9256209284573396, 0.068) [b] 
(0.9256819215625745, 0.069) [b] 
(0.9257855745306109, 0.07) [b] 
(0.9257872079950248, 0.071) [b] 
(0.9257994011549384, 0.072) [b] 
(0.9258008415997477, 0.073) [b] 
(0.9258055371484609, 0.074) [b] 
(0.9262338791434039, 0.077) [b] 
(0.9262361004228165, 0.078) [b] 
(0.9262665946394305, 0.079) [b] 
(0.9262862578814396, 0.08) [b] 
(0.9264963320823528, 0.081) [b] 
(0.9265782817692211, 0.082) [b] 
(0.9266610013830379, 0.083) [b] 
(0.9266795230725355, 0.084) [b] 
(0.9267383782632874, 0.085) [b] 
(0.9268802610934617, 0.086) [b] 
(0.9269663258702647, 0.087) [b] 
(0.9269922538768331, 0.088) [b] 
(0.9270962902071789, 0.089) [b] 
(0.9271122719423387, 0.09) [b] 
(0.9271201050812119, 0.091) [b] 
(0.9271306260900994, 0.092) [b] 
(0.9272685256334785, 0.093) [b] 
(0.9272936397887296, 0.094) [b] 
(0.9273103439069612, 0.095) [b] 
(0.9273219899032449, 0.096) [b] 
(0.9273831771178568, 0.097) [b] 
(0.9274488788226132, 0.098) [b] 
(0.9275423161560618, 0.099) [b] 
(0.9275543294657718, 0.1) [b] 
(0.9279736428551874, 0.101) [b] 
(0.9280112023590029, 0.102) [b] 
(0.9280217046421079, 0.103) [b] 
(0.9280573916622598, 0.104) [b] 
(0.9280804510229902, 0.105) [b] 
(0.9280956959706299, 0.106) [b] 
(0.9281041787501122, 0.107) [b] 
(0.9281190189327606, 0.108) [b] 
(0.9281669641382401, 0.109) [b] 
(0.9281706171062766, 0.11) [b] 
(0.9281754116268245, 0.111) [b] 
(0.9281781513528519, 0.112) [b] 
(0.9281808910788792, 0.113) [b] 
(0.9289513902836405, 0.114) [b] 
(0.9290488104469868, 0.115) [b] 
(0.9290529200360279, 0.116) [b] 
(0.9310163588645588, 0.117) [b] 
(0.9310339387732346, 0.118) [b] 
(0.9310353086362483, 0.119) [b] 
(0.9310385049832802, 0.12) [b] 
(0.9310435278143304, 0.121) [b] 
(0.9314946576005119, 0.122) [b] 
(0.9314964840845301, 0.123) [b] 
(0.9315119161179177, 0.124) [b] 
(0.9315128293599269, 0.125) [b] 
(0.9315322642914337, 0.126) [b] 
(0.9315327209124382, 0.127) [b] 
(0.9315434889576105, 0.128) [b] 
(0.9315455437521311, 0.129) [b] 
(0.9315493378837589, 0.131) [b] 
(0.9315504794362703, 0.132) [b] 
(0.9315895205321607, 0.133) [b] 
(0.9315936301212018, 0.134) [b] 
(0.9325356707447257, 0.135) [b] 
(0.9325379538497485, 0.136) [b] 
(0.9325407641575715, 0.137) [b] 
(0.9325416773995806, 0.138) [b] 
(0.932544417125608, 0.139) [b] 
(0.9325468734317015, 0.14) [b] 
(0.932550526399738, 0.141) [b] 
(0.9325514396417471, 0.142) [b] 
(0.932565823203391, 0.145) [b] 
(0.9325660515138933, 0.146) [b] 
(0.9325693537336187, 0.147) [b] 
(0.9325841939162671, 0.148) [b] 
(0.9325844222267694, 0.149) [b] 
(0.9325864770212899, 0.15) [b] 
(0.932588988436815, 0.151) [b] 
(0.9325935546468607, 0.153) [b] 
(0.9325953811308789, 0.154) [b] 
(0.9326093451887222, 0.155) [b] 
(0.9326296648234254, 0.156) [b] 
(0.9326298931339276, 0.159) [b] 
(0.9326303497549322, 0.16) [b] 
(0.932631034686439, 0.161) [b] 
(0.9326529524946582, 0.163) [b] 
(0.9326688528925928, 0.164) [b] 
(0.9326693095135974, 0.165) [b] 
(0.9326761588286658, 0.166) [b] 
(0.9326779853126841, 0.169) [b] 
(0.9326816382807206, 0.17) [b] 
(0.9326994750387114, 0.173) [b] 
(0.9327001599702183, 0.174) [b] 
(0.9327008449017251, 0.175) [b] 
(0.9327010732122274, 0.176) [b] 
(0.9327245891939625, 0.178) [b] 
(0.9327278475044951, 0.182) [b] 
(0.9327294456780111, 0.183) [b] 
(0.9327296739885134, 0.184) [b] 
(0.932732651304344, 0.186) [b] 
(0.9327365325828828, 0.187) [b] 
(0.932743153587449, 0.188) [b] 
(0.9327618750486362, 0.189) [b] 
(0.932826026628645, 0.19) [b] 
(0.9328299079071838, 0.191) [b] 
(0.9328456613318413, 0.192) [b] 
(0.9329040146654208, 0.193) [b] 
(0.9329074393229551, 0.194) [b] 
(0.9329083525649642, 0.195) [b] 
(0.9329213662635943, 0.196) [b] 
(0.9329222795056035, 0.197) [b] 
(0.9329277589576582, 0.198) [b] 
(0.9329337043100185, 0.199) [b] 
(0.9329741152689227, 0.2) [b] 
(0.9329937499721189, 0.201) [b] 
(0.9330147253232409, 0.202) [b] 
(0.9330243143643367, 0.203) [b] 
(0.9330496568300901, 0.204) [b] 
(0.933060844044702, 0.205) [b] 
(0.9332297938163914, 0.206) [b] 
(0.9332647253232407, 0.207) [b] 
(0.9333396111679895, 0.208) [b] 
(0.9333831919017996, 0.209) [b] 
(0.9334329104470456, 0.21) [b] 
(0.9335202470809166, 0.211) [b] 
(0.9335236717384509, 0.212) [b] 
(0.9335683648096844, 0.213) [b] 
(0.9336148870078984, 0.214) [b] 
(0.9336160285604098, 0.216) [b] 
(0.9336178550444281, 0.217) [b] 
(0.9336281290170309, 0.218) [b] 
(0.9336514166882637, 0.219) [b] 
(0.9336644011718058, 0.22) [b] 
(0.9338214787973765, 0.221) [b] 
(0.9338669125873309, 0.222) [b] 
(0.9338723920393857, 0.223) [b] 
(0.93393883039555, 0.224) [b] 
(0.9339808395279701, 0.225) [b] 
(0.9340201089343628, 0.226) [b] 
(0.9340607482037692, 0.227) [b] 
(0.9341644011718058, 0.228) [b] 
(0.9341876888430386, 0.229) [b] 
(0.9341940815371025, 0.23) [b] 
(0.9342366879557655, 0.231) [b] 
(0.9342378295082769, 0.232) [b] 
(0.9343202495996011, 0.233) [b] 
(0.934328240467181, 0.234) [b] 
(0.9343404353971132, 0.235) [b] 
(0.9343469776739199, 0.236) [b] 
(0.9343686671716367, 0.237) [b] 
(0.9344076816954012, 0.238) [b] 
(0.9345049525144519, 0.239) [b] 
(0.9349236739756389, 0.24) [b] 
(0.9349747889560245, 0.241) [b] 
(0.9350480500551315, 0.242) [b] 
(0.9351147167217981, 0.243) [b] 
(0.9351781604693069, 0.244) [b] 
(0.935182270058348, 0.245) [b] 
(0.9353656487101799, 0.246) [b] 
(0.9354140239645381, 0.247) [b] 
(0.9355318341504031, 0.248) [b] 
(0.9357110598613823, 0.249) [b] 
(0.9357364023271356, 0.25) [b] 
(0.9357816078065876, 0.251) [b] 
(0.9357873155691446, 0.252) [b] 
(0.9358249868020213, 0.253) [b] 
(0.9359103749298752, 0.254) [b] 
(0.9359160826924322, 0.255) [b] 
(0.9360384571216559, 0.256) [b] 
(0.9360644845189161, 0.257) [b] 
(0.9361254434230256, 0.258) [b] 
(0.9361382288111534, 0.259) [b] 
(0.936142795021199, 0.26) [b] 
(0.9361585484458566, 0.261) [b] 
(0.9361713338339844, 0.262) [b] 
(0.9362398269846693, 0.263) [b] 
(0.9362409685371808, 0.264) [b] 
(0.9362432516422036, 0.265) [b] 
(0.9362462196787332, 0.266) [b] 
(0.9362471329207424, 0.267) [b] 
(0.9363208772129797, 0.268) [b] 
(0.9363352607746236, 0.269) [b] 
(0.9363578897687703, 0.27) [b] 
(0.9363661089468525, 0.271) [b] 
(0.9364309491295009, 0.272) [b] 
(0.9365519536957109, 0.273) [b] 
(0.9367535518692268, 0.274) [b] 
(0.9367546934217382, 0.275) [b] 
(0.936758574700277, 0.276) [b] 
(0.9367718167094095, 0.277) [b] 
(0.9370871135130624, 0.278) [b] 
(0.9376084996627972, 0.279) [b] 
(0.9376358969230711, 0.28) [b] 
(0.9381137508043497, 0.281) [b] 
(0.9381283626764958, 0.282) [b] 
(0.9381388379231282, 0.283) [b] 
(0.9381390662336305, 0.284) [b] 
(0.9385255959139958, 0.285) [b] 
(0.9385358518622834, 0.286) [b] 
(0.9386223815426487, 0.287) [b] 
(0.938626719442192, 0.288) [b] 
(0.938654116702466, 0.289) [b] 
(0.9386956692138816, 0.29) [b] 
(0.9387214390855515, 0.291) [b] 
(0.9388269185376062, 0.292) [b] 
(0.9388563705924008, 0.293) [b] 
(0.9388748637430856, 0.294) [b] 
(0.9390250920535878, 0.295) [b] 
(0.9391707541540443, 0.296) [b] 
(0.9392463249303, 0.297) [b] 
(0.9392636765284735, 0.298) [b] 
(0.9393296582636332, 0.299) [b] 
(0.9395042887614072, 0.3) [b] 
(0.9396718686700829, 0.301) [b] 
(0.9397844257477085, 0.302) [b] 
(0.9398088113133316, 0.303) [b] 
(0.9398615510393591, 0.304) [b] 
(0.9400638341443819, 0.305) [b] 
(0.940086042226604, 0.306) [b] 
(0.9401031655142752, 0.307) [b] 
(0.9401317043270605, 0.308) [b] 
(0.9401990559252339, 0.309) [b] 
(0.9402061335508046, 0.31) [b] 
(0.9402353572950969, 0.311) [b] 
(0.9402410650576539, 0.312) [b] 
(0.9402568184823114, 0.313) [b] 
(0.9402906084366491, 0.314) [b] 
(0.9403077317243203, 0.315) [b] 
(0.9403239417699822, 0.316) [b] 
(0.9403353572950963, 0.317) [b] 
(0.9406985993042287, 0.318) [b] 
(0.9407013390302561, 0.319) [b] 
(0.940704535377288, 0.32) [b] 
(0.9407481712220369, 0.321) [b] 
(0.940909586747151, 0.322) [b] 
(0.9409146095782012, 0.323) [b] 
(0.9409150661992057, 0.324) [b] 
(0.9409173720806777, 0.325) [b] 
(0.9410029885190339, 0.326) [b] 
(0.9410125086847715, 0.327) [b] 
(0.9410141068582875, 0.328) [b] 
(0.9410718694153651, 0.329) [b] 
(0.9410861655345948, 0.33) [b] 
(0.9410886769501199, 0.331) [b] 
(0.9411135627948687, 0.332) [b] 
(0.9411382203291152, 0.333) [b] 
(0.9411477177184037, 0.334) [b] 
(0.9411502291339288, 0.335) [b] 
(0.9411665530618366, 0.336) [b] 
(0.941169292787864, 0.337) [b] 
(0.9411889560298731, 0.338) [b] 
(0.9415544811440283, 0.339) [b] 
(0.9415588190435716, 0.341) [b] 
(0.9416449100599382, 0.342) [b] 
(0.9416606634845958, 0.343) [b] 
(0.9416613484161026, 0.344) [b] 
(0.9416668278681574, 0.345) [b] 
(0.941668197731171, 0.346) [b] 
(0.941674749860379, 0.349) [b] 
(0.9416766218972956, 0.35) [b] 
(0.9416818958152973, 0.352) [b] 
(0.9416880601988589, 0.353) [b] 
(0.9417051675210238, 0.354) [b] 
(0.9417102359049724, 0.355) [b] 
(0.941710692525977, 0.356) [b] 
(0.9417319977303515, 0.357) [b] 
(0.9417326826618584, 0.358) [b] 
(0.9417329109723607, 0.359) [b] 
(0.9417379338034109, 0.36) [b] 
(0.9417463812919953, 0.361) [b] 
(0.9417722031552024, 0.362) [b] 
(0.9417753995022343, 0.363) [b] 
(0.9417758561232389, 0.364) [b] 
(0.941776769365248, 0.365) [b] 
(0.9417790524702708, 0.366) [b] 
(0.9418147600144235, 0.367) [b] 
(0.941815901566935, 0.368) [b] 
(0.9418250106645664, 0.369) [b] 
(0.9418290683975943, 0.37) [b] 
(0.9418315798131194, 0.371) [b] 
(0.9418318081236217, 0.372) [b] 
(0.941833248568431, 0.373) [b] 
(0.9418350750524492, 0.374) [b] 
(0.9418452191773, 0.375) [b] 
(0.9418454474878023, 0.377) [b] 
(0.9418739863005877, 0.378) [b] 
(0.9419118858439667, 0.38) [b] 
(0.9419196484010443, 0.381) [b] 
(0.9419201050220488, 0.382) [b] 
(0.9419210182640579, 0.383) [b] 
(0.9419226164375739, 0.384) [b] 
(0.9419230730585785, 0.385) [b] 
(0.9419239863005876, 0.386) [b] 
(0.9420617654095915, 0.387) [b] 
(0.942065418377628, 0.388) [b] 
(0.9420663316196372, 0.389) [b] 
(0.942067016551144, 0.39) [b] 
(0.9420803041131748, 0.391) [b] 
(0.942082130597193, 0.392) [b] 
(0.9422661488620332, 0.393) [b] 
(0.9422757373571183, 0.394) [b] 
(0.94228722121158, 0.395) [b] 
(0.9422881344535892, 0.396) [b] 
(0.9422931410119274, 0.397) [b] 
(0.942293597632932, 0.398) [b] 
(0.9422954696698486, 0.399) [b] 
(0.9425027756059217, 0.4) [b] 
(0.9425055836612966, 0.401) [b] 
(0.9425094649398353, 0.402) [b] 
(0.9425122729952102, 0.403) [b] 
(0.9425188939997764, 0.404) [b] 
(0.9425275236999541, 0.405) [b] 
(0.9425419072615979, 0.406) [b] 
(0.9425430488141093, 0.407) [b] 
(0.9425481627509562, 0.408) [b] 
(0.9425545554450201, 0.409) [b] 
(0.9425790302216627, 0.41) [b] 
(0.9425974827292128, 0.411) [b] 
(0.9426077567018155, 0.412) [b] 
(0.9427086699438246, 0.417) [b] 
(0.9427099295879752, 0.42) [b] 
(0.942710874321088, 0.421) [b] 
(0.942777769298257, 0.422) [b] 
(0.9427999154169784, 0.427) [b] 
(0.9428006003484852, 0.428) [b] 
(0.9428201248328184, 0.429) [b] 
(0.9428390746045079, 0.43) [b] 
(0.9429134766837168, 0.431) [b] 
(0.9429139333047214, 0.433) [b] 
(0.9429223807933058, 0.434) [b] 
(0.9429289329225138, 0.435) [b] 
(0.9429396629701103, 0.437) [b] 
(0.9429433159381468, 0.438) [b] 
(0.9429505617135162, 0.439) [b] 
(0.9429543057873494, 0.441) [b] 
(0.9429622332353453, 0.444) [b] 
(0.9429996739736767, 0.449) [b] 
(0.942999902284179, 0.45) [b] 
(0.9430218200923981, 0.453) [b] 
(0.9430220484029004, 0.454) [b] 
(0.9430248564582753, 0.456) [b] 
(0.9430278244948049, 0.459) [b] 
(0.9430324590341981, 0.46) [b] 
(0.9430386234177597, 0.461) [b] 
(0.9430413631437871, 0.462) [b] 
(0.9430891000851596, 0.465) [b] 
(0.9430928441589927, 0.468) [b] 
(0.9431340289711573, 0.472) [b] 
(0.9431559467793764, 0.475) [b] 
(0.9432244399300613, 0.476) [b] 
(0.9432482222740491, 0.479) [b] 
(0.9432957869620248, 0.48) [b] 
(0.9433106271446732, 0.49) [b] 
(0.94331131207618, 0.501) [b] 
(0.9433679330807462, 0.506) [b] 
(0.9434474222304773, 0.51) [b] 
(0.9434476505409796, 0.514) [b] 
(0.9434478788514818, 0.515) [b] 
(0.9434503857609282, 0.517) [b] 
(0.9434508423819328, 0.518) [b] 
(0.9434515273134396, 0.519) [b] 
(0.9434517556239419, 0.52) [b] 
(0.9434519839344442, 0.521) [b] 
(0.9434556369024807, 0.524) [b] 
(0.9434565501444898, 0.525) [b] 
(0.9434602031125263, 0.529) [b] 
(0.9434622579070469, 0.533) [b] 
(0.943468878911613, 0.535) [b] 
(0.9434748149846723, 0.536) [b] 
(0.943479381194718, 0.54) [b] 
(0.9434974177243983, 0.542) [b] 
(0.9435072350759965, 0.543) [b] 
(0.9435118012860422, 0.544) [b] 
(0.9435131711490559, 0.545) [b] 
(0.9435145410120696, 0.546) [b] 
(0.9435149976330741, 0.548) [b] 
(0.943532834391065, 0.549) [b] 
(0.9435506711490558, 0.55) [b] 
(0.9435584337061333, 0.553) [b] 
(0.9435636848476858, 0.554) [b] 
(0.943564598089695, 0.555) [b] 
(0.9435828914686903, 0.557) [b] 
(0.9435904257152656, 0.56) [b] 
(0.9435917955782793, 0.565) [b] 
(0.9435922521992839, 0.569) [b] 
(0.9436692107955591, 0.57) [b] 
(0.9436703523480705, 0.572) [b] 
(0.9436881891060613, 0.573) [b] 
(0.9436927102553194, 0.585) [b] 
(0.9436976894282485, 0.594) [b] 
(0.9436983743597553, 0.595) [b] 
(0.9436988264746812, 0.596) [b] 
(0.9437015662007086, 0.601) [b] 
(0.9437097853787908, 0.605) [b] 
(0.9437113835523068, 0.608) [b] 
(0.9437120684838136, 0.61) [b] 
(0.9437257671139506, 0.615) [b] 
(0.9437531643742246, 0.617) [b] 
(0.9437661488577668, 0.627) [b] 
(0.9437671425403853, 0.637) [b] 
(0.9439805547488741, 0.652) [b] 
(0.9439810113698787, 0.666) [b] 
(0.9439816963013855, 0.678) [b] 
(0.9439862625114311, 0.682) [b] 
(0.943986947442938, 0.693) [b] 
(0.943999604015285, 0.695) [b] 
(0.9440045078042409, 0.698) [b] 
(0.9440051927357478, 0.716) [b] 
(0.9440200329183962, 0.723) [b] 
(0.9442727579021328, 0.725) [b] 
(0.9442732145231374, 0.727) [b] 
(0.9442760225785123, 0.739) [b] 
(0.9442771641310237, 0.761) [b] 
(0.9444624957857639, 0.771) [b] 
(0.9444631807172708, 0.772) [b] 
(0.9444697328464787, 0.776) [b] 
(0.9444753489572284, 0.778) [b] 
(0.9444775096244424, 0.783) [b] 
(0.9444777379349447, 0.785) [b] 
(0.9444804776609721, 0.787) [b] 
(0.9444945179378463, 0.798) [b] 
(0.9445722074698839, 0.836) [b] 
(0.944576426327333, 0.847) [b] 
(0.9447354046267951, 0.852) [b] 
(0.9447358612477996, 0.859) [b] 
(0.9447365461793065, 0.869) [b] 
(0.944737002800311, 0.882) [b] 
(0.9447372311108133, 0.908) [b] 
(0.9447390575948316, 0.911) [b] 
(0.9450647920183144, 0.924) [b] 
(0.945108279733035, 0.934) [b] 
(0.9451300235903953, 0.935) [b] 
(0.9451517674477556, 0.938) [b] 
(0.9451601916138802, 0.959) [b] 
(0.9451658993764372, 0.981) [b] 
(0.9451665843079441, 1.021) [b] 
(0.9451680247527534, 1.027) [b] 
(0.9451708328081283, 1.055) [b] 
(0.9452146684245667, 1.086) [b] 
(0.9452585040410051, 1.087) [b] 
(0.9453530419052918, 1.093) [b] 
(0.9453684856822219, 1.124) [b] 
(0.945385334014471, 1.132) [b] 
(0.9453882149040898, 1.14) [b] 
(0.9454038869915221, 1.157) [b] 
(0.9454046072139268, 1.219) [b] 
(0.945405520455936, 1.239) [b] 
(0.9454064564743943, 1.249) [b] 
(0.9454083285113108, 1.251) [b] 
(0.945408503921134, 1.479) [b] 
(0.9454093809702496, 1.482) [b] 
(0.9454100826095422, 1.485) [b] 
(0.9454102580193653, 1.487) [b] 
(0.9454136826768995, 1.51) [b] 
(0.9454155091609178, 1.57) [b] 
(0.9454160353903871, 1.589) [b] 
(0.9454231130159578, 1.631) [b] 
(0.945432930367556, 1.636) [b] 
(0.9454334565970254, 1.688) [b] 
(0.9454338074166716, 1.698) [b] 
(0.9454346844657873, 1.699) [b] 
(0.9455230091441169, 1.805) [b] 
(0.945535734862764, 1.849) [b] 
(0.9455363408493662, 1.858) [b] 
(0.9455366438426673, 1.86) [b] 
(0.9455369468359685, 1.865) [b] 
(0.9455375528225707, 1.872) [b] 
(0.9455378558158718, 1.875) [b] 
(0.945540885748883, 1.884) [b] 
(0.9455411887421841, 1.886) [b] 
(0.9456951059347345, 1.89) [b] 
(0.9456996508342513, 1.903) [b] 
(0.9457002568208536, 1.905) [b] 
(0.9457038927404671, 1.912) [b] 
(0.9457051047136715, 1.918) [b] 
(0.9457069226734782, 1.921) [b] 
(0.9457081346466827, 1.926) [b] 
(0.945708362957185, 1.93) [b] 
(0.9457086659504861, 1.957) [b] 
(0.9457104839102928, 1.979) [b] 
(0.9457106593201159, 1.981) [b] 
(0.9457514290526664, 1.985) [b] 
(0.9457520350392686, 1.986) [b] 
(0.9486533589103838, 1.989) [b] 
(0.9490422832759409, 1.991) [b] 
(0.9496327414715006, 1.992) [b] 
(0.9497677033447713, 1.993) [b] 
(0.9498520545155656, 1.994) [b] 
(0.949852835911991, 1.996) [b] 
(0.9498546538717977, 1.999) [b] 
(0.9498883943401154, 2) [b] 
(0.9498890003267176, 2.002) [b] 
(0.9498893033200188, 2.008) [b] 
(0.9499061735541776, 2.009) [b] 
(0.949924466933173, 2.033) [b] 
(0.9499299208125932, 2.042) [b] 
(0.9499636612809109, 2.043) [b] 
(0.9499805315150698, 2.045) [b] 
(0.9499858328425221, 2.047) [b] 
(0.950002703076681, 2.048) [b] 
(0.9500060360029934, 2.049) [b] 
(0.9500397764713111, 2.054) [b] 
(0.9500412914378167, 2.06) [b] 
(0.9500581616719755, 2.063) [b] 
(0.95005937364518, 2.067) [b] 
(0.9500596766384811, 2.07) [b] 
(0.9500599796317822, 2.076) [b] 
(0.9500602826250834, 2.078) [b] 
(0.9500611916049867, 2.081) [b] 
(0.9500801413766762, 2.117) [b] 
(0.9500988628378634, 2.133) [b] 
(0.9501261322349642, 2.144) [b] 
(0.9501587806367907, 2.148) [b] 
(0.9501733925089368, 2.182) [b] 
(0.9501754473034574, 2.196) [b] 
(0.9501757981231036, 2.211) [b] 
(0.9501759735329267, 2.213) [b] 
(0.9501784292704505, 2.257) [b] 
(0.9501789554999198, 2.258) [b] 
(0.950179306319566, 2.259) [b] 
(0.95020295035637, 2.265) [b] 
(0.9502033011760163, 2.268) [b] 
(0.95020856347071, 2.27) [b] 
(0.9502124224868188, 2.272) [b] 
(0.9508703616190141, 2.281) [b] 
(0.9511571555997145, 2.282) [b] 
(0.951325857941303, 2.283) [b] 
(0.9513595984096207, 2.284) [b] 
(0.9513623253493307, 2.285) [b] 
(0.9513753098328729, 2.287) [b] 
(0.9514596610036672, 2.291) [b] 
(0.9514611759701728, 2.293) [b] 
(0.9515092194208902, 2.302) [b] 
(0.9515167942534182, 2.305) [b] 
(0.9515674049558948, 2.306) [b] 
(0.9516011454242125, 2.307) [b] 
(0.9516271143912968, 2.309) [b] 
(0.9516513209468881, 2.311) [b] 
(0.9516531389066948, 2.318) [b] 
(0.9516534279215577, 2.324) [b] 
(0.9516537169364206, 2.326) [b] 
(0.9516710515952655, 2.328) [b] 
(0.9516743845215779, 2.341) [b] 
(0.9516745771981533, 2.344) [b] 
(0.9516749280177995, 2.348) [b] 
(0.9516804074698543, 2.353) [b] 
(0.9518875664709655, 2.376) [b] 
(0.9518876628092532, 2.379) [b] 
(0.951904533043412, 2.381) [b] 
(0.9519056346998908, 2.385) [b] 
(0.9519058273764661, 2.386) [b] 
(0.9519059237147538, 2.387) [b] 
(0.9519065980827675, 2.389) [b] 
(0.9519075070626708, 2.39) [b] 
(0.9519100173569462, 2.391) [b] 
(0.9519122386363588, 2.393) [b] 
(0.9519131502189788, 2.395) [b] 
(0.9519132465572665, 2.396) [b] 
(0.951945997031404, 2.4) [b] 
(0.9519482183108166, 2.407) [b] 
(0.9519679805725416, 2.408) [b] 
(0.9519761657859404, 2.41) [b] 
(0.951979143101771, 2.42) [b] 
(0.9519803550749755, 2.428) [b] 
(0.95203560916926, 2.43) [b] 
(0.9520357845790831, 2.433) [b] 
(0.952036073593946, 2.436) [b] 
(0.9520380842303281, 2.443) [b] 
(0.9520470161778202, 2.447) [b] 
(0.9520472088543955, 2.449) [b] 
(0.9520614170500687, 2.451) [b] 
(0.9520619432795381, 2.452) [b] 
(0.9520621359561134, 2.456) [b] 
(0.9520623286326888, 2.457) [b] 
(0.9520632376125922, 2.458) [b] 
(0.9520635406058933, 2.46) [b] 
(0.9520717848675802, 2.466) [b] 
(0.9520723908541824, 2.487) [b] 
(0.9520726798690453, 2.488) [b] 
(0.9520728725456207, 2.49) [b] 
(0.9520729688839084, 2.492) [b] 
(0.952073065222196, 2.493) [b] 
(0.9520736712087983, 2.496) [b] 
(0.9520739742020994, 2.506) [b] 
(0.9520745801887016, 2.513) [b] 
(0.9520751861753038, 2.516) [b] 
(0.9520781634911345, 2.517) [b] 
(0.9520784525059974, 2.521) [b] 
(0.9520793614859008, 2.522) [b] 
(0.9520802704658041, 2.525) [b] 
(0.9520805734591052, 2.528) [b] 
(0.9520808764524064, 2.534) [b] 
(0.9520967804347635, 2.543) [b] 
(0.9521126844171207, 2.544) [b] 
(0.9521179857445731, 2.55) [b] 
(0.9521184674360114, 2.579) [b] 
(0.9521415267967419, 2.592) [b] 
(0.9521468281241943, 2.629) [b] 
(0.9521521294516466, 2.63) [b] 
(0.9521610145692967, 2.637) [b] 
(0.9521654571281217, 2.639) [b] 
(0.9521698996869468, 2.648) [b] 
(0.9529162684843457, 2.659) [b] 
(0.9529334454500348, 2.661) [b] 
(0.9529387467774871, 2.662) [b] 
(0.9529430410189094, 2.668) [b] 
(0.9529433918385556, 2.693) [b] 
(0.9529435672483787, 2.732) [b] 
(0.9529604374825376, 2.74) [b] 
(0.9529631772085649, 2.767) [b] 
(0.9529810139665558, 2.817) [b] 
(0.9529988507245466, 2.82) [b] 
(0.952999992277058, 2.826) [b] 
(0.9536363791153574, 2.9) [b] 
(0.9536365545251805, 2.961) [b] 
(0.9536367299350036, 2.967) [b] 
(0.9536456618824957, 2.974) [b] 
(0.9536461185035002, 2.981) [b] 
(0.9537230770997754, 3.041) [b] 
(0.9537283784272278, 3.046) [b] 
(0.9537452486613867, 3.083) [b] 
(0.9541213273221677, 3.107) [b] 
(0.9541414994518271, 3.127) [b] 
(0.9541424036816788, 3.136) [b] 
(0.9541428557966046, 3.137) [b] 
(0.9541430312064277, 3.147) [b] 
(0.9541539066154615, 3.15) [b] 
(0.9541567131726315, 3.153) [b] 
(0.9541568885824546, 3.16) [b] 
(0.9541737588166135, 3.176) [b] 
(0.9541829896994551, 3.205) [b] 
(0.954183441814381, 3.227) [b] 
(0.9541838939293068, 3.251) [b] 
(0.9541861152087193, 3.272) [b] 
(0.9544355479772038, 3.285) [b] 
(0.9544364250263194, 3.297) [b] 
(0.9544367758459656, 3.31) [b] 
(0.9545316127492166, 3.311) [b] 
(0.9545484829833755, 3.35) [b] 
(0.9545822234516932, 3.354) [b] 
(0.954599093685852, 3.356) [b] 
(0.9546120781693942, 3.358) [b] 
(0.9546380471364785, 3.366) [b] 
(0.9548467848259978, 3.424) [b] 
(0.9548476618751135, 3.435) [b] 
(0.9549215089378466, 3.499) [b] 
(0.9549492015863714, 3.5) [b] 
(0.9549676633520547, 3.505) [b] 
(0.9550648403940663, 3.54) [b] 
(0.9550657536360754, 3.544) [b] 
(0.9550835903940662, 3.556) [b] 
(0.955101427152057, 3.57) [b] 
(0.9551475815662652, 3.59) [b] 
(0.9551568124491069, 3.591) [b] 
(0.9551660433319485, 3.596) [b] 
(0.9551838800899394, 3.631) [b] 
(0.9552552271219028, 3.632) [b] 
(0.9552730638798936, 3.633) [b] 
(0.9552909006378845, 3.634) [b] 
(0.9553087373958753, 3.636) [b] 
(0.9553255767388954, 3.64) [b] 
(0.9553434134968862, 3.652) [b] 
(0.9553969237708588, 3.653) [b] 
(0.9554147605288497, 3.654) [b] 
(0.9554861075608131, 3.655) [b] 
(0.9555039443188039, 3.66) [b] 
(0.9555217810767948, 3.661) [b] 
(0.9555396178347856, 3.662) [b] 
(0.9555398461452879, 3.784) [b] 
(0.9555583079109712, 3.785) [b] 
(0.9555675387938128, 3.798) [b] 
(0.9555853755518037, 3.808) [b] 
(0.9556032123097945, 3.988) [b] 
(0.9556061803463242, 4.058) [b] 
(0.9556114816737765, 4.283) [b] 
(0.9556269254507066, 4.379) [b] 
(0.9557221374759854, 4.389) [b] 
(0.9557816449917846, 4.391) [b] 
(0.9558411525075838, 4.395) [b] 
(0.9558565962845139, 4.427) [b] 
(0.9558620501639341, 4.462) [b] 
(0.9558632621371386, 4.465) [b] 
(0.9558635651304397, 4.468) [b] 
(0.955864474110343, 4.47) [b] 
(0.9558687160165588, 4.479) [b] 
(0.9558720489428711, 4.483) [b] 
(0.955883950446031, 4.542) [b] 
(0.9558869804460614, 4.558) [b] 
(0.9558957672517938, 4.613) [b] 
(0.9558966762316972, 4.616) [b] 
(0.9559009624113395, 4.633) [b] 
(0.9559247654176591, 4.689) [b] 
(0.9559329462367894, 4.729) [b] 
(0.9559805522494288, 4.777) [b] 
(0.9560281582620682, 4.778) [b] 
(0.956115829494945, 4.78) [b] 
(0.9561377473031641, 4.781) [b] 
(0.9561486550620045, 4.822) [b] 
(0.9561510790084135, 4.828) [b] 
(0.9561522205609249, 4.831) [b] 
(0.9561591894068506, 4.986) [b] 
(0.9561594924001517, 4.995) [b] 
(0.9561607043733562, 5) [b] 
(0.9561610073666573, 5.003) [b] 
(0.9561613103599584, 5.006) [b] 
(0.9561900774832461, 5.05) [b] 
(0.956272269264068, 5.138) [b] 
(0.9562759051836816, 5.16) [b] 
(0.956277117156886, 5.484) [b] 
(0.9562777231434882, 5.487) [b] 
(0.9562955599014791, 5.52) [b] 
(0.9562963252735953, 5.916) [b] 
(0.9562970906457116, 5.987) [b] 
(0.956298232198223, 6.378) [b] 
(0.9562989171297298, 6.39) [b] 
(0.9562993737507344, 6.649) [b] 
(0.9562996767440355, 6.973) [b] 
(0.9563126612275776, 7) [b] 
(0.9563391576108213, 7.099) [b] 
(0.9563560278449802, 7.256) [b] 
(0.9564095381189528, 7.477) [b] 
(0.9564273748769436, 7.478) [b] 
(0.9564452116349345, 7.492) [b] 
(0.9564630483929253, 7.493) [b] 
(0.9564660783259364, 7.942) [b] 
(0.9564697142455499, 7.947) [b] 
(0.9564700172388511, 7.952) [b] 
(0.9564703202321522, 7.959) [b] 
(0.9564712292120555, 7.972) [b] 
(0.9564742591450667, 7.973) [b] 
(0.9564757741115723, 7.977) [b] 
(0.9564763800981745, 7.981) [b] 
(0.9564769860847767, 7.989) [b] 
(0.956517587187127, 8.019) [b] 
(0.9565233440598483, 8.044) [b] 
(0.9565254650129561, 8.049) [b] 
(0.9565260709995583, 8.053) [b] 
(0.9565263739928594, 8.056) [b] 
(0.9565272829727628, 8.063) [b] 
(0.956527888959365, 8.07) [b] 
(0.9565284949459673, 8.073) [b] 
(0.9565287979392684, 8.076) [b] 
(0.9565291009325695, 8.079) [b] 
(0.9565297069191717, 8.096) [b] 
(0.956530312905774, 8.134) [b] 
(0.9565333428387851, 8.149) [b] 
(0.9565351607985918, 8.152) [b] 
(0.956535766785194, 8.156) [b] 
(0.9565372817516996, 8.159) [b] 
(0.9565384937249041, 8.162) [b] 
(0.9565387967182052, 8.165) [b] 
(0.9565427356311198, 8.179) [b] 
(0.956543341617722, 8.182) [b] 
(0.9565439476043243, 8.188) [b] 
(0.9565460685574321, 8.193) [b] 
(0.9565475835239377, 8.197) [b] 
(0.9565478865172388, 8.203) [b] 
(0.9565481895105399, 8.206) [b] 
(0.956548492503841, 8.209) [b] 
(0.9565487954971421, 8.227) [b] 
(0.9565494014837443, 8.232) [b] 
(0.9565506134569488, 8.235) [b] 
(0.9565509164502499, 8.239) [b] 
(0.9565527344100566, 8.243) [b] 
(0.9565530374033577, 8.25) [b] 
(0.9565533403966588, 8.261) [b] 
(0.95655364338996, 8.263) [b] 
(0.9565579295696023, 8.869) [b] 
(0.9566007913660258, 8.87) [b] 
(0.9566038213660562, 8.875) [b] 
(0.956615722869216, 9.248) [b] 
(0.9566164078007229, 9.537) [b] 
(0.956717629205676, 9.559) [b] 
(0.9567344994398349, 9.605) [b] 
(0.9567682399081526, 9.607) [b] 
(0.9567851101423115, 9.671) [b] 
(0.9568357208447881, 9.673) [b] 
(0.9568525910789469, 9.703) [b] 
(0.9568530431938728, 9.783) [b] 
(0.9568543995386501, 9.803) [b] 
(0.9568553127806593, 10.052) [b] 
(0.9568571392646775, 10.06) [b] 
(0.9568573675751798, 10.149) [b] 
(0.956860792232714, 10.215) [b] 
(0.9568612488537186, 10.329) [b] 
(0.9568617009686444, 10.337) [b] 
(0.9568696234150958, 10.402) [b] 
(0.9568775458615473, 10.404) [b] 
(0.9568797065287613, 10.563) [b] 
(0.9568811469735706, 10.565) [b] 
(0.9568818671959753, 10.631) [b] 
(0.9568987374301342, 10.692) [b] 
(0.9569047974301949, 10.962) [b] 
(0.9569055176525996, 10.997) [b] 
(0.9569069580974089, 11.007) [b] 
(0.9569076783198136, 11.521) [b] 
(0.9569081349408182, 11.842) [b] 
(0.9569088551632229, 12.123) [b] 
(0.9569095753856276, 12.149) [b] 
(0.9569102956080323, 12.785) [b] 
(0.9569281323660231, 12.819) [b] 
(0.956945969124014, 12.822) [b] 
(0.9569464212389398, 12.998) [b] 
(0.9569468733538656, 13.008) [b] 
(0.9570238319501408, 13.438) [b] 
(0.9570280508075899, 13.757) [b] 
(0.9570285029225157, 13.788) [b] 
(0.9570289595435203, 14.376) [b] 
(0.9570330691325614, 14.471) [b] 
(0.9570415068474594, 15.551) [b] 
(0.9570457257049084, 15.555) [b] 
(0.9570499445623575, 15.612) [b] 
(0.9570503966772833, 15.789) [b] 
(0.957051766540297, 15.83) [b] 
(0.9570559853977461, 16.037) [b] 
(0.9570602042551951, 16.095) [b] 
(0.9570739028853321, 16.249) [b] 
(0.9570876015154691, 16.379) [b] 
(0.9571645601117443, 16.698) [b] 
(0.9571700395637991, 16.737) [b] 
(0.9571854833407292, 16.782) [b] 
(0.9571859399617337, 16.926) [b] 
(0.9571890287171197, 17.188) [b] 
(0.9572011969570774, 17.628) [b] 
(0.957244684671798, 17.864) [b] 
(0.9572625214297888, 17.908) [b] 
(0.9572803581877797, 18.342) [b] 
(0.9572858376398344, 19.787) [b] 
(0.9572906321603823, 19.838) [b] 
(0.9572948510178314, 19.943) [b] 
(0.957295307638836, 20.17) [b] 
(0.9572996018802582, 21.134) [b] 
(0.9573164721144171, 21.475) [b] 
(0.9573603077308553, 23.146) [b] 
(0.9575794858130472, 23.147) [b] 
(0.9575809262578565, 24.225) [b] 
(0.9575811545683588, 24.652) [b] 
(0.9575930560715187, 26.687) [b] 
(0.9576049575746786, 26.758) [b] 
(0.9576168590778384, 26.766) [b] 
(0.9576170873883407, 27.385) [b] 
(0.9576178076107454, 27.6) [b] 
(0.9576205473367728, 28.924) [b] 
(0.9576207756472751, 28.972) [b] 
(0.9576386124052659, 28.977) [b] 
(0.9576564491632568, 29.031) [b] 
(0.9576571693856615, 29.199) [b] 
(0.957657626006666, 30.082) [b] 
(0.957662420527214, 34.446) [b] 
(0.9576635620797254, 34.498) [b] 
(0.9576640187007299, 34.621) [b] 
(0.9576653885637436, 35.129) [b] 
(0.9576658451847482, 35.248) [b] 
(0.9576677172216648, 36.648) [b] 
(0.9576707472216951, 37.347) [b] 
(0.9576737772217254, 37.419) [b] 
(0.9576768072217557, 37.591) [b] 
(0.9576772638427603, 37.931) [b] 
(0.9576951006007511, 40.114) [b] 
(0.9576993948421734, 41.161) [b] 
(0.9577148386191034, 43.44) [b] 
(0.95771849158714, 44.379) [b] 
(0.9577207746921628, 44.429) [b] 
(0.9577212313131673, 44.526) [b] 
(0.957722135543019, 44.642) [b] 
(0.9577230397728707, 44.845) [b] 
(0.957724396117648, 44.859) [b] 
(0.9577310599558856, 45.259) [b] 
(0.9577355025147106, 45.266) [b] 
(0.9577399450735357, 45.267) [b] 
(0.9577443876323607, 45.414) [b] 
(0.9577466089117732, 45.42) [b] 
(0.9577468372222755, 45.56) [b] 
(0.9577646739802663, 46.263) [b] 
(0.9577668952596788, 47.201) [b] 
(0.9577671235701811, 47.205) [b] 
(0.9577682651226925, 47.266) [b] 
(0.9577698632962085, 47.31) [b] 
(0.9577700916067108, 47.711) [b] 
(0.9577703199172131, 48.262) [b] 
(0.9578059934331948, 51.225) [b] 
(0.9578238301911857, 51.226) [b] 
(0.9578416669491765, 51.229) [b] 
(0.9578595037071673, 51.239) [b] 
(0.9578773404651582, 51.24) [b] 
(0.9578910390952952, 52.92) [b] 
(0.9579047377254322, 52.922) [b] 
(0.9579184363555692, 52.923) [b] 
(0.9579321349857062, 52.924) [b] 
(0.9579458336158432, 52.925) [b] 
(0.9579595322459802, 52.926) [b] 
(0.9579732308761172, 52.928) [b] 
(0.9579869295062542, 52.929) [b] 
(0.9580828199172131, 52.93) [b] 
(0.9580965185473501, 52.932) [b] 
(0.9581102171774871, 52.934) [b] 
(0.9581239158076241, 52.937) [b] 
(0.9581376144377611, 52.951) [b] 
(0.9581513130678981, 52.954) [b] 
(0.9581650116980351, 52.986) [b] 
(0.958315696629542, 53.046) [b] 
(0.958356792519953, 53.049) [b] 
(0.95837049115009, 53.103) [b] 
(0.958384189780227, 54.039) [b] 
(0.9583849100026317, 55.551) [b] 
(0.9583853666236363, 55.915) [b] 
(0.958411589289333, 60.654) [b] 
(0.9584120414042588, 60.655) [b] 
(0.958413849863962, 60.657) [b] 
(0.9584147540938137, 60.658) [b] 
(0.9584174667833686, 60.668) [b] 
(0.9584205555387546, 60.675) [b] 
(0.9584246245730869, 60.705) [b] 
(0.9584250766880127, 60.71) [b] 
(0.9584255288029385, 60.731) [b] 
(0.9584259809178644, 60.738) [b] 
(0.9584262092283666, 62.16) [b] 
(0.9584289219179215, 62.674) [b] 
(0.9584293740328473, 63.108) [b] 
(0.958430278262699, 63.131) [b] 
(0.958443976892836, 63.266) [b] 
(0.958457675522973, 63.269) [b] 
(0.95847137415311, 63.273) [b] 
(0.958485072783247, 63.278) [b] 
(0.958498771413384, 63.439) [b] 
(0.958526168673658, 63.456) [b] 
(0.9585266207885839, 63.666) [b] 
(0.9585270729035097, 63.674) [b] 
(0.9585275250184355, 63.835) [b] 
(0.9585279771333614, 64.376) [b] 
(0.9585284292482872, 64.384) [b] 
(0.9585292962928761, 65.661) [b] 
(0.9585429949230131, 68.723) [b] 
(0.9585566935531501, 68.728) [b] 
(0.9585703921832871, 68.738) [b] 
(0.95869367985452, 68.739) [b] 
(0.958734775744931, 68.741) [b] 
(0.958748474375068, 68.751) [b] 
(0.958762173005205, 68.764) [b] 
(0.9587758716353421, 68.833) [b] 
(0.9588032688956161, 69.07) [b] 
(0.9588169675257531, 69.075) [b] 
(0.9588299520092952, 69.12) [b] 
(0.9588468222434541, 71.289) [b] 
(0.9588605208735911, 71.592) [b] 
(0.9588635508736214, 72.723) [b] 
(0.958865092286224, 72.781) [b] 
(0.9588681222862543, 73.168) [b] 
(0.9588818209163913, 74.114) [b] 
(0.9589777113273502, 74.115) [b] 
(0.9589914099574872, 74.12) [b] 
(0.9590188072177612, 74.121) [b] 
(0.9590325058478982, 74.122) [b] 
(0.9590736017383092, 74.123) [b] 
(0.9591009989985833, 74.124) [b] 
(0.9591283962588573, 74.125) [b] 
(0.9591557935191313, 74.126) [b] 
(0.9591694921492683, 74.14) [b] 
(0.9591831907794053, 74.141) [b] 
(0.9591968894095423, 74.142) [b] 
(0.9592123331864724, 77.075) [b] 
(0.9592277769634024, 77.698) [b] 
(0.9592432207403325, 78.896) [b] 
(0.9592729011056293, 78.934) [b] 
(0.9592733577266339, 78.998) [b] 
(0.9592735860371362, 79.337) [b] 
(0.9592738143476385, 79.397) [b] 
(0.9592763257631636, 79.721) [b] 
(0.9592917695400937, 81.226) [b] 
(0.9593072133170237, 81.632) [b] 
(0.9593226570939538, 82.939) [b] 
(0.9593228854044561, 86.845) [b] 
(0.9593233420254607, 86.887) [b] 
(0.9593235703359629, 86.929) [b] 
(0.9593237986464652, 86.974) [b] 
(0.959327679925004, 87.092) [b] 
(0.9593327027560542, 87.141) [b] 
(0.9593338443085656, 87.385) [b] 
(0.9593347803270239, 88.81) [b] 
(0.9593440112098656, 89.931) [b] 
(0.9593453810728793, 90.185) [b] 
(0.9593458376938838, 92.617) [b] 
(0.9593460660043861, 92.902) [b] 
(0.9593468313765023, 94.082) [b] 
(0.9593646681344932, 96.353) [b] 
(0.9593655813765023, 100.649) [b] 
(0.9593658096870046, 101.038) [b] 
(0.9593711110144569, 101.176) [b] 
(0.9593764123419093, 101.177) [b] 
(0.9593870149968141, 101.181) [b] 
(0.9593923163242665, 101.182) [b] 
(0.9593976176517188, 101.183) [b] 
(0.9594007064071048, 101.961) [b] 
(0.9594099372899465, 108.84) [b] 
(0.9595955098656939, 119.842) [b] 
(0.9596123800998527, 119.871) [b] 
(0.9596292503340116, 119.877) [b] 
(0.9596461205681704, 121.285) [b] 
(0.9596629908023293, 121.347) [b] 
(0.9596798610364882, 121.49) [b] 
(0.9596826007625155, 121.971) [b] 
(0.9596830573835201, 121.987) [b] 
(0.9596844272465338, 122.193) [b] 
(0.9596857971095475, 122.651) [b] 
(0.9597026673437064, 126.147) [b] 
(0.9597058636907383, 127.545) [b] 
(0.9597067769327474, 127.589) [b] 
(0.9597070052432497, 128.296) [b] 
(0.9597076901747565, 128.931) [b] 
(0.9597083751062634, 128.935) [b] 
(0.9597090600377702, 128.944) [b] 
(0.9597111148322908, 128.95) [b] 
(0.9597138545583181, 128.954) [b] 
(0.959714539489825, 128.965) [b] 
(0.9597152244213318, 128.969) [b] 
(0.9597159093528387, 128.982) [b] 
(0.9597165942843455, 129.083) [b] 
(0.9597172792158524, 129.095) [b] 
(0.9597341494500112, 129.252) [b] 
(0.9597348343815181, 130.355) [b] 
(0.9597355193130249, 130.365) [b] 
(0.9597362042445318, 131.521) [b] 
(0.9597392929999178, 140.31) [b] 
(0.9597669856484426, 140.902) [b] 
(0.9597762165312843, 140.904) [b] 
(0.959785447414126, 140.907) [b] 
(0.9597946782969676, 140.913) [b] 
(0.9598316018283342, 140.957) [b] 
(0.9598408327111758, 140.958) [b] 
(0.9598500635940175, 141.171) [b] 
(0.9598592944768591, 141.175) [b] 
(0.9598685253597008, 142.121) [b] 
(0.9598962180082257, 142.513) [b] 
(0.9599054488910673, 142.517) [b] 
(0.959914679773909, 142.53) [b] 
(0.9599239106567506, 142.551) [b] 
(0.9599331415395923, 142.702) [b] 
(0.959942372422434, 145.881) [b] 
(0.9599445937018465, 146.686) [b] 
(0.9599455069438556, 150.505) [b] 
(0.9600224655401308, 150.854) [b] 
(0.960099424136406, 151.297) [b] 
(0.9601172608943969, 165.45) [b] 
(0.9601264917772385, 166.609) [b] 
(0.9601357226600802, 166.885) [b] 
(0.9601368642125916, 167.359) [b] 
(0.9601377774546007, 167.448) [b] 
(0.9601507619381429, 178.215) [b] 
(0.9601537919381732, 179.179) [b] 
(0.9601626960477623, 195.517) [b] 
(0.9601652074632874, 195.598) [b] 
(0.9601754814358902, 195.713) [b] 
(0.9601830156824654, 196.124) [b] 
(0.9601883170099178, 207.772) [b] 
(0.9601936183373702, 208.472) [b] 
(0.9601989196648225, 208.563) [b] 
(0.9601993717797483, 212.559) [b] 
(0.9601998238946742, 212.58) [b] 
(0.9602119438947954, 213.055) [b] 
(0.9602149738948257, 213.076) [b] 
(0.9602184560097818, 213.097) [b] 
(0.9602189081247077, 213.201) [b] 
(0.9602198123545593, 215.029) [b] 
(0.9602202644694852, 215.03) [b] 
(0.960220716584411, 215.034) [b] 
(0.9602211686993368, 215.166) [b] 
(0.9602225250441142, 220.939) [b] 
(0.96022297715904, 221.034) [b] 
(0.9603768943515905, 222.134) [b] 
(0.9603775792830973, 224.512) [b] 
(0.9603954160410881, 224.984) [b] 
(0.960412286275247, 225.385) [b] 
(0.9604127428962516, 235.194) [b] 
(0.9604136561382607, 240.406) [b] 
(0.9604255576414206, 244.389) [b] 
(0.9604612621509001, 251.125) [b] 
(0.9604969666603796, 251.126) [b] 
(0.9605088681635395, 251.134) [b] 
(0.9605207696666994, 251.177) [b] 
(0.9605326711698593, 251.224) [b] 
(0.9605445726730192, 251.225) [b] 
(0.9605564741761791, 251.227) [b] 
(0.960568375679339, 251.742) [b] 
(0.9605704304738595, 261.37) [b] 
(0.9605713437158686, 261.432) [b] 
(0.9605720286473755, 261.764) [b] 
(0.9605722569578777, 262.362) [b] 
(0.96057248526838, 262.724) [b] 
(0.9605731701998869, 262.938) [b] 
(0.9605736268208914, 263.32) [b] 
(0.9605743117523983, 265.702) [b] 
(0.9605747683734028, 265.813) [b] 
(0.9606117798460823, 268.316) [b] 
(0.960612832305021, 271.376) [b] 
(0.9606135172365279, 277.677) [b] 
(0.9606188185639802, 285.142) [b] 
(0.9606241198914326, 285.227) [b] 
(0.9606455910985439, 285.87) [b] 
(0.9606498853399661, 285.876) [b] 
(0.9606570771427142, 286.899) [b] 
(0.9606623784701666, 287.649) [b] 
(0.9606626067806688, 288.289) [b] 
(0.9606628350911711, 290.107) [b] 
(0.9606630634016734, 299.732) [b] 
(0.9607400219979486, 318.594) [b] 
(0.9607578587559394, 320.174) [b] 
(0.9607631600833918, 329.01) [b] 
(0.9607696420850339, 329.545) [b] 
(0.9607746836418667, 329.547) [b] 
(0.9607754038642714, 329.551) [b] 
(0.9607761240866761, 329.599) [b] 
(0.9607768443090808, 329.607) [b] 
(0.9607790049762949, 329.78) [b] 
(0.9607797251986996, 329.783) [b] 
(0.9607804454211043, 329.788) [b] 
(0.9607826060883183, 329.789) [b] 
(0.960783326310723, 329.807) [b] 
(0.9607840465331278, 330.029) [b] 
(0.9607905285347699, 334.047) [b] 
(0.9607919689795792, 334.048) [b] 
(0.9607934094243885, 334.089) [b] 
(0.9607941296467932, 334.137) [b] 
(0.9607948498691979, 334.139) [b] 
(0.9607955700916027, 334.144) [b] 
(0.9608431761042421, 334.179) [b] 
(0.960855077607402, 334.181) [b] 
(0.9608557978298067, 334.214) [b] 
(0.9608676993329666, 334.24) [b] 
(0.9608684195553713, 334.354) [b] 
(0.9608713968712019, 334.849) [b] 
(0.9608721170936066, 336.716) [b] 
(0.9608840185967665, 337.493) [b] 
(0.9609078216030862, 337.523) [b] 
(0.9609197231062461, 347.366) [b] 
(0.9609435261125657, 347.367) [b] 
(0.9609554276157256, 347.375) [b] 
(0.9609673291188855, 347.38) [b] 
(0.9609792306220454, 347.382) [b] 
(0.9610268366346848, 347.434) [b] 
(0.9610387381378447, 347.601) [b] 
(0.9610506396410046, 347.633) [b] 
(0.9610625411441645, 348.064) [b] 
(0.9610632613665692, 348.133) [b] 
(0.9610639815889739, 348.695) [b] 
(0.9610647018113786, 348.855) [b] 
(0.9610654220337833, 350.298) [b] 
(0.9610726138365314, 357.248) [b] 
(0.9610755911523621, 360.757) [b] 
(0.9611550803020932, 365.74) [b] 
(0.9612080730685806, 365.808) [b] 
(0.9612345694518243, 366.344) [b] 
(0.961261065835068, 367.363) [b] 
(0.9613145761090406, 386.248) [b] 
(0.9613502496250224, 386.252) [b] 
(0.9613680863830132, 386.287) [b] 
(0.961385923141004, 386.399) [b] 
(0.9614013669179341, 389.745) [b] 
(0.9614044556733201, 391.7) [b] 
(0.96141635717648, 400.934) [b] 
(0.9614165854869823, 403.723) [b] 
(0.9614170421079868, 404.363) [b] 
(0.9614192633873994, 413.132) [b] 
(0.9614214846668119, 413.139) [b] 
(0.9614237059462244, 413.206) [b] 
(0.9614259272256369, 413.887) [b] 
(0.9614271550943988, 425.949) [b] 
(0.9614363859772405, 428.309) [b] 
(0.9614512725563938, 439.988) [b] 
(0.9614542498722245, 440.249) [b] 
(0.9614661513753844, 445.446) [b] 
(0.961475382258226, 446.923) [b] 
(0.9614872837613859, 455.866) [b] 
(0.9616197656776043, 457.957) [b] 
(0.9616327501611465, 479.595) [b] 
(0.9616457346446886, 479.608) [b] 
(0.9616500288861108, 499.562) [b] 
(0.9616543231275331, 499.571) [b] 
(0.9616572911640627, 510.213) [b] 
(0.9616579760955696, 510.27) [b] 
(0.9616582044060719, 510.389) [b] 
(0.9616584327165741, 511.549) [b] 
(0.9616626515740232, 516.533) [b] 
(0.9616718824568649, 565.99) [b] 
(0.9616741655618877, 585.095) [b] 
(0.9616755354249014, 585.148) [b] 
(0.961684766307743, 591.058) [b] 
(0.9616848626460307, 592.289) [b] 
(0.9616917119610991, 598.773) [b] 
(0.9616958215501402, 598.841) [b] 
(0.9616962781711448, 598.967) [b] 
(0.9616967347921493, 601.252) [b] 
(0.9616971914131539, 601.378) [b] 
(0.9616974197236562, 608.801) [b] 
(0.9616976480341585, 612.041) [b] 
(0.9617033557967155, 612.728) [b] 
(0.9617127485890503, 707.245) [b] 
(0.9617170347686926, 710.571) [b] 
(0.9617188527284993, 723.737) [b] 
(0.961719572950904, 724.52) [b] 
(0.9617202931733088, 742.641) [b] 
(0.9617212064153179, 751.815) [b] 
(0.9617216630363224, 758.084) [b] 
(0.9617308939191641, 758.24) [b] 
(0.9617311222296664, 813.622) [b] 
(0.9617479924638253, 831.633) [b] 
(0.9617648626979841, 831.646) [b] 
(0.961781732932143, 831.648) [b] 
(0.9617826689506013, 833.747) [b] 
(0.9617833538821081, 877.861) [b] 
(0.961784038813615, 881.804) [b] 
(0.9617849748320733, 885.173) [b] 
(0.9617942057149149, 960.908) [b] 
(0.9618034365977566, 961.69) [b] 
(0.9618126674805982, 961.7) [b] 
(0.9618154072066256, 962.997) [b] 
(0.9618322774407845, 964.106) [b] 
(0.9618491476749433, 971.263) [b] 
(0.9618496042959479, 976.869) [b] 
(0.9618518874009707, 988.262) [b] 
(0.961856181642393, 993.361) [b] 
(0.9618584647474158, 1005.68) [b] 
(0.9618607478524386, 1008.28) [b] 
(0.9618609761629409, 1008.34) [b] 
(0.9618612044734431, 1008.46) [b] 
(0.9618614327839454, 1009.24) [b] 
(0.9618635537370532, 1065.1) [b] 
(0.9618696767139835, 1075.35) [b] 
(0.9618712074582161, 1075.39) [b] 
(0.9618719728303323, 1075.4) [b] 
(0.9618727382024486, 1075.57) [b] 
(0.9618735035745648, 1075.58) [b] 
(0.961874268946681, 1075.64) [b] 
(0.9618757996909136, 1077.18) [b] 
(0.9618765650630299, 1081.42) [b] 
(0.9618773304351461, 1081.96) [b] 
(0.9618780958072624, 1082.08) [b] 
(0.9618834534120763, 1088.84) [b] 
(0.9618842187841925, 1088.86) [b] 
(0.961885749528425, 1088.87) [b] 
(0.9618865149005412, 1088.95) [b] 
(0.9618895449005715, 1164.33) [b] 
(0.961901188736188, 1166.36) [b] 
(0.9619057549462336, 1166.43) [b] 
(0.9619066681882428, 1166.5) [b] 
(0.9619071248092473, 1166.57) [b] 
(0.9619075814302519, 1166.78) [b] 
(0.9619082663617587, 1166.85) [b] 
(0.961908494672261, 1177.79) [b] 
(0.9619263314302519, 1199.9) [b] 
(0.9619417752071819, 1286.62) [b] 
(0.9619478352072426, 1313.49) [b] 
(0.9619508652072729, 1313.51) [b] 
(0.9619538952073032, 1313.53) [b] 
(0.9619569252073336, 1414.54) [b] 
(0.9619718449998852, 1415.94) [b] 
(0.9619727492297369, 1415.97) [b] 
(0.9619759140342176, 1415.99) [b] 
(0.9619781746088466, 1416) [b] 
(0.9619786267237724, 1416.12) [b] 
(0.9619795309536241, 1416.95) [b] 
(0.9619799830685499, 1417.97) [b] 
(0.9619804351834758, 1417.98) [b] 
(0.9619813394133274, 1420.75) [b] 
(0.9619858605625855, 1420.78) [b] 
(0.9619863126775113, 1420.79) [b] 
(0.9619867647924372, 1421.05) [b] 
(0.9619904177604737, 1421.93) [b] 
(0.9619922442444919, 1421.95) [b] 
(0.9619927008654965, 1422.16) [b] 
(0.961993157486501, 1422.55) [b] 
(0.9619936096014269, 1422.63) [b] 
(0.9619940617163527, 1422.97) [b] 
(0.9619977146843892, 1423.07) [b] 
(0.9619981713053938, 1423.41) [b] 
(0.9619990845474029, 1429.05) [b] 
(0.9620004544104166, 1436.96) [b] 
(0.9620013676524257, 1437.04) [b] 
(0.9620022718822774, 1441.32) [b] 
(0.962007697261387, 1530.84) [b] 
(0.9620081493763128, 1530.86) [b] 
(0.9620086014912387, 1531.42) [b] 
(0.9620090536061645, 1531.99) [b] 
(0.9620104099509419, 1556.83) [b] 
(0.9620108620658677, 1561.14) [b] 
(0.9620113141807936, 1561.29) [b] 
(0.9620242986643357, 1562.99) [b] 
(0.9620247507792615, 1564.39) [b] 
(0.9620377352628037, 1651.84) [b] 
(0.9620386394926553, 1695.43) [b] 
(0.9620411509081804, 1700.18) [b] 
(0.9620413792186827, 1700.29) [b] 
(0.9620414755569704, 1746.56) [b] 
(0.9620471833195274, 1754.41) [b] 
(0.9620496947350525, 1754.49) [b] 
(0.9620524344610799, 1754.57) [b] 
(0.9620531193925868, 1754.71) [b] 
(0.9620551741871073, 1755.46) [b] 
(0.962056544050121, 1756.78) [b] 
(0.9620583705341392, 1757.07) [b] 
(0.9620590554656461, 1761.19) [b] 
(0.9620597756880508, 1811.62) [b] 
(0.9621005454206013, 1815.16) [b] 
(0.9622228546182529, 1815.17) [b] 
(0.9622636243508034, 1815.39) [b] 
(0.9623043940833539, 1820.34) [b] 
(0.962319837860284, 1851.82) [b] 
(0.962335281637214, 1851.83) [b] 
(0.9623355099477163, 1921.66) [b] 
(0.9623373364317346, 1970.32) [b] 
(0.9623380213632414, 1970.39) [b] 
(0.962338477984246, 1970.66) [b] 
(0.9623393912262551, 1970.79) [b] 
(0.9623403044682642, 1972.25) [b] 
(0.9623405327787665, 1980.49) [b] 
(0.9623412177102734, 2016.1) [b] 
(0.962341314048561, 2075.89) [b] 
(0.9623419989800679, 2218.5) [b] 
(0.9623422272905702, 2285.16) [b] 
(0.9623426839115747, 2285.56) [b] 
(0.9623431405325793, 2286.2) [b] 
(0.9624200991288545, 2440.28) [b] 
(0.9624534183200423, 2457.52) [b] 
(0.9624556395994548, 2457.53) [b] 
(0.9624578608788673, 2457.55) [b] 
(0.9624689672759299, 2464.93) [b] 
(0.9624734098347549, 2464.94) [b] 
(0.96247785239358, 2464.96) [b] 
(0.9624800736729925, 2465.08) [b] 
(0.962482294952405, 2465.09) [b] 
(0.9624845162318175, 2465.35) [b] 
(0.9624847445423198, 2468.15) [b] 
(0.9624877745423501, 2601.69) [b] 
(0.9624908045423805, 2602.6) [b] 
(0.9624938345424108, 2607.15) [b] 
(0.9624945547648155, 2758.15) [b] 
(0.9624952749872202, 2758.2) [b] 
(0.9624959952096249, 2758.61) [b] 
(0.9624967154320296, 2758.71) [b] 
(0.9624974356544344, 2759.21) [b] 
(0.9625133396367914, 3021.15) [b] 
(0.9625135679472937, 3032.95) [b] 
(0.962514476927197, 3269.75) [b] 
(0.9625175656825831, 3348.77) [b] 
(0.9625180223035876, 3379.06) [b] 
(0.9625200770981082, 3383.56) [b] 
(0.9625205337191127, 3384.79) [b] 
(0.9625209903401173, 3395.65) [b] 
(0.9625212186506196, 3398.79) [b] 
(0.962522360203131, 3407.7) [b] 
},{(0.9581894246575344, 0.001) [c] 
(0.9581894246575344, 4.454787273972602) [c] 
(0.9581894246575344, 3600) [c] 
}}}{legend pos=north west}}
	% \subfloat[depth=8]{\input{src/tables/xscerror8.tex}}
	% \subfloat[depth=9]{\input{src/tables/xscerror9.tex}}
	% \subfloat[depth=10]{\cactus{Average Accuracy}{CPU time}{\budalg, \murtree, \cart}{{{(0.9139377867048767, 0) [a] 
(0.918538426817828, 0.001) [a] 
(0.9186932213383758, 0.002) [a] 
(0.9240326836595572, 0.003) [a] 
(0.9241710398239407, 0.004) [a] 
(0.9252761696540128, 0.005) [a] 
(0.9297644542139507, 0.006) [a] 
(0.9298302076386084, 0.007) [a] 
(0.9298726733920331, 0.008) [a] 
(0.9336982027274873, 0.009) [a] 
(0.937360942453515, 0.01) [a] 
(0.9373965588918711, 0.011) [a] 
(0.9373979287548848, 0.012) [a] 
(0.937410257522008, 0.013) [a] 
(0.9407229953754571, 0.014) [a] 
(0.943683379566923, 0.015) [a] 
(0.9437039275121285, 0.018) [a] 
(0.943714886416238, 0.019) [a] 
(0.9452148864162379, 0.02) [a] 
(0.9452203658682926, 0.021) [a] 
(0.9452655713477447, 0.022) [a] 
(0.9452874891559638, 0.023) [a] 
(0.9453107768271967, 0.024) [a] 
(0.945313516553224, 0.025) [a] 
(0.9453162562792514, 0.029) [a] 
(0.9470646124436348, 0.03) [a] 
(0.9470659823066485, 0.031) [a] 
(0.9509610507997992, 0.04) [a] 
(0.9509966672381553, 0.047) [a] 
(0.9510322836765115, 0.048) [a] 
(0.9523473065075618, 0.05) [a] 
(0.9523651147267399, 0.052) [a] 
(0.952400731165096, 0.053) [a] 
(0.9532815530829043, 0.058) [a] 
(0.954538722032676, 0.06) [a] 
(0.954542831621717, 0.062) [a] 
(0.9545551603888404, 0.067) [a] 
(0.9545811877861007, 0.068) [a] 
(0.954588722032676, 0.07) [a] 
(0.9545955713477444, 0.073) [a] 
(0.9546092699778814, 0.077) [a] 
(0.9546325576491144, 0.078) [a] 
(0.9546448864162377, 0.079) [a] 
(0.9546788590189775, 0.08) [a] 
(0.9546898179230872, 0.085) [a] 
(0.9546911877861008, 0.087) [a] 
(0.9547899549093884, 0.09) [a] 
(0.9547913247724021, 0.093) [a] 
(0.9548036535395255, 0.095) [a] 
(0.9548050234025391, 0.096) [a] 
(0.9548406398408953, 0.098) [a] 
(0.954842009703909, 0.099) [a] 
(0.9548810507997993, 0.1) [a] 
(0.9548920097039089, 0.106) [a] 
(0.9549043384710322, 0.107) [a] 
(0.9549057083340459, 0.109) [a] 
(0.9549213247724022, 0.11) [a] 
(0.9549692699778817, 0.119) [a] 
(0.9549814617587036, 0.12) [a] 
(0.9549828316217173, 0.121) [a] 
(0.954984201484731, 0.122) [a] 
(0.9549965302518543, 0.123) [a] 
(0.954997900114868, 0.128) [a] 
(0.9550013247724022, 0.13) [a] 
(0.9550054343614433, 0.131) [a] 
(0.9550068042244569, 0.134) [a] 
(0.9550081740874706, 0.137) [a] 
(0.9550232425806213, 0.138) [a] 
(0.9550570781970597, 0.14) [a] 
(0.9550584480600733, 0.143) [a] 
(0.955059817923087, 0.146) [a] 
(0.9550776261422651, 0.147) [a] 
(0.9550981740874706, 0.149) [a] 
(0.9551003658682925, 0.15) [a] 
(0.9551044754573336, 0.151) [a] 
(0.955107215183361, 0.152) [a] 
(0.9551085850463746, 0.153) [a] 
(0.9551099549093883, 0.156) [a] 
(0.9551455713477446, 0.157) [a] 
(0.9551484480600734, 0.16) [a] 
(0.9551498179230871, 0.162) [a] 
(0.9551607768271967, 0.163) [a] 
(0.9551621466902104, 0.164) [a] 
(0.9551635165532241, 0.167) [a] 
(0.9551658453203474, 0.17) [a] 
(0.9551685850463748, 0.173) [a] 
(0.9552288590189777, 0.174) [a] 
(0.9552466672381558, 0.175) [a] 
(0.9552480371011695, 0.176) [a] 
(0.9552494069641831, 0.177) [a] 
(0.9552503658682927, 0.18) [a] 
(0.9552517357313064, 0.181) [a] 
(0.9552531055943201, 0.187) [a] 
(0.9552539275121282, 0.19) [a] 
(0.9552552973751419, 0.192) [a] 
(0.9552566672381556, 0.198) [a] 
(0.9552676261422652, 0.199) [a] 
(0.9552692699778816, 0.2) [a] 
(0.9552706398408953, 0.204) [a] 
(0.955272009703909, 0.207) [a] 
(0.9552761192929501, 0.209) [a] 
(0.9552774891559638, 0.21) [a] 
(0.9552802288819912, 0.213) [a] 
(0.9553144754573337, 0.22) [a] 
(0.9553158453203474, 0.222) [a] 
(0.9553172151833611, 0.225) [a] 
(0.9553185850463748, 0.229) [a] 
(0.9553524206628132, 0.23) [a] 
(0.9553537905258269, 0.236) [a] 
(0.9553626946354159, 0.24) [a] 
(0.9553640644984296, 0.249) [a] 
(0.9553715987450049, 0.25) [a] 
(0.9553784480600733, 0.256) [a] 
(0.955379817923087, 0.257) [a] 
(0.9553963932655527, 0.26) [a] 
(0.9553977631285664, 0.268) [a] 
(0.9554500918956896, 0.27) [a] 
(0.9554514617587033, 0.274) [a] 
(0.955502831621717, 0.28) [a] 
(0.9555206398408951, 0.3) [a] 
(0.9555233795669225, 0.303) [a] 
(0.9555247494299361, 0.306) [a] 
(0.957987150691689, 0.31) [a] 
(0.9579885205547027, 0.312) [a] 
(0.9579918082259355, 0.32) [a] 
(0.9579931780889492, 0.329) [a] 
(0.9580086575410041, 0.33) [a] 
(0.9580233150752506, 0.35) [a] 
(0.9580246849382643, 0.352) [a] 
(0.9580424931574424, 0.358) [a] 
(0.9580438630204561, 0.359) [a] 
(0.9580459178149766, 0.36) [a] 
(0.9580472876779903, 0.365) [a] 
(0.9580479726094971, 0.37) [a] 
(0.9580493424725108, 0.372) [a] 
(0.9580507123355245, 0.375) [a] 
(0.9580696164451136, 0.38) [a] 
(0.9580709863081273, 0.381) [a] 
(0.958072356171141, 0.384) [a] 
(0.9581304383629218, 0.39) [a] 
(0.9581318082259355, 0.394) [a] 
(0.9581441369930588, 0.397) [a] 
(0.9581763287738806, 0.4) [a] 
(0.9582009863081272, 0.402) [a] 
(0.9582242739793602, 0.403) [a] 
(0.9582256438423739, 0.404) [a] 
(0.9582283835684012, 0.406) [a] 
(0.9582297534314149, 0.408) [a] 
(0.958277561650593, 0.41) [a] 
(0.9582789315136067, 0.411) [a] 
(0.958281671239634, 0.412) [a] 
(0.9582830411026477, 0.418) [a] 
(0.9583798904177163, 0.42) [a] 
(0.95838126028073, 0.424) [a] 
(0.9584356438423738, 0.43) [a] 
(0.9584370137053875, 0.438) [a] 
(0.9584856438423739, 0.44) [a] 
(0.9585361917875794, 0.45) [a] 
(0.9612960818757843, 0.465) [a] 
(0.9613052599579761, 0.47) [a] 
(0.9613312873552364, 0.476) [a] 
(0.9613323832456474, 0.48) [a] 
(0.9613337531086611, 0.487) [a] 
(0.9642625369235205, 0.49) [a] 
(0.9642803451426986, 0.491) [a] 
(0.9642811670605068, 0.5) [a] 
(0.9643300711700957, 0.51) [a] 
(0.9644039067865341, 0.52) [a] 
(0.9644340437728355, 0.525) [a] 
(0.9644477424029725, 0.536) [a] 
(0.9644759615810546, 0.55) [a] 
(0.9644766465125615, 0.57) [a] 
(0.9644780163755752, 0.575) [a] 
(0.9645137698002327, 0.58) [a] 
(0.9645334958276299, 0.59) [a] 
(0.964551304046808, 0.596) [a] 
(0.9645526739098217, 0.598) [a] 
(0.9645885643207807, 0.6) [a] 
(0.9646063725399587, 0.601) [a] 
(0.964607194457767, 0.61) [a] 
(0.9646085643207807, 0.614) [a] 
(0.9646099341837944, 0.62) [a] 
(0.9646103451426985, 0.63) [a] 
(0.9646108930879039, 0.64) [a] 
(0.9646117150057121, 0.65) [a] 
(0.9646888382933834, 0.66) [a] 
(0.9647070574714656, 0.67) [a] 
(0.9647248656906436, 0.678) [a] 
(0.9647262355536573, 0.705) [a] 
(0.964727605416671, 0.71) [a] 
(0.9647280163755751, 0.72) [a] 
(0.9648073314440683, 0.73) [a] 
(0.964807605416671, 0.81) [a] 
(0.9648080163755751, 0.82) [a] 
(0.9648085643207805, 0.84) [a] 
(0.9648177424029724, 0.85) [a] 
(0.9648178793892738, 0.88) [a] 
(0.9648534958276299, 0.885) [a] 
(0.9648536328139313, 0.9) [a] 
(0.9648814410331095, 0.91) [a] 
(0.9648817150057122, 0.93) [a] 
(0.964882399937219, 0.95) [a] 
(0.9648947287043423, 0.959) [a] 
(0.9649080163755752, 0.96) [a] 
(0.9649082903481779, 0.98) [a] 
(0.9649093862385889, 0.99) [a] 
(0.9649103451426985, 1) [a] 
(0.9649156876084518, 1.01) [a] 
(0.9649170574714655, 1.063) [a] 
(0.9649184273344792, 1.08) [a] 
(0.9655319178082182, 1.13) [a] 
(0.9657178082191771, 1.14) [a] 
(0.9657180821917798, 1.16) [a] 
(0.9657379452054784, 1.17) [a] 
(0.9657434246575332, 1.205) [a] 
(0.9657557534246564, 1.206) [a] 
(0.9657626027397248, 1.207) [a] 
(0.9657680821917796, 1.22) [a] 
(0.9657772602739714, 1.24) [a] 
(0.9658142465753412, 1.28) [a] 
(0.965823424657533, 1.29) [a] 
(0.9658247945205467, 1.344) [a] 
(0.9658275342465741, 1.376) [a] 
(0.9658357534246562, 1.377) [a] 
(0.9658426027397247, 1.378) [a] 
(0.965845342465752, 1.384) [a] 
(0.9658547945205466, 1.39) [a] 
(0.9658731506849302, 1.4) [a] 
(0.9658854794520535, 1.401) [a] 
(0.9658991780821905, 1.402) [a] 
(0.9659115068493138, 1.404) [a] 
(0.9659208219178069, 1.41) [a] 
(0.9659299999999987, 1.42) [a] 
(0.9659342465753412, 1.46) [a] 
(0.9659393150684918, 1.47) [a] 
(0.9659484931506837, 1.48) [a] 
(0.9659498630136973, 1.497) [a] 
(0.965951232876711, 1.506) [a] 
(0.9659526027397247, 1.519) [a] 
(0.9659568493150672, 1.52) [a] 
(0.965957534246574, 1.72) [a] 
(0.9659589041095877, 1.738) [a] 
(0.9659602739726014, 1.741) [a] 
(0.9659616438356151, 1.747) [a] 
(0.9659630136986288, 1.762) [a] 
(0.9659635616438342, 1.77) [a] 
(0.9659639726027384, 1.78) [a] 
(0.965965342465752, 1.797) [a] 
(0.9659673972602726, 1.81) [a] 
(0.9659687671232863, 1.816) [a] 
(0.9659730136986288, 1.82) [a] 
(0.9659743835616424, 1.879) [a] 
(0.9659757534246561, 1.892) [a] 
(0.9659894520547931, 1.907) [a] 
(0.9660017808219165, 1.91) [a] 
(0.9660031506849301, 1.911) [a] 
(0.9660045205479438, 1.913) [a] 
(0.9660058904109575, 1.916) [a] 
(0.9660072602739712, 1.92) [a] 
(0.9660086301369849, 1.93) [a] 
(0.9660117808219164, 1.94) [a] 
(0.9660131506849301, 1.948) [a] 
(0.9660135616438342, 1.95) [a] 
(0.9660149315068479, 1.954) [a] 
(0.9660163013698616, 1.956) [a] 
(0.9660176712328753, 1.961) [a] 
(0.966019041095889, 1.966) [a] 
(0.9660217808219164, 1.967) [a] 
(0.96602315068493, 1.968) [a] 
(0.9660258904109574, 1.974) [a] 
(0.9660275342465738, 1.98) [a] 
(0.9660289041095875, 1.982) [a] 
(0.9660302739726012, 1.987) [a] 
(0.9660330136986286, 1.992) [a] 
(0.9660343835616423, 1.995) [a] 
(0.9660526027397245, 2) [a] 
(0.9660539726027382, 2.006) [a] 
(0.9660567123287656, 2.013) [a] 
(0.9660580821917792, 2.056) [a] 
(0.9660594520547929, 2.091) [a] 
(0.9660608219178066, 2.122) [a] 
(0.9660621917808203, 2.13) [a] 
(0.966063561643834, 2.132) [a] 
(0.9660643835616423, 2.23) [a] 
(0.9660812328767108, 2.28) [a] 
(0.9660980821917793, 2.51) [a] 
(0.966099452054793, 2.565) [a] 
(0.9661104109589026, 2.572) [a] 
(0.9661117808219163, 2.583) [a] 
(0.9661145205479437, 2.591) [a] 
(0.9661199999999984, 2.592) [a] 
(0.9661241095890395, 2.593) [a] 
(0.9661843835616423, 2.668) [a] 
(0.9661953424657519, 2.671) [a] 
(0.9662076712328752, 2.673) [a] 
(0.9662199999999985, 2.674) [a] 
(0.966236849315067, 2.69) [a] 
(0.9662491780821904, 2.702) [a] 
(0.9662601369863, 2.704) [a] 
(0.9662615068493137, 2.714) [a] 
(0.9662628767123274, 2.717) [a] 
(0.966264246575341, 2.721) [a] 
(0.9662656164383547, 2.725) [a] 
(0.9662669863013684, 2.729) [a] 
(0.9662697260273958, 2.732) [a] 
(0.9662699999999985, 2.82) [a] 
(0.9662878082191766, 2.88) [a] 
(0.9662880821917793, 2.89) [a] 
(0.9663154794520533, 2.93) [a] 
(0.9663428767123273, 2.94) [a] 
(0.9663565753424643, 2.96) [a] 
(0.9663567123287657, 2.98) [a] 
(0.9663842465753412, 2.99) [a] 
(0.9663979452054782, 3) [a] 
(0.9664116438356152, 3.02) [a] 
(0.9664119178082179, 3.11) [a] 
(0.9664132876712316, 3.147) [a] 
(0.9664242465753411, 3.148) [a] 
(0.9664352054794506, 3.149) [a] 
(0.9664406849315054, 3.15) [a] 
(0.9664447945205464, 3.162) [a] 
(0.9664502739726012, 3.17) [a] 
(0.966455753424656, 3.179) [a] 
(0.9664776712328751, 3.18) [a] 
(0.9664872602739709, 3.181) [a] 
(0.9664927397260257, 3.188) [a] 
(0.9664982191780804, 3.198) [a] 
(0.9665036986301352, 3.228) [a] 
(0.9665050684931489, 3.254) [a] 
(0.9665064383561626, 3.257) [a] 
(0.9665219178082174, 3.4) [a] 
(0.9665287671232858, 3.401) [a] 
(0.9665315068493132, 3.409) [a] 
(0.9665342465753406, 3.41) [a] 
(0.966536986301368, 3.411) [a] 
(0.9665465753424639, 3.412) [a] 
(0.9665493150684913, 3.413) [a] 
(0.966554794520546, 3.415) [a] 
(0.9665553424657515, 3.57) [a] 
(0.9665560273972583, 3.59) [a] 
(0.9665568493150666, 3.6) [a] 
(0.9665705479452036, 3.75) [a] 
(0.9665715068493131, 3.8) [a] 
(0.9665728767123268, 3.838) [a] 
(0.9665742465753405, 3.85) [a] 
(0.9665756164383542, 3.864) [a] 
(0.9665769863013679, 3.904) [a] 
(0.9665783561643816, 3.914) [a] 
(0.9665797260273953, 3.917) [a] 
(0.966581095890409, 3.919) [a] 
(0.9665824657534227, 3.948) [a] 
(0.9665827397260254, 4.01) [a] 
(0.9665828767123268, 4.02) [a] 
(0.9665842465753405, 4.054) [a] 
(0.9665856164383542, 4.096) [a] 
(0.9665857534246556, 4.24) [a] 
(0.9665871232876693, 4.272) [a] 
(0.966588493150683, 4.343) [a] 
(0.9665898630136966, 4.512) [a] 
(0.9665904109589021, 4.53) [a] 
(0.9665917808219158, 4.544) [a] 
(0.9665931506849295, 4.574) [a] 
(0.9665945205479431, 4.624) [a] 
(0.9666123287671212, 4.645) [a] 
(0.9666127397260253, 4.67) [a] 
(0.9666305479452034, 4.686) [a] 
(0.9666319178082171, 4.849) [a] 
(0.9666320547945185, 4.85) [a] 
(0.9666443835616418, 4.874) [a] 
(0.9666487671232856, 5.16) [a] 
(0.9666542465753404, 5.467) [a] 
(0.9666569863013678, 5.489) [a] 
(0.9666583561643814, 5.498) [a] 
(0.9666597260273951, 5.518) [a] 
(0.9666610958904088, 5.532) [a] 
(0.9666624657534225, 5.544) [a] 
(0.9666638356164362, 5.556) [a] 
(0.9666652054794499, 5.57) [a] 
(0.9666665753424636, 5.577) [a] 
(0.9666679452054773, 5.582) [a] 
(0.966669315068491, 5.607) [a] 
(0.966673424657532, 5.615) [a] 
(0.9666747945205457, 5.657) [a] 
(0.9666761643835594, 5.66) [a] 
(0.9666775342465731, 5.665) [a] 
(0.9666789041095868, 5.685) [a] 
(0.9666802739726005, 5.707) [a] 
(0.9666816438356142, 5.716) [a] 
(0.9666830136986279, 5.737) [a] 
(0.9666843835616415, 5.746) [a] 
(0.9666857534246552, 5.793) [a] 
(0.9666871232876689, 5.801) [a] 
(0.9666884931506826, 5.803) [a] 
(0.9666898630136963, 5.843) [a] 
(0.96669123287671, 5.855) [a] 
(0.9667035616438333, 5.865) [a] 
(0.9667268493150663, 5.866) [a] 
(0.9667282191780799, 5.877) [a] 
(0.9667295890410936, 5.888) [a] 
(0.9667309589041073, 5.91) [a] 
(0.966732328767121, 5.92) [a] 
(0.9667336986301347, 5.934) [a] 
(0.9667350684931484, 5.939) [a] 
(0.9667364383561621, 5.979) [a] 
(0.9667378082191758, 6.026) [a] 
(0.9667391780821895, 6.029) [a] 
(0.9667405479452031, 6.047) [a] 
(0.9667460273972579, 6.102) [a] 
(0.9667693150684908, 6.181) [a] 
(0.9667939726027375, 6.182) [a] 
(0.9668049315068471, 6.187) [a] 
(0.9668172602739704, 6.222) [a] 
(0.9668227397260252, 6.243) [a] 
(0.9668460273972581, 6.246) [a] 
(0.9668630136986279, 6.25) [a] 
(0.9668672602739704, 6.35) [a] 
(0.9668795890410937, 6.453) [a] 
(0.9668809589041074, 6.625) [a] 
(0.9668979452054772, 6.8) [a] 
(0.9669020547945183, 6.809) [a] 
(0.9669047945205457, 6.814) [a] 
(0.966907534246573, 6.86) [a] 
(0.9669102739726004, 6.914) [a] 
(0.9669143835616415, 6.916) [a] 
(0.9669171232876689, 6.927) [a] 
(0.9669184931506826, 7.024) [a] 
(0.9669198630136963, 7.032) [a] 
(0.96692123287671, 7.035) [a] 
(0.9669226027397236, 7.049) [a] 
(0.9669394520547921, 7.19) [a] 
(0.9669408219178058, 7.294) [a] 
(0.9669531506849292, 7.328) [a] 
(0.9669654794520525, 7.343) [a] 
(0.9669668493150662, 7.385) [a] 
(0.9669682191780798, 7.417) [a] 
(0.9669691780821894, 7.42) [a] 
(0.9669706849315045, 7.44) [a] 
(0.9669734246575319, 7.543) [a] 
(0.9669747945205456, 7.648) [a] 
(0.9669790410958881, 7.7) [a] 
(0.9669791780821895, 7.76) [a] 
(0.9669795890410936, 7.8) [a] 
(0.9669802739726004, 7.85) [a] 
(0.966997123287669, 7.86) [a] 
(0.9669975342465731, 7.87) [a] 
(0.9669978082191758, 7.89) [a] 
(0.9669980821917785, 7.91) [a] 
(0.9669982191780799, 7.92) [a] 
(0.9669995890410936, 7.99) [a] 
(0.9670009589041073, 8.02) [a] 
(0.967002328767121, 8.028) [a] 
(0.9670036986301347, 8.058) [a] 
(0.9670050684931484, 8.083) [a] 
(0.9670064383561621, 8.09) [a] 
(0.9670065753424635, 8.15) [a] 
(0.9670079452054772, 8.164) [a] 
(0.9670093150684909, 8.179) [a] 
(0.9670094520547923, 8.18) [a] 
(0.967010821917806, 8.204) [a] 
(0.9670109589041074, 8.25) [a] 
(0.9670123287671211, 8.295) [a] 
(0.9670136986301348, 8.39) [a] 
(0.9670139726027375, 8.43) [a] 
(0.9670142465753402, 8.45) [a] 
(0.9670156164383539, 8.664) [a] 
(0.9670183561643813, 8.893) [a] 
(0.9670594520547923, 9.04) [a] 
(0.9670868493150663, 9.25) [a] 
(0.9671005479452033, 9.28) [a] 
(0.967100821917806, 9.65) [a] 
(0.9671035616438334, 9.733) [a] 
(0.9671039726027375, 9.97) [a] 
(0.9671041095890389, 9.98) [a] 
(0.9671042465753403, 10.05) [a] 
(0.967105616438354, 10.23) [a] 
(0.9671069863013677, 10.29) [a] 
(0.9671124657534225, 10.44) [a] 
(0.9671217808219156, 10.51) [a] 
(0.9671221917808197, 10.56) [a] 
(0.9671238356164361, 10.59) [a] 
(0.9671252054794498, 10.76) [a] 
(0.9671254794520525, 11.34) [a] 
(0.9671256164383539, 11.63) [a] 
(0.9671258904109566, 11.64) [a] 
(0.9671299999999977, 12.1) [a] 
(0.9671391780821895, 12.25) [a] 
(0.9671405479452032, 12.5) [a] 
(0.9671427397260252, 12.55) [a] 
(0.9671430136986279, 12.56) [a] 
(0.967143424657532, 12.58) [a] 
(0.9671436986301347, 12.59) [a] 
(0.9671439726027374, 12.67) [a] 
(0.9671442465753401, 12.74) [a] 
(0.9671443835616416, 12.84) [a] 
(0.9671457534246553, 12.96) [a] 
(0.9671467123287648, 13.64) [a] 
(0.9671473972602717, 13.68) [a] 
(0.9671475342465731, 13.69) [a] 
(0.9671478082191758, 13.7) [a] 
(0.9671480821917785, 13.71) [a] 
(0.9671482191780799, 13.73) [a] 
(0.9671489041095868, 13.75) [a] 
(0.9671495890410936, 13.76) [a] 
(0.967150136986299, 13.79) [a] 
(0.9671508219178059, 13.84) [a] 
(0.9671513698630113, 14.01) [a] 
(0.9671563013698606, 14.02) [a] 
(0.9672332876712305, 14.57) [a] 
(0.9672339726027374, 14.96) [a] 
(0.9672494520547922, 14.98) [a] 
(0.9672504109589017, 14.99) [a] 
(0.9672505479452032, 15.02) [a] 
(0.9672508219178059, 15.03) [a] 
(0.9672510958904086, 15.05) [a] 
(0.96725123287671, 15.07) [a] 
(0.9672515068493127, 15.27) [a] 
(0.9672516438356141, 15.6) [a] 
(0.9672521917808196, 15.61) [a] 
(0.967252328767121, 15.63) [a] 
(0.9672526027397237, 15.64) [a] 
(0.9672528767123264, 15.66) [a] 
(0.9672530136986278, 15.69) [a] 
(0.9672532876712305, 15.78) [a] 
(0.9672535616438332, 15.83) [a] 
(0.967253835616436, 15.84) [a] 
(0.9672542465753401, 15.9) [a] 
(0.9672545205479428, 15.93) [a] 
(0.9672546575342442, 16.32) [a] 
(0.9672561643835593, 16.42) [a] 
(0.967256438356162, 17.08) [a] 
(0.9672657534246552, 17.27) [a] 
(0.9672663013698606, 17.49) [a] 
(0.9672799999999976, 17.54) [a] 
(0.9672936986301346, 17.95) [a] 
(0.9673073972602716, 18.01) [a] 
(0.9673210958904086, 18.38) [a] 
(0.9673347945205456, 18.49) [a] 
(0.9673350684931483, 18.92) [a] 
(0.9673354794520524, 18.93) [a] 
(0.9673509589041073, 18.94) [a] 
(0.9673601369862991, 19.11) [a] 
(0.9673602739726005, 19.12) [a] 
(0.9673694520547923, 19.14) [a] 
(0.9673913698630114, 19.31) [a] 
(0.9673927397260251, 19.39) [a] 
(0.9673941095890388, 19.5) [a] 
(0.9674119178082168, 19.79) [a] 
(0.9674187671232852, 19.8) [a] 
(0.9674269863013674, 19.84) [a] 
(0.9674297260273947, 19.85) [a] 
(0.9674390410958879, 19.98) [a] 
(0.9674404109589015, 20.04) [a] 
(0.9674431506849289, 20.07) [a] 
(0.9674445205479426, 20.08) [a] 
(0.967444657534244, 20.12) [a] 
(0.9674460273972577, 20.14) [a] 
(0.9674528767123262, 20.49) [a] 
(0.9674556164383535, 20.5) [a] 
(0.9674583561643809, 20.51) [a] 
(0.9674675342465727, 20.57) [a] 
(0.9674730136986275, 20.94) [a] 
(0.9674883561643809, 21.17) [a] 
(0.9675020547945179, 21.72) [a] 
(0.9675157534246549, 21.74) [a] 
(0.9675294520547919, 21.75) [a] 
(0.967529863013696, 21.87) [a] 
(0.9675302739726002, 21.96) [a] 
(0.9675313698630111, 22.08) [a] 
(0.9675316438356139, 22.09) [a] 
(0.9675326027397234, 22.14) [a] 
(0.9675327397260248, 22.22) [a] 
(0.9675330136986275, 22.26) [a] 
(0.967549863013696, 22.62) [a] 
(0.9675567123287645, 23.36) [a] 
(0.9675569863013672, 23.51) [a] 
(0.9675573972602713, 23.52) [a] 
(0.9675742465753399, 25.06) [a] 
(0.9675747945205453, 26.38) [a] 
(0.9675790410958878, 26.52) [a] 
(0.9675794520547919, 26.68) [a] 
(0.967579863013696, 26.7) [a] 
(0.9675804109589015, 26.83) [a] 
(0.9675941095890385, 26.94) [a] 
(0.9675949315068467, 27.04) [a] 
(0.9676086301369837, 27.29) [a] 
(0.9676087671232851, 27.34) [a] 
(0.9676224657534221, 28.85) [a] 
(0.9676267123287646, 29.25) [a] 
(0.9676310958904084, 30.15) [a] 
(0.9676313698630111, 30.6) [a] 
(0.9676316438356138, 31.29) [a] 
(0.9676317808219153, 31.3) [a] 
(0.9676327397260248, 31.31) [a] 
(0.9676338356164358, 31.32) [a] 
(0.9676341095890385, 31.33) [a] 
(0.9676343835616412, 31.81) [a] 
(0.9676357534246549, 32.11) [a] 
(0.9676364383561618, 32.47) [a] 
(0.9676367123287645, 33.02) [a] 
(0.9676373972602713, 33.04) [a] 
(0.9676378082191754, 33.05) [a] 
(0.9676383561643809, 33.11) [a] 
(0.9676384931506823, 34.13) [a] 
(0.967638767123285, 34.14) [a] 
(0.9676394520547918, 34.16) [a] 
(0.9676397260273946, 34.17) [a] 
(0.9676402739726, 34.68) [a] 
(0.9676406849315041, 34.69) [a] 
(0.9676512328767096, 34.82) [a] 
(0.9676520547945179, 34.85) [a] 
(0.967653972602737, 34.86) [a] 
(0.9676547945205453, 35.15) [a] 
(0.9676571232876685, 35.2) [a] 
(0.9676620547945178, 35.41) [a] 
(0.9676630136986274, 35.43) [a] 
(0.9676634246575315, 35.44) [a] 
(0.9676867123287645, 35.63) [a] 
(0.9676990410958878, 35.66) [a] 
(0.9676997260273946, 36.31) [a] 
(0.9677001369862988, 36.32) [a] 
(0.9677771232876686, 36.51) [a] 
(0.9677773972602713, 36.85) [a] 
(0.9677783561643809, 36.92) [a] 
(0.9677938356164357, 38.44) [a] 
(0.9678093150684905, 38.47) [a] 
(0.9678120547945179, 38.64) [a] 
(0.9678257534246549, 38.72) [a] 
(0.9678394520547919, 38.77) [a] 
(0.9678408219178056, 38.93) [a] 
(0.9678613698630111, 39.36) [a] 
(0.9678639726027372, 39.37) [a] 
(0.9678667123287645, 39.67) [a] 
(0.967873561643833, 39.68) [a] 
(0.9678763013698604, 40.91) [a] 
(0.9678764383561618, 41.26) [a] 
(0.9678767123287645, 41.36) [a] 
(0.9678769863013672, 41.37) [a] 
(0.9678771232876686, 41.4) [a] 
(0.9678773972602713, 41.42) [a] 
(0.9678775342465727, 41.8) [a] 
(0.9678778082191755, 41.81) [a] 
(0.9678780821917782, 41.82) [a] 
(0.9678782191780796, 41.84) [a] 
(0.9678784931506823, 41.86) [a] 
(0.9678939726027371, 42.01) [a] 
(0.9678942465753398, 42.07) [a] 
(0.9678943835616413, 42.13) [a] 
(0.967894657534244, 42.24) [a] 
(0.9678949315068467, 42.29) [a] 
(0.9678952054794494, 43.36) [a] 
(0.9679089041095864, 44.78) [a] 
(0.9679102739726001, 44.84) [a] 
(0.9679116438356138, 44.85) [a] 
(0.9679130136986275, 44.87) [a] 
(0.9679143835616412, 44.95) [a] 
(0.9679157534246549, 45.35) [a] 
(0.9679171232876685, 46.2) [a] 
(0.9679184931506822, 46.44) [a] 
(0.9679198630136959, 46.76) [a] 
(0.9679202739726, 46.77) [a] 
(0.9679220547945179, 46.78) [a] 
(0.9679230136986274, 46.79) [a] 
(0.967923972602737, 46.8) [a] 
(0.9679243835616411, 46.82) [a] 
(0.9679247945205453, 46.83) [a] 
(0.9679253424657507, 46.88) [a] 
(0.9679261643835589, 46.9) [a] 
(0.9679275342465726, 46.93) [a] 
(0.9679298630136959, 46.95) [a] 
(0.9679302739726, 46.96) [a] 
(0.9679316438356137, 46.98) [a] 
(0.9679317808219151, 47.03) [a] 
(0.9679323287671205, 47.04) [a] 
(0.9679327397260247, 47.07) [a] 
(0.9679336986301342, 47.1) [a] 
(0.9679387671232849, 47.11) [a] 
(0.9679419178082164, 47.12) [a] 
(0.9679436986301342, 47.13) [a] 
(0.9679442465753396, 47.14) [a] 
(0.9679450684931479, 47.15) [a] 
(0.967946986301367, 47.16) [a] 
(0.9679491780821889, 47.17) [a] 
(0.9679510958904081, 47.18) [a] 
(0.9679524657534218, 47.19) [a] 
(0.96795328767123, 47.23) [a] 
(0.9679546575342437, 47.24) [a] 
(0.9679573972602712, 47.25) [a] 
(0.967960684931504, 47.26) [a] 
(0.9679610958904081, 47.29) [a] 
(0.9679615068493123, 47.3) [a] 
(0.9679620547945177, 47.44) [a] 
(0.9679631506849287, 47.46) [a] 
(0.9679636986301341, 47.48) [a] 
(0.9679650684931478, 47.49) [a] 
(0.9679664383561615, 47.5) [a] 
(0.9679682191780793, 47.51) [a] 
(0.9679699999999971, 47.52) [a] 
(0.9679719178082162, 47.53) [a] 
(0.9679723287671204, 47.54) [a] 
(0.9679727397260245, 47.55) [a] 
(0.9679732876712299, 47.58) [a] 
(0.9679750684931477, 47.7) [a] 
(0.9679779452054765, 47.71) [a] 
(0.9679786301369834, 47.72) [a] 
(0.9679791780821888, 47.78) [a] 
(0.9679795890410929, 47.79) [a] 
(0.9679797260273943, 47.95) [a] 
(0.967981095890408, 48.03) [a] 
(0.9679824657534217, 48.14) [a] 
(0.9679838356164354, 48.15) [a] 
(0.9679852054794491, 48.16) [a] 
(0.9679871232876682, 48.17) [a] 
(0.9679875342465724, 48.18) [a] 
(0.9679884931506819, 48.2) [a] 
(0.967988904109586, 48.21) [a] 
(0.9679912328767093, 48.22) [a] 
(0.967992602739723, 48.23) [a] 
(0.9679930136986271, 48.24) [a] 
(0.9679934246575312, 48.25) [a] 
(0.9679943835616408, 48.26) [a] 
(0.9679947945205449, 48.27) [a] 
(0.9679953424657504, 48.5) [a] 
(0.967996712328764, 49.16) [a] 
(0.9679971232876682, 49.17) [a] 
(0.9679975342465723, 49.18) [a] 
(0.9679984931506819, 49.19) [a] 
(0.9679998630136956, 49.24) [a] 
(0.9680002739725997, 49.38) [a] 
(0.9680016438356134, 49.79) [a] 
(0.9680030136986271, 49.8) [a] 
(0.9680049315068462, 49.81) [a] 
(0.968006712328764, 49.82) [a] 
(0.9680084931506818, 49.83) [a] 
(0.9680090410958873, 49.85) [a] 
(0.9680094520547914, 49.91) [a] 
(0.9680098630136955, 49.94) [a] 
(0.9680108219178051, 50.2) [a] 
(0.9680112328767092, 50.21) [a] 
(0.9680126027397229, 50.37) [a] 
(0.9680139726027366, 50.44) [a] 
(0.9680153424657503, 50.49) [a] 
(0.968016712328764, 50.51) [a] 
(0.9680172602739694, 50.55) [a] 
(0.9680180821917777, 50.56) [a] 
(0.9680194520547913, 50.6) [a] 
(0.9680199999999968, 50.61) [a] 
(0.9680204109589009, 50.75) [a] 
(0.9680213698630105, 50.92) [a] 
(0.9680249315068461, 50.93) [a] 
(0.9680258904109557, 50.94) [a] 
(0.968028219178079, 50.95) [a] 
(0.9680295890410927, 50.96) [a] 
(0.9680299999999968, 50.97) [a] 
(0.9680313698630105, 50.98) [a] 
(0.9680327397260242, 50.99) [a] 
(0.9680331506849283, 51) [a] 
(0.9680336986301338, 51.11) [a] 
(0.968034520547942, 51.22) [a] 
(0.9680350684931475, 52.17) [a] 
(0.9680354794520516, 52.18) [a] 
(0.9680358904109557, 52.23) [a] 
(0.9680363013698599, 52.45) [a] 
(0.9680365753424626, 52.46) [a] 
(0.968036712328764, 52.47) [a] 
(0.9680383561643804, 52.5) [a] 
(0.9680387671232845, 52.56) [a] 
(0.9680390410958872, 52.57) [a] 
(0.96803931506849, 52.58) [a] 
(0.9680394520547914, 52.61) [a] 
(0.9680399999999968, 52.62) [a] 
(0.9680401369862982, 52.63) [a] 
(0.9680404109589009, 52.64) [a] 
(0.9680406849315036, 52.7) [a] 
(0.9680410958904078, 52.73) [a] 
(0.9680420547945173, 52.86) [a] 
(0.9680443835616407, 52.88) [a] 
(0.9680447945205448, 52.89) [a] 
(0.9680461643835585, 52.9) [a] 
(0.9680463013698599, 53.03) [a] 
(0.9680476712328736, 53.34) [a] 
(0.9680480821917777, 53.41) [a] 
(0.9680494520547914, 53.47) [a] 
(0.9680498630136956, 53.62) [a] 
(0.9680501369862983, 53.65) [a] 
(0.968050410958901, 53.72) [a] 
(0.9680509589041064, 53.86) [a] 
(0.9680512328767091, 53.87) [a] 
(0.9680513698630105, 54.22) [a] 
(0.9680516438356133, 54.23) [a] 
(0.9680520547945174, 54.24) [a] 
(0.9680523287671201, 54.61) [a] 
(0.9680527397260242, 56.38) [a] 
(0.968053013698627, 56.41) [a] 
(0.9680536986301338, 56.48) [a] 
(0.9680539726027365, 56.5) [a] 
(0.9680541095890379, 56.52) [a] 
(0.9680543835616406, 56.55) [a] 
(0.9680546575342434, 56.59) [a] 
(0.9680547945205448, 56.6) [a] 
(0.9680552054794489, 56.61) [a] 
(0.9680563013698598, 56.63) [a] 
(0.9680564383561612, 56.66) [a] 
(0.9680573972602707, 56.72) [a] 
(0.9680576712328735, 56.75) [a] 
(0.9680578082191749, 56.76) [a] 
(0.9680580821917776, 56.79) [a] 
(0.9680583561643803, 56.81) [a] 
(0.9680584931506817, 56.87) [a] 
(0.9680587671232844, 56.88) [a] 
(0.9680590410958871, 56.9) [a] 
(0.9680591780821886, 56.92) [a] 
(0.9680594520547913, 56.94) [a] 
(0.968059726027394, 56.96) [a] 
(0.9680598630136954, 57.06) [a] 
(0.9680601369862981, 57.07) [a] 
(0.9680604109589008, 57.11) [a] 
(0.968060821917805, 57.47) [a] 
(0.9680613698630104, 57.59) [a] 
(0.9680621917808186, 57.69) [a] 
(0.9680626027397228, 57.83) [a] 
(0.9680653424657502, 58.85) [a] 
(0.9680654794520516, 58.96) [a] 
(0.9680747945205447, 59.49) [a] 
(0.9680901369862981, 59.52) [a] 
(0.9680993150684899, 60.17) [a] 
(0.9681763013698598, 60.25) [a] 
(0.9681776712328735, 60.33) [a] 
(0.9681790410958871, 60.34) [a] 
(0.9681798630136954, 61.12) [a] 
(0.9681804109589008, 61.17) [a] 
(0.9681812328767091, 61.26) [a] 
(0.9681824657534214, 61.38) [a] 
(0.9681838356164351, 61.92) [a] 
(0.9681879452054761, 62.78) [a] 
(0.9681884931506816, 64.28) [a] 
(0.9681894520547911, 64.4) [a] 
(0.9681898630136953, 64.65) [a] 
(0.968190136986298, 64.87) [a] 
(0.9681905479452021, 65.25) [a] 
(0.9681909589041062, 65.47) [a] 
(0.9681939726027363, 65.71) [a] 
(0.9681941095890377, 65.77) [a] 
(0.9681943835616404, 67.87) [a] 
(0.9681947945205446, 67.92) [a] 
(0.9681952054794487, 67.95) [a] 
(0.9681954794520514, 67.98) [a] 
(0.9681957534246541, 68.06) [a] 
(0.9681958904109556, 68.07) [a] 
(0.9681961643835583, 68.09) [a] 
(0.9681965753424624, 68.1) [a] 
(0.9682102739725994, 69.19) [a] 
(0.9682104109589008, 70.34) [a] 
(0.9682106849315035, 70.35) [a] 
(0.9682109589041062, 70.49) [a] 
(0.9682110958904077, 70.51) [a] 
(0.9682113698630104, 70.61) [a] 
(0.9682116438356131, 70.62) [a] 
(0.9682117808219145, 70.69) [a] 
(0.9682131506849282, 70.72) [a] 
(0.9682147945205446, 70.73) [a] 
(0.9682161643835583, 70.79) [a] 
(0.968216438356161, 70.8) [a] 
(0.9682178082191747, 70.83) [a] 
(0.9682179452054761, 70.95) [a] 
(0.9682182191780788, 70.96) [a] 
(0.9682213698630103, 72.14) [a] 
(0.9682223287671199, 72.15) [a] 
(0.968222739726024, 72.16) [a] 
(0.9682230136986267, 72.28) [a] 
(0.9682234246575309, 72.29) [a] 
(0.9682236986301336, 72.37) [a] 
(0.9682241095890377, 72.41) [a] 
(0.968226438356161, 72.84) [a] 
(0.9682286301369829, 72.86) [a] 
(0.9682326027397227, 73.37) [a] 
(0.9682343835616405, 73.5) [a] 
(0.9682354794520515, 73.6) [a] 
(0.9682361643835583, 73.61) [a] 
(0.968236438356161, 73.62) [a] 
(0.9682367123287637, 73.9) [a] 
(0.9682369863013665, 73.91) [a] 
(0.9682383561643801, 74.69) [a] 
(0.9682397260273938, 75.08) [a] 
(0.9682410958904075, 75.2) [a] 
(0.9682438356164349, 75.5) [a] 
(0.9682530136986267, 75.71) [a] 
(0.9682623287671198, 75.74) [a] 
(0.9682636986301335, 76.03) [a] 
(0.9682650684931472, 76.61) [a] 
(0.9682664383561609, 76.9) [a] 
(0.9682691780821883, 77.33) [a] 
(0.968270547945202, 77.42) [a] 
(0.9682712328767088, 79.22) [a] 
(0.9682726027397225, 80.42) [a] 
(0.9682739726027362, 80.52) [a] 
(0.9682753424657499, 81.21) [a] 
(0.9682767123287636, 81.66) [a] 
(0.9682808219178046, 81.99) [a] 
(0.968283561643832, 82) [a] 
(0.9682849315068457, 83.19) [a] 
(0.9682863013698594, 83.51) [a] 
(0.9682876712328731, 83.53) [a] 
(0.9682890410958868, 84.14) [a] 
(0.9682904109589004, 84.17) [a] 
(0.9682919178082156, 84.2) [a] 
(0.9682932876712292, 84.22) [a] 
(0.9682946575342429, 84.24) [a] 
(0.9682949315068456, 84.41) [a] 
(0.9682952054794484, 84.53) [a] 
(0.9682953424657498, 84.6) [a] 
(0.9682956164383525, 84.65) [a] 
(0.9682960273972566, 84.69) [a] 
(0.9683115068493114, 84.99) [a] 
(0.9683117808219142, 85.18) [a] 
(0.9683426027397224, 85.29) [a] 
(0.9683517808219142, 85.81) [a] 
(0.9683531506849279, 86.66) [a] 
(0.9683545205479416, 86.67) [a] 
(0.968357260273969, 86.68) [a] 
(0.9683586301369826, 86.69) [a] 
(0.9683599999999963, 87.66) [a] 
(0.96836136986301, 88.37) [a] 
(0.9683627397260237, 88.45) [a] 
(0.9683641095890374, 90.03) [a] 
(0.9683654794520511, 90.04) [a] 
(0.9683668493150648, 90.15) [a] 
(0.9683682191780785, 90.23) [a] 
(0.9683695890410922, 90.43) [a] 
(0.9683697260273936, 90.67) [a] 
(0.9683699999999963, 92.4) [a] 
(0.9683919178082153, 92.85) [a] 
(0.9683973972602701, 92.86) [a] 
(0.9684028767123248, 92.87) [a] 
(0.9684083561643796, 92.89) [a] 
(0.9684124657534207, 92.9) [a] 
(0.9684179452054754, 93.2) [a] 
(0.9684180821917768, 93.33) [a] 
(0.9684183561643795, 93.34) [a] 
(0.9684187671232837, 93.35) [a] 
(0.9684190410958864, 93.38) [a] 
(0.9684193150684891, 93.43) [a] 
(0.9684194520547905, 93.44) [a] 
(0.9684197260273932, 93.46) [a] 
(0.968419999999996, 93.5) [a] 
(0.9684204109589001, 93.53) [a] 
(0.9684210958904069, 93.55) [a] 
(0.9684213698630096, 93.59) [a] 
(0.9684227397260233, 93.63) [a] 
(0.9684228767123247, 93.74) [a] 
(0.9684242465753384, 95.24) [a] 
(0.9684247945205439, 95.65) [a] 
(0.968425205479448, 95.68) [a] 
(0.9684254794520507, 95.82) [a] 
(0.9684256164383521, 95.93) [a] 
(0.9684258904109548, 95.96) [a] 
(0.9684261643835576, 95.97) [a] 
(0.9684275342465712, 96) [a] 
(0.9684283561643795, 96.08) [a] 
(0.9684286301369822, 96.1) [a] 
(0.9684293150684891, 96.22) [a] 
(0.9684295890410918, 96.23) [a] 
(0.9684306849315027, 96.36) [a] 
(0.9684309589041055, 96.37) [a] 
(0.9684310958904069, 96.45) [a] 
(0.9684324657534206, 97.64) [a] 
(0.9684338356164343, 98.11) [a] 
(0.968439315068489, 99.24) [a] 
(0.9684394520547904, 102.04) [a] 
(0.9684397260273931, 102.71) [a] 
(0.9684399999999959, 103.28) [a] 
(0.9684402739725986, 103.29) [a] 
(0.9684406849315027, 103.45) [a] 
(0.9684409589041054, 103.46) [a] 
(0.9684410958904068, 103.48) [a] 
(0.968441506849311, 103.92) [a] 
(0.9684428767123247, 106.4) [a] 
(0.9684431506849274, 106.42) [a] 
(0.9684434246575301, 107.7) [a] 
(0.9684436986301328, 107.71) [a] 
(0.9684450684931465, 111.04) [a] 
(0.9684476712328725, 111.05) [a] 
(0.9684479452054752, 111.12) [a] 
(0.9684482191780779, 111.13) [a] 
(0.968448630136982, 111.15) [a] 
(0.9684490410958861, 111.3) [a] 
(0.9684494520547903, 118.7) [a] 
(0.968449726027393, 118.74) [a] 
(0.9684524657534204, 120.2) [a] 
(0.9684616438356122, 122.51) [a] 
(0.9684709589041053, 124.02) [a] 
(0.9684801369862971, 124.03) [a] 
(0.9684804109588998, 124.34) [a] 
(0.9684810958904067, 124.38) [a] 
(0.9684841095890367, 124.41) [a] 
(0.9684842465753382, 124.76) [a] 
(0.9684852054794477, 124.81) [a] 
(0.9684861643835573, 125.57) [a] 
(0.968487534246571, 125.8) [a] 
(0.9684879452054751, 127.32) [a] 
(0.9684882191780778, 128.91) [a] 
(0.9685060273972558, 133) [a] 
(0.9685087671232832, 133.1) [a] 
(0.9685142465753379, 133.2) [a] 
(0.9685297260273927, 134.14) [a] 
(0.9685338356164338, 135.6) [a] 
(0.9685352054794475, 144.24) [a] 
(0.9685354794520502, 145.38) [a] 
(0.9685358904109543, 145.4) [a] 
(0.968536164383557, 145.41) [a] 
(0.9685363013698585, 146.63) [a] 
(0.9685365753424612, 146.64) [a] 
(0.9685369863013653, 146.65) [a] 
(0.9685524657534201, 147.1) [a] 
(0.9685535616438311, 147.5) [a] 
(0.9685539726027352, 147.52) [a] 
(0.968554246575338, 147.85) [a] 
(0.9685545205479407, 147.91) [a] 
(0.9685549315068448, 147.94) [a] 
(0.9685552054794475, 147.95) [a] 
(0.9685553424657489, 147.96) [a] 
(0.9685558904109544, 148.11) [a] 
(0.9685580821917763, 148.15) [a] 
(0.968558356164379, 148.24) [a] 
(0.9685586301369817, 153.85) [a] 
(0.9685587671232831, 162.61) [a] 
(0.9685590410958859, 162.62) [a] 
(0.9685593150684886, 162.68) [a] 
(0.96855945205479, 162.69) [a] 
(0.9685597260273927, 162.96) [a] 
(0.9685599999999954, 163.66) [a] 
(0.9685602739725981, 163.68) [a] 
(0.9685605479452009, 163.9) [a] 
(0.9685606849315023, 164.71) [a] 
(0.968560958904105, 165.34) [a] 
(0.9685613698630091, 165.37) [a] 
(0.9685616438356118, 165.4) [a] 
(0.9685619178082145, 166.08) [a] 
(0.968562054794516, 168.94) [a] 
(0.9685638356164338, 182.65) [a] 
(0.9685663013698584, 182.66) [a] 
(0.9685695890410914, 182.67) [a] 
(0.9685706849315023, 182.68) [a] 
(0.968570958904105, 182.69) [a] 
(0.968572054794516, 182.7) [a] 
(0.9685724657534202, 182.72) [a] 
(0.9685730136986256, 183.69) [a] 
(0.968573150684927, 186.85) [a] 
(0.9685734246575297, 187.14) [a] 
(0.9685741095890366, 187.15) [a] 
(0.9685743835616393, 187.16) [a] 
(0.9685754794520502, 187.17) [a] 
(0.968575753424653, 187.51) [a] 
(0.9685758904109544, 187.87) [a] 
(0.9685895890410914, 187.99) [a] 
(0.9685980821917763, 188) [a] 
(0.9686024657534201, 188.01) [a] 
(0.968604246575338, 188.15) [a] 
(0.9686046575342421, 188.17) [a] 
(0.9686050684931462, 190.45) [a] 
(0.9686053424657489, 190.49) [a] 
(0.9686054794520503, 190.56) [a] 
(0.9686071232876667, 190.8) [a] 
(0.9686073972602695, 190.88) [a] 
(0.9686075342465709, 192.99) [a] 
(0.9686078082191736, 193.02) [a] 
(0.9686080821917763, 193.06) [a] 
(0.9686082191780777, 193.2) [a] 
(0.9686084931506804, 193.34) [a] 
(0.9686087671232831, 193.35) [a] 
(0.9686089041095846, 193.47) [a] 
(0.9686091780821873, 193.57) [a] 
(0.968610547945201, 195.9) [a] 
(0.9686149315068447, 198.1) [a] 
(0.9686152054794475, 203.63) [a] 
(0.9686153424657489, 204.45) [a] 
(0.9686167123287626, 212.1) [a] 
(0.9686169863013653, 215.76) [a] 
(0.968617260273968, 216.91) [a] 
(0.9686175342465707, 218.38) [a] 
(0.9686216438356118, 218.8) [a] 
(0.9686223287671186, 220.84) [a] 
(0.9686332876712281, 221.7) [a] 
(0.9686360273972555, 231.9) [a] 
(0.9686363013698582, 238.25) [a] 
(0.9686369863013651, 238.31) [a] 
(0.9686372602739678, 238.33) [a] 
(0.9686373972602692, 238.51) [a] 
(0.9686376712328719, 238.52) [a] 
(0.9686431506849267, 240.6) [a] 
(0.9686486301369814, 243) [a] 
(0.9686513698630088, 244.1) [a] 
(0.9687282191780773, 245.69) [a] 
(0.9687286301369814, 246.26) [a] 
(0.9687327397260225, 248.4) [a] 
(0.9687328767123239, 253.36) [a] 
(0.9687334246575293, 253.37) [a] 
(0.9687335616438307, 253.41) [a] 
(0.9687345205479403, 253.43) [a] 
(0.9687372602739677, 253.44) [a] 
(0.9687380821917759, 253.45) [a] 
(0.9687390410958855, 253.48) [a] 
(0.9687393150684882, 253.52) [a] 
(0.9687397260273923, 254.1) [a] 
(0.9687416438356115, 254.15) [a] 
(0.9687420547945156, 254.21) [a] 
(0.9687680821917759, 256) [a] 
(0.9687941095890362, 256.1) [a] 
(0.9687943835616389, 256.28) [a] 
(0.9687950684931458, 256.43) [a] 
(0.9687952054794472, 256.44) [a] 
(0.9687954794520499, 257.02) [a] 
(0.968795890410954, 257.66) [a] 
(0.9687964383561595, 257.75) [a] 
(0.9688228767123238, 258.45) [a] 
(0.9688242465753375, 258.9) [a] 
(0.9688352054794471, 259.7) [a] 
(0.9688354794520498, 268.48) [a] 
(0.968835890410954, 268.94) [a] 
(0.9688361643835567, 268.97) [a] 
(0.9689131506849266, 271.69) [a] 
(0.9689135616438307, 287.34) [a] 
(0.968925890410954, 287.8) [a] 
(0.9689260273972554, 287.81) [a] 
(0.9689265753424608, 287.89) [a] 
(0.9689267123287623, 287.9) [a] 
(0.968926986301365, 287.95) [a] 
(0.9689626027397212, 288.9) [a] 
(0.9689627397260226, 289.27) [a] 
(0.9689630136986254, 289.39) [a] 
(0.9689632876712281, 289.43) [a] 
(0.9689635616438308, 289.44) [a] 
(0.9689636986301322, 289.45) [a] 
(0.9689639726027349, 289.47) [a] 
(0.9689646575342418, 289.48) [a] 
(0.9689649315068445, 289.51) [a] 
(0.9689650684931459, 289.52) [a] 
(0.9689653424657486, 289.57) [a] 
(0.9689656164383513, 289.61) [a] 
(0.9689660273972555, 289.62) [a] 
(0.9689663013698582, 289.64) [a] 
(0.9689664383561596, 289.89) [a] 
(0.9689667123287623, 289.93) [a] 
(0.9689790410958856, 290.1) [a] 
(0.9689793150684883, 291.17) [a] 
(0.9689797260273925, 291.18) [a] 
(0.9689920547945158, 291.7) [a] 
(0.9689924657534199, 292.53) [a] 
(0.968992876712324, 298.6) [a] 
(0.9689938356164336, 298.67) [a] 
(0.9689942465753377, 299.05) [a] 
(0.9689947945205432, 299.21) [a] 
(0.9689975342465705, 299.22) [a] 
(0.9689979452054747, 299.23) [a] 
(0.9690002739725979, 299.24) [a] 
(0.969000684931502, 299.26) [a] 
(0.9690020547945157, 299.35) [a] 
(0.9690038356164336, 299.84) [a] 
(0.9690047945205431, 299.85) [a] 
(0.9690052054794472, 299.86) [a] 
(0.9690057534246527, 299.87) [a] 
(0.9690061643835568, 299.93) [a] 
(0.9690075342465705, 299.99) [a] 
(0.9690079452054746, 300.02) [a] 
(0.96900849315068, 300.22) [a] 
(0.9690098630136937, 300.23) [a] 
(0.9690102739725979, 300.24) [a] 
(0.9690116438356116, 300.71) [a] 
(0.9690226027397212, 300.8) [a] 
(0.9690235616438307, 300.82) [a] 
(0.9690239726027349, 300.96) [a] 
(0.969024383561639, 301.05) [a] 
(0.9690249315068444, 301.06) [a] 
(0.9690253424657486, 301.18) [a] 
(0.9690263013698581, 302.08) [a] 
(0.9690267123287623, 302.1) [a] 
(0.9690271232876664, 302.11) [a] 
(0.9690276712328718, 302.41) [a] 
(0.9690280821917759, 302.42) [a] 
(0.9690284931506801, 302.6) [a] 
(0.9690290410958855, 302.67) [a] 
(0.9690294520547896, 302.7) [a] 
(0.9690298630136938, 302.71) [a] 
(0.9690304109588992, 302.99) [a] 
(0.9690308219178033, 303.03) [a] 
(0.9690312328767074, 303.04) [a] 
(0.9690317808219129, 304.08) [a] 
(0.9690320547945156, 306.49) [a] 
(0.9690328767123239, 308.79) [a] 
(0.9690334246575293, 308.81) [a] 
(0.9690338356164334, 308.96) [a] 
(0.9690342465753375, 310.51) [a] 
(0.969034794520543, 311.33) [a] 
(0.9690352054794471, 311.36) [a] 
(0.9690356164383512, 311.41) [a] 
(0.9690369863013649, 311.87) [a] 
(0.9690379452054745, 313.07) [a] 
(0.9690383561643786, 313.37) [a] 
(0.9690393150684882, 313.4) [a] 
(0.9690406849315019, 313.41) [a] 
(0.969041095890406, 313.47) [a] 
(0.9690416438356114, 313.63) [a] 
(0.9690420547945156, 313.81) [a] 
(0.9690424657534197, 313.86) [a] 
(0.9690430136986251, 314.04) [a] 
(0.9690434246575292, 314.62) [a] 
(0.9690438356164334, 314.63) [a] 
(0.9690443835616388, 314.8) [a] 
(0.9690452054794471, 315.47) [a] 
(0.9690461643835566, 315.65) [a] 
(0.9690471232876662, 316.04) [a] 
(0.9690475342465703, 316.05) [a] 
(0.9690479452054744, 316.72) [a] 
(0.9690484931506799, 316.73) [a] 
(0.9690493150684881, 316.93) [a] 
(0.9690498630136936, 317.17) [a] 
(0.9690502739725977, 319.44) [a] 
(0.9690516438356114, 319.45) [a] 
(0.9690520547945155, 319.54) [a] 
(0.9690526027397209, 319.61) [a] 
(0.9690534246575292, 320.42) [a] 
(0.9690536986301319, 322.01) [a] 
(0.969055205479447, 322.02) [a] 
(0.9690554794520497, 322.04) [a] 
(0.9690557534246524, 322.06) [a] 
(0.9690558904109539, 322.08) [a] 
(0.9690561643835566, 322.27) [a] 
(0.9690568493150634, 322.28) [a] 
(0.9690571232876661, 322.29) [a] 
(0.9690572602739675, 322.32) [a] 
(0.9690575342465703, 322.33) [a] 
(0.969057808219173, 322.34) [a] 
(0.9690584931506798, 322.36) [a] 
(0.9690586301369812, 322.37) [a] 
(0.969058904109584, 322.39) [a] 
(0.9690591780821867, 322.4) [a] 
(0.9690593150684881, 322.42) [a] 
(0.9690598630136935, 322.43) [a] 
(0.9690612328767072, 322.49) [a] 
(0.9690613698630086, 322.79) [a] 
(0.9690616438356113, 324.12) [a] 
(0.9690620547945155, 324.14) [a] 
(0.9690623287671182, 324.38) [a] 
(0.9690626027397209, 324.4) [a] 
(0.9690631506849263, 325.33) [a] 
(0.9690635616438305, 325.46) [a] 
(0.9690758904109538, 327.2) [a] 
(0.9690768493150633, 327.24) [a] 
(0.9690891780821866, 327.6) [a] 
(0.96910150684931, 327.7) [a] 
(0.9691124657534196, 327.8) [a] 
(0.9691261643835566, 328.7) [a] 
(0.9691265753424607, 328.73) [a] 
(0.969138904109584, 328.8) [a] 
(0.9691393150684882, 329.12) [a] 
(0.9691398630136936, 329.8) [a] 
(0.969140410958899, 333.57) [a] 
(0.9691412328767073, 334.12) [a] 
(0.9691421917808168, 334.13) [a] 
(0.9691430136986251, 334.15) [a] 
(0.9691435616438305, 334.29) [a] 
(0.9691443835616388, 334.82) [a] 
(0.9691449315068442, 334.88) [a] 
(0.9691456164383511, 335.18) [a] 
(0.9691460273972552, 335.19) [a] 
(0.9691463013698579, 335.2) [a] 
(0.969146712328762, 337.25) [a] 
(0.9691468493150635, 337.64) [a] 
(0.9691471232876662, 353.92) [a] 
(0.9691473972602689, 353.98) [a] 
(0.9691482191780771, 354.01) [a] 
(0.9691484931506799, 354.04) [a] 
(0.9691487671232826, 354.08) [a] 
(0.969148904109584, 354.09) [a] 
(0.9691495890410908, 354.11) [a] 
(0.9691498630136935, 354.12) [a] 
(0.9691502739725977, 354.13) [a] 
(0.9691505479452004, 354.33) [a] 
(0.9691506849315018, 359.8) [a] 
(0.9691509589041045, 363.09) [a] 
(0.9691512328767072, 363.29) [a] 
(0.9691513698630086, 363.82) [a] 
(0.9691516438356114, 363.92) [a] 
(0.9691519178082141, 364.03) [a] 
(0.9691520547945155, 364.04) [a] 
(0.9691523287671182, 364.06) [a] 
(0.9691526027397209, 365.3) [a] 
(0.9691527397260223, 369.2) [a] 
(0.969153013698625, 369.22) [a] 
(0.9691532876712278, 370.28) [a] 
(0.9691535616438305, 376.1) [a] 
(0.96915452054794, 377.48) [a] 
(0.9691549315068442, 377.49) [a] 
(0.9691553424657483, 377.5) [a] 
(0.969155616438351, 377.53) [a] 
(0.9691560273972551, 380.97) [a] 
(0.9691564383561593, 381.97) [a] 
(0.9691568493150634, 397.8) [a] 
(0.9691571232876661, 397.89) [a] 
(0.969234109589036, 398.81) [a] 
(0.9692464383561593, 412) [a] 
(0.9692465753424607, 412.81) [a] 
(0.9692471232876662, 413.3) [a] 
(0.9692476712328716, 416.44) [a] 
(0.9692599999999949, 418) [a] 
(0.9692709589041045, 418.1) [a] 
(0.9692832876712278, 418.2) [a] 
(0.9693065753424608, 419.2) [a] 
(0.969342191780817, 419.3) [a] 
(0.9693545205479404, 419.4) [a] 
(0.9693549315068445, 420.96) [a] 
(0.9693553424657486, 436.26) [a] 
(0.9693556164383513, 441.77) [a] 
(0.969356986301365, 445.5) [a] 
(0.9693583561643787, 450.3) [a] 
(0.9693597260273924, 453.7) [a] 
(0.969360684931502, 455.72) [a] 
(0.9693616438356115, 455.73) [a] 
(0.9693626027397211, 456.85) [a] 
(0.9693634246575293, 458.28) [a] 
(0.9693643835616389, 458.32) [a] 
(0.9693646575342416, 461.81) [a] 
(0.969364794520543, 461.85) [a] 
(0.9693650684931457, 461.89) [a] 
(0.9693653424657485, 461.9) [a] 
(0.9693654794520499, 461.95) [a] 
(0.9693657534246526, 461.96) [a] 
(0.9693660273972553, 461.98) [a] 
(0.9693661643835567, 461.99) [a] 
(0.9693664383561594, 462.32) [a] 
(0.9693667123287621, 462.66) [a] 
(0.9693676712328717, 463.58) [a] 
(0.9693686301369813, 464.13) [a] 
(0.9693713698630086, 464.54) [a] 
(0.9693723287671182, 464.71) [a] 
(0.9693732876712278, 465.73) [a] 
(0.9693742465753373, 466.54) [a] 
(0.9693752054794469, 466.55) [a] 
(0.9693761643835564, 466.59) [a] 
(0.9693789041095838, 466.7) [a] 
(0.9693806849315016, 467.8) [a] 
(0.9693816438356112, 468.64) [a] 
(0.9693826027397208, 470.11) [a] 
(0.9693835616438303, 470.12) [a] 
(0.9693845205479399, 470.13) [a] 
(0.9693853424657481, 470.14) [a] 
(0.9693863013698577, 470.15) [a] 
(0.9693872602739673, 470.17) [a] 
(0.9693882191780768, 470.18) [a] 
(0.9693891780821864, 470.2) [a] 
(0.9693909589041042, 470.22) [a] 
(0.9693919178082138, 470.23) [a] 
(0.9693928767123233, 470.24) [a] 
(0.9693938356164329, 470.25) [a] 
(0.9693965753424603, 470.28) [a] 
(0.9693975342465698, 470.29) [a] 
(0.969399452054789, 470.31) [a] 
(0.9694004109588985, 470.32) [a] 
(0.9694013698630081, 470.33) [a] 
(0.9694031506849259, 470.34) [a] 
(0.9694041095890354, 470.36) [a] 
(0.9694060273972547, 470.4) [a] 
(0.9694069863013642, 470.46) [a] 
(0.9694078082191725, 470.51) [a] 
(0.9694097260273917, 470.66) [a] 
(0.969411643835611, 470.8) [a] 
(0.9694126027397205, 470.87) [a] 
(0.9694134246575288, 470.98) [a] 
(0.969415342465748, 471.05) [a] 
(0.9694163013698576, 471.09) [a] 
(0.9694172602739671, 471.14) [a] 
(0.9694182191780767, 471.2) [a] 
(0.969419041095885, 471.27) [a] 
(0.9694199999999945, 471.34) [a] 
(0.9694209589041041, 471.36) [a] 
(0.9694219178082136, 471.41) [a] 
(0.9694228767123232, 471.47) [a] 
(0.9694238356164327, 471.54) [a] 
(0.969424657534241, 471.65) [a] 
(0.9694256164383506, 472.31) [a] 
(0.969432465753419, 472.8) [a] 
(0.9694334246575286, 472.88) [a] 
(0.96943356164383, 473.36) [a] 
(0.9694354794520492, 474.71) [a] 
(0.969440958904104, 476.3) [a] 
(0.9694412328767067, 477.21) [a] 
(0.9694415068493094, 477.24) [a] 
(0.9694416438356108, 477.25) [a] 
(0.9694419178082135, 477.27) [a] 
(0.9694421917808163, 477.36) [a] 
(0.9694423287671177, 477.49) [a] 
(0.9694426027397204, 477.51) [a] 
(0.9694428767123231, 477.54) [a] 
(0.9694431506849258, 478.1) [a] 
(0.9694432876712272, 478.16) [a] 
(0.9694460273972546, 480.1) [a] 
(0.9694463013698573, 480.51) [a] 
(0.9694472602739669, 484.91) [a] 
(0.969447671232871, 488.04) [a] 
(0.9694480821917751, 488.05) [a] 
(0.9694482191780766, 490.31) [a] 
(0.9694484931506793, 490.32) [a] 
(0.9694489041095834, 490.33) [a] 
(0.969449863013693, 490.34) [a] 
(0.9694501369862957, 490.41) [a] 
(0.9694502739725971, 490.43) [a] 
(0.9694505479451998, 490.51) [a] 
(0.9694508219178025, 490.52) [a] 
(0.9694510958904052, 492.2) [a] 
(0.9694512328767066, 492.28) [a] 
(0.9694515068493094, 492.3) [a] 
(0.9694638356164327, 493.7) [a] 
(0.9694994520547889, 494) [a] 
(0.9695227397260219, 494.1) [a] 
(0.9695350684931452, 494.6) [a] 
(0.9695939726027344, 495.8) [a] 
(0.9696063013698577, 496) [a] 
(0.9696068493150631, 498.63) [a] 
(0.9696071232876659, 499.39) [a] 
(0.9696072602739673, 499.45) [a] 
(0.96960753424657, 501.95) [a] 
(0.9696079452054741, 503.61) [a] 
(0.9696082191780768, 514.3) [a] 
(0.9696083561643782, 526.3) [a] 
(0.9696097260273919, 531) [a] 
(0.9696110958904056, 533) [a] 
(0.9696879452054741, 544.89) [a] 
(0.9696882191780768, 545.89) [a] 
(0.9696936986301316, 546.2) [a] 
(0.9696950684931452, 549.7) [a] 
(0.9696978082191726, 550.4) [a] 
(0.9697101369862959, 553.3) [a] 
(0.9697105479452001, 554.4) [a] 
(0.9697110958904055, 589.75) [a] 
(0.9697119178082138, 589.82) [a] 
(0.969714246575337, 590.21) [a] 
(0.9697146575342411, 590.22) [a] 
(0.9697152054794466, 590.23) [a] 
(0.9697160273972548, 590.24) [a] 
(0.9697269863013644, 604) [a] 
(0.9697275342465699, 604.06) [a] 
(0.9697398630136932, 604.2) [a] 
(0.9697402739725973, 612.53) [a] 
(0.969741643835611, 612.54) [a] 
(0.9697430136986247, 612.59) [a] 
(0.9697439726027343, 612.6) [a] 
(0.969749452054789, 620.6) [a] 
(0.9697502739725973, 625.04) [a] 
(0.9697504109588987, 664.06) [a] 
(0.9697506849315014, 664.08) [a] 
(0.9697508219178028, 666.34) [a] 
(0.969751232876707, 673.72) [a] 
(0.9697515068493097, 698) [a] 
(0.9697516438356111, 712.43) [a] 
(0.9698576712328713, 736.92) [a] 
(0.9698841095890356, 737.11) [a] 
(0.9699106849315013, 737.13) [a] 
(0.9699901369862959, 743.38) [a] 
(0.9699905479452, 752.89) [a] 
(0.9699910958904054, 753.54) [a] 
(0.9699912328767069, 763.64) [a] 
(0.9699915068493096, 763.65) [a] 
(0.9699923287671178, 772.94) [a] 
(0.9699924657534192, 777.33) [a] 
(0.969992739726022, 777.35) [a] 
(0.9699930136986247, 778.04) [a] 
(0.9699931506849261, 779.08) [a] 
(0.9699934246575288, 779.11) [a] 
(0.9699936986301315, 779.86) [a] 
(0.9700030136986246, 783.42) [a] 
(0.9700213698630082, 783.43) [a] 
(0.9700306849315014, 789.3) [a] 
(0.9700398630136932, 789.31) [a] 
(0.9700401369862959, 789.82) [a] 
(0.9700402739725973, 789.83) [a] 
(0.9700495890410904, 798.12) [a] 
(0.9700587671232822, 798.17) [a] 
(0.9700772602739671, 798.22) [a] 
(0.9700956164383507, 798.35) [a] 
(0.9701049315068438, 798.78) [a] 
(0.9701052054794466, 800.34) [a] 
(0.9701143835616384, 800.69) [a] 
(0.9701236986301315, 800.7) [a] 
(0.9701373972602685, 805.5) [a] 
(0.9701376712328712, 813.88) [a] 
(0.9701379452054739, 813.92) [a] 
(0.9701380821917753, 813.95) [a] 
(0.970138356164378, 814.26) [a] 
(0.9701475342465699, 819.01) [a] 
(0.9701567123287617, 819.45) [a] 
(0.9701660273972548, 819.47) [a] 
(0.9701669863013643, 820.74) [a] 
(0.9701793150684876, 835.3) [a] 
(0.9701795890410904, 890.61) [a] 
(0.9701798630136931, 898.53) [a] 
(0.9701801369862958, 898.54) [a] 
(0.9701802739725972, 898.58) [a] 
(0.9701808219178026, 898.59) [a] 
(0.970180958904104, 905.31) [a] 
(0.9701810958904055, 915.99) [a] 
(0.970182054794515, 916.06) [a] 
(0.9701824657534192, 916.07) [a] 
(0.9701834246575287, 918.56) [a] 
(0.9701847945205424, 918.61) [a] 
(0.9701852054794465, 919.04) [a] 
(0.9701856164383507, 919.08) [a] 
(0.9701869863013644, 919.4) [a] 
(0.9701961643835562, 922.3) [a] 
(0.9702054794520493, 922.32) [a] 
(0.9702146575342411, 923.26) [a] 
(0.9702238356164329, 923.28) [a] 
(0.970233150684926, 923.33) [a] 
(0.9702423287671178, 923.4) [a] 
(0.9702516438356109, 923.5) [a] 
(0.9702521917808163, 925.87) [a] 
(0.9702534246575286, 926.24) [a] 
(0.9702565753424601, 926.29) [a] 
(0.9702830136986245, 938.66) [a] 
(0.9703095890410902, 938.67) [a] 
(0.9703105479451998, 939.27) [a] 
(0.9703126027397203, 939.28) [a] 
(0.9703130136986244, 939.8) [a] 
(0.9703134246575286, 939.9) [a] 
(0.9703398630136929, 941.79) [a] 
(0.9703408219178025, 944.32) [a] 
(0.9703499999999943, 948.9) [a] 
(0.9703591780821861, 949.34) [a] 
(0.9703684931506792, 950.38) [a] 
(0.970377671232871, 952.38) [a] 
(0.9703790410958847, 953) [a] 
(0.9703882191780765, 953.4) [a] 
(0.9703886301369806, 988.15) [a] 
(0.970389178082186, 988.2) [a] 
(0.9703895890410902, 988.29) [a] 
(0.9703898630136929, 989.8) [a] 
(0.970390273972597, 1001.1) [a] 
(0.9703995890410901, 1007.7) [a] 
(0.970408767123282, 1008.5) [a] 
(0.9704089041095834, 1022.4) [a] 
(0.970410273972597, 1088) [a] 
(0.9704116438356107, 1089) [a] 
(0.9704226027397203, 1090) [a] 
(0.9704239726027339, 1091) [a] 
(0.9704294520547887, 1092) [a] 
(0.9704295890410901, 1120) [a] 
(0.9704298630136928, 1142.2) [a] 
(0.970430273972597, 1142.3) [a] 
(0.9704387671232819, 1159.5) [a] 
(0.9704524657534189, 1168) [a] 
(0.9704647945205422, 1169) [a] 
(0.970465068493145, 1276.8) [a] 
(0.9704653424657477, 1742) [a] 
(0.9704658904109531, 1880.8) [a] 
(0.9704701369862956, 1896.7) [a] 
(0.9704705479451997, 1932.8) [a] 
(0.9704719178082134, 2019) [a] 
(0.9704746575342408, 2095) [a] 
(0.9704760273972545, 2096) [a] 
(0.9704773972602682, 2106) [a] 
(0.9704787671232818, 2108) [a] 
(0.9704789041095833, 2124.1) [a] 
(0.970479178082186, 2141.2) [a] 
(0.9704805479451997, 2185) [a] 
(0.9704819178082134, 2205) [a] 
(0.9704846575342407, 2211) [a] 
(0.9704873972602681, 2233) [a] 
(0.9704901369862955, 2235) [a] 
(0.9704915068493092, 2248) [a] 
(0.9705024657534187, 2315) [a] 
(0.9705065753424598, 2350) [a] 
(0.9705120547945145, 2466) [a] 
(0.9705134246575282, 2480) [a] 
(0.970518904109583, 2509) [a] 
(0.9705191780821857, 2519.5) [a] 
(0.9705193150684871, 2519.6) [a] 
(0.9705247945205419, 2584) [a] 
(0.9705302739725966, 2590) [a] 
(0.970530410958898, 2590.7) [a] 
(0.9705345205479391, 2591) [a] 
(0.9705349315068432, 2601.5) [a] 
(0.9705384931506789, 2601.7) [a] 
(0.9705430136986241, 2602.1) [a] 
(0.9705584931506789, 2607.5) [a] 
(0.9705621917808158, 2610) [a] 
(0.970593013698624, 2616.4) [a] 
(0.9706084931506789, 2616.5) [a] 
(0.9706239726027337, 2618.6) [a] 
(0.9706393150684871, 2625.8) [a] 
(0.9706410958904049, 2631.3) [a] 
(0.9706424657534186, 2631.4) [a] 
(0.970645205479446, 2632) [a] 
(0.9706465753424597, 2632.1) [a] 
(0.9706493150684871, 2650) [a] 
(0.9706506849315008, 2746) [a] 
(0.9706563013698569, 2746.4) [a] 
(0.9706571232876652, 2748.9) [a] 
(0.9706598630136926, 2758) [a] 
(0.9706636986301308, 2789.5) [a] 
(0.9706646575342404, 2789.8) [a] 
(0.9706654794520486, 2790.1) [a] 
(0.9706664383561582, 2790.2) [a] 
(0.9706702739725965, 2791.4) [a] 
(0.9706710958904048, 2793.7) [a] 
(0.9706752054794459, 2794) [a] 
(0.9706761643835554, 2865) [a] 
(0.9706916438356102, 3059) [a] 
},{(0.8376900139820977, 0) [b] 
(0.8767567398860433, 0.001) [b] 
(0.8886131303190717, 0.002) [b] 
(0.8916529000219553, 0.003) [b] 
(0.8946961797557202, 0.004) [b] 
(0.8976931120676619, 0.005) [b] 
(0.9005804475751058, 0.006) [b] 
(0.9024372274318829, 0.007) [b] 
(0.9044364170752185, 0.008) [b] 
(0.9058465305280985, 0.009) [b] 
(0.9097762597397734, 0.01) [b] 
(0.9106073847412837, 0.011) [b] 
(0.9118857143922223, 0.012) [b] 
(0.9121496708401005, 0.013) [b] 
(0.9132834383273725, 0.014) [b] 
(0.9138242152627452, 0.015) [b] 
(0.9146955191684331, 0.016) [b] 
(0.9154868223657393, 0.017) [b] 
(0.9164138986517395, 0.018) [b] 
(0.9180236492053733, 0.019) [b] 
(0.9192933292202428, 0.02) [b] 
(0.9196272152794488, 0.021) [b] 
(0.9203628330376294, 0.022) [b] 
(0.9204634756730997, 0.023) [b] 
(0.9206894680490959, 0.024) [b] 
(0.9210499000285238, 0.025) [b] 
(0.9211290222429078, 0.026) [b] 
(0.9212664646908405, 0.027) [b] 
(0.9217743796577806, 0.028) [b] 
(0.9219448778850695, 0.029) [b] 
(0.9220875662641906, 0.03) [b] 
(0.9221446740900906, 0.031) [b] 
(0.9223120886712041, 0.032) [b] 
(0.9224253046638345, 0.033) [b] 
(0.9225036265151643, 0.034) [b] 
(0.9226361057295169, 0.035) [b] 
(0.9226905266937181, 0.036) [b] 
(0.9226948480281462, 0.037) [b] 
(0.9227169816724856, 0.038) [b] 
(0.9228238482260045, 0.039) [b] 
(0.9231466781065716, 0.04) [b] 
(0.9233630930396741, 0.041) [b] 
(0.923472567127163, 0.042) [b] 
(0.9240577316363359, 0.043) [b] 
(0.9240721137461707, 0.044) [b] 
(0.9243131165036171, 0.045) [b] 
(0.9243951386657229, 0.046) [b] 
(0.9244253736783743, 0.047) [b] 
(0.9244275949577868, 0.048) [b] 
(0.924445365193087, 0.049) [b] 
(0.9245433935469244, 0.05) [b] 
(0.9248149482280588, 0.051) [b] 
(0.9249857219265589, 0.052) [b] 
(0.9250858779905208, 0.053) [b] 
(0.9251987085892817, 0.054) [b] 
(0.9252424695774072, 0.055) [b] 
(0.9252535263780999, 0.056) [b] 
(0.9252592881573373, 0.057) [b] 
(0.9252711790292384, 0.058) [b] 
(0.9252742529384615, 0.059) [b] 
(0.9252920896964524, 0.06) [b] 
(0.9253099264544432, 0.061) [b] 
(0.9253378739646343, 0.062) [b] 
(0.9253784398671685, 0.063) [b] 
(0.9254012709173968, 0.064) [b] 
(0.9254547567431163, 0.065) [b] 
(0.9254574964691437, 0.066) [b] 
(0.9254821825422029, 0.067) [b] 
(0.9256209284573396, 0.068) [b] 
(0.9256819215625745, 0.069) [b] 
(0.9257855745306109, 0.07) [b] 
(0.9257872079950248, 0.071) [b] 
(0.9257994011549384, 0.072) [b] 
(0.9258008415997477, 0.073) [b] 
(0.9258055371484609, 0.074) [b] 
(0.9262338791434039, 0.077) [b] 
(0.9262361004228165, 0.078) [b] 
(0.9262665946394305, 0.079) [b] 
(0.9262862578814396, 0.08) [b] 
(0.9264963320823528, 0.081) [b] 
(0.9265782817692211, 0.082) [b] 
(0.9266610013830379, 0.083) [b] 
(0.9266795230725355, 0.084) [b] 
(0.9267383782632874, 0.085) [b] 
(0.9268802610934617, 0.086) [b] 
(0.9269663258702647, 0.087) [b] 
(0.9269922538768331, 0.088) [b] 
(0.9270962902071789, 0.089) [b] 
(0.9271122719423387, 0.09) [b] 
(0.9271201050812119, 0.091) [b] 
(0.9271306260900994, 0.092) [b] 
(0.9272685256334785, 0.093) [b] 
(0.9272936397887296, 0.094) [b] 
(0.9273103439069612, 0.095) [b] 
(0.9273219899032449, 0.096) [b] 
(0.9273831771178568, 0.097) [b] 
(0.9274488788226132, 0.098) [b] 
(0.9275423161560618, 0.099) [b] 
(0.9275543294657718, 0.1) [b] 
(0.9279736428551874, 0.101) [b] 
(0.9280112023590029, 0.102) [b] 
(0.9280217046421079, 0.103) [b] 
(0.9280573916622598, 0.104) [b] 
(0.9280804510229902, 0.105) [b] 
(0.9280956959706299, 0.106) [b] 
(0.9281041787501122, 0.107) [b] 
(0.9281190189327606, 0.108) [b] 
(0.9281669641382401, 0.109) [b] 
(0.9281706171062766, 0.11) [b] 
(0.9281754116268245, 0.111) [b] 
(0.9281781513528519, 0.112) [b] 
(0.9281808910788792, 0.113) [b] 
(0.9289513902836405, 0.114) [b] 
(0.9290488104469868, 0.115) [b] 
(0.9290529200360279, 0.116) [b] 
(0.9310163588645588, 0.117) [b] 
(0.9310339387732346, 0.118) [b] 
(0.9310353086362483, 0.119) [b] 
(0.9310385049832802, 0.12) [b] 
(0.9310435278143304, 0.121) [b] 
(0.9314946576005119, 0.122) [b] 
(0.9314964840845301, 0.123) [b] 
(0.9315119161179177, 0.124) [b] 
(0.9315128293599269, 0.125) [b] 
(0.9315322642914337, 0.126) [b] 
(0.9315327209124382, 0.127) [b] 
(0.9315434889576105, 0.128) [b] 
(0.9315455437521311, 0.129) [b] 
(0.9315493378837589, 0.131) [b] 
(0.9315504794362703, 0.132) [b] 
(0.9315895205321607, 0.133) [b] 
(0.9315936301212018, 0.134) [b] 
(0.9325356707447257, 0.135) [b] 
(0.9325379538497485, 0.136) [b] 
(0.9325407641575715, 0.137) [b] 
(0.9325416773995806, 0.138) [b] 
(0.932544417125608, 0.139) [b] 
(0.9325468734317015, 0.14) [b] 
(0.932550526399738, 0.141) [b] 
(0.9325514396417471, 0.142) [b] 
(0.932565823203391, 0.145) [b] 
(0.9325660515138933, 0.146) [b] 
(0.9325693537336187, 0.147) [b] 
(0.9325841939162671, 0.148) [b] 
(0.9325844222267694, 0.149) [b] 
(0.9325864770212899, 0.15) [b] 
(0.932588988436815, 0.151) [b] 
(0.9325935546468607, 0.153) [b] 
(0.9325953811308789, 0.154) [b] 
(0.9326093451887222, 0.155) [b] 
(0.9326296648234254, 0.156) [b] 
(0.9326298931339276, 0.159) [b] 
(0.9326303497549322, 0.16) [b] 
(0.932631034686439, 0.161) [b] 
(0.9326529524946582, 0.163) [b] 
(0.9326688528925928, 0.164) [b] 
(0.9326693095135974, 0.165) [b] 
(0.9326761588286658, 0.166) [b] 
(0.9326779853126841, 0.169) [b] 
(0.9326816382807206, 0.17) [b] 
(0.9326994750387114, 0.173) [b] 
(0.9327001599702183, 0.174) [b] 
(0.9327008449017251, 0.175) [b] 
(0.9327010732122274, 0.176) [b] 
(0.9327245891939625, 0.178) [b] 
(0.9327278475044951, 0.182) [b] 
(0.9327294456780111, 0.183) [b] 
(0.9327296739885134, 0.184) [b] 
(0.932732651304344, 0.186) [b] 
(0.9327365325828828, 0.187) [b] 
(0.932743153587449, 0.188) [b] 
(0.9327618750486362, 0.189) [b] 
(0.932826026628645, 0.19) [b] 
(0.9328299079071838, 0.191) [b] 
(0.9328456613318413, 0.192) [b] 
(0.9329040146654208, 0.193) [b] 
(0.9329074393229551, 0.194) [b] 
(0.9329083525649642, 0.195) [b] 
(0.9329213662635943, 0.196) [b] 
(0.9329222795056035, 0.197) [b] 
(0.9329277589576582, 0.198) [b] 
(0.9329337043100185, 0.199) [b] 
(0.9329741152689227, 0.2) [b] 
(0.9329937499721189, 0.201) [b] 
(0.9330147253232409, 0.202) [b] 
(0.9330243143643367, 0.203) [b] 
(0.9330496568300901, 0.204) [b] 
(0.933060844044702, 0.205) [b] 
(0.9332297938163914, 0.206) [b] 
(0.9332647253232407, 0.207) [b] 
(0.9333396111679895, 0.208) [b] 
(0.9333831919017996, 0.209) [b] 
(0.9334329104470456, 0.21) [b] 
(0.9335202470809166, 0.211) [b] 
(0.9335236717384509, 0.212) [b] 
(0.9335683648096844, 0.213) [b] 
(0.9336148870078984, 0.214) [b] 
(0.9336160285604098, 0.216) [b] 
(0.9336178550444281, 0.217) [b] 
(0.9336281290170309, 0.218) [b] 
(0.9336514166882637, 0.219) [b] 
(0.9336644011718058, 0.22) [b] 
(0.9338214787973765, 0.221) [b] 
(0.9338669125873309, 0.222) [b] 
(0.9338723920393857, 0.223) [b] 
(0.93393883039555, 0.224) [b] 
(0.9339808395279701, 0.225) [b] 
(0.9340201089343628, 0.226) [b] 
(0.9340607482037692, 0.227) [b] 
(0.9341644011718058, 0.228) [b] 
(0.9341876888430386, 0.229) [b] 
(0.9341940815371025, 0.23) [b] 
(0.9342366879557655, 0.231) [b] 
(0.9342378295082769, 0.232) [b] 
(0.9343202495996011, 0.233) [b] 
(0.934328240467181, 0.234) [b] 
(0.9343404353971132, 0.235) [b] 
(0.9343469776739199, 0.236) [b] 
(0.9343686671716367, 0.237) [b] 
(0.9344076816954012, 0.238) [b] 
(0.9345049525144519, 0.239) [b] 
(0.9349236739756389, 0.24) [b] 
(0.9349747889560245, 0.241) [b] 
(0.9350480500551315, 0.242) [b] 
(0.9351147167217981, 0.243) [b] 
(0.9351781604693069, 0.244) [b] 
(0.935182270058348, 0.245) [b] 
(0.9353656487101799, 0.246) [b] 
(0.9354140239645381, 0.247) [b] 
(0.9355318341504031, 0.248) [b] 
(0.9357110598613823, 0.249) [b] 
(0.9357364023271356, 0.25) [b] 
(0.9357816078065876, 0.251) [b] 
(0.9357873155691446, 0.252) [b] 
(0.9358249868020213, 0.253) [b] 
(0.9359103749298752, 0.254) [b] 
(0.9359160826924322, 0.255) [b] 
(0.9360384571216559, 0.256) [b] 
(0.9360644845189161, 0.257) [b] 
(0.9361254434230256, 0.258) [b] 
(0.9361382288111534, 0.259) [b] 
(0.936142795021199, 0.26) [b] 
(0.9361585484458566, 0.261) [b] 
(0.9361713338339844, 0.262) [b] 
(0.9362398269846693, 0.263) [b] 
(0.9362409685371808, 0.264) [b] 
(0.9362432516422036, 0.265) [b] 
(0.9362462196787332, 0.266) [b] 
(0.9362471329207424, 0.267) [b] 
(0.9363208772129797, 0.268) [b] 
(0.9363352607746236, 0.269) [b] 
(0.9363578897687703, 0.27) [b] 
(0.9363661089468525, 0.271) [b] 
(0.9364309491295009, 0.272) [b] 
(0.9365519536957109, 0.273) [b] 
(0.9367535518692268, 0.274) [b] 
(0.9367546934217382, 0.275) [b] 
(0.936758574700277, 0.276) [b] 
(0.9367718167094095, 0.277) [b] 
(0.9370871135130624, 0.278) [b] 
(0.9376084996627972, 0.279) [b] 
(0.9376358969230711, 0.28) [b] 
(0.9381137508043497, 0.281) [b] 
(0.9381283626764958, 0.282) [b] 
(0.9381388379231282, 0.283) [b] 
(0.9381390662336305, 0.284) [b] 
(0.9385255959139958, 0.285) [b] 
(0.9385358518622834, 0.286) [b] 
(0.9386223815426487, 0.287) [b] 
(0.938626719442192, 0.288) [b] 
(0.938654116702466, 0.289) [b] 
(0.9386956692138816, 0.29) [b] 
(0.9387214390855515, 0.291) [b] 
(0.9388269185376062, 0.292) [b] 
(0.9388563705924008, 0.293) [b] 
(0.9388748637430856, 0.294) [b] 
(0.9390250920535878, 0.295) [b] 
(0.9391707541540443, 0.296) [b] 
(0.9392463249303, 0.297) [b] 
(0.9392636765284735, 0.298) [b] 
(0.9393296582636332, 0.299) [b] 
(0.9395042887614072, 0.3) [b] 
(0.9396718686700829, 0.301) [b] 
(0.9397844257477085, 0.302) [b] 
(0.9398088113133316, 0.303) [b] 
(0.9398615510393591, 0.304) [b] 
(0.9400638341443819, 0.305) [b] 
(0.940086042226604, 0.306) [b] 
(0.9401031655142752, 0.307) [b] 
(0.9401317043270605, 0.308) [b] 
(0.9401990559252339, 0.309) [b] 
(0.9402061335508046, 0.31) [b] 
(0.9402353572950969, 0.311) [b] 
(0.9402410650576539, 0.312) [b] 
(0.9402568184823114, 0.313) [b] 
(0.9402906084366491, 0.314) [b] 
(0.9403077317243203, 0.315) [b] 
(0.9403239417699822, 0.316) [b] 
(0.9403353572950963, 0.317) [b] 
(0.9406985993042287, 0.318) [b] 
(0.9407013390302561, 0.319) [b] 
(0.940704535377288, 0.32) [b] 
(0.9407481712220369, 0.321) [b] 
(0.940909586747151, 0.322) [b] 
(0.9409146095782012, 0.323) [b] 
(0.9409150661992057, 0.324) [b] 
(0.9409173720806777, 0.325) [b] 
(0.9410029885190339, 0.326) [b] 
(0.9410125086847715, 0.327) [b] 
(0.9410141068582875, 0.328) [b] 
(0.9410718694153651, 0.329) [b] 
(0.9410861655345948, 0.33) [b] 
(0.9410886769501199, 0.331) [b] 
(0.9411135627948687, 0.332) [b] 
(0.9411382203291152, 0.333) [b] 
(0.9411477177184037, 0.334) [b] 
(0.9411502291339288, 0.335) [b] 
(0.9411665530618366, 0.336) [b] 
(0.941169292787864, 0.337) [b] 
(0.9411889560298731, 0.338) [b] 
(0.9415544811440283, 0.339) [b] 
(0.9415588190435716, 0.341) [b] 
(0.9416449100599382, 0.342) [b] 
(0.9416606634845958, 0.343) [b] 
(0.9416613484161026, 0.344) [b] 
(0.9416668278681574, 0.345) [b] 
(0.941668197731171, 0.346) [b] 
(0.941674749860379, 0.349) [b] 
(0.9416766218972956, 0.35) [b] 
(0.9416818958152973, 0.352) [b] 
(0.9416880601988589, 0.353) [b] 
(0.9417051675210238, 0.354) [b] 
(0.9417102359049724, 0.355) [b] 
(0.941710692525977, 0.356) [b] 
(0.9417319977303515, 0.357) [b] 
(0.9417326826618584, 0.358) [b] 
(0.9417329109723607, 0.359) [b] 
(0.9417379338034109, 0.36) [b] 
(0.9417463812919953, 0.361) [b] 
(0.9417722031552024, 0.362) [b] 
(0.9417753995022343, 0.363) [b] 
(0.9417758561232389, 0.364) [b] 
(0.941776769365248, 0.365) [b] 
(0.9417790524702708, 0.366) [b] 
(0.9418147600144235, 0.367) [b] 
(0.941815901566935, 0.368) [b] 
(0.9418250106645664, 0.369) [b] 
(0.9418290683975943, 0.37) [b] 
(0.9418315798131194, 0.371) [b] 
(0.9418318081236217, 0.372) [b] 
(0.941833248568431, 0.373) [b] 
(0.9418350750524492, 0.374) [b] 
(0.9418452191773, 0.375) [b] 
(0.9418454474878023, 0.377) [b] 
(0.9418739863005877, 0.378) [b] 
(0.9419118858439667, 0.38) [b] 
(0.9419196484010443, 0.381) [b] 
(0.9419201050220488, 0.382) [b] 
(0.9419210182640579, 0.383) [b] 
(0.9419226164375739, 0.384) [b] 
(0.9419230730585785, 0.385) [b] 
(0.9419239863005876, 0.386) [b] 
(0.9420617654095915, 0.387) [b] 
(0.942065418377628, 0.388) [b] 
(0.9420663316196372, 0.389) [b] 
(0.942067016551144, 0.39) [b] 
(0.9420803041131748, 0.391) [b] 
(0.942082130597193, 0.392) [b] 
(0.9422661488620332, 0.393) [b] 
(0.9422757373571183, 0.394) [b] 
(0.94228722121158, 0.395) [b] 
(0.9422881344535892, 0.396) [b] 
(0.9422931410119274, 0.397) [b] 
(0.942293597632932, 0.398) [b] 
(0.9422954696698486, 0.399) [b] 
(0.9425027756059217, 0.4) [b] 
(0.9425055836612966, 0.401) [b] 
(0.9425094649398353, 0.402) [b] 
(0.9425122729952102, 0.403) [b] 
(0.9425188939997764, 0.404) [b] 
(0.9425275236999541, 0.405) [b] 
(0.9425419072615979, 0.406) [b] 
(0.9425430488141093, 0.407) [b] 
(0.9425481627509562, 0.408) [b] 
(0.9425545554450201, 0.409) [b] 
(0.9425790302216627, 0.41) [b] 
(0.9425974827292128, 0.411) [b] 
(0.9426077567018155, 0.412) [b] 
(0.9427086699438246, 0.417) [b] 
(0.9427099295879752, 0.42) [b] 
(0.942710874321088, 0.421) [b] 
(0.942777769298257, 0.422) [b] 
(0.9427999154169784, 0.427) [b] 
(0.9428006003484852, 0.428) [b] 
(0.9428201248328184, 0.429) [b] 
(0.9428390746045079, 0.43) [b] 
(0.9429134766837168, 0.431) [b] 
(0.9429139333047214, 0.433) [b] 
(0.9429223807933058, 0.434) [b] 
(0.9429289329225138, 0.435) [b] 
(0.9429396629701103, 0.437) [b] 
(0.9429433159381468, 0.438) [b] 
(0.9429505617135162, 0.439) [b] 
(0.9429543057873494, 0.441) [b] 
(0.9429622332353453, 0.444) [b] 
(0.9429996739736767, 0.449) [b] 
(0.942999902284179, 0.45) [b] 
(0.9430218200923981, 0.453) [b] 
(0.9430220484029004, 0.454) [b] 
(0.9430248564582753, 0.456) [b] 
(0.9430278244948049, 0.459) [b] 
(0.9430324590341981, 0.46) [b] 
(0.9430386234177597, 0.461) [b] 
(0.9430413631437871, 0.462) [b] 
(0.9430891000851596, 0.465) [b] 
(0.9430928441589927, 0.468) [b] 
(0.9431340289711573, 0.472) [b] 
(0.9431559467793764, 0.475) [b] 
(0.9432244399300613, 0.476) [b] 
(0.9432482222740491, 0.479) [b] 
(0.9432957869620248, 0.48) [b] 
(0.9433106271446732, 0.49) [b] 
(0.94331131207618, 0.501) [b] 
(0.9433679330807462, 0.506) [b] 
(0.9434474222304773, 0.51) [b] 
(0.9434476505409796, 0.514) [b] 
(0.9434478788514818, 0.515) [b] 
(0.9434503857609282, 0.517) [b] 
(0.9434508423819328, 0.518) [b] 
(0.9434515273134396, 0.519) [b] 
(0.9434517556239419, 0.52) [b] 
(0.9434519839344442, 0.521) [b] 
(0.9434556369024807, 0.524) [b] 
(0.9434565501444898, 0.525) [b] 
(0.9434602031125263, 0.529) [b] 
(0.9434622579070469, 0.533) [b] 
(0.943468878911613, 0.535) [b] 
(0.9434748149846723, 0.536) [b] 
(0.943479381194718, 0.54) [b] 
(0.9434974177243983, 0.542) [b] 
(0.9435072350759965, 0.543) [b] 
(0.9435118012860422, 0.544) [b] 
(0.9435131711490559, 0.545) [b] 
(0.9435145410120696, 0.546) [b] 
(0.9435149976330741, 0.548) [b] 
(0.943532834391065, 0.549) [b] 
(0.9435506711490558, 0.55) [b] 
(0.9435584337061333, 0.553) [b] 
(0.9435636848476858, 0.554) [b] 
(0.943564598089695, 0.555) [b] 
(0.9435828914686903, 0.557) [b] 
(0.9435904257152656, 0.56) [b] 
(0.9435917955782793, 0.565) [b] 
(0.9435922521992839, 0.569) [b] 
(0.9436692107955591, 0.57) [b] 
(0.9436703523480705, 0.572) [b] 
(0.9436881891060613, 0.573) [b] 
(0.9436927102553194, 0.585) [b] 
(0.9436976894282485, 0.594) [b] 
(0.9436983743597553, 0.595) [b] 
(0.9436988264746812, 0.596) [b] 
(0.9437015662007086, 0.601) [b] 
(0.9437097853787908, 0.605) [b] 
(0.9437113835523068, 0.608) [b] 
(0.9437120684838136, 0.61) [b] 
(0.9437257671139506, 0.615) [b] 
(0.9437531643742246, 0.617) [b] 
(0.9437661488577668, 0.627) [b] 
(0.9437671425403853, 0.637) [b] 
(0.9439805547488741, 0.652) [b] 
(0.9439810113698787, 0.666) [b] 
(0.9439816963013855, 0.678) [b] 
(0.9439862625114311, 0.682) [b] 
(0.943986947442938, 0.693) [b] 
(0.943999604015285, 0.695) [b] 
(0.9440045078042409, 0.698) [b] 
(0.9440051927357478, 0.716) [b] 
(0.9440200329183962, 0.723) [b] 
(0.9442727579021328, 0.725) [b] 
(0.9442732145231374, 0.727) [b] 
(0.9442760225785123, 0.739) [b] 
(0.9442771641310237, 0.761) [b] 
(0.9444624957857639, 0.771) [b] 
(0.9444631807172708, 0.772) [b] 
(0.9444697328464787, 0.776) [b] 
(0.9444753489572284, 0.778) [b] 
(0.9444775096244424, 0.783) [b] 
(0.9444777379349447, 0.785) [b] 
(0.9444804776609721, 0.787) [b] 
(0.9444945179378463, 0.798) [b] 
(0.9445722074698839, 0.836) [b] 
(0.944576426327333, 0.847) [b] 
(0.9447354046267951, 0.852) [b] 
(0.9447358612477996, 0.859) [b] 
(0.9447365461793065, 0.869) [b] 
(0.944737002800311, 0.882) [b] 
(0.9447372311108133, 0.908) [b] 
(0.9447390575948316, 0.911) [b] 
(0.9450647920183144, 0.924) [b] 
(0.945108279733035, 0.934) [b] 
(0.9451300235903953, 0.935) [b] 
(0.9451517674477556, 0.938) [b] 
(0.9451601916138802, 0.959) [b] 
(0.9451658993764372, 0.981) [b] 
(0.9451665843079441, 1.021) [b] 
(0.9451680247527534, 1.027) [b] 
(0.9451708328081283, 1.055) [b] 
(0.9452146684245667, 1.086) [b] 
(0.9452585040410051, 1.087) [b] 
(0.9453530419052918, 1.093) [b] 
(0.9453684856822219, 1.124) [b] 
(0.945385334014471, 1.132) [b] 
(0.9453882149040898, 1.14) [b] 
(0.9454038869915221, 1.157) [b] 
(0.9454046072139268, 1.219) [b] 
(0.945405520455936, 1.239) [b] 
(0.9454064564743943, 1.249) [b] 
(0.9454083285113108, 1.251) [b] 
(0.945408503921134, 1.479) [b] 
(0.9454093809702496, 1.482) [b] 
(0.9454100826095422, 1.485) [b] 
(0.9454102580193653, 1.487) [b] 
(0.9454136826768995, 1.51) [b] 
(0.9454155091609178, 1.57) [b] 
(0.9454160353903871, 1.589) [b] 
(0.9454231130159578, 1.631) [b] 
(0.945432930367556, 1.636) [b] 
(0.9454334565970254, 1.688) [b] 
(0.9454338074166716, 1.698) [b] 
(0.9454346844657873, 1.699) [b] 
(0.9455230091441169, 1.805) [b] 
(0.945535734862764, 1.849) [b] 
(0.9455363408493662, 1.858) [b] 
(0.9455366438426673, 1.86) [b] 
(0.9455369468359685, 1.865) [b] 
(0.9455375528225707, 1.872) [b] 
(0.9455378558158718, 1.875) [b] 
(0.945540885748883, 1.884) [b] 
(0.9455411887421841, 1.886) [b] 
(0.9456951059347345, 1.89) [b] 
(0.9456996508342513, 1.903) [b] 
(0.9457002568208536, 1.905) [b] 
(0.9457038927404671, 1.912) [b] 
(0.9457051047136715, 1.918) [b] 
(0.9457069226734782, 1.921) [b] 
(0.9457081346466827, 1.926) [b] 
(0.945708362957185, 1.93) [b] 
(0.9457086659504861, 1.957) [b] 
(0.9457104839102928, 1.979) [b] 
(0.9457106593201159, 1.981) [b] 
(0.9457514290526664, 1.985) [b] 
(0.9457520350392686, 1.986) [b] 
(0.9486533589103838, 1.989) [b] 
(0.9490422832759409, 1.991) [b] 
(0.9496327414715006, 1.992) [b] 
(0.9497677033447713, 1.993) [b] 
(0.9498520545155656, 1.994) [b] 
(0.949852835911991, 1.996) [b] 
(0.9498546538717977, 1.999) [b] 
(0.9498883943401154, 2) [b] 
(0.9498890003267176, 2.002) [b] 
(0.9498893033200188, 2.008) [b] 
(0.9499061735541776, 2.009) [b] 
(0.949924466933173, 2.033) [b] 
(0.9499299208125932, 2.042) [b] 
(0.9499636612809109, 2.043) [b] 
(0.9499805315150698, 2.045) [b] 
(0.9499858328425221, 2.047) [b] 
(0.950002703076681, 2.048) [b] 
(0.9500060360029934, 2.049) [b] 
(0.9500397764713111, 2.054) [b] 
(0.9500412914378167, 2.06) [b] 
(0.9500581616719755, 2.063) [b] 
(0.95005937364518, 2.067) [b] 
(0.9500596766384811, 2.07) [b] 
(0.9500599796317822, 2.076) [b] 
(0.9500602826250834, 2.078) [b] 
(0.9500611916049867, 2.081) [b] 
(0.9500801413766762, 2.117) [b] 
(0.9500988628378634, 2.133) [b] 
(0.9501261322349642, 2.144) [b] 
(0.9501587806367907, 2.148) [b] 
(0.9501733925089368, 2.182) [b] 
(0.9501754473034574, 2.196) [b] 
(0.9501757981231036, 2.211) [b] 
(0.9501759735329267, 2.213) [b] 
(0.9501784292704505, 2.257) [b] 
(0.9501789554999198, 2.258) [b] 
(0.950179306319566, 2.259) [b] 
(0.95020295035637, 2.265) [b] 
(0.9502033011760163, 2.268) [b] 
(0.95020856347071, 2.27) [b] 
(0.9502124224868188, 2.272) [b] 
(0.9508703616190141, 2.281) [b] 
(0.9511571555997145, 2.282) [b] 
(0.951325857941303, 2.283) [b] 
(0.9513595984096207, 2.284) [b] 
(0.9513623253493307, 2.285) [b] 
(0.9513753098328729, 2.287) [b] 
(0.9514596610036672, 2.291) [b] 
(0.9514611759701728, 2.293) [b] 
(0.9515092194208902, 2.302) [b] 
(0.9515167942534182, 2.305) [b] 
(0.9515674049558948, 2.306) [b] 
(0.9516011454242125, 2.307) [b] 
(0.9516271143912968, 2.309) [b] 
(0.9516513209468881, 2.311) [b] 
(0.9516531389066948, 2.318) [b] 
(0.9516534279215577, 2.324) [b] 
(0.9516537169364206, 2.326) [b] 
(0.9516710515952655, 2.328) [b] 
(0.9516743845215779, 2.341) [b] 
(0.9516745771981533, 2.344) [b] 
(0.9516749280177995, 2.348) [b] 
(0.9516804074698543, 2.353) [b] 
(0.9518875664709655, 2.376) [b] 
(0.9518876628092532, 2.379) [b] 
(0.951904533043412, 2.381) [b] 
(0.9519056346998908, 2.385) [b] 
(0.9519058273764661, 2.386) [b] 
(0.9519059237147538, 2.387) [b] 
(0.9519065980827675, 2.389) [b] 
(0.9519075070626708, 2.39) [b] 
(0.9519100173569462, 2.391) [b] 
(0.9519122386363588, 2.393) [b] 
(0.9519131502189788, 2.395) [b] 
(0.9519132465572665, 2.396) [b] 
(0.951945997031404, 2.4) [b] 
(0.9519482183108166, 2.407) [b] 
(0.9519679805725416, 2.408) [b] 
(0.9519761657859404, 2.41) [b] 
(0.951979143101771, 2.42) [b] 
(0.9519803550749755, 2.428) [b] 
(0.95203560916926, 2.43) [b] 
(0.9520357845790831, 2.433) [b] 
(0.952036073593946, 2.436) [b] 
(0.9520380842303281, 2.443) [b] 
(0.9520470161778202, 2.447) [b] 
(0.9520472088543955, 2.449) [b] 
(0.9520614170500687, 2.451) [b] 
(0.9520619432795381, 2.452) [b] 
(0.9520621359561134, 2.456) [b] 
(0.9520623286326888, 2.457) [b] 
(0.9520632376125922, 2.458) [b] 
(0.9520635406058933, 2.46) [b] 
(0.9520717848675802, 2.466) [b] 
(0.9520723908541824, 2.487) [b] 
(0.9520726798690453, 2.488) [b] 
(0.9520728725456207, 2.49) [b] 
(0.9520729688839084, 2.492) [b] 
(0.952073065222196, 2.493) [b] 
(0.9520736712087983, 2.496) [b] 
(0.9520739742020994, 2.506) [b] 
(0.9520745801887016, 2.513) [b] 
(0.9520751861753038, 2.516) [b] 
(0.9520781634911345, 2.517) [b] 
(0.9520784525059974, 2.521) [b] 
(0.9520793614859008, 2.522) [b] 
(0.9520802704658041, 2.525) [b] 
(0.9520805734591052, 2.528) [b] 
(0.9520808764524064, 2.534) [b] 
(0.9520967804347635, 2.543) [b] 
(0.9521126844171207, 2.544) [b] 
(0.9521179857445731, 2.55) [b] 
(0.9521184674360114, 2.579) [b] 
(0.9521415267967419, 2.592) [b] 
(0.9521468281241943, 2.629) [b] 
(0.9521521294516466, 2.63) [b] 
(0.9521610145692967, 2.637) [b] 
(0.9521654571281217, 2.639) [b] 
(0.9521698996869468, 2.648) [b] 
(0.9529162684843457, 2.659) [b] 
(0.9529334454500348, 2.661) [b] 
(0.9529387467774871, 2.662) [b] 
(0.9529430410189094, 2.668) [b] 
(0.9529433918385556, 2.693) [b] 
(0.9529435672483787, 2.732) [b] 
(0.9529604374825376, 2.74) [b] 
(0.9529631772085649, 2.767) [b] 
(0.9529810139665558, 2.817) [b] 
(0.9529988507245466, 2.82) [b] 
(0.952999992277058, 2.826) [b] 
(0.9536363791153574, 2.9) [b] 
(0.9536365545251805, 2.961) [b] 
(0.9536367299350036, 2.967) [b] 
(0.9536456618824957, 2.974) [b] 
(0.9536461185035002, 2.981) [b] 
(0.9537230770997754, 3.041) [b] 
(0.9537283784272278, 3.046) [b] 
(0.9537452486613867, 3.083) [b] 
(0.9541213273221677, 3.107) [b] 
(0.9541414994518271, 3.127) [b] 
(0.9541424036816788, 3.136) [b] 
(0.9541428557966046, 3.137) [b] 
(0.9541430312064277, 3.147) [b] 
(0.9541539066154615, 3.15) [b] 
(0.9541567131726315, 3.153) [b] 
(0.9541568885824546, 3.16) [b] 
(0.9541737588166135, 3.176) [b] 
(0.9541829896994551, 3.205) [b] 
(0.954183441814381, 3.227) [b] 
(0.9541838939293068, 3.251) [b] 
(0.9541861152087193, 3.272) [b] 
(0.9544355479772038, 3.285) [b] 
(0.9544364250263194, 3.297) [b] 
(0.9544367758459656, 3.31) [b] 
(0.9545316127492166, 3.311) [b] 
(0.9545484829833755, 3.35) [b] 
(0.9545822234516932, 3.354) [b] 
(0.954599093685852, 3.356) [b] 
(0.9546120781693942, 3.358) [b] 
(0.9546380471364785, 3.366) [b] 
(0.9548467848259978, 3.424) [b] 
(0.9548476618751135, 3.435) [b] 
(0.9549215089378466, 3.499) [b] 
(0.9549492015863714, 3.5) [b] 
(0.9549676633520547, 3.505) [b] 
(0.9550648403940663, 3.54) [b] 
(0.9550657536360754, 3.544) [b] 
(0.9550835903940662, 3.556) [b] 
(0.955101427152057, 3.57) [b] 
(0.9551475815662652, 3.59) [b] 
(0.9551568124491069, 3.591) [b] 
(0.9551660433319485, 3.596) [b] 
(0.9551838800899394, 3.631) [b] 
(0.9552552271219028, 3.632) [b] 
(0.9552730638798936, 3.633) [b] 
(0.9552909006378845, 3.634) [b] 
(0.9553087373958753, 3.636) [b] 
(0.9553255767388954, 3.64) [b] 
(0.9553434134968862, 3.652) [b] 
(0.9553969237708588, 3.653) [b] 
(0.9554147605288497, 3.654) [b] 
(0.9554861075608131, 3.655) [b] 
(0.9555039443188039, 3.66) [b] 
(0.9555217810767948, 3.661) [b] 
(0.9555396178347856, 3.662) [b] 
(0.9555398461452879, 3.784) [b] 
(0.9555583079109712, 3.785) [b] 
(0.9555675387938128, 3.798) [b] 
(0.9555853755518037, 3.808) [b] 
(0.9556032123097945, 3.988) [b] 
(0.9556061803463242, 4.058) [b] 
(0.9556114816737765, 4.283) [b] 
(0.9556269254507066, 4.379) [b] 
(0.9557221374759854, 4.389) [b] 
(0.9557816449917846, 4.391) [b] 
(0.9558411525075838, 4.395) [b] 
(0.9558565962845139, 4.427) [b] 
(0.9558620501639341, 4.462) [b] 
(0.9558632621371386, 4.465) [b] 
(0.9558635651304397, 4.468) [b] 
(0.955864474110343, 4.47) [b] 
(0.9558687160165588, 4.479) [b] 
(0.9558720489428711, 4.483) [b] 
(0.955883950446031, 4.542) [b] 
(0.9558869804460614, 4.558) [b] 
(0.9558957672517938, 4.613) [b] 
(0.9558966762316972, 4.616) [b] 
(0.9559009624113395, 4.633) [b] 
(0.9559247654176591, 4.689) [b] 
(0.9559329462367894, 4.729) [b] 
(0.9559805522494288, 4.777) [b] 
(0.9560281582620682, 4.778) [b] 
(0.956115829494945, 4.78) [b] 
(0.9561377473031641, 4.781) [b] 
(0.9561486550620045, 4.822) [b] 
(0.9561510790084135, 4.828) [b] 
(0.9561522205609249, 4.831) [b] 
(0.9561591894068506, 4.986) [b] 
(0.9561594924001517, 4.995) [b] 
(0.9561607043733562, 5) [b] 
(0.9561610073666573, 5.003) [b] 
(0.9561613103599584, 5.006) [b] 
(0.9561900774832461, 5.05) [b] 
(0.956272269264068, 5.138) [b] 
(0.9562759051836816, 5.16) [b] 
(0.956277117156886, 5.484) [b] 
(0.9562777231434882, 5.487) [b] 
(0.9562955599014791, 5.52) [b] 
(0.9562963252735953, 5.916) [b] 
(0.9562970906457116, 5.987) [b] 
(0.956298232198223, 6.378) [b] 
(0.9562989171297298, 6.39) [b] 
(0.9562993737507344, 6.649) [b] 
(0.9562996767440355, 6.973) [b] 
(0.9563126612275776, 7) [b] 
(0.9563391576108213, 7.099) [b] 
(0.9563560278449802, 7.256) [b] 
(0.9564095381189528, 7.477) [b] 
(0.9564273748769436, 7.478) [b] 
(0.9564452116349345, 7.492) [b] 
(0.9564630483929253, 7.493) [b] 
(0.9564660783259364, 7.942) [b] 
(0.9564697142455499, 7.947) [b] 
(0.9564700172388511, 7.952) [b] 
(0.9564703202321522, 7.959) [b] 
(0.9564712292120555, 7.972) [b] 
(0.9564742591450667, 7.973) [b] 
(0.9564757741115723, 7.977) [b] 
(0.9564763800981745, 7.981) [b] 
(0.9564769860847767, 7.989) [b] 
(0.956517587187127, 8.019) [b] 
(0.9565233440598483, 8.044) [b] 
(0.9565254650129561, 8.049) [b] 
(0.9565260709995583, 8.053) [b] 
(0.9565263739928594, 8.056) [b] 
(0.9565272829727628, 8.063) [b] 
(0.956527888959365, 8.07) [b] 
(0.9565284949459673, 8.073) [b] 
(0.9565287979392684, 8.076) [b] 
(0.9565291009325695, 8.079) [b] 
(0.9565297069191717, 8.096) [b] 
(0.956530312905774, 8.134) [b] 
(0.9565333428387851, 8.149) [b] 
(0.9565351607985918, 8.152) [b] 
(0.956535766785194, 8.156) [b] 
(0.9565372817516996, 8.159) [b] 
(0.9565384937249041, 8.162) [b] 
(0.9565387967182052, 8.165) [b] 
(0.9565427356311198, 8.179) [b] 
(0.956543341617722, 8.182) [b] 
(0.9565439476043243, 8.188) [b] 
(0.9565460685574321, 8.193) [b] 
(0.9565475835239377, 8.197) [b] 
(0.9565478865172388, 8.203) [b] 
(0.9565481895105399, 8.206) [b] 
(0.956548492503841, 8.209) [b] 
(0.9565487954971421, 8.227) [b] 
(0.9565494014837443, 8.232) [b] 
(0.9565506134569488, 8.235) [b] 
(0.9565509164502499, 8.239) [b] 
(0.9565527344100566, 8.243) [b] 
(0.9565530374033577, 8.25) [b] 
(0.9565533403966588, 8.261) [b] 
(0.95655364338996, 8.263) [b] 
(0.9565579295696023, 8.869) [b] 
(0.9566007913660258, 8.87) [b] 
(0.9566038213660562, 8.875) [b] 
(0.956615722869216, 9.248) [b] 
(0.9566164078007229, 9.537) [b] 
(0.956717629205676, 9.559) [b] 
(0.9567344994398349, 9.605) [b] 
(0.9567682399081526, 9.607) [b] 
(0.9567851101423115, 9.671) [b] 
(0.9568357208447881, 9.673) [b] 
(0.9568525910789469, 9.703) [b] 
(0.9568530431938728, 9.783) [b] 
(0.9568543995386501, 9.803) [b] 
(0.9568553127806593, 10.052) [b] 
(0.9568571392646775, 10.06) [b] 
(0.9568573675751798, 10.149) [b] 
(0.956860792232714, 10.215) [b] 
(0.9568612488537186, 10.329) [b] 
(0.9568617009686444, 10.337) [b] 
(0.9568696234150958, 10.402) [b] 
(0.9568775458615473, 10.404) [b] 
(0.9568797065287613, 10.563) [b] 
(0.9568811469735706, 10.565) [b] 
(0.9568818671959753, 10.631) [b] 
(0.9568987374301342, 10.692) [b] 
(0.9569047974301949, 10.962) [b] 
(0.9569055176525996, 10.997) [b] 
(0.9569069580974089, 11.007) [b] 
(0.9569076783198136, 11.521) [b] 
(0.9569081349408182, 11.842) [b] 
(0.9569088551632229, 12.123) [b] 
(0.9569095753856276, 12.149) [b] 
(0.9569102956080323, 12.785) [b] 
(0.9569281323660231, 12.819) [b] 
(0.956945969124014, 12.822) [b] 
(0.9569464212389398, 12.998) [b] 
(0.9569468733538656, 13.008) [b] 
(0.9570238319501408, 13.438) [b] 
(0.9570280508075899, 13.757) [b] 
(0.9570285029225157, 13.788) [b] 
(0.9570289595435203, 14.376) [b] 
(0.9570330691325614, 14.471) [b] 
(0.9570415068474594, 15.551) [b] 
(0.9570457257049084, 15.555) [b] 
(0.9570499445623575, 15.612) [b] 
(0.9570503966772833, 15.789) [b] 
(0.957051766540297, 15.83) [b] 
(0.9570559853977461, 16.037) [b] 
(0.9570602042551951, 16.095) [b] 
(0.9570739028853321, 16.249) [b] 
(0.9570876015154691, 16.379) [b] 
(0.9571645601117443, 16.698) [b] 
(0.9571700395637991, 16.737) [b] 
(0.9571854833407292, 16.782) [b] 
(0.9571859399617337, 16.926) [b] 
(0.9571890287171197, 17.188) [b] 
(0.9572011969570774, 17.628) [b] 
(0.957244684671798, 17.864) [b] 
(0.9572625214297888, 17.908) [b] 
(0.9572803581877797, 18.342) [b] 
(0.9572858376398344, 19.787) [b] 
(0.9572906321603823, 19.838) [b] 
(0.9572948510178314, 19.943) [b] 
(0.957295307638836, 20.17) [b] 
(0.9572996018802582, 21.134) [b] 
(0.9573164721144171, 21.475) [b] 
(0.9573603077308553, 23.146) [b] 
(0.9575794858130472, 23.147) [b] 
(0.9575809262578565, 24.225) [b] 
(0.9575811545683588, 24.652) [b] 
(0.9575930560715187, 26.687) [b] 
(0.9576049575746786, 26.758) [b] 
(0.9576168590778384, 26.766) [b] 
(0.9576170873883407, 27.385) [b] 
(0.9576178076107454, 27.6) [b] 
(0.9576205473367728, 28.924) [b] 
(0.9576207756472751, 28.972) [b] 
(0.9576386124052659, 28.977) [b] 
(0.9576564491632568, 29.031) [b] 
(0.9576571693856615, 29.199) [b] 
(0.957657626006666, 30.082) [b] 
(0.957662420527214, 34.446) [b] 
(0.9576635620797254, 34.498) [b] 
(0.9576640187007299, 34.621) [b] 
(0.9576653885637436, 35.129) [b] 
(0.9576658451847482, 35.248) [b] 
(0.9576677172216648, 36.648) [b] 
(0.9576707472216951, 37.347) [b] 
(0.9576737772217254, 37.419) [b] 
(0.9576768072217557, 37.591) [b] 
(0.9576772638427603, 37.931) [b] 
(0.9576951006007511, 40.114) [b] 
(0.9576993948421734, 41.161) [b] 
(0.9577148386191034, 43.44) [b] 
(0.95771849158714, 44.379) [b] 
(0.9577207746921628, 44.429) [b] 
(0.9577212313131673, 44.526) [b] 
(0.957722135543019, 44.642) [b] 
(0.9577230397728707, 44.845) [b] 
(0.957724396117648, 44.859) [b] 
(0.9577310599558856, 45.259) [b] 
(0.9577355025147106, 45.266) [b] 
(0.9577399450735357, 45.267) [b] 
(0.9577443876323607, 45.414) [b] 
(0.9577466089117732, 45.42) [b] 
(0.9577468372222755, 45.56) [b] 
(0.9577646739802663, 46.263) [b] 
(0.9577668952596788, 47.201) [b] 
(0.9577671235701811, 47.205) [b] 
(0.9577682651226925, 47.266) [b] 
(0.9577698632962085, 47.31) [b] 
(0.9577700916067108, 47.711) [b] 
(0.9577703199172131, 48.262) [b] 
(0.9578059934331948, 51.225) [b] 
(0.9578238301911857, 51.226) [b] 
(0.9578416669491765, 51.229) [b] 
(0.9578595037071673, 51.239) [b] 
(0.9578773404651582, 51.24) [b] 
(0.9578910390952952, 52.92) [b] 
(0.9579047377254322, 52.922) [b] 
(0.9579184363555692, 52.923) [b] 
(0.9579321349857062, 52.924) [b] 
(0.9579458336158432, 52.925) [b] 
(0.9579595322459802, 52.926) [b] 
(0.9579732308761172, 52.928) [b] 
(0.9579869295062542, 52.929) [b] 
(0.9580828199172131, 52.93) [b] 
(0.9580965185473501, 52.932) [b] 
(0.9581102171774871, 52.934) [b] 
(0.9581239158076241, 52.937) [b] 
(0.9581376144377611, 52.951) [b] 
(0.9581513130678981, 52.954) [b] 
(0.9581650116980351, 52.986) [b] 
(0.958315696629542, 53.046) [b] 
(0.958356792519953, 53.049) [b] 
(0.95837049115009, 53.103) [b] 
(0.958384189780227, 54.039) [b] 
(0.9583849100026317, 55.551) [b] 
(0.9583853666236363, 55.915) [b] 
(0.958411589289333, 60.654) [b] 
(0.9584120414042588, 60.655) [b] 
(0.958413849863962, 60.657) [b] 
(0.9584147540938137, 60.658) [b] 
(0.9584174667833686, 60.668) [b] 
(0.9584205555387546, 60.675) [b] 
(0.9584246245730869, 60.705) [b] 
(0.9584250766880127, 60.71) [b] 
(0.9584255288029385, 60.731) [b] 
(0.9584259809178644, 60.738) [b] 
(0.9584262092283666, 62.16) [b] 
(0.9584289219179215, 62.674) [b] 
(0.9584293740328473, 63.108) [b] 
(0.958430278262699, 63.131) [b] 
(0.958443976892836, 63.266) [b] 
(0.958457675522973, 63.269) [b] 
(0.95847137415311, 63.273) [b] 
(0.958485072783247, 63.278) [b] 
(0.958498771413384, 63.439) [b] 
(0.958526168673658, 63.456) [b] 
(0.9585266207885839, 63.666) [b] 
(0.9585270729035097, 63.674) [b] 
(0.9585275250184355, 63.835) [b] 
(0.9585279771333614, 64.376) [b] 
(0.9585284292482872, 64.384) [b] 
(0.9585292962928761, 65.661) [b] 
(0.9585429949230131, 68.723) [b] 
(0.9585566935531501, 68.728) [b] 
(0.9585703921832871, 68.738) [b] 
(0.95869367985452, 68.739) [b] 
(0.958734775744931, 68.741) [b] 
(0.958748474375068, 68.751) [b] 
(0.958762173005205, 68.764) [b] 
(0.9587758716353421, 68.833) [b] 
(0.9588032688956161, 69.07) [b] 
(0.9588169675257531, 69.075) [b] 
(0.9588299520092952, 69.12) [b] 
(0.9588468222434541, 71.289) [b] 
(0.9588605208735911, 71.592) [b] 
(0.9588635508736214, 72.723) [b] 
(0.958865092286224, 72.781) [b] 
(0.9588681222862543, 73.168) [b] 
(0.9588818209163913, 74.114) [b] 
(0.9589777113273502, 74.115) [b] 
(0.9589914099574872, 74.12) [b] 
(0.9590188072177612, 74.121) [b] 
(0.9590325058478982, 74.122) [b] 
(0.9590736017383092, 74.123) [b] 
(0.9591009989985833, 74.124) [b] 
(0.9591283962588573, 74.125) [b] 
(0.9591557935191313, 74.126) [b] 
(0.9591694921492683, 74.14) [b] 
(0.9591831907794053, 74.141) [b] 
(0.9591968894095423, 74.142) [b] 
(0.9592123331864724, 77.075) [b] 
(0.9592277769634024, 77.698) [b] 
(0.9592432207403325, 78.896) [b] 
(0.9592729011056293, 78.934) [b] 
(0.9592733577266339, 78.998) [b] 
(0.9592735860371362, 79.337) [b] 
(0.9592738143476385, 79.397) [b] 
(0.9592763257631636, 79.721) [b] 
(0.9592917695400937, 81.226) [b] 
(0.9593072133170237, 81.632) [b] 
(0.9593226570939538, 82.939) [b] 
(0.9593228854044561, 86.845) [b] 
(0.9593233420254607, 86.887) [b] 
(0.9593235703359629, 86.929) [b] 
(0.9593237986464652, 86.974) [b] 
(0.959327679925004, 87.092) [b] 
(0.9593327027560542, 87.141) [b] 
(0.9593338443085656, 87.385) [b] 
(0.9593347803270239, 88.81) [b] 
(0.9593440112098656, 89.931) [b] 
(0.9593453810728793, 90.185) [b] 
(0.9593458376938838, 92.617) [b] 
(0.9593460660043861, 92.902) [b] 
(0.9593468313765023, 94.082) [b] 
(0.9593646681344932, 96.353) [b] 
(0.9593655813765023, 100.649) [b] 
(0.9593658096870046, 101.038) [b] 
(0.9593711110144569, 101.176) [b] 
(0.9593764123419093, 101.177) [b] 
(0.9593870149968141, 101.181) [b] 
(0.9593923163242665, 101.182) [b] 
(0.9593976176517188, 101.183) [b] 
(0.9594007064071048, 101.961) [b] 
(0.9594099372899465, 108.84) [b] 
(0.9595955098656939, 119.842) [b] 
(0.9596123800998527, 119.871) [b] 
(0.9596292503340116, 119.877) [b] 
(0.9596461205681704, 121.285) [b] 
(0.9596629908023293, 121.347) [b] 
(0.9596798610364882, 121.49) [b] 
(0.9596826007625155, 121.971) [b] 
(0.9596830573835201, 121.987) [b] 
(0.9596844272465338, 122.193) [b] 
(0.9596857971095475, 122.651) [b] 
(0.9597026673437064, 126.147) [b] 
(0.9597058636907383, 127.545) [b] 
(0.9597067769327474, 127.589) [b] 
(0.9597070052432497, 128.296) [b] 
(0.9597076901747565, 128.931) [b] 
(0.9597083751062634, 128.935) [b] 
(0.9597090600377702, 128.944) [b] 
(0.9597111148322908, 128.95) [b] 
(0.9597138545583181, 128.954) [b] 
(0.959714539489825, 128.965) [b] 
(0.9597152244213318, 128.969) [b] 
(0.9597159093528387, 128.982) [b] 
(0.9597165942843455, 129.083) [b] 
(0.9597172792158524, 129.095) [b] 
(0.9597341494500112, 129.252) [b] 
(0.9597348343815181, 130.355) [b] 
(0.9597355193130249, 130.365) [b] 
(0.9597362042445318, 131.521) [b] 
(0.9597392929999178, 140.31) [b] 
(0.9597669856484426, 140.902) [b] 
(0.9597762165312843, 140.904) [b] 
(0.959785447414126, 140.907) [b] 
(0.9597946782969676, 140.913) [b] 
(0.9598316018283342, 140.957) [b] 
(0.9598408327111758, 140.958) [b] 
(0.9598500635940175, 141.171) [b] 
(0.9598592944768591, 141.175) [b] 
(0.9598685253597008, 142.121) [b] 
(0.9598962180082257, 142.513) [b] 
(0.9599054488910673, 142.517) [b] 
(0.959914679773909, 142.53) [b] 
(0.9599239106567506, 142.551) [b] 
(0.9599331415395923, 142.702) [b] 
(0.959942372422434, 145.881) [b] 
(0.9599445937018465, 146.686) [b] 
(0.9599455069438556, 150.505) [b] 
(0.9600224655401308, 150.854) [b] 
(0.960099424136406, 151.297) [b] 
(0.9601172608943969, 165.45) [b] 
(0.9601264917772385, 166.609) [b] 
(0.9601357226600802, 166.885) [b] 
(0.9601368642125916, 167.359) [b] 
(0.9601377774546007, 167.448) [b] 
(0.9601507619381429, 178.215) [b] 
(0.9601537919381732, 179.179) [b] 
(0.9601626960477623, 195.517) [b] 
(0.9601652074632874, 195.598) [b] 
(0.9601754814358902, 195.713) [b] 
(0.9601830156824654, 196.124) [b] 
(0.9601883170099178, 207.772) [b] 
(0.9601936183373702, 208.472) [b] 
(0.9601989196648225, 208.563) [b] 
(0.9601993717797483, 212.559) [b] 
(0.9601998238946742, 212.58) [b] 
(0.9602119438947954, 213.055) [b] 
(0.9602149738948257, 213.076) [b] 
(0.9602184560097818, 213.097) [b] 
(0.9602189081247077, 213.201) [b] 
(0.9602198123545593, 215.029) [b] 
(0.9602202644694852, 215.03) [b] 
(0.960220716584411, 215.034) [b] 
(0.9602211686993368, 215.166) [b] 
(0.9602225250441142, 220.939) [b] 
(0.96022297715904, 221.034) [b] 
(0.9603768943515905, 222.134) [b] 
(0.9603775792830973, 224.512) [b] 
(0.9603954160410881, 224.984) [b] 
(0.960412286275247, 225.385) [b] 
(0.9604127428962516, 235.194) [b] 
(0.9604136561382607, 240.406) [b] 
(0.9604255576414206, 244.389) [b] 
(0.9604612621509001, 251.125) [b] 
(0.9604969666603796, 251.126) [b] 
(0.9605088681635395, 251.134) [b] 
(0.9605207696666994, 251.177) [b] 
(0.9605326711698593, 251.224) [b] 
(0.9605445726730192, 251.225) [b] 
(0.9605564741761791, 251.227) [b] 
(0.960568375679339, 251.742) [b] 
(0.9605704304738595, 261.37) [b] 
(0.9605713437158686, 261.432) [b] 
(0.9605720286473755, 261.764) [b] 
(0.9605722569578777, 262.362) [b] 
(0.96057248526838, 262.724) [b] 
(0.9605731701998869, 262.938) [b] 
(0.9605736268208914, 263.32) [b] 
(0.9605743117523983, 265.702) [b] 
(0.9605747683734028, 265.813) [b] 
(0.9606117798460823, 268.316) [b] 
(0.960612832305021, 271.376) [b] 
(0.9606135172365279, 277.677) [b] 
(0.9606188185639802, 285.142) [b] 
(0.9606241198914326, 285.227) [b] 
(0.9606455910985439, 285.87) [b] 
(0.9606498853399661, 285.876) [b] 
(0.9606570771427142, 286.899) [b] 
(0.9606623784701666, 287.649) [b] 
(0.9606626067806688, 288.289) [b] 
(0.9606628350911711, 290.107) [b] 
(0.9606630634016734, 299.732) [b] 
(0.9607400219979486, 318.594) [b] 
(0.9607578587559394, 320.174) [b] 
(0.9607631600833918, 329.01) [b] 
(0.9607696420850339, 329.545) [b] 
(0.9607746836418667, 329.547) [b] 
(0.9607754038642714, 329.551) [b] 
(0.9607761240866761, 329.599) [b] 
(0.9607768443090808, 329.607) [b] 
(0.9607790049762949, 329.78) [b] 
(0.9607797251986996, 329.783) [b] 
(0.9607804454211043, 329.788) [b] 
(0.9607826060883183, 329.789) [b] 
(0.960783326310723, 329.807) [b] 
(0.9607840465331278, 330.029) [b] 
(0.9607905285347699, 334.047) [b] 
(0.9607919689795792, 334.048) [b] 
(0.9607934094243885, 334.089) [b] 
(0.9607941296467932, 334.137) [b] 
(0.9607948498691979, 334.139) [b] 
(0.9607955700916027, 334.144) [b] 
(0.9608431761042421, 334.179) [b] 
(0.960855077607402, 334.181) [b] 
(0.9608557978298067, 334.214) [b] 
(0.9608676993329666, 334.24) [b] 
(0.9608684195553713, 334.354) [b] 
(0.9608713968712019, 334.849) [b] 
(0.9608721170936066, 336.716) [b] 
(0.9608840185967665, 337.493) [b] 
(0.9609078216030862, 337.523) [b] 
(0.9609197231062461, 347.366) [b] 
(0.9609435261125657, 347.367) [b] 
(0.9609554276157256, 347.375) [b] 
(0.9609673291188855, 347.38) [b] 
(0.9609792306220454, 347.382) [b] 
(0.9610268366346848, 347.434) [b] 
(0.9610387381378447, 347.601) [b] 
(0.9610506396410046, 347.633) [b] 
(0.9610625411441645, 348.064) [b] 
(0.9610632613665692, 348.133) [b] 
(0.9610639815889739, 348.695) [b] 
(0.9610647018113786, 348.855) [b] 
(0.9610654220337833, 350.298) [b] 
(0.9610726138365314, 357.248) [b] 
(0.9610755911523621, 360.757) [b] 
(0.9611550803020932, 365.74) [b] 
(0.9612080730685806, 365.808) [b] 
(0.9612345694518243, 366.344) [b] 
(0.961261065835068, 367.363) [b] 
(0.9613145761090406, 386.248) [b] 
(0.9613502496250224, 386.252) [b] 
(0.9613680863830132, 386.287) [b] 
(0.961385923141004, 386.399) [b] 
(0.9614013669179341, 389.745) [b] 
(0.9614044556733201, 391.7) [b] 
(0.96141635717648, 400.934) [b] 
(0.9614165854869823, 403.723) [b] 
(0.9614170421079868, 404.363) [b] 
(0.9614192633873994, 413.132) [b] 
(0.9614214846668119, 413.139) [b] 
(0.9614237059462244, 413.206) [b] 
(0.9614259272256369, 413.887) [b] 
(0.9614271550943988, 425.949) [b] 
(0.9614363859772405, 428.309) [b] 
(0.9614512725563938, 439.988) [b] 
(0.9614542498722245, 440.249) [b] 
(0.9614661513753844, 445.446) [b] 
(0.961475382258226, 446.923) [b] 
(0.9614872837613859, 455.866) [b] 
(0.9616197656776043, 457.957) [b] 
(0.9616327501611465, 479.595) [b] 
(0.9616457346446886, 479.608) [b] 
(0.9616500288861108, 499.562) [b] 
(0.9616543231275331, 499.571) [b] 
(0.9616572911640627, 510.213) [b] 
(0.9616579760955696, 510.27) [b] 
(0.9616582044060719, 510.389) [b] 
(0.9616584327165741, 511.549) [b] 
(0.9616626515740232, 516.533) [b] 
(0.9616718824568649, 565.99) [b] 
(0.9616741655618877, 585.095) [b] 
(0.9616755354249014, 585.148) [b] 
(0.961684766307743, 591.058) [b] 
(0.9616848626460307, 592.289) [b] 
(0.9616917119610991, 598.773) [b] 
(0.9616958215501402, 598.841) [b] 
(0.9616962781711448, 598.967) [b] 
(0.9616967347921493, 601.252) [b] 
(0.9616971914131539, 601.378) [b] 
(0.9616974197236562, 608.801) [b] 
(0.9616976480341585, 612.041) [b] 
(0.9617033557967155, 612.728) [b] 
(0.9617127485890503, 707.245) [b] 
(0.9617170347686926, 710.571) [b] 
(0.9617188527284993, 723.737) [b] 
(0.961719572950904, 724.52) [b] 
(0.9617202931733088, 742.641) [b] 
(0.9617212064153179, 751.815) [b] 
(0.9617216630363224, 758.084) [b] 
(0.9617308939191641, 758.24) [b] 
(0.9617311222296664, 813.622) [b] 
(0.9617479924638253, 831.633) [b] 
(0.9617648626979841, 831.646) [b] 
(0.961781732932143, 831.648) [b] 
(0.9617826689506013, 833.747) [b] 
(0.9617833538821081, 877.861) [b] 
(0.961784038813615, 881.804) [b] 
(0.9617849748320733, 885.173) [b] 
(0.9617942057149149, 960.908) [b] 
(0.9618034365977566, 961.69) [b] 
(0.9618126674805982, 961.7) [b] 
(0.9618154072066256, 962.997) [b] 
(0.9618322774407845, 964.106) [b] 
(0.9618491476749433, 971.263) [b] 
(0.9618496042959479, 976.869) [b] 
(0.9618518874009707, 988.262) [b] 
(0.961856181642393, 993.361) [b] 
(0.9618584647474158, 1005.68) [b] 
(0.9618607478524386, 1008.28) [b] 
(0.9618609761629409, 1008.34) [b] 
(0.9618612044734431, 1008.46) [b] 
(0.9618614327839454, 1009.24) [b] 
(0.9618635537370532, 1065.1) [b] 
(0.9618696767139835, 1075.35) [b] 
(0.9618712074582161, 1075.39) [b] 
(0.9618719728303323, 1075.4) [b] 
(0.9618727382024486, 1075.57) [b] 
(0.9618735035745648, 1075.58) [b] 
(0.961874268946681, 1075.64) [b] 
(0.9618757996909136, 1077.18) [b] 
(0.9618765650630299, 1081.42) [b] 
(0.9618773304351461, 1081.96) [b] 
(0.9618780958072624, 1082.08) [b] 
(0.9618834534120763, 1088.84) [b] 
(0.9618842187841925, 1088.86) [b] 
(0.961885749528425, 1088.87) [b] 
(0.9618865149005412, 1088.95) [b] 
(0.9618895449005715, 1164.33) [b] 
(0.961901188736188, 1166.36) [b] 
(0.9619057549462336, 1166.43) [b] 
(0.9619066681882428, 1166.5) [b] 
(0.9619071248092473, 1166.57) [b] 
(0.9619075814302519, 1166.78) [b] 
(0.9619082663617587, 1166.85) [b] 
(0.961908494672261, 1177.79) [b] 
(0.9619263314302519, 1199.9) [b] 
(0.9619417752071819, 1286.62) [b] 
(0.9619478352072426, 1313.49) [b] 
(0.9619508652072729, 1313.51) [b] 
(0.9619538952073032, 1313.53) [b] 
(0.9619569252073336, 1414.54) [b] 
(0.9619718449998852, 1415.94) [b] 
(0.9619727492297369, 1415.97) [b] 
(0.9619759140342176, 1415.99) [b] 
(0.9619781746088466, 1416) [b] 
(0.9619786267237724, 1416.12) [b] 
(0.9619795309536241, 1416.95) [b] 
(0.9619799830685499, 1417.97) [b] 
(0.9619804351834758, 1417.98) [b] 
(0.9619813394133274, 1420.75) [b] 
(0.9619858605625855, 1420.78) [b] 
(0.9619863126775113, 1420.79) [b] 
(0.9619867647924372, 1421.05) [b] 
(0.9619904177604737, 1421.93) [b] 
(0.9619922442444919, 1421.95) [b] 
(0.9619927008654965, 1422.16) [b] 
(0.961993157486501, 1422.55) [b] 
(0.9619936096014269, 1422.63) [b] 
(0.9619940617163527, 1422.97) [b] 
(0.9619977146843892, 1423.07) [b] 
(0.9619981713053938, 1423.41) [b] 
(0.9619990845474029, 1429.05) [b] 
(0.9620004544104166, 1436.96) [b] 
(0.9620013676524257, 1437.04) [b] 
(0.9620022718822774, 1441.32) [b] 
(0.962007697261387, 1530.84) [b] 
(0.9620081493763128, 1530.86) [b] 
(0.9620086014912387, 1531.42) [b] 
(0.9620090536061645, 1531.99) [b] 
(0.9620104099509419, 1556.83) [b] 
(0.9620108620658677, 1561.14) [b] 
(0.9620113141807936, 1561.29) [b] 
(0.9620242986643357, 1562.99) [b] 
(0.9620247507792615, 1564.39) [b] 
(0.9620377352628037, 1651.84) [b] 
(0.9620386394926553, 1695.43) [b] 
(0.9620411509081804, 1700.18) [b] 
(0.9620413792186827, 1700.29) [b] 
(0.9620414755569704, 1746.56) [b] 
(0.9620471833195274, 1754.41) [b] 
(0.9620496947350525, 1754.49) [b] 
(0.9620524344610799, 1754.57) [b] 
(0.9620531193925868, 1754.71) [b] 
(0.9620551741871073, 1755.46) [b] 
(0.962056544050121, 1756.78) [b] 
(0.9620583705341392, 1757.07) [b] 
(0.9620590554656461, 1761.19) [b] 
(0.9620597756880508, 1811.62) [b] 
(0.9621005454206013, 1815.16) [b] 
(0.9622228546182529, 1815.17) [b] 
(0.9622636243508034, 1815.39) [b] 
(0.9623043940833539, 1820.34) [b] 
(0.962319837860284, 1851.82) [b] 
(0.962335281637214, 1851.83) [b] 
(0.9623355099477163, 1921.66) [b] 
(0.9623373364317346, 1970.32) [b] 
(0.9623380213632414, 1970.39) [b] 
(0.962338477984246, 1970.66) [b] 
(0.9623393912262551, 1970.79) [b] 
(0.9623403044682642, 1972.25) [b] 
(0.9623405327787665, 1980.49) [b] 
(0.9623412177102734, 2016.1) [b] 
(0.962341314048561, 2075.89) [b] 
(0.9623419989800679, 2218.5) [b] 
(0.9623422272905702, 2285.16) [b] 
(0.9623426839115747, 2285.56) [b] 
(0.9623431405325793, 2286.2) [b] 
(0.9624200991288545, 2440.28) [b] 
(0.9624534183200423, 2457.52) [b] 
(0.9624556395994548, 2457.53) [b] 
(0.9624578608788673, 2457.55) [b] 
(0.9624689672759299, 2464.93) [b] 
(0.9624734098347549, 2464.94) [b] 
(0.96247785239358, 2464.96) [b] 
(0.9624800736729925, 2465.08) [b] 
(0.962482294952405, 2465.09) [b] 
(0.9624845162318175, 2465.35) [b] 
(0.9624847445423198, 2468.15) [b] 
(0.9624877745423501, 2601.69) [b] 
(0.9624908045423805, 2602.6) [b] 
(0.9624938345424108, 2607.15) [b] 
(0.9624945547648155, 2758.15) [b] 
(0.9624952749872202, 2758.2) [b] 
(0.9624959952096249, 2758.61) [b] 
(0.9624967154320296, 2758.71) [b] 
(0.9624974356544344, 2759.21) [b] 
(0.9625133396367914, 3021.15) [b] 
(0.9625135679472937, 3032.95) [b] 
(0.962514476927197, 3269.75) [b] 
(0.9625175656825831, 3348.77) [b] 
(0.9625180223035876, 3379.06) [b] 
(0.9625200770981082, 3383.56) [b] 
(0.9625205337191127, 3384.79) [b] 
(0.9625209903401173, 3395.65) [b] 
(0.9625212186506196, 3398.79) [b] 
(0.962522360203131, 3407.7) [b] 
},{(0.9581894246575344, 0.001) [c] 
(0.9581894246575344, 4.454787273972602) [c] 
(0.9581894246575344, 3600) [c] 
}}}{legend pos=north west}}
	\caption{\label{fig:cactus}Accuracy over time, averaged across all data sets}
\end{figure}


\subsection{Factor analysis}

Finally, we report results of three variants, in order to analyse the impact of the factors described in Section~\ref{sec:ext}. For each variant, we report the average error (error), the ratio of optimality proofs (opt.) and the relative increase of cpu time  (cpu$^*$), on data sets for which an optimal tree has been found.

In the variant ``No heuristic'', the Gini impurity heuristic described in Section~\ref{sec:heuristic} is disabled, and replaced by simply selecting first the feature with minimum error. For shallow trees (depth 3 or 4), since in many cases the search space is completely exhausted, not computing the slightly more costly Gini impurity score may actually be a good choice and we observe run time reduction of about 15\% to 20\%. However, the accuracy of the trees decreases extremely rapidly for larger maximum depth. As a results, many less optimality proofs are obtained, and they take much longer to compute.

In the variant ``No preprocessing'', the preprocessing described in Section~\ref{sec:preprocessing} is disabled. The feature ordering is impacted by the removal of datapoints, and therefore it may happen that, by luck, a more acurate tree is found for the non-preprocessed data set than for the preprocessed one. However, in most cases, the preprocessing does pay off, yielding more optimality proofs, better accuracy, and shorter runtimes. We estimate that most of the gain is due to the removal of redundant features, and of inconsistent datapoints, whereas the fusion of datapoints accounts for only a slight speed-up.

In the variant ``No lower bound'', the lower bound technique described in Section~\ref{sec:lb} is disabled. In this case we observe an increase in computational time (up to 200\% for some data sets). However, the search space is explored in the same order, and it only slightly negatively affect accuracy and the number of optimality proofs in average.

\begin{table}[htbp]
\begin{center}
\begin{footnotesize}
\tabcolsep=7pt
\begin{tabular}{lrrrrrrrrrrrr}
\toprule
\multirow{2}{*}{$\mdepth$}&  \multicolumn{3}{c}{\budalg} & \multicolumn{3}{c}{\noheuristic} & \multicolumn{3}{c}{\nopreprocessing} & \multicolumn{3}{c}{\nolb}\\
\cmidrule(rr){2-4}\cmidrule(rr){5-7}\cmidrule(rr){8-10}\cmidrule(rr){11-13}
& \multicolumn{1}{c}{error} & \multicolumn{1}{c}{opt.} & \multicolumn{1}{c}{cpu} & \multicolumn{1}{c}{error} & \multicolumn{1}{c}{opt.} & \multicolumn{1}{c}{cpu$^*$} & \multicolumn{1}{c}{error} & \multicolumn{1}{c}{opt.} & \multicolumn{1}{c}{cpu$^*$} & \multicolumn{1}{c}{error} & \multicolumn{1}{c}{opt.} & \multicolumn{1}{c}{cpu$^*$} \\
\midrule

\texttt{3} & 793 & 0.93 & 67 & 817 & 0.93 & -1.6 & 793 & 0.93 & $\mathsmaller{+}$4.3 & 793 & 0.93 & -2.8\\
\texttt{4} & 696 & 0.78 & 388 & 789 & 0.79 & -61 & 696 & 0.72 & $\mathsmaller{+}$47 & 696 & 0.78 & $\mathsmaller{+}$13\\
\texttt{5} & 624 & 0.64 & 479 & 752 & 0.67 & -30 & 625 & 0.55 & $\mathsmaller{+}$276 & 624 & 0.64 & $\mathsmaller{+}$39\\
\texttt{7} & 508 & 0.48 & 1045 & 689 & 0.48 & $\mathsmaller{+}$17 & 509 & 0.43 & $\mathsmaller{+}$163 & 509 & 0.48 & $\mathsmaller{+}$30\\
\texttt{10} & 410 & 0.62 & 570 & 657 & 0.53 & $\mathsmaller{+}$66 & 410 & 0.43 & $\mathsmaller{+}$63 & 410 & 0.60 & $\mathsmaller{+}$8.0\\
\bottomrule
\end{tabular}

\end{footnotesize}
\end{center}
\caption{\label{tab:factor} Factor analysis}
\end{table}


% \subsection{Balancing size and accuracy}
%
% Most decision trees toolkits somehow try to balance size and accuracy. \budalg uses the standard approach to bound the maximum depth and searches for the tree with maximum accuracy within that limit. Other methods focus on size rather than depth.
% For instance, the algorithm \gosdt~\cite{NEURIPS2019_ac52c626} optimize a linear combination of classification error and number of leaves.
%
% In order to compare with such approaches, we designed a method to trade accuracy for size based on pruning. Given the tree of accuracy $\alpha$ found by \budalg, and given a target accuracy $\tau \leq \alpha$, we suppress the subtree of size $s_i$ and classification error $\alpha_i$ such that $\alpha_i/s_i$ is minimum, as long as the overall accuracy is not lower than $\tau$.  We did not manage, unfortunately, to obtain a relevant comparison with \gosdt, because no setting of the regularization parameter enabled us to obtain trees with more than a dozen leafs. Instead we experimented with \iti~\cite{Utgoff97decisiontree}. We ran it on every data set, and grouped the resulting trees in 4 classes depending on their depths. The first column of Table~\ref{tab:iti} shows the number of data sets in each class. Then for \iti, we report the average classification error and size of the trees.
% For \budalg, we report the same data before and after pruning.
% Over more than half of the data sets, \budalg can find trees that are both smaller and more accurate than those found by \iti. On the first and last classes, however, \iti's trees are slightly smaller, albeit less accurate.
%
%
% \begin{table}[htbp]
% \begin{center}
% \begin{footnotesize}
% \tabcolsep=5pt
% \begin{tabular}{lcrrrrrrrr}
\toprule
\multirow{2}{*}{}& & \multicolumn{3}{c}{\iti} & \multicolumn{5}{c}{\bfsh}\\
\cmidrule(rr){3-5}\cmidrule(rr){6-10}
&\multirow{1}{*}{\#} &  \multicolumn{1}{c}{error} & \multicolumn{1}{c}{size} & \multicolumn{1}{c}{depth} & \multicolumn{1}{c}{error} & \multicolumn{1}{c}{init e.} & \multicolumn{1}{c}{size} & \multicolumn{1}{c}{init s.} & \multicolumn{1}{c}{depth} \\
\midrule

\texttt{$\mdepth \in [0,5]$} & \multicolumn{1}{r}{9}  & 7.3 & \textbf{13.4} & 3.8 & \textbf{6.8} & 1.9 & 14.5 & 22.9 & 3.8\\
\texttt{$\mdepth \in [6,10]$} & \multicolumn{1}{r}{13}  & 38.1 & \textbf{41.0} & 7.5 & \textbf{32.2} & 2.5 & 45.2 & 104.9 & \textbf{7.2}\\
\texttt{$\mdepth \in [11,15]$} & \multicolumn{1}{r}{13}  & 93.1 & 129.8 & 13.1 & \textbf{89.8} & 20.1 & \textbf{110.8} & 198.8 & \textbf{12.5}\\
\texttt{$\mdepth \in [16,20]$} & \multicolumn{1}{r}{11}  & 1101.5 & 1036.6 & 17.7 & \textbf{907.3} & 406.0 & \textbf{993.5} & 1840.4 & 17.7\\
\bottomrule
\end{tabular}

% \end{footnotesize}
% \end{center}
% \caption{\label{tab:iti} ITI}
% \end{table}



\section{Conclusion}

We have introduced a simple, exact, iterative, memory-efficient and anytime algorithm for computing optimaly-accurate tree classifiers of bounded depth.
This algorithm is considerably more efficient than state-of-the-art exact algorithms. Moreover, it has no significant time nor memory overhead with respect to greedy heuristic methods.




\bibliographystyle{plain}
\bibliography{src/references}


\section*{Checklist}

\begin{enumerate}

\item For all authors...
\begin{enumerate}
  \item Do the main claims made in the abstract and introduction accurately reflect the paper's contributions and scope?
    \answerTODO{}
  \item Did you describe the limitations of your work?
    \answerTODO{}
  \item Did you discuss any potential negative societal impacts of your work?
    \answerTODO{}
  \item Have you read the ethics review guidelines and ensured that your paper conforms to them?
    \answerTODO{}
\end{enumerate}

\item If you are including theoretical results...
\begin{enumerate}
  \item Did you state the full set of assumptions of all theoretical results?
    \answerTODO{}
	\item Did you include complete proofs of all theoretical results?
    \answerTODO{}
\end{enumerate}

\item If you ran experiments...
\begin{enumerate}
  \item Did you include the code, data, and instructions needed to reproduce the main experimental results (either in the supplemental material or as a URL)?
    \answerTODO{}
  \item Did you specify all the training details (e.g., data splits, hyperparameters, how they were chosen)?
    \answerTODO{}
	\item Did you report error bars (e.g., with respect to the random seed after running experiments multiple times)?
    \answerTODO{}
	\item Did you include the total amount of compute and the type of resources used (e.g., type of GPUs, internal cluster, or cloud provider)?
    \answerTODO{}
\end{enumerate}

\item If you are using existing assets (e.g., code, data, models) or curating/releasing new assets...
\begin{enumerate}
  \item If your work uses existing assets, did you cite the creators?
    \answerTODO{}
  \item Did you mention the license of the assets?
    \answerTODO{}
  \item Did you include any new assets either in the supplemental material or as a URL?
    \answerTODO{}
  \item Did you discuss whether and how consent was obtained from people whose data you're using/curating?
    \answerTODO{}
  \item Did you discuss whether the data you are using/curating contains personally identifiable information or offensive content?
    \answerTODO{}
\end{enumerate}

\item If you used crowdsourcing or conducted research with human subjects...
\begin{enumerate}
  \item Did you include the full text of instructions given to participants and screenshots, if applicable?
    \answerTODO{}
  \item Did you describe any potential participant risks, with links to Institutional Review Board (IRB) approvals, if applicable?
    \answerTODO{}
  \item Did you include the estimated hourly wage paid to participants and the total amount spent on participant compensation?
    \answerTODO{}
\end{enumerate}

\end{enumerate}

%%%%%%%%%%%%%%%%%%%%%%%%%%%%%%%%%%%%%%%%%%%%%%%%%%%%%%%%%%%%

\appendix

\section{Appendix}


\subsection{Example of lower bound reasonning}

\begin{example}[Lower bound reasoning]
	\label{ex:lb}


	Figure~\ref{fig:lowerbound} shows a snapshot of the excution of \budalg. Every node is labelled with the feature test on that node, and with the values of $\best[\abranch]$ for the branch $\abranch$ ending on that node. When all subtrees of a branch $\abranch$ have been explored (hence $\opt[\abranch]=1$), this is marked by a ``$^*$''. We assume that the branch considered at Line~\ref{line:fail} is $\abranch = \{r, \bar{a}, \bar{c}, g\}$. For instance, we can suppose that a tree rooted at $\abranch$ with feature $e$ has been found (misclassifying 2 data points). Then, search moved to the sibling branch $\{r, \bar{a}, \bar{c}, \bar{g}\}$, which was then optimized for a total error of $4$, and now the pair $(\abranch,e)$ is popped out of \sequence. For all branches $\abranch'$ of $\abranch$, we give the values of $\lb{\abranch',\abranch}$ and $\best[\abranch']$ between brackets. Since there exists $\abranch'$ such that $\lb{\abranch',\abranch} \geq \best[\abranch']$ (e.g., $\emptyset$ and $\{r, \bar{a}\}$), we know that $\abranch$ cannot belong to an improving solution, and hence there is no need to try to extend it further.

	 % the current best classifier cannot be improved as long as


	\begin{figure}
	\begin{center}
% \subfloat[upper bounds] {
		\scalebox{1}{
			\begin{forest}
				for tree={%
					l sep=25pt,
					s sep=10pt,
					node options={shape=rectangle, minimum width=10pt, inner sep=1pt, font=\footnotesize},
		  		edge={-latex, shorten >=1pt, shorten <=1pt},
				}
				[{$r:[50,50]$}
					[{$a:[19,22]$}, edge={very thick}, edge label={node[midway,fill=white,inner sep=2pt,font=\scriptsize]{$b$}}
					 [{$h:15$}, edge label={node[midway,fill=white,inner sep=2pt,font=\scriptsize]{$a$}}
					 	 [{$e:10^*$}, edge label={node[midway,fill=white,inner sep=2pt,font=\scriptsize]{$\bar{h}$}}
					 	 ]
						 [{$c:3$}, edge label={node[midway,fill=white,inner sep=2pt,font=\scriptsize]{$h$}}
						 	[{$d:0^*$}, edge label={node[midway,fill=white,inner sep=2pt,font=\scriptsize]{$c$}}]
							[{$f:3^*$}, edge label={node[midway,fill=white,inner sep=2pt,font=\scriptsize]{$\bar{c}$}}]
						 ]
					 ]
					 [{$c:[19,17]$}, edge={very thick}, edge label={node[midway,fill=white,inner sep=2pt,font=\scriptsize]{$\bar{a}$}}
					 	[{$f:15^*$}, edge label={node[midway,fill=white,inner sep=2pt,font=\scriptsize]{$c$}}
							% [$\posclass$, edge label={node[midway,fill=white,inner sep=2pt,font=\scriptsize]{$f$}}]
							% [$\negclass$, edge label={node[midway,fill=white,inner sep=2pt,font=\scriptsize]{$\bar{f}$}}]
						]
						[{$g:[4,6]$}, edge={very thick}, edge label={node[midway,fill=white,inner sep=2pt,font=\scriptsize]{$\bar{c}$}}
							[{$e:2$}, edge={very thick}, edge label={node[midway,fill=white,inner sep=2pt,font=\scriptsize]{$g$}}
							]
							[{$h:4^*$}, edge label={node[midway,fill=white,inner sep=2pt,font=\scriptsize]{$\bar{g}$}}]
					 	]
					 ]
					]
					[{$d:31^*$}, edge label={node[midway,fill=white,inner sep=2pt,font=\scriptsize]{$\bar{b}$}}
					]
				]
			\end{forest}
		}
		% }
		% \subfloat[lower bounds w.r.t. $\{b,\bar{a},c,\bar{g}\}$] {
		% \scalebox{1}{
		% 	\begin{forest}
		% 		for tree={%
		% 			l sep=25pt,
		% 			s sep=10pt,
		% 			node options={shape=rectangle, minimum width=10pt, inner sep=1pt, font=\footnotesize},
		%   		edge={-latex, shorten >=1pt, shorten <=1pt},
		% 		}
		% 		[{$b,[49,50]$}
		% 			[{$a,[18,22]$}, edge={very thick}, edge label={node[midway,fill=white,inner sep=2pt,font=\scriptsize]{$b$}}
		% 			 [{$.$}, edge label={node[midway,fill=white,inner sep=2pt,font=\scriptsize]{$a$}}
		% 			 ]
		% 			 [{$c,[18,17]$}, edge={very thick}, edge label={node[midway,fill=white,inner sep=2pt,font=\scriptsize]{$\bar{a}$}}
		% 			 	[{$f,15^*$}, edge label={node[midway,fill=white,inner sep=2pt,font=\scriptsize]{$c$}}
		% 					% [$\posclass$, edge label={node[midway,fill=white,inner sep=2pt,font=\scriptsize]{$f$}}]
		% 					% [$\negclass$, edge label={node[midway,fill=white,inner sep=2pt,font=\scriptsize]{$\bar{f}$}}]
		% 				]
		% 				[{$g,[3,\infty]$}, edge={very thick}, edge label={node[midway,fill=white,inner sep=2pt,font=\scriptsize]{$\bar{c}$}}
		% 					[., edge={very thick}, edge label={node[midway,fill=white,inner sep=2pt,font=\scriptsize]{$g$}}
		% 					]
		% 					[{$h,3^*$}, edge label={node[midway,fill=white,inner sep=2pt,font=\scriptsize]{$\bar{g}$}}]
		% 			 	]
		% 			 ]
		% 			]
		% 			[{$d:31^*$}, edge label={node[midway,fill=white,inner sep=2pt,font=\scriptsize]{$\bar{b}$}}
		% 			]
		% 		]
		% 	\end{forest}
		% }
		% }
	\end{center}
	\caption{\label{fig:lowerbound} Example of lower bound computation w.r.t. the branch }
	%\caption{\label{fig:searchtree} The search tree for decision trees. \dynprog explores it depth first, whereas \budalg explores branches in the order given below the leaves.}
	\end{figure}

\end{example}


\subsection{Extra experiments on balancing size and accuracy}

Most decision trees toolkits somehow try to balance size and accuracy. \budalg uses the standard approach to bound the maximum depth and searches for the tree with maximum accuracy within that limit. Other methods focus on size rather than depth. 
For instance, the algorithm \gosdt~\cite{NEURIPS2019_ac52c626} optimize a linear combination of classification error and number of leaves. 

In order to compare with such approaches, we designed a method to trade accuracy for size based on pruning. Given the tree of accuracy $\alpha$ found by \budalg, and given a target accuracy $\tau \leq \alpha$, we suppress the subtree of size $s_i$ and classification error $\alpha_i$ such that $\alpha_i/s_i$ is minimum, as long as the overall accuracy is not lower than $\tau$.  We did not manage, unfortunately, to obtain a relevant comparison with \gosdt, because no setting of the regularization parameter enabled us to obtain trees with more than a dozen leafs. Instead we experimented with \iti~\cite{Utgoff97decisiontree}. We ran it on every data set, and grouped the resulting trees in 4 classes depending on their depths. The first column of Table~\ref{tab:iti} shows the number of data sets in each class. Then for \iti, we report the average classification error and size of the trees.
For \budalg, we report the same data before and after pruning. 
Over more than half of the data sets, \budalg can find trees that are both smaller and more accurate than those found by \iti. On the first and last classes, however, \iti's trees are slightly smaller, albeit less accurate.


\begin{table}[htbp]
\begin{center}
\begin{footnotesize}
\tabcolsep=5pt
\begin{tabular}{lcrrrrrrrr}
\toprule
\multirow{2}{*}{}& & \multicolumn{3}{c}{\iti} & \multicolumn{5}{c}{\bfsh}\\
\cmidrule(rr){3-5}\cmidrule(rr){6-10}
&\multirow{1}{*}{\#} &  \multicolumn{1}{c}{error} & \multicolumn{1}{c}{size} & \multicolumn{1}{c}{depth} & \multicolumn{1}{c}{error} & \multicolumn{1}{c}{init e.} & \multicolumn{1}{c}{size} & \multicolumn{1}{c}{init s.} & \multicolumn{1}{c}{depth} \\
\midrule

\texttt{$\mdepth \in [0,5]$} & \multicolumn{1}{r}{9}  & 7.3 & \textbf{13.4} & 3.8 & \textbf{6.8} & 1.9 & 14.5 & 22.9 & 3.8\\
\texttt{$\mdepth \in [6,10]$} & \multicolumn{1}{r}{13}  & 38.1 & \textbf{41.0} & 7.5 & \textbf{32.2} & 2.5 & 45.2 & 104.9 & \textbf{7.2}\\
\texttt{$\mdepth \in [11,15]$} & \multicolumn{1}{r}{13}  & 93.1 & 129.8 & 13.1 & \textbf{89.8} & 20.1 & \textbf{110.8} & 198.8 & \textbf{12.5}\\
\texttt{$\mdepth \in [16,20]$} & \multicolumn{1}{r}{11}  & 1101.5 & 1036.6 & 17.7 & \textbf{907.3} & 406.0 & \textbf{993.5} & 1840.4 & 17.7\\
\bottomrule
\end{tabular}

\end{footnotesize}
\end{center}
\caption{\label{tab:iti} ITI}
\end{table}


\subsection{Full experimental results}


The benchmark of classification data set we used is described in Table~\ref{tab:info}. It consists of 51 data sets 
commonly used in related work articles, to which we added some large data sets from Kaggle: \texttt{bank}, \texttt{titanic}, \texttt{surgical-deepnet} and \texttt{weather-aus}, as well as \texttt{mnist\_0}, \texttt{adult\_discretized} and \texttt{compas\_discretized}. We report the number of data points ($|\allex|$), the number of features ($|\features|$), the same parameters after preprocessing (respectively $|\allex|^*$ and $|\features|^*$), and the ``noise'' ratio, that is: $2|\posex \cap \negex|/(|\posex|+|\negex|)$.

\medskip

Then we report the raw data from our experimental comparison with the state of the art for $\mdepth=3,4,5,7,10$ and for the four size catagories in the following tables:

\tabcolsep=10pt
\begin{center}
\begin{tabular}{lrrrr}
	\toprule
	\multirow{2}{*}{$\mdepth$}& \multicolumn{2}{c}{$\numex < 1500$} & \multicolumn{2}{c}{$\numex \geq 1500$} \\
	& $\numfeat < 100$ & $\numfeat \geq 100$ & $\numfeat < 100$ & $\numfeat \geq 100$ \\
	 % & $\numex < 1500; \numfeat < 100$ & $\numex < 1500; \numfeat \geq 100$ & $\numex \geq 1500; \numfeat < 100$ & $\numex \geq 1500; \numfeat \geq 100$ \\
	 \midrule
	$3$ & Table~\ref{tab:all31} & Table~\ref{tab:all32}  & Table~\ref{tab:all33}  & Table~\ref{tab:all34} \\
	$4$ & Table~\ref{tab:all41} & Table~\ref{tab:all42}  & Table~\ref{tab:all43}  & Table~\ref{tab:all44} \\
	$5$ & Table~\ref{tab:all51} & Table~\ref{tab:all52}  & Table~\ref{tab:all53}  & Table~\ref{tab:all54} \\
	$7$ & Table~\ref{tab:all71} & Table~\ref{tab:all72}  & Table~\ref{tab:all73}  & Table~\ref{tab:all74} \\
	$10$ & Table~\ref{tab:all101} & Table~\ref{tab:all102}  & Table~\ref{tab:all103}  & Table~\ref{tab:all104} \\
	\bottomrule 
\end{tabular}
\end{center}


% in Tables~\ref{tab:all3},
% \ref{tab:all4},
% \ref{tab:all5},
% \ref{tab:all6},
% \ref{tab:all7},
% \ref{tab:all8},
% \ref{tab:all9} and \ref{tab:all10}, respectively.
For every instance, we give the classification error of the best tree found within a time limit of 1h for every method. Moreover, we give the CPU time taken by each method to prove optimality when optimality is proven (in which case we mark it by a ``$^*$''), and to find the best solution otherwise. Notice that \cp\ and \dleight\ are not anytime and hence only report a solution at the end of the time limit when optimality is not proven. In this case, we write $\geq1h$. 

\medskip

Every process was first run with a memory limit of 3.5GB. Many runs of \dleight, \cp\ and \binoct\ went well over that limit and were rerun with a limit of 50GB. Still, 138 runs of \binoct and 164 runs of \dleight (out of 460) went over the limit. As \binoct can output trees anytime, the data for these runs (up until the memory blow-out) are in the tables. For \dleight, however, this is marked as a ``-'' since there was no output.
%\binoct 138
%\dleight 164 %5+10+20+23+30+27+25+24




\begin{table}[htbp]%
\begin{center}%
\begin{scriptsize}%
\tabcolsep=10pt%
\begin{tabular}{lrrrrr}
\toprule
set & $|\allex|$ & $|\features|$ & $|\allex|^*$ & $|\features|^*$ & noise \\
\midrule
\texttt{hepatitis}& 137& 68& 136& 34& 0.0000\\
\texttt{lymph}& 148& 68& 148& 47& 0.0000\\
\texttt{wine1}& 178& 1276& 178& 646& 0.0000\\
\texttt{wine2}& 178& 1276& 178& 646& 0.0000\\
\texttt{wine3}& 178& 1276& 178& 646& 0.0000\\
\texttt{audiology}& 216& 148& 186& 84& 0.0000\\
\texttt{heart-cleveland}& 296& 95& 296& 54& 0.0000\\
\texttt{primary-tumor}& 336& 31& 240& 17& 0.0893\\
\texttt{ionosphere}& 351& 445& 350& 222& 0.0000\\
\texttt{vote}& 435& 48& 342& 48& 0.0000\\
\texttt{forest-fires}& 517& 989& 504& 656& 0.0155\\
\texttt{soybean}& 630& 50& 502& 43& 0.0063\\
\texttt{australian-credit}& 653& 125& 653& 74& 0.0000\\
\texttt{breast-cancer}& 683& 89& 449& 89& 0.0000\\
\texttt{breast-wisconsin}& 683& 120& 449& 60& 0.0000\\
\texttt{diabetes}& 768& 112& 768& 56& 0.0000\\
\texttt{anneal}& 812& 93& 495& 49& 0.0837\\
\texttt{vehicle}& 846& 252& 846& 126& 0.0000\\
\texttt{titanic}& 887& 333& 803& 333& 0.0361\\
\texttt{tic-tac-toe}& 958& 27& 958& 27& 0.0000\\
\texttt{german-credit}& 1000& 112& 998& 86& 0.0000\\
\texttt{yeast}& 1484& 89& 1418& 46& 0.0067\\
\texttt{car}& 1728& 21& 1728& 21& 0.0000\\
\texttt{segment}& 2310& 235& 2027& 114& 0.0000\\
\texttt{splice-1}& 3190& 287& 3005& 255& 0.0006\\
\texttt{kr-vs-kp}& 3196& 73& 3196& 38& 0.0000\\
\texttt{hypothyroid}& 3247& 88& 2527& 44& 0.0105\\
\texttt{compas\_discretized}& 6167& 25& 4181& 20& 0.5928\\
\texttt{pendigits}& 7494& 216& 7415& 108& 0.0000\\
\texttt{mushroom}& 8124& 119& 8124& 100& 0.0000\\
\texttt{surgical-deepnet}& 14635& 6047& 11733& 6046& 0.0000\\
\texttt{letter}& 20000& 224& 18200& 112& 0.0000\\
\texttt{taiwan\_binarised}& 30000& 205& 29112& 198& 0.0253\\
\texttt{adult\_discretized}& 30299& 59& 17804& 56& 0.2149\\
\texttt{bank}& 45211& 9531& 45211& 9530& 0.0000\\
\texttt{mnist\_8}& 60000& 784& 59987& 671& 0.0000\\
\texttt{mnist\_9}& 60000& 784& 59987& 671& 0.0000\\
\texttt{mnist\_0}& 60000& 784& 59987& 671& 0.0000\\
\texttt{mnist\_6}& 60000& 784& 59987& 671& 0.0000\\
\texttt{mnist\_5}& 60000& 784& 59987& 671& 0.0000\\
\texttt{mnist\_3}& 60000& 784& 59987& 671& 0.0000\\
\texttt{mnist\_2}& 60000& 784& 59987& 671& 0.0000\\
\texttt{mnist\_4}& 60000& 784& 59987& 671& 0.0000\\
\texttt{mnist\_7}& 60000& 784& 59987& 671& 0.0000\\
\texttt{mnist\_1}& 60000& 784& 59987& 671& 0.0000\\
\texttt{weather-aus}& 142193& 4759& 142151& 4756& 0.0000\\
\bottomrule
\end{tabular}
%
\end{scriptsize}%
\end{center}%
\caption{\label{tab:info} Benchmark and preprocessing data}%
\end{table}%


\begin{table}[htbp]%
\begin{center}%
\begin{scriptsize}%
\tabcolsep=2pt%
\begin{tabular}{lrrrrrrrrrrrr}
\toprule
\multirow{2}{*}{}&  \multicolumn{2}{c}{\budalg} & \multicolumn{2}{c}{\murtree} & \multicolumn{2}{c}{\dleight} & \multicolumn{2}{c}{\cp} & \multicolumn{2}{c}{binoct} & \multicolumn{2}{c}{\cart}\\
\cmidrule(rr){2-3}\cmidrule(rr){4-5}\cmidrule(rr){6-7}\cmidrule(rr){8-9}\cmidrule(rr){10-11}\cmidrule(rr){12-13}
& \multicolumn{1}{c}{error} & \multicolumn{1}{c}{cpu} & \multicolumn{1}{c}{error} & \multicolumn{1}{c}{cpu} & \multicolumn{1}{c}{error} & \multicolumn{1}{c}{cpu} & \multicolumn{1}{c}{error} & \multicolumn{1}{c}{cpu} & \multicolumn{1}{c}{error} & \multicolumn{1}{c}{cpu} & \multicolumn{1}{c}{error} & \multicolumn{1}{c}{cpu} \\
\midrule

\texttt{anneal} & 112 & 0.03$^*$ & 112 & 0.14$^*$ & 112 & 2.1$^*$ & 112 & 6.0$^*$ & 123 & 3042 & 149 & 0.00\\
\texttt{balance-scale} & 49 & 0.00$^*$ & 49 & 0.01$^*$ & 49 & 0.02$^*$ & 49 & 0.55$^*$ & - & - & 49 & 0.00\\
\texttt{banknote} & 36 & 0.01$^*$ & 36 & 0.02$^*$ & 36 & 0.09$^*$ & 36 & 0.88$^*$ & - & - & 118 & 0.00\\
\texttt{breast-cancer} & 24 & 0.16$^*$ & 24 & 0.07$^*$ & 24 & 0.89$^*$ & 24 & 5.7$^*$ & 25 & 3131 & 28 & 0.00\\
\texttt{heart-cleveland} & 41 & 0.05$^*$ & 41 & 0.12$^*$ & 41 & 3.5$^*$ & 41 & 6.8$^*$ & 42 & 870 & 43 & 0.00\\
\texttt{hepatitis} & 10 & 0.00$^*$ & 10 & 0.03$^*$ & 10 & 1.1$^*$ & 10 & 3.9$^*$ & 10 & 2314 & 16 & 0.00\\
\texttt{IndiansDiabetes} & 166 & 0.02$^*$ & 166 & 0.07$^*$ & 166 & 0.31$^*$ & 166 & 1.6$^*$ & - & - & 180 & 0.00\\
\texttt{iris} & 1 & 0.00$^*$ & 1 & 0.00$^*$ & 1 & 0.00$^*$ & 1 & 0.15$^*$ & - & - & 1 & 0.00\\
\texttt{lymph} & 12 & 0.01$^*$ & 12 & 0.03$^*$ & 12 & 0.56$^*$ & 12 & 3.7$^*$ & 14 & 2298 & 17 & 0.00\\
\texttt{messidor} & 366 & 0.25$^*$ & 366 & 0.63$^*$ & 366 & 4.9$^*$ & 366 & 5.3$^*$ & - & - & 384 & 0.00\\
\texttt{monk1} & 11 & 0.00$^*$ & 11 & 0.00$^*$ & 11 & 0.00$^*$ & 11 & 0.27$^*$ & - & - & 11 & 0.00\\
\texttt{monk2} & 42 & 0.00$^*$ & 42 & 0.00$^*$ & 42 & 0.01$^*$ & 42 & 0.37$^*$ & - & - & 57 & 0.00\\
\texttt{monk3} & 6 & 0.00$^*$ & 6 & 0.00$^*$ & 6 & 0.00$^*$ & 6 & 0.32$^*$ & - & - & 7 & 0.00\\
\texttt{primary-tumor} & 46 & 0.00$^*$ & 46 & 0.01$^*$ & 46 & 0.12$^*$ & 46 & 2.0$^*$ & 46 & 2722 & 53 & 0.00\\
\texttt{soybean} & 29 & 0.01$^*$ & 29 & 0.02$^*$ & 29 & 0.23$^*$ & 29 & 2.3$^*$ & 31 & 3098 & 47 & 0.00\\
\texttt{tic-tac-toe} & 216 & 0.01$^*$ & 216 & 0.02$^*$ & 216 & 0.11$^*$ & 216 & 1.8$^*$ & 232 & 1794 & 236 & 0.00\\
\texttt{vote} & 12 & 0.02$^*$ & 12 & 0.02$^*$ & 12 & 0.29$^*$ & 12 & 2.6$^*$ & 13 & 2763 & 14 & 0.00\\
\texttt{yeast} & 403 & 0.07$^*$ & 403 & 0.34$^*$ & 403 & 6.1$^*$ & 403 & 7.7$^*$ & 434 & 2683 & 418 & 0.00\\
\bottomrule
\end{tabular}
%
\end{scriptsize}%
\end{center}%
\caption{\label{tab:all31} Comparison with state of the art: $\numex<5000, \numfeat<250$, depth 3}%
\end{table}%

\begin{table}[htbp]%
\begin{center}%
\begin{scriptsize}%
\tabcolsep=2pt%
\begin{tabular}{lrrrrrrrrrrrr}
\toprule
\multirow{2}{*}{}&  \multicolumn{2}{c}{\budalg} & \multicolumn{2}{c}{\murtree} & \multicolumn{2}{c}{\dleight} & \multicolumn{2}{c}{\cp} & \multicolumn{2}{c}{binoct} & \multicolumn{2}{c}{\cart}\\
\cmidrule(rr){2-3}\cmidrule(rr){4-5}\cmidrule(rr){6-7}\cmidrule(rr){8-9}\cmidrule(rr){10-11}\cmidrule(rr){12-13}
& \multicolumn{1}{c}{error} & \multicolumn{1}{c}{cpu} & \multicolumn{1}{c}{error} & \multicolumn{1}{c}{cpu} & \multicolumn{1}{c}{error} & \multicolumn{1}{c}{cpu} & \multicolumn{1}{c}{error} & \multicolumn{1}{c}{cpu} & \multicolumn{1}{c}{error} & \multicolumn{1}{c}{cpu} & \multicolumn{1}{c}{error} & \multicolumn{1}{c}{cpu} \\
\midrule

\texttt{audiology} & 5 & 0.06$^*$ & 5 & 0.13$^*$ & 5 & 4.1$^*$ & 5 & 9.1$^*$ & 6 & 508 & 6 & 0.00\\
\texttt{australian-credit} & 73 & 0.14$^*$ & 73 & 0.35$^*$ & 73 & 9.7$^*$ & 73 & 14$^*$ & 87 & 192 & 87 & 0.00\\
\texttt{biodeg} & 164 & 5.4$^*$ & 164 & 12$^*$ & 164 & 141$^*$ & 164 & 90$^*$ & - & - & 184 & 0.01\\
\texttt{breast-wisconsin} & 15 & 0.05$^*$ & 15 & 0.20$^*$ & 15 & 5.6$^*$ & 15 & 11$^*$ & 18 & 1858 & 26 & 0.00\\
\texttt{diabetes} & 162 & 0.09$^*$ & 162 & 0.37$^*$ & 162 & 10$^*$ & 162 & 12$^*$ & 165 & 3501 & 177 & 0.00\\
\texttt{forest-fires} & 193 & 20$^*$ & 193 & 9.6$^*$ & - & - & 193 & 2836$^*$ & 198 & 3501 & 198 & 0.01\\
\texttt{german-credit} & 236 & 0.26$^*$ & 236 & 0.38$^*$ & 236 & 9.4$^*$ & 236 & 13$^*$ & 244 & 2329 & 251 & 0.00\\
\texttt{ionosphere} & 22 & 3.8$^*$ & 22 & 12$^*$ & 22 & 397$^*$ & 22 & 460$^*$ & 27 & 3268 & 29 & 0.01\\
\texttt{titanic} & 143 & 6.7$^*$ & 143 & 11$^*$ & 143 & 135$^*$ & 143 & 173$^*$ & 150 & 3362 & 148 & 0.01\\
\texttt{vehicle} & 26 & 0.93$^*$ & 26 & 2.2$^*$ & 26 & 63$^*$ & 26 & 66$^*$ & 42 & 3374 & 66 & 0.01\\
\texttt{wine1} & 43 & 16$^*$ & 43 & 9.0$^*$ & - & - & 43 & $\mathsmaller{\geq}1$h & 44 & 3507 & 45 & 0.00\\
\texttt{wine2} & 49 & 17$^*$ & 49 & 5.8$^*$ & - & - & 49 & $\mathsmaller{\geq}1$h & 57 & 3207 & 52 & 0.00\\
\texttt{wine3} & 33 & 16$^*$ & 33 & 8.4$^*$ & - & - & 33 & $\mathsmaller{\geq}1$h & 35 & 2814 & 35 & 0.00\\
\bottomrule
\end{tabular}
%
\end{scriptsize}%
\end{center}%
\caption{\label{tab:all32} Comparison with state of the art: $\numex<5000, \numfeat \geq 250$, depth 3}%
\end{table}%

\begin{table}[htbp]%
\begin{center}%
\begin{scriptsize}%
\tabcolsep=2pt%
\begin{tabular}{lrrrrrrrrrrrr}
\toprule
\multirow{2}{*}{}&  \multicolumn{2}{c}{\budalg} & \multicolumn{2}{c}{\murtree} & \multicolumn{2}{c}{\dleight} & \multicolumn{2}{c}{\cp} & \multicolumn{2}{c}{binoct} & \multicolumn{2}{c}{\cart}\\
\cmidrule(rr){2-3}\cmidrule(rr){4-5}\cmidrule(rr){6-7}\cmidrule(rr){8-9}\cmidrule(rr){10-11}\cmidrule(rr){12-13}
& \multicolumn{1}{c}{error} & \multicolumn{1}{c}{cpu} & \multicolumn{1}{c}{error} & \multicolumn{1}{c}{cpu} & \multicolumn{1}{c}{error} & \multicolumn{1}{c}{cpu} & \multicolumn{1}{c}{error} & \multicolumn{1}{c}{cpu} & \multicolumn{1}{c}{error} & \multicolumn{1}{c}{cpu} & \multicolumn{1}{c}{error} & \multicolumn{1}{c}{cpu} \\
\midrule

\texttt{adult\_discretized} & 5020 & 0.43$^*$ & 5020 & 0.84$^*$ & 5020 & 8.4$^*$ & 5020 & 6.4$^*$ & 5600 & 3503 & 5758 & 0.05\\
\texttt{car} & 192 & 0.01$^*$ & 192 & 0.01$^*$ & 192 & 0.03$^*$ & 192 & 1.7$^*$ & 192 & 1141 & 202 & 0.00\\
\texttt{car\_evaluation} & 202 & 0.00$^*$ & 202 & 0.01$^*$ & 202 & 0.02$^*$ & 202 & 0.44$^*$ & - & - & 226 & 0.00\\
\texttt{chess} & 0 & 0.00$^*$ & 0 & 0.00$^*$ & 0 & 0.00$^*$ & 0 & 0.04$^*$ & - & - & 0 & 0.00\\
\texttt{compas\_discretized} & 2004 & 0.00$^*$ & 2004 & 0.06$^*$ & 2004 & 0.21$^*$ & 2004 & 1.8$^*$ & 2032 & 806 & 2072 & 0.01\\
\texttt{HTRU\_2} & 401 & 1.2$^*$ & 401 & 3.6$^*$ & 401 & 12$^*$ & 401 & 5.7$^*$ & - & - & 422 & 0.05\\
\texttt{hypothyroid} & 61 & 0.07$^*$ & 61 & 0.41$^*$ & 61 & 3.8$^*$ & 61 & 6.6$^*$ & 62 & 2662 & 62 & 0.01\\
\texttt{kr-vs-kp} & 198 & 0.09$^*$ & 198 & 0.22$^*$ & 198 & 2.3$^*$ & 198 & 4.8$^*$ & 375 & 2200 & 306 & 0.01\\
\texttt{magic04} & 3446 & 3.8$^*$ & 3446 & 7.6$^*$ & 3446 & 26$^*$ & 3446 & 11$^*$ & - & - & 3788 & 0.06\\
\texttt{seismic\_bumps} & 160 & 0.28$^*$ & 160 & 1.1$^*$ & 160 & 5.3$^*$ & 160 & 7.3$^*$ & - & - & 170 & 0.01\\
\texttt{winequality-red} & 8 & 0.02$^*$ & 8 & 0.10$^*$ & 8 & 0.37$^*$ & 8 & 1.2$^*$ & - & - & 9 & 0.00\\
\bottomrule
\end{tabular}
%
\end{scriptsize}%
\end{center}%
\caption{\label{tab:all33} Comparison with state of the art: $\numex \geq 5000, \numfeat < 250$, depth 3}%
\end{table}%

\begin{table}[htbp]%
\begin{center}%
\begin{scriptsize}%
\tabcolsep=2pt%
\begin{tabular}{lrrrrrrrrrrrr}
\toprule
\multirow{2}{*}{}&  \multicolumn{2}{c}{\budalg} & \multicolumn{2}{c}{\murtree} & \multicolumn{2}{c}{\dleight} & \multicolumn{2}{c}{\cp} & \multicolumn{2}{c}{binoct} & \multicolumn{2}{c}{\cart}\\
\cmidrule(rr){2-3}\cmidrule(rr){4-5}\cmidrule(rr){6-7}\cmidrule(rr){8-9}\cmidrule(rr){10-11}\cmidrule(rr){12-13}
& \multicolumn{1}{c}{error} & \multicolumn{1}{c}{cpu} & \multicolumn{1}{c}{error} & \multicolumn{1}{c}{cpu} & \multicolumn{1}{c}{error} & \multicolumn{1}{c}{cpu} & \multicolumn{1}{c}{error} & \multicolumn{1}{c}{cpu} & \multicolumn{1}{c}{error} & \multicolumn{1}{c}{cpu} & \multicolumn{1}{c}{error} & \multicolumn{1}{c}{cpu} \\
\midrule

\texttt{bank} & 4453 & 259 & 5289 & 0.84 & - & - & 4453 & $\mathsmaller{\geq}1$h & - & - & 4462 & 33\\
\texttt{bank\_conv} & 428 & 9.9$^*$ & 428 & 16$^*$ & 428 & 112$^*$ & 428 & 73$^*$ & - & - & 438 & 0.02\\
\texttt{default\_credit} & 5327 & 232$^*$ & 5327 & 451$^*$ & 5327 & 1730$^*$ & 5327 & 510$^*$ & - & - & 5349 & 0.48\\
\texttt{hand\_posture} & 7645 & 181 & 7645 & 1134 & 10461 & $\mathsmaller{\geq}1$h & 8432 & $\mathsmaller{\geq}1$h & - & - & 8382 & 22\\
\texttt{letter} & 369 & 10$^*$ & 369 & 34$^*$ & 369 & 462$^*$ & 369 & 158$^*$ & 813 & 0.00 & 677 & 0.17\\
\texttt{mnist\_0} & 2557 & 1994$^*$ & 2557 & 568$^*$ & 3366 & $\mathsmaller{\geq}1$h & 2557 & $\mathsmaller{\geq}1$h & - & - & 3329 & 2.5\\
\texttt{mushroom} & 8 & 0.79$^*$ & 8 & 0.53$^*$ & 8 & 6.8$^*$ & 8 & 8.4$^*$ & 180 & 2728 & 280 & 0.02\\
\texttt{pendigits} & 47 & 3.3$^*$ & 47 & 11$^*$ & 47 & 126$^*$ & 47 & 70$^*$ & 477 & 2663 & 51 & 0.05\\
\texttt{segment} & 0 & 0.03$^*$ & 0 & 0.13$^*$ & 0 & 2.0$^*$ & 0 & 4.1$^*$ & 4 & 2865 & 5 & 0.01\\
\texttt{spambase} & 694 & 11$^*$ & 694 & 38$^*$ & 694 & 291$^*$ & 694 & 203$^*$ & - & - & 704 & 0.05\\
\texttt{splice-1} & 224 & 9.8$^*$ & 224 & 5.3$^*$ & 224 & 108$^*$ & 224 & 173$^*$ & 453 & 3502 & 279 & 0.03\\
\texttt{Statlog\_satellite} & 187 & 79$^*$ & 187 & 118$^*$ & 187 & 998$^*$ & 187 & 703$^*$ & - & - & 345 & 0.08\\
\texttt{Statlog\_shuttle} & 0 & 6.7$^*$ & 0 & 81$^*$ & 1 & $\mathsmaller{\geq}1$h & 0 & 39$^*$ & - & - & 58 & 1.7\\
\texttt{surgical-deepnet} & 2512 & 953 & 2512 & 3523 & - & - & 2512 & $\mathsmaller{\geq}1$h & - & - & 2924 & 5.7\\
\texttt{taiwan\_binarised} & 5326 & 48$^*$ & 5326 & 45$^*$ & 5326 & 511$^*$ & 5326 & 190$^*$ & 6636 & 0.00 & 5346 & 0.26\\
\texttt{weather-aus} & 1756 & 14 & 1756 & 611 & - & - & 1756 & $\mathsmaller{\geq}1$h & - & - & 1761 & 20\\
\bottomrule
\end{tabular}
%
\end{scriptsize}%
\end{center}%
\caption{\label{tab:all34} Comparison with state of the art: $\numex \geq 5000, \numfeat \geq 250$, depth 3}%
\end{table}%


\begin{table}[htbp]%
\begin{center}%
\begin{scriptsize}%
\tabcolsep=2pt%
\begin{tabular}{lrrrrrrrrrrrr}
\toprule
\multirow{2}{*}{}&  \multicolumn{2}{c}{\budalg} & \multicolumn{2}{c}{\murtree} & \multicolumn{2}{c}{\dleight} & \multicolumn{2}{c}{\cp} & \multicolumn{2}{c}{binoct} & \multicolumn{2}{c}{\cart}\\
\cmidrule(rr){2-3}\cmidrule(rr){4-5}\cmidrule(rr){6-7}\cmidrule(rr){8-9}\cmidrule(rr){10-11}\cmidrule(rr){12-13}
& \multicolumn{1}{c}{error} & \multicolumn{1}{c}{cpu} & \multicolumn{1}{c}{error} & \multicolumn{1}{c}{cpu} & \multicolumn{1}{c}{error} & \multicolumn{1}{c}{cpu} & \multicolumn{1}{c}{error} & \multicolumn{1}{c}{cpu} & \multicolumn{1}{c}{error} & \multicolumn{1}{c}{cpu} & \multicolumn{1}{c}{error} & \multicolumn{1}{c}{cpu} \\
\midrule

\texttt{anneal} & 91 & 1.5$^*$ & 91 & 5.0$^*$ & 91 & 102$^*$ & 91 & 193$^*$ & 108 & 2954 & 135 & 0.00\\
\texttt{balance-scale} & 48 & 0.04$^*$ & 48 & 0.05$^*$ & 48 & 0.22$^*$ & 48 & 1.8$^*$ & - & - & 49 & 0.00\\
\texttt{banknote} & 13 & 0.08$^*$ & 13 & 0.27$^*$ & 13 & 0.78$^*$ & 13 & 4.2$^*$ & - & - & 38 & 0.00\\
\texttt{breast-cancer} & 16 & 9.6$^*$ & 16 & 2.9$^*$ & 16 & 28$^*$ & 16 & 219$^*$ & 22 & 2746 & 21 & 0.00\\
\texttt{heart-cleveland} & 25 & 3.1$^*$ & 25 & 4.8$^*$ & 25 & 154$^*$ & 25 & 391$^*$ & 37 & 2750 & 38 & 0.00\\
\texttt{hepatitis} & 3 & 0.32$^*$ & 3 & 0.73$^*$ & 3 & 28$^*$ & 3 & 70$^*$ & 11 & 510 & 12 & 0.00\\
\texttt{IndiansDiabetes} & 149 & 0.90$^*$ & 149 & 1.3$^*$ & 149 & 7.3$^*$ & 149 & 16$^*$ & - & - & 166 & 0.00\\
\texttt{iris} & 1 & 0.00$^*$ & 1 & 0.00$^*$ & 1 & 0.00$^*$ & 1 & 0.92$^*$ & - & - & 1 & 0.00\\
\texttt{lymph} & 3 & 0.74$^*$ & 3 & 0.63$^*$ & 3 & 14$^*$ & 3 & 64$^*$ & 7 & 2987 & 10 & 0.00\\
\texttt{messidor} & 332 & 21$^*$ & 332 & 27$^*$ & 332 & 245$^*$ & 332 & 269$^*$ & - & - & 364 & 0.00\\
\texttt{monk1} & 2 & 0.00$^*$ & 2 & 0.01$^*$ & 2 & 0.01$^*$ & 2 & 1.5$^*$ & - & - & 11 & 0.00\\
\texttt{monk2} & 31 & 0.01$^*$ & 31 & 0.01$^*$ & 31 & 0.04$^*$ & 31 & 2.1$^*$ & - & - & 50 & 0.00\\
\texttt{monk3} & 4 & 0.00$^*$ & 4 & 0.01$^*$ & 4 & 0.01$^*$ & 4 & 1.0$^*$ & - & - & 5 & 0.00\\
\texttt{primary-tumor} & 34 & 0.03$^*$ & 34 & 0.11$^*$ & 34 & 2.0$^*$ & 34 & 5.6$^*$ & 38 & 3132 & 44 & 0.00\\
\texttt{soybean} & 14 & 0.62$^*$ & 14 & 0.46$^*$ & 14 & 5.1$^*$ & 14 & 22$^*$ & 22 & 2906 & 32 & 0.00\\
\texttt{tic-tac-toe} & 137 & 0.38$^*$ & 137 & 0.26$^*$ & 137 & 1.8$^*$ & 137 & 7.2$^*$ & 162 & 2511 & 150 & 0.00\\
\texttt{vote} & 5 & 1.2$^*$ & 5 & 0.50$^*$ & 5 & 7.6$^*$ & 5 & 21$^*$ & 12 & 3311 & 8 & 0.00\\
\texttt{yeast} & 366 & 3.4$^*$ & 366 & 18$^*$ & 366 & 257$^*$ & 366 & 386$^*$ & 438 & 888 & 394 & 0.01\\
\bottomrule
\end{tabular}
%
\end{scriptsize}%
\end{center}%
\caption{\label{tab:all41} Comparison with state of the art: $\numex<5000, \numfeat<250$, depth 4}%
\end{table}%

\begin{table}[htbp]%
\begin{center}%
\begin{scriptsize}%
\tabcolsep=2pt%
\begin{tabular}{lrrrrrrrrrrrr}
\toprule
\multirow{2}{*}{}&  \multicolumn{2}{c}{\budalg} & \multicolumn{2}{c}{\murtree} & \multicolumn{2}{c}{\dleight} & \multicolumn{2}{c}{\cp} & \multicolumn{2}{c}{binoct} & \multicolumn{2}{c}{\cart}\\
\cmidrule(rr){2-3}\cmidrule(rr){4-5}\cmidrule(rr){6-7}\cmidrule(rr){8-9}\cmidrule(rr){10-11}\cmidrule(rr){12-13}
& \multicolumn{1}{c}{error} & \multicolumn{1}{c}{cpu} & \multicolumn{1}{c}{error} & \multicolumn{1}{c}{cpu} & \multicolumn{1}{c}{error} & \multicolumn{1}{c}{cpu} & \multicolumn{1}{c}{error} & \multicolumn{1}{c}{cpu} & \multicolumn{1}{c}{error} & \multicolumn{1}{c}{cpu} & \multicolumn{1}{c}{error} & \multicolumn{1}{c}{cpu} \\
\midrule

\texttt{audiology} & 1 & 4.0$^*$ & 1 & 6.4$^*$ & 1 & 128$^*$ & 1 & 773$^*$ & 2 & 2687 & 3 & 0.00\\
\texttt{australian-credit} & 56 & 10$^*$ & 56 & 24$^*$ & 56 & 470$^*$ & 56 & 1170$^*$ & 83 & 3258 & 74 & 0.00\\
\texttt{biodeg} & 128 & 1511$^*$ & 128 & 1436$^*$ & - & - & 129 & $\mathsmaller{\geq}1$h & - & - & 148 & 0.01\\
\texttt{breast-wisconsin} & 7 & 3.1$^*$ & 7 & 9.3$^*$ & 7 & 245$^*$ & 7 & 662$^*$ & 15 & 3460 & 16 & 0.00\\
\texttt{diabetes} & 137 & 5.7$^*$ & 137 & 22$^*$ & 137 & 550$^*$ & 137 & 1001$^*$ & 180 & 2663 & 166 & 0.00\\
\texttt{forest-fires} & 173 & 15 & 171 & 2907$^*$ & - & - & 179 & $\mathsmaller{\geq}1$h & 196 & 3356 & 186 & 0.01\\
\texttt{german-credit} & 204 & 28$^*$ & 204 & 27$^*$ & 204 & 423$^*$ & 204 & 1008$^*$ & 236 & 3306 & 231 & 0.00\\
\texttt{ionosphere} & 7 & 730$^*$ & 7 & 1683$^*$ & - & - & 8 & $\mathsmaller{\geq}1$h & 24 & 751 & 27 & 0.01\\
\texttt{titanic} & 119 & 1604$^*$ & 119 & 2104$^*$ & - & - & 119 & $\mathsmaller{\geq}1$h & 135 & 3501 & 134 & 0.01\\
\texttt{vehicle} & 12 & 71$^*$ & 12 & 172$^*$ & - & - & 12 & $\mathsmaller{\geq}1$h & 30 & 3410 & 28 & 0.01\\
\texttt{wine1} & 37 & 1674 & 37 & 1831$^*$ & - & - & 39 & $\mathsmaller{\geq}1$h & 45 & 3506 & 42 & 0.01\\
\texttt{wine2} & 43 & 17 & 43 & 1833$^*$ & - & - & 46 & $\mathsmaller{\geq}1$h & 57 & 3232 & 47 & 0.01\\
\texttt{wine3} & 28 & 33 & 28 & 2537$^*$ & - & - & 30 & $\mathsmaller{\geq}1$h & 32 & 3388 & 32 & 0.01\\
\bottomrule
\end{tabular}
%
\end{scriptsize}%
\end{center}%
\caption{\label{tab:all42} Comparison with state of the art: $\numex<5000, \numfeat \geq 250$, depth 4}%
\end{table}%

\begin{table}[htbp]%
\begin{center}%
\begin{scriptsize}%
\tabcolsep=2pt%
\begin{tabular}{lrrrrrrrrrrrr}
\toprule
\multirow{2}{*}{}&  \multicolumn{2}{c}{\budalg} & \multicolumn{2}{c}{\murtree} & \multicolumn{2}{c}{\dleight} & \multicolumn{2}{c}{\cp} & \multicolumn{2}{c}{binoct} & \multicolumn{2}{c}{\cart}\\
\cmidrule(rr){2-3}\cmidrule(rr){4-5}\cmidrule(rr){6-7}\cmidrule(rr){8-9}\cmidrule(rr){10-11}\cmidrule(rr){12-13}
& \multicolumn{1}{c}{error} & \multicolumn{1}{c}{cpu} & \multicolumn{1}{c}{error} & \multicolumn{1}{c}{cpu} & \multicolumn{1}{c}{error} & \multicolumn{1}{c}{cpu} & \multicolumn{1}{c}{error} & \multicolumn{1}{c}{cpu} & \multicolumn{1}{c}{error} & \multicolumn{1}{c}{cpu} & \multicolumn{1}{c}{error} & \multicolumn{1}{c}{cpu} \\
\midrule

\texttt{adult\_discretized} & 4609 & 14$^*$ & 4609 & 30$^*$ & 4609 & 271$^*$ & 4609 & 246$^*$ & 5659 & 3392 & 5022 & 0.06\\
\texttt{car} & 136 & 0.19$^*$ & 136 & 0.16$^*$ & 136 & 0.36$^*$ & 136 & 2.8$^*$ & 178 & 871 & 178 & 0.00\\
\texttt{car\_evaluation} & 130 & 0.02$^*$ & 130 & 0.07$^*$ & 130 & 0.13$^*$ & 130 & 1.3$^*$ & - & - & 130 & 0.00\\
\texttt{chess} & 0 & 0.00$^*$ & 0 & 0.00$^*$ & 0 & 0.00$^*$ & 0 & 0.07$^*$ & - & - & 0 & 0.00\\
\texttt{compas\_discretized} & 1954 & 0.07$^*$ & 1954 & 1.0$^*$ & 1954 & 3.5$^*$ & 1954 & 6.3$^*$ & 1991 & 3390 & 1997 & 0.01\\
\texttt{HTRU\_2} & 385 & 74$^*$ & 385 & 122$^*$ & 385 & 450$^*$ & 385 & 295$^*$ & - & - & 409 & 0.05\\
\texttt{hypothyroid} & 53 & 2.9$^*$ & 53 & 16$^*$ & 53 & 181$^*$ & 53 & 254$^*$ & 55 & 3071 & 53 & 0.01\\
\texttt{kr-vs-kp} & 144 & 2.8$^*$ & 144 & 6.9$^*$ & 144 & 88$^*$ & 144 & 141$^*$ & 189 & 2850 & 189 & 0.01\\
\texttt{magic04} & 3112 & 232$^*$ & 3112 & 328$^*$ & 3112 & 1296$^*$ & 3112 & 800$^*$ & - & - & 3350 & 0.07\\
\texttt{seismic\_bumps} & 148 & 22$^*$ & 148 & 56$^*$ & 148 & 290$^*$ & 148 & 303$^*$ & - & - & 158 & 0.01\\
\texttt{winequality-red} & 4 & 0.62$^*$ & 4 & 1.2$^*$ & 4 & 4.3$^*$ & 4 & 12$^*$ & - & - & 8 & 0.00\\
\bottomrule
\end{tabular}
%
\end{scriptsize}%
\end{center}%
\caption{\label{tab:all43} Comparison with state of the art: $\numex \geq 5000, \numfeat < 250$, depth 4}%
\end{table}%

\begin{table}[htbp]%
\begin{center}%
\begin{scriptsize}%
\tabcolsep=2pt%
\begin{tabular}{lrrrrrrrrrrrr}
\toprule
\multirow{2}{*}{}&  \multicolumn{2}{c}{\budalg} & \multicolumn{2}{c}{\murtree} & \multicolumn{2}{c}{\dleight} & \multicolumn{2}{c}{\cp} & \multicolumn{2}{c}{binoct} & \multicolumn{2}{c}{\cart}\\
\cmidrule(rr){2-3}\cmidrule(rr){4-5}\cmidrule(rr){6-7}\cmidrule(rr){8-9}\cmidrule(rr){10-11}\cmidrule(rr){12-13}
& \multicolumn{1}{c}{error} & \multicolumn{1}{c}{cpu} & \multicolumn{1}{c}{error} & \multicolumn{1}{c}{cpu} & \multicolumn{1}{c}{error} & \multicolumn{1}{c}{cpu} & \multicolumn{1}{c}{error} & \multicolumn{1}{c}{cpu} & \multicolumn{1}{c}{error} & \multicolumn{1}{c}{cpu} & \multicolumn{1}{c}{error} & \multicolumn{1}{c}{cpu} \\
\midrule

\texttt{bank} & 4314 & 290 & 4686 & 2.5 & 4808 & $\mathsmaller{\geq}1$h & 5289 & $\mathsmaller{\geq}1$h & - & - & 4420 & 32\\
\texttt{bank\_conv} & 392 & 1963$^*$ & 392 & 1651$^*$ & - & - & 392 & $\mathsmaller{\geq}1$h & - & - & 408 & 0.04\\
\texttt{default\_credit} & 5270 & 209 & 5270 & 430 & 5306 & $\mathsmaller{\geq}1$h & 5270 & $\mathsmaller{\geq}1$h & - & - & 5306 & 0.69\\
\texttt{hand\_posture} & 4896 & 976 & 5778 & 1432 & 11021 & $\mathsmaller{\geq}1$h & 16265 & $\mathsmaller{\geq}1$h & - & - & 6098 & 27\\
\texttt{letter} & 261 & 1185$^*$ & 261 & 2956$^*$ & 335 & $\mathsmaller{\geq}1$h & 261 & $\mathsmaller{\geq}1$h & 813 & 0.00 & 462 & 0.20\\
\texttt{mnist\_0} & 2173 & 2158 & 1951 & 3542 & 3319 & $\mathsmaller{\geq}1$h & 5923 & $\mathsmaller{\geq}1$h & - & - & 2311 & 3.8\\
\texttt{mushroom} & 0 & 0.00$^*$ & 0 & 0.03$^*$ & 0 & 41$^*$ & 0 & 0.07$^*$ & 192 & 3354 & 4 & 0.02\\
\texttt{pendigits} & 13 & 230$^*$ & 13 & 833$^*$ & - & - & 14 & $\mathsmaller{\geq}1$h & 780 & 0.00 & 25 & 0.07\\
\texttt{segment} & 0 & 0.00$^*$ & 0 & 0.02$^*$ & 0 & 1.6$^*$ & 0 & 2.5$^*$ & 1 & 3501 & 1 & 0.01\\
\texttt{spambase} & 590 & 7.7 & 590 & 3295$^*$ & - & - & 590 & $\mathsmaller{\geq}1$h & - & - & 624 & 0.06\\
\texttt{splice-1} & 141 & 3241$^*$ & 141 & 644$^*$ & - & - & 141 & $\mathsmaller{\geq}1$h & 568 & 3416 & 141 & 0.03\\
\texttt{Statlog\_satellite} & 111 & 3571 & 116 & 1306 & - & - & 136 & $\mathsmaller{\geq}1$h & - & - & 204 & 0.08\\
\texttt{Statlog\_shuttle} & 0 & 0.64$^*$ & 0 & 41$^*$ & 1 & $\mathsmaller{\geq}1$h & 0 & 42$^*$ & - & - & 36 & 2.4\\
\texttt{surgical-deepnet} & 2269 & 49 & 2506 & 489 & - & - & 3690 & $\mathsmaller{\geq}1$h & - & - & 2704 & 6.2\\
\texttt{taiwan\_binarised} & 5273 & 6.2 & 5273 & 37 & 5307 & $\mathsmaller{\geq}1$h & 5273 & $\mathsmaller{\geq}1$h & 6521 & 75 & 5306 & 0.27\\
\texttt{weather-aus} & 1749 & 2525 & 1750 & 1243 & - & - & 1752 & $\mathsmaller{\geq}1$h & - & - & 1761 & 20\\
\bottomrule
\end{tabular}
%
\end{scriptsize}%
\end{center}%
\caption{\label{tab:all44} Comparison with state of the art: $\numex \geq 5000, \numfeat \geq 250$, depth 4}%
\end{table}%

\begin{table}[htbp]%
\begin{center}%
\begin{scriptsize}%
\tabcolsep=2pt%
\begin{tabular}{lrrrrrrrrrrrr}
\toprule
\multirow{2}{*}{}&  \multicolumn{2}{c}{\budalg} & \multicolumn{2}{c}{\murtree} & \multicolumn{2}{c}{\dleight} & \multicolumn{2}{c}{\cp} & \multicolumn{2}{c}{binoct} & \multicolumn{2}{c}{\cart}\\
\cmidrule(rr){2-3}\cmidrule(rr){4-5}\cmidrule(rr){6-7}\cmidrule(rr){8-9}\cmidrule(rr){10-11}\cmidrule(rr){12-13}
& \multicolumn{1}{c}{error} & \multicolumn{1}{c}{cpu} & \multicolumn{1}{c}{error} & \multicolumn{1}{c}{cpu} & \multicolumn{1}{c}{error} & \multicolumn{1}{c}{cpu} & \multicolumn{1}{c}{error} & \multicolumn{1}{c}{cpu} & \multicolumn{1}{c}{error} & \multicolumn{1}{c}{cpu} & \multicolumn{1}{c}{error} & \multicolumn{1}{c}{cpu} \\
\midrule

\texttt{anneal} & 70 & 44$^*$ & 70 & 148$^*$ & - & - & 75 & $\mathsmaller{\geq}1$h & 101 & 2995 & 123 & 0.00\\
\texttt{balance-scale} & 45 & 0.46$^*$ & 45 & 0.44$^*$ & 45 & 1.0$^*$ & 45 & 7.9$^*$ & - & - & 49 & 0.00\\
\texttt{banknote} & 3 & 0.88$^*$ & 3 & 2.2$^*$ & 3 & 3.9$^*$ & 3 & 34$^*$ & - & - & 15 & 0.00\\
\texttt{breast-cancer} & 6 & 725$^*$ & 6 & 72$^*$ & 6 & 438$^*$ & 6 & $\mathsmaller{\geq}1$h & 14 & 2894 & 16 & 0.00\\
\texttt{heart-cleveland} & 7 & 93$^*$ & 7 & 101$^*$ & - & - & 7 & $\mathsmaller{\geq}1$h & 26 & 3288 & 26 & 0.00\\
\texttt{hepatitis} & 0 & 0.05$^*$ & 0 & 0.18$^*$ & 0 & 71$^*$ & 0 & 12$^*$ & 6 & 3026 & 8 & 0.00\\
\texttt{IndiansDiabetes} & 125 & 30$^*$ & 125 & 19$^*$ & 125 & 125$^*$ & 125 & 410$^*$ & - & - & 162 & 0.00\\
\texttt{iris} & 1 & 0.00$^*$ & 1 & 0.00$^*$ & 1 & 0.01$^*$ & 1 & 1.2$^*$ & - & - & 1 & 0.00\\
\texttt{lymph} & 0 & 0.00$^*$ & 0 & 0.00$^*$ & 0 & 14$^*$ & 0 & 2.7$^*$ & 7 & 3380 & 4 & 0.00\\
\texttt{messidor} & 281 & 1522$^*$ & 281 & 855$^*$ & - & - & 292 & $\mathsmaller{\geq}1$h & - & - & 345 & 0.00\\
\texttt{monk1} & 0 & 0.00$^*$ & 0 & 0.00$^*$ & 0 & 0.00$^*$ & 0 & 0.23$^*$ & - & - & 9 & 0.00\\
\texttt{monk2} & 15 & 0.05$^*$ & 15 & 0.05$^*$ & 15 & 0.09$^*$ & 15 & 2.8$^*$ & - & - & 32 & 0.00\\
\texttt{monk3} & 2 & 0.03$^*$ & 2 & 0.03$^*$ & 2 & 0.03$^*$ & 2 & 2.2$^*$ & - & - & 5 & 0.00\\
\texttt{primary-tumor} & 26 & 0.38$^*$ & 26 & 1.5$^*$ & 26 & 24$^*$ & 26 & 103$^*$ & 34 & 3255 & 35 & 0.00\\
\texttt{soybean} & 8 & 20$^*$ & 8 & 7.6$^*$ & 8 & 63$^*$ & 8 & 752$^*$ & 14 & 3178 & 23 & 0.00\\
\texttt{tic-tac-toe} & 63 & 10$^*$ & 63 & 2.3$^*$ & 63 & 14$^*$ & 63 & 89$^*$ & 125 & 3052 & 78 & 0.00\\
\texttt{vote} & 1 & 24$^*$ & 1 & 6.1$^*$ & 1 & 45$^*$ & 1 & 522$^*$ & 8 & 1319 & 6 & 0.00\\
\texttt{yeast} & 313 & 139$^*$ & 313 & 558$^*$ & - & - & 315 & $\mathsmaller{\geq}1$h & 376 & 3456 & 367 & 0.01\\
\bottomrule
\end{tabular}
%
\end{scriptsize}%
\end{center}%
\caption{\label{tab:all51} Comparison with state of the art: $\numex<5000, \numfeat<250$, depth 5}%
\end{table}%

\begin{table}[htbp]%
\begin{center}%
\begin{scriptsize}%
\tabcolsep=2pt%
\begin{tabular}{lrrrrrrrrrrrr}
\toprule
\multirow{2}{*}{}&  \multicolumn{2}{c}{\budalg} & \multicolumn{2}{c}{\murtree} & \multicolumn{2}{c}{\dleight} & \multicolumn{2}{c}{\cp} & \multicolumn{2}{c}{binoct} & \multicolumn{2}{c}{\cart}\\
\cmidrule(rr){2-3}\cmidrule(rr){4-5}\cmidrule(rr){6-7}\cmidrule(rr){8-9}\cmidrule(rr){10-11}\cmidrule(rr){12-13}
& \multicolumn{1}{c}{error} & \multicolumn{1}{c}{cpu} & \multicolumn{1}{c}{error} & \multicolumn{1}{c}{cpu} & \multicolumn{1}{c}{error} & \multicolumn{1}{c}{cpu} & \multicolumn{1}{c}{error} & \multicolumn{1}{c}{cpu} & \multicolumn{1}{c}{error} & \multicolumn{1}{c}{cpu} & \multicolumn{1}{c}{error} & \multicolumn{1}{c}{cpu} \\
\midrule

\texttt{audiology} & 0 & 0.00$^*$ & 0 & 0.02$^*$ & 0 & 0.05$^*$ & 0 & 7.0$^*$ & 1 & 3083 & 2 & 0.00\\
\texttt{australian-credit} & 39 & 658$^*$ & 39 & 872$^*$ & - & - & 40 & $\mathsmaller{\geq}1$h & 72 & 3282 & 64 & 0.00\\
\texttt{biodeg} & 88 & 268 & 88 & 1141 & - & - & 356 & $\mathsmaller{\geq}1$h & - & - & 127 & 0.01\\
\texttt{breast-wisconsin} & 0 & 20$^*$ & 0 & 72$^*$ & - & - & 1 & $\mathsmaller{\geq}1$h & 16 & 3105 & 13 & 0.00\\
\texttt{diabetes} & 106 & 312$^*$ & 106 & 920$^*$ & - & - & 107 & $\mathsmaller{\geq}1$h & 160 & 3501 & 141 & 0.00\\
\texttt{forest-fires} & 156 & 777 & 149 & 2977 & - & - & 172 & $\mathsmaller{\geq}1$h & 207 & 3386 & 177 & 0.01\\
\texttt{german-credit} & 161 & 2741$^*$ & 161 & 973$^*$ & - & - & 161 & $\mathsmaller{\geq}1$h & 221 & 3504 & 209 & 0.01\\
\texttt{ionosphere} & 0 & 506$^*$ & 0 & 1340$^*$ & - & - & 4 & $\mathsmaller{\geq}1$h & 25 & 3386 & 17 & 0.01\\
\texttt{titanic} & 95 & 1428 & 95 & 1371 & - & - & 342 & $\mathsmaller{\geq}1$h & 149 & 3505 & 130 & 0.01\\
\texttt{vehicle} & 1 & 690 & 1 & 1540 & - & - & 218 & $\mathsmaller{\geq}1$h & 85 & 3502 & 23 & 0.01\\
\texttt{wine1} & 33 & 1154 & 33 & 287 & - & - & 38 & $\mathsmaller{\geq}1$h & 46 & 2910 & 39 & 0.01\\
\texttt{wine2} & 39 & 411 & 37 & 3400 & - & - & 42 & $\mathsmaller{\geq}1$h & 50 & 3197 & 44 & 0.01\\
\texttt{wine3} & 25 & 17 & 25 & 25 & - & - & 28 & $\mathsmaller{\geq}1$h & 37 & 3288 & 30 & 0.01\\
\bottomrule
\end{tabular}
%
\end{scriptsize}%
\end{center}%
\caption{\label{tab:all52} Comparison with state of the art: $\numex<5000, \numfeat \geq 250$, depth 5}%
\end{table}%

\begin{table}[htbp]%
\begin{center}%
\begin{scriptsize}%
\tabcolsep=2pt%
\begin{tabular}{lrrrrrrrrrrrr}
\toprule
\multirow{2}{*}{}&  \multicolumn{2}{c}{\budalg} & \multicolumn{2}{c}{\murtree} & \multicolumn{2}{c}{\dleight} & \multicolumn{2}{c}{\cp} & \multicolumn{2}{c}{binoct} & \multicolumn{2}{c}{\cart}\\
\cmidrule(rr){2-3}\cmidrule(rr){4-5}\cmidrule(rr){6-7}\cmidrule(rr){8-9}\cmidrule(rr){10-11}\cmidrule(rr){12-13}
& \multicolumn{1}{c}{error} & \multicolumn{1}{c}{cpu} & \multicolumn{1}{c}{error} & \multicolumn{1}{c}{cpu} & \multicolumn{1}{c}{error} & \multicolumn{1}{c}{cpu} & \multicolumn{1}{c}{error} & \multicolumn{1}{c}{cpu} & \multicolumn{1}{c}{error} & \multicolumn{1}{c}{cpu} & \multicolumn{1}{c}{error} & \multicolumn{1}{c}{cpu} \\
\midrule

\texttt{adult\_discretized} & 4423 & 725$^*$ & 4423 & 794$^*$ & 4442 & $\mathsmaller{\geq}1$h & 4423 & $\mathsmaller{\geq}1$h & 7157 & 20 & 4728 & 0.08\\
\texttt{car} & 86 & 2.4$^*$ & 86 & 1.2$^*$ & 86 & 2.7$^*$ & 86 & 21$^*$ & 138 & 3379 & 106 & 0.01\\
\texttt{car\_evaluation} & 90 & 0.13$^*$ & 90 & 0.37$^*$ & 90 & 0.49$^*$ & 90 & 4.9$^*$ & - & - & 116 & 0.00\\
\texttt{chess} & 0 & 0.00$^*$ & 0 & 0.00$^*$ & 0 & 0.00$^*$ & 0 & 0.08$^*$ & - & - & 0 & 0.00\\
\texttt{compas\_discretized} & 1919 & 1.1$^*$ & 1919 & 11$^*$ & 1919 & 26$^*$ & 1919 & 77$^*$ & 1952 & 3153 & 1968 & 0.01\\
\texttt{HTRU\_2} & 361 & 98 & 361 & 2724$^*$ & 369 & $\mathsmaller{\geq}1$h & 361 & $\mathsmaller{\geq}1$h & - & - & 394 & 0.06\\
\texttt{hypothyroid} & 44 & 87$^*$ & 44 & 343$^*$ & - & - & 45 & $\mathsmaller{\geq}1$h & 64 & 3324 & 50 & 0.01\\
\texttt{kr-vs-kp} & 81 & 65$^*$ & 81 & 150$^*$ & - & - & 81 & $\mathsmaller{\geq}1$h & 189 & 3502 & 189 & 0.01\\
\texttt{magic04} & 2882 & 756 & 2882 & 873 & 2910 & $\mathsmaller{\geq}1$h & 2882 & $\mathsmaller{\geq}1$h & - & - & 3179 & 0.11\\
\texttt{seismic\_bumps} & 132 & 1533$^*$ & 132 & 1617$^*$ & - & - & 134 & $\mathsmaller{\geq}1$h & - & - & 151 & 0.01\\
\texttt{winequality-red} & 3 & 16$^*$ & 3 & 18$^*$ & 3 & 39$^*$ & 3 & 232$^*$ & - & - & 6 & 0.00\\
\bottomrule
\end{tabular}
%
\end{scriptsize}%
\end{center}%
\caption{\label{tab:all53} Comparison with state of the art: $\numex \geq 5000, \numfeat < 250$, depth 5}%
\end{table}%

\begin{table}[htbp]%
\begin{center}%
\begin{scriptsize}%
\tabcolsep=2pt%
\begin{tabular}{lrrrrrrrrrrrr}
\toprule
\multirow{2}{*}{}&  \multicolumn{2}{c}{\budalg} & \multicolumn{2}{c}{\murtree} & \multicolumn{2}{c}{\dleight} & \multicolumn{2}{c}{\cp} & \multicolumn{2}{c}{binoct} & \multicolumn{2}{c}{\cart}\\
\cmidrule(rr){2-3}\cmidrule(rr){4-5}\cmidrule(rr){6-7}\cmidrule(rr){8-9}\cmidrule(rr){10-11}\cmidrule(rr){12-13}
& \multicolumn{1}{c}{error} & \multicolumn{1}{c}{cpu} & \multicolumn{1}{c}{error} & \multicolumn{1}{c}{cpu} & \multicolumn{1}{c}{error} & \multicolumn{1}{c}{cpu} & \multicolumn{1}{c}{error} & \multicolumn{1}{c}{cpu} & \multicolumn{1}{c}{error} & \multicolumn{1}{c}{cpu} & \multicolumn{1}{c}{error} & \multicolumn{1}{c}{cpu} \\
\midrule

\texttt{bank} & 4187 & 1152 & 4365 & 2093 & 4809 & $\mathsmaller{\geq}1$h & 5289 & $\mathsmaller{\geq}1$h & - & - & 4358 & 47\\
\texttt{bank\_conv} & 340 & 1662 & 340 & 1636 & - & - & 521 & $\mathsmaller{\geq}1$h & - & - & 379 & 0.04\\
\texttt{default\_credit} & 5181 & 3202 & 5251 & 3121 & 5334 & $\mathsmaller{\geq}1$h & 6636 & $\mathsmaller{\geq}1$h & - & - & 5273 & 1.0\\
\texttt{hand\_posture} & 3154 & 56 & 4482 & 1297 & 11736 & $\mathsmaller{\geq}1$h & 16265 & $\mathsmaller{\geq}1$h & - & - & 3377 & 42\\
\texttt{letter} & 168 & 3082 & 190 & 549 & 352 & $\mathsmaller{\geq}1$h & 813 & $\mathsmaller{\geq}1$h & 813 & 0.00 & 335 & 0.32\\
\texttt{mnist\_0} & 1714 & 284 & 2066 & 2149 & 3319 & $\mathsmaller{\geq}1$h & 5923 & $\mathsmaller{\geq}1$h & - & - & 2021 & 4.5\\
\texttt{mushroom} & 0 & 0.00$^*$ & 0 & 0.03$^*$ & 0 & 36$^*$ & 0 & 0.10$^*$ & 1930 & 19 & 3 & 0.03\\
\texttt{pendigits} & 0 & 284$^*$ & 0 & 1295$^*$ & - & - & 780 & $\mathsmaller{\geq}1$h & 751 & 30 & 11 & 0.07\\
\texttt{segment} & 0 & 0.00$^*$ & 0 & 0.02$^*$ & 0 & 1.0$^*$ & 0 & 2.0$^*$ & 41 & 2839 & 1 & 0.01\\
\texttt{spambase} & 501 & 219 & 501 & 2340 & - & - & 1813 & $\mathsmaller{\geq}1$h & - & - & 571 & 0.05\\
\texttt{splice-1} & 101 & 24 & 100 & 3308 & - & - & 1535 & $\mathsmaller{\geq}1$h & 814 & 16 & 117 & 0.04\\
\texttt{Statlog\_satellite} & 71 & 279 & 98 & 638 & - & - & 1072 & $\mathsmaller{\geq}1$h & - & - & 128 & 0.13\\
\texttt{Statlog\_shuttle} & 0 & 0.06$^*$ & 0 & 39$^*$ & 1 & $\mathsmaller{\geq}1$h & 0 & 34$^*$ & - & - & 10 & 2.8\\
\texttt{surgical-deepnet} & 2131 & 2168 & 2337 & 400 & - & - & 3690 & $\mathsmaller{\geq}1$h & - & - & 2245 & 8.4\\
\texttt{taiwan\_binarised} & 5200 & 105 & 5261 & 38 & 5412 & $\mathsmaller{\geq}1$h & 6636 & $\mathsmaller{\geq}1$h & 6636 & 0.00 & 5280 & 0.37\\
\texttt{weather-aus} & 1735 & 419 & 1735 & 1907 & - & - & 1761 & $\mathsmaller{\geq}1$h & - & - & 1751 & 26\\
\bottomrule
\end{tabular}
%
\end{scriptsize}%
\end{center}%
\caption{\label{tab:all54} Comparison with state of the art: $\numex \geq 5000, \numfeat \geq 250$, depth 5}%
\end{table}%


\begin{table}[htbp]%
\begin{center}%
\begin{scriptsize}%
\tabcolsep=2pt%
\begin{tabular}{lrrrrrrrrrrrr}
\toprule
\multirow{2}{*}{}&  \multicolumn{2}{c}{\budalg} & \multicolumn{2}{c}{\murtree} & \multicolumn{2}{c}{\dleight} & \multicolumn{2}{c}{\cp} & \multicolumn{2}{c}{binoct} & \multicolumn{2}{c}{\cart}\\
\cmidrule(rr){2-3}\cmidrule(rr){4-5}\cmidrule(rr){6-7}\cmidrule(rr){8-9}\cmidrule(rr){10-11}\cmidrule(rr){12-13}
& \multicolumn{1}{c}{error} & \multicolumn{1}{c}{cpu} & \multicolumn{1}{c}{error} & \multicolumn{1}{c}{cpu} & \multicolumn{1}{c}{error} & \multicolumn{1}{c}{cpu} & \multicolumn{1}{c}{error} & \multicolumn{1}{c}{cpu} & \multicolumn{1}{c}{error} & \multicolumn{1}{c}{cpu} & \multicolumn{1}{c}{error} & \multicolumn{1}{c}{cpu} \\
\midrule

\texttt{anneal} & 41 & 3036 & 50 & 836 & - & - & 187 & $\mathsmaller{\geq}1$h & 106 & 3386 & 96 & 0.00\\
\texttt{balance-scale} & 29 & 37$^*$ & 29 & 17$^*$ & 29 & 10$^*$ & 29 & 228$^*$ & - & - & 49 & 0.00\\
\texttt{banknote} & 2 & 0.00$^*$ & 2 & 69$^*$ & 2 & 80$^*$ & 2 & $\mathsmaller{\geq}1$h & - & - & 5 & 0.00\\
\texttt{breast-cancer} & 0 & 1007$^*$ & 0 & 150$^*$ & 0 & 450$^*$ & 1 & $\mathsmaller{\geq}1$h & 19 & 3313 & 8 & 0.00\\
\texttt{heart-cleveland} & 0 & 0.00$^*$ & 0 & 0.04$^*$ & - & - & 0 & 3.0$^*$ & 17 & 3368 & 6 & 0.01\\
\texttt{hepatitis} & 0 & 0.00$^*$ & 0 & 0.01$^*$ & 0 & 8.9$^*$ & 0 & 0.49$^*$ & 1 & 3436 & 1 & 0.00\\
\texttt{IndiansDiabetes} & 44 & 3343 & 44 & 1355$^*$ & - & - & 268 & $\mathsmaller{\geq}1$h & - & - & 113 & 0.00\\
\texttt{iris} & 1 & 0.00$^*$ & 1 & 0.00$^*$ & 1 & 0.01$^*$ & 1 & 2.6$^*$ & - & - & 1 & 0.00\\
\texttt{lymph} & 0 & 0.00$^*$ & 0 & 0.00$^*$ & 0 & 0.01$^*$ & 0 & 0.24$^*$ & 1 & 3431 & 0 & 0.00\\
\texttt{messidor} & 179 & 2456 & 203 & 1842 & - & - & 540 & $\mathsmaller{\geq}1$h & - & - & 305 & 0.01\\
\texttt{monk1} & 0 & 0.00$^*$ & 0 & 0.00$^*$ & 0 & 0.00$^*$ & 0 & 0.17$^*$ & - & - & 8 & 0.00\\
\texttt{monk2} & 0 & 0.00$^*$ & 0 & 0.02$^*$ & 0 & 0.00$^*$ & 0 & 0.78$^*$ & - & - & 5 & 0.00\\
\texttt{monk3} & 0 & 0.00$^*$ & 0 & 0.00$^*$ & 0 & 0.00$^*$ & 0 & 0.45$^*$ & - & - & 2 & 0.00\\
\texttt{primary-tumor} & 16 & 18$^*$ & 16 & 162$^*$ & 16 & 458$^*$ & 16 & $\mathsmaller{\geq}1$h & 24 & 3432 & 26 & 0.00\\
\texttt{soybean} & 2 & 19$^*$ & 2 & 1108$^*$ & - & - & 3 & $\mathsmaller{\geq}1$h & 13 & 1579 & 11 & 0.00\\
\texttt{tic-tac-toe} & 0 & 32$^*$ & 0 & 8.4$^*$ & 0 & 29$^*$ & 0 & 764$^*$ & 46 & 3449 & 22 & 0.00\\
\texttt{vote} & 0 & 0.00$^*$ & 0 & 0.00$^*$ & 0 & 0.17$^*$ & 0 & 3.2$^*$ & 2 & 3348 & 2 & 0.00\\
\texttt{yeast} & 182 & 3558 & 222 & 1088 & - & - & 463 & $\mathsmaller{\geq}1$h & 455 & 1968 & 306 & 0.02\\
\bottomrule
\end{tabular}
%
\end{scriptsize}%
\end{center}%
\caption{\label{tab:all71} Comparison with state of the art: $\numex<5000, \numfeat<250$, depth 7}%
\end{table}%

\begin{table}[htbp]%
\begin{center}%
\begin{scriptsize}%
\tabcolsep=2pt%
\begin{tabular}{lrrrrrrrrrrrr}
\toprule
\multirow{2}{*}{}&  \multicolumn{2}{c}{\budalg} & \multicolumn{2}{c}{\murtree} & \multicolumn{2}{c}{\dleight} & \multicolumn{2}{c}{\cp} & \multicolumn{2}{c}{binoct} & \multicolumn{2}{c}{\cart}\\
\cmidrule(rr){2-3}\cmidrule(rr){4-5}\cmidrule(rr){6-7}\cmidrule(rr){8-9}\cmidrule(rr){10-11}\cmidrule(rr){12-13}
& \multicolumn{1}{c}{error} & \multicolumn{1}{c}{cpu} & \multicolumn{1}{c}{error} & \multicolumn{1}{c}{cpu} & \multicolumn{1}{c}{error} & \multicolumn{1}{c}{cpu} & \multicolumn{1}{c}{error} & \multicolumn{1}{c}{cpu} & \multicolumn{1}{c}{error} & \multicolumn{1}{c}{cpu} & \multicolumn{1}{c}{error} & \multicolumn{1}{c}{cpu} \\
\midrule

\texttt{audiology} & 0 & 0.00$^*$ & 0 & 0.01$^*$ & 0 & 0.00$^*$ & 0 & 0.18$^*$ & 3 & 2177 & 0 & 0.00\\
\texttt{australian-credit} & 0 & 101$^*$ & 0 & 126$^*$ & - & - & 296 & $\mathsmaller{\geq}1$h & 85 & 3320 & 43 & 0.01\\
\texttt{biodeg} & 26 & 2775 & 95 & 876 & - & - & 356 & $\mathsmaller{\geq}1$h & - & - & 86 & 0.02\\
\texttt{breast-wisconsin} & 0 & 0.02$^*$ & 0 & 0.09$^*$ & - & - & 0 & 2805$^*$ & 12 & 3502 & 4 & 0.00\\
\texttt{diabetes} & 21 & 827 & 81 & 2000 & - & - & 268 & $\mathsmaller{\geq}1$h & 179 & 1988 & 100 & 0.01\\
\texttt{forest-fires} & 146 & 125 & 139 & 2254 & - & - & 247 & $\mathsmaller{\geq}1$h & 270 & 0.00 & 161 & 0.02\\
\texttt{german-credit} & 56 & 1192 & 87 & 341 & - & - & 300 & $\mathsmaller{\geq}1$h & 246 & 2598 & 150 & 0.01\\
\texttt{ionosphere} & 0 & 0.07$^*$ & 0 & 0.37$^*$ & - & - & 0 & 566$^*$ & 61 & 213 & 7 & 0.01\\
\texttt{titanic} & 72 & 442 & 97 & 149 & - & - & 342 & $\mathsmaller{\geq}1$h & 342 & 0.00 & 111 & 0.01\\
\texttt{vehicle} & 0 & 0.09$^*$ & 0 & 0.47$^*$ & - & - & 0 & 1178$^*$ & 210 & 25 & 4 & 0.01\\
\texttt{wine1} & 28 & 892 & 28 & 325 & - & - & 36 & $\mathsmaller{\geq}1$h & 57 & 122 & 33 & 0.01\\
\texttt{wine2} & 31 & 28 & 31 & 25 & - & - & 35 & $\mathsmaller{\geq}1$h & 71 & 0.00 & 38 & 0.01\\
\texttt{wine3} & 21 & 524 & 20 & 2925 & - & - & 24 & $\mathsmaller{\geq}1$h & 47 & 142 & 24 & 0.01\\
\bottomrule
\end{tabular}
%
\end{scriptsize}%
\end{center}%
\caption{\label{tab:all72} Comparison with state of the art: $\numex<5000, \numfeat \geq 250$, depth 7}%
\end{table}%

\begin{table}[htbp]%
\begin{center}%
\begin{scriptsize}%
\tabcolsep=2pt%
\begin{tabular}{lrrrrrrrrrrrr}
\toprule
\multirow{2}{*}{}&  \multicolumn{2}{c}{\budalg} & \multicolumn{2}{c}{\murtree} & \multicolumn{2}{c}{\dleight} & \multicolumn{2}{c}{\cp} & \multicolumn{2}{c}{binoct} & \multicolumn{2}{c}{\cart}\\
\cmidrule(rr){2-3}\cmidrule(rr){4-5}\cmidrule(rr){6-7}\cmidrule(rr){8-9}\cmidrule(rr){10-11}\cmidrule(rr){12-13}
& \multicolumn{1}{c}{error} & \multicolumn{1}{c}{cpu} & \multicolumn{1}{c}{error} & \multicolumn{1}{c}{cpu} & \multicolumn{1}{c}{error} & \multicolumn{1}{c}{cpu} & \multicolumn{1}{c}{error} & \multicolumn{1}{c}{cpu} & \multicolumn{1}{c}{error} & \multicolumn{1}{c}{cpu} & \multicolumn{1}{c}{error} & \multicolumn{1}{c}{cpu} \\
\midrule

\texttt{adult\_discretized} & 4191 & 534 & 4294 & 2016 & 4998 & $\mathsmaller{\geq}1$h & 7511 & $\mathsmaller{\geq}1$h & 7511 & 0.00 & 4481 & 0.09\\
\texttt{car} & 11 & 231$^*$ & 11 & 27$^*$ & 11 & 16$^*$ & 11 & 1678$^*$ & 80 & 3495 & 50 & 0.00\\
\texttt{car\_evaluation} & 80 & 0.00$^*$ & 80 & 7.4$^*$ & 80 & 4.2$^*$ & 80 & 123$^*$ & - & - & 80 & 0.00\\
\texttt{chess} & 0 & 0.00$^*$ & 0 & 0.00$^*$ & 0 & 0.00$^*$ & 0 & 0.13$^*$ & - & - & 0 & 0.00\\
\texttt{compas\_discretized} & 1852 & 198$^*$ & 1852 & 569$^*$ & 1852 & 575$^*$ & 1857 & $\mathsmaller{\geq}1$h & 1940 & 3504 & 1941 & 0.01\\
\texttt{HTRU\_2} & 297 & 3334 & 293 & 2992 & 601 & $\mathsmaller{\geq}1$h & 1639 & $\mathsmaller{\geq}1$h & - & - & 352 & 0.08\\
\texttt{hypothyroid} & 22 & 3478 & 23 & 590 & - & - & 277 & $\mathsmaller{\geq}1$h & 277 & 274 & 42 & 0.01\\
\texttt{kr-vs-kp} & 18 & 2550 & 21 & 2051 & - & - & 37 & $\mathsmaller{\geq}1$h & 1096 & 43 & 103 & 0.01\\
\texttt{magic04} & 2488 & 2773 & 2851 & 1512 & 3140 & $\mathsmaller{\geq}1$h & 6688 & $\mathsmaller{\geq}1$h & - & - & 2768 & 0.11\\
\texttt{seismic\_bumps} & 76 & 2389 & 97 & 193 & - & - & 170 & $\mathsmaller{\geq}1$h & - & - & 137 & 0.01\\
\texttt{winequality-red} & 2 & 0.01$^*$ & 2 & 1131$^*$ & - & - & 10 & $\mathsmaller{\geq}1$h & - & - & 4 & 0.00\\
\bottomrule
\end{tabular}
%
\end{scriptsize}%
\end{center}%
\caption{\label{tab:all73} Comparison with state of the art: $\numex \geq 5000, \numfeat < 250$, depth 7}%
\end{table}%

\begin{table}[htbp]%
\begin{center}%
\begin{scriptsize}%
\tabcolsep=2pt%
\begin{tabular}{lrrrrrrrrrrrr}
\toprule
\multirow{2}{*}{}&  \multicolumn{2}{c}{\budalg} & \multicolumn{2}{c}{\murtree} & \multicolumn{2}{c}{\dleight} & \multicolumn{2}{c}{\cp} & \multicolumn{2}{c}{binoct} & \multicolumn{2}{c}{\cart}\\
\cmidrule(rr){2-3}\cmidrule(rr){4-5}\cmidrule(rr){6-7}\cmidrule(rr){8-9}\cmidrule(rr){10-11}\cmidrule(rr){12-13}
& \multicolumn{1}{c}{error} & \multicolumn{1}{c}{cpu} & \multicolumn{1}{c}{error} & \multicolumn{1}{c}{cpu} & \multicolumn{1}{c}{error} & \multicolumn{1}{c}{cpu} & \multicolumn{1}{c}{error} & \multicolumn{1}{c}{cpu} & \multicolumn{1}{c}{error} & \multicolumn{1}{c}{cpu} & \multicolumn{1}{c}{error} & \multicolumn{1}{c}{cpu} \\
\midrule

\texttt{bank} & 3844 & 2369 & 4232 & 2003 & 4807 & $\mathsmaller{\geq}1$h & 5289 & $\mathsmaller{\geq}1$h & - & - & 4038 & 77\\
\texttt{bank\_conv} & 220 & 1642 & 319 & 1174 & - & - & 521 & $\mathsmaller{\geq}1$h & - & - & 303 & 0.06\\
\texttt{default\_credit} & 4935 & 222 & 5237 & 223 & 5412 & $\mathsmaller{\geq}1$h & 6636 & $\mathsmaller{\geq}1$h & - & - & 5153 & 1.0\\
\texttt{hand\_posture} & 749 & 2684 & 1418 & 461 & 14236 & $\mathsmaller{\geq}1$h & 16265 & $\mathsmaller{\geq}1$h & - & - & 962 & 78\\
\texttt{letter} & 68 & 177 & 131 & 86 & 488 & $\mathsmaller{\geq}1$h & 813 & $\mathsmaller{\geq}1$h & - & - & 153 & 0.31\\
\texttt{mnist\_0} & 1107 & 2895 & 1601 & 964 & 3296 & $\mathsmaller{\geq}1$h & 5923 & $\mathsmaller{\geq}1$h & - & - & 1323 & 8.5\\
\texttt{mushroom} & 0 & 0.00$^*$ & 0 & 0.02$^*$ & 0 & 10$^*$ & 0 & 0.15$^*$ & 4208 & 0.00 & 0 & 0.03\\
\texttt{pendigits} & 0 & 0.00$^*$ & 0 & 0.15$^*$ & - & - & 0 & 8.1$^*$ & 780 & 0.00 & 1 & 0.07\\
\texttt{segment} & 0 & 0.00$^*$ & 0 & 0.02$^*$ & 0 & 0.23$^*$ & 0 & 0.28$^*$ & 330 & 0.00 & 0 & 0.01\\
\texttt{spambase} & 352 & 3562 & 495 & 2265 & - & - & 1813 & $\mathsmaller{\geq}1$h & - & - & 462 & 0.08\\
\texttt{splice-1} & 29 & 3484 & 47 & 881 & - & - & 1535 & $\mathsmaller{\geq}1$h & 1655 & 0.00 & 58 & 0.05\\
\texttt{Statlog\_satellite} & 14 & 2428 & 89 & 2540 & - & - & 1072 & $\mathsmaller{\geq}1$h & - & - & 41 & 0.12\\
\texttt{Statlog\_shuttle} & 0 & 0.04$^*$ & 0 & 42$^*$ & 0 & 3163$^*$ & 0 & 14$^*$ & - & - & 4 & 2.8\\
\texttt{surgical-deepnet} & 1647 & 1248 & 1890 & 655 & - & - & 3690 & $\mathsmaller{\geq}1$h & - & - & 1871 & 9.9\\
\texttt{taiwan\_binarised} & 4896 & 1958 & 5189 & 3125 & 5412 & $\mathsmaller{\geq}1$h & 6636 & $\mathsmaller{\geq}1$h & - & - & 5161 & 0.58\\
\texttt{weather-aus} & 1685 & 2048 & 1724 & 3257 & - & - & 1761 & $\mathsmaller{\geq}1$h & - & - & 1721 & 27\\
\bottomrule
\end{tabular}
%
\end{scriptsize}%
\end{center}%
\caption{\label{tab:all74} Comparison with state of the art: $\numex \geq 5000, \numfeat \geq 250$, depth 7}%
\end{table}%


\begin{table}[htbp]%
\begin{center}%
\begin{scriptsize}%
\tabcolsep=2pt%
\begin{tabular}{lrrrrrrrrrrrr}
\toprule
\multirow{2}{*}{}&  \multicolumn{2}{c}{\budalg} & \multicolumn{2}{c}{\murtree} & \multicolumn{2}{c}{\dleight} & \multicolumn{2}{c}{\cp} & \multicolumn{2}{c}{binoct} & \multicolumn{2}{c}{\cart}\\
\cmidrule(rr){2-3}\cmidrule(rr){4-5}\cmidrule(rr){6-7}\cmidrule(rr){8-9}\cmidrule(rr){10-11}\cmidrule(rr){12-13}
& \multicolumn{1}{c}{error} & \multicolumn{1}{c}{cpu} & \multicolumn{1}{c}{error} & \multicolumn{1}{c}{cpu} & \multicolumn{1}{c}{error} & \multicolumn{1}{c}{cpu} & \multicolumn{1}{c}{error} & \multicolumn{1}{c}{cpu} & \multicolumn{1}{c}{error} & \multicolumn{1}{c}{cpu} & \multicolumn{1}{c}{error} & \multicolumn{1}{c}{cpu} \\
\midrule

\texttt{anneal} & 34 & 23$^*$ & 39 & 225 & - & - & 187 & $\mathsmaller{\geq}1$h & 625 & 0.00 & 59 & 0.00\\
\texttt{balance-scale} & 0 & 19$^*$ & 0 & 23$^*$ & 0 & 1.5$^*$ & 0 & 16$^*$ & - & - & 6 & 0.00\\
\texttt{banknote} & 2 & 0.00$^*$ & 2 & 1123$^*$ & 2 & 738$^*$ & 610 & $\mathsmaller{\geq}1$h & - & - & 2 & 0.00\\
\texttt{breast-cancer} & 0 & 0.00$^*$ & 0 & 0.01$^*$ & 0 & 0.00$^*$ & 0 & 2.4$^*$ & 239 & 0.00 & 0 & 0.00\\
\texttt{heart-cleveland} & 0 & 0.00$^*$ & 0 & 0.02$^*$ & 0 & 0.08$^*$ & 0 & 1.2$^*$ & 127 & 7.6 & 0 & 0.00\\
\texttt{hepatitis} & 0 & 0.00$^*$ & 0 & 0.00$^*$ & 0 & 0.00$^*$ & 0 & 1.3$^*$ & 19 & 2032 & 0 & 0.00\\
\texttt{IndiansDiabetes} & 8 & 4.7$^*$ & 44 & 1200 & - & - & 268 & $\mathsmaller{\geq}1$h & - & - & 63 & 0.00\\
\texttt{iris} & 1 & 0.00$^*$ & 1 & 0.00$^*$ & 1 & 0.01$^*$ & 1 & 21$^*$ & - & - & 1 & 0.00\\
\texttt{lymph} & 0 & 0.00$^*$ & 0 & 0.00$^*$ & 0 & 0.00$^*$ & 0 & 1.2$^*$ & 30 & 576 & 0 & 0.00\\
\texttt{messidor} & 66 & 604 & 168 & 456 & - & - & 540 & $\mathsmaller{\geq}1$h & - & - & 211 & 0.03\\
\texttt{monk1} & 0 & 0.00$^*$ & 0 & 0.00$^*$ & 0 & 0.00$^*$ & 0 & 0.64$^*$ & - & - & 0 & 0.00\\
\texttt{monk2} & 0 & 0.00$^*$ & 0 & 0.00$^*$ & 0 & 0.00$^*$ & 0 & 0.89$^*$ & - & - & 0 & 0.00\\
\texttt{monk3} & 0 & 0.00$^*$ & 0 & 0.00$^*$ & 0 & 0.00$^*$ & 0 & 0.99$^*$ & - & - & 0 & 0.00\\
\texttt{primary-tumor} & 15 & 0.00$^*$ & 15 & 1820 & - & - & 82 & $\mathsmaller{\geq}1$h & 31 & 3329 & 20 & 0.00\\
\texttt{soybean} & 2 & 0.00$^*$ & 2 & 18 & - & - & 92 & $\mathsmaller{\geq}1$h & 84 & 11 & 2 & 0.00\\
\texttt{tic-tac-toe} & 0 & 0.00$^*$ & 0 & 0.01$^*$ & 0 & 0.03$^*$ & 0 & 0.81$^*$ & 332 & 194 & 6 & 0.00\\
\texttt{vote} & 0 & 0.00$^*$ & 0 & 0.00$^*$ & 0 & 0.00$^*$ & 0 & 2.3$^*$ & 132 & 9.9 & 0 & 0.00\\
\texttt{yeast} & 28 & 1008 & 170 & 962 & - & - & 463 & $\mathsmaller{\geq}1$h & 463 & 0.00 & 185 & 0.01\\
\bottomrule
\end{tabular}
%
\end{scriptsize}%
\end{center}%
\caption{\label{tab:all101} Comparison with state of the art: $\numex<5000, \numfeat<250$, depth 10}%
\end{table}%

\begin{table}[htbp]%
\begin{center}%
\begin{scriptsize}%
\tabcolsep=2pt%
\begin{tabular}{lrrrrrrrrrrrr}
\toprule
\multirow{2}{*}{}&  \multicolumn{2}{c}{\budalg} & \multicolumn{2}{c}{\murtree} & \multicolumn{2}{c}{\dleight} & \multicolumn{2}{c}{\cp} & \multicolumn{2}{c}{binoct} & \multicolumn{2}{c}{\cart}\\
\cmidrule(rr){2-3}\cmidrule(rr){4-5}\cmidrule(rr){6-7}\cmidrule(rr){8-9}\cmidrule(rr){10-11}\cmidrule(rr){12-13}
& \multicolumn{1}{c}{error} & \multicolumn{1}{c}{cpu} & \multicolumn{1}{c}{error} & \multicolumn{1}{c}{cpu} & \multicolumn{1}{c}{error} & \multicolumn{1}{c}{cpu} & \multicolumn{1}{c}{error} & \multicolumn{1}{c}{cpu} & \multicolumn{1}{c}{error} & \multicolumn{1}{c}{cpu} & \multicolumn{1}{c}{error} & \multicolumn{1}{c}{cpu} \\
\midrule

\texttt{audiology} & 0 & 0.00$^*$ & 0 & 0.01$^*$ & 0 & 0.00$^*$ & 0 & 1.4$^*$ & 25 & 17 & 0 & 0.00\\
\texttt{australian-credit} & 0 & 0.04$^*$ & 0 & 0.27$^*$ & - & - & 0 & 464$^*$ & 357 & 0.00 & 12 & 0.01\\
\texttt{biodeg} & 1 & 1169$^*$ & 24 & 1652 & - & - & 356 & $\mathsmaller{\geq}1$h & - & - & 27 & 0.02\\
\texttt{breast-wisconsin} & 0 & 0.00$^*$ & 0 & 0.01$^*$ & 0 & 3.4$^*$ & 0 & 7.8$^*$ & 444 & 0.00 & 0 & 0.00\\
\texttt{diabetes} & 0 & 0.67$^*$ & 0 & 3.7$^*$ & - & - & 0 & 463$^*$ & 500 & 0.00 & 35 & 0.01\\
\texttt{forest-fires} & 113 & 942 & 119 & 458 & - & - & 247 & $\mathsmaller{\geq}1$h & - & - & 146 & 0.02\\
\texttt{german-credit} & 0 & 69$^*$ & 0 & 74$^*$ & - & - & 0 & 28$^*$ & 700 & 0.00 & 66 & 0.01\\
\texttt{ionosphere} & 0 & 0.00$^*$ & 0 & 0.14$^*$ & 0 & 110$^*$ & 0 & 8.1$^*$ & 225 & 0.00 & 0 & 0.01\\
\texttt{titanic} & 35 & 3059 & 77 & 1852 & - & - & 342 & $\mathsmaller{\geq}1$h & - & - & 78 & 0.01\\
\texttt{vehicle} & 0 & 0.00$^*$ & 0 & 0.08$^*$ & 0 & 0.37$^*$ & 0 & 4.2$^*$ & - & - & 0 & 0.01\\
\texttt{wine1} & 22 & 545 & 22 & 319 & - & - & 27 & $\mathsmaller{\geq}1$h & - & - & 25 & 0.01\\
\texttt{wine2} & 24 & 399 & 24 & 2440 & - & - & 29 & $\mathsmaller{\geq}1$h & - & - & 29 & 0.02\\
\texttt{wine3} & 16 & 272 & 15 & 151 & - & - & 19 & $\mathsmaller{\geq}1$h & - & - & 19 & 0.01\\
\bottomrule
\end{tabular}
%
\end{scriptsize}%
\end{center}%
\caption{\label{tab:all102} Comparison with state of the art: $\numex<5000, \numfeat \geq 250$, depth 10}%
\end{table}%

\begin{table}[htbp]%
\begin{center}%
\begin{scriptsize}%
\tabcolsep=2pt%
\begin{tabular}{lrrrrrrrrrrrr}
\toprule
\multirow{2}{*}{}&  \multicolumn{2}{c}{\budalg} & \multicolumn{2}{c}{\murtree} & \multicolumn{2}{c}{\dleight} & \multicolumn{2}{c}{\cp} & \multicolumn{2}{c}{binoct} & \multicolumn{2}{c}{\cart}\\
\cmidrule(rr){2-3}\cmidrule(rr){4-5}\cmidrule(rr){6-7}\cmidrule(rr){8-9}\cmidrule(rr){10-11}\cmidrule(rr){12-13}
& \multicolumn{1}{c}{error} & \multicolumn{1}{c}{cpu} & \multicolumn{1}{c}{error} & \multicolumn{1}{c}{cpu} & \multicolumn{1}{c}{error} & \multicolumn{1}{c}{cpu} & \multicolumn{1}{c}{error} & \multicolumn{1}{c}{cpu} & \multicolumn{1}{c}{error} & \multicolumn{1}{c}{cpu} & \multicolumn{1}{c}{error} & \multicolumn{1}{c}{cpu} \\
\midrule

\texttt{adult\_discretized} & 3841 & 2632 & 4052 & 1695 & 6200 & $\mathsmaller{\geq}1$h & 7511 & $\mathsmaller{\geq}1$h & - & - & 4148 & 0.12\\
\texttt{car} & 0 & 0.26$^*$ & 0 & 0.48$^*$ & 0 & 0.03$^*$ & 0 & 3.3$^*$ & 518 & 0.00 & 11 & 0.00\\
\texttt{car\_evaluation} & 80 & 0.00$^*$ & 80 & 112$^*$ & 80 & 9.2$^*$ & 80 & $\mathsmaller{\geq}1$h & - & - & 80 & 0.00\\
\texttt{chess} & 0 & 0.00$^*$ & 0 & 0.00$^*$ & 0 & 0.01$^*$ & 0 & 0.66$^*$ & - & - & 0 & 0.00\\
\texttt{compas\_discretized} & 1828 & 0.73$^*$ & 1842 & 2465 & - & - & 2809 & $\mathsmaller{\geq}1$h & 2809 & 0.00 & 1871 & 0.01\\
\texttt{HTRU\_2} & 219 & 550 & 299 & 1817 & 669 & $\mathsmaller{\geq}1$h & 1639 & $\mathsmaller{\geq}1$h & - & - & 293 & 0.08\\
\texttt{hypothyroid} & 17 & 0.96$^*$ & 17 & 517 & - & - & 277 & $\mathsmaller{\geq}1$h & - & - & 31 & 0.01\\
\texttt{kr-vs-kp} & 0 & 1897$^*$ & 24 & 711 & - & - & 784 & $\mathsmaller{\geq}1$h & - & - & 12 & 0.01\\
\texttt{magic04} & 1635 & 2746 & 2429 & 2759 & 3839 & $\mathsmaller{\geq}1$h & 6688 & $\mathsmaller{\geq}1$h & - & - & 2145 & 0.13\\
\texttt{seismic\_bumps} & 38 & 2591 & 88 & 3021 & - & - & 170 & $\mathsmaller{\geq}1$h & - & - & 101 & 0.01\\
\texttt{winequality-red} & 2 & 0.00$^*$ & 2 & 0.10 & - & - & 10 & $\mathsmaller{\geq}1$h & - & - & 2 & 0.00\\
\bottomrule
\end{tabular}
%
\end{scriptsize}%
\end{center}%
\caption{\label{tab:all103} Comparison with state of the art: $\numex \geq 5000, \numfeat < 250$, depth 10}%
\end{table}%

\begin{table}[htbp]%
\begin{center}%
\begin{scriptsize}%
\tabcolsep=2pt%
\begin{tabular}{lrrrrrrrrrrrr}
\toprule
\multirow{2}{*}{}&  \multicolumn{2}{c}{\budalg} & \multicolumn{2}{c}{\murtree} & \multicolumn{2}{c}{\dleight} & \multicolumn{2}{c}{\cp} & \multicolumn{2}{c}{binoct} & \multicolumn{2}{c}{\cart}\\
\cmidrule(rr){2-3}\cmidrule(rr){4-5}\cmidrule(rr){6-7}\cmidrule(rr){8-9}\cmidrule(rr){10-11}\cmidrule(rr){12-13}
& \multicolumn{1}{c}{error} & \multicolumn{1}{c}{cpu} & \multicolumn{1}{c}{error} & \multicolumn{1}{c}{cpu} & \multicolumn{1}{c}{error} & \multicolumn{1}{c}{cpu} & \multicolumn{1}{c}{error} & \multicolumn{1}{c}{cpu} & \multicolumn{1}{c}{error} & \multicolumn{1}{c}{cpu} & \multicolumn{1}{c}{error} & \multicolumn{1}{c}{cpu} \\
\midrule

\texttt{bank} & 3242 & 800 & 3767 & 3270 & 4826 & $\mathsmaller{\geq}1$h & 5289 & $\mathsmaller{\geq}1$h & - & - & 3327 & 102\\
\texttt{bank\_conv} & 169 & 2794 & 223 & 2607 & - & - & 521 & $\mathsmaller{\geq}1$h & - & - & 207 & 0.10\\
\texttt{default\_credit} & 4547 & 2019 & 5046 & 240 & 5412 & $\mathsmaller{\geq}1$h & 6636 & $\mathsmaller{\geq}1$h & - & - & 4762 & 1.3\\
\texttt{hand\_posture} & 334 & 39 & 450 & 426 & 15187 & $\mathsmaller{\geq}1$h & 16265 & $\mathsmaller{\geq}1$h & - & - & 530 & 88\\
\texttt{letter} & 0 & 79$^*$ & 0 & 278$^*$ & 725 & $\mathsmaller{\geq}1$h & 813 & $\mathsmaller{\geq}1$h & - & - & 21 & 0.31\\
\texttt{mnist\_0} & 383 & 413 & 880 & 101 & 3314 & $\mathsmaller{\geq}1$h & 5923 & $\mathsmaller{\geq}1$h & - & - & 477 & 8.5\\
\texttt{mushroom} & 0 & 0.00$^*$ & 0 & 0.02$^*$ & 0 & 1.1$^*$ & 0 & 1.2$^*$ & - & - & 0 & 0.04\\
\texttt{pendigits} & 0 & 0.00$^*$ & 0 & 0.11$^*$ & 0 & 1247$^*$ & 0 & 5.3$^*$ & - & - & 0 & 0.07\\
\texttt{segment} & 0 & 0.00$^*$ & 0 & 0.01$^*$ & 0 & 0.08$^*$ & 0 & 1.9$^*$ & - & - & 0 & 0.01\\
\texttt{spambase} & 262 & 546 & 381 & 440 & - & - & 1813 & $\mathsmaller{\geq}1$h & - & - & 332 & 0.09\\
\texttt{splice-1} & 5 & 1160 & 13 & 993 & - & - & 1535 & $\mathsmaller{\geq}1$h & - & - & 12 & 0.05\\
\texttt{Statlog\_satellite} & 3 & 219 & 11 & 3349 & - & - & 1072 & $\mathsmaller{\geq}1$h & - & - & 15 & 0.13\\
\texttt{Statlog\_shuttle} & 0 & 0.02$^*$ & 0 & 24$^*$ & 0 & 99$^*$ & 0 & 16$^*$ & - & - & 0 & 3.6\\
\texttt{surgical-deepnet} & 965 & 2865 & 1382 & 885 & - & - & 3690 & $\mathsmaller{\geq}1$h & - & - & 1089 & 14\\
\texttt{taiwan\_binarised} & 4217 & 1001 & 4993 & 1437 & - & - & 6636 & $\mathsmaller{\geq}1$h & - & - & 4710 & 0.54\\
\texttt{weather-aus} & 1601 & 2591 & 1675 & 2076 & - & - & 1761 & $\mathsmaller{\geq}1$h & - & - & 1642 & 32\\
\bottomrule
\end{tabular}
%
\end{scriptsize}%
\end{center}%
\caption{\label{tab:all104} Comparison with state of the art: $\numex \geq 5000, \numfeat \geq 250$, depth 10}%
\end{table}%




\begin{table}[htbp]%
\begin{center}%
\begin{scriptsize}%
\tabcolsep=2pt%
\begin{tabular}{lrrrrrrrr}
\toprule
\multirow{2}{*}{}&  \multicolumn{2}{c}{\budalg} & \multicolumn{2}{c}{\noheuristic} & \multicolumn{2}{c}{\nopreprocessing} & \multicolumn{2}{c}{\nolb}\\
\cmidrule(rr){2-3}\cmidrule(rr){4-5}\cmidrule(rr){6-7}\cmidrule(rr){8-9}
& \multicolumn{1}{c}{error} & \multicolumn{1}{c}{cpu} & \multicolumn{1}{c}{error} & \multicolumn{1}{c}{cpu} & \multicolumn{1}{c}{error} & \multicolumn{1}{c}{cpu} & \multicolumn{1}{c}{error} & \multicolumn{1}{c}{cpu} \\
\midrule

\texttt{anneal} & 112 & 0.03$^*$ & 112 & 0.02$^*$ & 112 & 0.17$^*$ & 112 & 0.03$^*$\\
\texttt{balance-scale} & 49 & 0.00$^*$ & 49 & 0.00$^*$ & 49 & 0.00$^*$ & 49 & 0.00$^*$\\
\texttt{banknote} & 36 & 0.01$^*$ & 36 & 0.00$^*$ & 36 & 0.01$^*$ & 36 & 0.01$^*$\\
\texttt{breast-cancer} & 24 & 0.16$^*$ & 24 & 0.08$^*$ & 24 & 0.10$^*$ & 24 & 0.10$^*$\\
\texttt{heart-cleveland} & 41 & 0.05$^*$ & 41 & 0.03$^*$ & 41 & 0.20$^*$ & 41 & 0.05$^*$\\
\texttt{hepatitis} & 10 & 0.00$^*$ & 10 & 0.00$^*$ & 10 & 0.07$^*$ & 10 & 0.01$^*$\\
\texttt{IndiansDiabetes} & 166 & 0.02$^*$ & 166 & 0.02$^*$ & 166 & 0.03$^*$ & 166 & 0.03$^*$\\
\texttt{iris} & 1 & 0.00$^*$ & 1 & 0.00$^*$ & 1 & 0.00$^*$ & 1 & 0.00$^*$\\
\texttt{lymph} & 12 & 0.01$^*$ & 12 & 0.01$^*$ & 12 & 0.06$^*$ & 12 & 0.02$^*$\\
\texttt{messidor} & 366 & 0.25$^*$ & 366 & 0.23$^*$ & 366 & 0.26$^*$ & 366 & 0.28$^*$\\
\texttt{monk1} & 11 & 0.00$^*$ & 11 & 0.00$^*$ & 11 & 0.00$^*$ & 11 & 0.00$^*$\\
\texttt{monk2} & 42 & 0.00$^*$ & 42 & 0.00$^*$ & 42 & 0.00$^*$ & 42 & 0.00$^*$\\
\texttt{monk3} & 6 & 0.00$^*$ & 6 & 0.00$^*$ & 6 & 0.00$^*$ & 6 & 0.00$^*$\\
\texttt{primary-tumor} & 46 & 0.00$^*$ & 46 & 0.00$^*$ & 46 & 0.01$^*$ & 46 & 0.00$^*$\\
\texttt{soybean} & 29 & 0.01$^*$ & 29 & 0.01$^*$ & 29 & 0.03$^*$ & 29 & 0.02$^*$\\
\texttt{tic-tac-toe} & 216 & 0.01$^*$ & 216 & 0.01$^*$ & 216 & 0.01$^*$ & 216 & 0.01$^*$\\
\texttt{vote} & 12 & 0.02$^*$ & 12 & 0.02$^*$ & 12 & 0.03$^*$ & 12 & 0.03$^*$\\
\texttt{yeast} & 403 & 0.07$^*$ & 403 & 0.07$^*$ & 403 & 0.36$^*$ & 403 & 0.07$^*$\\
\bottomrule
\end{tabular}
%
\end{scriptsize}%
\end{center}%
\caption{\label{tab:all31} Comparison with state of the art: $\numex<5000, \numfeat<250$, depth 3}%
\end{table}%

\begin{table}[htbp]%
\begin{center}%
\begin{scriptsize}%
\tabcolsep=2pt%
\begin{tabular}{lrrrrrrrr}
\toprule
\multirow{2}{*}{}&  \multicolumn{2}{c}{\budalg} & \multicolumn{2}{c}{\noheuristic} & \multicolumn{2}{c}{\nopreprocessing} & \multicolumn{2}{c}{\nolb}\\
\cmidrule(rr){2-3}\cmidrule(rr){4-5}\cmidrule(rr){6-7}\cmidrule(rr){8-9}
& \multicolumn{1}{c}{error} & \multicolumn{1}{c}{cpu} & \multicolumn{1}{c}{error} & \multicolumn{1}{c}{cpu} & \multicolumn{1}{c}{error} & \multicolumn{1}{c}{cpu} & \multicolumn{1}{c}{error} & \multicolumn{1}{c}{cpu} \\
\midrule

\texttt{audiology} & 5 & 0.06$^*$ & 5 & 0.04$^*$ & 5 & 0.33$^*$ & 5 & 0.06$^*$\\
\texttt{australian-credit} & 73 & 0.14$^*$ & 73 & 0.11$^*$ & 73 & 0.55$^*$ & 73 & 0.13$^*$\\
\texttt{biodeg} & 164 & 5.4$^*$ & 164 & 4.6$^*$ & 164 & 5.7$^*$ & 164 & 6.1$^*$\\
\texttt{breast-wisconsin} & 15 & 0.05$^*$ & 15 & 0.04$^*$ & 15 & 0.28$^*$ & 15 & 0.06$^*$\\
\texttt{diabetes} & 162 & 0.09$^*$ & 162 & 0.08$^*$ & 162 & 0.50$^*$ & 162 & 0.09$^*$\\
\texttt{forest-fires} & 193 & 20$^*$ & 193 & 16$^*$ & 193 & 65$^*$ & 193 & 20$^*$\\
\texttt{german-credit} & 236 & 0.26$^*$ & 236 & 0.20$^*$ & 236 & 0.54$^*$ & 236 & 0.26$^*$\\
\texttt{ionosphere} & 22 & 3.8$^*$ & 22 & 3.0$^*$ & 22 & 22$^*$ & 22 & 4.2$^*$\\
\texttt{titanic} & 143 & 6.7$^*$ & 143 & 5.5$^*$ & 143 & 6.6$^*$ & 143 & 6.7$^*$\\
\texttt{vehicle} & 26 & 0.93$^*$ & 26 & 0.59$^*$ & 26 & 3.5$^*$ & 26 & 0.83$^*$\\
\texttt{wine1} & 43 & 16$^*$ & 43 & 14$^*$ & 43 & 120$^*$ & 43 & 17$^*$\\
\texttt{wine2} & 49 & 17$^*$ & 49 & 14$^*$ & 49 & 118$^*$ & 49 & 17$^*$\\
\texttt{wine3} & 33 & 16$^*$ & 33 & 13$^*$ & 33 & 118$^*$ & 33 & 16$^*$\\
\bottomrule
\end{tabular}
%
\end{scriptsize}%
\end{center}%
\caption{\label{tab:all32} Comparison with state of the art: $\numex<5000, \numfeat \geq 250$, depth 3}%
\end{table}%

\begin{table}[htbp]%
\begin{center}%
\begin{scriptsize}%
\tabcolsep=2pt%
\begin{tabular}{lrrrrrrrr}
\toprule
\multirow{2}{*}{}&  \multicolumn{2}{c}{\budalg} & \multicolumn{2}{c}{\noheuristic} & \multicolumn{2}{c}{\nopreprocessing} & \multicolumn{2}{c}{\nolb}\\
\cmidrule(rr){2-3}\cmidrule(rr){4-5}\cmidrule(rr){6-7}\cmidrule(rr){8-9}
& \multicolumn{1}{c}{error} & \multicolumn{1}{c}{cpu} & \multicolumn{1}{c}{error} & \multicolumn{1}{c}{cpu} & \multicolumn{1}{c}{error} & \multicolumn{1}{c}{cpu} & \multicolumn{1}{c}{error} & \multicolumn{1}{c}{cpu} \\
\midrule

\texttt{adult\_discretized} & 5020 & 0.43$^*$ & 5020 & 0.25$^*$ & 5020 & 0.72$^*$ & 5020 & 0.27$^*$\\
\texttt{car} & 192 & 0.01$^*$ & 192 & 0.00$^*$ & 192 & 0.00$^*$ & 192 & 0.01$^*$\\
\texttt{car\_evaluation} & 202 & 0.00$^*$ & 202 & 0.00$^*$ & 202 & 0.00$^*$ & 202 & 0.01$^*$\\
\texttt{chess} & 0 & 0.00$^*$ & 0 & 0.00$^*$ & 0 & 0.00$^*$ & 0 & 0.00$^*$\\
\texttt{compas\_discretized} & 2004 & 0.00$^*$ & 2004 & 0.00$^*$ & 2004 & 0.03$^*$ & 2004 & 0.00$^*$\\
\texttt{HTRU\_2} & 401 & 1.2$^*$ & 401 & 1.2$^*$ & 401 & 2.0$^*$ & 401 & 1.4$^*$\\
\texttt{hypothyroid} & 61 & 0.07$^*$ & 61 & 0.07$^*$ & 61 & 0.33$^*$ & 61 & 0.08$^*$\\
\texttt{kr-vs-kp} & 198 & 0.09$^*$ & 198 & 0.06$^*$ & 198 & 0.22$^*$ & 198 & 0.07$^*$\\
\texttt{magic04} & 3446 & 3.8$^*$ & 3446 & 3.5$^*$ & 3446 & 3.3$^*$ & 3446 & 3.2$^*$\\
\texttt{seismic\_bumps} & 160 & 0.28$^*$ & 160 & 0.28$^*$ & 160 & 0.41$^*$ & 160 & 0.32$^*$\\
\texttt{winequality-red} & 8 & 0.02$^*$ & 8 & 0.02$^*$ & 8 & 0.03$^*$ & 8 & 0.03$^*$\\
\bottomrule
\end{tabular}
%
\end{scriptsize}%
\end{center}%
\caption{\label{tab:all33} Comparison with state of the art: $\numex \geq 5000, \numfeat < 250$, depth 3}%
\end{table}%

\begin{table}[htbp]%
\begin{center}%
\begin{scriptsize}%
\tabcolsep=2pt%
\begin{tabular}{lrrrrrrrr}
\toprule
\multirow{2}{*}{}&  \multicolumn{2}{c}{\budalg} & \multicolumn{2}{c}{\noheuristic} & \multicolumn{2}{c}{\nopreprocessing} & \multicolumn{2}{c}{\nolb}\\
\cmidrule(rr){2-3}\cmidrule(rr){4-5}\cmidrule(rr){6-7}\cmidrule(rr){8-9}
& \multicolumn{1}{c}{error} & \multicolumn{1}{c}{cpu} & \multicolumn{1}{c}{error} & \multicolumn{1}{c}{cpu} & \multicolumn{1}{c}{error} & \multicolumn{1}{c}{cpu} & \multicolumn{1}{c}{error} & \multicolumn{1}{c}{cpu} \\
\midrule

\texttt{bank} & 4453 & 259 & 4383 & 84 & 4453 & 226 & 4453 & 257\\
\texttt{bank\_conv} & 428 & 9.9$^*$ & 428 & 5.6$^*$ & 428 & 6.0$^*$ & 428 & 6.3$^*$\\
\texttt{default\_credit} & 5327 & 232$^*$ & 5327 & 289$^*$ & 5327 & 256$^*$ & 5327 & 220$^*$\\
\texttt{hand\_posture} & 7645 & 181 & 9132 & 189 & 7645 & 153 & 7645 & 172\\
\texttt{letter} & 369 & 10$^*$ & 369 & 8.4$^*$ & 369 & 45$^*$ & 369 & 8.2$^*$\\
\texttt{mnist\_0} & 2557 & 1994$^*$ & 2557 & 1832$^*$ & 2557 & 1792$^*$ & 2557 & 1867$^*$\\
\texttt{mushroom} & 8 & 0.79$^*$ & 8 & 0.60$^*$ & 8 & 0.76$^*$ & 8 & 0.68$^*$\\
\texttt{pendigits} & 47 & 3.3$^*$ & 47 & 3.1$^*$ & 47 & 13$^*$ & 47 & 3.6$^*$\\
\texttt{segment} & 0 & 0.03$^*$ & 0 & 0.03$^*$ & 0 & 0.20$^*$ & 0 & 0.03$^*$\\
\texttt{spambase} & 694 & 11$^*$ & 694 & 9.6$^*$ & 694 & 12$^*$ & 694 & 11$^*$\\
\texttt{splice-1} & 224 & 9.8$^*$ & 224 & 8.2$^*$ & 224 & 11$^*$ & 224 & 9.8$^*$\\
\texttt{Statlog\_satellite} & 187 & 79$^*$ & 187 & 63$^*$ & 187 & 67$^*$ & 187 & 85$^*$\\
\texttt{Statlog\_shuttle} & 0 & 6.7$^*$ & 0 & 84$^*$ & 0 & 13$^*$ & 0 & 6.2$^*$\\
\texttt{surgical-deepnet} & 2512 & 953 & 2524 & 1304 & 2512 & 907 & 2512 & 918\\
\texttt{taiwan\_binarised} & 5326 & 48$^*$ & 5326 & 28$^*$ & 5326 & 45$^*$ & 5326 & 33$^*$\\
\texttt{weather-aus} & 1756 & 14 & 1756 & 1.3 & 1756 & 12 & 1756 & 13\\
\bottomrule
\end{tabular}
%
\end{scriptsize}%
\end{center}%
\caption{\label{tab:all34} Comparison with state of the art: $\numex \geq 5000, \numfeat \geq 250$, depth 3}%
\end{table}%


\begin{table}[htbp]%
\begin{center}%
\begin{scriptsize}%
\tabcolsep=2pt%
\begin{tabular}{lrrrrrrrr}
\toprule
\multirow{2}{*}{}&  \multicolumn{2}{c}{\budalg} & \multicolumn{2}{c}{\noheuristic} & \multicolumn{2}{c}{\nopreprocessing} & \multicolumn{2}{c}{\nolb}\\
\cmidrule(rr){2-3}\cmidrule(rr){4-5}\cmidrule(rr){6-7}\cmidrule(rr){8-9}
& \multicolumn{1}{c}{error} & \multicolumn{1}{c}{cpu} & \multicolumn{1}{c}{error} & \multicolumn{1}{c}{cpu} & \multicolumn{1}{c}{error} & \multicolumn{1}{c}{cpu} & \multicolumn{1}{c}{error} & \multicolumn{1}{c}{cpu} \\
\midrule

\texttt{anneal} & 91 & 1.5$^*$ & 91 & 1.0$^*$ & 91 & 11$^*$ & 91 & 1.3$^*$\\
\texttt{balance-scale} & 48 & 0.04$^*$ & 48 & 0.04$^*$ & 48 & 0.04$^*$ & 48 & 0.04$^*$\\
\texttt{banknote} & 13 & 0.08$^*$ & 13 & 0.07$^*$ & 13 & 0.19$^*$ & 13 & 0.10$^*$\\
\texttt{breast-cancer} & 16 & 9.6$^*$ & 16 & 7.6$^*$ & 16 & 9.1$^*$ & 16 & 8.9$^*$\\
\texttt{heart-cleveland} & 25 & 3.1$^*$ & 25 & 2.3$^*$ & 25 & 19$^*$ & 25 & 3.3$^*$\\
\texttt{hepatitis} & 3 & 0.32$^*$ & 3 & 0.20$^*$ & 3 & 3.0$^*$ & 3 & 0.31$^*$\\
\texttt{IndiansDiabetes} & 149 & 0.90$^*$ & 149 & 0.89$^*$ & 149 & 0.97$^*$ & 149 & 0.98$^*$\\
\texttt{iris} & 1 & 0.00$^*$ & 1 & 0.00$^*$ & 1 & 0.00$^*$ & 1 & 0.00$^*$\\
\texttt{lymph} & 3 & 0.74$^*$ & 3 & 0.57$^*$ & 3 & 2.4$^*$ & 3 & 0.91$^*$\\
\texttt{messidor} & 332 & 21$^*$ & 332 & 19$^*$ & 332 & 21$^*$ & 332 & 23$^*$\\
\texttt{monk1} & 2 & 0.00$^*$ & 2 & 0.00$^*$ & 2 & 0.00$^*$ & 2 & 0.00$^*$\\
\texttt{monk2} & 31 & 0.01$^*$ & 31 & 0.01$^*$ & 31 & 0.01$^*$ & 31 & 0.01$^*$\\
\texttt{monk3} & 4 & 0.00$^*$ & 4 & 0.00$^*$ & 4 & 0.00$^*$ & 4 & 0.00$^*$\\
\texttt{primary-tumor} & 34 & 0.03$^*$ & 34 & 0.02$^*$ & 34 & 0.22$^*$ & 34 & 0.03$^*$\\
\texttt{soybean} & 14 & 0.62$^*$ & 14 & 0.50$^*$ & 14 & 1.1$^*$ & 14 & 0.71$^*$\\
\texttt{tic-tac-toe} & 137 & 0.38$^*$ & 137 & 0.34$^*$ & 137 & 0.38$^*$ & 137 & 0.38$^*$\\
\texttt{vote} & 5 & 1.2$^*$ & 5 & 0.91$^*$ & 5 & 1.2$^*$ & 5 & 1.4$^*$\\
\texttt{yeast} & 366 & 3.4$^*$ & 366 & 3.0$^*$ & 366 & 29$^*$ & 366 & 3.4$^*$\\
\bottomrule
\end{tabular}
%
\end{scriptsize}%
\end{center}%
\caption{\label{tab:all41} Comparison with state of the art: $\numex<5000, \numfeat<250$, depth 4}%
\end{table}%

\begin{table}[htbp]%
\begin{center}%
\begin{scriptsize}%
\tabcolsep=2pt%
\begin{tabular}{lrrrrrrrr}
\toprule
\multirow{2}{*}{}&  \multicolumn{2}{c}{\budalg} & \multicolumn{2}{c}{\noheuristic} & \multicolumn{2}{c}{\nopreprocessing} & \multicolumn{2}{c}{\nolb}\\
\cmidrule(rr){2-3}\cmidrule(rr){4-5}\cmidrule(rr){6-7}\cmidrule(rr){8-9}
& \multicolumn{1}{c}{error} & \multicolumn{1}{c}{cpu} & \multicolumn{1}{c}{error} & \multicolumn{1}{c}{cpu} & \multicolumn{1}{c}{error} & \multicolumn{1}{c}{cpu} & \multicolumn{1}{c}{error} & \multicolumn{1}{c}{cpu} \\
\midrule

\texttt{audiology} & 1 & 4.0$^*$ & 1 & 3.2$^*$ & 1 & 29$^*$ & 1 & 4.5$^*$\\
\texttt{australian-credit} & 56 & 10$^*$ & 56 & 8.5$^*$ & 56 & 68$^*$ & 56 & 11$^*$\\
\texttt{biodeg} & 128 & 1511$^*$ & 128 & 1153$^*$ & 128 & 1447$^*$ & 128 & 1620$^*$\\
\texttt{breast-wisconsin} & 7 & 3.1$^*$ & 7 & 2.1$^*$ & 7 & 33$^*$ & 7 & 3.4$^*$\\
\texttt{diabetes} & 137 & 5.7$^*$ & 137 & 4.8$^*$ & 137 & 59$^*$ & 137 & 6.0$^*$\\
\texttt{forest-fires} & 173 & 15 & 173 & 11 & 173 & 48 & 173 & 15\\
\texttt{german-credit} & 204 & 28$^*$ & 204 & 22$^*$ & 204 & 66$^*$ & 204 & 29$^*$\\
\texttt{ionosphere} & 7 & 730$^*$ & 7 & 548$^*$ & 8 & 55 & 7 & 1026$^*$\\
\texttt{titanic} & 119 & 1604$^*$ & 119 & 1318$^*$ & 119 & 1620$^*$ & 119 & 1722$^*$\\
\texttt{vehicle} & 12 & 71$^*$ & 12 & 60$^*$ & 12 & 706$^*$ & 12 & 91$^*$\\
\texttt{wine1} & 37 & 1674 & 37 & 1808 & 38 & 2248 & 37 & 1617\\
\texttt{wine2} & 43 & 17 & 43 & 0.02 & 43 & 110 & 43 & 16\\
\texttt{wine3} & 28 & 33 & 28 & 190 & 28 & 222 & 28 & 33\\
\bottomrule
\end{tabular}
%
\end{scriptsize}%
\end{center}%
\caption{\label{tab:all42} Comparison with state of the art: $\numex<5000, \numfeat \geq 250$, depth 4}%
\end{table}%

\begin{table}[htbp]%
\begin{center}%
\begin{scriptsize}%
\tabcolsep=2pt%
\begin{tabular}{lrrrrrrrr}
\toprule
\multirow{2}{*}{}&  \multicolumn{2}{c}{\budalg} & \multicolumn{2}{c}{\noheuristic} & \multicolumn{2}{c}{\nopreprocessing} & \multicolumn{2}{c}{\nolb}\\
\cmidrule(rr){2-3}\cmidrule(rr){4-5}\cmidrule(rr){6-7}\cmidrule(rr){8-9}
& \multicolumn{1}{c}{error} & \multicolumn{1}{c}{cpu} & \multicolumn{1}{c}{error} & \multicolumn{1}{c}{cpu} & \multicolumn{1}{c}{error} & \multicolumn{1}{c}{cpu} & \multicolumn{1}{c}{error} & \multicolumn{1}{c}{cpu} \\
\midrule

\texttt{adult\_discretized} & 4609 & 14$^*$ & 4609 & 14$^*$ & 4609 & 43$^*$ & 4609 & 14$^*$\\
\texttt{car} & 136 & 0.19$^*$ & 136 & 0.16$^*$ & 136 & 0.14$^*$ & 136 & 0.16$^*$\\
\texttt{car\_evaluation} & 130 & 0.02$^*$ & 130 & 0.02$^*$ & 130 & 0.04$^*$ & 130 & 0.03$^*$\\
\texttt{chess} & 0 & 0.00$^*$ & 0 & 0.00$^*$ & 0 & 0.00$^*$ & 0 & 0.00$^*$\\
\texttt{compas\_discretized} & 1954 & 0.07$^*$ & 1954 & 0.05$^*$ & 1954 & 0.69$^*$ & 1954 & 0.07$^*$\\
\texttt{HTRU\_2} & 385 & 74$^*$ & 385 & 77$^*$ & 385 & 106$^*$ & 385 & 75$^*$\\
\texttt{hypothyroid} & 53 & 2.9$^*$ & 53 & 2.5$^*$ & 53 & 23$^*$ & 53 & 3.1$^*$\\
\texttt{kr-vs-kp} & 144 & 2.8$^*$ & 144 & 2.4$^*$ & 144 & 14$^*$ & 144 & 2.5$^*$\\
\texttt{magic04} & 3112 & 232$^*$ & 3112 & 259$^*$ & 3112 & 290$^*$ & 3112 & 265$^*$\\
\texttt{seismic\_bumps} & 148 & 22$^*$ & 148 & 20$^*$ & 148 & 31$^*$ & 148 & 25$^*$\\
\texttt{winequality-red} & 4 & 0.62$^*$ & 4 & 0.62$^*$ & 4 & 0.96$^*$ & 4 & 0.85$^*$\\
\bottomrule
\end{tabular}
%
\end{scriptsize}%
\end{center}%
\caption{\label{tab:all43} Comparison with state of the art: $\numex \geq 5000, \numfeat < 250$, depth 4}%
\end{table}%

\begin{table}[htbp]%
\begin{center}%
\begin{scriptsize}%
\tabcolsep=2pt%
\begin{tabular}{lrrrrrrrr}
\toprule
\multirow{2}{*}{}&  \multicolumn{2}{c}{\budalg} & \multicolumn{2}{c}{\noheuristic} & \multicolumn{2}{c}{\nopreprocessing} & \multicolumn{2}{c}{\nolb}\\
\cmidrule(rr){2-3}\cmidrule(rr){4-5}\cmidrule(rr){6-7}\cmidrule(rr){8-9}
& \multicolumn{1}{c}{error} & \multicolumn{1}{c}{cpu} & \multicolumn{1}{c}{error} & \multicolumn{1}{c}{cpu} & \multicolumn{1}{c}{error} & \multicolumn{1}{c}{cpu} & \multicolumn{1}{c}{error} & \multicolumn{1}{c}{cpu} \\
\midrule

\texttt{bank} & 4314 & 290 & 4326 & 1102 & 4314 & 258 & 4314 & 308\\
\texttt{bank\_conv} & 392 & 1963$^*$ & 392 & 1150$^*$ & 392 & 1288$^*$ & 392 & 1291$^*$\\
\texttt{default\_credit} & 5270 & 209 & 5270 & 183 & 5270 & 167 & 5270 & 199\\
\texttt{hand\_posture} & 4896 & 976 & 10096 & 28 & 4896 & 872 & 4896 & 969\\
\texttt{letter} & 261 & 1185$^*$ & 261 & 813$^*$ & 261 & 292 & 261 & 1407$^*$\\
\texttt{mnist\_0} & 2173 & 2158 & 2229 & 3292 & 2173 & 1844 & 2173 & 2444\\
\texttt{mushroom} & 0 & 0.00$^*$ & 0 & 0.00$^*$ & 0 & 0.00$^*$ & 0 & 0.00$^*$\\
\texttt{pendigits} & 13 & 230$^*$ & 13 & 237$^*$ & 13 & 1871$^*$ & 13 & 341$^*$\\
\texttt{segment} & 0 & 0.00$^*$ & 0 & 0.00$^*$ & 0 & 0.00$^*$ & 0 & 0.00$^*$\\
\texttt{spambase} & 590 & 7.7 & 590 & 3504$^*$ & 590 & 7.5 & 590 & 7.5\\
\texttt{splice-1} & 141 & 3241$^*$ & 141 & 2519$^*$ & 141 & 0.00 & 141 & 3563$^*$\\
\texttt{Statlog\_satellite} & 111 & 3571 & 120 & 1141 & 111 & 3479 & 114 & 3294\\
\texttt{Statlog\_shuttle} & 0 & 0.64$^*$ & 0 & 0.79$^*$ & 0 & 1.6$^*$ & 0 & 0.83$^*$\\
\texttt{surgical-deepnet} & 2269 & 49 & 2414 & 1479 & 2269 & 46 & 2269 & 51\\
\texttt{taiwan\_binarised} & 5273 & 6.2 & 5273 & 39 & 5273 & 6.2 & 5273 & 7.1\\
\texttt{weather-aus} & 1749 & 2525 & 1750 & 2646 & 1749 & 2142 & 1749 & 2638\\
\bottomrule
\end{tabular}
%
\end{scriptsize}%
\end{center}%
\caption{\label{tab:all44} Comparison with state of the art: $\numex \geq 5000, \numfeat \geq 250$, depth 4}%
\end{table}%

\begin{table}[htbp]%
\begin{center}%
\begin{scriptsize}%
\tabcolsep=2pt%
\begin{tabular}{lrrrrrrrr}
\toprule
\multirow{2}{*}{}&  \multicolumn{2}{c}{\budalg} & \multicolumn{2}{c}{\noheuristic} & \multicolumn{2}{c}{\nopreprocessing} & \multicolumn{2}{c}{\nolb}\\
\cmidrule(rr){2-3}\cmidrule(rr){4-5}\cmidrule(rr){6-7}\cmidrule(rr){8-9}
& \multicolumn{1}{c}{error} & \multicolumn{1}{c}{cpu} & \multicolumn{1}{c}{error} & \multicolumn{1}{c}{cpu} & \multicolumn{1}{c}{error} & \multicolumn{1}{c}{cpu} & \multicolumn{1}{c}{error} & \multicolumn{1}{c}{cpu} \\
\midrule

\texttt{anneal} & 70 & 44$^*$ & 70 & 38$^*$ & 70 & 736$^*$ & 70 & 50$^*$\\
\texttt{balance-scale} & 45 & 0.46$^*$ & 45 & 0.47$^*$ & 45 & 0.50$^*$ & 45 & 0.51$^*$\\
\texttt{banknote} & 3 & 0.88$^*$ & 3 & 0.87$^*$ & 3 & 2.4$^*$ & 3 & 1.6$^*$\\
\texttt{breast-cancer} & 6 & 725$^*$ & 6 & 604$^*$ & 6 & 764$^*$ & 6 & 764$^*$\\
\texttt{heart-cleveland} & 7 & 93$^*$ & 7 & 78$^*$ & 7 & 1224$^*$ & 7 & 156$^*$\\
\texttt{hepatitis} & 0 & 0.05$^*$ & 0 & 0.09$^*$ & 0 & 0.44$^*$ & 0 & 0.05$^*$\\
\texttt{IndiansDiabetes} & 125 & 30$^*$ & 125 & 28$^*$ & 125 & 31$^*$ & 125 & 37$^*$\\
\texttt{iris} & 1 & 0.00$^*$ & 1 & 0.00$^*$ & 1 & 0.00$^*$ & 1 & 0.00$^*$\\
\texttt{lymph} & 0 & 0.00$^*$ & 0 & 0.02$^*$ & 0 & 0.00$^*$ & 0 & 0.00$^*$\\
\texttt{messidor} & 281 & 1522$^*$ & 281 & 1373$^*$ & 281 & 1527$^*$ & 281 & 1719$^*$\\
\texttt{monk1} & 0 & 0.00$^*$ & 0 & 0.00$^*$ & 0 & 0.00$^*$ & 0 & 0.00$^*$\\
\texttt{monk2} & 15 & 0.05$^*$ & 15 & 0.05$^*$ & 15 & 0.05$^*$ & 15 & 0.07$^*$\\
\texttt{monk3} & 2 & 0.03$^*$ & 2 & 0.02$^*$ & 2 & 0.02$^*$ & 2 & 0.03$^*$\\
\texttt{primary-tumor} & 26 & 0.38$^*$ & 26 & 0.43$^*$ & 26 & 6.7$^*$ & 26 & 0.46$^*$\\
\texttt{soybean} & 8 & 20$^*$ & 8 & 16$^*$ & 8 & 40$^*$ & 8 & 26$^*$\\
\texttt{tic-tac-toe} & 63 & 10$^*$ & 63 & 8.7$^*$ & 63 & 9.3$^*$ & 63 & 11$^*$\\
\texttt{vote} & 1 & 24$^*$ & 1 & 21$^*$ & 1 & 26$^*$ & 1 & 45$^*$\\
\texttt{yeast} & 313 & 139$^*$ & 313 & 123$^*$ & 313 & 2348$^*$ & 313 & 151$^*$\\
\bottomrule
\end{tabular}
%
\end{scriptsize}%
\end{center}%
\caption{\label{tab:all51} Comparison with state of the art: $\numex<5000, \numfeat<250$, depth 5}%
\end{table}%

\begin{table}[htbp]%
\begin{center}%
\begin{scriptsize}%
\tabcolsep=2pt%
\begin{tabular}{lrrrrrrrr}
\toprule
\multirow{2}{*}{}&  \multicolumn{2}{c}{\budalg} & \multicolumn{2}{c}{\noheuristic} & \multicolumn{2}{c}{\nopreprocessing} & \multicolumn{2}{c}{\nolb}\\
\cmidrule(rr){2-3}\cmidrule(rr){4-5}\cmidrule(rr){6-7}\cmidrule(rr){8-9}
& \multicolumn{1}{c}{error} & \multicolumn{1}{c}{cpu} & \multicolumn{1}{c}{error} & \multicolumn{1}{c}{cpu} & \multicolumn{1}{c}{error} & \multicolumn{1}{c}{cpu} & \multicolumn{1}{c}{error} & \multicolumn{1}{c}{cpu} \\
\midrule

\texttt{audiology} & 0 & 0.00$^*$ & 0 & 0.00$^*$ & 0 & 0.00$^*$ & 0 & 0.00$^*$\\
\texttt{australian-credit} & 39 & 658$^*$ & 39 & 513$^*$ & 40 & 40 & 39 & 839$^*$\\
\texttt{biodeg} & 88 & 268 & 88 & 680 & 88 & 271 & 88 & 323\\
\texttt{breast-wisconsin} & 0 & 20$^*$ & 0 & 16$^*$ & 0 & 478$^*$ & 0 & 31$^*$\\
\texttt{diabetes} & 106 & 312$^*$ & 106 & 245$^*$ & 106 & 1425 & 106 & 357$^*$\\
\texttt{forest-fires} & 156 & 777 & 157 & 61 & 156 & 2891 & 156 & 760\\
\texttt{german-credit} & 161 & 2741$^*$ & 161 & 2037$^*$ & 161 & 82 & 161 & 2885$^*$\\
\texttt{ionosphere} & 0 & 506$^*$ & 0 & 444$^*$ & 2 & 1746 & 0 & 806$^*$\\
\texttt{titanic} & 95 & 1428 & 95 & 1057 & 95 & 1464 & 95 & 1465\\
\texttt{vehicle} & 1 & 690 & 1 & 3525$^*$ & 3 & 42 & 1 & 1142\\
\texttt{wine1} & 33 & 1154 & 33 & 950 & 34 & 1319 & 33 & 1158\\
\texttt{wine2} & 39 & 411 & 37 & 13 & 39 & 2756 & 39 & 409\\
\texttt{wine3} & 25 & 17 & 25 & 90 & 25 & 100 & 25 & 16\\
\bottomrule
\end{tabular}
%
\end{scriptsize}%
\end{center}%
\caption{\label{tab:all52} Comparison with state of the art: $\numex<5000, \numfeat \geq 250$, depth 5}%
\end{table}%

\begin{table}[htbp]%
\begin{center}%
\begin{scriptsize}%
\tabcolsep=2pt%
\begin{tabular}{lrrrrrrrr}
\toprule
\multirow{2}{*}{}&  \multicolumn{2}{c}{\budalg} & \multicolumn{2}{c}{\noheuristic} & \multicolumn{2}{c}{\nopreprocessing} & \multicolumn{2}{c}{\nolb}\\
\cmidrule(rr){2-3}\cmidrule(rr){4-5}\cmidrule(rr){6-7}\cmidrule(rr){8-9}
& \multicolumn{1}{c}{error} & \multicolumn{1}{c}{cpu} & \multicolumn{1}{c}{error} & \multicolumn{1}{c}{cpu} & \multicolumn{1}{c}{error} & \multicolumn{1}{c}{cpu} & \multicolumn{1}{c}{error} & \multicolumn{1}{c}{cpu} \\
\midrule

\texttt{adult\_discretized} & 4423 & 725$^*$ & 4423 & 693$^*$ & 4423 & 2388$^*$ & 4423 & 755$^*$\\
\texttt{car} & 86 & 2.4$^*$ & 86 & 2.5$^*$ & 86 & 2.5$^*$ & 86 & 2.9$^*$\\
\texttt{car\_evaluation} & 90 & 0.13$^*$ & 90 & 0.13$^*$ & 90 & 0.40$^*$ & 90 & 0.23$^*$\\
\texttt{chess} & 0 & 0.00$^*$ & 0 & 0.00$^*$ & 0 & 0.00$^*$ & 0 & 0.00$^*$\\
\texttt{compas\_discretized} & 1919 & 1.1$^*$ & 1919 & 1.1$^*$ & 1919 & 14$^*$ & 1919 & 1.3$^*$\\
\texttt{HTRU\_2} & 361 & 98 & 361 & 3301$^*$ & 361 & 92 & 361 & 73\\
\texttt{hypothyroid} & 44 & 87$^*$ & 44 & 85$^*$ & 44 & 1539$^*$ & 44 & 103$^*$\\
\texttt{kr-vs-kp} & 81 & 65$^*$ & 81 & 65$^*$ & 81 & 823$^*$ & 81 & 81$^*$\\
\texttt{magic04} & 2882 & 756 & 2882 & 91 & 2867 & 3184 & 2867 & 3455\\
\texttt{seismic\_bumps} & 132 & 1533$^*$ & 132 & 1243$^*$ & 132 & 1914$^*$ & 132 & 1708$^*$\\
\texttt{winequality-red} & 3 & 16$^*$ & 3 & 15$^*$ & 3 & 28$^*$ & 3 & 24$^*$\\
\bottomrule
\end{tabular}
%
\end{scriptsize}%
\end{center}%
\caption{\label{tab:all53} Comparison with state of the art: $\numex \geq 5000, \numfeat < 250$, depth 5}%
\end{table}%

\begin{table}[htbp]%
\begin{center}%
\begin{scriptsize}%
\tabcolsep=2pt%
\begin{tabular}{lrrrrrrrr}
\toprule
\multirow{2}{*}{}&  \multicolumn{2}{c}{\budalg} & \multicolumn{2}{c}{\noheuristic} & \multicolumn{2}{c}{\nopreprocessing} & \multicolumn{2}{c}{\nolb}\\
\cmidrule(rr){2-3}\cmidrule(rr){4-5}\cmidrule(rr){6-7}\cmidrule(rr){8-9}
& \multicolumn{1}{c}{error} & \multicolumn{1}{c}{cpu} & \multicolumn{1}{c}{error} & \multicolumn{1}{c}{cpu} & \multicolumn{1}{c}{error} & \multicolumn{1}{c}{cpu} & \multicolumn{1}{c}{error} & \multicolumn{1}{c}{cpu} \\
\midrule

\texttt{bank} & 4187 & 1152 & 4309 & 1113 & 4187 & 1073 & 4187 & 1205\\
\texttt{bank\_conv} & 340 & 1662 & 346 & 2252 & 340 & 1059 & 340 & 984\\
\texttt{default\_credit} & 5181 & 3202 & 5183 & 3381 & 5181 & 1269 & 5181 & 1411\\
\texttt{hand\_posture} & 3154 & 56 & 9813 & 11 & 3154 & 56 & 3154 & 58\\
\texttt{letter} & 168 & 3082 & 172 & 2110 & 192 & 208 & 173 & 2313\\
\texttt{mnist\_0} & 1714 & 284 & 2075 & 1862 & 1714 & 241 & 1714 & 300\\
\texttt{mushroom} & 0 & 0.00$^*$ & 0 & 0.00$^*$ & 0 & 0.00$^*$ & 0 & 0.00$^*$\\
\texttt{pendigits} & 0 & 284$^*$ & 0 & 725$^*$ & 2 & 55 & 0 & 447$^*$\\
\texttt{segment} & 0 & 0.00$^*$ & 0 & 0.00$^*$ & 0 & 0.00$^*$ & 0 & 0.00$^*$\\
\texttt{spambase} & 501 & 219 & 501 & 935 & 501 & 274 & 501 & 234\\
\texttt{splice-1} & 101 & 24 & 101 & 1861 & 101 & 26 & 101 & 26\\
\texttt{Statlog\_satellite} & 71 & 279 & 99 & 1003 & 71 & 255 & 71 & 286\\
\texttt{Statlog\_shuttle} & 0 & 0.06$^*$ & 0 & 85$^*$ & 0 & 0.10$^*$ & 0 & 0.06$^*$\\
\texttt{surgical-deepnet} & 2131 & 2168 & 2310 & 2836 & 2131 & 1932 & 2131 & 2286\\
\texttt{taiwan\_binarised} & 5200 & 105 & 5201 & 3306 & 5200 & 83 & 5200 & 115\\
\texttt{weather-aus} & 1735 & 419 & 1749 & 1835 & 1735 & 350 & 1735 & 401\\
\bottomrule
\end{tabular}
%
\end{scriptsize}%
\end{center}%
\caption{\label{tab:all54} Comparison with state of the art: $\numex \geq 5000, \numfeat \geq 250$, depth 5}%
\end{table}%


\begin{table}[htbp]%
\begin{center}%
\begin{scriptsize}%
\tabcolsep=2pt%
\begin{tabular}{lrrrrrrrr}
\toprule
\multirow{2}{*}{}&  \multicolumn{2}{c}{\budalg} & \multicolumn{2}{c}{\noheuristic} & \multicolumn{2}{c}{\nopreprocessing} & \multicolumn{2}{c}{\nolb}\\
\cmidrule(rr){2-3}\cmidrule(rr){4-5}\cmidrule(rr){6-7}\cmidrule(rr){8-9}
& \multicolumn{1}{c}{error} & \multicolumn{1}{c}{cpu} & \multicolumn{1}{c}{error} & \multicolumn{1}{c}{cpu} & \multicolumn{1}{c}{error} & \multicolumn{1}{c}{cpu} & \multicolumn{1}{c}{error} & \multicolumn{1}{c}{cpu} \\
\midrule

\texttt{anneal} & 41 & 3036 & 49 & 2818 & 58 & 272 & 50 & 232\\
\texttt{balance-scale} & 29 & 37$^*$ & 29 & 39$^*$ & 29 & 37$^*$ & 29 & 40$^*$\\
\texttt{banknote} & 2 & 0.00$^*$ & 2 & 0.01$^*$ & 2 & 283$^*$ & 2 & 0.00$^*$\\
\texttt{breast-cancer} & 0 & 1007$^*$ & 0 & 824$^*$ & 0 & 1024$^*$ & 0 & 1194$^*$\\
\texttt{heart-cleveland} & 0 & 0.00$^*$ & 0 & 3.0$^*$ & 0 & 0.03$^*$ & 0 & 0.00$^*$\\
\texttt{hepatitis} & 0 & 0.00$^*$ & 0 & 0.00$^*$ & 0 & 0.00$^*$ & 0 & 0.00$^*$\\
\texttt{IndiansDiabetes} & 44 & 3343 & 45 & 3464 & 44 & 3448 & 47 & 579\\
\texttt{iris} & 1 & 0.00$^*$ & 1 & 0.00$^*$ & 1 & 0.01$^*$ & 1 & 0.00$^*$\\
\texttt{lymph} & 0 & 0.00$^*$ & 0 & 0.01$^*$ & 0 & 0.00$^*$ & 0 & 0.00$^*$\\
\texttt{messidor} & 179 & 2456 & 172 & 3162 & 179 & 3217 & 179 & 2901\\
\texttt{monk1} & 0 & 0.00$^*$ & 0 & 0.00$^*$ & 0 & 0.00$^*$ & 0 & 0.00$^*$\\
\texttt{monk2} & 0 & 0.00$^*$ & 0 & 0.00$^*$ & 0 & 0.00$^*$ & 0 & 0.00$^*$\\
\texttt{monk3} & 0 & 0.00$^*$ & 0 & 0.00$^*$ & 0 & 0.00$^*$ & 0 & 0.00$^*$\\
\texttt{primary-tumor} & 16 & 18$^*$ & 16 & 17$^*$ & 16 & 2866$^*$ & 16 & 39$^*$\\
\texttt{soybean} & 2 & 19$^*$ & 2 & 6.1$^*$ & 2 & 729 & 2 & 32$^*$\\
\texttt{tic-tac-toe} & 0 & 32$^*$ & 0 & 83$^*$ & 0 & 31$^*$ & 0 & 100$^*$\\
\texttt{vote} & 0 & 0.00$^*$ & 0 & 0.04$^*$ & 0 & 0.00$^*$ & 0 & 0.00$^*$\\
\texttt{yeast} & 182 & 3558 & 234 & 1611 & 210 & 1191 & 203 & 410\\
\bottomrule
\end{tabular}
%
\end{scriptsize}%
\end{center}%
\caption{\label{tab:all71} Comparison with state of the art: $\numex<5000, \numfeat<250$, depth 7}%
\end{table}%

\begin{table}[htbp]%
\begin{center}%
\begin{scriptsize}%
\tabcolsep=2pt%
\begin{tabular}{lrrrrrrrr}
\toprule
\multirow{2}{*}{}&  \multicolumn{2}{c}{\budalg} & \multicolumn{2}{c}{\noheuristic} & \multicolumn{2}{c}{\nopreprocessing} & \multicolumn{2}{c}{\nolb}\\
\cmidrule(rr){2-3}\cmidrule(rr){4-5}\cmidrule(rr){6-7}\cmidrule(rr){8-9}
& \multicolumn{1}{c}{error} & \multicolumn{1}{c}{cpu} & \multicolumn{1}{c}{error} & \multicolumn{1}{c}{cpu} & \multicolumn{1}{c}{error} & \multicolumn{1}{c}{cpu} & \multicolumn{1}{c}{error} & \multicolumn{1}{c}{cpu} \\
\midrule

\texttt{audiology} & 0 & 0.00$^*$ & 0 & 0.00$^*$ & 0 & 0.00$^*$ & 0 & 0.00$^*$\\
\texttt{australian-credit} & 0 & 101$^*$ & 0 & 477$^*$ & 0 & 1002$^*$ & 0 & 153$^*$\\
\texttt{biodeg} & 26 & 2775 & 57 & 3064 & 26 & 2548 & 26 & 3341\\
\texttt{breast-wisconsin} & 0 & 0.02$^*$ & 0 & 0.23$^*$ & 0 & 0.33$^*$ & 0 & 0.03$^*$\\
\texttt{diabetes} & 21 & 827 & 27 & 238 & 26 & 3164 & 21 & 1324\\
\texttt{forest-fires} & 146 & 125 & 142 & 140 & 132 & 1346 & 146 & 124\\
\texttt{german-credit} & 56 & 1192 & 117 & 2789 & 56 & 2472 & 56 & 1446\\
\texttt{ionosphere} & 0 & 0.07$^*$ & 0 & 0.07$^*$ & 0 & 0.49$^*$ & 0 & 0.07$^*$\\
\texttt{titanic} & 72 & 442 & 78 & 2696 & 72 & 471 & 72 & 500\\
\texttt{vehicle} & 0 & 0.09$^*$ & 0 & 196$^*$ & 0 & 0.66$^*$ & 0 & 0.10$^*$\\
\texttt{wine1} & 28 & 892 & 28 & 2666 & 29 & 487 & 28 & 892\\
\texttt{wine2} & 31 & 28 & 31 & 23 & 31 & 168 & 31 & 28\\
\texttt{wine3} & 21 & 524 & 21 & 1062 & 20 & 296 & 21 & 531\\
\bottomrule
\end{tabular}
%
\end{scriptsize}%
\end{center}%
\caption{\label{tab:all72} Comparison with state of the art: $\numex<5000, \numfeat \geq 250$, depth 7}%
\end{table}%

\begin{table}[htbp]%
\begin{center}%
\begin{scriptsize}%
\tabcolsep=2pt%
\begin{tabular}{lrrrrrrrr}
\toprule
\multirow{2}{*}{}&  \multicolumn{2}{c}{\budalg} & \multicolumn{2}{c}{\noheuristic} & \multicolumn{2}{c}{\nopreprocessing} & \multicolumn{2}{c}{\nolb}\\
\cmidrule(rr){2-3}\cmidrule(rr){4-5}\cmidrule(rr){6-7}\cmidrule(rr){8-9}
& \multicolumn{1}{c}{error} & \multicolumn{1}{c}{cpu} & \multicolumn{1}{c}{error} & \multicolumn{1}{c}{cpu} & \multicolumn{1}{c}{error} & \multicolumn{1}{c}{cpu} & \multicolumn{1}{c}{error} & \multicolumn{1}{c}{cpu} \\
\midrule

\texttt{adult\_discretized} & 4191 & 534 & 4203 & 686 & 4162 & 2418 & 4191 & 553\\
\texttt{car} & 11 & 231$^*$ & 11 & 256$^*$ & 11 & 233$^*$ & 11 & 627$^*$\\
\texttt{car\_evaluation} & 80 & 0.00$^*$ & 80 & 0.00$^*$ & 80 & 27$^*$ & 80 & 0.00$^*$\\
\texttt{chess} & 0 & 0.00$^*$ & 0 & 0.00$^*$ & 0 & 0.00$^*$ & 0 & 0.00$^*$\\
\texttt{compas\_discretized} & 1852 & 198$^*$ & 1852 & 184$^*$ & 1852 & 2030 & 1852 & 299$^*$\\
\texttt{HTRU\_2} & 297 & 3334 & 324 & 2967 & 298 & 3052 & 297 & 2460\\
\texttt{hypothyroid} & 22 & 3478 & 23 & 147 & 27 & 113 & 23 & 171\\
\texttt{kr-vs-kp} & 18 & 2550 & 18 & 1423 & 34 & 3090 & 21 & 1756\\
\texttt{magic04} & 2488 & 2773 & 2512 & 2481 & 2488 & 2657 & 2488 & 2586\\
\texttt{seismic\_bumps} & 76 & 2389 & 96 & 1453 & 78 & 1580 & 77 & 1217\\
\texttt{winequality-red} & 2 & 0.01$^*$ & 2 & 0.13$^*$ & 2 & 0.43 & 2 & 0.00$^*$\\
\bottomrule
\end{tabular}
%
\end{scriptsize}%
\end{center}%
\caption{\label{tab:all73} Comparison with state of the art: $\numex \geq 5000, \numfeat < 250$, depth 7}%
\end{table}%

\begin{table}[htbp]%
\begin{center}%
\begin{scriptsize}%
\tabcolsep=2pt%
\begin{tabular}{lrrrrrrrr}
\toprule
\multirow{2}{*}{}&  \multicolumn{2}{c}{\budalg} & \multicolumn{2}{c}{\noheuristic} & \multicolumn{2}{c}{\nopreprocessing} & \multicolumn{2}{c}{\nolb}\\
\cmidrule(rr){2-3}\cmidrule(rr){4-5}\cmidrule(rr){6-7}\cmidrule(rr){8-9}
& \multicolumn{1}{c}{error} & \multicolumn{1}{c}{cpu} & \multicolumn{1}{c}{error} & \multicolumn{1}{c}{cpu} & \multicolumn{1}{c}{error} & \multicolumn{1}{c}{cpu} & \multicolumn{1}{c}{error} & \multicolumn{1}{c}{cpu} \\
\midrule

\texttt{bank} & 3844 & 2369 & 4303 & 252 & 3844 & 2351 & 3844 & 2460\\
\texttt{bank\_conv} & 220 & 1642 & 288 & 1459 & 220 & 1459 & 220 & 1442\\
\texttt{default\_credit} & 4935 & 222 & 5054 & 557 & 4935 & 256 & 4935 & 187\\
\texttt{hand\_posture} & 749 & 2684 & 8944 & 3595 & 749 & 2401 & 749 & 2702\\
\texttt{letter} & 68 & 177 & 168 & 2143 & 70 & 3525 & 68 & 193\\
\texttt{mnist\_0} & 1107 & 2895 & 1556 & 1539 & 1107 & 2983 & 1107 & 2735\\
\texttt{mushroom} & 0 & 0.00$^*$ & 0 & 0.00$^*$ & 0 & 0.00$^*$ & 0 & 0.00$^*$\\
\texttt{pendigits} & 0 & 0.00$^*$ & 0 & 3.5$^*$ & 0 & 0.00$^*$ & 0 & 0.00$^*$\\
\texttt{segment} & 0 & 0.00$^*$ & 0 & 0.00$^*$ & 0 & 0.00$^*$ & 0 & 0.00$^*$\\
\texttt{spambase} & 352 & 3562 & 373 & 2535 & 357 & 2501 & 357 & 2249\\
\texttt{splice-1} & 29 & 3484 & 46 & 3380 & 29 & 3575 & 29 & 3408\\
\texttt{Statlog\_satellite} & 14 & 2428 & 54 & 308 & 14 & 2062 & 14 & 2407\\
\texttt{Statlog\_shuttle} & 0 & 0.04$^*$ & 0 & 25$^*$ & 0 & 0.06$^*$ & 0 & 0.04$^*$\\
\texttt{surgical-deepnet} & 1647 & 1248 & 2246 & 3102 & 1647 & 1086 & 1647 & 1288\\
\texttt{taiwan\_binarised} & 4896 & 1958 & 5016 & 2961 & 4909 & 1426 & 4896 & 2055\\
\texttt{weather-aus} & 1685 & 2048 & 1747 & 1685 & 1685 & 1948 & 1685 & 2083\\
\bottomrule
\end{tabular}
%
\end{scriptsize}%
\end{center}%
\caption{\label{tab:all74} Comparison with state of the art: $\numex \geq 5000, \numfeat \geq 250$, depth 7}%
\end{table}%


\begin{table}[htbp]%
\begin{center}%
\begin{scriptsize}%
\tabcolsep=2pt%
\begin{tabular}{lrrrrrrrr}
\toprule
\multirow{2}{*}{}&  \multicolumn{2}{c}{\budalg} & \multicolumn{2}{c}{\noheuristic} & \multicolumn{2}{c}{\nopreprocessing} & \multicolumn{2}{c}{\nolb}\\
\cmidrule(rr){2-3}\cmidrule(rr){4-5}\cmidrule(rr){6-7}\cmidrule(rr){8-9}
& \multicolumn{1}{c}{error} & \multicolumn{1}{c}{cpu} & \multicolumn{1}{c}{error} & \multicolumn{1}{c}{cpu} & \multicolumn{1}{c}{error} & \multicolumn{1}{c}{cpu} & \multicolumn{1}{c}{error} & \multicolumn{1}{c}{cpu} \\
\midrule

\texttt{anneal} & 34 & 23$^*$ & 36 & 1986 & 36 & 661 & 34 & 32$^*$\\
\texttt{balance-scale} & 0 & 19$^*$ & 0 & 45$^*$ & 0 & 21$^*$ & 0 & 61$^*$\\
\texttt{banknote} & 2 & 0.00$^*$ & 2 & 0.00$^*$ & 2 & 0.00 & 2 & 0.00$^*$\\
\texttt{breast-cancer} & 0 & 0.00$^*$ & 0 & 0.31$^*$ & 0 & 0.00$^*$ & 0 & 0.00$^*$\\
\texttt{heart-cleveland} & 0 & 0.00$^*$ & 0 & 0.00$^*$ & 0 & 0.00$^*$ & 0 & 0.00$^*$\\
\texttt{hepatitis} & 0 & 0.00$^*$ & 0 & 0.00$^*$ & 0 & 0.00$^*$ & 0 & 0.00$^*$\\
\texttt{IndiansDiabetes} & 8 & 4.7$^*$ & 13 & 667 & 8 & 1732 & 8 & 8.6$^*$\\
\texttt{iris} & 1 & 0.00$^*$ & 1 & 0.00$^*$ & 1 & 0.02$^*$ & 1 & 0.00$^*$\\
\texttt{lymph} & 0 & 0.00$^*$ & 0 & 0.00$^*$ & 0 & 0.00$^*$ & 0 & 0.00$^*$\\
\texttt{messidor} & 66 & 604 & 62 & 1498 & 66 & 1515 & 66 & 774\\
\texttt{monk1} & 0 & 0.00$^*$ & 0 & 0.00$^*$ & 0 & 0.00$^*$ & 0 & 0.00$^*$\\
\texttt{monk2} & 0 & 0.00$^*$ & 0 & 0.00$^*$ & 0 & 0.00$^*$ & 0 & 0.00$^*$\\
\texttt{monk3} & 0 & 0.00$^*$ & 0 & 0.00$^*$ & 0 & 0.00$^*$ & 0 & 0.00$^*$\\
\texttt{primary-tumor} & 15 & 0.00$^*$ & 15 & 0.00$^*$ & 15 & 0.28 & 15 & 0.00$^*$\\
\texttt{soybean} & 2 & 0.00$^*$ & 2 & 0.43$^*$ & 2 & 0.00 & 2 & 0.00$^*$\\
\texttt{tic-tac-toe} & 0 & 0.00$^*$ & 0 & 0.00$^*$ & 0 & 0.00$^*$ & 0 & 0.00$^*$\\
\texttt{vote} & 0 & 0.00$^*$ & 0 & 0.00$^*$ & 0 & 0.00$^*$ & 0 & 0.00$^*$\\
\texttt{yeast} & 28 & 1008 & 68 & 2610 & 67 & 466 & 28 & 1633\\
\bottomrule
\end{tabular}
%
\end{scriptsize}%
\end{center}%
\caption{\label{tab:all101} Comparison with state of the art: $\numex<5000, \numfeat<250$, depth 10}%
\end{table}%

\begin{table}[htbp]%
\begin{center}%
\begin{scriptsize}%
\tabcolsep=2pt%
\begin{tabular}{lrrrrrrrr}
\toprule
\multirow{2}{*}{}&  \multicolumn{2}{c}{\budalg} & \multicolumn{2}{c}{\noheuristic} & \multicolumn{2}{c}{\nopreprocessing} & \multicolumn{2}{c}{\nolb}\\
\cmidrule(rr){2-3}\cmidrule(rr){4-5}\cmidrule(rr){6-7}\cmidrule(rr){8-9}
& \multicolumn{1}{c}{error} & \multicolumn{1}{c}{cpu} & \multicolumn{1}{c}{error} & \multicolumn{1}{c}{cpu} & \multicolumn{1}{c}{error} & \multicolumn{1}{c}{cpu} & \multicolumn{1}{c}{error} & \multicolumn{1}{c}{cpu} \\
\midrule

\texttt{audiology} & 0 & 0.00$^*$ & 0 & 0.00$^*$ & 0 & 0.00$^*$ & 0 & 0.00$^*$\\
\texttt{australian-credit} & 0 & 0.04$^*$ & 0 & 0.15$^*$ & 0 & 0.26$^*$ & 0 & 0.04$^*$\\
\texttt{biodeg} & 1 & 1169$^*$ & 40 & 1739 & 1 & 2928 & 1 & 1342$^*$\\
\texttt{breast-wisconsin} & 0 & 0.00$^*$ & 0 & 0.00$^*$ & 0 & 0.00$^*$ & 0 & 0.00$^*$\\
\texttt{diabetes} & 0 & 0.67$^*$ & 0 & 3026$^*$ & 0 & 11$^*$ & 0 & 0.60$^*$\\
\texttt{forest-fires} & 113 & 942 & 114 & 3068 & 118 & 3167 & 113 & 1003\\
\texttt{german-credit} & 0 & 69$^*$ & 62 & 2594 & 0 & 173$^*$ & 0 & 96$^*$\\
\texttt{ionosphere} & 0 & 0.00$^*$ & 0 & 0.02$^*$ & 0 & 0.00$^*$ & 0 & 0.00$^*$\\
\texttt{titanic} & 35 & 3059 & 52 & 943 & 45 & 1077 & 42 & 180\\
\texttt{vehicle} & 0 & 0.00$^*$ & 0 & 60$^*$ & 0 & 0.00$^*$ & 0 & 0.00$^*$\\
\texttt{wine1} & 22 & 545 & 20 & 1469 & 22 & 3227 & 22 & 539\\
\texttt{wine2} & 24 & 399 & 21 & 20 & 24 & 2832 & 24 & 415\\
\texttt{wine3} & 16 & 272 & 17 & 690 & 18 & 1802 & 16 & 270\\
\bottomrule
\end{tabular}
%
\end{scriptsize}%
\end{center}%
\caption{\label{tab:all102} Comparison with state of the art: $\numex<5000, \numfeat \geq 250$, depth 10}%
\end{table}%

\begin{table}[htbp]%
\begin{center}%
\begin{scriptsize}%
\tabcolsep=2pt%
\begin{tabular}{lrrrrrrrr}
\toprule
\multirow{2}{*}{}&  \multicolumn{2}{c}{\budalg} & \multicolumn{2}{c}{\noheuristic} & \multicolumn{2}{c}{\nopreprocessing} & \multicolumn{2}{c}{\nolb}\\
\cmidrule(rr){2-3}\cmidrule(rr){4-5}\cmidrule(rr){6-7}\cmidrule(rr){8-9}
& \multicolumn{1}{c}{error} & \multicolumn{1}{c}{cpu} & \multicolumn{1}{c}{error} & \multicolumn{1}{c}{cpu} & \multicolumn{1}{c}{error} & \multicolumn{1}{c}{cpu} & \multicolumn{1}{c}{error} & \multicolumn{1}{c}{cpu} \\
\midrule

\texttt{adult\_discretized} & 3841 & 2632 & 4119 & 3075 & 3775 & 2994 & 3841 & 2988\\
\texttt{car} & 0 & 0.26$^*$ & 0 & 21$^*$ & 0 & 0.32$^*$ & 0 & 0.44$^*$\\
\texttt{car\_evaluation} & 80 & 0.00$^*$ & 80 & 0.00$^*$ & 80 & 0.00 & 80 & 0.00$^*$\\
\texttt{chess} & 0 & 0.00$^*$ & 0 & 0.00$^*$ & 0 & 0.00$^*$ & 0 & 0.00$^*$\\
\texttt{compas\_discretized} & 1828 & 0.73$^*$ & 1828 & 9.1$^*$ & 1828 & 323 & 1828 & 1.4$^*$\\
\texttt{HTRU\_2} & 219 & 550 & 272 & 340 & 218 & 638 & 219 & 559\\
\texttt{hypothyroid} & 17 & 0.96$^*$ & 17 & 40$^*$ & 17 & 72 & 17 & 1.5$^*$\\
\texttt{kr-vs-kp} & 0 & 1897$^*$ & 0 & 752$^*$ & 5 & 86 & 1 & 400\\
\texttt{magic04} & 1635 & 2746 & 2180 & 1847 & 1653 & 1768 & 1658 & 143\\
\texttt{seismic\_bumps} & 38 & 2591 & 98 & 874 & 45 & 319 & 45 & 1015\\
\texttt{winequality-red} & 2 & 0.00$^*$ & 2 & 0.00$^*$ & 2 & 0.00 & 2 & 0.00$^*$\\
\bottomrule
\end{tabular}
%
\end{scriptsize}%
\end{center}%
\caption{\label{tab:all103} Comparison with state of the art: $\numex \geq 5000, \numfeat < 250$, depth 10}%
\end{table}%

\begin{table}[htbp]%
\begin{center}%
\begin{scriptsize}%
\tabcolsep=2pt%
\begin{tabular}{lrrrrrrrr}
\toprule
\multirow{2}{*}{}&  \multicolumn{2}{c}{\budalg} & \multicolumn{2}{c}{\noheuristic} & \multicolumn{2}{c}{\nopreprocessing} & \multicolumn{2}{c}{\nolb}\\
\cmidrule(rr){2-3}\cmidrule(rr){4-5}\cmidrule(rr){6-7}\cmidrule(rr){8-9}
& \multicolumn{1}{c}{error} & \multicolumn{1}{c}{cpu} & \multicolumn{1}{c}{error} & \multicolumn{1}{c}{cpu} & \multicolumn{1}{c}{error} & \multicolumn{1}{c}{cpu} & \multicolumn{1}{c}{error} & \multicolumn{1}{c}{cpu} \\
\midrule

\texttt{bank} & 3242 & 800 & 4200 & 20 & 3245 & 851 & 3242 & 845\\
\texttt{bank\_conv} & 169 & 2794 & 262 & 896 & 172 & 2851 & 174 & 2555\\
\texttt{default\_credit} & 4547 & 2019 & 4954 & 495 & 4561 & 1878 & 4549 & 1171\\
\texttt{hand\_posture} & 334 & 39 & 8927 & 2467 & 334 & 34 & 334 & 35\\
\texttt{letter} & 0 & 79$^*$ & 88 & 1825 & 0 & 1535$^*$ & 0 & 104$^*$\\
\texttt{mnist\_0} & 383 & 413 & 1721 & 3235 & 383 & 404 & 383 & 450\\
\texttt{mushroom} & 0 & 0.00$^*$ & 0 & 0.00$^*$ & 0 & 0.00$^*$ & 0 & 0.00$^*$\\
\texttt{pendigits} & 0 & 0.00$^*$ & 0 & 0.00$^*$ & 0 & 0.00$^*$ & 0 & 0.00$^*$\\
\texttt{segment} & 0 & 0.00$^*$ & 0 & 0.00$^*$ & 0 & 0.00$^*$ & 0 & 0.00$^*$\\
\texttt{spambase} & 262 & 546 & 321 & 2700 & 272 & 610 & 262 & 562\\
\texttt{splice-1} & 5 & 1160 & 12 & 1676 & 4 & 3506 & 5 & 1205\\
\texttt{Statlog\_satellite} & 3 & 219 & 14 & 1016 & 3 & 195 & 3 & 215\\
\texttt{Statlog\_shuttle} & 0 & 0.02$^*$ & 0 & 0.02$^*$ & 0 & 0.02$^*$ & 0 & 0.02$^*$\\
\texttt{surgical-deepnet} & 965 & 2865 & 1849 & 3204 & 965 & 3133 & 965 & 3192\\
\texttt{taiwan\_binarised} & 4217 & 1001 & 4896 & 2890 & 4189 & 1046 & 4217 & 1041\\
\texttt{weather-aus} & 1601 & 2591 & 1734 & 2391 & 1603 & 1988 & 1601 & 2758\\
\bottomrule
\end{tabular}
%
\end{scriptsize}%
\end{center}%
\caption{\label{tab:all104} Comparison with state of the art: $\numex \geq 5000, \numfeat \geq 250$, depth 10}%
\end{table}%



\begin{table}[htbp]
\begin{center}
\begin{footnotesize}
\tabcolsep=5pt
\begin{tabular}{lrrrrrr}
\toprule
\multirow{2}{*}{}&  \multicolumn{2}{c}{\iti} & \multicolumn{4}{c}{\bfss}\\
\cmidrule(rr){2-3}\cmidrule(rr){4-7}
& \multicolumn{1}{c}{error} & \multicolumn{1}{c}{size} & \multicolumn{1}{c}{error} & \multicolumn{1}{c}{init e.} & \multicolumn{1}{c}{size} & \multicolumn{1}{c}{init s.} \\
\midrule

\texttt{HTRU\_2} & 329 & 195 & \textbf{306} & 235 & \textbf{163} & 609\\
\texttt{IndiansDiabetes} & 116 & \textbf{99} & \textbf{95} & 8 & 133 & 417\\
\texttt{Statlog\_satellite} & 62 & 101 & \textbf{57} & 8 & \textbf{77} & 181\\
\texttt{Statlog\_shuttle} & 0 & \textbf{17} & 0 & 0 & 21 & 21\\
\texttt{adult\_discretized} & 3801 & 1693 & \textbf{3631} & 3300 & \textbf{1041} & 2535\\
\texttt{anneal} & 66 & 75 & \textbf{62} & 34 & \textbf{61} & 157\\
\texttt{audiology} & 4 & 13 & \textbf{3} & 0 & 13 & 21\\
\texttt{australian-credit} & 57 & \textbf{37} & \textbf{47} & 2 & 43 & 161\\
\texttt{balance-scale} & 49 & \textbf{21} & \textbf{48} & 48 & 27 & 27\\
\texttt{bank} & 2931 & \textbf{1201} & \textbf{1921} & 810 & 2341 & 5089\\
\texttt{bank\_conv} & 276 & 223 & \textbf{233} & 90 & 223 & 575\\
\texttt{banknote} & 6 & \textbf{41} & \textbf{4} & 2 & 43 & 49\\
\texttt{biodeg} & 84 & \textbf{85} & \textbf{75} & 25 & 95 & 215\\
\texttt{breast-cancer} & 23 & 21 & \textbf{21} & 5 & 21 & 57\\
\texttt{breast-wisconsin} & 13 & \textbf{21} & 13 & 1 & 23 & 53\\
\texttt{car} & 10 & \textbf{69} & 10 & 0 & 71 & 93\\
\texttt{car\_evaluation} & 80 & 51 & 80 & 80 & \textbf{21} & 21\\
\texttt{chess} & 0 & 3 & 0 & 0 & 3 & 3\\
\texttt{compas\_discretized} & 1863 & 489 & \textbf{1858} & 1828 & \textbf{167} & 285\\
\texttt{default\_credit} & 3723 & \textbf{2353} & \textbf{2798} & 1609 & 3583 & 7723\\
\texttt{diabetes} & 106 & \textbf{73} & \textbf{90} & 13 & 95 & 271\\
\texttt{forest-fires} & 134 & 75 & \textbf{130} & 112 & \textbf{67} & 103\\
\texttt{german-credit} & 146 & \textbf{105} & \textbf{115} & 14 & 113 & 341\\
\texttt{hand\_posture} & 226 & 523 & \textbf{208} & 156 & \textbf{189} & 385\\
\texttt{heart-cleveland} & 39 & \textbf{19} & \textbf{37} & 7 & 25 & 63\\
\texttt{hepatitis} & 14 & \textbf{13} & \textbf{13} & 3 & 17 & 29\\
\texttt{hypothyroid} & 49 & \textbf{33} & \textbf{45} & 22 & 43 & 109\\
\texttt{ionosphere} & 22 & 17 & \textbf{21} & 4 & \textbf{15} & 43\\
\texttt{iris} & 1 & 7 & 1 & 1 & \textbf{5} & 5\\
\texttt{kr-vs-kp} & 7 & \textbf{77} & \textbf{6} & 0 & 87 & 111\\
\texttt{letter} & 69 & 165 & \textbf{57} & 0 & \textbf{161} & 363\\
\texttt{lymph} & 11 & 17 & \textbf{10} & 3 & 17 & 31\\
\texttt{magic04} & 1786 & \textbf{1415} & \textbf{1484} & 770 & 1741 & 4077\\
\texttt{messidor} & 212 & \textbf{171} & \textbf{184} & 39 & 189 & 555\\
\texttt{mnist\_0} & 332 & \textbf{499} & \textbf{268} & 30 & 549 & 1157\\
\texttt{monk1} & 9 & 13 & \textbf{5} & 0 & 13 & 15\\
\texttt{monk2} & 24 & 41 & \textbf{23} & 0 & \textbf{29} & 67\\
\texttt{monk3} & 5 & \textbf{13} & \textbf{4} & 4 & 21 & 23\\
\texttt{mushroom} & 0 & 21 & 0 & 0 & \textbf{17} & 17\\
\texttt{pendigits} & 20 & 43 & 20 & 0 & \textbf{23} & 45\\
\texttt{primary-tumor} & 39 & \textbf{31} & \textbf{37} & 16 & 37 & 113\\
\texttt{segment} & 3 & 11 & \textbf{2} & 0 & \textbf{9} & 11\\
\texttt{seismic\_bumps} & 143 & \textbf{73} & \textbf{125} & 71 & 77 & 289\\
\texttt{soybean} & 21 & \textbf{35} & \textbf{20} & 3 & 51 & 85\\
\texttt{spambase} & 338 & 307 & \textbf{299} & 178 & \textbf{285} & 595\\
\texttt{splice-1} & 71 & 63 & \textbf{65} & 9 & 63 & 197\\
\texttt{surgical-deepnet} & 962 & \textbf{455} & \textbf{124} & 124 & 2281 & 2281\\
\texttt{taiwan\_binarised} & 3845 & \textbf{2379} & \textbf{3108} & 2022 & 3281 & 7243\\
\texttt{tic-tac-toe} & 29 & 75 & \textbf{24} & 0 & \textbf{59} & 97\\
\texttt{titanic} & 107 & 65 & \textbf{106} & 75 & \textbf{43} & 151\\
\texttt{vehicle} & 25 & \textbf{25} & \textbf{21} & 0 & 33 & 73\\
\texttt{vote} & 12 & \textbf{13} & 12 & 5 & 19 & 29\\
\texttt{weather-aus} & \textbf{1160} & 643 & 1371 & 1371 & \textbf{511} & 511\\
\texttt{wine1} & 31 & 27 & \textbf{29} & 17 & \textbf{19} & 35\\
\texttt{wine2} & 24 & 31 & \textbf{22} & 14 & 31 & 47\\
\texttt{wine3} & 26 & 23 & \textbf{25} & 15 & \textbf{19} & 35\\
\texttt{winequality-red} & 10 & \textbf{5} & \textbf{9} & 9 & 7 & 7\\
\texttt{yeast} & 232 & \textbf{177} & \textbf{175} & 13 & 271 & 705\\
\bottomrule
\end{tabular}

\end{footnotesize}
\end{center}
\caption{\label{tab:iti} Comparison with ITI}
\end{table}



\end{document}


% We report in Table~\ref{tab:summaryacc} data averaged over the 47 data sets described above, for
%
%
% the average accuracy found within the one hour time limit for \budalg and \murtree
%
% on relatively shallow trees (3,4 and 5) in tables~\ref{tab:d3}, \ref{tab:d4} and \ref{tab:d5}, respectively.
% We give the minimum \emph{error}, the cpu in seconds \emph{time} and size of the search space (\emph{choices}) required to prove optimality (when a proof is given, as markes by a 1 in the column \emph{opt}) or to find the best solution (otherwise).


% \begin{table}[htbp]
% \begin{center}
% \begin{normalsize}
% \tabcolsep=3.7pt
% \begin{tabular}{lrrrrrrrrr}
\toprule
\multirow{2}{*}{$\mdepth$}&  \multicolumn{3}{c}{\budalg} & \multicolumn{3}{c}{\murtree} & \multicolumn{3}{c}{\dleight}\\
\cmidrule(rr){2-4}\cmidrule(rr){5-7}\cmidrule(rr){8-10}
& \multicolumn{1}{c}{error} & \multicolumn{1}{c}{cpu} & \multicolumn{1}{c}{opt.} & \multicolumn{1}{c}{error} & \multicolumn{1}{c}{cpu$^*$} & \multicolumn{1}{c}{opt.} & \multicolumn{1}{c}{error$^*$} & \multicolumn{1}{c}{cpu$^*$} & \multicolumn{1}{c}{opt.} \\
\midrule

\texttt{3} & \textbf{1328} & \textbf{465} & 0.93 & 1346 & $\mathsmaller{\times}$1.83 & 0.93 & $\mathsmaller{+}$190 & $\mathsmaller{\times}$44 & 0.63\\
\texttt{4} & 1144 & \textbf{594} & 0.61 & \textbf{1141} & $\mathsmaller{\times}$2.58 & \textbf{0.70} & $\mathsmaller{+}$416 & $\mathsmaller{\times}$229 & 0.48\\
\texttt{5} & \textbf{1010} & \textbf{826} & 0.52 & 1054 & $\mathsmaller{\times}$2.53 & 0.52 & $\mathsmaller{+}$738 & $\mathsmaller{\times}$529 & 0.26\\
\texttt{6} & \textbf{889} & \textbf{1139} & 0.41 & 1010 & $\mathsmaller{\times}$3.35 & 0.41 & $\mathsmaller{+}$1050 & $\mathsmaller{\times}$576 & 0.24\\
\texttt{7} & \textbf{789} & \textbf{1215} & 0.39 & 947 & $\mathsmaller{\times}$6.52 & 0.39 & $\mathsmaller{+}$377 & $\mathsmaller{\times}$179 & 0.24\\
\texttt{8} & \textbf{704} & \textbf{792} & \textbf{0.43} & 860 & $\mathsmaller{\times}$6918 & 0.39 & $\mathsmaller{+}$702 & $\mathsmaller{\times}$3615 & 0.26\\
\texttt{9} & \textbf{637} & \textbf{788} & \textbf{0.43} & 789 & $\mathsmaller{\times}$4.09 & 0.35 & $\mathsmaller{+}$943 & $\mathsmaller{\times}$3835 & 0.28\\
\texttt{10} & \textbf{575} & \textbf{678} & \textbf{0.52} & 713 & $\mathsmaller{\times}$4.01 & 0.39 & $\mathsmaller{+}$1021 & $\mathsmaller{\times}$9725 & 0.30\\
\bottomrule
\end{tabular}

% \end{normalsize}
% \end{center}
% \caption{\label{tab:summary} Comparison with the state of the art: computing optimal trees}
% \end{table}


\begin{table}[htbp]
\begin{center}
\begin{footnotesize}
\tabcolsep=1.7pt
\begin{tabular}{lrrrrrrrrrrrrr}
\toprule
\multirow{2}{*}{$\mdepth$}&  \multicolumn{4}{c}{\budalg} & \multicolumn{4}{c}{\murtree} & \multicolumn{5}{c}{\dleight}\\
\cmidrule(rr){2-5}\cmidrule(rr){6-9}\cmidrule(rr){10-14}
& \multicolumn{1}{c}{error} & \multicolumn{1}{c}{acc.} & \multicolumn{1}{c}{cpu} & \multicolumn{1}{c}{opt.} & \multicolumn{1}{c}{error} & \multicolumn{1}{c}{acc.} & \multicolumn{1}{c}{cpu$^*$} & \multicolumn{1}{c}{opt.} & \multicolumn{1}{c}{error$^*$} & \multicolumn{1}{c}{acc.$^*$} & \multicolumn{1}{c}{cpu$^*$} & \multicolumn{1}{c}{sol.} & \multicolumn{1}{c}{opt.} \\
\midrule

\texttt{3} & 151 & 0.8845 & 4 & 1.00 & 151 & 0.8845 & $\mathsmaller{\times}$1.24 & 1.00 & 0 & $\mathsmaller{+}$0.00\% & $\mathsmaller{\times}$19 & 0.86 & 0.86\\
\texttt{4} & 126 & 0.9109 & 305 & 0.88 & 126 & 0.9110 & $\mathsmaller{\times}$2.02 & 0.98 & 0 & $\mathsmaller{+}$0.00\% & $\mathsmaller{\times}$35 & 0.71 & 0.71\\
\texttt{5} & 105 & 0.9330 & 339 & 0.78 & 108 & 0.9313 & $\mathsmaller{\times}$2.07 & 0.76 & 0 & $\mathsmaller{+}$0.00\% & $\mathsmaller{\times}$56 & 0.49 & 0.49\\
\texttt{7} & 73 & 0.9603 & 824 & 0.61 & 111 & 0.9392 & $\mathsmaller{\times}$352 & 0.63 & 0 & $\mathsmaller{+}$0.00\% & $\mathsmaller{\times}$50 & 0.45 & 0.45\\
\texttt{10} & 57 & 0.9744 & 354 & 0.76 & 135 & 0.9345 & $\mathsmaller{\times}$324 & 0.57 & 0 & $\mathsmaller{+}$0.00\% & $\mathsmaller{\times}$572 & 0.49 & 0.49\\
\bottomrule
\end{tabular}

\end{footnotesize}
\end{center}
\caption{\label{tab:summaryaccsmall} Comparison with the state of the art on data sets with at most 10000 datapoints: computing optimal trees}
\end{table}

\begin{table}[htbp]
\begin{center}
\begin{footnotesize}
\tabcolsep=1.7pt
\begin{tabular}{lrrrrrrrrrrrrr}
\toprule
\multirow{2}{*}{$\mdepth$}&  \multicolumn{4}{c}{\budalg} & \multicolumn{4}{c}{\murtree} & \multicolumn{5}{c}{\dleight}\\
\cmidrule(rr){2-5}\cmidrule(rr){6-9}\cmidrule(rr){10-14}
& \multicolumn{1}{c}{error} & \multicolumn{1}{c}{acc.} & \multicolumn{1}{c}{cpu} & \multicolumn{1}{c}{opt.} & \multicolumn{1}{c}{error} & \multicolumn{1}{c}{acc.} & \multicolumn{1}{c}{cpu$^*$} & \multicolumn{1}{c}{opt.} & \multicolumn{1}{c}{error$^*$} & \multicolumn{1}{c}{acc.$^*$} & \multicolumn{1}{c}{cpu$^*$} & \multicolumn{1}{c}{sol.} & \multicolumn{1}{c}{opt.} \\
\midrule

\texttt{3} & \textbf{3527} & \textbf{0.9212} & 1330 & 0.81 & 3579 & 0.9201 & $\mathsmaller{\times}$0.72 & 0.81 & $\mathsmaller{+}$544 & -0.98\% & $\mathsmaller{\times}$26 & 0.88 & 0.19\\
\texttt{4} & 3042 & \textbf{0.9308} & 1227 & 0.12 & \textbf{3034} & 0.9301 & $\mathsmaller{\times}$2.31 & 0.12 & $\mathsmaller{+}$1041 & -1.87\% & $\mathsmaller{\times}$19 & 0.88 & 0.06\\
\texttt{5} & \textbf{2692} & \textbf{0.9376} & 1735 & 0.06 & 2819 & 0.9346 & $\mathsmaller{\times}$1.10 & 0.06 & $\mathsmaller{+}$1371 & -2.50\% & - & 0.88 & 0.00\\
\texttt{6} & \textbf{2377} & \textbf{0.9444} & 1737 & 0.00 & 2720 & 0.9371 & - & 0.00 & $\mathsmaller{+}$2013 & -3.59\% & - & 0.75 & 0.00\\
\texttt{7} & \textbf{2110} & \textbf{0.9497} & 2134 & 0.00 & 2553 & 0.9409 & - & 0.00 & $\mathsmaller{+}$1207 & -3.06\% & - & 0.31 & 0.00\\
\texttt{8} & \textbf{1876} & \textbf{0.9551} & 1411 & 0.00 & 2319 & 0.9459 & - & 0.00 & $\mathsmaller{+}$1905 & -4.10\% & - & 0.44 & 0.00\\
\texttt{9} & \textbf{1691} & \textbf{0.9593} & 1469 & 0.00 & 2114 & 0.9501 & - & 0.00 & $\mathsmaller{+}$2474 & -5.18\% & - & 0.50 & 0.00\\
\texttt{10} & \textbf{1518} & \textbf{0.9631} & 1363 & 0.06 & 1902 & 0.9542 & $\mathsmaller{\times}$3.51 & 0.06 & $\mathsmaller{+}$2808 & -5.18\% & - & 0.50 & 0.00\\
\bottomrule
\end{tabular}

\end{footnotesize}
\end{center}
\caption{\label{tab:summaryacclarge} Comparison with the state of the art on data sets with more than 10000 datapoints: computing optimal trees}
\end{table}


\begin{table}[htbp]
\begin{center}
\begin{footnotesize}
\tabcolsep=1.7pt
\begin{tabular}{lrrrrrrrrrrrrr}
\toprule
&  \multicolumn{4}{c}{\budalg} & \multicolumn{4}{c}{\murtree} & \multicolumn{5}{c}{\dleight}\\
\cmidrule(rr){2-5}\cmidrule(rr){6-9}\cmidrule(rr){10-14}
& \multicolumn{1}{c}{error} & \multicolumn{1}{c}{acc.} & \multicolumn{1}{c}{cpu} & \multicolumn{1}{c}{opt.} & \multicolumn{1}{c}{error} & \multicolumn{1}{c}{acc.} & \multicolumn{1}{c}{cpu} & \multicolumn{1}{c}{opt.} & \multicolumn{1}{c}{error$^*$} & \multicolumn{1}{c}{acc.$^*$} & \multicolumn{1}{c}{cpu$^*$} & \multicolumn{1}{c}{sol.} & \multicolumn{1}{c}{opt.} \\
\midrule

\texttt{D = 3} & \textbf{1328} & \textbf{0.8921} & 465 & 0.93 & 1346 & 0.8917 & \textbf{234} & 0.93 & $\mathsmaller{+}$190 & -0.3\% & $\mathsmaller{\times}$44 & 0.87 & 0.63\\
\texttt{D = 4} & 1144 & \textbf{0.9133} & \textbf{594} & 0.61 & \textbf{1143} & 0.9129 & 731 & 0.61 & $\mathsmaller{+}$416 & -0.7\% & $\mathsmaller{\times}$229 & 0.76 & 0.48\\
\texttt{D = 5} & \textbf{1010} & \textbf{0.9298} & \textbf{826} & 0.52 & 1054 & 0.9282 & 951 & 0.52 & $\mathsmaller{+}$738 & -1.3\% & $\mathsmaller{\times}$529 & 0.57 & 0.26\\
\texttt{D = 6} & \textbf{889} & \textbf{0.9436} & 1139 & \textbf{0.41} & 1009 & 0.9390 & \textbf{1097} & 0.37 & $\mathsmaller{+}$1050 & -1.9\% & $\mathsmaller{\times}$576 & 0.50 & 0.24\\
\texttt{D = 7} & \textbf{789} & \textbf{0.9510} & 1215 & 0.39 & 944 & 0.9438 & \textbf{937} & 0.39 & $\mathsmaller{+}$377 & -1.0\% & $\mathsmaller{\times}$179 & 0.35 & 0.24\\
\texttt{D = 8} & \textbf{704} & \textbf{0.9564} & \textbf{792} & \textbf{0.43} & 861 & 0.9501 & 1080 & 0.39 & $\mathsmaller{+}$702 & -1.5\% & $\mathsmaller{\times}$3615 & 0.41 & 0.26\\
\texttt{D = 9} & \textbf{637} & \textbf{0.9609} & \textbf{788} & \textbf{0.43} & 790 & 0.9534 & 867 & 0.35 & $\mathsmaller{+}$943 & -2.0\% & $\mathsmaller{\times}$3835 & 0.46 & 0.28\\
\texttt{D = 10} & \textbf{575} & \textbf{0.9650} & \textbf{678} & \textbf{0.52} & 714 & 0.9572 & 1002 & 0.39 & $\mathsmaller{+}$1021 & -1.9\% & $\mathsmaller{\times}$9725 & 0.48 & 0.30\\
\bottomrule
\end{tabular}

\end{footnotesize}
\end{center}
\caption{\label{tab:summaryacc} Comparison with the state of the art: computing optimal trees}
\end{table}

\begin{table}[htbp]
\begin{center}
\begin{footnotesize}
\tabcolsep=1.7pt
\begin{tabular}{lrrrrrrrrrrrr}
\toprule
\multirow{2}{*}{$\mdepth$}&  \multicolumn{4}{c}{\budalg} & \multicolumn{4}{c}{\cp} & \multicolumn{4}{c}{\binoct}\\
\cmidrule(rr){2-5}\cmidrule(rr){6-9}\cmidrule(rr){10-13}
& \multicolumn{1}{c}{error} & \multicolumn{1}{c}{acc.} & \multicolumn{1}{c}{cpu} & \multicolumn{1}{c}{opt.} & \multicolumn{1}{c}{error$^*$} & \multicolumn{1}{c}{acc.$^*$} & \multicolumn{1}{c}{cpu$^*$} & \multicolumn{1}{c}{opt.} & \multicolumn{1}{c}{error$^*$} & \multicolumn{1}{c}{acc.$^*$} & \multicolumn{1}{c}{cpu$^*$} & \multicolumn{1}{c}{opt.} \\
\midrule

\texttt{3} & 1138 & 0.8957 & 309 & 0.94 & $\mathsmaller{+}$11 & -0.01\% & $\mathsmaller{\times}$32 & 0.76 & $\mathsmaller{+}$106 & -1.84\% & - & 0.00\\
\texttt{4} & 967 & 0.9169 & 513 & 0.68 & $\mathsmaller{+}$614 & -1.26\% & $\mathsmaller{\times}$53 & 0.56 & $\mathsmaller{+}$144 & -3.80\% & - & 0.00\\
\texttt{5} & 843 & 0.9343 & 707 & 0.56 & $\mathsmaller{+}$850 & -4.63\% & $\mathsmaller{\times}$50 & 0.34 & $\mathsmaller{+}$276 & -6.39\% & - & 0.00\\
\texttt{7} & 646 & 0.9567 & 1186 & 0.44 & $\mathsmaller{+}$1154 & -8.69\% & $\mathsmaller{\times}$1970 & 0.34 & $\mathsmaller{+}$432 & -15.05\% & - & 0.00\\
\texttt{10} & 476 & 0.9705 & 637 & 0.55 & $\mathsmaller{+}$1345 & -9.82\% & $\mathsmaller{\times}$236 & 0.38 & $\mathsmaller{+}$320 & -38.00\% & - & 0.00\\
\bottomrule
\end{tabular}

\end{footnotesize}
\end{center}
\caption{\label{tab:summaryacc} Comparison with the state of the art: computing optimal trees}
\end{table}


\begin{table}[htbp]
\begin{center}
\begin{footnotesize}
\tabcolsep=1.7pt
\begin{tabular}{lrrrrrrrrrrrrr}
\toprule
\multirow{2}{*}{$\mdepth$}&  \multicolumn{3}{c}{\budalg} & \multicolumn{3}{c}{\murtree} & \multicolumn{4}{c}{\dleight} & \multicolumn{3}{c}{\cp}\\
\cmidrule(rr){2-4}\cmidrule(rr){5-7}\cmidrule(rr){8-11}\cmidrule(rr){12-14}
& \multicolumn{1}{c}{opt.} & \multicolumn{1}{c}{acc.} & \multicolumn{1}{c}{cpu} & \multicolumn{1}{c}{opt.} & \multicolumn{1}{c}{acc.$^*$} & \multicolumn{1}{c}{cpu$^*$} & \multicolumn{1}{c}{opt.} & \multicolumn{1}{c}{acc.$^*$} & \multicolumn{1}{c}{cpu$^*$} & \multicolumn{1}{c}{sol.} & \multicolumn{1}{c}{opt.} & \multicolumn{1}{c}{acc.$^*$} & \multicolumn{1}{c}{cpu$^*$} \\
\midrule

\texttt{3} & 0.94 & 0.8957 & 309 & 0.94 & -0.06\% & $\mathsmaller{\times}$1.58 & 0.68 & $\mathsmaller{+}$0.00\% & $\mathsmaller{\times}$19 & 0.68 & 0.76 & $\mathsmaller{+}$0.00\% & $\mathsmaller{\times}$32\\
\texttt{4} & 0.68 & 0.9169 & 513 & 0.77 & -0.34\% & $\mathsmaller{\times}$11 & 0.52 & -0.52\% & $\mathsmaller{\times}$34 & 0.70 & 0.55 & -1.10\% & $\mathsmaller{\times}$55\\
\texttt{5} & 0.56 & 0.9343 & 707 & 0.56 & -0.68\% & $\mathsmaller{\times}$69 & 0.34 & -0.92\% & $\mathsmaller{\times}$56 & 0.54 & 0.34 & -4.22\% & $\mathsmaller{\times}$50\\
\texttt{7} & 0.44 & 0.9567 & 1186 & 0.45 & -2.58\% & $\mathsmaller{\times}$362 & 0.31 & -0.57\% & $\mathsmaller{\times}$50 & 0.38 & 0.34 & -8.05\% & $\mathsmaller{\times}$1970\\
\texttt{10} & 0.55 & 0.9705 & 637 & 0.41 & -4.58\% & $\mathsmaller{\times}$330 & 0.35 & -1.26\% & $\mathsmaller{\times}$588 & 0.46 & 0.38 & -9.62\% & $\mathsmaller{\times}$236\\
\bottomrule
\end{tabular}

\end{footnotesize}
\end{center}
\caption{\label{tab:summaryacc} Comparison with the state of the art: computing optimal trees}
\end{table}






% \begin{figure}
% 	\subfloat[depth=3]{\cactus{Average Accuracy}{CPU time}{\budalg, \murtree, \cart}{{{(0.875686014267174, 0) [a] 
(0.8815520559338406, 0.01) [a] 
(0.8838416392671739, 0.02) [a] 
(0.8843585142671738, 0.03) [a] 
(0.8847739309338405, 0.04) [a] 
(0.8849243476005072, 0.05) [a] 
(0.8849847642671739, 0.06) [a] 
(0.8851051809338406, 0.07) [a] 
(0.8852078892671739, 0.08) [a] 
(0.8852476809338405, 0.11) [a] 
(0.8852885142671738, 0.12) [a] 
(0.8854964309338405, 0.14) [a] 
(0.8855005976005071, 0.17) [a] 
(0.8856228892671738, 0.2) [a] 
(0.8856230976005072, 0.22) [a] 
(0.8869746661860897, 0.25) [a] 
(0.886980499519423, 0.27) [a] 
(0.8869850828527563, 0.28) [a] 
(0.8870207078527563, 0.39) [a] 
(0.8870707078527563, 0.4) [a] 
(0.887090499519423, 0.43) [a] 
(0.8915034415711883, 0.44) [a] 
(0.8916205249045217, 0.47) [a] 
(0.8916215665711884, 0.6) [a] 
(0.8917403165711884, 0.68) [a] 
(0.8921241666666666, 0.82) [a] 
(0.8921325, 0.86) [a] 
(0.8921339583333333, 0.87) [a] 
(0.892135625, 0.98) [a] 
(0.8921360416666666, 1.02) [a] 
(0.8921364583333332, 1.16) [a] 
(0.8922766666666665, 1.17) [a] 
(0.8922887499999999, 1.25) [a] 
(0.8922889583333332, 1.46) [a] 
(0.8923658333333332, 1.64) [a] 
(0.8923670833333331, 1.65) [a] 
(0.8923679166666665, 1.71) [a] 
(0.8923835416666664, 1.94) [a] 
(0.8924070833333331, 2) [a] 
(0.8924104166666664, 2.06) [a] 
(0.8924135416666664, 2.1) [a] 
(0.892413958333333, 2.13) [a] 
(0.8924143749999996, 2.21) [a] 
(0.8924266666666664, 2.44) [a] 
(0.8924308333333331, 2.46) [a] 
(0.8924337499999997, 2.47) [a] 
(0.892435208333333, 2.48) [a] 
(0.8924379166666664, 2.49) [a] 
(0.892460833333333, 2.52) [a] 
(0.8925062499999997, 2.78) [a] 
(0.8925072916666664, 2.79) [a] 
(0.8925433333333331, 2.84) [a] 
(0.8925441666666665, 2.85) [a] 
(0.8925443749999998, 2.86) [a] 
(0.8925933333333331, 2.9) [a] 
(0.8926658333333332, 2.92) [a] 
(0.8927395833333331, 3) [a] 
(0.8927408333333331, 3.01) [a] 
(0.8927458333333331, 3.27) [a] 
(0.8927479166666664, 4.75) [a] 
(0.8929010416666664, 5.31) [a] 
(0.8929166666666664, 5.52) [a] 
(0.8929635416666664, 6.58) [a] 
(0.8929754166666664, 6.68) [a] 
(0.8929862499999996, 7.46) [a] 
(0.892992083333333, 8.33) [a] 
(0.8929935416666663, 8.43) [a] 
(0.8929947916666663, 8.73) [a] 
(0.8930006249999997, 8.79) [a] 
(0.893007708333333, 9.23) [a] 
(0.8930216666666663, 9.44) [a] 
(0.893025833333333, 9.71) [a] 
(0.8930587499999997, 9.97) [a] 
(0.8930822916666664, 11.52) [a] 
(0.8930954166666664, 11.53) [a] 
(0.893125208333333, 11.57) [a] 
(0.8931368749999996, 11.58) [a] 
(0.893195208333333, 11.66) [a] 
(0.893220833333333, 11.69) [a] 
(0.893228958333333, 11.84) [a] 
(0.893229583333333, 12.96) [a] 
(0.8932299999999996, 12.99) [a] 
(0.893230208333333, 13.77) [a] 
(0.893231458333333, 13.9) [a] 
(0.8932354166666663, 13.91) [a] 
(0.8932460416666663, 13.92) [a] 
(0.8932656249999996, 13.93) [a] 
(0.8932704166666663, 13.96) [a] 
(0.8932929166666662, 15.54) [a] 
(0.8933010416666662, 16.49) [a] 
(0.8933031249999995, 16.54) [a] 
(0.8933043749999995, 16.56) [a] 
(0.8933445833333328, 17.23) [a] 
(0.8933474999999994, 17.47) [a] 
(0.8933545833333327, 17.7) [a] 
(0.893361666666666, 17.95) [a] 
(0.8934449999999994, 18.08) [a] 
(0.8934493749999993, 18.11) [a] 
(0.8934564583333326, 18.31) [a] 
(0.893463541666666, 18.63) [a] 
(0.8934679166666659, 18.71) [a] 
(0.8934749999999992, 18.93) [a] 
(0.8934806249999993, 19.04) [a] 
(0.8934862499999993, 19.22) [a] 
(0.8934962499999992, 19.52) [a] 
(0.8934977083333325, 19.56) [a] 
(0.8934991666666658, 19.62) [a] 
(0.8935047916666659, 19.78) [a] 
(0.8935106249999992, 19.94) [a] 
(0.8935147916666659, 20.09) [a] 
(0.8935233333333326, 20.37) [a] 
(0.8935289583333326, 20.61) [a] 
(0.8935362499999993, 20.8) [a] 
(0.8935433333333326, 21.07) [a] 
(0.8935504166666659, 21.3) [a] 
(0.8935545833333326, 21.51) [a] 
(0.893560416666666, 21.65) [a] 
(0.8935799999999993, 23.45) [a] 
(0.8935831249999993, 26.89) [a] 
(0.8935837499999992, 30.92) [a] 
(0.8935931249999992, 31.29) [a] 
(0.8936268749999992, 37.11) [a] 
(0.8936441666666658, 37.12) [a] 
(0.8936447916666658, 49.31) [a] 
(0.8936629166666658, 53.69) [a] 
(0.893668124999999, 53.7) [a] 
(0.893672499999999, 56.77) [a] 
(0.8936733333333323, 66.05) [a] 
(0.8936843749999991, 66.06) [a] 
(0.8936870833333325, 71.58) [a] 
(0.8936929166666658, 71.93) [a] 
(0.8936935416666658, 73.88) [a] 
(0.8936941666666658, 73.99) [a] 
(0.8936970833333324, 111.78) [a] 
(0.8936972916666658, 113.71) [a] 
(0.8937081249999991, 147.05) [a] 
(0.893711874999999, 196.97) [a] 
(0.893718124999999, 214.45) [a] 
(0.8937310416666657, 557.9) [a] 
(0.8937410416666657, 909.3) [a] 
(0.8937466666666657, 909.36) [a] 
(0.893748124999999, 909.39) [a] 
},{(0.8179105759656972, 0) [b] 
(0.8379130696122398, 0.001) [b] 
(0.8405932268074437, 0.002) [b] 
(0.8422043774308753, 0.003) [b] 
(0.8457562789903939, 0.004) [b] 
(0.8473965399451646, 0.005) [b] 
(0.8474478535740645, 0.006) [b] 
(0.8475311869073978, 0.007) [b] 
(0.848416889291446, 0.008) [b] 
(0.8507429479974137, 0.009) [b] 
(0.8509599618863025, 0.01) [b] 
(0.8576242473648263, 0.011) [b] 
(0.8577075806981597, 0.012) [b] 
(0.8577380833823959, 0.013) [b] 
(0.8577521220167176, 0.015) [b] 
(0.8580393368534154, 0.016) [b] 
(0.8588893368534154, 0.018) [b] 
(0.858980844906124, 0.019) [b] 
(0.8629991462871759, 0.021) [b] 
(0.8637031512934337, 0.022) [b] 
(0.8641701428801665, 0.023) [b] 
(0.864226297417453, 0.025) [b] 
(0.8657545105835972, 0.028) [b] 
(0.8658288308906076, 0.03) [b] 
(0.8658801445195075, 0.031) [b] 
(0.8659314581484073, 0.033) [b] 
(0.8668099387106349, 0.04) [b] 
(0.8676492733245041, 0.047) [b] 
(0.8677405927689485, 0.048) [b] 
(0.8683091691578374, 0.049) [b] 
(0.8687532652915887, 0.05) [b] 
(0.8706526005242573, 0.053) [b] 
(0.8706831032084935, 0.057) [b] 
(0.8709140145385428, 0.067) [b] 
(0.8720020222754867, 0.068) [b] 
(0.872232933605536, 0.069) [b] 
(0.8723516420575778, 0.078) [b] 
(0.8725018510439937, 0.083) [b] 
(0.873042061491194, 0.086) [b] 
(0.8733667368158693, 0.094) [b] 
(0.873398640848539, 0.101) [b] 
(0.8735217692961356, 0.103) [b] 
(0.8745751400826525, 0.109) [b] 
(0.8755114696706674, 0.113) [b] 
(0.8760281927071198, 0.124) [b] 
(0.8775497282876442, 0.13) [b] 
(0.8776090825136651, 0.131) [b] 
(0.8777702688450641, 0.135) [b] 
(0.8778021728777339, 0.138) [b] 
(0.877808589056771, 0.139) [b] 
(0.8778536828518647, 0.14) [b] 
(0.8780234843147696, 0.164) [b] 
(0.8782464009814362, 0.181) [b] 
(0.8783103469862549, 0.196) [b] 
(0.8784060590842641, 0.234) [b] 
(0.8787896993297938, 0.265) [b] 
(0.8795555574325156, 0.277) [b] 
(0.8796753786327967, 0.338) [b] 
(0.8799094610298004, 0.366) [b] 
(0.8799497576126502, 0.403) [b] 
(0.8799789242793168, 0.508) [b] 
(0.8800959654778187, 0.562) [b] 
(0.8801258265889298, 0.584) [b] 
(0.8806483960333742, 0.813) [b] 
(0.8817862432555964, 0.89) [b] 
(0.882116798811152, 0.966) [b] 
(0.8827515210333743, 1.057) [b] 
(0.8828685622318762, 1.115) [b] 
(0.8828692566763205, 1.174) [b] 
(0.8829879651283623, 1.412) [b] 
(0.8830254651283623, 1.61) [b] 
(0.88743789518235, 1.899) [b] 
(0.8875826868490166, 1.921) [b] 
(0.8875889368490166, 1.947) [b] 
(0.8878823396267944, 2.051) [b] 
(0.8882531729601277, 2.099) [b] 
(0.8882615129654653, 2.115) [b] 
(0.8887694990765764, 2.127) [b] 
(0.8889379018543542, 2.129) [b] 
(0.8891472768543541, 2.204) [b] 
(0.8893542212987986, 2.229) [b] 
(0.8893611657432431, 2.233) [b] 
(0.8899097768543542, 2.268) [b] 
(0.8902983185210209, 2.271) [b] 
(0.8905493601876876, 2.283) [b] 
(0.8905774851876876, 2.293) [b] 
(0.8911413740765766, 2.302) [b] 
(0.8913257490765766, 2.313) [b] 
(0.8914427902750784, 2.352) [b] 
(0.8914917486084117, 2.454) [b] 
(0.8914948736084117, 2.48) [b] 
(0.8914952208306339, 2.501) [b] 
(0.8915302902750784, 2.555) [b] 
(0.8915768180528562, 2.558) [b] 
(0.8916518180528562, 2.639) [b] 
(0.8916976513861895, 2.657) [b] 
(0.8917358458306339, 2.829) [b] 
(0.8917556374973006, 2.956) [b] 
(0.891778206941745, 2.964) [b] 
(0.8918049430528562, 2.996) [b] 
(0.891810845830634, 3.148) [b] 
(0.8918334152750784, 4.004) [b] 
(0.8918344569417451, 4.265) [b] 
(0.8918882763861896, 4.548) [b] 
(0.8919365402750785, 5.027) [b] 
(0.891955984719523, 7.135) [b] 
(0.8919570263861897, 7.414) [b] 
(0.8919670958306342, 9.721) [b] 
(0.8920140706521298, 10.067) [b] 
(0.8920803900965742, 10.214) [b] 
(0.8921088623187965, 11.163) [b] 
(0.8921119873187965, 14.638) [b] 
(0.8921352512076853, 22.431) [b] 
(0.8922883762076853, 22.719) [b] 
(0.8922928900965742, 22.841) [b] 
(0.8923439317632409, 22.894) [b] 
(0.8923446262076853, 22.994) [b] 
(0.8923602512076853, 24.303) [b] 
(0.8923720567632408, 26.471) [b] 
(0.8923734456521297, 48.438) [b] 
(0.8924137422349795, 59.205) [b] 
(0.8924245061238684, 65.867) [b] 
(0.8924276311238684, 73.12) [b] 
(0.8924338811238683, 90.042) [b] 
(0.8924377005683128, 98.501) [b] 
(0.892450547790535, 285.014) [b] 
(0.8930555472211238, 481.63) [b] 
(0.8933359822513026, 555.776) [b] 
(0.8933445234197345, 567.67) [b] 
(0.8933452559921302, 613.139) [b] 
(0.8933623383289939, 3426.9) [b] 
},{(0.8784109791666664, 0.001) [c] 
(0.8784109791666664, 1.7516523333333336) [c] 
(0.8784109791666664, 3600) [c] 
}}}{legend pos=north west}}
% 	\subfloat[depth=5]{\cactus{Average Accuracy}{CPU time}{\budalg, \murtree, \cart}{{{(0.8976572034057506, 0) [a] 
(0.9060691392671738, 0.01) [a] 
(0.9085149031560626, 0.02) [a] 
(0.9130899726005072, 0.03) [a] 
(0.9134299726005072, 0.04) [a] 
(0.9139026809338406, 0.05) [a] 
(0.9142541392671738, 0.06) [a] 
(0.914311847600507, 0.07) [a] 
(0.9143391392671737, 0.08) [a] 
(0.9145468476005072, 0.09) [a] 
(0.9146047642671737, 0.1) [a] 
(0.9146778892671737, 0.11) [a] 
(0.9147153892671738, 0.12) [a] 
(0.9147568476005071, 0.13) [a] 
(0.9147970559338404, 0.14) [a] 
(0.9148520559338404, 0.15) [a] 
(0.9151424726005073, 0.16) [a] 
(0.9152299726005072, 0.17) [a] 
(0.9152641392671739, 0.18) [a] 
(0.9153966392671739, 0.19) [a] 
(0.9154995559338406, 0.2) [a] 
(0.9155283059338406, 0.21) [a] 
(0.917732999519423, 0.22) [a] 
(0.9177475828527564, 0.23) [a] 
(0.9177942495194231, 0.24) [a] 
(0.9179975828527563, 0.25) [a] 
(0.9179984161860897, 0.26) [a] 
(0.918052374519423, 0.27) [a] 
(0.9182340411860896, 0.28) [a] 
(0.918236749519423, 0.29) [a] 
(0.9183409161860896, 0.3) [a] 
(0.9184646661860896, 0.31) [a] 
(0.9184652911860895, 0.32) [a] 
(0.9184950828527562, 0.33) [a] 
(0.9185211245194228, 0.34) [a] 
(0.9185252911860895, 0.35) [a] 
(0.9185467495194228, 0.36) [a] 
(0.9185677911860896, 0.37) [a] 
(0.9185679995194229, 0.39) [a] 
(0.9186194578527562, 0.4) [a] 
(0.9186788328527563, 0.41) [a] 
(0.9187107078527562, 0.42) [a] 
(0.9187346661860895, 0.43) [a] 
(0.9187554995194228, 0.44) [a] 
(0.9187971661860895, 0.45) [a] 
(0.9232117749045216, 0.47) [a] 
(0.9232128165711881, 0.48) [a] 
(0.9232134415711881, 0.49) [a] 
(0.9232396915711879, 0.5) [a] 
(0.9232405249045211, 0.51) [a] 
(0.923243024904521, 0.52) [a] 
(0.923244899904521, 0.53) [a] 
(0.923248024904521, 0.55) [a] 
(0.9232573999045209, 0.56) [a] 
(0.9232601082378542, 0.57) [a] 
(0.9232665665711876, 0.58) [a] 
(0.9232676082378543, 0.6) [a] 
(0.9232971915711876, 0.61) [a] 
(0.9233003165711876, 0.63) [a] 
(0.9233007332378542, 0.65) [a] 
(0.9233251082378543, 0.66) [a] 
(0.9236878165711876, 0.68) [a] 
(0.9237296915711876, 0.69) [a] 
(0.9237767749045208, 0.71) [a] 
(0.9239569832378541, 0.72) [a] 
(0.9239973999045208, 0.73) [a] 
(0.9240455249045207, 0.74) [a] 
(0.9240726082378541, 0.75) [a] 
(0.9240730249045207, 0.77) [a] 
(0.9241434415711873, 0.78) [a] 
(0.9241503165711874, 0.8) [a] 
(0.9241509415711874, 0.81) [a] 
(0.9241530249045207, 0.82) [a] 
(0.9241555249045207, 0.83) [a] 
(0.924158233237854, 0.84) [a] 
(0.9241584415711873, 0.85) [a] 
(0.9241830249045206, 0.86) [a] 
(0.9246581249999989, 0.87) [a] 
(0.9246672916666655, 0.88) [a] 
(0.9246677083333321, 0.93) [a] 
(0.9246756249999988, 0.95) [a] 
(0.9247531249999987, 0.97) [a] 
(0.9247887499999988, 0.98) [a] 
(0.9248191666666655, 1.01) [a] 
(0.9248333333333322, 1.04) [a] 
(0.9248337499999988, 1.05) [a] 
(0.9248397916666654, 1.06) [a] 
(0.9248722916666654, 1.07) [a] 
(0.9249124999999987, 1.1) [a] 
(0.9249160416666654, 1.11) [a] 
(0.9249966666666654, 1.12) [a] 
(0.924997083333332, 1.16) [a] 
(0.9249974999999986, 1.18) [a] 
(0.925000208333332, 1.21) [a] 
(0.925002083333332, 1.25) [a] 
(0.9250135416666654, 1.26) [a] 
(0.9250187499999988, 1.32) [a] 
(0.9250197916666654, 1.34) [a] 
(0.9250479166666654, 1.45) [a] 
(0.9250797916666654, 1.46) [a] 
(0.9251760416666654, 1.47) [a] 
(0.9252464583333321, 1.5) [a] 
(0.925247083333332, 1.55) [a] 
(0.9254083333333321, 1.62) [a] 
(0.925499583333332, 1.73) [a] 
(0.9255060416666654, 1.74) [a] 
(0.9255974999999987, 1.81) [a] 
(0.9256281249999987, 1.82) [a] 
(0.925634583333332, 1.92) [a] 
(0.9256672916666654, 1.93) [a] 
(0.925667708333332, 1.94) [a] 
(0.9256806249999987, 2.06) [a] 
(0.9256872916666654, 2.09) [a] 
(0.9257654166666655, 2.1) [a] 
(0.9257666666666654, 2.16) [a] 
(0.9257681249999987, 2.19) [a] 
(0.925792708333332, 2.27) [a] 
(0.925852083333332, 2.29) [a] 
(0.925852708333332, 2.31) [a] 
(0.9258543749999987, 2.35) [a] 
(0.9258552083333319, 2.36) [a] 
(0.9258558333333319, 2.39) [a] 
(0.9259152083333319, 2.4) [a] 
(0.9259162499999984, 2.42) [a] 
(0.9259164583333318, 2.43) [a] 
(0.9259756249999985, 2.49) [a] 
(0.9259772916666652, 2.55) [a] 
(0.9259777083333318, 2.56) [a] 
(0.9259804166666652, 2.57) [a] 
(0.9259808333333318, 2.59) [a] 
(0.9259818749999985, 2.61) [a] 
(0.9259877083333319, 2.65) [a] 
(0.9259887499999986, 2.79) [a] 
(0.9259897916666653, 2.8) [a] 
(0.9259902083333319, 2.84) [a] 
(0.9259910416666651, 2.85) [a] 
(0.9259914583333317, 2.9) [a] 
(0.9260574999999984, 2.92) [a] 
(0.9260599999999984, 2.93) [a] 
(0.9261193749999984, 2.97) [a] 
(0.9261449999999984, 3.01) [a] 
(0.9261706249999985, 3.02) [a] 
(0.9261964583333318, 3.13) [a] 
(0.9261983333333318, 3.17) [a] 
(0.9262006249999984, 3.2) [a] 
(0.9262014583333317, 3.23) [a] 
(0.926201666666665, 3.25) [a] 
(0.926202291666665, 3.44) [a] 
(0.9262864583333317, 3.49) [a] 
(0.9262870833333317, 3.5) [a] 
(0.9263156249999982, 3.52) [a] 
(0.9263435416666648, 3.53) [a] 
(0.9263445833333315, 3.61) [a] 
(0.9263702083333315, 3.62) [a] 
(0.9263958333333315, 3.63) [a] 
(0.9263985416666649, 3.73) [a] 
(0.9264243749999982, 3.75) [a] 
(0.9265854166666648, 3.81) [a] 
(0.9266660416666648, 3.84) [a] 
(0.9266856249999981, 3.92) [a] 
(0.9267183333333314, 3.93) [a] 
(0.9267885416666648, 4.01) [a] 
(0.926835624999998, 4.28) [a] 
(0.9268589583333314, 4.29) [a] 
(0.926882499999998, 4.35) [a] 
(0.9268827083333314, 4.55) [a] 
(0.9268835416666648, 4.58) [a] 
(0.9268841666666647, 4.67) [a] 
(0.9268847916666647, 4.8) [a] 
(0.9269129166666646, 4.82) [a] 
(0.9269585416666647, 4.84) [a] 
(0.9269858333333314, 4.89) [a] 
(0.9269885416666648, 5.22) [a] 
(0.9270156249999981, 5.24) [a] 
(0.9270364583333315, 5.4) [a] 
(0.9271068749999981, 5.46) [a] 
(0.9271124999999981, 5.63) [a] 
(0.9271424999999981, 5.67) [a] 
(0.9272129166666647, 6) [a] 
(0.927214999999998, 6.01) [a] 
(0.9272158333333314, 6.2) [a] 
(0.9272170833333313, 6.22) [a] 
(0.9272406249999979, 6.47) [a] 
(0.9272433333333313, 6.55) [a] 
(0.9273070833333313, 6.69) [a] 
(0.9273389583333314, 6.71) [a] 
(0.9273708333333314, 6.72) [a] 
(0.9273718749999981, 6.94) [a] 
(0.9273729166666648, 6.99) [a] 
(0.9274433333333314, 7.33) [a] 
(0.9274477083333313, 7.46) [a] 
(0.9274497916666646, 7.63) [a] 
(0.927464374999998, 7.64) [a] 
(0.927470624999998, 7.71) [a] 
(0.927547499999998, 8.24) [a] 
(0.9275477083333313, 8.28) [a] 
(0.9276247916666647, 8.31) [a] 
(0.9276252083333313, 8.7) [a] 
(0.927628749999998, 9.05) [a] 
(0.927705624999998, 9.27) [a] 
(0.9277827083333313, 9.34) [a] 
(0.9277883333333313, 9.46) [a] 
(0.9277958333333314, 9.54) [a] 
(0.927796249999998, 9.66) [a] 
(0.927796874999998, 10.31) [a] 
(0.9278562499999979, 10.58) [a] 
(0.9278704166666646, 10.71) [a] 
(0.9279872916666646, 11.08) [a] 
(0.928104374999998, 15.76) [a] 
(0.928129999999998, 16.03) [a] 
(0.9281366666666647, 16.15) [a] 
(0.9282537499999981, 16.7) [a] 
(0.9282793749999981, 17.75) [a] 
(0.9282820833333315, 17.94) [a] 
(0.9282831249999982, 17.95) [a] 
(0.9282835416666648, 17.99) [a] 
(0.9283445833333315, 18.03) [a] 
(0.9283702083333315, 19.25) [a] 
(0.9284006249999982, 19.91) [a] 
(0.9284012499999982, 20.09) [a] 
(0.9284020833333315, 20.1) [a] 
(0.9284027083333315, 20.16) [a] 
(0.9284041666666648, 21.08) [a] 
(0.9284297916666648, 21.25) [a] 
(0.9284312499999982, 21.47) [a] 
(0.9284318749999981, 21.84) [a] 
(0.9284329166666648, 22.07) [a] 
(0.9284333333333314, 23.03) [a] 
(0.9284343749999981, 23.06) [a] 
(0.9284570833333314, 23.55) [a] 
(0.9284585416666647, 23.59) [a] 
(0.9284591666666647, 23.61) [a] 
(0.9284656249999981, 23.81) [a] 
(0.9284772916666647, 23.99) [a] 
(0.9284779166666647, 24.02) [a] 
(0.9284816666666647, 24.93) [a] 
(0.9285120833333314, 24.97) [a] 
(0.928513124999998, 24.99) [a] 
(0.928513749999998, 25.02) [a] 
(0.9285152083333313, 25.09) [a] 
(0.9285202083333314, 25.19) [a] 
(0.9285266666666647, 25.68) [a] 
(0.9285779166666647, 25.91) [a] 
(0.9286035416666647, 25.96) [a] 
(0.9286097916666647, 25.97) [a] 
(0.9286104166666647, 26.13) [a] 
(0.9286145833333314, 26.15) [a] 
(0.9286152083333313, 26.17) [a] 
(0.928618124999998, 26.41) [a] 
(0.9286222916666647, 27.9) [a] 
(0.9286735416666646, 28.47) [a] 
(0.9286991666666646, 28.52) [a] 
(0.9287131249999979, 28.54) [a] 
(0.9287270833333312, 28.55) [a] 
(0.9287343749999979, 28.66) [a] 
(0.9287349999999979, 29.71) [a] 
(0.9287656249999978, 31.92) [a] 
(0.9288827083333312, 32.84) [a] 
(0.9289072916666645, 33.38) [a] 
(0.9289906249999978, 36.02) [a] 
(0.9290322916666646, 36.13) [a] 
(0.9290947916666646, 36.14) [a] 
(0.9291364583333314, 36.16) [a] 
(0.9291572916666647, 36.22) [a] 
(0.9291956249999981, 38.47) [a] 
(0.9291979166666647, 38.72) [a] 
(0.9291995833333314, 38.94) [a] 
(0.929199999999998, 38.98) [a] 
(0.9292004166666646, 39.14) [a] 
(0.9292239583333313, 39.23) [a] 
(0.9292245833333312, 39.53) [a] 
(0.929225624999998, 39.57) [a] 
(0.9292285416666646, 40.1) [a] 
(0.9292295833333313, 40.99) [a] 
(0.9292322916666647, 42.88) [a] 
(0.9292335416666646, 44.54) [a] 
(0.9292349999999979, 44.78) [a] 
(0.9292720833333312, 46.23) [a] 
(0.9292724999999978, 47.09) [a] 
(0.9292741666666645, 51.35) [a] 
(0.9292762499999978, 51.36) [a] 
(0.9292768749999978, 52.52) [a] 
(0.9292779166666645, 52.53) [a] 
(0.9292783333333311, 52.56) [a] 
(0.9292797916666644, 52.58) [a] 
(0.9292808333333311, 57.12) [a] 
(0.9292839583333311, 57.14) [a] 
(0.9292847916666644, 57.2) [a] 
(0.9292860416666644, 58.42) [a] 
(0.9293262499999977, 63.21) [a] 
(0.9293272916666643, 65.33) [a] 
(0.9293543749999976, 69.17) [a] 
(0.9293599999999976, 71.35) [a] 
(0.9293631249999976, 71.63) [a] 
(0.9293635416666642, 71.64) [a] 
(0.9293641666666642, 72.69) [a] 
(0.9293645833333308, 72.79) [a] 
(0.9293656249999975, 78.4) [a] 
(0.9293906249999975, 82.68) [a] 
(0.9293918749999974, 82.69) [a] 
(0.9293933333333307, 82.7) [a] 
(0.9293937499999974, 84.51) [a] 
(0.9294012499999974, 85.24) [a] 
(0.9294020833333307, 85.25) [a] 
(0.9294110416666641, 85.43) [a] 
(0.9294166666666641, 87.01) [a] 
(0.9294181249999974, 91.52) [a] 
(0.929418541666664, 91.55) [a] 
(0.9294193749999974, 91.58) [a] 
(0.9294204166666641, 91.65) [a] 
(0.9294208333333307, 91.66) [a] 
(0.9294268749999974, 94.98) [a] 
(0.9294302083333308, 94.99) [a] 
(0.9294306249999974, 95) [a] 
(0.9294331249999974, 95.05) [a] 
(0.9294339583333308, 104.63) [a] 
(0.9294381249999975, 111.54) [a] 
(0.9294383333333308, 127.17) [a] 
(0.9294481249999975, 129.95) [a] 
(0.9294487499999975, 137.63) [a] 
(0.9294633333333309, 143.13) [a] 
(0.9295227083333308, 143.99) [a] 
(0.9295233333333308, 144.15) [a] 
(0.9295243749999975, 144.16) [a] 
(0.9295314583333308, 145.19) [a] 
(0.9295377083333308, 146.58) [a] 
(0.9295414583333308, 146.81) [a] 
(0.9295445833333308, 151.49) [a] 
(0.9295452083333308, 154.44) [a] 
(0.9295502083333308, 154.45) [a] 
(0.9295558333333308, 156.32) [a] 
(0.9295579166666641, 156.33) [a] 
(0.9295602083333308, 156.53) [a] 
(0.9295668749999975, 159.31) [a] 
(0.9295683333333308, 159.32) [a] 
(0.9295695833333307, 159.82) [a] 
(0.9295699999999973, 161.01) [a] 
(0.9295702083333307, 170.48) [a] 
(0.9295737499999974, 192.74) [a] 
(0.9295914583333308, 197.27) [a] 
(0.9295924999999975, 197.46) [a] 
(0.9295968749999974, 202.57) [a] 
(0.9296041666666641, 202.66) [a] 
(0.9297212499999975, 203.14) [a] 
(0.9297356249999975, 203.91) [a] 
(0.9297393749999975, 204.01) [a] 
(0.9297397916666641, 204.13) [a] 
(0.9297629166666641, 210.66) [a] 
(0.9297804166666641, 210.94) [a] 
(0.9297879166666642, 211.03) [a] 
(0.9297960416666642, 211.48) [a] 
(0.9298039583333308, 211.57) [a] 
(0.9298170833333308, 211.85) [a] 
(0.9298199999999974, 238.11) [a] 
(0.9298433333333308, 241.96) [a] 
(0.9298668749999974, 242.71) [a] 
(0.9298683333333307, 260.63) [a] 
(0.9298687499999974, 261.33) [a] 
(0.929872916666664, 264.08) [a] 
(0.9298977083333307, 268.2) [a] 
(0.9299004166666641, 283.51) [a] 
(0.9299114583333308, 283.78) [a] 
(0.9299197916666642, 283.79) [a] 
(0.9299479166666642, 283.81) [a] 
(0.9299487499999975, 285.19) [a] 
(0.9299510416666642, 285.5) [a] 
(0.9299514583333308, 288.07) [a] 
(0.9299629166666642, 294.51) [a] 
(0.9299656249999976, 294.58) [a] 
(0.9299662499999976, 299.39) [a] 
(0.9299983333333309, 302.82) [a] 
(0.9299987499999975, 303.82) [a] 
(0.9300039583333308, 390.04) [a] 
(0.9300199999999975, 390.05) [a] 
(0.9300227083333309, 390.12) [a] 
(0.9300462499999975, 391.94) [a] 
(0.9300560416666642, 395.26) [a] 
(0.9300637499999975, 397.09) [a] 
(0.9300649999999975, 403.19) [a] 
(0.9300664583333308, 403.35) [a] 
(0.9301835416666642, 410.51) [a] 
(0.9301837499999975, 418.82) [a] 
(0.9301839583333309, 419.45) [a] 
(0.9301870833333309, 421.53) [a] 
(0.9301893749999975, 440.68) [a] 
(0.9301935416666642, 452.12) [a] 
(0.9302529166666642, 490.66) [a] 
(0.9302539583333309, 496.93) [a] 
(0.9302545833333309, 496.94) [a] 
(0.9302554166666642, 496.95) [a] 
(0.9302591666666642, 498.64) [a] 
(0.9302597916666642, 498.69) [a] 
(0.9302606249999975, 498.8) [a] 
(0.9302610416666641, 498.91) [a] 
(0.9302612499999975, 500.97) [a] 
(0.9302614583333308, 501.69) [a] 
(0.9302624999999975, 501.81) [a] 
(0.9303218749999975, 506) [a] 
(0.9303222916666641, 534.73) [a] 
(0.9303243749999974, 534.8) [a] 
(0.930324791666664, 534.9) [a] 
(0.9303662499999974, 543.34) [a] 
(0.930366666666664, 553.63) [a] 
(0.9303706249999973, 582.49) [a] 
(0.930372291666664, 582.66) [a] 
(0.930374791666664, 582.67) [a] 
(0.9303749999999974, 592.97) [a] 
(0.9303752083333308, 630.13) [a] 
(0.9303760416666641, 630.16) [a] 
(0.9303766666666641, 630.35) [a] 
(0.9303770833333307, 630.85) [a] 
(0.930382291666664, 632.37) [a] 
(0.9303835416666639, 632.38) [a] 
(0.9303845833333306, 656.59) [a] 
(0.930409166666664, 690.24) [a] 
(0.9304102083333307, 707.86) [a] 
(0.9304133333333306, 712.65) [a] 
(0.930414166666664, 722.32) [a] 
(0.9304143749999974, 722.86) [a] 
(0.930416666666664, 729.97) [a] 
(0.9304570833333307, 776.98) [a] 
(0.9304581249999974, 802.4) [a] 
(0.9304591666666641, 802.45) [a] 
(0.9304606249999974, 805.93) [a] 
(0.930461041666664, 871.02) [a] 
(0.9304612499999974, 886.23) [a] 
(0.930461666666664, 886.8) [a] 
(0.9304627083333307, 896.33) [a] 
(0.9304637499999974, 899.38) [a] 
(0.9304658333333307, 909.41) [a] 
(0.9304720833333306, 909.52) [a] 
(0.9304956249999973, 1135.4) [a] 
(0.9304960416666639, 1151.9) [a] 
(0.9306131249999973, 1154.5) [a] 
(0.9306135416666639, 1182.4) [a] 
(0.9306137499999972, 1182.5) [a] 
(0.9306166666666639, 1312.8) [a] 
(0.9306399999999972, 1420.6) [a] 
(0.9306635416666639, 1427.2) [a] 
(0.9306870833333305, 1427.7) [a] 
(0.9306912499999972, 1474.5) [a] 
(0.9306943749999972, 1569.5) [a] 
(0.9306954166666638, 1691.4) [a] 
(0.9306958333333304, 1693.3) [a] 
(0.9306970833333303, 1695) [a] 
(0.9307270833333303, 1700.9) [a] 
(0.930731874999997, 1701) [a] 
(0.930738749999997, 1706.8) [a] 
(0.9307393749999969, 1792.6) [a] 
(0.930773749999997, 1915.2) [a] 
(0.9307789583333302, 1915.3) [a] 
(0.9307818749999969, 1915.8) [a] 
(0.9307877083333302, 1915.9) [a] 
(0.9307945833333302, 1935.6) [a] 
(0.9307949999999968, 1982.3) [a] 
(0.9307954166666634, 1982.4) [a] 
(0.9307956249999968, 1982.9) [a] 
(0.9307960416666634, 2030) [a] 
(0.9307974999999967, 2096.3) [a] 
(0.9307985416666633, 2097.2) [a] 
(0.93079958333333, 2097.4) [a] 
(0.93080083333333, 2162.3) [a] 
(0.9308037499999966, 2162.5) [a] 
(0.9308052083333299, 2162.6) [a] 
(0.9308066666666632, 2162.7) [a] 
(0.9308093749999966, 2162.9) [a] 
(0.9308137499999966, 2163) [a] 
(0.9308152083333299, 2163.1) [a] 
(0.9308193749999966, 2163.2) [a] 
(0.9308208333333299, 2163.3) [a] 
(0.9308237499999965, 2163.4) [a] 
(0.9308252083333298, 2163.6) [a] 
(0.9308279166666632, 2163.8) [a] 
(0.9308293749999965, 2164) [a] 
(0.9308308333333298, 2164.1) [a] 
(0.9308322916666631, 2164.2) [a] 
(0.9308337499999965, 2164.4) [a] 
(0.9308349999999964, 2164.5) [a] 
(0.9308364583333297, 2164.7) [a] 
(0.930837916666663, 2164.8) [a] 
(0.9308408333333297, 2165.5) [a] 
(0.930842291666663, 2166) [a] 
(0.9308435416666629, 2167.2) [a] 
(0.9308464583333296, 2167.6) [a] 
(0.9308468749999962, 2407.8) [a] 
(0.9308470833333296, 2408.1) [a] 
(0.9308474999999962, 2411.4) [a] 
(0.9308499999999962, 2842.3) [a] 
(0.9308568749999961, 2842.4) [a] 
(0.9308587499999962, 2842.5) [a] 
(0.9308589583333295, 2842.8) [a] 
(0.9308649999999963, 2844.5) [a] 
(0.9308652083333296, 2844.6) [a] 
(0.9308662499999963, 2846.5) [a] 
(0.9308677083333297, 2846.6) [a] 
(0.9308687499999964, 3010.7) [a] 
(0.9308708333333297, 3011.3) [a] 
(0.9308729166666629, 3082.5) [a] 
(0.9308808333333296, 3111) [a] 
(0.930881041666663, 3229.5) [a] 
},{(0.8296745796170372, 0) [b] 
(0.8681085992941731, 0.001) [b] 
(0.8706998777805137, 0.002) [b] 
(0.8749793708672606, 0.003) [b] 
(0.8763643978681644, 0.004) [b] 
(0.8787834133406778, 0.005) [b] 
(0.8793536363288303, 0.006) [b] 
(0.8804969463365279, 0.007) [b] 
(0.8808358228666292, 0.008) [b] 
(0.8817114953368342, 0.009) [b] 
(0.883533077678795, 0.01) [b] 
(0.8844245953707681, 0.011) [b] 
(0.884486033507935, 0.012) [b] 
(0.8851128480580613, 0.013) [b] 
(0.8857916416796269, 0.015) [b] 
(0.8869000031978531, 0.016) [b] 
(0.8872124461827099, 0.018) [b] 
(0.8873541128493766, 0.019) [b] 
(0.8874811817285738, 0.02) [b] 
(0.8877028129342476, 0.021) [b] 
(0.8879460430299181, 0.022) [b] 
(0.8879527994214644, 0.024) [b] 
(0.8879914598456806, 0.026) [b] 
(0.8880219625299168, 0.028) [b] 
(0.8881067038732626, 0.031) [b] 
(0.8881303149843738, 0.033) [b] 
(0.8881914572315047, 0.035) [b] 
(0.8882245260145735, 0.036) [b] 
(0.8883092673579193, 0.039) [b] 
(0.888320387365036, 0.04) [b] 
(0.8883231673668153, 0.043) [b] 
(0.8886407718302531, 0.047) [b] 
(0.8889949384969198, 0.049) [b] 
(0.8890029246080309, 0.051) [b] 
(0.8893024131853792, 0.053) [b] 
(0.8894431789511449, 0.054) [b] 
(0.8897407483955894, 0.055) [b] 
(0.8898111312784722, 0.056) [b] 
(0.8898815141613551, 0.058) [b] 
(0.8899145829444238, 0.063) [b] 
(0.8899466638396091, 0.064) [b] 
(0.8899771665238453, 0.065) [b] 
(0.890010235306914, 0.066) [b] 
(0.8906952772153602, 0.069) [b] 
(0.8907049034956556, 0.072) [b] 
(0.8910934885651001, 0.075) [b] 
(0.891152842791121, 0.077) [b] 
(0.8913285372355654, 0.078) [b] 
(0.8913424372444614, 0.08) [b] 
(0.8915277506828998, 0.082) [b] 
(0.8916777506828998, 0.085) [b] 
(0.8917034074973497, 0.088) [b] 
(0.8917096574973497, 0.091) [b] 
(0.8919833301848153, 0.093) [b] 
(0.8920068175955632, 0.096) [b] 
(0.8921928295003251, 0.098) [b] 
(0.8934758844759688, 0.099) [b] 
(0.8935210233648577, 0.1) [b] 
(0.8943974219704515, 0.103) [b] 
(0.8944358849797687, 0.104) [b] 
(0.8945028988686575, 0.105) [b] 
(0.8945775516464353, 0.106) [b] 
(0.8950126210908798, 0.108) [b] 
(0.8955907460908797, 0.109) [b] 
(0.895753593313102, 0.11) [b] 
(0.895787968313102, 0.111) [b] 
(0.8976093946551745, 0.113) [b] 
(0.8976580057662856, 0.114) [b] 
(0.8979923807662856, 0.115) [b] 
(0.8982510613218412, 0.116) [b] 
(0.8984302279885078, 0.117) [b] 
(0.8987309224329523, 0.118) [b] 
(0.8988035352801745, 0.119) [b] 
(0.8988087436135078, 0.12) [b] 
(0.8994611326211052, 0.122) [b] 
(0.9002804210106182, 0.123) [b] 
(0.9003424249788722, 0.125) [b] 
(0.9013262693921056, 0.127) [b] 
(0.9013567720763418, 0.128) [b] 
(0.9019126748541195, 0.133) [b] 
(0.9020721950174682, 0.134) [b] 
(0.9021443956329877, 0.136) [b] 
(0.9021612866118537, 0.137) [b] 
(0.9022232905801076, 0.141) [b] 
(0.9022857905801076, 0.143) [b] 
(0.9022988277098531, 0.15) [b] 
(0.9024610304236934, 0.152) [b] 
(0.9025695373681378, 0.154) [b] 
(0.9027051710486934, 0.155) [b] 
(0.9027370750813631, 0.156) [b] 
(0.902863422790258, 0.158) [b] 
(0.9028915000589013, 0.166) [b] 
(0.9029220027431375, 0.167) [b] 
(0.902935039872883, 0.172) [b] 
(0.9029655425571192, 0.173) [b] 
(0.9030076584600842, 0.174) [b] 
(0.9033201584600843, 0.181) [b] 
(0.9035170994916607, 0.182) [b] 
(0.9035587661583274, 0.184) [b] 
(0.9035618911583274, 0.192) [b] 
(0.903574928288073, 0.195) [b] 
(0.9035763034709723, 0.213) [b] 
(0.9035997908817202, 0.231) [b] 
(0.9036112492150535, 0.24) [b] 
(0.9036529158817203, 0.253) [b] 
(0.9037965940426398, 0.258) [b] 
(0.9038275356578747, 0.261) [b] 
(0.9045933937605964, 0.286) [b] 
(0.9046142270939298, 0.296) [b] 
(0.9049649215383743, 0.303) [b] 
(0.9049929988070176, 0.307) [b] 
(0.9054492031865796, 0.343) [b] 
(0.9055703837421352, 0.351) [b] 
(0.9056313891106076, 0.38) [b] 
(0.9061640279994965, 0.393) [b] 
(0.9061695835550521, 0.414) [b] 
(0.9061702711465017, 0.425) [b] 
(0.9062312765149741, 0.428) [b] 
(0.9062333598483074, 0.439) [b] 
(0.9062638625325437, 0.441) [b] 
(0.9063809037310455, 0.447) [b] 
(0.9064291290396875, 0.458) [b] 
(0.9064353790396875, 0.461) [b] 
(0.9064360666311372, 0.464) [b] 
(0.9073079416311371, 0.48) [b] 
(0.9073619927635084, 0.483) [b] 
(0.9073924954477446, 0.484) [b] 
(0.9074159828584925, 0.497) [b] 
(0.907426117445812, 0.498) [b] 
(0.907750681846222, 0.509) [b] 
(0.9077608164335416, 0.535) [b] 
(0.9077641946293148, 0.537) [b] 
(0.9077735696293148, 0.546) [b] 
(0.9077798196293148, 0.554) [b] 
(0.9078033070400626, 0.569) [b] 
(0.9078060848178404, 0.577) [b] 
(0.9078125009968775, 0.605) [b] 
(0.9078138898857664, 0.611) [b] 
(0.9079111121079886, 0.635) [b] 
(0.9079144903037618, 0.664) [b] 
(0.9080315315022637, 0.671) [b] 
(0.9080329203911526, 0.676) [b] 
(0.9080922746171735, 0.682) [b] 
(0.9081542785854274, 0.686) [b] 
(0.9081709452520941, 0.69) [b] 
(0.9081942091409829, 0.721) [b] 
(0.9081952508076496, 0.769) [b] 
(0.9082187382183975, 0.848) [b] 
(0.9082679895974362, 0.856) [b] 
(0.908281878486325, 0.859) [b] 
(0.9084319884678956, 0.866) [b] 
(0.908446571801229, 0.931) [b] 
(0.9084700592119769, 0.933) [b] 
(0.908481170323088, 0.99) [b] 
(0.908486031434199, 1) [b] 
(0.9084929758786435, 1.035) [b] 
(0.9085914786367207, 1.053) [b] 
(0.9086161043262401, 1.064) [b] 
(0.9086491731093088, 1.151) [b] 
(0.9087662143078107, 1.2) [b] 
(0.9087992830908794, 1.227) [b] 
(0.9088104030979962, 1.292) [b] 
(0.9088128336535518, 1.315) [b] 
(0.9088263753202185, 1.41) [b] 
(0.9088288058757741, 1.549) [b] 
(0.908945847074276, 1.561) [b] 
(0.908947583185387, 1.664) [b] 
(0.9089531387409426, 1.74) [b] 
(0.9089671773752643, 1.776) [b] 
(0.9089918301530421, 1.797) [b] 
(0.9091530164844411, 1.807) [b] 
(0.9092866491549564, 1.834) [b] 
(0.9092998435994009, 1.871) [b] 
(0.9094241491549564, 1.916) [b] 
(0.9094656213630473, 1.928) [b] 
(0.909593905435297, 1.973) [b] 
(0.9140063354892846, 2.086) [b] 
(0.9140166493610296, 2.098) [b] 
(0.9141649132499184, 2.121) [b] 
(0.9143055382499184, 2.128) [b] 
(0.9144338223221681, 2.159) [b] 
(0.9144421556555014, 2.198) [b] 
(0.9146178500999458, 2.213) [b] 
(0.9147907667666124, 2.218) [b] 
(0.9147977112110569, 2.221) [b] 
(0.9150428500999458, 2.255) [b] 
(0.9150442252828451, 2.264) [b] 
(0.9152970030606229, 2.295) [b] 
(0.9153667947272895, 2.336) [b] 
(0.9154195725050673, 2.383) [b] 
(0.9155063780606228, 2.415) [b] 
(0.9155292947272895, 2.614) [b] 
(0.9155570725050672, 2.661) [b] 
(0.9155667947272894, 2.697) [b] 
(0.9156974112372163, 2.704) [b] 
(0.9158998668276029, 2.711) [b] 
(0.916044097951297, 2.716) [b] 
(0.9160506287767933, 2.746) [b] 
(0.9160702212532824, 2.755) [b] 
(0.9161406041361653, 2.78) [b] 
(0.916141645802832, 2.834) [b] 
(0.916148061981869, 2.847) [b] 
(0.9162395700345777, 2.937) [b] 
(0.9162700727188139, 2.938) [b] 
(0.9163005754030501, 2.963) [b] 
(0.9163336441861188, 2.991) [b] 
(0.9163667129691876, 2.993) [b] 
(0.9163788657469654, 3.03) [b] 
(0.9164264351914098, 3.065) [b] 
(0.9164777488203096, 3.173) [b] 
(0.9164787904869763, 3.275) [b] 
(0.9165301041158762, 3.434) [b] 
(0.9165918516257006, 3.513) [b] 
(0.916601921070145, 3.553) [b] 
(0.9169538354845594, 3.801) [b] 
(0.9171649841332081, 3.802) [b] 
(0.917235367016091, 3.807) [b] 
(0.9175872814305054, 3.858) [b] 
(0.9176576643133882, 3.859) [b] 
(0.9177280471962711, 3.882) [b] 
(0.917798430079154, 4.534) [b] 
(0.9178265073477972, 5.213) [b] 
(0.9178778209766971, 5.888) [b] 
(0.917903477791147, 5.898) [b] 
(0.9179291346055969, 5.902) [b] 
(0.9179537602951162, 6.297) [b] 
(0.9179582741840051, 6.394) [b] 
(0.9180352446273549, 6.396) [b] 
(0.9180609014418047, 6.407) [b] 
(0.9181247095071442, 6.715) [b] 
(0.9181566135398139, 6.727) [b] 
(0.9181885175724837, 6.743) [b] 
(0.9187227056066717, 6.807) [b] 
(0.9187820598326926, 6.838) [b] 
(0.9188414140587134, 7.107) [b] 
(0.9188685407948246, 7.129) [b] 
(0.9189132342390527, 7.181) [b] 
(0.9190493773460859, 7.184) [b] 
(0.9190728647568338, 7.238) [b] 
(0.9190963521675817, 7.248) [b] 
(0.9190997303633549, 7.297) [b] 
(0.9191031085591281, 7.32) [b] 
(0.919162462785149, 7.494) [b] 
(0.9191631503765987, 7.55) [b] 
(0.9191638379680483, 7.551) [b] 
(0.919164525559498, 7.566) [b] 
(0.9191672759252967, 7.597) [b] 
(0.9191676231475189, 7.788) [b] 
(0.9191696859218679, 7.951) [b] 
(0.9191710611047672, 7.952) [b] 
(0.9191717486962169, 7.962) [b] 
(0.9192022513804531, 7.965) [b] 
(0.9192070645206007, 8.346) [b] 
(0.919209147853934, 8.373) [b] 
(0.9192223422983785, 8.594) [b] 
(0.9194641217954772, 9.044) [b] 
(0.9194706526209735, 9.171) [b] 
(0.9194977793570847, 9.196) [b] 
(0.9195005593588639, 9.464) [b] 
(0.9195599135848848, 9.664) [b] 
(0.9195654691404404, 10.052) [b] 
(0.9195959718246766, 10.239) [b] 
(0.9196264745089128, 10.241) [b] 
(0.919656977193149, 10.243) [b] 
(0.9196874798773852, 10.25) [b] 
(0.9197179825616214, 10.251) [b] 
(0.9197484852458576, 10.689) [b] 
(0.9198188681287405, 11.03) [b] 
(0.9198493708129767, 11.196) [b] 
(0.9203105253268655, 13.391) [b] 
(0.9203919055351989, 13.393) [b] 
(0.9204190322713101, 13.399) [b] 
(0.9204732857435323, 13.401) [b] 
(0.9205275392157545, 13.42) [b] 
(0.9205546659518656, 13.485) [b] 
(0.9205817926879768, 13.531) [b] 
(0.9207174263685324, 13.549) [b] 
(0.9207716798407546, 13.552) [b] 
(0.9207988065768657, 13.596) [b] 
(0.9208259333129769, 15.213) [b] 
(0.9208680492159419, 15.276) [b] 
(0.9208820878502636, 15.277) [b] 
(0.9208961264845853, 15.291) [b] 
(0.920910165118907, 15.3) [b] 
(0.9209492765081435, 15.83) [b] 
(0.9209753507676346, 15.831) [b] 
(0.9209818693325074, 15.835) [b] 
(0.9209949064622529, 15.842) [b] 
(0.9210079435919984, 15.894) [b] 
(0.9210325692815178, 15.977) [b] 
(0.9210466079158395, 16.732) [b] 
(0.9210737346519506, 17.015) [b] 
(0.9211694467499598, 17.025) [b] 
(0.9212332548152993, 17.026) [b] 
(0.921265158847969, 17.118) [b] 
(0.9212922855840802, 17.146) [b] 
(0.9213053227138257, 17.148) [b] 
(0.9213324494499369, 17.197) [b] 
(0.9213643534826066, 17.53) [b] 
(0.9213962575152763, 17.531) [b] 
(0.921428161547946, 17.535) [b] 
(0.9214600655806158, 17.549) [b] 
(0.9214861398401069, 17.659) [b] 
(0.9214991769698524, 18.392) [b] 
(0.9215056955347252, 18.398) [b] 
(0.9215839183131983, 18.415) [b] 
(0.9216543011960812, 18.49) [b] 
(0.921724684078964, 18.958) [b] 
(0.9217663507456307, 20.54) [b] 
(0.9218496840789641, 20.541) [b] 
(0.9218705174122974, 20.568) [b] 
(0.9218913507456308, 20.691) [b] 
(0.9219121840789641, 20.78) [b] 
(0.9220996840789641, 20.789) [b] 
(0.9221830174122975, 20.791) [b] 
(0.9222135200965337, 21.379) [b] 
(0.9222440227807699, 21.387) [b] 
(0.9223050281492423, 21.39) [b] 
(0.9223053753714645, 21.693) [b] 
(0.92231841250121, 27.798) [b] 
(0.9223510053255738, 27.813) [b] 
(0.9223548247700183, 29.178) [b] 
(0.9223596858811294, 29.64) [b] 
(0.9223603803255738, 29.969) [b] 
(0.9223812136589071, 30.107) [b] 
(0.9224203250481436, 30.738) [b] 
(0.9224268436130164, 30.74) [b] 
(0.9224407325019053, 31.597) [b] 
(0.9224650380574608, 31.612) [b] 
(0.9224678158352386, 31.657) [b] 
(0.922468510279683, 32.339) [b] 
(0.9224692047241274, 33.79) [b] 
(0.9224705936130163, 33.812) [b] 
(0.9224771121778891, 34.985) [b] 
(0.9224805844001113, 38.296) [b] 
(0.9225062412145612, 39.497) [b] 
(0.922531898029011, 39.838) [b] 
(0.922535023029011, 40.675) [b] 
(0.922553773029011, 41.693) [b] 
(0.9225794298434609, 42.297) [b] 
(0.9226387840694817, 43.577) [b] 
(0.9226644408839316, 43.68) [b] 
(0.9226900976983815, 44.048) [b] 
(0.9227157545128314, 46.693) [b] 
(0.9227185345146106, 50.997) [b] 
(0.9227746890518972, 54.257) [b] 
(0.9228308435891838, 54.271) [b] 
(0.9228448822235055, 54.278) [b] 
(0.9228589208578272, 54.282) [b] 
(0.9228729594921489, 54.284) [b] 
(0.9228869981264706, 54.414) [b] 
(0.9228925536820262, 56.931) [b] 
(0.9229342203486929, 57.304) [b] 
(0.9230175536820262, 57.456) [b] 
(0.9230383870153596, 57.458) [b] 
(0.9230800536820263, 57.479) [b] 
(0.9230835259042485, 57.509) [b] 
(0.9231460259042484, 57.545) [b] 
(0.9233072122356475, 62.578) [b] 
(0.9233475088184973, 62.646) [b] 
(0.923387805401347, 63.582) [b] 
(0.9235048465998489, 71.422) [b] 
(0.9235058882665156, 78.203) [b] 
(0.9236999854887378, 79.168) [b] 
(0.92370658271096, 79.665) [b] 
(0.9237107493776266, 80.257) [b] 
(0.9237122145224181, 85.438) [b] 
(0.923735701933166, 88.195) [b] 
(0.9237613587476159, 92.058) [b] 
(0.9238126723765158, 92.06) [b] 
(0.9238383291909656, 92.091) [b] 
(0.9238639860054155, 92.828) [b] 
(0.9238896428198654, 92.879) [b] 
(0.9238899900420876, 97.265) [b] 
(0.9239156468565375, 101.217) [b] 
(0.9239669604854374, 101.222) [b] 
(0.9239926172998872, 101.253) [b] 
(0.9240182741143371, 101.906) [b] 
(0.924043930928787, 101.966) [b] 
(0.9240633753732315, 103.063) [b] 
(0.9240644170398982, 103.41) [b] 
(0.9240665003732315, 104.577) [b] 
(0.9240675420398982, 105.918) [b] 
(0.9240692781510093, 106.325) [b] 
(0.9240696253732315, 109.137) [b] 
(0.9240748337065647, 109.546) [b] 
(0.9240888723408864, 126.291) [b] 
(0.9242059135393883, 133.304) [b] 
(0.9242201496504994, 135.126) [b] 
(0.9243149413171661, 135.689) [b] 
(0.9243902885393883, 135.824) [b] 
(0.9244020940949439, 136.2) [b] 
(0.9244204968727217, 136.685) [b] 
(0.9244221085319922, 138.963) [b] 
(0.9244332196431033, 152.484) [b] 
(0.9244363446431033, 152.701) [b] 
(0.9245840987802192, 154.148) [b] 
(0.9246087244697385, 154.171) [b] 
(0.9246104605808496, 154.362) [b] 
(0.9246361173952995, 160.262) [b] 
(0.9246385479508551, 160.575) [b] 
(0.9246631736403744, 166.552) [b] 
(0.9246888304548243, 175.697) [b] 
(0.9247193331390605, 183.83) [b] 
(0.9247449899535104, 193.266) [b] 
(0.9247470732868437, 193.517) [b] 
(0.9247727301012936, 202.522) [b] 
(0.9247983869157435, 202.523) [b] 
(0.9248240437301933, 202.869) [b] 
(0.92482508539686, 209.187) [b] 
(0.9248261270635267, 209.308) [b] 
(0.9248517838779766, 212.173) [b] 
(0.924893103322421, 215.727) [b] 
(0.925436891045915, 219.827) [b] 
(0.9255138614892647, 222.502) [b] 
(0.9255167085454087, 253.216) [b] 
(0.9255181320734807, 289.211) [b] 
(0.9257522144704844, 289.598) [b] 
(0.9258692556689863, 299.14) [b] 
(0.9258716862245419, 299.622) [b] 
(0.9258781024035789, 323.02) [b] 
(0.9258815746258011, 358.098) [b] 
(0.9259003246258011, 358.528) [b] 
(0.9259176857369121, 358.889) [b] 
(0.9259263662924677, 359.088) [b] 
(0.9259404049267894, 363.918) [b] 
(0.9259544435611111, 363.926) [b] 
(0.9259613880055556, 369.195) [b] 
(0.9260795408355295, 374.44) [b] 
(0.926344471391085, 423.562) [b] 
(0.9263479436133072, 423.719) [b] 
(0.9263545408355294, 424.2) [b] 
(0.9263555825021961, 424.346) [b] 
(0.9263576658355294, 424.518) [b] 
(0.9263580130577516, 424.663) [b] 
(0.9263659991688626, 428.665) [b] 
(0.9263673880577515, 429.475) [b] 
(0.9263677352799737, 431.218) [b] 
(0.9265342880643946, 448.653) [b] 
(0.9265371351205386, 456.73) [b] 
(0.9265478990094275, 460.85) [b] 
(0.9265583156760941, 461.624) [b] 
(0.9265788017872053, 472.078) [b] 
(0.9270163670954603, 513.827) [b] 
(0.9270228979209567, 518.336) [b] 
(0.9270833145876234, 520.225) [b] 
(0.9270895645876234, 520.291) [b] 
(0.9270926895876234, 521.288) [b] 
(0.9271083145876233, 527.279) [b] 
(0.9271135229209566, 528.284) [b] 
(0.9271138159499149, 546.585) [b] 
(0.9271158992832482, 547.892) [b] 
(0.9273450659499148, 559.73) [b] 
(0.927351663172137, 559.836) [b] 
(0.9273898576165814, 560.054) [b] 
(0.927392288172137, 560.39) [b] 
(0.9273926353943592, 560.947) [b] 
(0.9273943715054703, 561.889) [b] 
(0.9273971492832481, 563.048) [b] 
(0.9273981909499148, 564.75) [b] 
(0.927425317686026, 565.887) [b] 
(0.9274562204638037, 597.769) [b] 
(0.927461428797137, 597.824) [b] 
(0.9274631649082481, 597.956) [b] 
(0.9274649010193592, 598.086) [b] 
(0.927520803797137, 624.553) [b] 
(0.9275277482415815, 624.627) [b] 
(0.927531567686026, 624.806) [b] 
(0.9275909219120468, 627.85) [b] 
(0.9275947413564913, 645.023) [b] 
(0.9276006441342691, 645.077) [b] 
(0.927624131545017, 655.529) [b] 
(0.9276476189557649, 657.421) [b] 
(0.9277240078446537, 673.556) [b] 
(0.9277243550668759, 673.641) [b] 
(0.9277253967335426, 794.88) [b] 
(0.9277358134002093, 795.444) [b] 
(0.9277420634002093, 953.978) [b] 
(0.927773967432879, 971.454) [b] 
(0.9277906340995457, 998.702) [b] 
(0.9278141215102935, 1017.28) [b] 
(0.9278142680247726, 1083.25) [b] 
(0.9278212124692171, 1109.95) [b] 
(0.9278222541358838, 1110.39) [b] 
(0.927825726358106, 1150.3) [b] 
(0.9278285041358838, 1159.26) [b] 
(0.9279055874692171, 1184.77) [b] 
(0.9279142680247727, 1184.86) [b] 
(0.927916351358106, 1188.34) [b] 
(0.9279621846914393, 1246.02) [b] 
(0.927963226358106, 1246.98) [b] 
(0.9279646152469949, 1247.41) [b] 
(0.9279649624692171, 1247.81) [b] 
(0.9279701708025504, 1248.27) [b] 
(0.9279705180247726, 1248.7) [b] 
(0.9280031569136615, 1296.76) [b] 
(0.9280094069136615, 1296.85) [b] 
(0.9280104485803282, 1297.02) [b] 
(0.9280146152469949, 1297.2) [b] 
(0.9280153096914393, 1297.44) [b] 
(0.9280278096914393, 1368.67) [b] 
(0.9280285041358837, 1368.73) [b] 
(0.9280295458025504, 1368.86) [b] 
(0.9280298930247726, 1369.06) [b] 
(0.9280323235803282, 1369.3) [b] 
(0.928091677806349, 1476.16) [b] 
(0.9281510320323699, 1476.18) [b] 
(0.9281538098101477, 1571.79) [b] 
(0.9281772972208956, 1798.77) [b] 
(0.9281828527764512, 1828.71) [b] 
(0.9281831999986734, 1829.63) [b] 
(0.9281908388875623, 1830.23) [b] 
(0.9281911861097845, 1830.93) [b] 
(0.9281915333320067, 1836.01) [b] 
(0.9281918805542289, 1836.71) [b] 
(0.9285207043046986, 1839.92) [b] 
(0.9285911665369421, 1839.93) [b] 
(0.9286146539476899, 1840) [b] 
(0.9286851161799334, 1840.88) [b] 
(0.9287086035906813, 1842.81) [b] 
(0.928755578412177, 1844.85) [b] 
(0.9287790658229249, 1844.88) [b] 
(0.9288025532336728, 1845.22) [b] 
(0.9288028462626311, 1923.7) [b] 
(0.928919887461133, 1977.94) [b] 
(0.9289233596833552, 2036.45) [b] 
(0.9289237069055774, 2036.55) [b] 
(0.9289282207944662, 2036.89) [b] 
(0.9289292624611329, 2036.98) [b] 
(0.9289303041277996, 2038.77) [b] 
(0.9289313457944663, 2054.03) [b] 
(0.9289760436187422, 2094.42) [b] 
(0.9289995310294901, 2119.15) [b] 
(0.9290465058509857, 2119.25) [b] 
(0.929071131540505, 2251.76) [b] 
(0.9290718259849494, 2298.06) [b] 
(0.9290791176516161, 2298.14) [b] 
(0.9290815482071717, 2298.24) [b] 
(0.9290857148738384, 2298.32) [b] 
(0.9290898815405051, 2300.91) [b] 
(0.9290926615422843, 2507.54) [b] 
(0.9291253004311733, 2518.12) [b] 
(0.9291287726533954, 2518.23) [b] 
(0.9291541198756177, 2518.51) [b] 
(0.9291565504311733, 2518.64) [b] 
(0.92916071709784, 2519.95) [b] 
(0.9291634948756178, 2520.12) [b] 
(0.9291780782089512, 2520.37) [b] 
(0.9291822448756178, 2520.53) [b] 
(0.9291968282089512, 2544.66) [b] 
(0.929237124791801, 2621.99) [b] 
(0.9292399047935802, 2859.84) [b] 
(0.9292464356190766, 3121.47) [b] 
(0.9293057898450975, 3398.97) [b] 
(0.9293651440711184, 3399.01) [b] 
},{(0.9101962708333334, 0.001) [c] 
(0.9101962708333334, 2.8048337916666664) [c] 
(0.9101962708333334, 3600) [c] 
}}}{legend pos=north west}}
% 	\subfloat[depth=7]{\cactus{Average Accuracy}{CPU time}{\budalg, \murtree, \cart}{{{(0.9082726060274735, 0) [a] 
(0.9161712920449518, 0.01) [a] 
(0.9210576809338408, 0.02) [a] 
(0.9250673337116188, 0.03) [a] 
(0.9292812226005075, 0.04) [a] 
(0.9294024726005073, 0.05) [a] 
(0.9294203892671741, 0.06) [a] 
(0.929734139267174, 0.07) [a] 
(0.9299326809338406, 0.08) [a] 
(0.930259139267174, 0.09) [a] 
(0.9303139309338406, 0.1) [a] 
(0.9303558059338407, 0.11) [a] 
(0.9304814309338407, 0.12) [a] 
(0.9304828892671739, 0.13) [a] 
(0.9304937226005073, 0.14) [a] 
(0.9305249726005074, 0.15) [a] 
(0.9306637226005074, 0.16) [a] 
(0.9307772642671742, 0.17) [a] 
(0.9307776809338408, 0.19) [a] 
(0.9307853892671741, 0.2) [a] 
(0.9308099726005075, 0.21) [a] 
(0.9309449726005077, 0.22) [a] 
(0.9309453892671743, 0.23) [a] 
(0.9309468476005076, 0.24) [a] 
(0.9336063328527567, 0.25) [a] 
(0.93363841618609, 0.26) [a] 
(0.9336388328527566, 0.27) [a] 
(0.9336400828527566, 0.28) [a] 
(0.9336407078527565, 0.29) [a] 
(0.9336421661860899, 0.3) [a] 
(0.9337259161860899, 0.31) [a] 
(0.9337467495194233, 0.32) [a] 
(0.9337675828527566, 0.33) [a] 
(0.9338309161860899, 0.34) [a] 
(0.9338313328527567, 0.35) [a] 
(0.93383216618609, 0.37) [a] 
(0.9338323745194234, 0.39) [a] 
(0.9338325828527567, 0.4) [a] 
(0.9338342495194234, 0.43) [a] 
(0.9338794578527567, 0.44) [a] 
(0.9338923745194234, 0.46) [a] 
(0.9338988328527568, 0.47) [a] 
(0.9339232078527568, 0.5) [a] 
(0.9383428165711888, 0.51) [a] 
(0.9383663582378554, 0.53) [a] 
(0.938367399904522, 0.54) [a] 
(0.9383682332378553, 0.55) [a] 
(0.9383684415711887, 0.57) [a] 
(0.9383686499045221, 0.58) [a] 
(0.9383690665711887, 0.6) [a] 
(0.938486774904522, 0.61) [a] 
(0.9384876082378553, 0.62) [a] 
(0.9386503165711887, 0.63) [a] 
(0.9387319832378554, 0.64) [a] 
(0.9387869832378554, 0.65) [a] 
(0.9387871915711887, 0.66) [a] 
(0.9388415665711888, 0.69) [a] 
(0.9388434415711887, 0.7) [a] 
(0.9388440665711887, 0.71) [a] 
(0.938844274904522, 0.73) [a] 
(0.9388446915711887, 0.75) [a] 
(0.9388548999045221, 0.76) [a] 
(0.9388884415711887, 0.77) [a] 
(0.9388888582378553, 0.79) [a] 
(0.9388894832378553, 0.8) [a] 
(0.9390521915711887, 0.81) [a] 
(0.9390526082378553, 0.82) [a] 
(0.9390559415711887, 0.83) [a] 
(0.9390601082378552, 0.84) [a] 
(0.9390676082378553, 0.85) [a] 
(0.9390680249045219, 0.86) [a] 
(0.9390821915711886, 0.87) [a] 
(0.9391188582378551, 0.95) [a] 
(0.93972625, 0.96) [a] 
(0.9397264583333333, 1.02) [a] 
(0.9397306249999999, 1.06) [a] 
(0.9397547916666666, 1.09) [a] 
(0.9397552083333333, 1.11) [a] 
(0.9397770833333332, 1.17) [a] 
(0.9397774999999998, 1.2) [a] 
(0.9398381249999999, 1.24) [a] 
(0.9398414583333332, 1.25) [a] 
(0.9398818749999999, 1.41) [a] 
(0.9399633333333333, 1.45) [a] 
(0.9399695833333332, 1.52) [a] 
(0.9399699999999999, 1.56) [a] 
(0.9399704166666665, 1.57) [a] 
(0.9399724999999998, 1.69) [a] 
(0.9399737499999997, 1.72) [a] 
(0.9399741666666663, 1.73) [a] 
(0.9399756249999996, 1.77) [a] 
(0.939977083333333, 1.79) [a] 
(0.9399777083333329, 1.8) [a] 
(0.9399783333333329, 1.81) [a] 
(0.9399787499999995, 1.86) [a] 
(0.9401583333333329, 1.93) [a] 
(0.9404149999999997, 1.94) [a] 
(0.940415208333333, 1.96) [a] 
(0.9404160416666664, 1.99) [a] 
(0.9404172916666663, 2) [a] 
(0.9404177083333329, 2.02) [a] 
(0.9405429166666663, 2.04) [a] 
(0.9409020833333331, 2.05) [a] 
(0.9409022916666665, 2.08) [a] 
(0.9409162499999998, 2.1) [a] 
(0.9409224999999999, 2.11) [a] 
(0.940964375, 2.12) [a] 
(0.9409935416666666, 2.13) [a] 
(0.9410191666666666, 2.24) [a] 
(0.9410704166666666, 2.31) [a] 
(0.9411106249999999, 2.42) [a] 
(0.9411618749999998, 2.44) [a] 
(0.9411629166666665, 2.45) [a] 
(0.9412033333333332, 2.47) [a] 
(0.9412043749999999, 2.5) [a] 
(0.9412085416666666, 2.54) [a] 
(0.9412087499999999, 2.57) [a] 
(0.9412229166666666, 2.61) [a] 
(0.9414539583333332, 2.62) [a] 
(0.9414541666666666, 2.72) [a] 
(0.9414797916666666, 2.75) [a] 
(0.9417108333333333, 2.77) [a] 
(0.9417110416666666, 2.82) [a] 
(0.9417366666666667, 2.89) [a] 
(0.9417370833333333, 2.93) [a] 
(0.9417733333333332, 3.03) [a] 
(0.941809375, 3.04) [a] 
(0.9418302083333333, 3.19) [a] 
(0.941884375, 3.22) [a] 
(0.941890625, 3.29) [a] 
(0.9418916666666667, 3.46) [a] 
(0.941905625, 4.06) [a] 
(0.9419185416666667, 4.2) [a] 
(0.9419504166666667, 4.58) [a] 
(0.9420141666666667, 4.67) [a] 
(0.9420145833333333, 4.8) [a] 
(0.9420354166666667, 4.85) [a] 
(0.9420358333333333, 4.95) [a] 
(0.9420677083333333, 5.16) [a] 
(0.9420681249999999, 6.24) [a] 
(0.9420683333333333, 6.25) [a] 
(0.9420691666666666, 6.26) [a] 
(0.942069375, 6.28) [a] 
(0.9420697916666666, 6.31) [a] 
(0.9420704166666666, 6.33) [a] 
(0.9420708333333332, 6.34) [a] 
(0.9420714583333332, 6.37) [a] 
(0.9420716666666665, 6.57) [a] 
(0.9420727083333332, 6.73) [a] 
(0.9420731249999998, 7.7) [a] 
(0.9420987499999999, 7.73) [a] 
(0.9421020833333332, 8.02) [a] 
(0.9421277083333333, 8.05) [a] 
(0.942140625, 8.49) [a] 
(0.9421470833333333, 8.55) [a] 
(0.9421679166666667, 8.77) [a] 
(0.94218875, 8.78) [a] 
(0.942189375, 10.15) [a] 
(0.942215, 10.3) [a] 
(0.9422410416666667, 10.68) [a] 
(0.9422541666666667, 10.69) [a] 
(0.9422797916666668, 10.72) [a] 
(0.9422862500000001, 10.73) [a] 
(0.9423125000000001, 11.48) [a] 
(0.9424102083333334, 11.5) [a] 
(0.9426447916666667, 11.51) [a] 
(0.9426450000000001, 11.67) [a] 
(0.9426454166666667, 11.69) [a] 
(0.9427093750000001, 12.08) [a] 
(0.9428050000000001, 12.1) [a] 
(0.9428056250000001, 12.45) [a] 
(0.9428060416666667, 12.82) [a] 
(0.9428127083333334, 12.96) [a] 
(0.9428191666666668, 12.97) [a] 
(0.9428195833333334, 13.62) [a] 
(0.9429183333333334, 13.65) [a] 
(0.9429189583333334, 13.66) [a] 
(0.9429204166666667, 13.7) [a] 
(0.9429206250000001, 13.75) [a] 
(0.9429214583333334, 13.76) [a] 
(0.9429220833333334, 13.8) [a] 
(0.9429225, 13.85) [a] 
(0.9431566666666668, 13.91) [a] 
(0.9432743750000001, 13.94) [a] 
(0.9432810416666668, 13.96) [a] 
(0.9432814583333334, 14.03) [a] 
(0.9432816666666668, 14.08) [a] 
(0.9433010416666668, 14.29) [a] 
(0.9433020833333335, 15.04) [a] 
(0.9433025000000002, 15.73) [a] 
(0.9433089583333335, 16.11) [a] 
(0.9433093750000001, 16.48) [a] 
(0.9433160416666668, 16.86) [a] 
(0.9433170833333335, 17.31) [a] 
(0.9433175000000001, 17.44) [a] 
(0.9434345833333335, 17.79) [a] 
(0.9434350000000001, 18.05) [a] 
(0.9434558333333335, 18.24) [a] 
(0.9434562500000001, 18.3) [a] 
(0.9434770833333335, 18.51) [a] 
(0.9434979166666668, 18.75) [a] 
(0.9434987500000002, 19.19) [a] 
(0.9435318750000001, 19.5) [a] 
(0.9435325000000001, 19.73) [a] 
(0.9435331250000001, 20) [a] 
(0.9435335416666667, 20.05) [a] 
(0.94354, 21.19) [a] 
(0.9436358333333335, 22.66) [a] 
(0.9436677083333335, 22.67) [a] 
(0.9436995833333335, 22.68) [a] 
(0.9437633333333335, 22.7) [a] 
(0.9437954166666669, 22.71) [a] 
(0.9438272916666669, 22.89) [a] 
(0.944139791666667, 23.41) [a] 
(0.9441814583333337, 23.43) [a] 
(0.9442231250000004, 23.44) [a] 
(0.9442775000000004, 24.55) [a] 
(0.9444402083333339, 24.57) [a] 
(0.9444672916666672, 24.94) [a] 
(0.9444991666666672, 25.01) [a] 
(0.9445208333333339, 25.15) [a] 
(0.9445537500000006, 25.16) [a] 
(0.944555208333334, 25.17) [a] 
(0.9446093750000006, 26.06) [a] 
(0.9446366666666673, 26.07) [a] 
(0.9446637500000007, 26.21) [a] 
(0.944690833333334, 26.23) [a] 
(0.9447179166666674, 26.35) [a] 
(0.9447387500000007, 26.54) [a] 
(0.9447595833333341, 26.55) [a] 
(0.9447914583333341, 26.61) [a] 
(0.9447922916666674, 26.78) [a] 
(0.9449091666666675, 27.14) [a] 
(0.9449362500000008, 27.29) [a] 
(0.9449635416666675, 27.3) [a] 
(0.9450806250000009, 27.33) [a] 
(0.9451125000000009, 27.69) [a] 
(0.9451395833333343, 27.73) [a] 
(0.9451714583333343, 28.68) [a] 
(0.9451720833333342, 30.27) [a] 
(0.9451725000000009, 30.41) [a] 
(0.9451729166666675, 30.42) [a] 
(0.9451731250000008, 30.48) [a] 
(0.9451739583333341, 30.62) [a] 
(0.9451741666666674, 30.63) [a] 
(0.945174583333334, 30.64) [a] 
(0.9451750000000007, 30.69) [a] 
(0.9451760416666674, 30.72) [a] 
(0.9451762500000007, 30.76) [a] 
(0.9451766666666673, 30.78) [a] 
(0.9452037500000007, 30.8) [a] 
(0.9452041666666673, 31.05) [a] 
(0.9452250000000006, 31.06) [a] 
(0.9452254166666673, 31.12) [a] 
(0.9453295833333341, 31.66) [a] 
(0.9453504166666674, 31.73) [a] 
(0.9453506250000008, 32.63) [a] 
(0.9453510416666674, 33.49) [a] 
(0.9453927083333341, 33.64) [a] 
(0.9453931250000007, 34.14) [a] 
(0.9454252083333341, 34.34) [a] 
(0.9454570833333341, 34.74) [a] 
(0.9454766666666674, 38.36) [a] 
(0.9455037500000008, 39.46) [a] 
(0.9455245833333341, 40.01) [a] 
(0.9455250000000007, 42.81) [a] 
(0.9455252083333341, 42.97) [a] 
(0.9455570833333341, 44.98) [a] 
(0.9455604166666675, 45.9) [a] 
(0.9455812500000008, 46.88) [a] 
(0.9456020833333342, 46.97) [a] 
(0.9456025000000008, 47.54) [a] 
(0.9456027083333342, 47.64) [a] 
(0.9456031250000008, 47.68) [a] 
(0.9456035416666674, 47.73) [a] 
(0.9456037500000007, 47.76) [a] 
(0.9456045833333341, 48.52) [a] 
(0.9456050000000007, 49.21) [a] 
(0.9456056250000007, 50.92) [a] 
(0.9456362500000006, 52.21) [a] 
(0.9456366666666672, 52.47) [a] 
(0.9456372916666672, 52.63) [a] 
(0.9456581250000006, 53.99) [a] 
(0.9456585416666672, 54.87) [a] 
(0.9456706250000005, 56.52) [a] 
(0.9456710416666672, 60.6) [a] 
(0.9456712500000005, 60.69) [a] 
(0.9456777083333339, 61.63) [a] 
(0.9456779166666672, 63.11) [a] 
(0.9457612500000007, 64.45) [a] 
(0.945782083333334, 64.47) [a] 
(0.9458029166666674, 64.49) [a] 
(0.9458041666666673, 65.06) [a] 
(0.9458104166666673, 65.07) [a] 
(0.945814583333334, 65.08) [a] 
(0.9458154166666674, 65.09) [a] 
(0.9458181250000006, 65.22) [a] 
(0.9459304166666672, 66.22) [a] 
(0.9460006250000005, 66.23) [a] 
(0.9460010416666671, 66.61) [a] 
(0.9460150000000004, 66.64) [a] 
(0.9460431250000004, 66.66) [a] 
(0.9460437500000004, 66.78) [a] 
(0.9460577083333337, 66.88) [a] 
(0.9460583333333337, 67.02) [a] 
(0.9460587500000003, 67.06) [a] 
(0.9460870833333336, 67.07) [a] 
(0.9460875000000002, 67.09) [a] 
(0.9460881250000002, 67.23) [a] 
(0.9460885416666668, 67.25) [a] 
(0.9461447916666668, 67.66) [a] 
(0.94615875, 67.7) [a] 
(0.9461729166666667, 67.72) [a] 
(0.946186875, 67.77) [a] 
(0.946215, 67.96) [a] 
(0.9462289583333333, 68.66) [a] 
(0.9462429166666666, 68.68) [a] 
(0.9462714583333333, 68.69) [a] 
(0.9462716666666666, 68.7) [a] 
(0.9462858333333333, 68.71) [a] 
(0.9463137499999998, 68.86) [a] 
(0.9463279166666665, 68.97) [a] 
(0.9463283333333331, 70.66) [a] 
(0.9464454166666665, 72.09) [a] 
(0.9464870833333332, 72.17) [a] 
(0.9464874999999998, 76.11) [a] 
(0.9464877083333332, 76.45) [a] 
(0.9464881249999998, 76.82) [a] 
(0.9464916666666665, 78.64) [a] 
(0.9465037499999999, 86.61) [a] 
(0.9465158333333332, 86.64) [a] 
(0.9465164583333332, 87.11) [a] 
(0.9465168749999998, 87.16) [a] 
(0.9465172916666664, 87.18) [a] 
(0.9465293749999998, 87.21) [a] 
(0.9465295833333331, 87.22) [a] 
(0.9465304166666665, 87.3) [a] 
(0.9465310416666665, 87.31) [a] 
(0.9465314583333331, 87.35) [a] 
(0.9465316666666664, 87.36) [a] 
(0.9465679166666665, 87.37) [a] 
(0.9465683333333331, 87.49) [a] 
(0.9465687499999997, 88.85) [a] 
(0.9465960416666664, 90.3) [a] 
(0.9465964583333331, 91.96) [a] 
(0.946597083333333, 91.97) [a] 
(0.9465974999999996, 92.02) [a] 
(0.9465979166666663, 92.05) [a] 
(0.9465981249999996, 92.06) [a] 
(0.9465985416666662, 92.09) [a] 
(0.9465989583333329, 92.41) [a] 
(0.9465991666666662, 92.44) [a] 
(0.9465995833333328, 93.44) [a] 
(0.9465999999999994, 96.71) [a] 
(0.9466318749999995, 97.18) [a] 
(0.9468002083333327, 98.36) [a] 
(0.9468143749999993, 98.43) [a] 
(0.9468147916666659, 98.67) [a] 
(0.9468708333333327, 98.7) [a] 
(0.9468849999999993, 98.72) [a] 
(0.9469131249999992, 99.97) [a] 
(0.9469410416666659, 100.12) [a] 
(0.946969166666666, 100.14) [a] 
(0.946969791666666, 107.75) [a] 
(0.9469702083333326, 107.79) [a] 
(0.9469706249999992, 107.8) [a] 
(0.9469708333333325, 107.83) [a] 
(0.9469712499999992, 107.84) [a] 
(0.9469718749999991, 107.85) [a] 
(0.9469722916666657, 107.9) [a] 
(0.9469729166666657, 107.91) [a] 
(0.9469737499999991, 107.92) [a] 
(0.9470279166666657, 111.06) [a] 
(0.9470549999999991, 111.13) [a] 
(0.9470806249999991, 114.23) [a] 
(0.9471833333333325, 114.33) [a] 
(0.9472089583333325, 115.91) [a] 
(0.9473116666666659, 116.01) [a] 
(0.9473118749999992, 118.02) [a] 
(0.9473183333333326, 118.54) [a] 
(0.9473187499999992, 119.66) [a] 
(0.9473220833333326, 119.82) [a] 
(0.9473339583333326, 120.25) [a] 
(0.9473756249999993, 125.13) [a] 
(0.9473964583333326, 125.14) [a] 
(0.947417291666666, 125.64) [a] 
(0.947468541666666, 125.71) [a] 
(0.9474824999999993, 126.47) [a] 
(0.9474858333333327, 126.63) [a] 
(0.947489166666666, 126.74) [a] 
(0.9475404166666661, 127.54) [a] 
(0.9475674999999995, 128.14) [a] 
(0.9475677083333328, 132.47) [a] 
(0.9475679166666662, 134.68) [a] 
(0.9475683333333328, 134.74) [a] 
(0.9475687499999994, 134.78) [a] 
(0.9475727083333327, 138.88) [a] 
(0.9475729166666661, 138.89) [a] 
(0.9475731249999995, 139.53) [a] 
(0.9475862499999995, 146) [a] 
(0.9475927083333329, 146.04) [a] 
(0.9475931249999995, 148.58) [a] 
(0.9475933333333328, 148.6) [a] 
(0.9476206249999996, 150.53) [a] 
(0.9476477083333329, 150.54) [a] 
(0.9476481249999995, 152.21) [a] 
(0.9476485416666661, 152.26) [a] 
(0.9476487499999995, 153) [a] 
(0.9476495833333328, 153.02) [a] 
(0.9476497916666662, 154.22) [a] 
(0.9476502083333328, 154.24) [a] 
(0.9476506249999994, 154.26) [a] 
(0.9476508333333328, 156.26) [a] 
(0.9476510416666661, 159.85) [a] 
(0.9476514583333328, 163.66) [a] 
(0.9476518749999994, 166.62) [a] 
(0.947652291666666, 167.15) [a] 
(0.9476731249999993, 167.58) [a] 
(0.9476733333333327, 168.25) [a] 
(0.947674166666666, 168.74) [a] 
(0.9476949999999994, 169.64) [a] 
(0.9476970833333327, 186.71) [a] 
(0.9476974999999993, 186.77) [a] 
(0.9476985416666659, 187.25) [a] 
(0.9476989583333325, 187.36) [a] 
(0.9476991666666659, 187.37) [a] 
(0.9476995833333325, 187.71) [a] 
(0.9476999999999991, 187.87) [a] 
(0.9477004166666657, 188.09) [a] 
(0.9477047916666657, 196.26) [a] 
(0.947706249999999, 196.27) [a] 
(0.9477077083333323, 196.99) [a] 
(0.947709374999999, 200.36) [a] 
(0.9477099999999989, 200.52) [a] 
(0.9477104166666656, 200.92) [a] 
(0.9477114583333323, 202.43) [a] 
(0.947712499999999, 219.69) [a] 
(0.9477129166666656, 219.73) [a] 
(0.9477131249999989, 219.76) [a] 
(0.9477141666666656, 220.11) [a] 
(0.9477164583333323, 221.75) [a] 
(0.9477420833333323, 223.5) [a] 
(0.9477677083333323, 226.01) [a] 
(0.9477885416666657, 241.27) [a] 
(0.9478093749999991, 241.85) [a] 
(0.947810624999999, 297.43) [a] 
(0.9478247916666657, 297.73) [a] 
(0.947838749999999, 299.6) [a] 
(0.9478808333333323, 299.71) [a] 
(0.9478949999999989, 299.73) [a] 
(0.9479420833333322, 309) [a] 
(0.9479654166666656, 309.01) [a] 
(0.9479889583333322, 309.11) [a] 
(0.9480124999999988, 315.77) [a] 
(0.9480127083333322, 328.3) [a] 
(0.9480137499999989, 365.04) [a] 
(0.9480168749999989, 365.5) [a] 
(0.9480231249999989, 365.51) [a] 
(0.9480252083333323, 365.54) [a] 
(0.9480283333333323, 365.56) [a] 
(0.948029374999999, 365.69) [a] 
(0.9480304166666657, 369.66) [a] 
(0.9480314583333324, 369.81) [a] 
(0.948031874999999, 392.96) [a] 
(0.9480322916666656, 393) [a] 
(0.9480329166666656, 413.58) [a] 
(0.9480335416666655, 413.79) [a] 
(0.9480345833333322, 413.8) [a] 
(0.9480349999999989, 414.82) [a] 
(0.9480354166666655, 421.67) [a] 
(0.9480447916666654, 427.97) [a] 
(0.9480541666666655, 427.98) [a] 
(0.9480552083333322, 428.01) [a] 
(0.9480562499999989, 428.02) [a] 
(0.9480583333333323, 428.3) [a] 
(0.948062499999999, 428.33) [a] 
(0.9480656249999991, 428.6) [a] 
(0.9480660416666657, 429.89) [a] 
(0.9480670833333324, 432.05) [a] 
(0.9480672916666658, 432.22) [a] 
(0.9480677083333324, 432.44) [a] 
(0.948068124999999, 432.45) [a] 
(0.9480685416666657, 433.63) [a] 
(0.948068749999999, 443.97) [a] 
(0.948069374999999, 461.02) [a] 
(0.9480697916666656, 461.03) [a] 
(0.9480718749999989, 461.07) [a] 
(0.9480720833333323, 461.08) [a] 
(0.948073124999999, 461.17) [a] 
(0.9480735416666656, 461.18) [a] 
(0.9480739583333322, 461.2) [a] 
(0.9480741666666656, 461.74) [a] 
(0.9480745833333322, 485.69) [a] 
(0.9480749999999988, 485.74) [a] 
(0.9480752083333321, 485.77) [a] 
(0.9480772916666654, 485.83) [a] 
(0.9480783333333321, 485.85) [a] 
(0.9480791666666655, 485.94) [a] 
(0.9480793749999988, 485.95) [a] 
(0.9480814583333321, 486.13) [a] 
(0.9480818749999987, 487.03) [a] 
(0.9481987499999988, 491.78) [a] 
(0.9484099999999986, 501.64) [a] 
(0.9484335416666653, 501.66) [a] 
(0.9484806249999985, 501.83) [a] 
(0.9485039583333319, 502.21) [a] 
(0.9485274999999985, 502.23) [a] 
(0.9485510416666652, 502.24) [a] 
(0.9485745833333318, 502.33) [a] 
(0.9485979166666652, 502.34) [a] 
(0.9486449999999985, 502.35) [a] 
(0.9486918749999985, 503) [a] 
(0.9487154166666651, 503.1) [a] 
(0.9487389583333318, 503.12) [a] 
(0.9487624999999984, 506.58) [a] 
(0.9487858333333318, 509.98) [a] 
(0.9488093749999984, 510.55) [a] 
(0.9489264583333318, 518.25) [a] 
(0.9489318749999984, 525.28) [a] 
(0.9489327083333318, 525.3) [a] 
(0.9489345833333317, 525.32) [a] 
(0.9489381249999984, 525.45) [a] 
(0.948961666666665, 525.88) [a] 
(0.9489733333333317, 533.18) [a] 
(0.948975416666665, 533.19) [a] 
(0.9489987499999983, 533.74) [a] 
(0.9490002083333317, 581.72) [a] 
(0.949001666666665, 581.73) [a] 
(0.9490064583333316, 581.74) [a] 
(0.9490535416666649, 605.87) [a] 
(0.9490770833333315, 605.88) [a] 
(0.9491004166666649, 605.96) [a] 
(0.9491239583333315, 606.08) [a] 
(0.9491474999999981, 618.46) [a] 
(0.9491479166666648, 625.68) [a] 
(0.9491483333333314, 629.38) [a] 
(0.949148749999998, 638.95) [a] 
(0.9491491666666646, 638.96) [a] 
(0.949149374999998, 639.06) [a] 
(0.9491497916666646, 641.65) [a] 
(0.9491502083333312, 648.27) [a] 
(0.9491504166666646, 648.95) [a] 
(0.9491506249999979, 659.04) [a] 
(0.9491510416666645, 659.08) [a] 
(0.9491514583333311, 659.13) [a] 
(0.9491516666666645, 659.14) [a] 
(0.9491520833333311, 659.23) [a] 
(0.9491524999999977, 661.22) [a] 
(0.9491527083333311, 661.24) [a] 
(0.9491797916666644, 674.76) [a] 
(0.9492068749999978, 683.91) [a] 
(0.9492072916666644, 685.06) [a] 
(0.949207708333331, 685.07) [a] 
(0.9492081249999976, 696.32) [a] 
(0.9492087499999976, 696.33) [a] 
(0.9492091666666642, 696.42) [a] 
(0.9492095833333308, 696.64) [a] 
(0.9492102083333308, 696.65) [a] 
(0.9492106249999974, 700.87) [a] 
(0.9492112499999974, 728.27) [a] 
(0.949211666666664, 728.35) [a] 
(0.949212291666664, 728.36) [a] 
(0.9492131249999973, 728.59) [a] 
(0.9492133333333307, 748.4) [a] 
(0.949213541666664, 775.79) [a] 
(0.9492137499999974, 782.65) [a] 
(0.9492152083333307, 802.84) [a] 
(0.949216666666664, 802.91) [a] 
(0.9492170833333307, 867.98) [a] 
(0.9492174999999973, 867.99) [a] 
(0.9492185416666639, 868) [a] 
(0.9492187499999972, 868.04) [a] 
(0.9492191666666638, 868.06) [a] 
(0.9492195833333305, 872.87) [a] 
(0.9493366666666638, 880.02) [a] 
(0.9493372916666638, 882.54) [a] 
(0.9493377083333304, 882.55) [a] 
(0.9493379166666638, 882.56) [a] 
(0.9493383333333304, 882.65) [a] 
(0.949338749999997, 886.65) [a] 
(0.949339374999997, 886.73) [a] 
(0.9493397916666636, 886.74) [a] 
(0.949339999999997, 886.75) [a] 
(0.9493404166666636, 890.84) [a] 
(0.9493408333333302, 897.44) [a] 
(0.9493414583333302, 897.46) [a] 
(0.9493418749999968, 897.48) [a] 
(0.9493420833333301, 897.5) [a] 
(0.9493424999999968, 900.92) [a] 
(0.9493429166666634, 900.93) [a] 
(0.9493435416666633, 900.94) [a] 
(0.94934395833333, 900.95) [a] 
(0.9493441666666633, 918.25) [a] 
(0.9493712499999967, 993.31) [a] 
(0.9493985416666634, 993.52) [a] 
(0.9494068749999968, 1008.9) [a] 
(0.9494143749999968, 1009) [a] 
(0.94942395833333, 1009.6) [a] 
(0.9494274999999968, 1009.9) [a] 
(0.9494281249999967, 1010.3) [a] 
(0.9494285416666634, 1013.1) [a] 
(0.94942895833333, 1013.3) [a] 
(0.9494593749999967, 1016.5) [a] 
(0.94947895833333, 1055.2) [a] 
(0.9494985416666633, 1056.5) [a] 
(0.9494989583333299, 1058.9) [a] 
(0.9494993749999965, 1061.1) [a] 
(0.9495385416666632, 1063.2) [a] 
(0.9495389583333298, 1064.3) [a] 
(0.9495454166666631, 1064.7) [a] 
(0.9495456249999965, 1099.6) [a] 
(0.9495470833333297, 1130) [a] 
(0.9495481249999963, 1130.1) [a] 
(0.9495483333333297, 1130.9) [a] 
(0.9495487499999963, 1131.4) [a] 
(0.9495493749999963, 1132) [a] 
(0.9495497916666629, 1139.1) [a] 
(0.9495502083333295, 1157.4) [a] 
(0.9495529166666627, 1177.5) [a] 
(0.9495535416666627, 1177.6) [a] 
(0.9495581249999959, 1177.7) [a] 
(0.9495587499999959, 1178) [a] 
(0.9495606249999959, 1178.4) [a] 
(0.9495608333333293, 1180.1) [a] 
(0.9495622916666626, 1225.2) [a] 
(0.9495629166666626, 1339.9) [a] 
(0.9495633333333292, 1341.1) [a] 
(0.9495841666666626, 1385.6) [a] 
(0.9495845833333292, 1398.6) [a] 
(0.9495914583333291, 1399.2) [a] 
(0.949606458333329, 1399.3) [a] 
(0.9496085416666623, 1400.7) [a] 
(0.9496162499999956, 1400.8) [a] 
(0.949619583333329, 1410.2) [a] 
(0.9496431249999956, 1422.8) [a] 
(0.949666458333329, 1434.5) [a] 
(0.9496672916666623, 1439.2) [a] 
(0.9496677083333289, 1770) [a] 
(0.9496679166666623, 1793.9) [a] 
(0.9496681249999956, 1933.1) [a] 
(0.949668333333329, 1933.5) [a] 
(0.9496687499999956, 1988.9) [a] 
(0.949668958333329, 1989.1) [a] 
(0.9496691666666623, 1989.7) [a] 
(0.9496756249999957, 2014.9) [a] 
(0.9497279166666625, 2048.9) [a] 
(0.9497345833333292, 2052.1) [a] 
(0.9497410416666625, 2053.6) [a] 
(0.9497412499999959, 2054.1) [a] 
(0.9497416666666625, 2080.6) [a] 
(0.9497483333333292, 2327.1) [a] 
(0.9497547916666625, 2327.2) [a] 
(0.9497549999999959, 2471.8) [a] 
(0.9497554166666625, 2504.6) [a] 
(0.9497556249999959, 2536.5) [a] 
(0.9497558333333292, 2673.9) [a] 
(0.9497689583333292, 3397.5) [a] 
(0.9497754166666625, 3405.6) [a] 
(0.9498012499999958, 3422.7) [a] 
(0.9498524999999958, 3422.9) [a] 
(0.9499293749999959, 3424) [a] 
(0.9499808333333292, 3424.3) [a] 
(0.9500064583333292, 3425.3) [a] 
(0.9500299999999958, 3431.4) [a] 
(0.9500558333333291, 3499.9) [a] 
(0.950107083333329, 3500.2) [a] 
(0.9501839583333291, 3501.2) [a] 
(0.9502354166666624, 3501.5) [a] 
(0.9502610416666625, 3502.6) [a] 
(0.9502616666666625, 3575) [a] 
},{(0.8320577141972472, 0) [b] 
(0.8785612655830084, 0.001) [b] 
(0.88766039577434, 0.002) [b] 
(0.8904697129059341, 0.003) [b] 
(0.8933552480410565, 0.004) [b] 
(0.8952262991649315, 0.005) [b] 
(0.8964664304149172, 0.006) [b] 
(0.8973171179255857, 0.007) [b] 
(0.897544470024074, 0.008) [b] 
(0.8979746672205384, 0.009) [b] 
(0.8985733898382936, 0.01) [b] 
(0.8989129789102164, 0.011) [b] 
(0.8994975094264307, 0.013) [b] 
(0.9005339578435448, 0.014) [b] 
(0.9014228778684537, 0.015) [b] 
(0.9015344674582934, 0.016) [b] 
(0.9019498619270643, 0.017) [b] 
(0.9020206952603976, 0.018) [b] 
(0.9023516090652087, 0.019) [b] 
(0.902663326289818, 0.02) [b] 
(0.9026719404486476, 0.021) [b] 
(0.9027321497785713, 0.023) [b] 
(0.9031447978902969, 0.024) [b] 
(0.903295930223186, 0.025) [b] 
(0.9032987102249652, 0.027) [b] 
(0.9033221976357131, 0.028) [b] 
(0.9033628226357131, 0.03) [b] 
(0.9033659476357131, 0.031) [b] 
(0.903699486223562, 0.032) [b] 
(0.9040586816258608, 0.034) [b] 
(0.9041665555142061, 0.036) [b] 
(0.9043073212799718, 0.037) [b] 
(0.9044480870457375, 0.038) [b] 
(0.9048416681268187, 0.039) [b] 
(0.9048472281303771, 0.04) [b] 
(0.905040129364945, 0.045) [b] 
(0.905064242019266, 0.048) [b] 
(0.9050906331321564, 0.049) [b] 
(0.9051041747988231, 0.05) [b] 
(0.9051392442432675, 0.051) [b] 
(0.9051457750687639, 0.052) [b] 
(0.9059395272927653, 0.053) [b] 
(0.9059402217372097, 0.054) [b] 
(0.905947166181654, 0.056) [b] 
(0.9060497934394537, 0.057) [b] 
(0.9060837808054567, 0.058) [b] 
(0.9061864080632563, 0.06) [b] 
(0.9062098954740042, 0.062) [b] 
(0.9062515621406709, 0.068) [b] 
(0.9063135661089249, 0.071) [b] 
(0.906326398466999, 0.072) [b] 
(0.906388402435253, 0.074) [b] 
(0.9069525545951498, 0.075) [b] 
(0.9069589707741869, 0.076) [b] 
(0.9070464707741869, 0.077) [b] 
(0.9072591798028338, 0.08) [b] 
(0.9074091798028338, 0.083) [b] 
(0.9075911242472782, 0.084) [b] 
(0.9076451143422622, 0.09) [b] 
(0.9076646700368806, 0.092) [b] 
(0.9076917967729917, 0.099) [b] 
(0.9078092736248435, 0.107) [b] 
(0.9078333862791645, 0.11) [b] 
(0.9080017890569423, 0.111) [b] 
(0.9082402873208312, 0.112) [b] 
(0.9083982360111706, 0.113) [b] 
(0.9084620440765101, 0.114) [b] 
(0.9084943608746443, 0.115) [b] 
(0.908495048466094, 0.116) [b] 
(0.9103127300436626, 0.119) [b] 
(0.910400298575043, 0.12) [b] 
(0.9104564531123296, 0.121) [b] 
(0.91060930319754, 0.122) [b] 
(0.9106310498920424, 0.123) [b] 
(0.9106615525762786, 0.125) [b] 
(0.9106650247985008, 0.127) [b] 
(0.9107407192429452, 0.131) [b] 
(0.91157404626678, 0.137) [b] 
(0.9117136296001133, 0.139) [b] 
(0.9118345193486626, 0.144) [b] 
(0.9118411165708848, 0.148) [b] 
(0.9118551552052065, 0.15) [b] 
(0.9119061968718732, 0.151) [b] 
(0.9119065440940954, 0.153) [b] 
(0.9119225163163176, 0.156) [b] 
(0.9119818705423385, 0.157) [b] 
(0.9121058288756718, 0.158) [b] 
(0.9122256205423385, 0.159) [b] 
(0.9122311760978941, 0.16) [b] 
(0.9124468672354602, 0.161) [b] 
(0.9126197839021268, 0.165) [b] 
(0.9126673533465712, 0.166) [b] 
(0.9128395755687935, 0.167) [b] 
(0.9128878394576824, 0.168) [b] 
(0.9130048533465712, 0.169) [b] 
(0.9132017283465712, 0.17) [b] 
(0.913993395013238, 0.171) [b] 
(0.9141385339021268, 0.172) [b] 
(0.9142065894576824, 0.173) [b] 
(0.9145982561243491, 0.174) [b] 
(0.9146645755687934, 0.175) [b] 
(0.9150034644576823, 0.176) [b] 
(0.9153340200132379, 0.177) [b] 
(0.9154531172354602, 0.178) [b] 
(0.9154670061243491, 0.179) [b] 
(0.91553436723546, 0.18) [b] 
(0.9155392283465711, 0.181) [b] 
(0.9159600616799044, 0.182) [b] 
(0.9160819366799045, 0.183) [b] 
(0.9165173533465711, 0.184) [b] 
(0.9165677005687933, 0.185) [b] 
(0.91666874223546, 0.186) [b] 
(0.9167114505687933, 0.187) [b] 
(0.9169638811243489, 0.188) [b] 
(0.917082957572592, 0.189) [b] 
(0.9171270547948143, 0.19) [b] 
(0.9171277492392587, 0.191) [b] 
(0.9173228881281476, 0.192) [b] 
(0.9176699069711457, 0.193) [b] 
(0.9176956014155901, 0.194) [b] 
(0.9177004625267012, 0.195) [b] 
(0.9177008097489234, 0.196) [b] 
(0.9180181708600345, 0.197) [b] 
(0.9181469903044789, 0.198) [b] 
(0.9182167819711455, 0.199) [b] 
(0.9183702541933677, 0.201) [b] 
(0.9183766703724048, 0.203) [b] 
(0.9189995870390715, 0.205) [b] 
(0.9190311842612937, 0.207) [b] 
(0.9192995870390714, 0.209) [b] 
(0.9194193787057381, 0.211) [b] 
(0.9201124718714376, 0.214) [b] 
(0.9206694163158821, 0.217) [b] 
(0.9207476390943552, 0.219) [b] 
(0.9209729267803516, 0.221) [b] 
(0.9210091491416366, 0.223) [b] 
(0.9211278575936783, 0.226) [b] 
(0.9211754270381227, 0.228) [b] 
(0.9211785520381227, 0.229) [b] 
(0.9212219548159005, 0.231) [b] 
(0.9212795937047894, 0.232) [b] 
(0.9212927881492339, 0.234) [b] 
(0.9213297070125072, 0.243) [b] 
(0.9213300542347294, 0.248) [b] 
(0.9213413315248775, 0.249) [b] 
(0.9214006857508984, 0.256) [b] 
(0.9214071019299355, 0.265) [b] 
(0.9214091852632688, 0.27) [b] 
(0.9217653106193942, 0.272) [b] 
(0.9225311687221159, 0.286) [b] 
(0.9225546561328638, 0.289) [b] 
(0.9225997950217527, 0.297) [b] 
(0.9226591492477736, 0.303) [b] 
(0.9226910532804433, 0.315) [b] 
(0.9227867653784525, 0.316) [b] 
(0.9227881889065245, 0.33) [b] 
(0.9227981536030283, 0.332) [b] 
(0.9228686158352718, 0.335) [b] 
(0.922899118519508, 0.336) [b] 
(0.9229601238879804, 0.337) [b] 
(0.9231017376833277, 0.339) [b] 
(0.9231627430518001, 0.34) [b] 
(0.9232908680518002, 0.347) [b] 
(0.9234545737800771, 0.352) [b] 
(0.9234574208362211, 0.357) [b] 
(0.9235184262046935, 0.361) [b] 
(0.9235343984269158, 0.366) [b] 
(0.9236468571446017, 0.368) [b] 
(0.9236705045150142, 0.375) [b] 
(0.9236738827107874, 0.39) [b] 
(0.923684017298107, 0.391) [b] 
(0.9236957063527513, 0.401) [b] 
(0.9239689633886454, 0.402) [b] 
(0.9241272967219787, 0.408) [b] 
(0.9241321098621263, 0.409) [b] 
(0.924132797453576, 0.412) [b] 
(0.924136923002274, 0.423) [b] 
(0.9241390063356073, 0.431) [b] 
(0.9241636320251266, 0.463) [b] 
(0.9241967008081954, 0.465) [b] 
(0.924245952187234, 0.468) [b] 
(0.9242705778767534, 0.478) [b] 
(0.9243132591218283, 0.483) [b] 
(0.9243378848113476, 0.488) [b] 
(0.9243625105008669, 0.493) [b] 
(0.9243833438342003, 0.503) [b] 
(0.9246184678978707, 0.528) [b] 
(0.9246215928978707, 0.544) [b] 
(0.9246247178978707, 0.548) [b] 
(0.9249492822982807, 0.552) [b] 
(0.924949976742725, 0.556) [b] 
(0.9249717234372274, 0.56) [b] 
(0.9249727651038941, 0.584) [b] 
(0.9249769317705608, 0.588) [b] 
(0.9249800567705608, 0.607) [b] 
(0.9249821401038941, 0.611) [b] 
(0.9249831817705608, 0.616) [b] 
(0.9250133901038942, 0.625) [b] 
(0.925397742683327, 0.628) [b] 
(0.9254112843499936, 0.631) [b] 
(0.9254154510166603, 0.652) [b] 
(0.9255336038466342, 0.667) [b] 
(0.9255367288466342, 0.675) [b] 
(0.9256499232910786, 0.756) [b] 
(0.9256541938752946, 0.768) [b] 
(0.9257979702105639, 0.807) [b] 
(0.9259150114090657, 0.81) [b] 
(0.9259217678006122, 0.811) [b] 
(0.9259473913059078, 0.845) [b] 
(0.925950769501681, 0.852) [b] 
(0.9260110511374835, 0.858) [b] 
(0.9260152178041502, 0.863) [b] 
(0.9260185959999234, 0.876) [b] 
(0.9260326346342451, 1.112) [b] 
(0.9260446909614056, 1.145) [b] 
(0.9260929162700476, 1.149) [b] 
(0.9262099574685495, 1.157) [b] 
(0.9262148185796606, 1.161) [b] 
(0.9262179435796606, 1.228) [b] 
(0.9262299999068211, 1.255) [b] 
(0.9262420562339816, 1.256) [b] 
(0.9262739602666513, 1.346) [b] 
(0.9262777797110958, 1.38) [b] 
(0.9262784741555402, 1.388) [b] 
(0.9263025868098612, 1.423) [b] 
(0.9263146431370217, 1.426) [b] 
(0.9263210593160588, 1.477) [b] 
(0.9263221009827255, 1.523) [b] 
(0.9266321208239953, 1.683) [b] 
(0.9267657534945105, 1.785) [b] 
(0.9267851071916197, 1.829) [b] 
(0.9268378849693975, 1.85) [b] 
(0.9268383457717095, 1.876) [b] 
(0.9268876513272651, 1.907) [b] 
(0.9269291235353561, 1.949) [b] 
(0.9313415535893437, 2.041) [b] 
(0.9313672104037936, 2.062) [b] 
(0.9313681320084178, 2.074) [b] 
(0.9313690536130421, 2.077) [b] 
(0.931447835427492, 2.104) [b] 
(0.9314513076497142, 2.165) [b] 
(0.9316061687608252, 2.236) [b] 
(0.9316700576497141, 2.258) [b] 
(0.9317634604274919, 2.261) [b] 
(0.9317738770941586, 2.318) [b] 
(0.931852696538603, 2.326) [b] 
(0.9318537382052697, 2.332) [b] 
(0.9318547638066238, 2.377) [b] 
(0.9318552033500612, 2.421) [b] 
(0.9318616195290983, 2.438) [b] 
(0.9319084945290983, 2.509) [b] 
(0.931915786195765, 2.547) [b] 
(0.9319627610172606, 2.589) [b] 
(0.9319932637014968, 2.832) [b] 
(0.9320445773303967, 2.836) [b] 
(0.9321888084540908, 2.845) [b] 
(0.9322401220829907, 2.878) [b] 
(0.932246652908487, 3.196) [b] 
(0.9322532501307091, 3.331) [b] 
(0.9322803768668203, 3.36) [b] 
(0.9323974180653222, 3.549) [b] 
(0.9324443928868178, 3.722) [b] 
(0.9324461289979289, 3.91) [b] 
(0.9324612560098214, 3.984) [b] 
(0.9324619436012711, 3.985) [b] 
(0.9324646939670698, 3.99) [b] 
(0.9324688195157678, 3.998) [b] 
(0.9324701946986671, 4.079) [b] 
(0.9324708822901168, 4.081) [b] 
(0.9324814807432955, 4.12) [b] 
(0.9325095580119388, 4.232) [b] 
(0.9325235966462605, 4.234) [b] 
(0.9325516739149038, 4.246) [b] 
(0.9325523615063535, 4.534) [b] 
(0.9326064126387247, 4.705) [b] 
(0.9326300600091371, 4.706) [b] 
(0.9326537073795494, 4.713) [b] 
(0.9326672201626423, 4.714) [b] 
(0.9326880534959756, 4.731) [b] 
(0.9326894286788749, 4.733) [b] 
(0.9327102620122083, 4.742) [b] 
(0.9327213731233194, 4.907) [b] 
(0.9327234564566527, 5.002) [b] 
(0.9327244981233194, 5.036) [b] 
(0.9327862456331437, 5.124) [b] 
(0.932795000877074, 5.148) [b] 
(0.9327959224816983, 5.164) [b] 
(0.9328019129117558, 5.174) [b] 
(0.9328086693033022, 5.264) [b] 
(0.9328120474990754, 5.266) [b] 
(0.9328154256948487, 5.268) [b] 
(0.9328221820863951, 5.277) [b] 
(0.9328225293086173, 5.31) [b] 
(0.9329837156400164, 5.703) [b] 
(0.9329870938357896, 5.952) [b] 
(0.9329877814272393, 6.35) [b] 
(0.9329941976062763, 6.67) [b] 
(0.933223364272943, 7.595) [b] 
(0.9332650309396097, 7.599) [b] 
(0.9333066976062764, 7.6) [b] 
(0.9333275309396097, 7.603) [b] 
(0.9333483642729431, 7.627) [b] 
(0.9334733642729431, 7.675) [b] 
(0.9335358642729431, 7.676) [b] 
(0.9335566976062765, 7.708) [b] 
(0.9335897663893452, 7.765) [b] 
(0.9336559039554828, 7.766) [b] 
(0.9337120584927694, 7.897) [b] 
(0.9337260971270911, 7.898) [b] 
(0.9337541743957344, 7.902) [b] 
(0.9337682130300561, 7.908) [b] 
(0.9337962902986994, 7.916) [b] 
(0.9339226380075942, 7.993) [b] 
(0.9339647539105592, 7.994) [b] 
(0.9339671844661148, 8.219) [b] 
(0.9339812231004365, 8.311) [b] 
(0.9340083498365477, 8.361) [b] 
(0.934089730044881, 8.362) [b] 
(0.9341168567809922, 8.384) [b] 
(0.9341234540032144, 8.475) [b] 
(0.9341777074754366, 8.773) [b] 
(0.9342048342115478, 8.799) [b] 
(0.9342265809060502, 9.24) [b] 
(0.9342537076421613, 9.502) [b] 
(0.9342544020866057, 9.549) [b] 
(0.9343570293444053, 9.657) [b] 
(0.9343826861588552, 9.658) [b] 
(0.9344083429733051, 9.66) [b] 
(0.934433999787755, 9.672) [b] 
(0.9344596566022049, 9.784) [b] 
(0.9344853134166548, 9.98) [b] 
(0.9345109702311046, 9.982) [b] 
(0.9346135974889043, 10.035) [b] 
(0.9346392543033542, 10.037) [b] 
(0.934664911117804, 10.039) [b] 
(0.9346905679322539, 10.051) [b] 
(0.9347162247467038, 10.171) [b] 
(0.934743351482815, 10.33) [b] 
(0.9347690082972648, 10.37) [b] 
(0.9347946651117147, 10.372) [b] 
(0.9348010812907518, 10.618) [b] 
(0.9348074974697889, 10.724) [b] 
(0.9348139136488259, 10.763) [b] 
(0.9348374010595738, 12.43) [b] 
(0.9348494573867343, 14.701) [b] 
(0.934853624053401, 16.561) [b] 
(0.934877736707722, 16.59) [b] 
(0.9348897930348825, 16.601) [b] 
(0.9348939185835805, 17.043) [b] 
(0.9348987317237282, 17.045) [b] 
(0.9348994193151778, 17.046) [b] 
(0.9349114756423383, 17.233) [b] 
(0.9349235319694988, 17.316) [b] 
(0.9349355882966593, 17.384) [b] 
(0.934938018852215, 17.406) [b] 
(0.9350620267887229, 17.82) [b] 
(0.9351860347252308, 17.823) [b] 
(0.9352480386934847, 17.861) [b] 
(0.9353100426617387, 17.899) [b] 
(0.9353431114448074, 18.236) [b] 
(0.9353750154774771, 18.426) [b] 
(0.9354069195101469, 18.427) [b] 
(0.9354689234784008, 18.519) [b] 
(0.935475520700623, 18.839) [b] 
(0.935537524668877, 19.161) [b] 
(0.9355389135577659, 20.059) [b] 
(0.9355509698849264, 21.396) [b] 
(0.9355744572956742, 21.997) [b] 
(0.9355754989623409, 22.63) [b] 
(0.9355989863730888, 23.153) [b] 
(0.9356795795387883, 25.256) [b] 
(0.9356802739832327, 25.438) [b] 
(0.9357107766674689, 25.64) [b] 
(0.9357303323620872, 26.149) [b] 
(0.93573685092696, 26.784) [b] 
(0.93574310092696, 26.822) [b] 
(0.9357549064825156, 26.829) [b] 
(0.93575560092696, 26.867) [b] 
(0.9358176048952139, 27.481) [b] 
(0.9358196882285472, 27.951) [b] 
(0.935851592261217, 29.787) [b] 
(0.9358834962938867, 29.789) [b] 
(0.9358900271193831, 29.98) [b] 
(0.9359161504213684, 29.986) [b] 
(0.9359226812468648, 29.992) [b] 
(0.9359292120723612, 30.004) [b] 
(0.935981458676332, 30.011) [b] 
(0.9359879895018284, 30.065) [b] 
(0.9360020281361501, 30.5) [b] 
(0.9360301054047934, 30.504) [b] 
(0.9360367026270155, 30.547) [b] 
(0.936043233452512, 30.782) [b] 
(0.9360572720868336, 30.846) [b] 
(0.9360853493554769, 30.847) [b] 
(0.936091765534514, 32.916) [b] 
(0.9361188922706252, 33.273) [b] 
(0.936190731351085, 34.082) [b] 
(0.936229916304063, 34.089) [b] 
(0.9363469575025649, 34.529) [b] 
(0.936364318613676, 35.52) [b] 
(0.9363726519470094, 35.574) [b] 
(0.9363809852803426, 35.695) [b] 
(0.9363834158358982, 35.984) [b] 
(0.9363844575025649, 37.754) [b] 
(0.9363882769470094, 38.77) [b] 
(0.9364201809796792, 39.224) [b] 
(0.9364503893130125, 41.227) [b] 
(0.9364507365352347, 43.128) [b] 
(0.936452111718134, 43.241) [b] 
(0.9364618339403562, 46.465) [b] 
(0.9364621811625784, 46.531) [b] 
(0.9364625283848006, 47.476) [b] 
(0.9364649589403562, 47.538) [b] 
(0.9364683371361294, 47.818) [b] 
(0.936548930301829, 47.913) [b] 
(0.9365496178932786, 54.515) [b] 
(0.9365503054847283, 54.536) [b] 
(0.936550993076178, 54.918) [b] 
(0.9365645058592709, 56.755) [b] 
(0.9365780186423638, 56.762) [b] 
(0.936581396838137, 56.765) [b] 
(0.9365847750339102, 56.775) [b] 
(0.9365881532296835, 56.863) [b] 
(0.9365949096212299, 56.898) [b] 
(0.936611800600096, 56.902) [b] 
(0.9366185569916424, 56.949) [b] 
(0.9366219351874157, 57.089) [b] 
(0.9366254074096378, 57.236) [b] 
(0.9366267962985267, 58.251) [b] 
(0.9366301744943, 58.475) [b] 
(0.9366335526900732, 58.479) [b] 
(0.9366403090816195, 58.521) [b] 
(0.9366674358177307, 58.615) [b] 
(0.9366802681758049, 61.275) [b] 
(0.9366866843548419, 61.292) [b] 
(0.9366900625506152, 62.692) [b] 
(0.936715719365065, 63.661) [b] 
(0.9367560159479148, 65.107) [b] 
(0.9368855298368037, 65.816) [b] 
(0.9368914326145815, 65.883) [b] 
(0.9368917798368037, 65.991) [b] 
(0.9369015020590259, 66.158) [b] 
(0.9369042798368037, 66.386) [b] 
(0.936925113170137, 66.403) [b] 
(0.9369486005808849, 67.64) [b] 
(0.9370010311364405, 68.214) [b] 
(0.9370090172475516, 68.307) [b] 
(0.9370346740620015, 68.325) [b] 
(0.9370380522577747, 69.453) [b] 
(0.9370383994799969, 70.66) [b] 
(0.937043260591108, 70.823) [b] 
(0.9370547189244414, 72.595) [b] 
(0.9370581911466636, 73.578) [b] 
(0.937059656291455, 73.723) [b] 
(0.9371234643567945, 74.366) [b] 
(0.9371244899581486, 77.623) [b] 
(0.9371501467725984, 78.774) [b] 
(0.9371758035870483, 85.08) [b] 
(0.9371764980314927, 86.086) [b] 
(0.9372547208099659, 91.948) [b] 
(0.9372612393748386, 91.963) [b] 
(0.9372873136343297, 93.644) [b] 
(0.9373108010450776, 96.346) [b] 
(0.9373510976279273, 97.657) [b] 
(0.9373517920723717, 99.245) [b] 
(0.937357000405705, 99.75) [b] 
(0.9373653337390384, 99.755) [b] 
(0.9373663754057051, 99.763) [b] 
(0.9373715837390384, 99.771) [b] 
(0.9373736670723717, 99.936) [b] 
(0.9373767920723717, 100.018) [b] 
(0.9373823476279273, 100.3) [b] 
(0.937383389294594, 100.604) [b] 
(0.9373844309612607, 100.738) [b] 
(0.9373875559612607, 101.182) [b] 
(0.9373885976279274, 101.247) [b] 
(0.9373896392945941, 101.329) [b] 
(0.9373906809612608, 102.865) [b] 
(0.9374163377757107, 107.911) [b] 
(0.9374933082190604, 107.912) [b] 
(0.9375189650335103, 107.957) [b] 
(0.9375446218479602, 113.409) [b] 
(0.93757027866241, 113.978) [b] 
(0.9376021826950798, 115.498) [b] 
(0.9376278395095297, 117.574) [b] 
(0.9377048099528794, 117.575) [b] 
(0.9377304667673293, 117.625) [b] 
(0.9377561235817792, 123.729) [b] 
(0.9377817803962291, 124.363) [b] 
(0.9377821276184513, 127.407) [b] 
(0.9377828152099009, 129.859) [b] 
(0.9377835028013506, 129.861) [b] 
(0.9377841903928003, 129.874) [b] 
(0.9377848779842499, 129.888) [b] 
(0.9377855655756996, 130.126) [b] 
(0.9378258621585494, 130.354) [b] 
(0.9378589309416181, 132.472) [b] 
(0.9378599726082848, 136.916) [b] 
(0.937887099344396, 136.989) [b] 
(0.9378881410110627, 137.002) [b] 
(0.9379011781408082, 137.34) [b] 
(0.9379014711697665, 138.633) [b] 
(0.9381355535667703, 138.817) [b] 
(0.9381376369001035, 141.496) [b] 
(0.9381424980112146, 142.048) [b] 
(0.938145935968463, 145.551) [b] 
(0.938149060968463, 145.721) [b] 
(0.938153186517161, 146.255) [b] 
(0.9382158217523278, 147.664) [b] 
(0.9382165161967722, 148.584) [b] 
(0.9382168634189944, 150.466) [b] 
(0.9382296751716421, 155.58) [b] 
(0.9382302612295588, 186.376) [b] 
(0.9382366774085958, 199.391) [b] 
(0.9382430935876329, 201.292) [b] 
(0.9382665809983808, 203.373) [b] 
(0.9382900684091287, 203.385) [b] 
(0.9383135558198765, 203.414) [b] 
(0.9383370432306244, 204.16) [b] 
(0.9383605306413723, 206.606) [b] 
(0.9383840180521201, 207.346) [b] 
(0.9383904342311572, 211.529) [b] 
(0.9384223382638269, 217.692) [b] 
(0.9385818584271756, 217.694) [b] 
(0.9386137624598453, 217.695) [b] 
(0.938645666492515, 217.702) [b] 
(0.9386775705251847, 217.712) [b] 
(0.9387094745578545, 217.846) [b] 
(0.9387732826231939, 217.884) [b] 
(0.9388051866558637, 217.885) [b] 
(0.9388370906885334, 217.93) [b] 
(0.9389008987538728, 223.74) [b] 
(0.9389328027865426, 223.744) [b] 
(0.9389647068192123, 223.944) [b] 
(0.9389952095034485, 228.509) [b] 
(0.9390562148719209, 228.51) [b] 
(0.9390600343163654, 230.425) [b] 
(0.9390601808308445, 235.685) [b] 
(0.9390836682415924, 245.282) [b] 
(0.9390870464373656, 253.946) [b] 
(0.9390877340288153, 263.663) [b] 
(0.939088421620265, 263.67) [b] 
(0.9390891092117146, 263.812) [b] 
(0.939090484394614, 273.724) [b] 
(0.9390918595775133, 273.725) [b] 
(0.939105898211835, 283.244) [b] 
(0.9391177037673906, 283.665) [b] 
(0.9391399259896128, 283.725) [b] 
(0.9391718300222825, 287.876) [b] 
(0.9392675421202917, 287.878) [b] 
(0.9392994461529615, 287.879) [b] 
(0.9393313501856312, 287.894) [b] 
(0.9393951582509706, 287.967) [b] 
(0.9394270622836404, 287.969) [b] 
(0.9394284511725293, 291.95) [b] 
(0.9394768511269763, 295.882) [b] 
(0.9394886566825319, 303.265) [b] 
(0.9394994205714208, 303.318) [b] 
(0.9396164617699226, 303.77) [b] 
(0.939617156214367, 304.105) [b] 
(0.9396192395477003, 304.838) [b] 
(0.9396830476130398, 308.167) [b] 
(0.9397149516457095, 308.17) [b] 
(0.9397468556783792, 308.171) [b] 
(0.9397475501228236, 311.377) [b] 
(0.9397527584561569, 318.275) [b] 
(0.9397592770210297, 323.684) [b] 
(0.9397723141507752, 323.685) [b] 
(0.9399389808174419, 325.078) [b] 
(0.9403348141507752, 325.08) [b] 
(0.9403556474841086, 325.195) [b] 
(0.9404598141507753, 325.249) [b] 
(0.940501480817442, 325.25) [b] 
(0.9405223141507754, 325.252) [b] 
(0.9405431474841087, 325.256) [b] 
(0.9406056474841087, 326.346) [b] 
(0.940626480817442, 326.36) [b] 
(0.9406681474841088, 326.37) [b] 
(0.9406889808174421, 329.141) [b] 
(0.9407098141507755, 329.149) [b] 
(0.9407306474841088, 335.777) [b] 
(0.9407514808174422, 338.007) [b] 
(0.9407833848501119, 344.452) [b] 
(0.9408250515167786, 350.543) [b] 
(0.940845884850112, 350.544) [b] 
(0.9408730115862232, 358.298) [b] 
(0.9409146782528899, 361.618) [b] 
(0.9409563449195566, 361.619) [b] 
(0.9409980115862233, 361.815) [b] 
(0.9410188449195567, 362.511) [b] 
(0.9411873085314165, 363.405) [b] 
(0.9412715403373464, 363.407) [b] 
(0.9412855789716681, 363.409) [b] 
(0.9413136562403114, 363.482) [b] 
(0.9413276948746331, 363.483) [b] 
(0.9413417335089548, 363.504) [b] 
(0.9413557721432765, 363.818) [b] 
(0.9413698107775982, 363.85) [b] 
(0.9413731889733714, 364.377) [b] 
(0.9413940223067048, 365.929) [b] 
(0.9414080609410265, 366.302) [b] 
(0.9414361382096698, 367.748) [b] 
(0.9414782541126347, 369.35) [b] 
(0.9415203700155997, 369.351) [b] 
(0.9415344086499214, 369.417) [b] 
(0.9416186404558513, 379.256) [b] 
(0.941632679090173, 379.26) [b] 
(0.9416467177244947, 379.272) [b] 
(0.9416607563588164, 380.296) [b] 
(0.9417421365671498, 386.795) [b] 
(0.9417692633032609, 386.796) [b] 
(0.9417963900393721, 386.797) [b] 
(0.9418506435115943, 386.803) [b] 
(0.9418777702477055, 386.818) [b] 
(0.9419320237199277, 387.473) [b] 
(0.9419591504560388, 388.862) [b] 
(0.94198627719215, 388.875) [b] 
(0.9420003158264717, 390.194) [b] 
(0.9420143544607934, 390.199) [b] 
(0.9420283930951151, 390.265) [b] 
(0.9420564703637584, 390.939) [b] 
(0.9420985862667234, 392.617) [b] 
(0.9423326686637271, 411.769) [b] 
(0.9423394250552735, 412.336) [b] 
(0.9423428032510467, 413.418) [b] 
(0.94236363658438, 419.296) [b] 
(0.9423844699177134, 419.301) [b] 
(0.9424053032510468, 419.491) [b] 
(0.9424261365843801, 420.312) [b] 
(0.9424678032510468, 447.472) [b] 
(0.9424962738124864, 462.864) [b] 
(0.9424976489953857, 480.68) [b] 
(0.9425101489953857, 481.828) [b] 
(0.9425181351064967, 481.929) [b] 
(0.9425216073287189, 482.843) [b] 
(0.9425257739953856, 483.018) [b] 
(0.94252646843983, 486) [b] 
(0.9425473017731634, 487.715) [b] 
(0.9425681351064967, 492.27) [b] 
(0.9425684823287189, 492.746) [b] 
(0.9425698712176078, 493.95) [b] 
(0.94257021843983, 510.367) [b] 
(0.9425715936227294, 555.776) [b] 
(0.9425770943543267, 555.777) [b] 
(0.9425777819457763, 555.779) [b] 
(0.9425798447201253, 555.796) [b] 
(0.942580532311575, 555.798) [b] 
(0.9425812199030247, 555.853) [b] 
(0.9425819074944743, 555.927) [b] 
(0.9425878102722521, 578.477) [b] 
(0.9425881574944743, 579.338) [b] 
(0.942742324161141, 586.951) [b] 
(0.9427440602722521, 587.027) [b] 
(0.9427451019389188, 587.187) [b] 
(0.9427482269389188, 589.903) [b] 
(0.942748574161141, 589.996) [b] 
(0.9427510047166966, 590.213) [b] 
(0.9427513519389188, 593.69) [b] 
(0.9427836436055854, 595.638) [b] 
(0.9427940602722521, 598.346) [b] 
(0.9427975324944743, 598.407) [b] 
(0.942798574161141, 598.97) [b] 
(0.9427989213833632, 599.319) [b] 
(0.9427992686055854, 599.741) [b] 
(0.9428123302565781, 614.704) [b] 
(0.9428188610820745, 614.709) [b] 
(0.9428384535585634, 614.715) [b] 
(0.9428449843840598, 614.721) [b] 
(0.942845331606282, 622.154) [b] 
(0.9428456788285042, 631.762) [b] 
(0.9428458253429833, 658.377) [b] 
(0.943188895608329, 681.641) [b] 
(0.9431988603048328, 740.028) [b] 
(0.9432320380713055, 828.793) [b] 
(0.9432908155007727, 836.648) [b] 
(0.9433104079772616, 836.655) [b] 
(0.943316938802758, 836.896) [b] 
(0.9433176332472024, 878.04) [b] 
(0.9433402125604964, 881.884) [b] 
(0.9433467311253692, 933.911) [b] 
(0.9433532496902419, 934.169) [b] 
(0.9433597682551147, 934.177) [b] 
(0.9434768094536166, 936.037) [b] 
(0.943486878898061, 944.215) [b] 
(0.9434955594536166, 944.431) [b] 
(0.9435018094536166, 945.03) [b] 
(0.943502503898061, 1037.36) [b] 
(0.9435049344536166, 1045.87) [b] 
(0.9435073650091722, 1078.7) [b] 
(0.9435084066758389, 1084.98) [b] 
(0.9435355334119501, 1119.46) [b] 
(0.9435626601480612, 1147.98) [b] 
(0.9435827990369501, 1148.55) [b] 
(0.9435907851480612, 1148.62) [b] 
(0.9435925212591723, 1148.7) [b] 
(0.9435932157036167, 1148.76) [b] 
(0.9435942573702834, 1148.9) [b] 
(0.9435946045925055, 1149.04) [b] 
(0.9435959934813944, 1158.21) [b] 
(0.9436178684813944, 1188.31) [b] 
(0.9436737712591722, 1188.41) [b] 
(0.9436789795925055, 1189.02) [b] 
(0.9436796740369499, 1195.27) [b] 
(0.9436860902159869, 1230.55) [b] 
(0.9438875731302359, 1262.61) [b] 
(0.9439278697130856, 1263.44) [b] 
(0.9439681662959354, 1263.99) [b] 
(0.9440084628787851, 1267.99) [b] 
(0.9440132760189328, 1288.61) [b] 
(0.9440146512018321, 1289.25) [b] 
(0.9440194643419798, 1289.27) [b] 
(0.9440213075512283, 1340.63) [b] 
(0.9440247797734505, 1438.57) [b] 
(0.9440254742178948, 1447.32) [b] 
(0.9440275575512281, 1484.41) [b] 
(0.944028946440117, 1500.93) [b] 
(0.9440546032545669, 1759.21) [b] 
(0.9440802600690168, 1759.59) [b] 
(0.9441059168834667, 1761.46) [b] 
(0.9441315736979166, 1763.15) [b] 
(0.9441572305123664, 1765.47) [b] 
(0.9441575777345886, 1802.23) [b] 
(0.9441603555123664, 1802.28) [b] 
(0.9441610499568108, 1802.95) [b] 
(0.9441867067712607, 1815.52) [b] 
(0.9442123635857106, 1815.56) [b] 
(0.9442179191412662, 1885.54) [b] 
(0.9442435759557161, 1975.6) [b] 
(0.944269232770166, 1975.66) [b] 
(0.9442948895846158, 1975.82) [b] 
(0.9443205463990657, 1979.11) [b] 
(0.9443462032135156, 1981.78) [b] 
(0.9443466640158277, 2012.47) [b] 
(0.9443723208302776, 2027.94) [b] 
(0.9443979776447274, 2029.02) [b] 
(0.9444236344591773, 2031.96) [b] 
(0.9444492912736272, 2270.08) [b] 
(0.9445306714819606, 2279.91) [b] 
(0.9445447101162823, 2290.44) [b] 
(0.944558748750604, 2290.67) [b] 
(0.944600864653569, 2291.2) [b] 
(0.9446279913896801, 2296.18) [b] 
(0.9446420300240018, 2360.33) [b] 
(0.9446427244684462, 2483.51) [b] 
(0.9446444605795573, 2488.9) [b] 
(0.9446451550240017, 2493.2) [b] 
(0.9446458494684461, 2494.57) [b] 
(0.9446555716906683, 2601.7) [b] 
(0.944659738357335, 2601.71) [b] 
(0.9446611272462239, 2601.75) [b] 
(0.9446632105795572, 2601.87) [b] 
(0.9446639050240015, 2602.01) [b] 
(0.9446659883573348, 2603.22) [b] 
(0.9446687661351126, 2603.24) [b] 
(0.9446708494684459, 2603.32) [b] 
(0.9446722383573348, 2615.3) [b] 
(0.9446729328017792, 2615.42) [b] 
(0.9446736272462236, 2615.52) [b] 
(0.9446846287094182, 2676.95) [b] 
(0.9447190082819014, 2677.24) [b] 
(0.944719695873351, 2677.26) [b] 
(0.9447210710562504, 2677.41) [b] 
(0.9447231338305994, 2677.99) [b] 
(0.944723821422049, 2678) [b] 
(0.9447251966049484, 2679.93) [b] 
(0.9447561382201832, 2689.28) [b] 
(0.9447623265432302, 2689.34) [b] 
(0.9447637017261296, 2689.35) [b] 
(0.9447671396833779, 2689.43) [b] 
(0.9447705776406262, 2689.48) [b] 
(0.9447712720850706, 2720.96) [b] 
(0.944771966529515, 2758.33) [b] 
(0.9447726609739594, 2761.94) [b] 
(0.9447733554184038, 2762.07) [b] 
(0.9447740498628482, 2762.34) [b] 
(0.9447747443072926, 2762.81) [b] 
(0.9447761194901919, 3013.04) [b] 
(0.9447774946730912, 3013.1) [b] 
(0.9447781822645409, 3013.7) [b] 
(0.9447788698559906, 3014.9) [b] 
(0.9447795574474402, 3014.92) [b] 
(0.9447797039619193, 3019.79) [b] 
(0.9448068306980305, 3286.48) [b] 
(0.9448075182894802, 3327.61) [b] 
(0.9448082058809298, 3327.64) [b] 
(0.9448088934723795, 3327.65) [b] 
(0.9448095810638292, 3327.75) [b] 
(0.9448102686552788, 3327.76) [b] 
(0.9448116438381782, 3328.29) [b] 
(0.9448123314296278, 3335.9) [b] 
(0.9448130190210775, 3336.82) [b] 
(0.9448131655355566, 3483.49) [b] 
},{(0.9320879791666666, 0.001) [c] 
(0.9320879791666666, 3.683737729166666) [c] 
(0.9320879791666666, 3600) [c] 
}}}{legend pos=north west}}
%
% 	\subfloat[depth=8]{\cactus{Average Accuracy}{CPU time}{\budalg, \murtree, \cart}{{{(0.9114044084619304, 0) [a] 
(0.9220837226005072, 0.01) [a] 
(0.9268135142671737, 0.02) [a] 
(0.9317708753782848, 0.03) [a] 
(0.9363843476005069, 0.04) [a] 
(0.9365616392671735, 0.05) [a] 
(0.9367974726005068, 0.06) [a] 
(0.9369528892671735, 0.07) [a] 
(0.93698080593384, 0.08) [a] 
(0.93698268093384, 0.09) [a] 
(0.9370133059338396, 0.1) [a] 
(0.9370739309338396, 0.11) [a] 
(0.9372039309338397, 0.12) [a] 
(0.9373478892671727, 0.13) [a] 
(0.9373487226005061, 0.14) [a] 
(0.9373495559338394, 0.15) [a] 
(0.9373914309338393, 0.16) [a] 
(0.9373955976005057, 0.17) [a] 
(0.9375370559338391, 0.18) [a] 
(0.937537680933839, 0.19) [a] 
(0.9375405976005056, 0.2) [a] 
(0.9375422642671722, 0.21) [a] 
(0.9375964309338389, 0.22) [a] 
(0.9376241392671723, 0.23) [a] 
(0.9376393476005054, 0.24) [a] 
(0.9376664309338386, 0.25) [a] 
(0.9376683059338385, 0.27) [a] 
(0.9409625828527545, 0.28) [a] 
(0.9409692495194212, 0.29) [a] 
(0.9409840411860877, 0.3) [a] 
(0.9409909161860878, 0.31) [a] 
(0.9409942495194211, 0.32) [a] 
(0.9410182078527544, 0.33) [a] 
(0.9410190411860877, 0.34) [a] 
(0.9410425828527543, 0.37) [a] 
(0.941046124519421, 0.39) [a] 
(0.9410600828527543, 0.4) [a] 
(0.9410634161860877, 0.41) [a] 
(0.9410640411860877, 0.42) [a] 
(0.9410667495194209, 0.43) [a] 
(0.9410682078527542, 0.44) [a] 
(0.941069249519421, 0.45) [a] 
(0.9454911499045197, 0.47) [a] 
(0.9454915665711864, 0.48) [a] 
(0.945491983237853, 0.49) [a] 
(0.9455398999045199, 0.51) [a] 
(0.9455401082378533, 0.52) [a] 
(0.9456771915711868, 0.53) [a] 
(0.9456807332378535, 0.55) [a] 
(0.9456811499045201, 0.56) [a] 
(0.9457080249045202, 0.6) [a] 
(0.9457117749045203, 0.61) [a] 
(0.945725941571187, 0.63) [a] 
(0.945807191571187, 0.64) [a] 
(0.9458076082378536, 0.67) [a] 
(0.9458436499045203, 0.68) [a] 
(0.9458467749045202, 0.7) [a] 
(0.9458478165711869, 0.71) [a] 
(0.9459867749045202, 0.72) [a] 
(0.9460690665711868, 0.73) [a] 
(0.9460701082378535, 0.74) [a] 
(0.9460973999045202, 0.75) [a] 
(0.9464201082378536, 0.76) [a] 
(0.9464480249045202, 0.77) [a] 
(0.9465298999045202, 0.78) [a] 
(0.9465578165711868, 0.79) [a] 
(0.9465851082378535, 0.8) [a] 
(0.9465855249045201, 0.81) [a] 
(0.9465903165711866, 0.85) [a] 
(0.94659052490452, 0.92) [a] 
(0.9472960416666649, 0.98) [a] 
(0.9472993749999983, 1.03) [a] 
(0.9473556249999983, 1.05) [a] 
(0.9473695833333315, 1.06) [a] 
(0.9473837499999982, 1.07) [a] 
(0.9473870833333315, 1.09) [a] 
(0.9475666666666648, 1.11) [a] 
(0.9477270833333316, 1.12) [a] 
(0.9478041666666649, 1.15) [a] 
(0.9478074999999982, 1.16) [a] 
(0.9478079166666649, 1.17) [a] 
(0.9478081249999982, 1.21) [a] 
(0.9478089583333315, 1.26) [a] 
(0.9478091666666648, 1.27) [a] 
(0.947988749999998, 1.33) [a] 
(0.9481431249999981, 1.34) [a] 
(0.9481441666666648, 1.35) [a] 
(0.9481697916666648, 1.37) [a] 
(0.9482216666666647, 1.38) [a] 
(0.9482249999999981, 1.41) [a] 
(0.948226249999998, 1.42) [a] 
(0.948587499999998, 1.43) [a] 
(0.9486131249999981, 1.45) [a] 
(0.9487758333333314, 1.53) [a] 
(0.9488933333333313, 1.54) [a] 
(0.949098749999998, 1.68) [a] 
(0.9492527083333312, 1.69) [a] 
(0.9493020833333312, 1.7) [a] 
(0.9493024999999978, 1.77) [a] 
(0.9493027083333312, 1.8) [a] 
(0.9493031249999978, 1.83) [a] 
(0.9493033333333312, 1.91) [a] 
(0.9493035416666645, 1.95) [a] 
(0.9493039583333311, 2.09) [a] 
(0.9493043749999978, 2.34) [a] 
(0.949318333333331, 2.59) [a] 
(0.9493502083333311, 2.79) [a] 
(0.9494141666666644, 2.8) [a] 
(0.9494460416666645, 2.81) [a] 
(0.9494474999999978, 2.87) [a] 
(0.9494504166666644, 2.95) [a] 
(0.9494568749999978, 3.08) [a] 
(0.9494577083333311, 3.63) [a] 
(0.9494583333333311, 3.66) [a] 
(0.9494779166666644, 3.77) [a] 
(0.9494785416666643, 3.93) [a] 
(0.9494787499999977, 4.1) [a] 
(0.9496454166666646, 4.16) [a] 
(0.9496662499999979, 4.17) [a] 
(0.9496870833333313, 4.21) [a] 
(0.949728749999998, 4.38) [a] 
(0.9497289583333314, 4.55) [a] 
(0.9497560416666647, 4.9) [a] 
(0.9497564583333313, 5) [a] 
(0.949756874999998, 5.19) [a] 
(0.9497708333333312, 5.2) [a] 
(0.9497849999999979, 5.23) [a] 
(0.9498437499999978, 5.24) [a] 
(0.949857708333331, 5.25) [a] 
(0.9499543749999976, 5.47) [a] 
(0.9499822916666643, 5.48) [a] 
(0.9499964583333309, 5.69) [a] 
(0.9499972916666642, 5.84) [a] 
(0.9500291666666643, 6.03) [a] 
(0.9502206249999977, 6.08) [a] 
(0.9502589583333311, 6.1) [a] 
(0.9502908333333311, 6.13) [a] 
(0.9503227083333311, 6.15) [a] 
(0.9503545833333311, 6.16) [a] 
(0.9504185416666644, 6.2) [a] 
(0.950418958333331, 6.5) [a] 
(0.9504254166666644, 6.52) [a] 
(0.950425833333331, 6.53) [a] 
(0.950426458333331, 6.72) [a] 
(0.9504268749999976, 6.8) [a] 
(0.9504304166666643, 7.21) [a] 
(0.9504308333333309, 7.34) [a] 
(0.9504627083333309, 7.59) [a] 
(0.9504835416666643, 8.13) [a] 
(0.9504868749999976, 8.15) [a] 
(0.9504935416666643, 8.16) [a] 
(0.950517708333331, 8.24) [a] 
(0.9505245833333311, 8.31) [a] 
(0.9505249999999977, 8.36) [a] 
(0.9505283333333311, 8.5) [a] 
(0.9505316666666644, 8.59) [a] 
(0.9505418749999979, 9) [a] 
(0.9505424999999978, 9.5) [a] 
(0.9505429166666645, 9.65) [a] 
(0.9505491666666644, 10.85) [a] 
(0.9505495833333311, 11.27) [a] 
(0.950550208333331, 11.41) [a] 
(0.9505568749999977, 12.06) [a] 
(0.9505637499999978, 12.07) [a] 
(0.9505956249999978, 12.84) [a] 
(0.9506274999999978, 13.21) [a] 
(0.9506279166666645, 13.67) [a] 
(0.9506283333333311, 13.68) [a] 
(0.9506285416666644, 13.69) [a] 
(0.9506287499999978, 14.49) [a] 
(0.9506297916666644, 14.53) [a] 
(0.950630208333331, 14.55) [a] 
(0.9506306249999976, 14.63) [a] 
(0.9506312499999976, 14.66) [a] 
(0.9506316666666642, 14.67) [a] 
(0.9506381249999976, 14.7) [a] 
(0.9506385416666642, 14.84) [a] 
(0.9506449999999975, 14.98) [a] 
(0.9506514583333309, 14.99) [a] 
(0.9506524999999976, 15.46) [a] 
(0.9506664583333309, 15.84) [a] 
(0.9506668749999975, 15.89) [a] 
(0.9506670833333308, 15.9) [a] 
(0.9506679166666642, 15.96) [a] 
(0.9506683333333308, 16.04) [a] 
(0.9507806249999974, 16.19) [a] 
(0.9507924999999974, 16.4) [a] 
(0.9508287499999973, 16.44) [a] 
(0.950829166666664, 16.57) [a] 
(0.9508572916666639, 16.58) [a] 
(0.9508712499999972, 16.59) [a] 
(0.9508854166666638, 16.62) [a] 
(0.9508993749999971, 16.63) [a] 
(0.9509618749999972, 16.86) [a] 
(0.9510243749999971, 16.87) [a] 
(0.9510452083333305, 16.89) [a] 
(0.9510660416666639, 16.9) [a] 
(0.9511077083333306, 16.95) [a] 
(0.9512743749999975, 17.01) [a] 
(0.9512952083333308, 17.04) [a] 
(0.9513993749999976, 17.11) [a] 
(0.9513999999999976, 17.57) [a] 
(0.9514208333333309, 17.83) [a] 
(0.9514416666666643, 18.33) [a] 
(0.9514422916666643, 20.17) [a] 
(0.9514429166666643, 20.19) [a] 
(0.9514443749999976, 20.2) [a] 
(0.9514447916666642, 20.22) [a] 
(0.9514449999999975, 20.24) [a] 
(0.9514452083333309, 20.69) [a] 
(0.9514708333333309, 20.78) [a] 
(0.9514712499999975, 20.87) [a] 
(0.9514968749999976, 20.9) [a] 
(0.951650833333331, 20.94) [a] 
(0.9516518749999977, 21.08) [a] 
(0.951703333333331, 21.28) [a] 
(0.951728958333331, 21.9) [a] 
(0.9517545833333311, 22.02) [a] 
(0.9517804166666644, 22.06) [a] 
(0.9519085416666645, 22.07) [a] 
(0.9519599999999978, 22.42) [a] 
(0.9519602083333312, 23.1) [a] 
(0.9519606249999978, 23.11) [a] 
(0.9519610416666644, 23.13) [a] 
(0.9519620833333311, 23.16) [a] 
(0.9519627083333311, 23.19) [a] 
(0.951963333333331, 23.3) [a] 
(0.9519697916666644, 23.32) [a] 
(0.9520060416666645, 23.49) [a] 
(0.9520181249999978, 23.54) [a] 
(0.9520437499999979, 23.55) [a] 
(0.9520452083333312, 23.64) [a] 
(0.9520454166666645, 23.68) [a] 
(0.9520458333333311, 23.69) [a] 
(0.9520477083333311, 23.7) [a] 
(0.9520485416666643, 23.72) [a] 
(0.9520499999999976, 23.83) [a] 
(0.952050833333331, 23.95) [a] 
(0.9520812499999977, 24.18) [a] 
(0.952081458333331, 24.25) [a] 
(0.9520818749999976, 24.36) [a] 
(0.9520822916666642, 24.37) [a] 
(0.9520824999999976, 24.41) [a] 
(0.9520833333333308, 24.44) [a] 
(0.9520835416666642, 24.69) [a] 
(0.9520839583333308, 24.7) [a] 
(0.9521095833333308, 24.82) [a] 
(0.9521099999999975, 24.93) [a] 
(0.9521102083333308, 27.35) [a] 
(0.9521106249999974, 27.83) [a] 
(0.952111041666664, 27.86) [a] 
(0.9521112499999974, 27.89) [a] 
(0.952111666666664, 27.9) [a] 
(0.9521120833333306, 28.1) [a] 
(0.952112291666664, 28.11) [a] 
(0.9521127083333306, 28.18) [a] 
(0.9521131249999972, 28.77) [a] 
(0.9521133333333306, 28.78) [a] 
(0.9521137499999972, 29.36) [a] 
(0.9521139583333306, 29.38) [a] 
(0.9521147916666638, 29.39) [a] 
(0.9521962499999972, 29.65) [a] 
(0.9522504166666639, 29.76) [a] 
(0.9522506249999972, 29.95) [a] 
(0.9522508333333306, 30.03) [a] 
(0.9522512499999972, 30.06) [a] 
(0.9522516666666638, 30.94) [a] 
(0.9522520833333304, 32.38) [a] 
(0.9522524999999971, 32.75) [a] 
(0.9522797916666638, 33.18) [a] 
(0.9523068749999971, 33.22) [a] 
(0.9523079166666638, 34.08) [a] 
(0.9523087499999972, 34.09) [a] 
(0.9523089583333305, 34.36) [a] 
(0.9523093749999971, 34.5) [a] 
(0.9523097916666637, 34.51) [a] 
(0.9523368749999971, 36.78) [a] 
(0.9523910416666638, 36.79) [a] 
(0.9523916666666637, 37.16) [a] 
(0.9523924999999971, 37.19) [a] 
(0.9523927083333305, 37.43) [a] 
(0.9524199999999972, 37.57) [a] 
(0.9524204166666638, 37.75) [a] 
(0.9524210416666637, 37.86) [a] 
(0.9524418749999971, 37.91) [a] 
(0.9524424999999971, 38.21) [a] 
(0.9524695833333304, 40.62) [a] 
(0.9524704166666638, 40.83) [a] 
(0.9524710416666637, 40.84) [a] 
(0.9524714583333304, 40.86) [a] 
(0.9524716666666637, 41.05) [a] 
(0.9524720833333303, 41.06) [a] 
(0.952472499999997, 41.08) [a] 
(0.9524727083333303, 41.09) [a] 
(0.9524731249999969, 41.37) [a] 
(0.9524733333333303, 42.21) [a] 
(0.9524737499999969, 42.22) [a] 
(0.9524741666666635, 43.14) [a] 
(0.9524745833333301, 43.15) [a] 
(0.9525914583333301, 43.31) [a] 
(0.9525916666666635, 43.35) [a] 
(0.9525924999999967, 43.36) [a] 
(0.9525935416666634, 43.4) [a] 
(0.9525941666666634, 43.75) [a] 
(0.95259458333333, 44.57) [a] 
(0.9525949999999966, 44.89) [a] 
(0.95259520833333, 45.15) [a] 
(0.95259583333333, 45.2) [a] 
(0.9525962499999966, 45.21) [a] 
(0.9525977083333299, 45.8) [a] 
(0.9526116666666632, 46.64) [a] 
(0.9526118749999966, 46.93) [a] 
(0.9526122916666632, 46.98) [a] 
(0.9526404166666631, 47) [a] 
(0.9526410416666631, 47.03) [a] 
(0.9526552083333297, 47.06) [a] 
(0.9526695833333296, 47.18) [a] 
(0.9526699999999962, 47.26) [a] 
(0.9526839583333295, 47.71) [a] 
(0.9526989583333294, 47.73) [a] 
(0.9527410416666626, 47.74) [a] 
(0.9527416666666626, 47.92) [a] 
(0.9527570833333292, 48.28) [a] 
(0.9527712499999959, 48.39) [a] 
(0.9527716666666625, 52.89) [a] 
(0.9527731249999958, 56.23) [a] 
(0.9527741666666624, 56.24) [a] 
(0.9527747916666623, 56.25) [a] 
(0.9527956249999957, 57.82) [a] 
(0.952809583333329, 58.35) [a] 
(0.9528099999999956, 62.13) [a] 
(0.952837083333329, 63.6) [a] 
(0.9528374999999956, 65.81) [a] 
(0.9528377083333289, 65.84) [a] 
(0.9528381249999955, 65.85) [a] 
(0.9528383333333289, 66.37) [a] 
(0.9528387499999955, 66.41) [a] 
(0.9528391666666621, 66.64) [a] 
(0.9528395833333287, 66.87) [a] 
(0.9528397916666621, 66.88) [a] 
(0.9528408333333288, 67.13) [a] 
(0.9528679166666622, 70.13) [a] 
(0.9528681249999955, 70.24) [a] 
(0.9528685416666621, 70.31) [a] 
(0.9528691666666621, 70.32) [a] 
(0.9528724999999955, 71.78) [a] 
(0.9528731249999954, 73.38) [a] 
(0.9528870833333287, 74.82) [a] 
(0.9529152083333288, 75.34) [a] 
(0.9529293749999954, 75.44) [a] 
(0.9529433333333287, 75.46) [a] 
(0.952957291666662, 76.01) [a] 
(0.952963541666662, 76.97) [a] 
(0.9529777083333286, 77.48) [a] 
(0.9529781249999952, 80.8) [a] 
(0.9529785416666618, 81.37) [a] 
(0.9529789583333285, 82.65) [a] 
(0.9529799999999952, 83.9) [a] 
(0.9529804166666618, 83.95) [a] 
(0.9529806249999951, 83.96) [a] 
(0.9529808333333285, 88.47) [a] 
(0.9530016666666619, 88.9) [a] 
(0.9530224999999952, 89.02) [a] 
(0.9530641666666619, 89.06) [a] 
(0.9530849999999953, 91.52) [a] 
(0.9530852083333287, 94.02) [a] 
(0.9530862499999954, 96.93) [a] 
(0.9532033333333287, 97.33) [a] 
(0.9532035416666621, 104.8) [a] 
(0.9532049999999954, 119.52) [a] 
(0.9532070833333287, 119.53) [a] 
(0.9532104166666621, 119.54) [a] 
(0.9532108333333287, 145.69) [a] 
(0.953211041666662, 148.93) [a] 
(0.9532114583333287, 155.7) [a] 
(0.9532118749999953, 155.72) [a] 
(0.9532120833333286, 156.29) [a] 
(0.953212916666662, 156.3) [a] 
(0.9532139583333287, 156.34) [a] 
(0.953214166666662, 156.36) [a] 
(0.9532145833333286, 156.44) [a] 
(0.9532149999999953, 156.45) [a] 
(0.9532420833333286, 156.54) [a] 
(0.953242291666662, 163.36) [a] 
(0.9532427083333286, 163.37) [a] 
(0.9532433333333286, 163.4) [a] 
(0.9532437499999952, 163.48) [a] 
(0.9532441666666618, 163.5) [a] 
(0.9532443749999951, 167.27) [a] 
(0.9532447916666618, 168.76) [a] 
(0.9532449999999951, 178.11) [a] 
(0.9532454166666617, 178.52) [a] 
(0.953285624999995, 182.45) [a] 
(0.9532860416666616, 182.81) [a] 
(0.9532864583333283, 182.83) [a] 
(0.9532866666666616, 183.27) [a] 
(0.9532870833333282, 183.72) [a] 
(0.9532874999999948, 187.72) [a] 
(0.9532877083333282, 187.81) [a] 
(0.9532908333333282, 187.82) [a] 
(0.9532912499999948, 193.61) [a] 
(0.9532916666666614, 201.47) [a] 
(0.9533172916666615, 220.22) [a] 
(0.9533429166666615, 220.43) [a] 
(0.9533972916666615, 226.72) [a] 
(0.9534243749999949, 226.88) [a] 
(0.9534247916666615, 226.91) [a] 
(0.9534252083333281, 226.92) [a] 
(0.9534254166666615, 226.93) [a] 
(0.9534262499999947, 226.95) [a] 
(0.9534264583333281, 226.98) [a] 
(0.9534268749999947, 227.08) [a] 
(0.9534272916666613, 227.12) [a] 
(0.9534279166666613, 227.17) [a] 
(0.9534283333333279, 227.2) [a] 
(0.9534285416666612, 227.23) [a] 
(0.9534289583333279, 227.26) [a] 
(0.9534299999999946, 227.29) [a] 
(0.9534306249999945, 227.33) [a] 
(0.9534310416666611, 227.42) [a] 
(0.9534320833333279, 227.49) [a] 
(0.9534324999999945, 227.58) [a] 
(0.9534331249999944, 227.59) [a] 
(0.9534337499999944, 227.6) [a] 
(0.9534358333333277, 227.63) [a] 
(0.9534362499999943, 227.76) [a] 
(0.953437291666661, 227.88) [a] 
(0.953437916666661, 227.89) [a] 
(0.9534383333333276, 227.92) [a] 
(0.9534970833333276, 229.57) [a] 
(0.9535099999999943, 230.24) [a] 
(0.9535362499999943, 230.25) [a] 
(0.9535427083333277, 231.58) [a] 
(0.9535545833333277, 238) [a] 
(0.9535549999999943, 241.29) [a] 
(0.9535554166666609, 245.32) [a] 
(0.9535562499999942, 248.68) [a] 
(0.9535566666666608, 248.94) [a] 
(0.9535568749999942, 249.71) [a] 
(0.9535572916666608, 251.08) [a] 
(0.9535574999999942, 284.46) [a] 
(0.9535579166666608, 286.52) [a] 
(0.9535581249999941, 286.56) [a] 
(0.9535589583333274, 286.6) [a] 
(0.9535595833333274, 286.68) [a] 
(0.953559999999994, 286.71) [a] 
(0.9535606249999939, 286.72) [a] 
(0.9535610416666606, 286.73) [a] 
(0.9535612499999939, 286.74) [a] 
(0.9535620833333273, 286.75) [a] 
(0.9535627083333272, 287.18) [a] 
(0.9535691666666606, 287.7) [a] 
(0.9535756249999939, 288.42) [a] 
(0.9535760416666605, 293.65) [a] 
(0.9535762499999939, 319.58) [a] 
(0.9535766666666605, 329.1) [a] 
(0.9535777083333272, 336.55) [a] 
(0.9535781249999938, 338.89) [a] 
(0.9535783333333272, 338.99) [a] 
(0.9535787499999938, 339.01) [a] 
(0.9535791666666604, 339.06) [a] 
(0.9535793749999938, 341.71) [a] 
(0.9535797916666604, 360.37) [a] 
(0.953580208333327, 360.38) [a] 
(0.9535804166666604, 360.39) [a] 
(0.9535924999999937, 361.17) [a] 
(0.9536133333333271, 362.2) [a] 
(0.9536254166666605, 365.52) [a] 
(0.9536374999999938, 365.95) [a] 
(0.9536389583333271, 379.67) [a] 
(0.9536393749999937, 379.78) [a] 
(0.9536397916666604, 384.13) [a] 
(0.9537568749999937, 391.44) [a] 
(0.9537579166666603, 394.47) [a] 
(0.9537583333333269, 394.5) [a] 
(0.9537585416666603, 394.51) [a] 
(0.9537793749999937, 408.99) [a] 
(0.953800208333327, 409.38) [a] 
(0.9538006249999936, 415.66) [a] 
(0.9538012499999936, 416.79) [a] 
(0.953918333333327, 420.18) [a] 
(0.9539191666666603, 428.34) [a] 
(0.9539204166666603, 428.35) [a] 
(0.9539412499999936, 472.55) [a] 
(0.9539829166666604, 472.64) [a] 
(0.9540245833333271, 473.05) [a] 
(0.9540454166666604, 473.06) [a] 
(0.9540870833333271, 473.07) [a] 
(0.9541106249999938, 558.4) [a] 
(0.9541341666666604, 558.49) [a] 
(0.9541356249999937, 566.97) [a] 
(0.9541564583333271, 577.24) [a] 
(0.9542735416666605, 612.8) [a] 
(0.9542739583333271, 623.99) [a] 
(0.954274583333327, 649.82) [a] 
(0.954275208333327, 660.78) [a] 
(0.9542754166666604, 674.07) [a] 
(0.954275833333327, 734.12) [a] 
(0.9542760416666604, 734.13) [a] 
(0.9542768749999937, 734.14) [a] 
(0.9542770833333271, 734.24) [a] 
(0.9542774999999937, 734.27) [a] 
(0.9542781249999936, 734.28) [a] 
(0.9542785416666603, 734.33) [a] 
(0.9542799999999936, 734.5) [a] 
(0.9542802083333269, 734.57) [a] 
(0.9542806249999936, 734.58) [a] 
(0.9542810416666602, 734.61) [a] 
(0.9542812499999935, 735.08) [a] 
(0.9542816666666601, 735.19) [a] 
(0.9542820833333268, 740.86) [a] 
(0.9542824999999934, 745.69) [a] 
(0.9542956249999933, 750.49) [a] 
(0.95429666666666, 752.64) [a] 
(0.9542968749999934, 758.78) [a] 
(0.9542970833333267, 769.22) [a] 
(0.9543252083333268, 771.71) [a] 
(0.9543391666666601, 771.72) [a] 
(0.9543531249999934, 771.81) [a] 
(0.95436729166666, 771.86) [a] 
(0.9544233333333266, 772.02) [a] 
(0.95446541666666, 773.01) [a] 
(0.9544795833333266, 773.02) [a] 
(0.9544935416666599, 775.58) [a] 
(0.9545077083333265, 776.41) [a] 
(0.9545085416666598, 780.25) [a] 
(0.9545087499999931, 780.29) [a] 
(0.9545227083333264, 784.42) [a] 
(0.9545366666666597, 784.74) [a] 
(0.9545508333333264, 784.76) [a] 
(0.954578749999993, 825.73) [a] 
(0.9546068749999931, 825.76) [a] 
(0.9546210416666597, 825.91) [a] 
(0.9546489583333263, 832.71) [a] 
(0.954663124999993, 832.95) [a] 
(0.9546770833333262, 832.97) [a] 
(0.9546772916666596, 926.8) [a] 
(0.954677499999993, 926.81) [a] 
(0.954705624999993, 931.77) [a] 
(0.9547195833333263, 938.13) [a] 
(0.9547337499999929, 938.16) [a] 
(0.9547339583333263, 942.69) [a] 
(0.9547341666666597, 942.7) [a] 
(0.954734374999993, 942.82) [a] 
(0.9547377083333264, 959.9) [a] 
(0.9547379166666597, 1038.6) [a] 
(0.9547412499999931, 1113.2) [a] 
(0.9547445833333265, 1116) [a] 
(0.9547449999999931, 1151.9) [a] 
(0.9547454166666597, 1160.2) [a] 
(0.954753124999993, 1171.5) [a] 
(0.954753749999993, 1171.6) [a] 
(0.9547545833333263, 1171.8) [a] 
(0.9547558333333263, 1171.9) [a] 
(0.9547566666666596, 1172.1) [a] 
(0.9547649999999928, 1172.5) [a] 
(0.9547656249999927, 1172.8) [a] 
(0.954767083333326, 1173) [a] 
(0.9547704166666593, 1173.6) [a] 
(0.9547712499999926, 1176.6) [a] 
(0.9547718749999926, 1183) [a] 
(0.9547724999999926, 1190.6) [a] 
(0.9547739583333259, 1192.8) [a] 
(0.9547747916666592, 1222.7) [a] 
(0.9547760416666591, 1225.2) [a] 
(0.9547849999999923, 1230.7) [a] 
(0.9547856249999923, 1267.3) [a] 
(0.9547862499999923, 1267.4) [a] 
(0.9547866666666589, 1267.5) [a] 
(0.9547872916666589, 1267.6) [a] 
(0.9547877083333255, 1270.9) [a] 
(0.9547881249999921, 1271) [a] 
(0.9547883333333255, 1300.9) [a] 
(0.9547891666666588, 1301) [a] 
(0.9547893749999922, 1302.2) [a] 
(0.9547897916666588, 1302.7) [a] 
(0.9547902083333254, 1305.6) [a] 
(0.9547908333333254, 1360.8) [a] 
(0.954791874999992, 1360.9) [a] 
(0.9547935416666585, 1361) [a] 
(0.9547939583333251, 1361.2) [a] 
(0.9547943749999918, 1363.1) [a] 
(0.9547945833333251, 1376.3) [a] 
(0.9547954166666585, 1392.9) [a] 
(0.9547956249999918, 1404) [a] 
(0.9547960416666584, 1404.1) [a] 
(0.954796458333325, 1408.6) [a] 
(0.9547968749999917, 1450.1) [a] 
(0.954797083333325, 1482) [a] 
(0.9547974999999916, 1485) [a] 
(0.9547987499999916, 1508.8) [a] 
(0.9547991666666582, 1511.6) [a] 
(0.9547995833333248, 1523.1) [a] 
(0.9548127083333247, 1603.1) [a] 
(0.9548202083333245, 1603.5) [a] 
(0.9548216666666578, 1603.6) [a] 
(0.954830624999991, 1604.1) [a] 
(0.9548470833333244, 1604.2) [a] 
(0.9548506249999911, 1604.4) [a] 
(0.954851249999991, 1604.8) [a] 
(0.9548691666666577, 1611) [a] 
(0.9548927083333243, 1654) [a] 
(0.9549135416666577, 1675.8) [a] 
(0.9549552083333244, 1697.7) [a] 
(0.9549760416666577, 1713.4) [a] 
(0.9549764583333243, 1717.5) [a] 
(0.9549766666666577, 1726.3) [a] 
(0.9549770833333243, 1757.4) [a] 
(0.9549772916666577, 1816.4) [a] 
(0.954979374999991, 1981.3) [a] 
(0.9549854166666577, 1982.5) [a] 
(0.9549902083333243, 1991.5) [a] 
(0.9550158333333243, 2166) [a] 
(0.9550414583333243, 2166.1) [a] 
(0.9550927083333244, 2213.1) [a] 
(0.9551066666666577, 2259.7) [a] 
(0.9551322916666577, 2352.5) [a] 
(0.955158124999991, 2397.2) [a] 
(0.955183749999991, 2400.2) [a] 
(0.9552095833333243, 2445.1) [a] 
(0.9552127083333243, 2545.9) [a] 
(0.9552141666666575, 2546) [a] 
(0.9552152083333241, 2569.4) [a] 
(0.9552154166666574, 2575.8) [a] 
(0.9552156249999908, 2575.9) [a] 
(0.9552164583333241, 2576.6) [a] 
(0.9552166666666575, 2576.7) [a] 
(0.9552168749999909, 2660.3) [a] 
(0.9552233333333242, 2662.9) [a] 
(0.9553066666666576, 2724.4) [a] 
(0.955327499999991, 2724.7) [a] 
(0.9553691666666577, 2727.9) [a] 
(0.9553693749999911, 2836.4) [a] 
(0.9553708333333244, 2876.2) [a] 
(0.9553710416666578, 2919.3) [a] 
(0.9553712499999911, 3055.3) [a] 
(0.9553714583333245, 3251.2) [a] 
(0.9553720833333245, 3475.2) [a] 
},{(0.8382336937226955, 0) [b] 
(0.878129069674984, 0.001) [b] 
(0.8890266899770825, 0.002) [b] 
(0.8944582788610956, 0.003) [b] 
(0.898171366543596, 0.004) [b] 
(0.9008209088814668, 0.005) [b] 
(0.9025653564900693, 0.006) [b] 
(0.9031911654707528, 0.007) [b] 
(0.903514854814555, 0.008) [b] 
(0.9041084196741267, 0.009) [b] 
(0.9043417768766402, 0.01) [b] 
(0.9048647031821019, 0.011) [b] 
(0.9050166151295679, 0.012) [b] 
(0.9068590580565562, 0.013) [b] 
(0.9069349946841997, 0.014) [b] 
(0.9069544546966541, 0.015) [b] 
(0.9077637726609434, 0.016) [b] 
(0.9078419498740146, 0.017) [b] 
(0.9081963951749187, 0.018) [b] 
(0.9087018638232044, 0.019) [b] 
(0.909215367033434, 0.02) [b] 
(0.9093198809337828, 0.021) [b] 
(0.9097051241771922, 0.022) [b] 
(0.9097825565713902, 0.023) [b] 
(0.9097893129629366, 0.024) [b] 
(0.9102601973136455, 0.025) [b] 
(0.9102629773154247, 0.026) [b] 
(0.9103099521369203, 0.027) [b] 
(0.9103512076239001, 0.028) [b] 
(0.9103581131743773, 0.029) [b] 
(0.91037117482537, 0.03) [b] 
(0.9103862551263584, 0.031) [b] 
(0.9104608804921571, 0.032) [b] 
(0.9104702554921571, 0.033) [b] 
(0.9105039130537647, 0.036) [b] 
(0.9106626109937028, 0.038) [b] 
(0.9107082532084357, 0.042) [b] 
(0.910727808903054, 0.044) [b] 
(0.9108337116808318, 0.045) [b] 
(0.9108806865023275, 0.046) [b] 
(0.9108950887924756, 0.047) [b] 
(0.910901619617972, 0.048) [b] 
(0.911276619617972, 0.049) [b] 
(0.9115978001735277, 0.05) [b] 
(0.9116200223957499, 0.052) [b] 
(0.9116432862846388, 0.053) [b] 
(0.9116516196179721, 0.054) [b] 
(0.9116800918401943, 0.055) [b] 
(0.9116870362846387, 0.056) [b] 
(0.9116905085068608, 0.057) [b] 
(0.9118154669589026, 0.059) [b] 
(0.9118459696431388, 0.06) [b] 
(0.9118792501051527, 0.061) [b] 
(0.911900083438486, 0.062) [b] 
(0.9119042501051526, 0.063) [b] 
(0.9119375430267113, 0.064) [b] 
(0.911950604677704, 0.065) [b] 
(0.911989789630682, 0.072) [b] 
(0.9121640744122584, 0.074) [b] 
(0.912366895592814, 0.076) [b] 
(0.9123915212823334, 0.077) [b] 
(0.9125415212823333, 0.078) [b] 
(0.9125665212823334, 0.079) [b] 
(0.9125911469718527, 0.081) [b] 
(0.9130788726327167, 0.083) [b] 
(0.9132145063132723, 0.085) [b] 
(0.9132673004055695, 0.087) [b] 
(0.91327935673273, 0.089) [b] 
(0.9133980651847717, 0.09) [b] 
(0.9134258158742911, 0.091) [b] 
(0.91347095476318, 0.096) [b] 
(0.9137109900703094, 0.099) [b] 
(0.9137381168064206, 0.101) [b] 
(0.9138380763946459, 0.102) [b] 
(0.9140064791724236, 0.104) [b] 
(0.9140658333984445, 0.105) [b] 
(0.9148851217879576, 0.107) [b] 
(0.9148892884546242, 0.109) [b] 
(0.9153877072724169, 0.113) [b] 
(0.9161037062924213, 0.115) [b] 
(0.9161068312924213, 0.116) [b] 
(0.9161335297735379, 0.12) [b] 
(0.9161591865879878, 0.122) [b] 
(0.9162359226990989, 0.123) [b] 
(0.9162515476990989, 0.126) [b] 
(0.9163217408707072, 0.127) [b] 
(0.9164200113109587, 0.128) [b] 
(0.9165502196442921, 0.129) [b] 
(0.9165550807554032, 0.13) [b] 
(0.9165571640887364, 0.131) [b] 
(0.9165779974220698, 0.132) [b] 
(0.9165800807554031, 0.133) [b] 
(0.9166338249398586, 0.136) [b] 
(0.9166560471620807, 0.137) [b] 
(0.9167067416065251, 0.14) [b] 
(0.9167342155648585, 0.141) [b] 
(0.9168871524148859, 0.142) [b] 
(0.9169128092293358, 0.143) [b] 
(0.9169135036737802, 0.144) [b] 
(0.9169395077104523, 0.145) [b] 
(0.9169651645249022, 0.146) [b] 
(0.9169675950804578, 0.147) [b] 
(0.9169905117471244, 0.148) [b] 
(0.9170185890157677, 0.149) [b] 
(0.9170758806824344, 0.151) [b] 
(0.9171092140157677, 0.152) [b] 
(0.9173602932925784, 0.153) [b] 
(0.9173675849592451, 0.154) [b] 
(0.9173700155148007, 0.155) [b] 
(0.9173703627370229, 0.156) [b] 
(0.9180120294036896, 0.159) [b] 
(0.9180137655148007, 0.162) [b] 
(0.9180179321814674, 0.163) [b] 
(0.9180182794036896, 0.166) [b] 
(0.9180394599592452, 0.167) [b] 
(0.9180498766259119, 0.168) [b] 
(0.9180769599592452, 0.169) [b] 
(0.9181036960703562, 0.17) [b] 
(0.9181693210703562, 0.172) [b] 
(0.9182731405148007, 0.174) [b] 
(0.9183786960703563, 0.175) [b] 
(0.9184352932925784, 0.176) [b] 
(0.9186335571814673, 0.177) [b] 
(0.9186977932925785, 0.178) [b] 
(0.918914112737023, 0.179) [b] 
(0.918925223848134, 0.18) [b] 
(0.9192338515323701, 0.181) [b] 
(0.9194598403277175, 0.182) [b] 
(0.9196851875499397, 0.183) [b] 
(0.9200296319943841, 0.184) [b] 
(0.9202560208832731, 0.185) [b] 
(0.9205261597721618, 0.186) [b] 
(0.9205535903277173, 0.187) [b] 
(0.9213087986610506, 0.188) [b] 
(0.9213351875499395, 0.189) [b] 
(0.9219962986610507, 0.19) [b] 
(0.9225685208832728, 0.191) [b] 
(0.9228803264388283, 0.192) [b] 
(0.9231025486610505, 0.193) [b] 
(0.9232678264388282, 0.194) [b] 
(0.9234317153277171, 0.195) [b] 
(0.9235157431054948, 0.196) [b] 
(0.9238848403277169, 0.197) [b] 
(0.924000812549939, 0.198) [b] 
(0.9245848403277169, 0.199) [b] 
(0.9246428264388279, 0.2) [b] 
(0.9254247708832724, 0.201) [b] 
(0.9255216458832721, 0.202) [b] 
(0.9256799792166056, 0.203) [b] 
(0.9257143542166055, 0.204) [b] 
(0.9257424792166055, 0.205) [b] 
(0.9259436035552298, 0.206) [b] 
(0.9261147841107853, 0.207) [b] 
(0.9261696452218964, 0.208) [b] 
(0.9261738118885631, 0.209) [b] 
(0.9263123535552296, 0.21) [b] 
(0.9263227702218961, 0.211) [b] 
(0.9264252007774516, 0.212) [b] 
(0.9264297146663404, 0.213) [b] 
(0.9264300618885626, 0.214) [b] 
(0.9265109646663403, 0.217) [b] 
(0.9265123535552292, 0.218) [b] 
(0.926569992444118, 0.219) [b] 
(0.9265752007774514, 0.22) [b] 
(0.9266099229996736, 0.221) [b] 
(0.9266255479996736, 0.222) [b] 
(0.9266366591107846, 0.223) [b] 
(0.9266745063330069, 0.224) [b] 
(0.9266908257774513, 0.225) [b] 
(0.9266939507774513, 0.226) [b] 
(0.9267213813330069, 0.227) [b] 
(0.9272772841107846, 0.233) [b] 
(0.9272776313330068, 0.235) [b] 
(0.9272939507774512, 0.237) [b] 
(0.9272942979996734, 0.239) [b] 
(0.92729533966634, 0.244) [b] 
(0.9273317979996734, 0.245) [b] 
(0.9273852702218955, 0.246) [b] 
(0.9273866591107844, 0.248) [b] 
(0.9274300618885623, 0.249) [b] 
(0.9274877007774511, 0.25) [b] 
(0.9274995063330067, 0.251) [b] 
(0.9275290202218957, 0.252) [b] 
(0.9275422146663401, 0.253) [b] 
(0.9275477702218957, 0.255) [b] 
(0.9275894368885624, 0.262) [b] 
(0.9276519368885624, 0.263) [b] 
(0.9276540202218957, 0.264) [b] 
(0.9276567979996735, 0.276) [b] 
(0.9284226561023953, 0.283) [b] 
(0.9284461435131431, 0.293) [b] 
(0.9284666296242543, 0.303) [b] 
(0.9286159351798099, 0.312) [b] 
(0.9286394225905578, 0.321) [b] 
(0.9287112975905578, 0.322) [b] 
(0.9287127211186298, 0.33) [b] 
(0.9287226858151336, 0.332) [b] 
(0.9289567682121374, 0.338) [b] 
(0.9289667329086412, 0.343) [b] 
(0.9291115245753079, 0.347) [b] 
(0.929117940754345, 0.35) [b] 
(0.9291207185321227, 0.353) [b] 
(0.9292844242603997, 0.363) [b] 
(0.9293482323257392, 0.365) [b] 
(0.9293510793818831, 0.367) [b] 
(0.9293653146626029, 0.37) [b] 
(0.9293710087748908, 0.373) [b] 
(0.9294569639399317, 0.376) [b] 
(0.9294637203314782, 0.378) [b] 
(0.929480734220367, 0.379) [b] 
(0.929593192938053, 0.384) [b] 
(0.9296679108587059, 0.386) [b] 
(0.9296778755552098, 0.388) [b] 
(0.9297348166780887, 0.396) [b] 
(0.9297428027891997, 0.402) [b] 
(0.9297466222336442, 0.411) [b] 
(0.9298916453817924, 0.436) [b] 
(0.9299398706904344, 0.437) [b] 
(0.9300569118889362, 0.442) [b] 
(0.9300689682160967, 0.443) [b] 
(0.9300717459938745, 0.445) [b] 
(0.9300751241896478, 0.448) [b] 
(0.930078502385421, 0.468) [b] 
(0.9300905587125815, 0.475) [b] 
(0.9300936837125815, 0.493) [b] 
(0.9302690309348037, 0.511) [b] 
(0.9302693781570259, 0.543) [b] 
(0.9302763226014704, 0.55) [b] 
(0.9306008870018804, 0.58) [b] 
(0.9307179282003822, 0.624) [b] 
(0.931102280779815, 0.655) [b] 
(0.9311029752242594, 0.657) [b] 
(0.9311043641131483, 0.658) [b] 
(0.9311061002242594, 0.661) [b] 
(0.9313879587825102, 0.674) [b] 
(0.9314136155969601, 0.687) [b] 
(0.9315937723951879, 0.713) [b] 
(0.9316051606197637, 0.752) [b] 
(0.9316089800642082, 0.759) [b] 
(0.931646132841986, 0.76) [b] 
(0.9316952770671838, 0.771) [b] 
(0.9317271810998535, 0.797) [b] 
(0.9317292644331868, 0.815) [b] 
(0.9317611684658565, 0.835) [b] 
(0.9317654390500725, 0.851) [b] 
(0.9319092153853418, 0.89) [b] 
(0.9319119931631196, 0.894) [b] 
(0.9319376166684151, 0.928) [b] 
(0.932054657866917, 0.934) [b] 
(0.9320908268483985, 0.952) [b] 
(0.9321149395027195, 0.957) [b] 
(0.9321177865588635, 0.97) [b] 
(0.9322973842600129, 0.984) [b] 
(0.9323094405871734, 0.988) [b] 
(0.9323607542160732, 0.999) [b] 
(0.9323864110305231, 1.01) [b] 
(0.9325660087316725, 1.056) [b] 
(0.9326173223605724, 1.071) [b] 
(0.9326429791750223, 1.081) [b] 
(0.9326942928039221, 1.09) [b] 
(0.932719949618372, 1.095) [b] 
(0.9327712632472719, 1.157) [b] 
(0.9327969200617218, 1.161) [b] 
(0.932802128395055, 1.223) [b] 
(0.9328277852095049, 1.341) [b] 
(0.9328790988384048, 1.373) [b] 
(0.9329047556528547, 1.397) [b] 
(0.9329560692817546, 1.428) [b] 
(0.9329769026150879, 1.491) [b] 
(0.9330238774365835, 1.539) [b] 
(0.9330379160709052, 1.783) [b] 
(0.9331715487414205, 1.852) [b] 
(0.9331909024385296, 1.897) [b] 
(0.9331918240431539, 1.905) [b] 
(0.9331987360778358, 1.927) [b] 
(0.9332001184847721, 1.93) [b] 
(0.9332386601514387, 1.953) [b] 
(0.9332391209537508, 1.965) [b] 
(0.9332418857676236, 1.979) [b] 
(0.9332547181256977, 1.999) [b] 
(0.9332654820145866, 2.001) [b] 
(0.9376779120685742, 2.015) [b] 
(0.9377227037352409, 2.036) [b] 
(0.9377641759433318, 2.062) [b] 
(0.937811398165554, 2.067) [b] 
(0.9378147763613273, 2.101) [b] 
(0.9378526235835495, 2.12) [b] 
(0.9379217208057717, 2.124) [b] 
(0.9379290124724384, 2.13) [b] 
(0.9380158180279939, 2.131) [b] 
(0.9380469155561889, 2.185) [b] 
(0.9380519843816223, 2.209) [b] 
(0.9380558038260668, 2.25) [b] 
(0.9380766371594002, 2.253) [b] 
(0.9380775587640244, 2.258) [b] 
(0.9380784803686487, 2.265) [b] 
(0.9380789199120861, 2.363) [b] 
(0.9380792129410445, 2.383) [b] 
(0.9380802385423985, 2.385) [b] 
(0.938080678085836, 2.387) [b] 
(0.938080824600315, 2.388) [b] 
(0.9381138933833838, 2.391) [b] 
(0.9381143329268212, 2.43) [b] 
(0.9381146259557795, 2.432) [b] 
(0.9381286645901012, 2.436) [b] 
(0.9381474145901012, 2.453) [b] 
(0.9381478541335386, 2.463) [b] 
(0.9381674098281569, 2.466) [b] 
(0.9381688910382611, 2.491) [b] 
(0.9381720160382611, 2.775) [b] 
(0.9381753942340343, 2.866) [b] 
(0.938176435900701, 2.936) [b] 
(0.9381816442340343, 2.962) [b] 
(0.938192060900701, 2.969) [b] 
(0.9381941442340342, 3.036) [b] 
(0.9383383753577283, 3.061) [b] 
(0.9383463614688394, 3.087) [b] 
(0.9384634026673413, 3.118) [b] 
(0.93849647145041, 3.142) [b] 
(0.9385308510228931, 3.38) [b] 
(0.9385322262057925, 3.381) [b] 
(0.9385384145288395, 3.384) [b] 
(0.9385404978621728, 3.385) [b] 
(0.9385411854536224, 3.399) [b] 
(0.9385418730450721, 3.454) [b] 
(0.9385425606365217, 3.522) [b] 
(0.9385480613681191, 3.611) [b] 
(0.9385494365510184, 3.614) [b] 
(0.9385501241424681, 3.626) [b] 
(0.9386284940484242, 3.824) [b] 
(0.9386480865249132, 3.831) [b] 
(0.9386532948582464, 4.065) [b] 
(0.9386663565092391, 4.113) [b] 
(0.938883370398128, 4.222) [b] 
(0.9389104971342391, 4.224) [b] 
(0.9389376238703503, 4.225) [b] 
(0.9389918773425725, 4.23) [b] 
(0.9390190040786837, 4.236) [b] 
(0.9390272985203019, 4.239) [b] 
(0.9390293818536352, 4.29) [b] 
(0.9390434204879569, 4.367) [b] 
(0.9390855363909218, 4.394) [b] 
(0.939112663127033, 4.433) [b] 
(0.9391397898631442, 4.499) [b] 
(0.9391418526374932, 4.503) [b] 
(0.9391524510906719, 4.511) [b] 
(0.9391531386821216, 4.527) [b] 
(0.939158668309867, 4.675) [b] 
(0.9391905723425368, 4.783) [b] 
(0.9392224763752065, 4.784) [b] 
(0.9392231708196509, 5.088) [b] 
(0.9392299272111974, 5.097) [b] 
(0.9392333054069706, 5.101) [b] 
(0.9392501963858366, 5.103) [b] 
(0.9392535745816099, 5.104) [b] 
(0.9392840772658461, 5.44) [b] 
(0.9393081899201671, 6.007) [b] 
(0.9393146060992041, 6.065) [b] 
(0.939338093509952, 6.144) [b] 
(0.9393422601766187, 6.16) [b] 
(0.9393486763556558, 6.167) [b] 
(0.9393497180223225, 6.17) [b] 
(0.9393511004292588, 6.185) [b] 
(0.9393530637005498, 6.19) [b] 
(0.9394148112103742, 6.216) [b] 
(0.9394235664543045, 6.241) [b] 
(0.9394267920704893, 6.245) [b] 
(0.9394277136751136, 6.249) [b] 
(0.9394286352797379, 6.265) [b] 
(0.9394521226904857, 6.273) [b] 
(0.9394544267020464, 6.284) [b] 
(0.9394604171321039, 6.298) [b] 
(0.9394613387367282, 6.301) [b] 
(0.939464564352913, 6.306) [b] 
(0.9394666476862463, 6.307) [b] 
(0.9394671084885584, 6.322) [b] 
(0.9395062198777949, 6.453) [b] 
(0.9395303325321159, 6.463) [b] 
(0.9395306797543381, 6.798) [b] 
(0.9395931797543381, 6.986) [b] 
(0.9396348464210048, 6.987) [b] 
(0.9396469027481653, 7.055) [b] 
(0.9396496805259431, 7.28) [b] 
(0.9397479509661947, 7.412) [b] 
(0.9397619896005164, 7.421) [b] 
(0.9398828793490657, 7.441) [b] 
(0.9399249952520307, 7.463) [b] 
(0.9399390338863524, 7.464) [b] 
(0.9399418116641302, 7.556) [b] 
(0.9400627014126794, 7.659) [b] 
(0.9400767400470011, 7.987) [b] 
(0.9400887963741617, 7.996) [b] 
(0.9401129090284827, 7.997) [b] 
(0.9401249653556432, 8.064) [b] 
(0.9401530426242865, 8.317) [b] 
(0.940165098951447, 8.876) [b] 
(0.9401692656181136, 9.902) [b] 
(0.9401827784012065, 10.656) [b] 
(0.9401861565969797, 10.66) [b] 
(0.9401895347927529, 10.89) [b] 
(0.9401929129885261, 10.92) [b] 
(0.9401949963218594, 11.196) [b] 
(0.9401977466876581, 11.688) [b] 
(0.9402011846449064, 11.689) [b] 
(0.9402032474192554, 11.697) [b] 
(0.9402039350107051, 11.933) [b] 
(0.9402053101936044, 12.323) [b] 
(0.9402066853765038, 12.367) [b] 
(0.9402073729679534, 12.851) [b] 
(0.9402080605594031, 12.858) [b] 
(0.9402385632436393, 13.118) [b] 
(0.9402690659278755, 13.119) [b] 
(0.9403009699605452, 13.306) [b] 
(0.9403023588494341, 15.293) [b] 
(0.9403065255161008, 15.332) [b] 
(0.9403685294843548, 15.772) [b] 
(0.9404305334526087, 15.773) [b] 
(0.9404312210440584, 15.779) [b] 
(0.9404319086355081, 15.785) [b] 
(0.9404683669688414, 16.187) [b] 
(0.9404777419688414, 16.397) [b] 
(0.9404841581478784, 20.356) [b] 
(0.9404906767127512, 21.433) [b] 
(0.9405047153470729, 21.913) [b] 
(0.9405187539813946, 21.937) [b] 
(0.9405211845369502, 22.014) [b] 
(0.9405263928702835, 22.246) [b] 
(0.9405404315046052, 23.396) [b] 
(0.9405544701389269, 23.431) [b] 
(0.9405572479167047, 24.194) [b] 
(0.9405617618055936, 24.241) [b] 
(0.9405631506944825, 24.35) [b] 
(0.9405839840278158, 25.706) [b] 
(0.9407010252263177, 26.033) [b] 
(0.9407044034220909, 26.432) [b] 
(0.9407077816178642, 26.45) [b] 
(0.9407111598136374, 26.962) [b] 
(0.9407514563964872, 28.922) [b] 
(0.9407535397298205, 36.821) [b] 
(0.940765596056981, 40.648) [b] 
(0.9407701099458698, 41.239) [b] 
(0.940770457168092, 41.766) [b] 
(0.9407939445788399, 43.986) [b] 
(0.9408210713149511, 46.346) [b] 
(0.9408332240927288, 47.514) [b] 
(0.9408544046482844, 47.568) [b] 
(0.9408564879816177, 47.794) [b] 
(0.9408571824260621, 47.926) [b] 
(0.940858571314951, 48.195) [b] 
(0.9409160006430954, 48.802) [b] 
(0.9409193788388687, 48.804) [b] 
(0.9409227570346419, 48.806) [b] 
(0.9409328916219615, 48.809) [b] 
(0.9409362698177347, 48.873) [b] 
(0.9409531607966007, 48.917) [b] 
(0.9409768081670131, 48.918) [b] 
(0.9409801863627864, 48.934) [b] 
(0.9409835645585596, 48.948) [b] 
(0.9409869427543328, 48.949) [b] 
(0.9410188467870025, 50.366) [b] 
(0.9410507508196723, 50.368) [b] 
(0.9411783669503512, 50.369) [b] 
(0.9412102709830209, 50.37) [b] 
(0.9412421750156906, 50.377) [b] 
(0.9412740790483604, 50.397) [b] 
(0.9413059830810301, 50.399) [b] 
(0.9413378871136998, 50.401) [b] 
(0.9413697911463695, 50.402) [b] 
(0.9413906244797029, 51.005) [b] 
(0.9414309210625527, 52.099) [b] 
(0.9414375182847748, 52.554) [b] 
(0.941437865506997, 52.714) [b] 
(0.9414396016181081, 52.805) [b] 
(0.9414399488403303, 52.981) [b] 
(0.9414634362510782, 53.598) [b] 
(0.9414735708383979, 55.166) [b] 
(0.9414769490341711, 55.219) [b] 
(0.9415088530668408, 56.494) [b] 
(0.9415726611321803, 56.495) [b] 
(0.94160456516485, 56.497) [b] 
(0.9417002772628592, 56.498) [b] 
(0.9417321812955289, 56.509) [b] 
(0.9418278933935381, 56.51) [b] 
(0.9418597974262078, 56.575) [b] 
(0.9418917014588776, 56.61) [b] 
(0.9419236054915473, 56.613) [b] 
(0.9419356618187078, 57.491) [b] 
(0.9419477181458683, 57.738) [b] 
(0.9419597744730288, 58.519) [b] 
(0.9419604689174732, 61.15) [b] 
(0.9419611633619176, 61.411) [b] 
(0.941961857806362, 62.987) [b] 
(0.9419694966952509, 63.039) [b] 
(0.9419760939174731, 63.089) [b] 
(0.9419781772508063, 63.209) [b] 
(0.941992215885128, 64.983) [b] 
(0.9419936810299195, 70.772) [b] 
(0.9419940282521417, 71.652) [b] 
(0.9419957864258915, 80.157) [b] 
(0.9420166197592249, 82.924) [b] 
(0.9420374530925583, 83.039) [b] 
(0.9420582864258916, 83.045) [b] 
(0.9420723250602133, 83.425) [b] 
(0.9420931583935467, 83.569) [b] 
(0.94211399172688, 83.903) [b] 
(0.9421458957595498, 84.804) [b] 
(0.9421789645426185, 88.786) [b] 
(0.9421879032314642, 88.975) [b] 
(0.9421968419203098, 88.977) [b] 
(0.9421982171032092, 88.988) [b] 
(0.9422009674690078, 88.991) [b] 
(0.9422016550604575, 89.056) [b] 
(0.9422023426519072, 89.061) [b] 
(0.9422037178348065, 89.158) [b] 
(0.9422057806091555, 89.161) [b] 
(0.9422085309749542, 89.509) [b] 
(0.9427096128562888, 90.01) [b] 
(0.9427103004477385, 91.512) [b] 
(0.9427109880391882, 91.513) [b] 
(0.9427123632220875, 91.515) [b] 
(0.9428735495534866, 91.706) [b] 
(0.9428970369642344, 95.225) [b] 
(0.9428977245556841, 96.084) [b] 
(0.9428984121471338, 96.097) [b] 
(0.9428990997385834, 101.53) [b] 
(0.9428997873300331, 101.531) [b] 
(0.9429004749214828, 101.534) [b] 
(0.942900822143705, 105.772) [b] 
(0.9429015165881494, 106.104) [b] 
(0.9429018638103716, 114.574) [b] 
(0.9429029054770383, 123.195) [b] 
(0.942907072143705, 123.2) [b] 
(0.9429081138103717, 123.276) [b] 
(0.9429091554770384, 123.515) [b] 
(0.9429112388103716, 128.027) [b] 
(0.9429122804770383, 129.095) [b] 
(0.942912968068488, 132.58) [b] 
(0.942974972036742, 136.263) [b] 
(0.9430369760049959, 136.264) [b] 
(0.9430989799732499, 136.266) [b] 
(0.9431609839415038, 136.28) [b] 
(0.9432229879097578, 136.332) [b] 
(0.9432849918780117, 136.348) [b] 
(0.9433469958462657, 136.36) [b] 
(0.9434089998145196, 137.26) [b] 
(0.9434492963973694, 140.129) [b] 
(0.9435113003656234, 144.783) [b] 
(0.9435226885901992, 147.508) [b] 
(0.943530327479088, 153.108) [b] 
(0.9435320635901991, 153.433) [b] 
(0.9435355358124213, 154.015) [b] 
(0.9435362302568657, 157.853) [b] 
(0.943538313590199, 157.956) [b] 
(0.9435390080346434, 159.138) [b] 
(0.9435393552568656, 161.588) [b] 
(0.9435533938911873, 170.19) [b] 
(0.9435662262492615, 170.274) [b] 
(0.9435943035179047, 171.291) [b] 
(0.9436177909286526, 188.257) [b] 
(0.9436181381508748, 192.789) [b] 
(0.943618485373097, 208.993) [b] 
(0.9436419727838449, 209.635) [b] 
(0.9437233529921782, 223.455) [b] 
(0.9437504797282894, 223.457) [b] 
(0.9438318599366228, 223.562) [b] 
(0.9438589866727339, 223.563) [b] 
(0.9438861134088451, 223.582) [b] 
(0.9439132401449563, 225.012) [b] 
(0.9439340734782896, 230.065) [b] 
(0.9439612002144008, 231.931) [b] 
(0.9440862002144008, 233.376) [b] 
(0.9441695335477341, 233.377) [b] 
(0.9441903668810675, 233.641) [b] 
(0.9442737002144008, 233.642) [b] 
(0.9443153668810675, 233.643) [b] 
(0.9443570335477343, 233.644) [b] 
(0.9443778668810676, 233.645) [b] 
(0.9443842830601047, 235.806) [b] 
(0.9443846302823269, 248.108) [b] 
(0.9443849775045491, 248.352) [b] 
(0.9443856719489935, 248.476) [b] 
(0.9443881025045491, 248.662) [b] 
(0.9443884497267713, 249.208) [b] 
(0.9443898386156602, 251.335) [b] 
(0.9443962547946972, 255.931) [b] 
(0.9444026709737343, 258.807) [b] 
(0.9444090871527714, 259.197) [b] 
(0.944507357593023, 268.311) [b] 
(0.9445354348616662, 268.312) [b] 
(0.944549473495988, 268.321) [b] 
(0.9445775507646312, 268.38) [b] 
(0.9445915893989529, 269.014) [b] 
(0.9446056280332746, 269.149) [b] 
(0.9446196666675963, 269.361) [b] 
(0.944633705301918, 269.381) [b] 
(0.9447179371078479, 271.294) [b] 
(0.9447460143764912, 271.295) [b] 
(0.9447600530108129, 271.309) [b] 
(0.9447881302794562, 271.335) [b] 
(0.9448021689137779, 271.337) [b] 
(0.9448162075480996, 271.4) [b] 
(0.9448675211769995, 276.56) [b] 
(0.9448931779914493, 276.566) [b] 
(0.9449188348058992, 276.594) [b] 
(0.9449444916203491, 276.68) [b] 
(0.944970148434799, 276.684) [b] 
(0.9449841870691207, 279.307) [b] 
(0.9450098438835706, 280.151) [b] 
(0.9450355006980204, 280.96) [b] 
(0.9450611575124703, 287.078) [b] 
(0.9450868143269202, 287.197) [b] 
(0.9451124711413701, 287.354) [b] 
(0.94516378477027, 297.278) [b] 
(0.9451894415847198, 297.289) [b] 
(0.9452150983991697, 297.316) [b] 
(0.9452407552136196, 297.41) [b] 
(0.9452664120280695, 297.414) [b] 
(0.9452920688425194, 301.136) [b] 
(0.9453177256569693, 302.017) [b] 
(0.9453412130677171, 303.621) [b] 
(0.9453415602899393, 308.188) [b] 
(0.9453672171043892, 308.712) [b] 
(0.9453928739188391, 308.8) [b] 
(0.945418530733289, 309.011) [b] 
(0.9454186772477681, 325.187) [b] 
(0.9454421646585159, 329.489) [b] 
(0.9454462902072139, 345.915) [b] 
(0.9454469777986636, 345.916) [b] 
(0.9454483529815629, 345.95) [b] 
(0.9454490405730126, 345.968) [b] 
(0.9454497281644623, 345.982) [b] 
(0.9454504157559119, 346.034) [b] 
(0.9454524785302609, 346.073) [b] 
(0.9454531661217106, 366.331) [b] 
(0.9454538537131603, 366.333) [b] 
(0.9454545481576047, 391.344) [b] 
(0.9455715893561065, 406.218) [b] 
(0.9455977126580919, 439.448) [b] 
(0.9456173051345809, 439.454) [b] 
(0.94563689761107, 439.462) [b] 
(0.945656490087559, 439.464) [b] 
(0.9456630209130554, 439.484) [b] 
(0.9456695517385518, 439.586) [b] 
(0.9456729299343251, 467.696) [b] 
(0.9456796863258714, 473.208) [b] 
(0.9456830645216446, 473.209) [b] 
(0.9456864427174179, 473.214) [b] 
(0.9456898209131911, 473.216) [b] 
(0.9456931991089643, 473.217) [b] 
(0.9456999555005108, 473.32) [b] 
(0.9457067118920572, 473.321) [b] 
(0.9457168464793768, 473.326) [b] 
(0.94572022467515, 473.328) [b] 
(0.9457236028709233, 473.369) [b] 
(0.9457269810666965, 474.895) [b] 
(0.9457371156540161, 474.929) [b] 
(0.9457454489873494, 481) [b] 
(0.9457464906540161, 481.052) [b] 
(0.945747879542905, 481.366) [b] 
(0.9457506573206828, 481.481) [b] 
(0.945751004542905, 481.964) [b] 
(0.9457543827386782, 488.185) [b] 
(0.9457577609344514, 488.232) [b] 
(0.9457611391302246, 488.566) [b] 
(0.9457653057968913, 534.83) [b] 
(0.9457698196857802, 534.894) [b] 
(0.9457712085746691, 535.189) [b] 
(0.9457715557968913, 536.919) [b] 
(0.9457722502413357, 537.774) [b] 
(0.9457725974635579, 540.352) [b] 
(0.9457791160284307, 541.95) [b] 
(0.9457798104728751, 544.813) [b] 
(0.9457863290377478, 551.632) [b] 
(0.94578667625997, 557.962) [b] 
(0.9458148012599701, 563.134) [b] 
(0.9458172318155257, 563.184) [b] 
(0.9458936207044145, 569.712) [b] 
(0.9459498707044145, 569.81) [b] 
(0.9459547318155256, 570.463) [b] 
(0.9459568151488589, 571.983) [b] 
(0.9459578568155256, 576.529) [b] 
(0.9459595929266367, 583.827) [b] 
(0.9459602873710811, 583.872) [b] 
(0.9459606345933033, 584.057) [b] 
(0.9459675790377478, 589.06) [b] 
(0.9459947057738589, 590.988) [b] 
(0.9459981779960811, 591.814) [b] 
(0.9460092891071922, 595.648) [b] 
(0.9460157052862292, 612.401) [b] 
(0.9460484222503899, 747.695) [b] 
(0.9460491098418395, 836.01) [b] 
(0.9460504850247389, 836.011) [b] 
(0.9460518602076382, 836.015) [b] 
(0.9460725963116837, 843.896) [b] 
(0.9460730571139958, 876.358) [b] 
(0.9460870957483175, 909.535) [b] 
(0.9461151730169608, 909.566) [b] 
(0.9461292116512825, 909.667) [b] 
(0.9461432502856042, 909.676) [b] 
(0.9461572889199259, 909.793) [b] 
(0.9461713275542476, 909.986) [b] 
(0.9461853661885693, 911.024) [b] 
(0.9461855127030484, 1002.53) [b] 
(0.9463025539015503, 1003.46) [b] 
(0.9463260413122981, 1013.72) [b] 
(0.946349528723046, 1015.55) [b] 
(0.9463635673573677, 1038.76) [b] 
(0.9463640281596798, 1040.87) [b] 
(0.946364375381902, 1077.42) [b] 
(0.9463914587152353, 1113.67) [b] 
(0.9463963198263464, 1113.73) [b] 
(0.9463966670485686, 1113.91) [b] 
(0.9463977087152353, 1116.42) [b] 
(0.946401875381902, 1116.49) [b] 
(0.9464067364930131, 1116.57) [b] 
(0.9464074309374575, 1118.01) [b] 
(0.9464077781596797, 1119.71) [b] 
(0.9464141943387168, 1132.93) [b] 
(0.9464206105177538, 1132.95) [b] 
(0.9464207570322329, 1451.91) [b] 
(0.9465446097648154, 1588.32) [b] 
(0.9465511283296881, 1588.42) [b] 
(0.9465576468945609, 1588.92) [b] 
(0.946583721154052, 1589.15) [b] 
(0.9465854572651631, 1594.1) [b] 
(0.9465858044873853, 1602.54) [b] 
(0.946592323052258, 1615.85) [b] 
(0.9466314344414946, 1617.32) [b] 
(0.9466379530063673, 1617.34) [b] 
(0.9466444715712401, 1617.73) [b] 
(0.9466493326823512, 1645.24) [b] 
(0.9466701660156845, 1655.17) [b] 
(0.9466897584921736, 1670.16) [b] 
(0.94669628931767, 1670.18) [b] 
(0.9467028201431664, 1670.34) [b] 
(0.9467035145876108, 1685.84) [b] 
(0.9467243479209442, 1695.04) [b] 
(0.9467451812542775, 1695.12) [b] 
(0.9470576812542776, 1695.13) [b] 
(0.9470993479209443, 1695.14) [b] 
(0.9471201812542777, 1695.18) [b] 
(0.9473701812542776, 1695.28) [b] 
(0.947391014587611, 1695.29) [b] 
(0.9474118479209443, 1696.82) [b] 
(0.9474326812542777, 1697.44) [b] 
(0.9474330284764999, 1697.68) [b] 
(0.947434764587611, 1700.02) [b] 
(0.9474351118098332, 1700.85) [b] 
(0.9476017784765, 1705.87) [b] 
(0.9476226118098333, 1705.88) [b] 
(0.9476434451431667, 1705.89) [b] 
(0.9476642784765, 1705.95) [b] 
(0.9476851118098334, 1705.97) [b] 
(0.9477059451431668, 1705.98) [b] 
(0.9477267784765001, 1706.24) [b] 
(0.9477476118098335, 1706.25) [b] 
(0.9477710992205813, 1718.48) [b] 
(0.9477945866313292, 1721.47) [b] 
(0.9478362532979959, 1762.33) [b] 
(0.9478570866313293, 1762.34) [b] 
(0.9478779199646626, 1762.35) [b] 
(0.947898753297996, 1762.78) [b] 
(0.9479195866313294, 1762.8) [b] 
(0.9479404199646627, 1762.84) [b] 
(0.9479612532979961, 1769.73) [b] 
(0.9481240137146628, 1816.17) [b] 
(0.948151140450774, 1816.18) [b] 
(0.9481782671868851, 1816.19) [b] 
(0.9482053939229963, 1816.2) [b] 
(0.9482325206591075, 1816.39) [b] 
(0.9482596473952186, 1816.45) [b] 
(0.9483139008674409, 1816.66) [b] 
(0.948802182117441, 1816.67) [b] 
(0.9488293088535522, 1816.68) [b] 
(0.94901919600633, 1816.77) [b] 
(0.9490734494785523, 1817.12) [b] 
(0.9491005762146635, 1817.14) [b] 
(0.9491548296868857, 1818.08) [b] 
(0.9492362098952191, 1822.77) [b] 
(0.9492570432285524, 1838.27) [b] 
(0.9492827000430023, 1851.1) [b] 
(0.9493369535152246, 1854.49) [b] 
(0.949418333723558, 1880.84) [b] 
(0.9494454604596692, 1880.85) [b] 
(0.9494465021263359, 1907.94) [b] 
(0.949473628862447, 1912.07) [b] 
(0.9494826566402248, 1914.6) [b] 
(0.949504878862447, 1914.69) [b] 
(0.949508003862447, 1916.73) [b] 
(0.9495086983068914, 1917.11) [b] 
(0.9495093927513358, 1917.93) [b] 
(0.9495107816402247, 1925.22) [b] 
(0.9495121705291136, 1931.32) [b] 
(0.9495125177513358, 1961.89) [b] 
(0.949512864973558, 1961.95) [b] 
(0.9495135594180024, 1963.43) [b] 
(0.9495146010846691, 1964.13) [b] 
(0.949540257899119, 1969.68) [b] 
(0.9495673846352302, 2021.73) [b] 
(0.9496216381074525, 2029.06) [b] 
(0.9496487648435636, 2029.07) [b] 
(0.9497572717880081, 2029.13) [b] 
(0.9497843985241192, 2029.38) [b] 
(0.9497984371584409, 2073.72) [b] 
(0.9498124757927626, 2083.79) [b] 
(0.9498265144270843, 2085.1) [b] 
(0.9498278896099837, 2198.96) [b] 
(0.9498285772014333, 2198.99) [b] 
(0.949829264792883, 2199.16) [b] 
(0.9498433034272047, 2259.49) [b] 
(0.9498689602416546, 2315.09) [b] 
(0.9498946170561045, 2315.11) [b] 
(0.9499972443139041, 2315.14) [b] 
(0.950022901128354, 2316.16) [b] 
(0.9500485579428039, 2316.17) [b] 
(0.9500742147572537, 2327.52) [b] 
(0.9500745619794759, 2397.74) [b] 
(0.950076298090587, 2397.79) [b] 
(0.9500766453128092, 2397.89) [b] 
(0.9500769925350314, 2398.44) [b] 
(0.9501026493494813, 2454.41) [b] 
(0.9501283061639312, 2454.42) [b] 
(0.9502309334217308, 2454.45) [b] 
(0.9502565902361807, 2455.51) [b] 
(0.9502822470506306, 2455.52) [b] 
(0.9502884970506306, 2467.43) [b] 
(0.9502919692728528, 2467.44) [b] 
(0.950294052606186, 2467.47) [b] 
(0.9503197094206359, 2467.51) [b] 
(0.9503210983095248, 2467.79) [b] 
(0.9503349871984137, 2469.39) [b] 
(0.9503391538650803, 2469.4) [b] 
(0.950343320531747, 2469.42) [b] 
(0.9503440149761914, 2469.64) [b] 
(0.95034676534199, 2481.38) [b] 
(0.9503481405248894, 2483.24) [b] 
(0.9503495157077887, 2483.27) [b] 
(0.9503502032992384, 2483.28) [b] 
(0.9503736907099862, 2487.18) [b] 
(0.9503743851544306, 2489.3) [b] 
(0.9503978725651785, 2490.95) [b] 
(0.9503985601566282, 2496.79) [b] 
(0.9504006229309772, 2498.6) [b] 
(0.9504013105224268, 2498.72) [b] 
(0.9504019981138765, 2499.17) [b] 
(0.9504026857053262, 2506.06) [b] 
(0.9504030329275484, 2518.82) [b] 
(0.9504037273719927, 2521.67) [b] 
(0.9504870607053261, 2609.93) [b] 
(0.9504904389010993, 2803.97) [b] 
(0.950491126492549, 2840.57) [b] 
(0.9504938768583476, 2840.58) [b] 
(0.9504959396326966, 2840.63) [b] 
(0.9505026960242431, 2847.34) [b] 
(0.9505298227603542, 2894.72) [b] 
(0.9505569494964654, 2894.74) [b] 
(0.9508282168575766, 2895.23) [b] 
(0.9508553435936877, 2895.25) [b] 
(0.9509367238020211, 2895.35) [b] 
(0.9510452307464656, 2895.36) [b] 
(0.9510473140797989, 2908.65) [b] 
(0.95105217519091, 2908.7) [b] 
(0.9511355085242434, 2979.34) [b] 
(0.9511980085242435, 2979.35) [b] 
},{(0.9403091041666664, 0.001) [c] 
(0.9403091041666664, 3.510360354166668) [c] 
(0.9403091041666664, 3600) [c] 
}}}{legend pos=north west}}
% 	\subfloat[depth=9]{\cactus{Average Accuracy}{CPU time}{\budalg, \murtree, \cart}{{{(0.9200019225068743, 0) [a] 
(0.9266353892671741, 0.01) [a] 
(0.9317151809338406, 0.02) [a] 
(0.934541014267174, 0.03) [a] 
(0.9364792781560626, 0.04) [a] 
(0.9438495559338403, 0.05) [a] 
(0.9438605976005068, 0.06) [a] 
(0.9442853892671736, 0.07) [a] 
(0.9445437226005069, 0.08) [a] 
(0.9445676809338402, 0.09) [a] 
(0.9447670559338399, 0.1) [a] 
(0.9450195559338391, 0.11) [a] 
(0.9452937226005056, 0.12) [a] 
(0.9453672642671722, 0.13) [a] 
(0.9454118476005053, 0.14) [a] 
(0.9455551809338384, 0.15) [a] 
(0.9455891392671716, 0.16) [a] 
(0.9456141392671715, 0.17) [a] 
(0.945647055933838, 0.18) [a] 
(0.9456660142671711, 0.19) [a] 
(0.945669139267171, 0.2) [a] 
(0.9456780976005041, 0.21) [a] 
(0.9456808059338374, 0.22) [a] 
(0.945683097600504, 0.23) [a] 
(0.9456874726005038, 0.24) [a] 
(0.9457366392671704, 0.25) [a] 
(0.9457391392671705, 0.26) [a] 
(0.9458228892671705, 0.28) [a] 
(0.9494979995194198, 0.29) [a] 
(0.9494988328527532, 0.3) [a] 
(0.9495042495194198, 0.31) [a] 
(0.9496502911860865, 0.32) [a] 
(0.9497679995194198, 0.33) [a] 
(0.9499367495194199, 0.34) [a] 
(0.9499394578527532, 0.35) [a] 
(0.9499442495194199, 0.36) [a] 
(0.9499800828527533, 0.37) [a] 
(0.95008424951942, 0.4) [a] 
(0.9502015411860868, 0.45) [a] 
(0.9502019578527534, 0.46) [a] 
(0.9502290411860868, 0.47) [a] 
(0.9503748745194203, 0.48) [a] 
(0.9503752911860869, 0.49) [a] 
(0.9503769578527536, 0.5) [a] 
(0.9548036499045187, 0.51) [a] 
(0.9548040665711853, 0.52) [a] 
(0.9548042749045187, 0.53) [a] 
(0.9548317749045188, 0.55) [a] 
(0.9548526082378521, 0.58) [a] 
(0.9548840665711855, 0.59) [a] 
(0.9549378165711856, 0.6) [a] 
(0.9549919832378523, 0.61) [a] 
(0.9553119832378523, 0.62) [a] 
(0.9553940665711856, 0.63) [a] 
(0.9554176082378523, 0.65) [a] 
(0.9554240665711856, 0.66) [a] 
(0.9554513582378523, 0.67) [a] 
(0.9554932332378524, 0.68) [a] 
(0.9555140665711858, 0.71) [a] 
(0.9555411499045191, 0.72) [a] 
(0.9555417749045191, 0.76) [a] 
(0.9556055249045191, 0.79) [a] 
(0.9556290665711857, 0.9) [a] 
(0.955643024904519, 0.94) [a] 
(0.9556494832378524, 1.02) [a] 
(0.955663649904519, 1.03) [a] 
(0.9556651082378523, 1.06) [a] 
(0.9556665665711856, 1.07) [a] 
(0.9564808333333306, 1.09) [a] 
(0.9565041666666639, 1.18) [a] 
(0.9565277083333306, 1.23) [a] 
(0.9565512499999972, 1.24) [a] 
(0.9565793749999971, 1.26) [a] 
(0.9566027083333305, 1.27) [a] 
(0.9566047916666638, 1.35) [a] 
(0.9566095833333305, 1.37) [a] 
(0.9566122916666638, 1.39) [a] 
(0.9566402083333305, 1.46) [a] 
(0.9566916666666638, 1.49) [a] 
(0.9566922916666638, 1.5) [a] 
(0.9567437499999971, 1.51) [a] 
(0.9567472916666637, 1.59) [a] 
(0.9567508333333303, 1.6) [a] 
(0.956754999999997, 1.61) [a] 
(0.9567577083333303, 1.62) [a] 
(0.9567612499999969, 1.63) [a] 
(0.9567645833333301, 1.64) [a] 
(0.9567666666666634, 1.65) [a] 
(0.9567674999999968, 1.66) [a] 
(0.9567737499999968, 1.67) [a] 
(0.9568214583333301, 1.69) [a] 
(0.9568235416666634, 1.75) [a] 
(0.95685208333333, 1.76) [a] 
(0.9568541666666633, 1.77) [a] 
(0.9568881249999966, 1.78) [a] 
(0.9568914583333299, 1.79) [a] 
(0.9568970833333299, 1.8) [a] 
(0.9568979166666632, 1.82) [a] 
(0.9569120833333299, 1.85) [a] 
(0.9569391666666632, 1.98) [a] 
(0.9569395833333298, 1.99) [a] 
(0.9569399999999965, 2.02) [a] 
(0.9569487499999965, 2.05) [a] 
(0.9569552083333298, 2.07) [a] 
(0.9569566666666631, 2.2) [a] 
(0.9569572916666631, 2.24) [a] 
(0.9569579166666631, 2.25) [a] 
(0.9569593749999964, 2.32) [a] 
(0.9569658333333297, 2.51) [a] 
(0.956967916666663, 2.55) [a] 
(0.9569766666666629, 2.57) [a] 
(0.9569906249999962, 2.62) [a] 
(0.9570735416666629, 2.63) [a] 
(0.9571033333333295, 2.64) [a] 
(0.9571062499999962, 2.65) [a] 
(0.9571083333333295, 2.66) [a] 
(0.9571116666666628, 2.67) [a] 
(0.9571131249999961, 2.68) [a] 
(0.9571154166666628, 2.69) [a] 
(0.9571158333333294, 2.7) [a] 
(0.9571164583333294, 2.73) [a] 
(0.957116874999996, 2.74) [a] 
(0.9571189583333293, 2.75) [a] 
(0.957120624999996, 2.76) [a] 
(0.9571227083333292, 2.77) [a] 
(0.9571239583333292, 2.78) [a] 
(0.9571245833333292, 2.79) [a] 
(0.9571256249999959, 2.8) [a] 
(0.9571262499999958, 2.82) [a] 
(0.9571295833333292, 2.84) [a] 
(0.9571299999999958, 2.85) [a] 
(0.9571304166666624, 2.92) [a] 
(0.9571316666666624, 2.93) [a] 
(0.9571337499999957, 2.94) [a] 
(0.957135208333329, 2.95) [a] 
(0.9571366666666623, 2.98) [a] 
(0.9571372916666623, 3.01) [a] 
(0.9571381249999956, 3.18) [a] 
(0.957138333333329, 3.33) [a] 
(0.9571456249999957, 3.34) [a] 
(0.9571691666666623, 3.54) [a] 
(0.9571947916666623, 3.73) [a] 
(0.957195833333329, 3.77) [a] 
(0.9572279166666624, 3.78) [a] 
(0.9572379166666625, 3.8) [a] 
(0.9572520833333291, 4.03) [a] 
(0.9572527083333291, 4.14) [a] 
(0.9572943749999958, 4.19) [a] 
(0.9573568749999959, 4.25) [a] 
(0.9573777083333292, 4.26) [a] 
(0.9573781249999959, 4.48) [a] 
(0.9573989583333292, 4.53) [a] 
(0.9573991666666626, 4.55) [a] 
(0.957399374999996, 4.6) [a] 
(0.9574133333333292, 4.77) [a] 
(0.9574174999999958, 4.79) [a] 
(0.9574189583333291, 4.8) [a] 
(0.9574191666666625, 4.98) [a] 
(0.9574312499999958, 5.03) [a] 
(0.9574322916666625, 5.29) [a] 
(0.9574531249999959, 5.34) [a] 
(0.9574770833333293, 5.43) [a] 
(0.9574897916666626, 5.48) [a] 
(0.957489999999996, 5.5) [a] 
(0.957515624999996, 5.53) [a] 
(0.957541249999996, 5.64) [a] 
(0.957541874999996, 6) [a] 
(0.9575433333333293, 6.01) [a] 
(0.9575572916666626, 6.08) [a] 
(0.9575714583333292, 6.09) [a] 
(0.9575718749999959, 6.26) [a] 
(0.9575722916666625, 6.32) [a] 
(0.9575724999999958, 6.33) [a] 
(0.9575729166666624, 6.34) [a] 
(0.9575735416666624, 6.54) [a] 
(0.957573958333329, 6.55) [a] 
(0.9575868749999957, 6.77) [a] 
(0.9575872916666623, 6.78) [a] 
(0.9576012499999956, 6.8) [a] 
(0.9576016666666622, 6.81) [a] 
(0.9576020833333289, 6.91) [a] 
(0.9576033333333288, 6.95) [a] 
(0.9576241666666622, 7.19) [a] 
(0.9576247916666621, 7.2) [a] 
(0.9576256249999955, 7.21) [a] 
(0.9576262499999955, 7.22) [a] 
(0.9576402083333287, 7.26) [a] 
(0.9576437499999954, 7.28) [a] 
(0.9579399999999952, 7.31) [a] 
(0.9579412499999952, 7.33) [a] 
(0.9579427083333285, 7.44) [a] 
(0.9579831249999952, 7.73) [a] 
(0.9579845833333285, 7.82) [a] 
(0.9579852083333285, 7.83) [a] 
(0.9579858333333284, 7.84) [a] 
(0.9579866666666618, 7.85) [a] 
(0.9579883333333284, 7.99) [a] 
(0.9579887499999951, 8) [a] 
(0.9579891666666617, 8.03) [a] 
(0.957989374999995, 8.13) [a] 
(0.957989999999995, 8.55) [a] 
(0.9579914583333283, 9.06) [a] 
(0.957991874999995, 9.41) [a] 
(0.9579922916666617, 9.46) [a] 
(0.9579927083333283, 9.86) [a] 
(0.958023124999995, 10.09) [a] 
(0.9580241666666617, 10.11) [a] 
(0.9580245833333283, 10.3) [a] 
(0.9580260416666616, 10.74) [a] 
(0.9580274999999949, 10.84) [a] 
(0.9580289583333282, 11.49) [a] 
(0.9580316666666616, 11.5) [a] 
(0.958052499999995, 11.7) [a] 
(0.9580733333333283, 11.71) [a] 
(0.9580735416666617, 11.76) [a] 
(0.958079999999995, 12.05) [a] 
(0.9580802083333284, 12.12) [a] 
(0.9580804166666618, 12.27) [a] 
(0.9580808333333284, 12.3) [a] 
(0.958081249999995, 12.33) [a] 
(0.9580816666666616, 12.37) [a] 
(0.9580820833333282, 12.38) [a] 
(0.9580822916666616, 12.4) [a] 
(0.9580827083333282, 12.43) [a] 
(0.9580831249999948, 12.44) [a] 
(0.9580833333333282, 12.46) [a] 
(0.9580837499999948, 12.48) [a] 
(0.9580841666666614, 12.49) [a] 
(0.9580872916666614, 12.5) [a] 
(0.958087708333328, 12.51) [a] 
(0.9580881249999946, 12.53) [a] 
(0.958088333333328, 12.54) [a] 
(0.9580887499999946, 12.55) [a] 
(0.9580891666666612, 12.6) [a] 
(0.9580895833333278, 14.34) [a] 
(0.9580897916666612, 14.39) [a] 
(0.9580899999999946, 14.41) [a] 
(0.9580908333333279, 14.44) [a] 
(0.9580914583333279, 14.79) [a] 
(0.9580918749999945, 14.8) [a] 
(0.9580924999999945, 14.82) [a] 
(0.9580929166666611, 14.86) [a] 
(0.9580939583333277, 14.88) [a] 
(0.9580943749999943, 14.89) [a] 
(0.9580945833333276, 14.9) [a] 
(0.9580956249999943, 14.91) [a] 
(0.9580964583333277, 15.08) [a] 
(0.958097916666661, 15.2) [a] 
(0.9580981249999944, 15.53) [a] 
(0.9580987499999943, 16.15) [a] 
(0.958099166666661, 16.71) [a] 
(0.9581006249999943, 17.27) [a] 
(0.9581020833333276, 17.35) [a] 
(0.9581022916666609, 17.79) [a] 
(0.9581093749999943, 17.99) [a] 
(0.9581095833333276, 18.29) [a] 
(0.958109791666661, 18.45) [a] 
(0.9581099999999944, 18.66) [a] 
(0.9581108333333277, 18.89) [a] 
(0.9581110416666611, 19.14) [a] 
(0.9581114583333277, 19.15) [a] 
(0.9581118749999943, 19.16) [a] 
(0.9581120833333276, 19.17) [a] 
(0.9581131249999943, 19.19) [a] 
(0.9581139583333277, 19.22) [a] 
(0.9581145833333277, 19.23) [a] 
(0.9581149999999943, 19.29) [a] 
(0.9581152083333276, 19.3) [a] 
(0.958116041666661, 19.33) [a] 
(0.9581170833333277, 19.37) [a] 
(0.958117291666661, 19.4) [a] 
(0.9581183333333277, 19.41) [a] 
(0.9581520833333279, 19.42) [a] 
(0.9581591666666612, 19.43) [a] 
(0.9581597916666612, 19.5) [a] 
(0.9581604166666612, 19.53) [a] 
(0.9581624999999946, 19.72) [a] 
(0.9581635416666613, 19.73) [a] 
(0.9581649999999946, 19.9) [a] 
(0.9581660416666613, 20.02) [a] 
(0.9581664583333279, 20.47) [a] 
(0.9581666666666613, 20.52) [a] 
(0.9581670833333279, 20.53) [a] 
(0.9581681249999946, 20.56) [a] 
(0.9581693749999945, 20.67) [a] 
(0.9581697916666612, 20.73) [a] 
(0.9581718749999945, 21.18) [a] 
(0.9581724999999944, 21.24) [a] 
(0.958172916666661, 21.25) [a] 
(0.9581733333333277, 21.27) [a] 
(0.9581739583333276, 21.37) [a] 
(0.9581756249999943, 22.61) [a] 
(0.9581760416666609, 22.62) [a] 
(0.9581781249999942, 22.65) [a] 
(0.9581795833333275, 23.15) [a] 
(0.9581802083333275, 23.16) [a] 
(0.9581810416666608, 23.18) [a] 
(0.9581843749999941, 23.19) [a] 
(0.9581849999999941, 23.23) [a] 
(0.9581864583333274, 23.9) [a] 
(0.9581879166666607, 24.19) [a] 
(0.9581885416666607, 24.2) [a] 
(0.9581889583333273, 24.23) [a] 
(0.9581893749999939, 24.25) [a] 
(0.9581908333333272, 24.3) [a] 
(0.9581910416666606, 24.32) [a] 
(0.9581914583333272, 24.33) [a] 
(0.9581920833333272, 24.34) [a] 
(0.9581929166666605, 24.35) [a] 
(0.9581931249999939, 24.39) [a] 
(0.9581941666666606, 24.44) [a] 
(0.9581945833333272, 24.47) [a] 
(0.9581949999999938, 24.49) [a] 
(0.9581952083333272, 24.5) [a] 
(0.9581956249999938, 24.51) [a] 
(0.9581960416666604, 24.58) [a] 
(0.9581962499999938, 24.77) [a] 
(0.9581966666666604, 24.92) [a] 
(0.9582174999999937, 24.94) [a] 
(0.958231458333327, 25.03) [a] 
(0.958232083333327, 25.55) [a] 
(0.9582322916666604, 25.69) [a] 
(0.958232708333327, 25.71) [a] 
(0.9582356249999936, 25.8) [a] 
(0.9582360416666602, 25.82) [a] 
(0.9582372916666602, 25.86) [a] 
(0.9582374999999935, 25.99) [a] 
(0.9582379166666601, 26.04) [a] 
(0.9582383333333268, 26.92) [a] 
(0.9582412499999934, 27.02) [a] 
(0.9582427083333267, 27.1) [a] 
(0.9582456249999933, 27.49) [a] 
(0.9582460416666599, 27.6) [a] 
(0.9582470833333265, 27.61) [a] 
(0.9582479166666598, 27.63) [a] 
(0.9582485416666597, 27.7) [a] 
(0.9582502083333263, 27.71) [a] 
(0.9582504166666597, 27.72) [a] 
(0.9582508333333263, 27.78) [a] 
(0.9582512499999929, 27.97) [a] 
(0.9583683333333263, 28.22) [a] 
(0.9583691666666596, 28.28) [a] 
(0.9583706249999929, 28.42) [a] 
(0.9583733333333262, 28.43) [a] 
(0.9583747916666595, 29.91) [a] 
(0.9583752083333261, 29.98) [a] 
(0.9583781249999928, 30.07) [a] 
(0.9583793749999927, 31.63) [a] 
(0.9584074999999926, 31.75) [a] 
(0.9584214583333259, 31.83) [a] 
(0.9584220833333259, 33.69) [a] 
(0.9584362499999926, 34.49) [a] 
(0.9584368749999925, 34.59) [a] 
(0.9584383333333258, 34.6) [a] 
(0.9584416666666592, 34.61) [a] 
(0.9584458333333258, 34.63) [a] 
(0.9584597916666591, 34.77) [a] 
(0.9584608333333258, 36.2) [a] 
(0.9584754166666591, 36.21) [a] 
(0.9584756249999925, 36.29) [a] 
(0.9584764583333258, 36.3) [a] 
(0.9584766666666592, 36.33) [a] 
(0.9584770833333258, 36.34) [a] 
(0.9584781249999925, 36.38) [a] 
(0.9584839583333259, 36.4) [a] 
(0.9584906249999926, 36.41) [a] 
(0.958490833333326, 37.16) [a] 
(0.9584910416666593, 37.17) [a] 
(0.9584918749999927, 37.18) [a] 
(0.958492083333326, 37.19) [a] 
(0.9585087499999927, 37.22) [a] 
(0.9585091666666593, 38.66) [a] 
(0.9585093749999927, 38.67) [a] 
(0.9585118749999927, 38.68) [a] 
(0.9585122916666593, 38.71) [a] 
(0.9585124999999927, 38.76) [a] 
(0.9585129166666593, 38.77) [a] 
(0.9585133333333259, 38.79) [a] 
(0.9585137499999925, 38.8) [a] 
(0.9585149999999925, 38.82) [a] 
(0.9585158333333258, 38.83) [a] 
(0.9585160416666592, 38.87) [a] 
(0.9585164583333258, 38.89) [a] 
(0.9585189583333258, 38.95) [a] 
(0.9585195833333258, 39.16) [a] 
(0.9585199999999924, 39.88) [a] 
(0.9585216666666591, 40.9) [a] 
(0.9585231249999924, 40.91) [a] 
(0.958523541666659, 41.08) [a] 
(0.9585252083333257, 41.09) [a] 
(0.9585262499999924, 41.1) [a] 
(0.9585270833333257, 41.13) [a] 
(0.9585281249999924, 42.9) [a] 
(0.9585291666666591, 42.93) [a] 
(0.9585324999999925, 42.94) [a] 
(0.9585335416666592, 42.96) [a] 
(0.9585339583333258, 43.06) [a] 
(0.9585374999999925, 43.16) [a] 
(0.9585408333333258, 43.17) [a] 
(0.9585429166666593, 43.62) [a] 
(0.958543958333326, 43.99) [a] 
(0.9585449999999927, 44) [a] 
(0.958546458333326, 44.03) [a] 
(0.9585466666666593, 44.26) [a] 
(0.9585470833333259, 45.78) [a] 
(0.9585477083333259, 46.2) [a] 
(0.9585487499999926, 46.3) [a] 
(0.9585629166666593, 46.44) [a] 
(0.9585768749999926, 46.57) [a] 
(0.9586049999999925, 47.04) [a] 
(0.9586470833333257, 47.06) [a] 
(0.9586504166666591, 47.28) [a] 
(0.9586508333333257, 47.6) [a] 
(0.9586743749999923, 48.63) [a] 
(0.9586979166666589, 48.71) [a] 
(0.9586983333333255, 48.76) [a] 
(0.9587122916666588, 48.77) [a] 
(0.9587591666666588, 48.8) [a] 
(0.9587827083333255, 48.83) [a] 
(0.9587833333333254, 51.42) [a] 
(0.9587974999999921, 52.28) [a] 
(0.9588114583333254, 52.29) [a] 
(0.958811874999992, 52.79) [a] 
(0.9588122916666586, 52.81) [a] 
(0.958812499999992, 52.82) [a] 
(0.9588129166666586, 52.83) [a] 
(0.9588143749999919, 52.97) [a] 
(0.9588154166666586, 53.03) [a] 
(0.9588158333333252, 53.07) [a] 
(0.9588172916666585, 53.19) [a] 
(0.9588210416666585, 53.2) [a] 
(0.9588216666666585, 53.24) [a] 
(0.9588220833333251, 53.25) [a] 
(0.9588231249999917, 53.26) [a] 
(0.9588235416666583, 53.27) [a] 
(0.9588239583333249, 53.29) [a] 
(0.9588241666666583, 53.3) [a] 
(0.9588245833333249, 53.33) [a] 
(0.9588249999999915, 53.46) [a] 
(0.9588252083333249, 53.47) [a] 
(0.9588254166666582, 53.53) [a] 
(0.9588258333333248, 53.56) [a] 
(0.9588262499999914, 53.85) [a] 
(0.9588266666666581, 54.36) [a] 
(0.9588268749999914, 54.68) [a] 
(0.9588277083333248, 54.69) [a] 
(0.9588279166666581, 54.75) [a] 
(0.9588281249999915, 55.04) [a] 
(0.9588285416666581, 55.16) [a] 
(0.9588289583333247, 55.23) [a] 
(0.958829791666658, 55.31) [a] 
(0.9588302083333247, 55.67) [a] 
(0.9588537499999913, 56.05) [a] 
(0.9588818749999912, 56.27) [a] 
(0.9589099999999912, 56.41) [a] 
(0.9589239583333244, 56.42) [a] 
(0.9589379166666577, 56.5) [a] 
(0.9589660416666577, 57.75) [a] 
(0.9589941666666576, 57.76) [a] 
(0.9589945833333242, 57.8) [a] 
(0.9589952083333242, 58.11) [a] 
(0.9589956249999908, 58.15) [a] 
(0.9589997916666575, 58.27) [a] 
(0.9590008333333242, 58.41) [a] 
(0.9590018749999909, 58.45) [a] 
(0.9590020833333243, 60.06) [a] 
(0.959003124999991, 60.15) [a] 
(0.9590033333333243, 61.55) [a] 
(0.9590037499999909, 66.77) [a] 
(0.9590058333333242, 67.49) [a] 
(0.9590079166666576, 67.5) [a] 
(0.9590089583333243, 67.51) [a] 
(0.959009999999991, 67.65) [a] 
(0.9590110416666577, 68.5) [a] 
(0.9590116666666577, 74.52) [a] 
(0.959012499999991, 74.56) [a] 
(0.9590139583333244, 74.82) [a] 
(0.9590149999999911, 74.83) [a] 
(0.9590152083333244, 74.88) [a] 
(0.959015624999991, 75.57) [a] 
(0.959021874999991, 76.31) [a] 
(0.9590339583333244, 76.63) [a] 
(0.9590460416666577, 76.66) [a] 
(0.9591631249999911, 79.52) [a] 
(0.9591679166666578, 80.47) [a] 
(0.9591683333333244, 81.54) [a] 
(0.959168749999991, 83.15) [a] 
(0.9591689583333244, 83.17) [a] 
(0.9591708333333243, 83.25) [a] 
(0.9591712499999909, 83.26) [a] 
(0.9591716666666575, 83.27) [a] 
(0.9591718749999909, 83.28) [a] 
(0.9591743749999909, 83.31) [a] 
(0.9591749999999909, 83.35) [a] 
(0.9591758333333241, 83.41) [a] 
(0.9591760416666575, 83.46) [a] 
(0.9591762499999908, 83.49) [a] 
(0.9591766666666575, 83.53) [a] 
(0.9591770833333241, 83.56) [a] 
(0.9591772916666574, 83.75) [a] 
(0.959177708333324, 83.91) [a] 
(0.9591818749999907, 84) [a] 
(0.9591822916666574, 84.09) [a] 
(0.959182708333324, 84.1) [a] 
(0.9591829166666573, 84.13) [a] 
(0.9591847916666574, 84.14) [a] 
(0.9591849999999907, 84.28) [a] 
(0.9591852083333241, 84.82) [a] 
(0.9591866666666574, 88.74) [a] 
(0.959187083333324, 89.4) [a] 
(0.959188333333324, 90.07) [a] 
(0.9591891666666573, 91) [a] 
(0.9591897916666573, 91.3) [a] 
(0.959196458333324, 91.95) [a] 
(0.959197083333324, 92.41) [a] 
(0.959203333333324, 97.27) [a] 
(0.959203958333324, 97.49) [a] 
(0.9592081249999905, 98.13) [a] 
(0.9592083333333239, 99.32) [a] 
(0.9592087499999905, 99.33) [a] 
(0.9592093749999905, 99.34) [a] 
(0.9592097916666571, 99.37) [a] 
(0.9592102083333237, 99.38) [a] 
(0.9592112499999904, 99.44) [a] 
(0.9592114583333238, 99.49) [a] 
(0.9592139583333237, 99.78) [a] 
(0.9592145833333237, 99.79) [a] 
(0.9592162499999903, 99.81) [a] 
(0.9592170833333237, 100.42) [a] 
(0.9592174999999903, 100.88) [a] 
(0.9592179166666569, 100.91) [a] 
(0.9592181249999903, 101.77) [a] 
(0.9592206249999903, 104.51) [a] 
(0.9592210416666569, 104.56) [a] 
(0.9592212499999903, 104.64) [a] 
(0.9592220833333236, 104.65) [a] 
(0.9592239583333236, 104.66) [a] 
(0.959224166666657, 104.71) [a] 
(0.9592249999999903, 104.74) [a] 
(0.9592252083333237, 104.75) [a] 
(0.9592256249999903, 104.76) [a] 
(0.9592260416666569, 104.97) [a] 
(0.9592262499999903, 105.83) [a] 
(0.9592664583333236, 106.22) [a] 
(0.9592806249999902, 106.23) [a] 
(0.9592864583333236, 106.24) [a] 
(0.9592868749999902, 106.47) [a] 
(0.9593808333333235, 106.68) [a] 
(0.9594041666666568, 106.69) [a] 
(0.9594277083333235, 106.77) [a] 
(0.9594512499999901, 106.89) [a] 
(0.9594747916666567, 107.07) [a] 
(0.9595216666666567, 107.08) [a] 
(0.9595452083333234, 107.5) [a] 
(0.9595499999999899, 107.89) [a] 
(0.9595514583333232, 107.9) [a] 
(0.9595520833333232, 108.12) [a] 
(0.9595547916666565, 108.13) [a] 
(0.9595568749999898, 108.15) [a] 
(0.9595583333333231, 108.25) [a] 
(0.9595597916666564, 108.55) [a] 
(0.9595610416666563, 108.56) [a] 
(0.9595618749999897, 108.57) [a] 
(0.9595631249999896, 108.63) [a] 
(0.959563958333323, 108.64) [a] 
(0.9595656249999895, 108.65) [a] 
(0.9595662499999895, 108.66) [a] 
(0.9595666666666561, 108.67) [a] 
(0.9595670833333227, 108.73) [a] 
(0.9595672916666561, 108.74) [a] 
(0.9595677083333227, 108.78) [a] 
(0.9595683333333227, 108.84) [a] 
(0.959569166666656, 108.91) [a] 
(0.959569791666656, 108.97) [a] 
(0.9595702083333226, 109.12) [a] 
(0.959571041666656, 109.32) [a] 
(0.9595716666666559, 109.46) [a] 
(0.9595722916666559, 109.47) [a] 
(0.9595737499999892, 109.72) [a] 
(0.9595758333333225, 109.73) [a] 
(0.9595772916666558, 109.85) [a] 
(0.9595785416666558, 109.96) [a] 
(0.9595793749999891, 110.37) [a] 
(0.9595814583333224, 110.38) [a] 
(0.9595820833333224, 110.41) [a] 
(0.9595827083333224, 110.47) [a] 
(0.9595835416666557, 110.75) [a] 
(0.959585624999989, 110.93) [a] 
(0.9595883333333223, 110.96) [a] 
(0.9595889583333223, 111.49) [a] 
(0.9595897916666556, 111.68) [a] 
(0.959589999999989, 111.85) [a] 
(0.9595906249999889, 112.19) [a] 
(0.9595912499999889, 112.26) [a] 
(0.9595920833333222, 112.33) [a] 
(0.9595927083333222, 112.43) [a] 
(0.9595941666666555, 112.49) [a] 
(0.9595995833333222, 112.61) [a] 
(0.9596004166666555, 112.66) [a] 
(0.9596010416666555, 112.84) [a] 
(0.9596016666666555, 112.85) [a] 
(0.9596024999999888, 113.16) [a] 
(0.9596031249999888, 113.27) [a] 
(0.9596058333333221, 113.37) [a] 
(0.9596066666666554, 113.9) [a] 
(0.9596079166666553, 114.04) [a] 
(0.9596087499999887, 114.3) [a] 
(0.9596093749999887, 114.52) [a] 
(0.9596097916666553, 116.31) [a] 
(0.9596104166666553, 121.01) [a] 
(0.9596133333333219, 121.07) [a] 
(0.9596139583333219, 122.57) [a] 
(0.9596145833333218, 122.6) [a] 
(0.9596154166666552, 122.64) [a] 
(0.9596160416666552, 122.72) [a] 
(0.9596166666666551, 123.46) [a] 
(0.9596174999999885, 123.51) [a] 
(0.9596181249999884, 123.52) [a] 
(0.9596187499999884, 124.62) [a] 
(0.9596195833333218, 125.19) [a] 
(0.9596208333333217, 125.53) [a] 
(0.959621666666655, 125.54) [a] 
(0.9596237499999883, 126.37) [a] 
(0.9596249999999883, 126.38) [a] 
(0.9596258333333216, 126.41) [a] 
(0.9596264583333216, 126.86) [a] 
(0.9596270833333216, 126.98) [a] 
(0.9596279166666549, 127.03) [a] 
(0.9596299999999882, 127.09) [a] 
(0.9596364583333216, 128.66) [a] 
(0.9596370833333215, 129.58) [a] 
(0.9596377083333215, 130.7) [a] 
(0.9596385416666549, 135) [a] 
(0.9596389583333215, 145.56) [a] 
(0.9596393749999881, 145.67) [a] 
(0.9596395833333214, 145.7) [a] 
(0.9596404166666548, 145.71) [a] 
(0.9596435416666548, 145.79) [a] 
(0.9596437499999881, 145.84) [a] 
(0.9596441666666548, 145.85) [a] 
(0.9596445833333214, 145.87) [a] 
(0.9596447916666547, 145.9) [a] 
(0.9596452083333213, 145.91) [a] 
(0.959645624999988, 146.01) [a] 
(0.9596458333333213, 146.02) [a] 
(0.9596462499999879, 146.03) [a] 
(0.9596472916666546, 146.08) [a] 
(0.9596708333333213, 147.03) [a] 
(0.9596710416666546, 147.67) [a] 
(0.9596714583333212, 148.04) [a] 
(0.9596720833333212, 149.07) [a] 
(0.9596729166666546, 149.43) [a] 
(0.9596733333333212, 150.67) [a] 
(0.9596737499999878, 150.71) [a] 
(0.9596739583333211, 150.72) [a] 
(0.9596743749999878, 150.74) [a] 
(0.9596747916666544, 150.75) [a] 
(0.9596754166666543, 150.83) [a] 
(0.9596760416666543, 150.88) [a] 
(0.9596764583333209, 150.92) [a] 
(0.9596774999999876, 151) [a] 
(0.9596779166666543, 151.05) [a] 
(0.9596785416666542, 151.08) [a] 
(0.9596789583333208, 151.17) [a] 
(0.9596791666666542, 151.24) [a] 
(0.9596795833333208, 151.26) [a] 
(0.9596799999999874, 156.4) [a] 
(0.9596806249999874, 161.95) [a] 
(0.9596808333333208, 164.12) [a] 
(0.9596810416666541, 165.8) [a] 
(0.9596818749999875, 165.92) [a] 
(0.9596820833333208, 165.93) [a] 
(0.9596829166666542, 165.94) [a] 
(0.9596831249999875, 165.98) [a] 
(0.9596839583333208, 166) [a] 
(0.9596841666666541, 166.02) [a] 
(0.9596849999999875, 166.03) [a] 
(0.9596856249999874, 166.98) [a] 
(0.9596864583333208, 169.93) [a] 
(0.9596866666666541, 174.09) [a] 
(0.9596902083333209, 174.1) [a] 
(0.9596937499999876, 174.12) [a] 
(0.9596989583333209, 174.13) [a] 
(0.9596991666666542, 174.16) [a] 
(0.9596995833333208, 174.42) [a] 
(0.9596997916666542, 175.73) [a] 
(0.9597008333333209, 175.74) [a] 
(0.9597012499999875, 175.76) [a] 
(0.9597016666666541, 175.77) [a] 
(0.9597018749999875, 176.41) [a] 
(0.9597022916666541, 176.42) [a] 
(0.9597027083333207, 184.26) [a] 
(0.9597033333333207, 184.29) [a] 
(0.9597037499999873, 184.32) [a] 
(0.9597039583333207, 184.33) [a] 
(0.9597043749999873, 184.35) [a] 
(0.9597047916666539, 184.4) [a] 
(0.9597054166666539, 184.53) [a] 
(0.9597058333333205, 184.59) [a] 
(0.9597060416666539, 184.68) [a] 
(0.9597064583333205, 184.72) [a] 
(0.9597068749999871, 184.74) [a] 
(0.9597070833333204, 185.02) [a] 
(0.9597074999999871, 185.69) [a] 
(0.9597077083333204, 186.54) [a] 
(0.959708124999987, 186.95) [a] 
(0.9597089583333204, 188.33) [a] 
(0.959709374999987, 188.37) [a] 
(0.959709999999987, 188.38) [a] 
(0.9597108333333203, 188.39) [a] 
(0.9597110416666537, 188.42) [a] 
(0.9597114583333203, 188.57) [a] 
(0.9597127083333202, 188.6) [a] 
(0.9597131249999868, 188.61) [a] 
(0.9597137499999868, 188.62) [a] 
(0.9597141666666534, 188.63) [a] 
(0.9597143749999868, 188.88) [a] 
(0.9597147916666534, 192.71) [a] 
(0.95971520833332, 192.72) [a] 
(0.9597154166666534, 192.75) [a] 
(0.95971583333332, 192.76) [a] 
(0.9597162499999866, 194.22) [a] 
(0.9597168749999866, 200.74) [a] 
(0.95972020833332, 205.38) [a] 
(0.9597268749999867, 212.34) [a] 
(0.9597272916666533, 218.57) [a] 
(0.9597274999999866, 218.59) [a] 
(0.9597279166666532, 218.6) [a] 
(0.9597283333333199, 218.66) [a] 
(0.9597299999999865, 218.69) [a] 
(0.9597341666666532, 218.7) [a] 
(0.9597347916666532, 218.71) [a] 
(0.9597352083333198, 218.77) [a] 
(0.9597358333333198, 218.79) [a] 
(0.959736666666653, 218.8) [a] 
(0.9597368749999864, 218.86) [a] 
(0.959737291666653, 219.3) [a] 
(0.9597377083333196, 219.34) [a] 
(0.959737916666653, 220.55) [a] 
(0.9597383333333196, 220.88) [a] 
(0.9597387499999862, 221.59) [a] 
(0.9597389583333196, 221.63) [a] 
(0.9597397916666528, 222.26) [a] 
(0.9597404166666528, 222.3) [a] 
(0.9597408333333194, 222.31) [a] 
(0.9597420833333193, 222.32) [a] 
(0.9597439583333194, 222.33) [a] 
(0.9597441666666527, 222.52) [a] 
(0.9597449999999861, 222.56) [a] 
(0.9597452083333194, 222.71) [a] 
(0.9597456249999861, 225.9) [a] 
(0.9597460416666527, 239.16) [a] 
(0.9597477083333193, 239.17) [a] 
(0.959748124999986, 239.26) [a] 
(0.9597487499999859, 239.27) [a] 
(0.9597491666666526, 239.3) [a] 
(0.9597493749999859, 239.31) [a] 
(0.9597497916666525, 239.32) [a] 
(0.9597502083333191, 239.33) [a] 
(0.9597504166666525, 239.35) [a] 
(0.9597512499999858, 239.36) [a] 
(0.9597514583333192, 239.37) [a] 
(0.9597518749999858, 239.38) [a] 
(0.9597529166666525, 239.42) [a] 
(0.9597533333333191, 239.43) [a] 
(0.9597535416666525, 239.46) [a] 
(0.9597545833333192, 239.48) [a] 
(0.9597549999999858, 239.54) [a] 
(0.9597554166666524, 239.55) [a] 
(0.9597556249999858, 239.56) [a] 
(0.9597560416666524, 239.59) [a] 
(0.959756458333319, 239.86) [a] 
(0.9597566666666524, 240.36) [a] 
(0.959757083333319, 241.08) [a] 
(0.9597574999999856, 244.8) [a] 
(0.9597579166666522, 245.94) [a] 
(0.9597583333333188, 245.95) [a] 
(0.9597585416666522, 246.05) [a] 
(0.9597589583333188, 246.09) [a] 
(0.9597593749999854, 246.25) [a] 
(0.959759791666652, 246.71) [a] 
(0.9597608333333186, 246.74) [a] 
(0.9597618749999852, 247.98) [a] 
(0.9597620833333186, 248.12) [a] 
(0.9597629166666519, 249.97) [a] 
(0.9597662499999853, 258.57) [a] 
(0.9597666666666519, 258.66) [a] 
(0.9597672916666519, 259.83) [a] 
(0.9597679166666518, 259.89) [a] 
(0.9597685416666518, 260.03) [a] 
(0.9597693749999852, 260.55) [a] 
(0.9597697916666518, 260.78) [a] 
(0.9597729166666518, 260.79) [a] 
(0.9597733333333184, 260.8) [a] 
(0.9597739583333184, 260.81) [a] 
(0.959774374999985, 260.86) [a] 
(0.9597745833333183, 260.87) [a] 
(0.9597754166666516, 260.9) [a] 
(0.9597756249999849, 260.93) [a] 
(0.9597764583333183, 260.94) [a] 
(0.9597766666666516, 260.96) [a] 
(0.9597770833333182, 260.97) [a] 
(0.9597774999999849, 261.03) [a] 
(0.9597777083333182, 261.1) [a] 
(0.9597781249999848, 261.11) [a] 
(0.9597785416666514, 262.09) [a] 
(0.959778958333318, 262.73) [a] 
(0.9597791666666514, 262.75) [a] 
(0.9597797916666514, 264.34) [a] 
(0.9597806249999847, 264.42) [a] 
(0.9597812499999847, 264.53) [a] 
(0.9597816666666513, 264.54) [a] 
(0.9597881249999847, 264.55) [a] 
(0.959788333333318, 264.69) [a] 
(0.9597891666666514, 274.23) [a] 
(0.9597893749999847, 274.24) [a] 
(0.9597897916666513, 274.28) [a] 
(0.9597904166666513, 274.31) [a] 
(0.9597912499999847, 274.33) [a] 
(0.9597918749999846, 274.34) [a] 
(0.9597922916666513, 274.35) [a] 
(0.959793333333318, 274.37) [a] 
(0.9597935416666513, 276.72) [a] 
(0.9597939583333179, 278.08) [a] 
(0.9597943749999845, 283.18) [a] 
(0.9597945833333179, 283.19) [a] 
(0.9597949999999845, 283.21) [a] 
(0.9597954166666511, 283.27) [a] 
(0.9597956249999845, 283.73) [a] 
(0.9597960416666511, 283.96) [a] 
(0.9597964583333177, 288.65) [a] 
(0.9597968749999843, 288.66) [a] 
(0.9597974999999843, 288.68) [a] 
(0.9597979166666509, 290.51) [a] 
(0.9597989583333176, 290.88) [a] 
(0.9597999999999843, 291.12) [a] 
(0.9598256249999844, 297.88) [a] 
(0.9598512499999844, 298.08) [a] 
(0.9598768749999844, 298.92) [a] 
(0.9599024999999844, 299.12) [a] 
(0.9599283333333177, 299.2) [a] 
(0.959954166666651, 300.19) [a] 
(0.959954791666651, 300.51) [a] 
(0.959956666666651, 300.52) [a] 
(0.9599568749999844, 300.53) [a] 
(0.959957291666651, 300.54) [a] 
(0.9599577083333176, 300.57) [a] 
(0.959957916666651, 300.59) [a] 
(0.9599583333333176, 300.6) [a] 
(0.9599589583333176, 300.63) [a] 
(0.9599593749999842, 300.64) [a] 
(0.9599597916666508, 300.65) [a] 
(0.9599604166666508, 300.67) [a] 
(0.9599608333333174, 300.68) [a] 
(0.9599620833333173, 300.71) [a] 
(0.959962499999984, 300.72) [a] 
(0.9599641666666506, 300.86) [a] 
(0.959964999999984, 300.88) [a] 
(0.9599660416666507, 300.9) [a] 
(0.9599666666666506, 300.97) [a] 
(0.9599670833333173, 301.01) [a] 
(0.9599672916666506, 301.54) [a] 
(0.9599677083333172, 301.56) [a] 
(0.9599702083333171, 301.61) [a] 
(0.9599704166666505, 302.14) [a] 
(0.9599708333333171, 302.17) [a] 
(0.9599712499999837, 302.53) [a] 
(0.9599714583333171, 302.8) [a] 
(0.9600889583333171, 303.26) [a] 
(0.9600893749999837, 303.31) [a] 
(0.9600895833333171, 303.35) [a] 
(0.9600899999999837, 303.38) [a] 
(0.9600904166666503, 303.39) [a] 
(0.9600906249999837, 303.43) [a] 
(0.9600910416666503, 303.46) [a] 
(0.9600914583333169, 303.48) [a] 
(0.9600916666666502, 304.13) [a] 
(0.9600918749999836, 304.53) [a] 
(0.9600922916666502, 307.9) [a] 
(0.9600927083333168, 311.52) [a] 
(0.9600929166666502, 315.03) [a] 
(0.9602099999999836, 316.84) [a] 
(0.9602106249999836, 325.36) [a] 
(0.9602110416666502, 330.09) [a] 
(0.9602116666666501, 330.1) [a] 
(0.9602122916666501, 330.13) [a] 
(0.9602127083333167, 330.15) [a] 
(0.9602143749999834, 332.06) [a] 
(0.9602164583333167, 336.91) [a] 
(0.9602168749999833, 336.92) [a] 
(0.9602174999999833, 336.94) [a] 
(0.9602179166666499, 337.06) [a] 
(0.9602185416666499, 337.13) [a] 
(0.9602189583333165, 337.18) [a] 
(0.9602191666666499, 337.62) [a] 
(0.9602195833333165, 337.63) [a] 
(0.9602452083333165, 339.86) [a] 
(0.9602456249999831, 340.02) [a] 
(0.9602712499999831, 341.33) [a] 
(0.9602714583333165, 343.1) [a] 
(0.9602743749999831, 345.67) [a] 
(0.9602752083333165, 346.57) [a] 
(0.9602754166666498, 346.6) [a] 
(0.9602758333333165, 346.93) [a] 
(0.9602762499999831, 346.99) [a] 
(0.9602764583333164, 347) [a] 
(0.960276874999983, 352.94) [a] 
(0.960277499999983, 352.96) [a] 
(0.9602779166666496, 352.97) [a] 
(0.9602783333333162, 353.01) [a] 
(0.9602785416666496, 353.02) [a] 
(0.9602789583333162, 353.04) [a] 
(0.9602793749999828, 353.05) [a] 
(0.9602795833333162, 353.07) [a] 
(0.9602799999999828, 353.11) [a] 
(0.9602827083333162, 353.16) [a] 
(0.9602835416666494, 353.19) [a] 
(0.9602837499999828, 353.25) [a] 
(0.9602841666666494, 353.68) [a] 
(0.960284583333316, 353.72) [a] 
(0.9602856249999827, 353.86) [a] 
(0.9602858333333161, 354.28) [a] 
(0.9602862499999827, 356.31) [a] 
(0.9602868749999827, 356.32) [a] 
(0.9603149999999826, 356.46) [a] 
(0.9603164583333159, 358.41) [a] 
(0.9603174999999826, 366.01) [a] 
(0.9603181249999826, 374.92) [a] 
(0.9603185416666492, 375.03) [a] 
(0.9603187499999826, 375.09) [a] 
(0.9603195833333159, 375.1) [a] 
(0.9603197916666493, 375.18) [a] 
(0.9603206249999826, 375.21) [a] 
(0.960320833333316, 375.27) [a] 
(0.9603212499999826, 375.28) [a] 
(0.9603218749999826, 375.41) [a] 
(0.9603227083333159, 375.55) [a] 
(0.9603233333333159, 375.86) [a] 
(0.9603235416666492, 376.17) [a] 
(0.9603239583333159, 376.19) [a] 
(0.9603304166666492, 376.43) [a] 
(0.9603370833333159, 376.49) [a] 
(0.9603379166666492, 377.27) [a] 
(0.9603381249999826, 377.28) [a] 
(0.9603391666666493, 377.37) [a] 
(0.9603395833333159, 377.38) [a] 
(0.9603402083333159, 377.45) [a] 
(0.9603406249999825, 377.95) [a] 
(0.9603418749999825, 377.96) [a] 
(0.9603422916666491, 378.02) [a] 
(0.9603433333333157, 378.03) [a] 
(0.960344791666649, 380.91) [a] 
(0.9603468749999823, 380.96) [a] 
(0.9603474999999823, 380.97) [a] 
(0.9603479166666489, 381.19) [a] 
(0.9603483333333155, 381.38) [a] 
(0.9603487499999821, 382.19) [a] 
(0.9603493749999821, 382.25) [a] 
(0.9603508333333154, 387.75) [a] 
(0.9603529166666487, 389.64) [a] 
(0.9603541666666486, 395.17) [a] 
(0.960355624999982, 395.21) [a] 
(0.9603558333333153, 396.23) [a] 
(0.9603562499999819, 396.24) [a] 
(0.9603566666666485, 401.41) [a] 
(0.9603570833333152, 401.97) [a] 
(0.9603572916666485, 402) [a] 
(0.9603577083333151, 402.1) [a] 
(0.9603581249999817, 402.11) [a] 
(0.9603583333333151, 402.46) [a] 
(0.9603591666666484, 402.57) [a] 
(0.9603595833333151, 403.51) [a] 
(0.9603599999999817, 412.24) [a] 
(0.9603604166666483, 414.72) [a] 
(0.9603606249999816, 415.69) [a] 
(0.9603610416666483, 423.16) [a] 
(0.9603614583333149, 424.11) [a] 
(0.9603616666666482, 424.2) [a] 
(0.9603624999999816, 424.22) [a] 
(0.9603629166666482, 426.57) [a] 
(0.9603639583333149, 426.58) [a] 
(0.9603641666666483, 426.6) [a] 
(0.9603645833333149, 426.62) [a] 
(0.9603649999999815, 426.73) [a] 
(0.9603656249999815, 426.74) [a] 
(0.9603683333333147, 435.25) [a] 
(0.9603689583333147, 435.26) [a] 
(0.960369791666648, 435.27) [a] 
(0.960370416666648, 435.35) [a] 
(0.960371041666648, 435.8) [a] 
(0.9603724999999813, 435.81) [a] 
(0.9603729166666479, 439.85) [a] 
(0.9603756249999813, 439.86) [a] 
(0.9603762499999813, 439.91) [a] 
(0.9603764583333146, 441.88) [a] 
(0.9603768749999813, 444.55) [a] 
(0.960377916666648, 444.56) [a] 
(0.9603785416666479, 448.96) [a] 
(0.9603799999999812, 449.28) [a] 
(0.9603808333333146, 453.72) [a] 
(0.9603814583333146, 459.45) [a] 
(0.9603824999999813, 459.46) [a] 
(0.9603829166666479, 459.52) [a] 
(0.9603835416666479, 459.53) [a] 
(0.9603839583333145, 459.54) [a] 
(0.9603841666666478, 459.57) [a] 
(0.9603845833333144, 459.6) [a] 
(0.9603852083333144, 459.69) [a] 
(0.9603858333333144, 460.85) [a] 
(0.9603860416666478, 460.89) [a] 
(0.9603866666666477, 460.95) [a] 
(0.9603870833333144, 462.04) [a] 
(0.9603877083333143, 470.55) [a] 
(0.9603881249999809, 473.41) [a] 
(0.9603885416666476, 473.42) [a] 
(0.9603887499999809, 473.81) [a] 
(0.9603895833333143, 474.25) [a] 
(0.9603902083333142, 488.29) [a] 
(0.9603908333333142, 493.13) [a] 
(0.9603935416666475, 493.15) [a] 
(0.9603964583333141, 493.29) [a] 
(0.9603966666666475, 494.1) [a] 
(0.9603970833333141, 494.63) [a] 
(0.9603977083333141, 495.26) [a] 
(0.9603997916666474, 495.37) [a] 
(0.9604008333333139, 497.37) [a] 
(0.9604010416666473, 497.56) [a] 
(0.9604014583333139, 500.27) [a] 
(0.9604018749999805, 503.38) [a] 
(0.9604020833333139, 503.39) [a] 
(0.9604024999999805, 503.96) [a] 
(0.9605193749999805, 505.05) [a] 
(0.9605197916666471, 517.68) [a] 
(0.9605204166666471, 521.92) [a] 
(0.9605210416666471, 521.93) [a] 
(0.9605218749999804, 521.94) [a] 
(0.9605220833333138, 522.06) [a] 
(0.9605239583333138, 522.21) [a] 
(0.9605247916666472, 523.84) [a] 
(0.9605249999999805, 523.85) [a] 
(0.9605254166666471, 523.86) [a] 
(0.9605258333333138, 523.88) [a] 
(0.9605260416666471, 523.94) [a] 
(0.9605264583333137, 524.97) [a] 
(0.9605268749999804, 526.82) [a] 
(0.9605277083333137, 532.41) [a] 
(0.960527916666647, 532.42) [a] 
(0.9605281249999804, 543.02) [a] 
(0.9605287499999804, 548.69) [a] 
(0.960529166666647, 548.71) [a] 
(0.9605293749999804, 549.74) [a] 
(0.9605308333333137, 551.84) [a] 
(0.9605312499999803, 573.96) [a] 
(0.9605314583333137, 574.04) [a] 
(0.9605318749999803, 574.05) [a] 
(0.9605322916666469, 574.07) [a] 
(0.9605324999999802, 574.83) [a] 
(0.9605329166666469, 574.85) [a] 
(0.9605333333333135, 582.12) [a] 
(0.9605343749999802, 582.17) [a] 
(0.9605347916666468, 582.73) [a] 
(0.9605352083333134, 593.49) [a] 
(0.9605354166666468, 593.72) [a] 
(0.9605360416666467, 594.61) [a] 
(0.9605366666666467, 594.62) [a] 
(0.9605374999999801, 594.7) [a] 
(0.96053812499998, 594.71) [a] 
(0.9605389583333134, 598.64) [a] 
(0.9605391666666467, 598.68) [a] 
(0.9605393749999801, 599.22) [a] 
(0.9605397916666467, 611.68) [a] 
(0.9605402083333133, 614.34) [a] 
(0.9605412499999799, 614.35) [a] 
(0.9605433333333132, 614.36) [a] 
(0.9605441666666465, 614.42) [a] 
(0.9605443749999799, 614.45) [a] 
(0.9605447916666465, 614.59) [a] 
(0.9605458333333132, 615.22) [a] 
(0.9605464583333132, 615.23) [a] 
(0.9605472916666465, 615.28) [a] 
(0.9605474999999799, 615.35) [a] 
(0.9605479166666465, 615.36) [a] 
(0.9605483333333131, 615.58) [a] 
(0.9605487499999797, 615.59) [a] 
(0.9605489583333131, 615.62) [a] 
(0.9605491666666465, 617.38) [a] 
(0.9605495833333131, 621.31) [a] 
(0.9605499999999797, 623.47) [a] 
(0.9605504166666463, 635.02) [a] 
(0.9605506249999797, 635.03) [a] 
(0.9605510416666463, 635.07) [a] 
(0.9605514583333129, 635.08) [a] 
(0.9605516666666463, 635.09) [a] 
(0.9605520833333129, 635.34) [a] 
(0.9605524999999795, 645.32) [a] 
(0.9605533333333128, 654.53) [a] 
(0.9605539583333128, 654.88) [a] 
(0.9605554166666461, 654.9) [a] 
(0.9605566666666461, 655.08) [a] 
(0.9605577083333128, 655.32) [a] 
(0.9605579166666461, 655.35) [a] 
(0.9605583333333128, 656.26) [a] 
(0.9605585416666461, 658.98) [a] 
(0.9605589583333127, 658.99) [a] 
(0.9605593749999793, 659.36) [a] 
(0.9605595833333127, 662.18) [a] 
(0.9605599999999793, 662.19) [a] 
(0.9605608333333127, 668.86) [a] 
(0.9605612499999793, 669.67) [a] 
(0.9605614583333126, 669.96) [a] 
(0.9605624999999793, 673.34) [a] 
(0.960562916666646, 674.98) [a] 
(0.9605631249999793, 675.01) [a] 
(0.9605635416666459, 676.14) [a] 
(0.9605639583333125, 682.14) [a] 
(0.9605643749999792, 688.63) [a] 
(0.9605647916666458, 689.04) [a] 
(0.9605654166666457, 689.09) [a] 
(0.9605656249999791, 689.19) [a] 
(0.9605664583333123, 689.29) [a] 
(0.960566874999979, 689.94) [a] 
(0.9605670833333123, 690.07) [a] 
(0.9606295833333124, 690.26) [a] 
(0.9606504166666457, 690.3) [a] 
(0.9606508333333124, 694.23) [a] 
(0.9607133333333124, 698.04) [a] 
(0.9607139583333124, 710.52) [a] 
(0.9607141666666458, 719.41) [a] 
(0.9607145833333124, 719.44) [a] 
(0.960714999999979, 720.51) [a] 
(0.9607152083333124, 720.52) [a] 
(0.9607166666666457, 734.29) [a] 
(0.9607170833333123, 740.87) [a] 
(0.9607172916666457, 740.88) [a] 
(0.9607191666666457, 769.13) [a] 
(0.9607193749999791, 769.18) [a] 
(0.9607197916666457, 770.21) [a] 
(0.9607202083333123, 775.57) [a] 
(0.9607410416666456, 776.51) [a] 
(0.9607416666666456, 782.85) [a] 
(0.9607420833333122, 803.99) [a] 
(0.9607422916666456, 804.04) [a] 
(0.9607427083333122, 804.05) [a] 
(0.9607431249999788, 804.19) [a] 
(0.9607433333333122, 806.63) [a] 
(0.9607437499999788, 809.71) [a] 
(0.9607441666666454, 858.08) [a] 
(0.960744583333312, 858.36) [a] 
(0.9607447916666454, 858.39) [a] 
(0.960745208333312, 858.62) [a] 
(0.9607456249999786, 858.63) [a] 
(0.9607460416666452, 869.05) [a] 
(0.9607464583333118, 869.25) [a] 
(0.9607466666666452, 869.27) [a] 
(0.9607470833333118, 869.33) [a] 
(0.9607474999999784, 882.77) [a] 
(0.9607477083333118, 882.78) [a] 
(0.9607481249999784, 882.85) [a] 
(0.960748541666645, 882.86) [a] 
(0.9607487499999784, 883.15) [a] 
(0.9607493749999784, 884.88) [a] 
(0.9607499999999783, 885.06) [a] 
(0.9607508333333117, 885.11) [a] 
(0.9607514583333117, 885.41) [a] 
(0.9607518749999783, 889.43) [a] 
(0.9607522916666449, 889.45) [a] 
(0.9607524999999782, 889.47) [a] 
(0.9607529166666449, 889.51) [a] 
(0.9607533333333115, 889.7) [a] 
(0.9607672916666448, 898.77) [a] 
(0.9607679166666447, 899.68) [a] 
(0.960781874999978, 899.69) [a] 
(0.9607827083333114, 899.7) [a] 
(0.9607833333333113, 899.79) [a] 
(0.960797499999978, 899.84) [a] 
(0.9608114583333113, 905.78) [a] 
(0.9608256249999779, 905.79) [a] 
(0.9608260416666445, 905.94) [a] 
(0.9608399999999778, 906.3) [a] 
(0.9608539583333111, 909.78) [a] 
(0.9608541666666445, 915.33) [a] 
(0.9608545833333111, 915.35) [a] 
(0.9608549999999777, 915.59) [a] 
(0.9608552083333111, 926.97) [a] 
(0.9608556249999777, 926.98) [a] 
(0.9608560416666443, 927.86) [a] 
(0.9608564583333109, 944.31) [a] 
(0.9608566666666443, 945.42) [a] 
(0.9608570833333109, 945.52) [a] 
(0.9608577083333109, 954.81) [a] 
(0.9608581249999775, 954.82) [a] 
(0.9608583333333108, 954.83) [a] 
(0.9608585416666442, 962.69) [a] 
(0.9608589583333108, 968.44) [a] 
(0.9608595833333108, 968.49) [a] 
(0.9608599999999774, 968.5) [a] 
(0.9608608333333106, 972.49) [a] 
(0.9608618749999773, 972.5) [a] 
(0.9608622916666439, 972.51) [a] 
(0.9608629166666439, 972.54) [a] 
(0.9608643749999772, 972.55) [a] 
(0.9608647916666438, 972.56) [a] 
(0.9608654166666438, 972.62) [a] 
(0.9608660416666438, 972.63) [a] 
(0.9608668749999771, 972.64) [a] 
(0.9608693749999772, 972.65) [a] 
(0.9608697916666438, 972.74) [a] 
(0.9608702083333104, 972.79) [a] 
(0.960870624999977, 972.84) [a] 
(0.9608708333333104, 973.28) [a] 
(0.960871249999977, 973.34) [a] 
(0.9608716666666436, 973.35) [a] 
(0.9608722916666436, 973.37) [a] 
(0.9608727083333102, 973.47) [a] 
(0.9608731249999768, 985.63) [a] 
(0.9608733333333102, 985.92) [a] 
(0.9608737499999768, 986.52) [a] 
(0.9608741666666434, 989.8) [a] 
(0.9608747916666434, 994.21) [a] 
(0.9608749999999767, 994.77) [a] 
(0.9608754166666433, 994.79) [a] 
(0.96087583333331, 994.89) [a] 
(0.9608760416666433, 995.19) [a] 
(0.9608764583333099, 995.78) [a] 
(0.9608768749999765, 996.1) [a] 
(0.9608770833333099, 996.34) [a] 
(0.9608774999999765, 997.27) [a] 
(0.9608779166666431, 998.89) [a] 
(0.9608781249999765, 1005.1) [a] 
(0.9608785416666431, 1005.2) [a] 
(0.9608787499999765, 1007.5) [a] 
(0.9608789583333098, 1016.5) [a] 
(0.9608791666666432, 1018.8) [a] 
(0.9608820833333097, 1019) [a] 
(0.9608822916666431, 1019.2) [a] 
(0.9608827083333097, 1019.4) [a] 
(0.960883541666643, 1019.8) [a] 
(0.960884166666643, 1019.9) [a] 
(0.9608845833333096, 1022.6) [a] 
(0.9608849999999762, 1029.6) [a] 
(0.9608885416666428, 1032.5) [a] 
(0.9608912499999761, 1032.9) [a] 
(0.9608927083333094, 1036.8) [a] 
(0.9608929166666428, 1062.9) [a] 
(0.9608937499999761, 1067.7) [a] 
(0.9608945833333093, 1073.1) [a] 
(0.9608966666666426, 1073.2) [a] 
(0.9608972916666426, 1088.6) [a] 
(0.9609114583333093, 1089.7) [a] 
(0.9609135416666426, 1090.7) [a] 
(0.9609156249999758, 1091.2) [a] 
(0.9609170833333092, 1110) [a] 
(0.9609174999999758, 1117.5) [a] 
(0.9609179166666424, 1136.3) [a] 
(0.960918333333309, 1201.9) [a] 
(0.960918958333309, 1202) [a] 
(0.9609191666666423, 1204.4) [a] 
(0.960919583333309, 1230.8) [a] 
(0.9609210416666423, 1277.9) [a] 
(0.9609212499999756, 1302.9) [a] 
(0.9609216666666422, 1303.7) [a] 
(0.9609233333333088, 1350.7) [a] 
(0.9609237499999754, 1350.8) [a] 
(0.9609239583333088, 1350.9) [a] 
(0.9609252083333087, 1368.5) [a] 
(0.9609260416666421, 1373.6) [a] 
(0.9609266666666421, 1373.8) [a] 
(0.960927291666642, 1373.9) [a] 
(0.9609287499999754, 1374) [a] 
(0.960929166666642, 1380.6) [a] 
(0.9609431249999753, 1383.7) [a] 
(0.9609570833333085, 1385.4) [a] 
(0.9609712499999752, 1386.8) [a] 
(0.9609852083333085, 1389.3) [a] 
(0.9609993749999751, 1389.4) [a] 
(0.9609997916666417, 1390.3) [a] 
(0.9609999999999751, 1390.5) [a] 
(0.9610004166666417, 1398.5) [a] 
(0.9610008333333083, 1419.1) [a] 
(0.9610043749999749, 1429.3) [a] 
(0.9610085416666415, 1430) [a] 
(0.9610091666666415, 1430.2) [a] 
(0.9610097916666415, 1431.7) [a] 
(0.9610112499999748, 1431.8) [a] 
(0.9610114583333081, 1432.5) [a] 
(0.9610118749999748, 1447.4) [a] 
(0.9610122916666414, 1448.1) [a] 
(0.961012708333308, 1449.8) [a] 
(0.9610137499999746, 1466.6) [a] 
(0.9610139583333079, 1466.8) [a] 
(0.9610147916666413, 1491.2) [a] 
(0.9610154166666413, 1494.8) [a] 
(0.9610160416666412, 1506.6) [a] 
(0.9610164583333078, 1506.7) [a] 
(0.9610170833333078, 1511.1) [a] 
(0.9610172916666412, 1529.1) [a] 
(0.9611343749999746, 1540.8) [a] 
(0.9611483333333078, 1553.4) [a] 
(0.9611622916666411, 1557) [a] 
(0.9611624999999745, 1565.4) [a] 
(0.9611629166666411, 1584.2) [a] 
(0.9611631249999745, 1606) [a] 
(0.9611645833333078, 1615.2) [a] 
(0.9611652083333078, 1615.5) [a] 
(0.9611666666666411, 1615.7) [a] 
(0.9611681249999744, 1628.5) [a] 
(0.9611687499999744, 1688.7) [a] 
(0.9611693749999743, 1688.9) [a] 
(0.9611708333333077, 1691) [a] 
(0.961172291666641, 1693.8) [a] 
(0.961172916666641, 1712) [a] 
(0.9611735416666409, 1737.8) [a] 
(0.9611743749999743, 1737.9) [a] 
(0.9611745833333076, 1792.2) [a] 
(0.961174791666641, 1793) [a] 
(0.961175416666641, 1800.1) [a] 
(0.9611758333333076, 1809.1) [a] 
(0.9611764583333076, 1820.2) [a] 
(0.9611906249999742, 1987.7) [a] 
(0.9612045833333075, 1988) [a] 
(0.9612187499999741, 1989) [a] 
(0.9612466666666407, 1995.3) [a] 
(0.9612747916666406, 1995.4) [a] 
(0.961298124999974, 2117) [a] 
(0.9612985416666406, 2193.2) [a] 
(0.9612989583333073, 2196.5) [a] 
(0.9613002083333072, 2220.3) [a] 
(0.9613006249999738, 2220.4) [a] 
(0.9613010416666404, 2221) [a] 
(0.9613016666666404, 2230) [a] 
(0.961302083333307, 2230.3) [a] 
(0.9613022916666404, 2243.3) [a] 
(0.961302708333307, 2243.6) [a] 
(0.9613031249999736, 2243.9) [a] 
(0.9613041666666403, 2296.2) [a] 
(0.9613047916666403, 2312.1) [a] 
(0.961305833333307, 2345.6) [a] 
(0.961306458333307, 2345.7) [a] 
(0.9613068749999736, 2346.2) [a] 
(0.9613072916666402, 2358.3) [a] 
(0.9613074999999736, 2359.2) [a] 
(0.9613079166666402, 2365.4) [a] 
(0.9613083333333068, 2522.5) [a] 
(0.9613085416666401, 2667.2) [a] 
(0.9613089583333068, 2684.8) [a] 
(0.9613095833333067, 2892.6) [a] 
(0.9613104166666401, 2922.7) [a] 
(0.9613108333333067, 3042.7) [a] 
(0.96132479166664, 3064.7) [a] 
(0.9613389583333066, 3064.8) [a] 
(0.9613529166666399, 3142) [a] 
(0.96138104166664, 3216.4) [a] 
(0.9613816666666399, 3343.4) [a] 
(0.9613822916666399, 3343.6) [a] 
(0.9613831249999732, 3343.8) [a] 
(0.9613837499999732, 3384.4) [a] 
(0.9613977083333065, 3433.3) [a] 
(0.9614118749999732, 3460.9) [a] 
(0.9614327083333065, 3477.7) [a] 
(0.9614535416666399, 3477.8) [a] 
(0.9614952083333066, 3478.1) [a] 
(0.9615368749999733, 3478.3) [a] 
(0.9615374999999733, 3484.1) [a] 
(0.9615583333333066, 3491.5) [a] 
(0.96157916666664, 3491.7) [a] 
(0.9615999999999734, 3492.9) [a] 
(0.9616208333333067, 3498.2) [a] 
(0.9616416666666401, 3512.7) [a] 
(0.9616624999999734, 3512.9) [a] 
},{(0.8358251991712193, 0) [b] 
(0.875817128502228, 0.001) [b] 
(0.888819875722926, 0.002) [b] 
(0.8937210161167515, 0.003) [b] 
(0.8981338233213114, 0.004) [b] 
(0.9013146337593991, 0.005) [b] 
(0.9024984583801634, 0.006) [b] 
(0.9032370897057654, 0.007) [b] 
(0.904214712481939, 0.008) [b] 
(0.9048904817057728, 0.009) [b] 
(0.9053926017521443, 0.01) [b] 
(0.9056470824548902, 0.011) [b] 
(0.906475340930739, 0.012) [b] 
(0.9074947999280464, 0.013) [b] 
(0.9077159298581299, 0.014) [b] 
(0.907776554838793, 0.015) [b] 
(0.9086585100412636, 0.016) [b] 
(0.9093891105751846, 0.017) [b] 
(0.9095356109409832, 0.018) [b] 
(0.9098159060798722, 0.019) [b] 
(0.9098499067693915, 0.02) [b] 
(0.9105192690275724, 0.021) [b] 
(0.9110094973785283, 0.022) [b] 
(0.9110365229447139, 0.023) [b] 
(0.9114300026568866, 0.024) [b] 
(0.9114863253942839, 0.025) [b] 
(0.9115172670095187, 0.026) [b] 
(0.9115244854264648, 0.027) [b] 
(0.9115315891969515, 0.028) [b] 
(0.9116280706028229, 0.029) [b] 
(0.9116766131839863, 0.032) [b] 
(0.9117200091087694, 0.033) [b] 
(0.911720696700219, 0.034) [b] 
(0.9117321550335523, 0.035) [b] 
(0.9117418772557745, 0.036) [b] 
(0.9118654468189272, 0.037) [b] 
(0.9119080774888063, 0.039) [b] 
(0.9119094663776952, 0.04) [b] 
(0.9119190926579905, 0.043) [b] 
(0.9120329815468794, 0.046) [b] 
(0.9120371070955774, 0.047) [b] 
(0.9120762920485554, 0.048) [b] 
(0.9123738614929999, 0.049) [b] 
(0.9123772994502483, 0.05) [b] 
(0.9127314661169149, 0.054) [b] 
(0.9127585928530261, 0.062) [b] 
(0.9127857195891372, 0.063) [b] 
(0.9128670997974706, 0.065) [b] 
(0.9133823240143736, 0.066) [b] 
(0.9133885740143736, 0.067) [b] 
(0.9134073240143736, 0.069) [b] 
(0.9134902893515057, 0.07) [b] 
(0.9134965393515057, 0.072) [b] 
(0.9134979282403946, 0.073) [b] 
(0.9135076504626168, 0.074) [b] 
(0.9137000115737279, 0.075) [b] 
(0.9138204976848391, 0.076) [b] 
(0.9140191087959503, 0.077) [b] 
(0.9140253587959503, 0.078) [b] 
(0.9140295254626168, 0.079) [b] 
(0.9140349869917234, 0.08) [b] 
(0.914037764769501, 0.081) [b] 
(0.9140398481028343, 0.082) [b] 
(0.9141054523288552, 0.084) [b] 
(0.9141373563615249, 0.086) [b] 
(0.9144965517638238, 0.087) [b] 
(0.9157073641868534, 0.088) [b] 
(0.9157087530757422, 0.089) [b] 
(0.9157108364090755, 0.092) [b] 
(0.9157184752979644, 0.093) [b] 
(0.9157261141868532, 0.094) [b] 
(0.9159254328451342, 0.095) [b] 
(0.916018862376999, 0.096) [b] 
(0.9160199040436657, 0.097) [b] 
(0.9160792582696866, 0.098) [b] 
(0.9161959249363533, 0.102) [b] 
(0.916364327714131, 0.104) [b] 
(0.916406443617096, 0.107) [b] 
(0.9164515825059849, 0.108) [b] 
(0.9164547075059849, 0.109) [b] 
(0.9165016823274805, 0.112) [b] 
(0.9170900133199898, 0.114) [b] 
(0.9175581781139973, 0.115) [b] 
(0.9176124315862195, 0.119) [b] 
(0.9176197232528862, 0.12) [b] 
(0.9177824836695528, 0.121) [b] 
(0.917809610405664, 0.122) [b] 
(0.9178853048501084, 0.123) [b] 
(0.9179124315862196, 0.128) [b] 
(0.9179940664185858, 0.129) [b] 
(0.9183031810019192, 0.131) [b] 
(0.9183448476685859, 0.134) [b] 
(0.9183469310019193, 0.14) [b] 
(0.9183483061848187, 0.142) [b] 
(0.918350389518152, 0.144) [b] 
(0.9183510771096016, 0.145) [b] 
(0.9183631334367621, 0.146) [b] 
(0.9191879773818308, 0.15) [b] 
(0.9191893662707197, 0.151) [b] 
(0.9198821002608675, 0.152) [b] 
(0.9202669524776163, 0.153) [b] 
(0.9204725542154378, 0.154) [b] 
(0.9205079708821045, 0.155) [b] 
(0.920661911768804, 0.156) [b] 
(0.9206664256576929, 0.157) [b] 
(0.9207313562132484, 0.158) [b] 
(0.9208364140266035, 0.159) [b] 
(0.9209133844699533, 0.16) [b] 
(0.9209140789143977, 0.163) [b] 
(0.9209279678032866, 0.166) [b] 
(0.920934912247731, 0.167) [b] 
(0.9209862258766309, 0.171) [b] 
(0.9209886564321865, 0.172) [b] 
(0.920991086987742, 0.173) [b] 
(0.9209935175432975, 0.174) [b] 
(0.921012961987742, 0.175) [b] 
(0.9210327536544085, 0.176) [b] 
(0.9210337953210751, 0.177) [b] 
(0.9210344897655195, 0.178) [b] 
(0.921038309209964, 0.179) [b] 
(0.9210615730988528, 0.18) [b] 
(0.9210817119877417, 0.181) [b] 
(0.9210865730988529, 0.184) [b] 
(0.9211479650354603, 0.185) [b] 
(0.9211580344799046, 0.186) [b] 
(0.9211597705910157, 0.187) [b] 
(0.9211662891558885, 0.191) [b] 
(0.9211683724892218, 0.192) [b] 
(0.9211842660540945, 0.193) [b] 
(0.9212425993874278, 0.194) [b] 
(0.9213110021652057, 0.195) [b] 
(0.9213457243874278, 0.196) [b] 
(0.9213675993874278, 0.197) [b] 
(0.9214408632763167, 0.198) [b] 
(0.9214807938318722, 0.199) [b] 
(0.9215616966096499, 0.2) [b] 
(0.9215849604985388, 0.201) [b] 
(0.9215908632763166, 0.202) [b] 
(0.9218078771652054, 0.203) [b] 
(0.9219023216096498, 0.204) [b] 
(0.9219398216096498, 0.205) [b] 
(0.9219929466096497, 0.206) [b] 
(0.9221360021652052, 0.207) [b] 
(0.9222748910540939, 0.208) [b] 
(0.922470377165205, 0.209) [b] 
(0.9225061410540938, 0.21) [b] 
(0.9226672521652048, 0.211) [b] 
(0.9227419049429826, 0.212) [b] 
(0.9229186410540937, 0.213) [b] 
(0.9229259327207604, 0.214) [b] 
(0.9230557938318715, 0.215) [b] 
(0.9231422521652046, 0.216) [b] 
(0.9233766271652046, 0.217) [b] 
(0.9235491966096491, 0.218) [b] 
(0.9236342660540936, 0.219) [b] 
(0.9238735021652047, 0.22) [b] 
(0.924522460498538, 0.221) [b] 
(0.9245287104985378, 0.222) [b] 
(0.9245425993874267, 0.223) [b] 
(0.9245509327207599, 0.224) [b] 
(0.9246627382763155, 0.225) [b] 
(0.9246707243874267, 0.226) [b] 
(0.9249068901177637, 0.227) [b] 
(0.9249305012288748, 0.228) [b] 
(0.9249426540066525, 0.229) [b] 
(0.925366959562208, 0.23) [b] 
(0.925690223451097, 0.231) [b] 
(0.9259520290066526, 0.232) [b] 
(0.926056462071992, 0.233) [b] 
(0.9261512133268839, 0.234) [b] 
(0.926185935549106, 0.235) [b] 
(0.9262576464369681, 0.236) [b] 
(0.926335771436968, 0.237) [b] 
(0.9263688009508568, 0.238) [b] 
(0.9275563877564122, 0.239) [b] 
(0.9277345995619678, 0.24) [b] 
(0.9277422384508567, 0.241) [b] 
(0.9278033495619677, 0.242) [b] 
(0.9278290440064121, 0.243) [b] 
(0.9278752245619676, 0.244) [b] 
(0.927922794006412, 0.245) [b] 
(0.9279592523397452, 0.246) [b] 
(0.9279658495619673, 0.247) [b] 
(0.9281464485203006, 0.248) [b] 
(0.9283000075480783, 0.249) [b] 
(0.9283434537286338, 0.25) [b] 
(0.9285138530341893, 0.251) [b] 
(0.9285781759508559, 0.252) [b] 
(0.9287785231730781, 0.253) [b] 
(0.9288104676175225, 0.254) [b] 
(0.928812898173078, 0.255) [b] 
(0.9289090787286335, 0.256) [b] 
(0.9289407193536335, 0.257) [b] 
(0.9291608582425224, 0.258) [b] 
(0.9291959276869667, 0.259) [b] 
(0.9292091221314112, 0.26) [b] 
(0.9293747471314111, 0.261) [b] 
(0.9293980110203, 0.262) [b] 
(0.9298733582425222, 0.263) [b] 
(0.9298792610202999, 0.264) [b] 
(0.9299850132780996, 0.265) [b] 
(0.9302265670547827, 0.266) [b] 
(0.930292886499227, 0.267) [b] 
(0.9304508726103381, 0.268) [b] 
(0.9304658031658936, 0.269) [b] 
(0.9304772614992267, 0.27) [b] 
(0.9304849003881155, 0.271) [b] 
(0.9304923512241063, 0.272) [b] 
(0.9305156151129953, 0.273) [b] 
(0.9305329762241064, 0.274) [b] 
(0.9305381845574395, 0.275) [b] 
(0.9305409623352173, 0.276) [b] 
(0.930824962701016, 0.277) [b] 
(0.9313925770079003, 0.278) [b] 
(0.9322389906661777, 0.281) [b] 
(0.9327848239995109, 0.282) [b] 
(0.9328032267772887, 0.285) [b] 
(0.9329726712217331, 0.286) [b] 
(0.9330837823328442, 0.287) [b] 
(0.933132842068354, 0.288) [b] 
(0.933206359153016, 0.289) [b] 
(0.9332067063752382, 0.29) [b] 
(0.9334296230419048, 0.291) [b] 
(0.9334317063752381, 0.292) [b] 
(0.933542470264127, 0.294) [b] 
(0.9335528869307936, 0.295) [b] 
(0.933594206375238, 0.296) [b] 
(0.933634831375238, 0.297) [b] 
(0.9336674702641267, 0.298) [b] 
(0.9336712897085712, 0.299) [b] 
(0.9336754563752379, 0.3) [b] 
(0.9336896924863489, 0.301) [b] 
(0.9336994147085711, 0.302) [b] 
(0.9337014980419043, 0.306) [b] 
(0.9337070535974598, 0.308) [b] 
(0.9342768452641264, 0.309) [b] 
(0.9342910813752375, 0.31) [b] 
(0.9343292354079071, 0.311) [b] 
(0.9343455548523516, 0.315) [b] 
(0.9343998083245738, 0.322) [b] 
(0.934426935060685, 0.323) [b] 
(0.9347253291579072, 0.325) [b] 
(0.9347795826301294, 0.326) [b] 
(0.9348081328943125, 0.328) [b] 
(0.9348129940054236, 0.329) [b] 
(0.9348274725908163, 0.33) [b] 
(0.9348684448130385, 0.331) [b] 
(0.9348715698130385, 0.332) [b] 
(0.934906673896666, 0.333) [b] 
(0.9349177850077771, 0.334) [b] 
(0.9349188266744438, 0.335) [b] 
(0.934919173896666, 0.336) [b] 
(0.9349304928428757, 0.337) [b] 
(0.9349738956206535, 0.34) [b] 
(0.9350318817317645, 0.341) [b] 
(0.9350436872873201, 0.342) [b] 
(0.9350943817317645, 0.343) [b] 
(0.9351113015942044, 0.344) [b] 
(0.9351212662907082, 0.345) [b] 
(0.9351241133468522, 0.346) [b] 
(0.9351312309872121, 0.348) [b] 
(0.9351385226538788, 0.349) [b] 
(0.9351392170983231, 0.35) [b] 
(0.9351423420983231, 0.351) [b] 
(0.9351524115427676, 0.353) [b] 
(0.9351965087649898, 0.355) [b] 
(0.9352140084328333, 0.357) [b] 
(0.9352153836157326, 0.37) [b] 
(0.9353790893440096, 0.371) [b] 
(0.9354215197334869, 0.375) [b] 
(0.9354222073249365, 0.376) [b] 
(0.9354364426056563, 0.378) [b] 
(0.9354407131898722, 0.379) [b] 
(0.9354678399259834, 0.381) [b] 
(0.9354721105101993, 0.382) [b] 
(0.9354778046224872, 0.385) [b] 
(0.9354784922139369, 0.386) [b] 
(0.9354813392700808, 0.387) [b] 
(0.9354820268615305, 0.388) [b] 
(0.9354879296393083, 0.394) [b] 
(0.9355223046393083, 0.399) [b] 
(0.9356347633569942, 0.401) [b] 
(0.9356447280534981, 0.406) [b] 
(0.9356589633342178, 0.408) [b] 
(0.935686090070329, 0.413) [b] 
(0.9357069234036623, 0.414) [b] 
(0.9357909912626524, 0.416) [b] 
(0.9358479357070969, 0.425) [b] 
(0.9359205356387675, 0.426) [b] 
(0.9359233134165453, 0.44) [b] 
(0.9359448411943231, 0.444) [b] 
(0.9359656745276564, 0.447) [b] 
(0.935996230083212, 0.464) [b] 
(0.9359979661943231, 0.474) [b] 
(0.9360522196665453, 0.477) [b] 
(0.9361064731387675, 0.481) [b] 
(0.9361586216577495, 0.505) [b] 
(0.9362002883244163, 0.512) [b] 
(0.9363442473530632, 0.519) [b] 
(0.9363539695752854, 0.562) [b] 
(0.9366785339756953, 0.607) [b] 
(0.9368111728645842, 0.616) [b] 
(0.9368118673090285, 0.631) [b] 
(0.9368125617534729, 0.672) [b] 
(0.9371969143329058, 0.685) [b] 
(0.9372011849171217, 0.698) [b] 
(0.9374830434753725, 0.717) [b] 
(0.9374894596544096, 0.718) [b] 
(0.9374958758334466, 0.72) [b] 
(0.9375058405299505, 0.722) [b] 
(0.9375122567089875, 0.734) [b] 
(0.9375186728880246, 0.749) [b] 
(0.9376368257179984, 0.767) [b] 
(0.9376493257179984, 0.845) [b] 
(0.9376607139425742, 0.847) [b] 
(0.9376621374706462, 0.848) [b] 
(0.9377241414389001, 0.883) [b] 
(0.9378053824664658, 0.947) [b] 
(0.937808854688688, 0.958) [b] 
(0.9378152708677251, 0.963) [b] 
(0.9379590472029944, 0.987) [b] 
(0.93798467070829, 1.025) [b] 
(0.9379987093426116, 1.064) [b] 
(0.937999403787056, 1.101) [b] 
(0.9380022508432, 1.105) [b] 
(0.9381014571924063, 1.117) [b] 
(0.9381345259754751, 1.119) [b] 
(0.938194721494655, 1.122) [b] 
(0.9383156112432043, 1.224) [b] 
(0.9383669248721042, 1.295) [b] 
(0.9383707443165487, 1.296) [b] 
(0.9383964011309985, 1.299) [b] 
(0.9383997793267718, 1.366) [b] 
(0.9384254361412216, 1.387) [b] 
(0.9384288143369949, 1.416) [b] 
(0.938430550448106, 1.432) [b] 
(0.9384430504481059, 1.491) [b] 
(0.9384440921147726, 1.512) [b] 
(0.9384454810036615, 1.561) [b] 
(0.9384458282258837, 1.592) [b] 
(0.9386473111401326, 1.612) [b] 
(0.9387809438106478, 1.695) [b] 
(0.938800297507757, 1.737) [b] 
(0.9388012191123812, 1.745) [b] 
(0.9388016799146933, 1.747) [b] 
(0.9388288066508045, 1.749) [b] 
(0.9388297282554288, 1.757) [b] 
(0.9388366402901106, 1.777) [b] 
(0.9388371010924227, 1.78) [b] 
(0.9388426307201682, 1.786) [b] 
(0.9388440131271045, 1.789) [b] 
(0.9388444739294166, 1.821) [b] 
(0.938845856336353, 1.844) [b] 
(0.9388486211502257, 1.847) [b] 
(0.93884954275485, 1.85) [b] 
(0.9388578371964682, 1.882) [b] 
(0.9388601412080289, 1.889) [b] 
(0.9388606020103409, 1.894) [b] 
(0.9388619844172773, 1.896) [b] 
(0.9389089592387729, 1.9) [b] 
(0.9389324466495208, 1.908) [b] 
(0.9389536272050764, 1.926) [b] 
(0.9389956410939653, 1.946) [b] 
(0.9390371133020562, 1.953) [b] 
(0.9390544744131673, 1.98) [b] 
(0.9390638494131672, 1.991) [b] 
(0.9390873368239151, 2.01) [b] 
(0.9434997668779027, 2.014) [b] 
(0.9435138055122244, 2.038) [b] 
(0.9435278441465461, 2.06) [b] 
(0.9435356777858522, 2.067) [b] 
(0.9435471978436553, 2.089) [b] 
(0.9435522666690886, 2.111) [b] 
(0.9436246046320517, 2.126) [b] 
(0.9436487172863727, 2.128) [b] 
(0.9436705922863726, 2.141) [b] 
(0.9436962491008225, 2.153) [b] 
(0.9437018046563781, 2.155) [b] 
(0.9437348734394468, 2.157) [b] 
(0.9437973734394468, 2.18) [b] 
(0.9437987558463832, 2.183) [b] 
(0.9437996774510075, 2.202) [b] 
(0.9438482885621186, 2.21) [b] 
(0.9438492101667428, 2.211) [b] 
(0.9438501317713671, 2.215) [b] 
(0.9438515141783035, 2.221) [b] 
(0.9438519749806156, 2.223) [b] 
(0.94385266942506, 2.228) [b] 
(0.9439483815230691, 2.265) [b] 
(0.9439802855557389, 2.266) [b] 
(0.9440121895884086, 2.278) [b] 
(0.9440170506995197, 2.287) [b] 
(0.9440427075139696, 2.308) [b] 
(0.944043147057407, 2.359) [b] 
(0.9440435866008444, 2.361) [b] 
(0.9440438796298027, 2.38) [b] 
(0.9440441726587611, 2.401) [b] 
(0.9440451982601151, 2.403) [b] 
(0.9440456378035526, 2.405) [b] 
(0.9440457843180317, 2.409) [b] 
(0.94404607734699, 2.41) [b] 
(0.9442801597439937, 2.449) [b] 
(0.9442805992874311, 2.45) [b] 
(0.9442808923163895, 2.456) [b] 
(0.9442811853453478, 2.462) [b] 
(0.9443273117076394, 2.463) [b] 
(0.9443304367076394, 2.478) [b] 
(0.9443308762510768, 2.494) [b] 
(0.9443311692800351, 2.496) [b] 
(0.9443316088234726, 2.524) [b] 
(0.9443368171568058, 2.563) [b] 
(0.9443385532679169, 2.637) [b] 
(0.944340289379028, 2.69) [b] 
(0.9443659461934779, 2.763) [b] 
(0.944414765186404, 2.912) [b] 
(0.9445589963100981, 2.943) [b] 
(0.9445610590844471, 2.984) [b] 
(0.9445665598160444, 2.989) [b] 
(0.9445672474074941, 3.036) [b] 
(0.9446194940114648, 3.057) [b] 
(0.9446260248369611, 3.064) [b] 
(0.9446267124284108, 3.072) [b] 
(0.9446329624284108, 3.091) [b] 
(0.944633309650633, 3.111) [b] 
(0.9447146898589663, 3.286) [b] 
(0.9447418165950775, 3.288) [b] 
(0.9447689433311887, 3.292) [b] 
(0.9448859845296905, 3.375) [b] 
(0.9448925153551868, 3.498) [b] 
(0.9449166280095078, 3.943) [b] 
(0.9449183641206189, 3.949) [b] 
(0.9449424767749399, 3.952) [b] 
(0.9449507712165581, 4.011) [b] 
(0.9449526144258066, 4.013) [b] 
(0.9449590656581763, 4.021) [b] 
(0.9449705239915097, 4.046) [b] 
(0.9449838872585612, 4.134) [b] 
(0.9449963289209885, 4.224) [b] 
(0.9450282329536582, 4.31) [b] 
(0.945060136986328, 4.312) [b] 
(0.9451877531170069, 4.313) [b] 
(0.9452196571496766, 4.314) [b] 
(0.9452515611823463, 4.32) [b] 
(0.945283465215016, 4.322) [b] 
(0.9452898813940531, 4.323) [b] 
(0.9453217854267228, 4.324) [b] 
(0.9453323838799016, 4.371) [b] 
(0.94533422708915, 4.379) [b] 
(0.9453661311218198, 4.506) [b] 
(0.9453980351544895, 4.507) [b] 
(0.9454299391871592, 4.508) [b] 
(0.945461843219829, 4.51) [b] 
(0.9455256512851684, 4.511) [b] 
(0.9455575553178381, 4.515) [b] 
(0.9456213633831776, 4.517) [b] 
(0.9456532674158473, 4.519) [b] 
(0.9457132487171603, 4.521) [b] 
(0.9458153459214383, 4.522) [b] 
(0.9458208755491838, 4.526) [b] 
(0.9458349141835055, 4.528) [b] 
(0.9458489528178272, 4.531) [b] 
(0.9458496404092769, 4.536) [b] 
(0.9458636790435986, 4.554) [b] 
(0.9459198335808852, 4.578) [b] 
(0.9459619494838502, 4.579) [b] 
(0.945963338372739, 4.667) [b] 
(0.9459651815819875, 4.674) [b] 
(0.9459792202163092, 4.745) [b] 
(0.945992732999402, 5.002) [b] 
(0.9460197585655876, 5.003) [b] 
(0.9460332713486804, 5.014) [b] 
(0.9460366495444537, 5.018) [b] 
(0.9460697183275224, 5.041) [b] 
(0.9461002210117586, 5.136) [b] 
(0.9461103555990782, 5.19) [b] 
(0.9461434243821469, 5.219) [b] 
(0.9461642577154803, 5.229) [b] 
(0.9461707762803531, 5.354) [b] 
(0.9461725123914642, 5.411) [b] 
(0.9461996391275753, 5.538) [b] 
(0.9462267658636865, 5.54) [b] 
(0.9462301440594597, 5.599) [b] 
(0.9462335222552329, 5.606) [b] 
(0.9462475608895546, 5.95) [b] 
(0.9463646020880565, 6.028) [b] 
(0.9463694631991676, 6.174) [b] 
(0.9463740712222888, 6.501) [b] 
(0.9463754536292251, 6.517) [b] 
(0.9463800616523463, 6.522) [b] 
(0.9463809832569706, 6.526) [b] 
(0.946442730766795, 6.555) [b] 
(0.9464514860107253, 6.579) [b] 
(0.9464570156384707, 6.584) [b] 
(0.946457937243095, 6.585) [b] 
(0.9464588588477193, 6.589) [b] 
(0.9464597804523436, 6.604) [b] 
(0.9464643884754648, 6.637) [b] 
(0.9464666924870254, 6.64) [b] 
(0.9464685356962739, 6.643) [b] 
(0.9464745261263314, 6.657) [b] 
(0.9464754477309557, 6.659) [b] 
(0.9464786733471405, 6.666) [b] 
(0.9464791341494526, 6.673) [b] 
(0.9464795949517647, 6.69) [b] 
(0.9464800557540768, 6.694) [b] 
(0.9464805165563889, 6.697) [b] 
(0.946480977358701, 6.703) [b] 
(0.9464816718031454, 6.846) [b] 
(0.9467638967620945, 6.949) [b] 
(0.9469948080921438, 6.988) [b] 
(0.9470204649065936, 7.096) [b] 
(0.9470607614894434, 7.149) [b] 
(0.9473943000772923, 7.62) [b] 
(0.9474199568917422, 7.622) [b] 
(0.9474216930028533, 7.624) [b] 
(0.947498663446203, 7.627) [b] 
(0.9475243202606529, 7.633) [b] 
(0.9476012907040027, 7.642) [b] 
(0.9476782611473524, 7.737) [b] 
(0.9477039179618023, 7.908) [b] 
(0.9477295747762522, 8.329) [b] 
(0.94776672755403, 8.552) [b] 
(0.9477923843684799, 8.847) [b] 
(0.9478180411829298, 9.091) [b] 
(0.9478436979973797, 9.104) [b] 
(0.9478693548118295, 9.504) [b] 
(0.9478901881451629, 10.092) [b] 
(0.9479526881451629, 10.093) [b] 
(0.947979814881274, 10.353) [b] 
(0.9480069416173852, 10.427) [b] 
(0.9480124423489825, 10.934) [b] 
(0.9480131299404322, 10.936) [b] 
(0.9480138175318819, 10.966) [b] 
(0.9480151927147812, 11.039) [b] 
(0.9480172554891302, 11.046) [b] 
(0.9480429123035801, 11.968) [b] 
(0.94806856911803, 12.208) [b] 
(0.9480689163402521, 12.296) [b] 
(0.948185957538754, 12.516) [b] 
(0.9481904714276429, 12.972) [b] 
(0.9481911658720873, 13.024) [b] 
(0.9481915130943095, 13.115) [b] 
(0.9481925547609762, 13.343) [b] 
(0.9482182115754261, 13.346) [b] 
(0.9482598782420928, 13.494) [b] 
(0.948287004978204, 13.501) [b] 
(0.9483141317143151, 13.503) [b] 
(0.9483412584504263, 13.504) [b] 
(0.9483683851865374, 13.506) [b] 
(0.9483955119226486, 13.524) [b] 
(0.9484226386587598, 13.525) [b] 
(0.9485040188670931, 13.527) [b] 
(0.9485311456032043, 13.528) [b] 
(0.9486125258115377, 13.529) [b] 
(0.9486396525476488, 13.534) [b] 
(0.94866677928376, 13.543) [b] 
(0.9486939060198711, 13.545) [b] 
(0.9487210327559823, 13.546) [b] 
(0.9487481594920935, 13.547) [b] 
(0.9487752862282046, 13.548) [b] 
(0.948796119561538, 13.867) [b] 
(0.9488101581958597, 15.448) [b] 
(0.9488241968301814, 15.788) [b] 
(0.9488269746079592, 16.68) [b] 
(0.9488304468301814, 17.352) [b] 
(0.9488325301635147, 17.363) [b] 
(0.9488339190524036, 17.397) [b] 
(0.948834613496848, 17.453) [b] 
(0.9488602703112978, 17.935) [b] 
(0.9488837577220457, 19.735) [b] 
(0.9488879243887124, 20.51) [b] 
(0.9488913025844856, 21.299) [b] 
(0.9488946807802588, 21.327) [b] 
(0.9488980589760321, 21.411) [b] 
(0.948923715790482, 22.499) [b] 
(0.9489357721176425, 23.533) [b] 
(0.948947828444803, 23.536) [b] 
(0.9489488701114697, 23.631) [b] 
(0.9489629087457914, 23.692) [b] 
(0.9490799499442932, 27.453) [b] 
(0.949098352722071, 28.853) [b] 
(0.9490997416109599, 28.907) [b] 
(0.9491004360554043, 29.009) [b] 
(0.9491007832776265, 29.213) [b] 
(0.9491011304998487, 29.314) [b] 
(0.9491139628579228, 29.45) [b] 
(0.9491420401265661, 29.547) [b] 
(0.9491493317932328, 30.641) [b] 
(0.9491614845710106, 30.686) [b] 
(0.9496739546769211, 33.731) [b] 
(0.94967534356581, 36.232) [b] 
(0.9496760380102544, 38.569) [b] 
(0.94967846856581, 38.679) [b] 
(0.9496798574546989, 38.725) [b] 
(0.9496819407880321, 39.005) [b] 
(0.9496884593529049, 46.238) [b] 
(0.9496888065751271, 46.773) [b] 
(0.9496891537973493, 47.497) [b] 
(0.9496908899084604, 48.499) [b] 
(0.9496915843529048, 48.544) [b] 
(0.9496926260195715, 48.593) [b] 
(0.9496933204640159, 48.65) [b] 
(0.9496943621306826, 48.894) [b] 
(0.9497007783097197, 49.14) [b] 
(0.9497071944887567, 49.141) [b] 
(0.9497172639332012, 49.687) [b] 
(0.9497206421289744, 53.798) [b] 
(0.9497240203247477, 53.799) [b] 
(0.9497442894993869, 53.9) [b] 
(0.949750705678424, 56.654) [b] 
(0.9497510529006462, 56.983) [b] 
(0.9498130568689002, 60.414) [b] 
(0.949930098067402, 61.592) [b] 
(0.9499509314007354, 66.118) [b] 
(0.9499512786229576, 66.966) [b] 
(0.9499747660337055, 68.626) [b] 
(0.9499762311784969, 68.906) [b] 
(0.9499807450673858, 70.362) [b] 
(0.9499814395118302, 70.414) [b] 
(0.9499824811784969, 70.567) [b] 
(0.9499831756229413, 71.299) [b] 
(0.9500066630336892, 71.89) [b] 
(0.9500087142363972, 73.642) [b] 
(0.9500485945404776, 79.146) [b] 
(0.950058220820773, 79.148) [b] 
(0.9500589084122226, 79.22) [b] 
(0.9500595960036723, 79.228) [b] 
(0.9500650967352696, 79.334) [b] 
(0.9500657843267193, 79.351) [b] 
(0.950066471918169, 79.423) [b] 
(0.9500675135848357, 80.181) [b] 
(0.950069596918169, 80.187) [b] 
(0.9500709721010683, 80.362) [b] 
(0.950072013767735, 81.509) [b] 
(0.950075138767735, 81.517) [b] 
(0.9500754859899572, 81.753) [b] 
(0.9500765276566239, 82.188) [b] 
(0.9500786109899572, 82.359) [b] 
(0.9500796526566239, 83.318) [b] 
(0.9500817359899572, 84.686) [b] 
(0.9501140276566238, 84.693) [b] 
(0.9501181943232905, 84.7) [b] 
(0.9501213193232905, 84.72) [b] 
(0.9501223609899572, 84.83) [b] 
(0.9501338193232906, 85.119) [b] 
(0.9501386324634382, 85.141) [b] 
(0.9501406952377872, 85.143) [b] 
(0.9501413828292369, 85.171) [b] 
(0.9501427580121362, 85.208) [b] 
(0.9501434456035859, 85.21) [b] 
(0.9501441331950355, 85.4) [b] 
(0.9501506640205319, 86.869) [b] 
(0.9501513516119816, 86.945) [b] 
(0.950152046056426, 88.272) [b] 
(0.9501620107529298, 92.958) [b] 
(0.9501720801973743, 95.341) [b] 
(0.950179371864041, 95.391) [b] 
(0.9501797190862632, 95.695) [b] 
(0.9501818024195965, 95.999) [b] 
(0.9501883332450929, 96.239) [b] 
(0.950190069356204, 97.26) [b] 
(0.9501911110228707, 97.386) [b] 
(0.9501914582450929, 97.941) [b] 
(0.9501928334279922, 102.9) [b] 
(0.9501935210194419, 102.917) [b] 
(0.9501942086108915, 102.922) [b] 
(0.9501955974997804, 106.371) [b] 
(0.9502023538913269, 110.115) [b] 
(0.9502440205579936, 116.813) [b] 
(0.9502856872246603, 116.814) [b] 
(0.9503091746354082, 117.167) [b] 
(0.9503326620461561, 117.175) [b] 
(0.9503561494569039, 117.503) [b] 
(0.950356295971383, 128.724) [b] 
(0.9503566431936052, 136.064) [b] 
(0.9503706818279269, 137.663) [b] 
(0.9503717234945936, 151.822) [b] 
(0.9503727651612603, 152.607) [b] 
(0.950373806827927, 152.645) [b] 
(0.9503748484945937, 153.937) [b] 
(0.9503758901612605, 155.697) [b] 
(0.9503769318279272, 155.983) [b] 
(0.9503772790501493, 159.552) [b] 
(0.9503776262723715, 159.719) [b] 
(0.9503779734945937, 161.607) [b] 
(0.9503790151612604, 163.89) [b] 
(0.950435169698547, 168.474) [b] 
(0.950477285601512, 168.475) [b] 
(0.9504913242358337, 168.533) [b] 
(0.9505053628701554, 168.648) [b] 
(0.9505194015044771, 169.118) [b] 
(0.9505334401387988, 169.119) [b] 
(0.9505474787731205, 169.12) [b] 
(0.9505615174074422, 169.121) [b] 
(0.9505679335864793, 170.585) [b] 
(0.9506849747849812, 182.614) [b] 
(0.9506883529807544, 226.707) [b] 
(0.9506887002029766, 230.455) [b] 
(0.9506890474251988, 230.554) [b] 
(0.9506904363140877, 231.101) [b] 
(0.9506914779807544, 231.708) [b] 
(0.9506918252029766, 232.214) [b] 
(0.9508088664014784, 236.568) [b] 
(0.9509259075999803, 237.013) [b] 
(0.9509266020444247, 258.304) [b] 
(0.9509272964888691, 259.38) [b] 
(0.9509279840803188, 267.512) [b] 
(0.9509286716717684, 267.556) [b] 
(0.9509988648433767, 268.161) [b] 
(0.9510129034776984, 268.167) [b] 
(0.9510269421120201, 268.173) [b] 
(0.9510409807463418, 268.197) [b] 
(0.9510416683377915, 268.624) [b] 
(0.9510430435206908, 272.626) [b] 
(0.9510570821550125, 273.58) [b] 
(0.9510711207893342, 273.605) [b] 
(0.9511132366922992, 273.627) [b] 
(0.9511272753266209, 273.629) [b] 
(0.9511413139609426, 273.722) [b] 
(0.9511553525952643, 273.725) [b] 
(0.951169391229586, 273.734) [b] 
(0.9511834298639077, 274.869) [b] 
(0.9511974684982294, 274.875) [b] 
(0.9512016351648961, 277.949) [b] 
(0.9512085796093406, 278.02) [b] 
(0.9512099547922399, 281.405) [b] 
(0.9512106423836896, 281.407) [b] 
(0.9512113299751392, 282.429) [b] 
(0.9512120244195836, 284.642) [b] 
(0.9512148021973614, 288.066) [b] 
(0.9512670488013322, 300.601) [b] 
(0.9512735796268286, 300.728) [b] 
(0.9512866412778213, 300.734) [b] 
(0.9512880301667102, 310.059) [b] 
(0.9513020688010319, 312.282) [b] 
(0.951302215315511, 315.84) [b] 
(0.9513029029069606, 344.617) [b] 
(0.9513035904984103, 344.621) [b] 
(0.95130427808986, 344.625) [b] 
(0.9513049656813096, 346.641) [b] 
(0.9513056532727593, 346.698) [b] 
(0.9513070284556586, 346.705) [b] 
(0.9513077160471083, 346.744) [b] 
(0.951308403638558, 346.745) [b] 
(0.9513292369718913, 358.34) [b] 
(0.9513917369718913, 358.341) [b] 
(0.9514125703052246, 358.342) [b] 
(0.9514542369718914, 358.343) [b] 
(0.9514750703052247, 358.344) [b] 
(0.9515167369718914, 358.345) [b] 
(0.9515375703052248, 358.346) [b] 
(0.9515584036385581, 358.347) [b] 
(0.9516625703052248, 358.349) [b] 
(0.9519125703052248, 358.35) [b] 
(0.9519542369718915, 358.359) [b] 
(0.9520167369718915, 358.364) [b] 
(0.9520375703052248, 358.467) [b] 
(0.9520792369718916, 358.578) [b] 
(0.9521209036385583, 358.581) [b] 
(0.9521417369718916, 358.587) [b] 
(0.952162570305225, 358.594) [b] 
(0.9521834036385584, 358.856) [b] 
(0.9522042369718917, 359.701) [b] 
(0.9522250703052251, 361.295) [b] 
(0.9522257578966747, 361.836) [b] 
(0.9524340912300081, 363.94) [b] 
(0.9524549245633415, 363.941) [b] 
(0.9524757578966748, 363.942) [b] 
(0.9525382578966748, 363.947) [b] 
(0.9525590912300081, 363.953) [b] 
(0.9526007578966749, 364.026) [b] 
(0.9526215912300082, 364.167) [b] 
(0.9526424245633416, 364.169) [b] 
(0.9526632578966749, 364.179) [b] 
(0.9526840912300083, 372.021) [b] 
(0.9527415205581528, 415.978) [b] 
(0.952744898753926, 415.981) [b] 
(0.9527482769496992, 415.983) [b] 
(0.9527516551454724, 415.993) [b] 
(0.9527550333412457, 416.027) [b] 
(0.952761789732792, 416.111) [b] 
(0.9527719243201116, 417.329) [b] 
(0.952778680711658, 417.33) [b] 
(0.9527854371032045, 417.334) [b] 
(0.9527989498862973, 417.34) [b] 
(0.9528023280820705, 417.341) [b] 
(0.9528057062778438, 417.367) [b] 
(0.952809084473617, 417.535) [b] 
(0.9528124626693902, 417.943) [b] 
(0.9528158408651635, 417.95) [b] 
(0.9528179241984968, 448.253) [b] 
(0.9528186186429412, 448.371) [b] 
(0.9528193130873855, 448.541) [b] 
(0.9528258316522583, 459.355) [b] 
(0.9528293038744805, 462.277) [b] 
(0.9528696004573303, 464.375) [b] 
(0.95290989704018, 464.627) [b] 
(0.9529501936230298, 464.878) [b] 
(0.9529533186230298, 468.899) [b] 
(0.9531393305277917, 470.152) [b] 
(0.9532013344960456, 470.161) [b] 
(0.9532633384642996, 470.175) [b] 
(0.9533253424325535, 470.261) [b] 
(0.9533873464008075, 470.769) [b] 
(0.9534276429836572, 472.749) [b] 
(0.953467939566507, 472.936) [b] 
(0.9535082361493568, 473.876) [b] 
(0.9535702401176107, 475.818) [b] 
(0.95369112986616, 483.547) [b] 
(0.9537314264490098, 483.737) [b] 
(0.9537328499770817, 483.821) [b] 
(0.9537731465599315, 483.992) [b] 
(0.9538134431427813, 484.118) [b] 
(0.953853739725631, 484.631) [b] 
(0.9538575591700755, 516.793) [b] 
(0.9538586008367422, 516.848) [b] 
(0.9538589480589644, 517.191) [b] 
(0.9538592952811866, 517.306) [b] 
(0.9538596425034088, 517.477) [b] 
(0.9538658925034088, 535.949) [b] 
(0.953866239725631, 536.263) [b] 
(0.9538995730589643, 545.151) [b] 
(0.9539179758367421, 545.227) [b] 
(0.9539221425034088, 545.482) [b] 
(0.9539231841700755, 545.764) [b] 
(0.9539242258367422, 545.895) [b] 
(0.9539252675034089, 546.35) [b] 
(0.9539263091700756, 549.415) [b] 
(0.9539593779531443, 557.054) [b] 
(0.9541052112864776, 564.879) [b] 
(0.9541677112864776, 564.88) [b] 
(0.954188544619811, 564.889) [b] 
(0.9542093779531443, 564.927) [b] 
(0.954251044619811, 565.002) [b] 
(0.9542927112864777, 565.003) [b] 
(0.9543034751753666, 572.624) [b] 
(0.954304169619811, 572.681) [b] 
(0.9543048640642554, 572.801) [b] 
(0.9543052112864776, 577.152) [b] 
(0.9543156279531443, 586.008) [b] 
(0.9543159751753665, 586.059) [b] 
(0.9543163223975887, 586.111) [b] 
(0.9543166696198109, 586.222) [b] 
(0.9543173640642553, 586.491) [b] 
(0.9543180585086997, 587.187) [b] 
(0.9543244746877367, 650.267) [b] 
(0.9543516620241519, 664.229) [b] 
(0.954351808538631, 674.454) [b] 
(0.9543619461894977, 694.959) [b] 
(0.954362640633942, 822.303) [b] 
(0.9543636823006088, 848.703) [b] 
(0.954365765633942, 848.708) [b] 
(0.9543709739672753, 848.713) [b] 
(0.954372015633942, 848.734) [b] 
(0.9543730573006087, 848.741) [b] 
(0.9543845156339421, 848.747) [b] 
(0.9543855573006088, 849.276) [b] 
(0.9543865989672755, 849.282) [b] 
(0.9543876406339422, 852.848) [b] 
(0.9543886823006089, 854.14) [b] 
(0.9544121697113568, 898.778) [b] 
(0.954412516933579, 911.888) [b] 
(0.9544135586002457, 1018.72) [b] 
(0.9544370460109936, 1037.02) [b] 
(0.9544373932332157, 1039.47) [b] 
(0.9544380876776601, 1042.68) [b] 
(0.9544384348998823, 1045.45) [b] 
(0.9544619223106302, 1053.47) [b] 
(0.9544660889772969, 1171.38) [b] 
(0.9544674778661858, 1172.26) [b] 
(0.9544931346806357, 1173.15) [b] 
(0.9544948707917468, 1173.18) [b] 
(0.9545205276061967, 1173.29) [b] 
(0.9545246942728633, 1176.25) [b] 
(0.9545281664950855, 1179.48) [b] 
(0.9545285137173077, 1180.11) [b] 
(0.9545312914950855, 1183.78) [b] 
(0.9545569483095354, 1208.03) [b] 
(0.9546194483095355, 1219.72) [b] 
(0.9547236149762022, 1219.79) [b] 
(0.9547444483095355, 1220.02) [b] 
(0.9547701051239854, 1224.45) [b] 
(0.9547957619384353, 1234.49) [b] 
(0.9548214187528852, 1234.62) [b] 
(0.9548422520862185, 1242.24) [b] 
(0.9548630854195519, 1242.25) [b] 
(0.9548839187528853, 1243.78) [b] 
(0.9549095755673351, 1270.39) [b] 
(0.9549106172340018, 1283.07) [b] 
(0.9549362740484517, 1288.52) [b] 
(0.9550533152469536, 1299.26) [b] 
(0.9550543569136203, 1308.71) [b] 
(0.9550550513580647, 1337.52) [b] 
(0.9550564402469536, 1337.54) [b] 
(0.955057134691398, 1337.66) [b] 
(0.9550578291358424, 1338.37) [b] 
(0.9550585235802868, 1340.81) [b] 
(0.9550613013580646, 1354.93) [b] 
(0.9550640791358423, 1354.94) [b] 
(0.9550647735802867, 1355.18) [b] 
(0.9550712921451595, 1387.34) [b] 
(0.9550947795559074, 1570.67) [b] 
(0.9550968628892407, 1588.01) [b] 
(0.9550985990003518, 1588.07) [b] 
(0.955098946222574, 1588.43) [b] 
(0.9551079740003517, 1665.7) [b] 
(0.9551149184447962, 1665.78) [b] 
(0.9551180434447962, 1665.85) [b] 
(0.9551187378892406, 1666.16) [b] 
(0.9551190851114628, 1668.16) [b] 
(0.9551331237457845, 1766.82) [b] 
(0.9551471623801062, 1767.28) [b] 
(0.9551574762518512, 1788.03) [b] 
(0.9551595390262002, 1788.04) [b] 
(0.9551609142090995, 1788.06) [b] 
(0.9551664149406969, 1788.13) [b] 
(0.9551671025321465, 1788.65) [b] 
(0.9551684777150459, 1788.69) [b] 
(0.9551691653064955, 1788.84) [b] 
(0.9551698528979452, 1788.85) [b] 
(0.9551712280808445, 1791.06) [b] 
(0.9551719156722942, 1791.23) [b] 
(0.9551732908551935, 1793.81) [b] 
(0.9551753536295425, 1793.91) [b] 
(0.9551767288124419, 1793.98) [b] 
(0.9551781039953412, 1794.11) [b] 
(0.9551787915867909, 1805.23) [b] 
(0.9551794791782405, 1833.5) [b] 
(0.9551815419525895, 1848.16) [b] 
(0.9551829171354889, 1928.4) [b] 
(0.9551842923183882, 1933.64) [b] 
(0.9551849799098379, 1937.74) [b] 
(0.9551856675012875, 1937.75) [b] 
(0.9551863550927372, 1937.76) [b] 
(0.9551870426841869, 1937.78) [b] 
(0.9551877302756365, 1937.82) [b] 
(0.9551897930499855, 1949.68) [b] 
(0.9551904874944299, 1990.49) [b] 
(0.9551929180499855, 2123.38) [b] 
(0.9551943069388744, 2123.42) [b] 
(0.9551946541610966, 2123.51) [b] 
(0.9551950013833188, 2123.6) [b] 
(0.955195348605541, 2123.92) [b] 
(0.9551956958277632, 2124.2) [b] 
(0.9551960430499854, 2135.25) [b] 
(0.9551963902722076, 2137.75) [b] 
(0.9551988208277632, 2143.28) [b] 
(0.9551998624944299, 2143.32) [b] 
(0.9552002097166521, 2162.14) [b] 
(0.9552008973081018, 2568.54) [b] 
(0.9552054111969907, 2573.52) [b] 
(0.9552074945303239, 2573.58) [b] 
(0.9552088834192128, 2574.02) [b] 
(0.955209230641435, 2574.42) [b] 
(0.9552099250858794, 2575.78) [b] 
(0.9552106195303238, 2585.94) [b] 
(0.955213997726097, 2610.69) [b] 
(0.9552945908917966, 3223.6) [b] 
(0.9552979690875698, 3343.55) [b] 
(0.9553081036748894, 3343.6) [b] 
(0.9553114818706626, 3343.62) [b] 
(0.9553249946537554, 3344.62) [b] 
(0.9553317510453019, 3344.65) [b] 
(0.9553351292410751, 3344.72) [b] 
(0.9553418856326215, 3359.1) [b] 
(0.9553452638283947, 3440.27) [b] 
(0.9553486420241679, 3442.55) [b] 
(0.9553517670241679, 3471.64) [b] 
(0.9553531559130568, 3474.87) [b] 
(0.9553555864686124, 3507.34) [b] 
},{(0.9475928541666668, 0.001) [c] 
(0.9475928541666668, 3.9281546249999995) [c] 
(0.9475928541666668, 3600) [c] 
}}}{legend pos=north west}}
% 	\subfloat[depth=10]{\cactus{Average Accuracy}{CPU time}{\budalg, \murtree, \cart}{{{(0.9293122642671738, 0) [a] 
(0.9315914309338404, 0.01) [a] 
(0.9382080976005072, 0.02) [a] 
(0.9382839309338403, 0.03) [a] 
(0.9436158753782845, 0.04) [a] 
(0.9472002503782844, 0.05) [a] 
(0.9489364309338397, 0.06) [a] 
(0.9492443476005058, 0.07) [a] 
(0.9493087226005057, 0.08) [a] 
(0.949327680933839, 0.09) [a] 
(0.949373305933839, 0.1) [a] 
(0.949380805933839, 0.11) [a] 
(0.9494524726005055, 0.12) [a] 
(0.9495170559338386, 0.13) [a] 
(0.9497018476005049, 0.14) [a] 
(0.9497320559338381, 0.15) [a] 
(0.9498166392671713, 0.16) [a] 
(0.9499162226005045, 0.17) [a] 
(0.9500635142671708, 0.18) [a] 
(0.9500849726005042, 0.19) [a] 
(0.9500858059338375, 0.2) [a] 
(0.9500978892671709, 0.21) [a] 
(0.950354972600504, 0.22) [a] 
(0.9504010142671707, 0.23) [a] 
(0.9505035142671708, 0.24) [a] 
(0.9505105976005042, 0.25) [a] 
(0.950514347600504, 0.26) [a] 
(0.950515597600504, 0.27) [a] 
(0.9505168476005039, 0.28) [a] 
(0.9542611245194197, 0.29) [a] 
(0.9542619578527529, 0.3) [a] 
(0.9542623745194196, 0.31) [a] 
(0.9542640411860862, 0.33) [a] 
(0.9542944578527529, 0.35) [a] 
(0.9542961245194196, 0.36) [a] 
(0.9543171661860863, 0.37) [a] 
(0.9543182078527529, 0.38) [a] 
(0.9543396661860862, 0.39) [a] 
(0.9543404995194196, 0.4) [a] 
(0.9543615411860863, 0.41) [a] 
(0.9543619578527529, 0.43) [a] 
(0.9544040411860861, 0.44) [a] 
(0.9544179995194194, 0.45) [a] 
(0.954475291186086, 0.46) [a] 
(0.9589059415711847, 0.47) [a] 
(0.9589061499045181, 0.49) [a] 
(0.9590776082378515, 0.5) [a] 
(0.9591046915711848, 0.55) [a] 
(0.9591367749045181, 0.57) [a] 
(0.9591434415711848, 0.58) [a] 
(0.9591436499045182, 0.59) [a] 
(0.9591440665711848, 0.6) [a] 
(0.9591582332378514, 0.63) [a] 
(0.9591619832378514, 0.67) [a] 
(0.959162399904518, 0.68) [a] 
(0.9591626082378514, 0.76) [a] 
(0.9591628165711847, 0.79) [a] 
(0.9591634415711847, 0.8) [a] 
(0.959164274904518, 0.81) [a] 
(0.959164899904518, 0.82) [a] 
(0.9591663582378513, 0.83) [a] 
(0.9591676082378513, 0.84) [a] 
(0.9591684415711846, 0.85) [a] 
(0.9591690665711846, 0.91) [a] 
(0.9591701082378512, 0.92) [a] 
(0.9591709415711844, 0.95) [a] 
(0.9591773999045178, 0.96) [a] 
(0.9591778165711844, 0.97) [a] 
(0.9591798999045177, 1) [a] 
(0.959180733237851, 1.01) [a] 
(0.9591834415711843, 1.03) [a] 
(0.9591898999045176, 1.09) [a] 
(0.9601224999999959, 1.12) [a] 
(0.960129374999996, 1.14) [a] 
(0.9602904166666626, 1.16) [a] 
(0.9604118749999959, 1.17) [a] 
(0.9604124999999959, 1.18) [a] 
(0.9604135416666625, 1.19) [a] 
(0.9604137499999958, 1.23) [a] 
(0.9604141666666625, 1.25) [a] 
(0.9604145833333291, 1.26) [a] 
(0.9604149999999957, 1.27) [a] 
(0.960415208333329, 1.28) [a] 
(0.9604156249999957, 1.3) [a] 
(0.9604160416666623, 1.33) [a] 
(0.9604164583333289, 1.35) [a] 
(0.9604168749999955, 1.36) [a] 
(0.9604172916666621, 1.37) [a] 
(0.9604174999999955, 1.44) [a] 
(0.9604181249999955, 1.49) [a] 
(0.9604185416666621, 1.53) [a] 
(0.9604249999999954, 1.54) [a] 
(0.9604256249999954, 1.55) [a] 
(0.960428541666662, 1.56) [a] 
(0.9604318749999953, 1.57) [a] 
(0.9604347916666619, 1.58) [a] 
(0.9604368749999952, 1.59) [a] 
(0.9604389583333285, 1.6) [a] 
(0.9604395833333285, 1.63) [a] 
(0.9604402083333284, 1.7) [a] 
(0.9604466666666618, 1.72) [a] 
(0.9604481249999951, 1.74) [a] 
(0.9604993749999952, 1.93) [a] 
(0.9604997916666618, 1.97) [a] 
(0.9605027083333284, 2.05) [a] 
(0.9605041666666617, 2.07) [a] 
(0.9605047916666617, 2.27) [a] 
(0.9605564583333283, 2.32) [a] 
(0.9605568749999949, 2.37) [a] 
(0.9605574999999948, 2.38) [a] 
(0.9605604166666615, 2.39) [a] 
(0.9605631249999947, 2.4) [a] 
(0.9605645833333281, 2.41) [a] 
(0.9605679166666613, 2.42) [a] 
(0.9605687499999946, 2.44) [a] 
(0.9605699999999946, 2.45) [a] 
(0.9605704166666612, 2.46) [a] 
(0.9605716666666612, 2.47) [a] 
(0.9605720833333278, 2.49) [a] 
(0.9605733333333277, 2.59) [a] 
(0.9605737499999943, 2.61) [a] 
(0.9605743749999943, 2.62) [a] 
(0.9606368749999943, 2.65) [a] 
(0.9607491666666609, 2.69) [a] 
(0.9607497916666609, 2.73) [a] 
(0.9607512499999942, 2.79) [a] 
(0.9607533333333275, 2.8) [a] 
(0.9607539583333274, 2.81) [a] 
(0.9607547916666608, 2.84) [a] 
(0.9607554166666608, 2.85) [a] 
(0.9607791666666607, 2.88) [a] 
(0.9607999999999941, 2.89) [a] 
(0.9608208333333275, 2.91) [a] 
(0.9608272916666608, 2.94) [a] 
(0.9608506249999942, 2.96) [a] 
(0.9608645833333275, 2.97) [a] 
(0.9608652083333274, 3.13) [a] 
(0.9608660416666608, 3.42) [a] 
(0.9608666666666608, 3.47) [a] 
(0.9608672916666607, 3.56) [a] 
(0.9608814583333274, 3.57) [a] 
(0.960881874999994, 4.01) [a] 
(0.960882499999994, 4.04) [a] 
(0.9608839583333273, 4.08) [a] 
(0.9608843749999939, 4.09) [a] 
(0.9608847916666605, 4.11) [a] 
(0.9608852083333271, 4.27) [a] 
(0.9608866666666604, 4.35) [a] 
(0.9608881249999938, 4.37) [a] 
(0.9608885416666604, 4.4) [a] 
(0.960888958333327, 4.5) [a] 
(0.9608893749999936, 4.51) [a] 
(0.960889583333327, 4.57) [a] 
(0.9608908333333269, 5.05) [a] 
(0.9608922916666602, 5.13) [a] 
(0.9609062499999935, 5.21) [a] 
(0.9609064583333269, 5.53) [a] 
(0.9609320833333269, 5.62) [a] 
(0.9609577083333269, 5.63) [a] 
(0.9609606249999936, 5.74) [a] 
(0.9609612499999935, 5.75) [a] 
(0.9609627083333269, 6.33) [a] 
(0.9609641666666602, 6.34) [a] 
(0.9609783333333268, 7.19) [a] 
(0.9609922916666601, 7.22) [a] 
(0.9609924999999935, 7.49) [a] 
(0.9609929166666601, 7.5) [a] 
(0.96099416666666, 7.54) [a] 
(0.9609956249999934, 7.64) [a] 
(0.9610377083333266, 7.73) [a] 
(0.9610379166666599, 8.84) [a] 
(0.9610393749999933, 9.64) [a] 
(0.9610402083333266, 9.67) [a] 
(0.9610406249999932, 9.86) [a] 
(0.9610408333333266, 10.04) [a] 
(0.9610412499999932, 10.05) [a] 
(0.9610416666666598, 10.16) [a] 
(0.9610418749999932, 10.26) [a] 
(0.9610422916666598, 10.27) [a] 
(0.9610427083333264, 10.28) [a] 
(0.961043124999993, 10.32) [a] 
(0.9610433333333264, 10.97) [a] 
(0.961043749999993, 10.99) [a] 
(0.9610441666666596, 11.04) [a] 
(0.961044374999993, 11.33) [a] 
(0.9610447916666596, 11.34) [a] 
(0.9610452083333262, 13.03) [a] 
(0.9611622916666596, 13.81) [a] 
(0.9611685416666597, 13.91) [a] 
(0.9611706249999931, 13.92) [a] 
(0.9611716666666598, 13.94) [a] 
(0.9611720833333264, 14.17) [a] 
(0.961172499999993, 14.18) [a] 
(0.9611727083333264, 14.23) [a] 
(0.961173124999993, 14.35) [a] 
(0.9611735416666596, 14.47) [a] 
(0.9611737499999929, 14.51) [a] 
(0.9611741666666596, 14.52) [a] 
(0.9611949999999929, 14.53) [a] 
(0.9612158333333263, 14.66) [a] 
(0.9612162499999929, 14.68) [a] 
(0.9612633333333261, 14.79) [a] 
(0.9612868749999928, 15.08) [a] 
(0.9612872916666594, 15.27) [a] 
(0.9612874999999927, 15.33) [a] 
(0.9612879166666594, 15.5) [a] 
(0.961288333333326, 15.58) [a] 
(0.961288958333326, 15.59) [a] 
(0.9612893749999926, 15.63) [a] 
(0.9612902083333259, 15.74) [a] 
(0.9612904166666593, 15.75) [a] 
(0.9612908333333259, 15.85) [a] 
(0.9612912499999925, 15.86) [a] 
(0.9612916666666592, 15.87) [a] 
(0.9612920833333258, 15.99) [a] 
(0.9612924999999924, 16.02) [a] 
(0.9612927083333258, 16.03) [a] 
(0.9612935416666591, 16.05) [a] 
(0.9612947916666591, 16.06) [a] 
(0.9612952083333257, 16.07) [a] 
(0.9612956249999923, 16.1) [a] 
(0.9612958333333257, 16.11) [a] 
(0.9612962499999923, 16.14) [a] 
(0.9612966666666589, 16.16) [a] 
(0.9612968749999923, 16.17) [a] 
(0.9612972916666589, 16.56) [a] 
(0.9612981249999922, 16.82) [a] 
(0.9612985416666588, 17.66) [a] 
(0.9612991666666588, 18.39) [a] 
(0.9613039583333254, 18.4) [a] 
(0.9613047916666587, 18.41) [a] 
(0.9613122916666585, 18.52) [a] 
(0.9613143749999918, 18.53) [a] 
(0.9613149999999918, 18.54) [a] 
(0.9613156249999918, 18.55) [a] 
(0.9613164583333251, 18.56) [a] 
(0.9613170833333251, 18.57) [a] 
(0.9613177083333251, 18.58) [a] 
(0.9613245833333249, 18.74) [a] 
(0.9613281249999915, 18.75) [a] 
(0.9613295833333249, 18.88) [a] 
(0.9613302083333248, 19.03) [a] 
(0.9613308333333248, 19.04) [a] 
(0.9613541666666582, 19.16) [a] 
(0.9613543749999915, 19.92) [a] 
(0.9613552083333249, 19.97) [a] 
(0.9613554166666582, 19.98) [a] 
(0.9613562499999916, 20) [a] 
(0.9613572916666583, 20.18) [a] 
(0.9613577083333249, 20.19) [a] 
(0.9613579166666583, 20.22) [a] 
(0.9613587499999916, 20.26) [a] 
(0.9613591666666582, 20.28) [a] 
(0.9614104166666583, 20.32) [a] 
(0.9614360416666583, 20.36) [a] 
(0.9615387499999917, 20.37) [a] 
(0.9615643749999917, 20.41) [a] 
(0.961615833333325, 20.42) [a] 
(0.9616162499999916, 20.76) [a] 
(0.9616418749999917, 20.77) [a] 
(0.9616674999999917, 20.82) [a] 
(0.9616679166666583, 21.21) [a] 
(0.961674583333325, 21.32) [a] 
(0.961700208333325, 21.72) [a] 
(0.9617237499999917, 21.73) [a] 
(0.9617493749999917, 21.76) [a] 
(0.9617497916666583, 21.91) [a] 
(0.9617499999999917, 22.41) [a] 
(0.9617504166666583, 22.42) [a] 
(0.961751458333325, 22.43) [a] 
(0.9617524999999917, 22.45) [a] 
(0.9617529166666583, 22.47) [a] 
(0.9617531249999917, 22.48) [a] 
(0.9617535416666583, 22.65) [a] 
(0.9617599999999916, 23.9) [a] 
(0.961766458333325, 23.91) [a] 
(0.961767083333325, 24.14) [a] 
(0.9617677083333249, 24.17) [a] 
(0.9617685416666583, 24.22) [a] 
(0.9617712499999915, 24.41) [a] 
(0.9617777083333249, 24.81) [a] 
(0.9617789583333248, 25.83) [a] 
(0.9617797916666582, 25.87) [a] 
(0.9617804166666581, 26.21) [a] 
(0.9617845833333247, 26.22) [a] 
(0.961787291666658, 26.23) [a] 
(0.9617902083333246, 26.24) [a] 
(0.961790416666658, 26.32) [a] 
(0.9617908333333246, 26.33) [a] 
(0.9617912499999912, 26.35) [a] 
(0.9617918749999912, 26.44) [a] 
(0.9617945833333245, 26.5) [a] 
(0.9617954166666578, 26.55) [a] 
(0.9618018749999911, 26.83) [a] 
(0.9618022916666578, 28.42) [a] 
(0.9618024999999911, 28.49) [a] 
(0.9618029166666577, 28.52) [a] 
(0.9618033333333244, 28.54) [a] 
(0.9618035416666577, 28.56) [a] 
(0.961804374999991, 28.57) [a] 
(0.9618045833333244, 28.58) [a] 
(0.9618052083333244, 28.95) [a] 
(0.9618058333333244, 29.04) [a] 
(0.9618072916666577, 29.21) [a] 
(0.9618079166666577, 30.29) [a] 
(0.9618085416666576, 30.38) [a] 
(0.9618152083333243, 30.55) [a] 
(0.9618214583333242, 33.98) [a] 
(0.9618220833333242, 33.99) [a] 
(0.9618227083333242, 34.69) [a] 
(0.9618235416666575, 34.87) [a] 
(0.9618247916666575, 34.88) [a] 
(0.9618256249999908, 34.9) [a] 
(0.9618262499999908, 34.91) [a] 
(0.9618268749999908, 35.05) [a] 
(0.9618283333333241, 35.63) [a] 
(0.961828958333324, 35.7) [a] 
(0.9618299999999907, 35.78) [a] 
(0.9618304166666574, 37.37) [a] 
(0.9618306249999907, 37.38) [a] 
(0.9618308333333241, 38.02) [a] 
(0.9618314583333241, 38.33) [a] 
(0.9619485416666574, 38.44) [a] 
(0.9619506249999908, 40.72) [a] 
(0.9619508333333242, 42.53) [a] 
(0.9619512499999908, 42.64) [a] 
(0.9619518749999908, 42.8) [a] 
(0.9619522916666574, 42.82) [a] 
(0.9619524999999908, 42.83) [a] 
(0.9619529166666574, 42.86) [a] 
(0.9619535416666574, 42.88) [a] 
(0.961953958333324, 42.99) [a] 
(0.9619543749999906, 44.69) [a] 
(0.9619547916666572, 47.06) [a] 
(0.9619756249999906, 47.11) [a] 
(0.9619760416666572, 49.81) [a] 
(0.9619764583333238, 50.32) [a] 
(0.9619766666666572, 50.33) [a] 
(0.9619770833333238, 50.56) [a] 
(0.9619774999999904, 50.74) [a] 
(0.9619983333333237, 51.98) [a] 
(0.9620191666666571, 52.76) [a] 
(0.9620195833333237, 55.27) [a] 
(0.9620199999999903, 55.38) [a] 
(0.9620202083333237, 57.36) [a] 
(0.9620206249999903, 57.38) [a] 
(0.9620210416666569, 57.39) [a] 
(0.9620212499999903, 57.4) [a] 
(0.9620216666666569, 57.41) [a] 
(0.9620220833333235, 57.44) [a] 
(0.9620222916666569, 57.61) [a] 
(0.9620227083333235, 57.64) [a] 
(0.9620231249999901, 58.14) [a] 
(0.9620439583333235, 58.97) [a] 
(0.9620856249999902, 59.03) [a] 
(0.9621064583333235, 59.08) [a] 
(0.9621272916666569, 59.12) [a] 
(0.9621481249999903, 59.56) [a] 
(0.9621689583333236, 59.64) [a] 
(0.962169791666657, 60.77) [a] 
(0.9622868749999903, 63.47) [a] 
(0.9623077083333237, 70.03) [a] 
(0.962328541666657, 70.25) [a] 
(0.9623493749999904, 70.63) [a] 
(0.9623495833333238, 70.86) [a] 
(0.9623499999999904, 70.9) [a] 
(0.962350416666657, 70.91) [a] 
(0.9623506249999904, 70.92) [a] 
(0.962351041666657, 71.43) [a] 
(0.9623512499999903, 72.01) [a] 
(0.962351666666657, 72.03) [a] 
(0.9623520833333236, 72.05) [a] 
(0.9623522916666569, 72.06) [a] 
(0.9623527083333235, 72.09) [a] 
(0.9623531249999902, 72.12) [a] 
(0.9623533333333235, 72.15) [a] 
(0.9623537499999901, 72.16) [a] 
(0.9623541666666567, 72.21) [a] 
(0.9623543749999901, 73.23) [a] 
(0.9623554166666568, 74.31) [a] 
(0.9623560416666568, 76.34) [a] 
(0.9623566666666568, 76.85) [a] 
(0.9623574999999901, 76.86) [a] 
(0.9623581249999901, 76.96) [a] 
(0.9623583333333234, 77.35) [a] 
(0.96235874999999, 77.38) [a] 
(0.9623591666666567, 77.39) [a] 
(0.9623597916666566, 79.86) [a] 
(0.96236062499999, 79.87) [a] 
(0.9623608333333233, 80.14) [a] 
(0.9623610416666567, 82.32) [a] 
(0.96236187499999, 86.55) [a] 
(0.96236249999999, 86.61) [a] 
(0.96236312499999, 88) [a] 
(0.9623635416666566, 88.03) [a] 
(0.9623639583333232, 88.04) [a] 
(0.9623641666666566, 88.07) [a] 
(0.9623645833333232, 88.11) [a] 
(0.9623649999999898, 88.22) [a] 
(0.9623885416666564, 88.55) [a] 
(0.9623887499999898, 88.87) [a] 
(0.9623891666666564, 88.88) [a] 
(0.962389583333323, 88.97) [a] 
(0.9623897916666564, 88.98) [a] 
(0.962390208333323, 89.14) [a] 
(0.9623906249999896, 89.16) [a] 
(0.9623910416666562, 96.58) [a] 
(0.9623914583333228, 99.11) [a] 
(0.9623920833333228, 99.66) [a] 
(0.9623924999999894, 100.67) [a] 
(0.962392916666656, 100.71) [a] 
(0.9623931249999894, 100.82) [a] 
(0.9623937499999894, 101) [a] 
(0.962394166666656, 101.02) [a] 
(0.9623945833333226, 101.78) [a] 
(0.9623949999999892, 102.02) [a] 
(0.9623954166666558, 104.7) [a] 
(0.9623956249999892, 105.68) [a] 
(0.9623964583333225, 105.79) [a] 
(0.9623966666666559, 105.87) [a] 
(0.9623970833333225, 105.93) [a] 
(0.9623974999999891, 105.94) [a] 
(0.9623977083333225, 105.96) [a] 
(0.9623981249999891, 105.97) [a] 
(0.9623985416666557, 106.09) [a] 
(0.9623987499999891, 106.11) [a] 
(0.9623995833333223, 106.22) [a] 
(0.9623997916666557, 106.23) [a] 
(0.9624002083333223, 106.31) [a] 
(0.9624006249999889, 106.32) [a] 
(0.9624008333333223, 106.35) [a] 
(0.9624012499999889, 106.36) [a] 
(0.9624016666666555, 106.37) [a] 
(0.9624018749999889, 106.4) [a] 
(0.9624022916666555, 106.42) [a] 
(0.9624258333333221, 110.25) [a] 
(0.9624262499999887, 112.57) [a] 
(0.9624264583333221, 112.58) [a] 
(0.9624266666666554, 116.02) [a] 
(0.9624272916666554, 119.24) [a] 
(0.9624322916666553, 119.25) [a] 
(0.9624329166666553, 119.26) [a] 
(0.9624343749999886, 119.27) [a] 
(0.9624349999999886, 119.3) [a] 
(0.9624364583333219, 119.31) [a] 
(0.9624370833333219, 119.32) [a] 
(0.9624377083333219, 119.46) [a] 
(0.9624381249999885, 119.54) [a] 
(0.9624385416666551, 119.55) [a] 
(0.9624387499999885, 119.56) [a] 
(0.9624395833333218, 119.6) [a] 
(0.9624410416666551, 119.63) [a] 
(0.9624452083333218, 119.64) [a] 
(0.9624466666666551, 119.65) [a] 
(0.9624472916666551, 119.66) [a] 
(0.9624479166666551, 119.67) [a] 
(0.9624481249999884, 119.69) [a] 
(0.962448541666655, 119.7) [a] 
(0.962449791666655, 119.73) [a] 
(0.962450416666655, 119.87) [a] 
(0.9624512499999883, 119.9) [a] 
(0.9624514583333217, 119.94) [a] 
(0.9624518749999883, 119.98) [a] 
(0.9624522916666549, 120) [a] 
(0.9624524999999883, 120.03) [a] 
(0.9624529166666549, 120.04) [a] 
(0.9624533333333215, 120.07) [a] 
(0.9624535416666549, 120.08) [a] 
(0.9624539583333215, 122.89) [a] 
(0.9624545833333215, 122.92) [a] 
(0.9624552083333214, 126.33) [a] 
(0.9624560416666548, 126.38) [a] 
(0.9624564583333214, 128.81) [a] 
(0.9624566666666547, 133.1) [a] 
(0.9624570833333214, 133.12) [a] 
(0.9624577083333213, 133.4) [a] 
(0.9624585416666547, 133.77) [a] 
(0.9624591666666547, 135.09) [a] 
(0.962460624999988, 135.26) [a] 
(0.9624612499999879, 135.3) [a] 
(0.9624616666666546, 135.33) [a] 
(0.9624622916666545, 135.61) [a] 
(0.9624629166666545, 135.62) [a] 
(0.9624637499999878, 135.79) [a] 
(0.9624649999999878, 135.8) [a] 
(0.9624658333333211, 135.82) [a] 
(0.9624664583333211, 135.87) [a] 
(0.9624666666666545, 137.16) [a] 
(0.9624670833333211, 137.17) [a] 
(0.9624674999999877, 137.18) [a] 
(0.9624677083333211, 137.19) [a] 
(0.962468333333321, 137.2) [a] 
(0.9624687499999877, 137.68) [a] 
(0.9624693749999876, 137.72) [a] 
(0.9625114583333209, 139.3) [a] 
(0.9626518749999875, 139.31) [a] 
(0.9626799999999874, 139.32) [a] 
(0.9627220833333207, 139.34) [a] 
(0.9627360416666539, 139.6) [a] 
(0.9627366666666539, 152.83) [a] 
(0.9627402083333205, 152.99) [a] 
(0.9627408333333205, 153) [a] 
(0.9627422916666538, 153.01) [a] 
(0.9627443749999871, 153.02) [a] 
(0.9627449999999871, 153.07) [a] 
(0.9627464583333204, 153.27) [a] 
(0.9627479166666537, 153.28) [a] 
(0.9627485416666537, 153.31) [a] 
(0.9627491666666537, 153.32) [a] 
(0.962749999999987, 153.36) [a] 
(0.9627504166666536, 153.37) [a] 
(0.9627516666666536, 153.39) [a] 
(0.9627531249999869, 153.43) [a] 
(0.9627552083333202, 153.44) [a] 
(0.9627579166666534, 153.45) [a] 
(0.9627587499999868, 153.46) [a] 
(0.9627593749999868, 153.51) [a] 
(0.9627599999999867, 153.68) [a] 
(0.96276145833332, 153.87) [a] 
(0.96276208333332, 153.93) [a] 
(0.9627649999999867, 154.54) [a] 
(0.9627656249999866, 154.55) [a] 
(0.9627662499999866, 154.61) [a] 
(0.96276708333332, 154.65) [a] 
(0.9627683333333199, 154.91) [a] 
(0.9627691666666532, 154.92) [a] 
(0.9627697916666532, 154.93) [a] 
(0.9627704166666532, 155.04) [a] 
(0.9627712499999865, 155.16) [a] 
(0.9627724999999865, 155.17) [a] 
(0.9627739583333198, 155.2) [a] 
(0.9628020833333197, 158.13) [a] 
(0.962803541666653, 160.67) [a] 
(0.9628177083333197, 168.81) [a] 
(0.962831666666653, 170.77) [a] 
(0.9628383333333197, 184.29) [a] 
(0.9628389583333197, 198.39) [a] 
(0.9628393749999863, 206.47) [a] 
(0.9628397916666529, 206.53) [a] 
(0.9628537499999862, 208.54) [a] 
(0.9628539583333195, 211.73) [a] 
(0.9628541666666529, 211.83) [a] 
(0.9628545833333195, 211.84) [a] 
(0.9628549999999861, 211.87) [a] 
(0.9628556249999861, 211.88) [a] 
(0.9628560416666527, 211.93) [a] 
(0.9628564583333193, 212.22) [a] 
(0.962870624999986, 217.57) [a] 
(0.9628714583333193, 219.57) [a] 
(0.9628716666666527, 219.85) [a] 
(0.962871874999986, 222.72) [a] 
(0.9628722916666527, 222.73) [a] 
(0.9628727083333193, 222.8) [a] 
(0.9628729166666526, 222.81) [a] 
(0.9628733333333193, 222.88) [a] 
(0.9628737499999859, 222.93) [a] 
(0.9628739583333192, 227.75) [a] 
(0.9628879166666525, 231.08) [a] 
(0.9629018749999858, 231.42) [a] 
(0.9629022916666524, 231.8) [a] 
(0.9629024999999858, 231.82) [a] 
(0.9630193749999858, 240.42) [a] 
(0.9630195833333192, 241.2) [a] 
(0.9630199999999858, 241.23) [a] 
(0.9630206249999858, 242.14) [a] 
(0.9630212499999857, 242.51) [a] 
(0.963022708333319, 242.66) [a] 
(0.9630247916666523, 242.67) [a] 
(0.9630254166666523, 243.32) [a] 
(0.9630262499999857, 243.48) [a] 
(0.9630268749999856, 243.49) [a] 
(0.9630283333333189, 243.52) [a] 
(0.9630295833333189, 243.71) [a] 
(0.9630304166666522, 243.83) [a] 
(0.9630308333333188, 245.43) [a] 
(0.9630312499999855, 245.97) [a] 
(0.9630318749999854, 246.22) [a] 
(0.9630324999999854, 246.23) [a] 
(0.9630333333333188, 246.24) [a] 
(0.9630339583333187, 248.97) [a] 
(0.9630343749999853, 249.06) [a] 
(0.9630349999999853, 249.59) [a] 
(0.963049166666652, 254.2) [a] 
(0.9630495833333186, 263.19) [a] 
(0.963166666666652, 265.97) [a] 
(0.9632068749999853, 266.5) [a] 
(0.9632074999999852, 274.29) [a] 
(0.9632079166666518, 285.43) [a] 
(0.9632087499999852, 285.6) [a] 
(0.9632102083333185, 285.86) [a] 
(0.9632104166666519, 287.57) [a] 
(0.9632243749999851, 300.46) [a] 
(0.9632385416666518, 302.32) [a] 
(0.9632524999999851, 302.33) [a] 
(0.9632529166666517, 307.7) [a] 
(0.963253124999985, 307.72) [a] 
(0.9632535416666517, 307.73) [a] 
(0.9632539583333183, 307.74) [a] 
(0.9632541666666516, 307.8) [a] 
(0.9632545833333183, 307.85) [a] 
(0.9632549999999849, 307.92) [a] 
(0.9632552083333182, 307.93) [a] 
(0.9632556249999848, 307.94) [a] 
(0.9632560416666515, 308.01) [a] 
(0.9632562499999848, 308.06) [a] 
(0.9632566666666514, 308.09) [a] 
(0.963257083333318, 308.21) [a] 
(0.9632574999999847, 308.74) [a] 
(0.963257708333318, 308.75) [a] 
(0.9632581249999846, 308.76) [a] 
(0.9632585416666513, 308.78) [a] 
(0.9632591666666512, 316.9) [a] 
(0.9632597916666512, 316.91) [a] 
(0.9632672916666511, 316.92) [a] 
(0.9632687499999845, 317.14) [a] 
(0.9632708333333178, 317.24) [a] 
(0.9632714583333177, 317.29) [a] 
(0.9632722916666511, 317.3) [a] 
(0.963272916666651, 320.53) [a] 
(0.963273541666651, 321.51) [a] 
(0.963279166666651, 329.18) [a] 
(0.9632810416666511, 329.4) [a] 
(0.9632824999999844, 329.66) [a] 
(0.9632839583333177, 337.94) [a] 
(0.9632962499999843, 338.06) [a] 
(0.9632964583333177, 358.77) [a] 
(0.9632968749999843, 388.96) [a] 
(0.9634139583333177, 424.82) [a] 
(0.9634143749999843, 433.16) [a] 
(0.9634147916666509, 433.29) [a] 
(0.9634154166666509, 433.39) [a] 
(0.9634158333333175, 439.22) [a] 
(0.9634164583333175, 452.75) [a] 
(0.9634168749999841, 462.04) [a] 
(0.9634172916666507, 480.25) [a] 
(0.9634179166666507, 480.37) [a] 
(0.9634189583333173, 480.39) [a] 
(0.9634195833333172, 480.4) [a] 
(0.9634199999999838, 480.45) [a] 
(0.9634204166666505, 480.46) [a] 
(0.9634206249999838, 480.51) [a] 
(0.963421458333317, 480.52) [a] 
(0.9634216666666504, 480.53) [a] 
(0.963422083333317, 480.54) [a] 
(0.9634224999999836, 481.39) [a] 
(0.9634231249999836, 482.87) [a] 
(0.963423333333317, 489.41) [a] 
(0.9634243749999837, 489.43) [a] 
(0.9634247916666503, 489.47) [a] 
(0.9634252083333169, 489.48) [a] 
(0.9634254166666503, 489.49) [a] 
(0.9634258333333169, 489.51) [a] 
(0.9634262499999835, 489.57) [a] 
(0.9634264583333169, 489.62) [a] 
(0.9634268749999835, 493.76) [a] 
(0.9634272916666501, 501.71) [a] 
(0.9634274999999834, 501.72) [a] 
(0.9634279166666501, 501.73) [a] 
(0.9634283333333167, 503.94) [a] 
(0.96342854166665, 503.96) [a] 
(0.9634287499999834, 511.92) [a] 
(0.96342916666665, 512.02) [a] 
(0.9634295833333166, 512.08) [a] 
(0.9634302083333166, 512.12) [a] 
(0.9634306249999832, 512.13) [a] 
(0.9634308333333166, 512.18) [a] 
(0.9634316666666499, 512.25) [a] 
(0.9634318749999833, 512.3) [a] 
(0.9634322916666499, 513) [a] 
(0.9634327083333165, 518.31) [a] 
(0.9634329166666499, 519.2) [a] 
(0.9634333333333165, 527.88) [a] 
(0.9634337499999831, 527.93) [a] 
(0.9634339583333165, 527.95) [a] 
(0.9634343749999831, 528.23) [a] 
(0.9634347916666497, 528.96) [a] 
(0.9635516666666497, 541.46) [a] 
(0.9635520833333163, 556.65) [a] 
(0.9635522916666497, 557.02) [a] 
(0.9635541666666497, 557.03) [a] 
(0.9635545833333163, 565.09) [a] 
(0.9635547916666497, 596.43) [a] 
(0.9635549999999831, 644.35) [a] 
(0.9635554166666497, 644.36) [a] 
(0.9635558333333163, 644.51) [a] 
(0.9635560416666497, 644.52) [a] 
(0.963556249999983, 699.33) [a] 
(0.9635566666666496, 699.51) [a] 
(0.9635570833333162, 720.5) [a] 
(0.9637183333333161, 791.33) [a] 
(0.9637585416666494, 791.55) [a] 
(0.9637989583333161, 791.57) [a] 
(0.9639197916666494, 798.73) [a] 
(0.9639199999999828, 836.01) [a] 
(0.9639204166666494, 852.15) [a] 
(0.963920833333316, 884.61) [a] 
(0.9639210416666494, 902.13) [a] 
(0.9639212499999827, 958.94) [a] 
(0.9639216666666494, 963.07) [a] 
(0.963922083333316, 963.23) [a] 
(0.9639222916666493, 963.28) [a] 
(0.963922708333316, 963.31) [a] 
(0.9639508333333159, 985.55) [a] 
(0.9639647916666492, 985.6) [a] 
(0.9639789583333158, 985.73) [a] 
(0.9639929166666491, 985.78) [a] 
(0.9640068749999824, 986.25) [a] 
(0.964021041666649, 1009.4) [a] 
(0.9640489583333156, 1009.9) [a] 
(0.9640891666666489, 1020.2) [a] 
(0.9641295833333156, 1020.3) [a] 
(0.9641697916666488, 1023.8) [a] 
(0.9641754166666487, 1052.6) [a] 
(0.9641760416666487, 1052.7) [a] 
(0.964177499999982, 1053.2) [a] 
(0.964184374999982, 1053.4) [a] 
(0.964240624999982, 1076.7) [a] 
(0.9642687499999819, 1076.8) [a] 
(0.9642968749999818, 1083.7) [a] 
(0.9643247916666484, 1138.2) [a] 
(0.9643389583333151, 1139.5) [a] 
(0.9643670833333151, 1139.6) [a] 
(0.9643810416666484, 1139.7) [a] 
(0.9643949999999817, 1139.8) [a] 
(0.9643954166666483, 1149.2) [a] 
(0.9643956249999817, 1149.3) [a] 
(0.9643960416666483, 1157.3) [a] 
(0.9644102083333149, 1171.4) [a] 
(0.9644241666666482, 1171.9) [a] 
(0.9644247916666482, 1172) [a] 
(0.9644387499999815, 1173.2) [a] 
(0.9644529166666481, 1175.6) [a] 
(0.9644668749999814, 1176.8) [a] 
(0.9644797916666481, 1200) [a] 
(0.9644939583333147, 1243.8) [a] 
(0.964507916666648, 1244) [a] 
(0.9645083333333146, 1258) [a] 
(0.9645089583333146, 1323.3) [a] 
(0.9645093749999812, 1323.6) [a] 
(0.9645097916666479, 1323.7) [a] 
(0.9645110416666478, 1332.5) [a] 
(0.9645114583333144, 1332.6) [a] 
(0.964511874999981, 1333.1) [a] 
(0.9645129166666476, 1335.7) [a] 
(0.9645135416666476, 1369.2) [a] 
(0.9645139583333142, 1370) [a] 
(0.9645145833333142, 1371.2) [a] 
(0.9645149999999808, 1372.1) [a] 
(0.9645154166666474, 1383.9) [a] 
(0.9645156249999808, 1517.3) [a] 
(0.964518958333314, 1610.6) [a] 
(0.9645204166666473, 1611) [a] 
(0.9645216666666473, 1614.7) [a] 
(0.9645220833333139, 1615) [a] 
(0.9645235416666472, 1634.8) [a] 
(0.9645304166666471, 1634.9) [a] 
(0.9645318749999804, 1646.4) [a] 
(0.9645352083333137, 1646.5) [a] 
(0.964537291666647, 1647.2) [a] 
(0.9645437499999804, 1719) [a] 
(0.9645452083333137, 1737.2) [a] 
(0.964547291666647, 2436.9) [a] 
(0.9645477083333136, 2437.1) [a] 
(0.964547916666647, 2437.2) [a] 
(0.9645483333333136, 2437.4) [a] 
(0.9645489583333136, 2439.9) [a] 
(0.9645493749999802, 2440) [a] 
(0.9645504166666468, 2440.6) [a] 
(0.9645510416666467, 2446.3) [a] 
(0.9645520833333133, 2446.4) [a] 
(0.9645524999999799, 2491.3) [a] 
(0.9645758333333133, 3164) [a] 
(0.9645993749999799, 3164.4) [a] 
(0.9646229166666466, 3388.2) [a] 
},{(0.8409019251496185, 0) [b] 
(0.8794239765593256, 0.001) [b] 
(0.8941673956904447, 0.002) [b] 
(0.8980740368648389, 0.003) [b] 
(0.9007888562503038, 0.004) [b] 
(0.9032523422438659, 0.005) [b] 
(0.9043891045563026, 0.006) [b] 
(0.90555831365529, 0.007) [b] 
(0.9060748465270825, 0.008) [b] 
(0.9072333621897868, 0.009) [b] 
(0.9077799506233065, 0.01) [b] 
(0.9082630420074872, 0.011) [b] 
(0.908907881506822, 0.012) [b] 
(0.9099842106403497, 0.013) [b] 
(0.9105338251738081, 0.014) [b] 
(0.9114343656786252, 0.015) [b] 
(0.9127861665149333, 0.016) [b] 
(0.9136942873309926, 0.017) [b] 
(0.9140815761482136, 0.018) [b] 
(0.9143084870713347, 0.019) [b] 
(0.9144112377608541, 0.02) [b] 
(0.91483915437218, 0.021) [b] 
(0.9154262017553301, 0.022) [b] 
(0.9157128072076507, 0.023) [b] 
(0.915716245164899, 0.024) [b] 
(0.9158078676522398, 0.025) [b] 
(0.9158477479563203, 0.026) [b] 
(0.9158944671689204, 0.027) [b] 
(0.916006101401039, 0.028) [b] 
(0.9160067889924887, 0.029) [b] 
(0.9160562955768644, 0.03) [b] 
(0.9161662639497443, 0.031) [b] 
(0.9162617249336855, 0.032) [b] 
(0.9162898499336856, 0.033) [b] 
(0.9162926208584686, 0.034) [b] 
(0.9163214334499182, 0.035) [b] 
(0.9163297599302568, 0.036) [b] 
(0.9164407649820239, 0.037) [b] 
(0.9164456192401401, 0.038) [b] 
(0.9164786880232089, 0.041) [b] 
(0.9164814658009867, 0.042) [b] 
(0.9164856324676534, 0.043) [b] 
(0.9165048850282439, 0.045) [b] 
(0.916543390149425, 0.046) [b] 
(0.9169017234827583, 0.047) [b] 
(0.916904473848557, 0.049) [b] 
(0.9173079460707793, 0.05) [b] 
(0.9173159321818903, 0.052) [b] 
(0.9173180155152237, 0.053) [b] 
(0.917321393710997, 0.054) [b] 
(0.9173247719067702, 0.055) [b] 
(0.9173424802401035, 0.056) [b] 
(0.9174048458018798, 0.058) [b] 
(0.917408223997653, 0.059) [b] 
(0.9174691326914961, 0.06) [b] 
(0.9175217811882533, 0.061) [b] 
(0.9176727577894618, 0.062) [b] 
(0.9177385807933781, 0.063) [b] 
(0.9177396224600448, 0.064) [b] 
(0.917766749196156, 0.07) [b] 
(0.9179307379065405, 0.073) [b] 
(0.9179349045732071, 0.075) [b] 
(0.9179376823509849, 0.076) [b] 
(0.9179411545732071, 0.077) [b] 
(0.9187630266349147, 0.078) [b] 
(0.918813764482137, 0.079) [b] 
(0.9188408912182482, 0.08) [b] 
(0.9190833391349148, 0.081) [b] 
(0.9190840335793592, 0.082) [b] 
(0.9190854224682481, 0.083) [b] 
(0.9190909780238037, 0.085) [b] 
(0.9190944502460259, 0.086) [b] 
(0.9191041724682482, 0.087) [b] 
(0.9191125058015815, 0.088) [b] 
(0.9191263946904704, 0.089) [b] 
(0.9191329919126927, 0.09) [b] 
(0.9193008454093747, 0.091) [b] 
(0.9193057065204858, 0.092) [b] 
(0.9193101263829258, 0.093) [b] 
(0.9193428086745925, 0.094) [b] 
(0.9193713242995926, 0.095) [b] 
(0.9193803520773702, 0.096) [b] 
(0.9194944145773701, 0.097) [b] 
(0.9195841281190368, 0.098) [b] 
(0.9196119492995923, 0.099) [b] 
(0.9198198486051479, 0.1) [b] 
(0.91984207082737, 0.101) [b] 
(0.9199582166607033, 0.102) [b] 
(0.9199596055495922, 0.103) [b] 
(0.9201325222162589, 0.104) [b] 
(0.9201339111051476, 0.105) [b] 
(0.9201852999940364, 0.106) [b] 
(0.9201873833273697, 0.107) [b] 
(0.9201958828397401, 0.108) [b] 
(0.9202674106175177, 0.109) [b] 
(0.9202687995064066, 0.11) [b] 
(0.9202715772841844, 0.111) [b] 
(0.9202743550619621, 0.112) [b] 
(0.9202760911730732, 0.113) [b] 
(0.9207442559670806, 0.115) [b] 
(0.9214527531580918, 0.116) [b] 
(0.9214537948247585, 0.117) [b] 
(0.9214642114914252, 0.118) [b] 
(0.9220494174839345, 0.119) [b] 
(0.9220521952617123, 0.12) [b] 
(0.9220535841506012, 0.122) [b] 
(0.9220556674839345, 0.123) [b] 
(0.9220570563728233, 0.124) [b] 
(0.9220584452617121, 0.125) [b] 
(0.9220601813728232, 0.126) [b] 
(0.9220608758172676, 0.127) [b] 
(0.9220619174839343, 0.128) [b] 
(0.9221938619283787, 0.131) [b] 
(0.922195945261712, 0.132) [b] 
(0.9222045471599181, 0.133) [b] 
(0.9222094082710292, 0.134) [b] 
(0.922210797159918, 0.135) [b] 
(0.9222128804932513, 0.136) [b] 
(0.9222142693821402, 0.137) [b] 
(0.9222177416043624, 0.138) [b] 
(0.9222934360488069, 0.139) [b] 
(0.922320562784918, 0.145) [b] 
(0.9223413961182514, 0.146) [b] 
(0.9223622294515847, 0.149) [b] 
(0.9232602179573319, 0.151) [b] 
(0.9237476974318803, 0.152) [b] 
(0.9237990110607802, 0.153) [b] 
(0.9250048813399264, 0.154) [b] 
(0.9255436744433747, 0.155) [b] 
(0.9258575342545405, 0.156) [b] 
(0.9259088478834404, 0.157) [b] 
(0.9260704652890234, 0.164) [b] 
(0.926198749361273, 0.167) [b] 
(0.9262227076946062, 0.17) [b] 
(0.9262465423275763, 0.171) [b] 
(0.926247583994243, 0.175) [b] 
(0.9262479312164652, 0.176) [b] 
(0.9262486256609096, 0.178) [b] 
(0.9263255961042594, 0.18) [b] 
(0.9263776041776036, 0.181) [b] 
(0.9264095486220479, 0.182) [b] 
(0.9264865190653977, 0.184) [b] 
(0.9265440826942976, 0.185) [b] 
(0.9265597076942976, 0.186) [b] 
(0.9265614438054087, 0.187) [b] 
(0.9265617910276309, 0.188) [b] 
(0.9265819299165198, 0.191) [b] 
(0.9265843604720754, 0.192) [b] 
(0.9265850549165198, 0.193) [b] 
(0.9265982493609642, 0.194) [b] 
(0.9266013743609642, 0.195) [b] 
(0.9266399594304088, 0.196) [b] 
(0.9266778066526309, 0.197) [b] 
(0.9267041955415196, 0.198) [b] 
(0.9267069733192973, 0.199) [b] 
(0.9267118344304083, 0.2) [b] 
(0.9267153066526305, 0.201) [b] 
(0.9267156538748527, 0.202) [b] 
(0.9267413106893025, 0.203) [b] 
(0.9267423523559691, 0.204) [b] 
(0.9267440884670802, 0.205) [b] 
(0.9267447829115246, 0.206) [b] 
(0.9267475606893023, 0.207) [b] 
(0.9267871440226356, 0.208) [b] 
(0.9267885329115245, 0.209) [b] 
(0.9267902690226355, 0.211) [b] 
(0.9268200925037521, 0.212) [b] 
(0.9268270369481965, 0.213) [b] 
(0.9268273841704187, 0.215) [b] 
(0.9268291202815298, 0.216) [b] 
(0.9268308563926408, 0.217) [b] 
(0.926831203614863, 0.218) [b] 
(0.9268318980593074, 0.219) [b] 
(0.9268391897259741, 0.22) [b] 
(0.9270728702815298, 0.221) [b] 
(0.9271190508370853, 0.222) [b] 
(0.9271273841704187, 0.223) [b] 
(0.9271770369481965, 0.224) [b] 
(0.9272023841704187, 0.225) [b] 
(0.9273075925037519, 0.226) [b] 
(0.9273079397259741, 0.227) [b] 
(0.9273305091704186, 0.228) [b] 
(0.9274259952815297, 0.229) [b] 
(0.927513842503752, 0.23) [b] 
(0.9275197452815297, 0.231) [b] 
(0.9275242591704184, 0.232) [b] 
(0.9275537730593073, 0.233) [b] 
(0.9275749536148629, 0.234) [b] 
(0.9276847301198043, 0.235) [b] 
(0.9277354245642487, 0.236) [b] 
(0.9277538273420264, 0.237) [b] 
(0.9277562578975818, 0.238) [b] 
(0.9278946152522095, 0.239) [b] 
(0.9280019069188762, 0.24) [b] 
(0.9280095183957859, 0.241) [b] 
(0.928228615618008, 0.242) [b] 
(0.9283931989513413, 0.243) [b] 
(0.9284761850624523, 0.244) [b] 
(0.9285470183957855, 0.245) [b] 
(0.92873833784023, 0.246) [b] 
(0.9288678517291188, 0.247) [b] 
(0.928935560062452, 0.248) [b] 
(0.9289466711735631, 0.249) [b] 
(0.9290372961735632, 0.25) [b] 
(0.9290661156180076, 0.251) [b] 
(0.9292591711735632, 0.252) [b] 
(0.9293310461735631, 0.253) [b] 
(0.9293761850624519, 0.254) [b] 
(0.9294171161667053, 0.255) [b] 
(0.9294730189444831, 0.256) [b] 
(0.9295087828333719, 0.257) [b] 
(0.9295178106111496, 0.258) [b] 
(0.9295292689444827, 0.259) [b] 
(0.9299212828333717, 0.26) [b] 
(0.9300053106111493, 0.261) [b] 
(0.930842463388927, 0.262) [b] 
(0.9310275328333714, 0.263) [b] 
(0.931049060611149, 0.264) [b] 
(0.9310560050555933, 0.265) [b] 
(0.9310882967222598, 0.266) [b] 
(0.9310952411667043, 0.267) [b] 
(0.9310962828333709, 0.268) [b] 
(0.9314636439444818, 0.269) [b] 
(0.9316011439444816, 0.27) [b] 
(0.9317174633889259, 0.271) [b] 
(0.9318476717222592, 0.272) [b] 
(0.9318730189444814, 0.273) [b] 
(0.931908435611148, 0.274) [b] 
(0.9319403800555924, 0.275) [b] 
(0.9319525328333701, 0.276) [b] 
(0.9325733661667035, 0.277) [b] 
(0.9325837828333701, 0.278) [b] 
(0.9325966300555921, 0.279) [b] 
(0.9327771856111476, 0.28) [b] 
(0.9327907272778141, 0.281) [b] 
(0.9328112133889251, 0.282) [b] 
(0.9328271856111472, 0.283) [b] 
(0.9335982520472023, 0.284) [b] 
(0.9336322798249799, 0.285) [b] 
(0.9336610992694243, 0.286) [b] 
(0.9336683909360908, 0.287) [b] 
(0.9338277659360908, 0.288) [b] 
(0.9343985992694241, 0.289) [b] 
(0.9344246409360909, 0.29) [b] 
(0.9345795020472021, 0.291) [b] 
(0.9345836687138688, 0.292) [b] 
(0.9346454742694242, 0.293) [b] 
(0.934676377047202, 0.294) [b] 
(0.9346965159360908, 0.295) [b] 
(0.9352673492694242, 0.296) [b] 
(0.9352759634282539, 0.297) [b] 
(0.9353717967615872, 0.298) [b] 
(0.9354124217615872, 0.299) [b] 
(0.9358158939838095, 0.3) [b] 
(0.9358419356504761, 0.301) [b] 
(0.9359224912060317, 0.302) [b] 
(0.9360013106504762, 0.303) [b] 
(0.9363464495393649, 0.304) [b] 
(0.9363641578726981, 0.305) [b] 
(0.9364308245393648, 0.306) [b] 
(0.9364478384282535, 0.307) [b] 
(0.9366204078726978, 0.308) [b] 
(0.9366818662060311, 0.309) [b] 
(0.9367408939838088, 0.31) [b] 
(0.9367839495393642, 0.311) [b] 
(0.9368238800949197, 0.312) [b] 
(0.9368537412060307, 0.313) [b] 
(0.9368568662060307, 0.314) [b] 
(0.9368933245393639, 0.315) [b] 
(0.9369426300949194, 0.316) [b] 
(0.9369565189838083, 0.317) [b] 
(0.9371106856504748, 0.318) [b] 
(0.9374252689838082, 0.319) [b] 
(0.9375804773171414, 0.32) [b] 
(0.9377865298403519, 0.321) [b] 
(0.9378174326181298, 0.322) [b] 
(0.937877502062574, 0.323) [b] 
(0.9379163909514628, 0.324) [b] 
(0.9379306270625738, 0.325) [b] 
(0.9379514603959069, 0.326) [b] 
(0.9379747242847957, 0.327) [b] 
(0.9379934742847956, 0.328) [b] 
(0.9379993770625732, 0.329) [b] 
(0.9380174326181286, 0.33) [b] 
(0.9380282311462005, 0.331) [b] 
(0.9380355228128671, 0.332) [b] 
(0.9380454875093709, 0.333) [b] 
(0.9380465291760375, 0.334) [b] 
(0.9380479180649264, 0.335) [b] 
(0.9380493415929984, 0.336) [b] 
(0.9380906610374428, 0.337) [b] 
(0.938634758259665, 0.338) [b] 
(0.9387563133470558, 0.339) [b] 
(0.9387697156265987, 0.34) [b] 
(0.9390027017377097, 0.341) [b] 
(0.9390129443780696, 0.342) [b] 
(0.9390188471558473, 0.343) [b] 
(0.9391428054891805, 0.344) [b] 
(0.9391556527114027, 0.345) [b] 
(0.9392299582669582, 0.346) [b] 
(0.9392348193780694, 0.347) [b] 
(0.9392771804891803, 0.348) [b] 
(0.9392785693780691, 0.349) [b] 
(0.9393077360447357, 0.35) [b] 
(0.939309819378069, 0.351) [b] 
(0.9393222146301283, 0.352) [b] 
(0.9393465894640501, 0.353) [b] 
(0.939351103352939, 0.354) [b] 
(0.9393645749108481, 0.355) [b] 
(0.9393705816061751, 0.356) [b] 
(0.9393709288283973, 0.357) [b] 
(0.9393764843839529, 0.36) [b] 
(0.9393837760506196, 0.361) [b] 
(0.9393851995786916, 0.366) [b] 
(0.9393855468009138, 0.367) [b] 
(0.9393862412453582, 0.368) [b] 
(0.9393928384675804, 0.369) [b] 
(0.9394022134675803, 0.37) [b] 
(0.9394447565382017, 0.371) [b] 
(0.9394822999409795, 0.372) [b] 
(0.9395013616936273, 0.373) [b] 
(0.9395027505825162, 0.374) [b] 
(0.9395055283602939, 0.375) [b] 
(0.9395069172491828, 0.377) [b] 
(0.9395090005825161, 0.378) [b] 
(0.9395096950269605, 0.379) [b] 
(0.9395100422491827, 0.381) [b] 
(0.9395103894714049, 0.382) [b] 
(0.9395114311380715, 0.383) [b] 
(0.9395152505825158, 0.384) [b] 
(0.9395159450269602, 0.385) [b] 
(0.9395166394714046, 0.386) [b] 
(0.9395169866936268, 0.387) [b] 
(0.9395183755825157, 0.39) [b] 
(0.9396820813107927, 0.391) [b] 
(0.9396824285330149, 0.393) [b] 
(0.9396852755891588, 0.396) [b] 
(0.9397026358698786, 0.398) [b] 
(0.939716975898539, 0.4) [b] 
(0.9397242675652057, 0.401) [b] 
(0.9397342322617095, 0.402) [b] 
(0.9397402389570365, 0.404) [b] 
(0.9397939882927234, 0.406) [b] 
(0.9397991966260567, 0.407) [b] 
(0.9398009327371678, 0.408) [b] 
(0.9398140224336716, 0.409) [b] 
(0.9398258279892272, 0.414) [b] 
(0.9398338141003383, 0.415) [b] 
(0.939850480767005, 0.416) [b] 
(0.9398511752114493, 0.418) [b] 
(0.9398567307670049, 0.42) [b] 
(0.939858466878116, 0.421) [b] 
(0.9398605502114493, 0.422) [b] 
(0.9398668002114493, 0.425) [b] 
(0.9398674946558937, 0.426) [b] 
(0.9399827311513574, 0.428) [b] 
(0.940063286706913, 0.432) [b] 
(0.9400732514034168, 0.433) [b] 
(0.9400874866841366, 0.435) [b] 
(0.9400960278525684, 0.437) [b] 
(0.9401054028525684, 0.438) [b] 
(0.9401110969648563, 0.439) [b] 
(0.9401114441870785, 0.442) [b] 
(0.9401472012229726, 0.443) [b] 
(0.9401482428896393, 0.444) [b] 
(0.9402048401118615, 0.446) [b] 
(0.9402631701236293, 0.447) [b] 
(0.940264211790296, 0.448) [b] 
(0.940290215826968, 0.449) [b] 
(0.9403161610430174, 0.45) [b] 
(0.9403168554874618, 0.451) [b] 
(0.9403527232939, 0.454) [b] 
(0.9403569938781159, 0.458) [b] 
(0.9403583827670048, 0.46) [b] 
(0.940361760962778, 0.461) [b] 
(0.9403624554072224, 0.462) [b] 
(0.9404423470055597, 0.463) [b] 
(0.9404433886722263, 0.464) [b] 
(0.9404458192277819, 0.465) [b] 
(0.9404619300067365, 0.466) [b] 
(0.9404640133400698, 0.467) [b] 
(0.9405146701545197, 0.468) [b] 
(0.9405157118211864, 0.469) [b] 
(0.9406161942785755, 0.471) [b] 
(0.9406283470563532, 0.472) [b] 
(0.9406290415007976, 0.473) [b] 
(0.9406304303896865, 0.474) [b] 
(0.9406321665007976, 0.476) [b] 
(0.9406422359452421, 0.477) [b] 
(0.9406584589390693, 0.478) [b] 
(0.9406588061612915, 0.48) [b] 
(0.9406622783835137, 0.481) [b] 
(0.9406688756057359, 0.482) [b] 
(0.9406695700501803, 0.483) [b] 
(0.9406716533835136, 0.484) [b] 
(0.9412650561612913, 0.495) [b] 
(0.941271932075788, 0.501) [b] 
(0.9412882515202324, 0.503) [b] 
(0.9412892931868991, 0.509) [b] 
(0.9412899807783488, 0.51) [b] 
(0.9412906752227932, 0.513) [b] 
(0.9413888426610648, 0.514) [b] 
(0.9414852932783487, 0.515) [b] 
(0.9414901543894598, 0.525) [b] 
(0.941490501611682, 0.532) [b] 
(0.9414946682783487, 0.533) [b] 
(0.9414977932783487, 0.534) [b] 
(0.9415325589033487, 0.535) [b] 
(0.9415690172366821, 0.536) [b] 
(0.941573531125571, 0.537) [b] 
(0.9415738783477932, 0.538) [b] 
(0.9415745727922376, 0.539) [b] 
(0.9415776977922375, 0.54) [b] 
(0.9415780450144597, 0.541) [b] 
(0.9415797811255708, 0.542) [b] 
(0.9415842950144597, 0.543) [b] 
(0.9416179755700153, 0.544) [b] 
(0.9416214477922374, 0.545) [b] 
(0.941660990713796, 0.546) [b] 
(0.9417265348902466, 0.547) [b] 
(0.9417282710013577, 0.548) [b] 
(0.9418569287987033, 0.549) [b] 
(0.9418929994980396, 0.55) [b] 
(0.9419568075633791, 0.551) [b] 
(0.9419613214522679, 0.552) [b] 
(0.941988448188379, 0.556) [b] 
(0.9420318509661568, 0.557) [b] 
(0.9420894898550457, 0.559) [b] 
(0.9421012954106013, 0.56) [b] 
(0.9421019898550457, 0.561) [b] 
(0.9421033787439346, 0.562) [b] 
(0.942222850497992, 0.564) [b] 
(0.9422235449424364, 0.565) [b] 
(0.9422249338313253, 0.566) [b] 
(0.9422544477202142, 0.568) [b] 
(0.9422547949424364, 0.571) [b] 
(0.9422690310535474, 0.572) [b] 
(0.9422718088313252, 0.573) [b] 
(0.9422735449424363, 0.574) [b] 
(0.9423280184195505, 0.576) [b] 
(0.9423731168966647, 0.577) [b] 
(0.9424050209293344, 0.579) [b] 
(0.9424369249620042, 0.58) [b] 
(0.9424688289946739, 0.583) [b] 
(0.9425326370600133, 0.584) [b] 
(0.9425645410926831, 0.586) [b] 
(0.9426081842841825, 0.587) [b] 
(0.9426400883168522, 0.588) [b] 
(0.9426501577612967, 0.59) [b] 
(0.9426518938724078, 0.592) [b] 
(0.9426574494279634, 0.593) [b] 
(0.9426591855390745, 0.594) [b] 
(0.9426595327612967, 0.596) [b] 
(0.942660227205741, 0.597) [b] 
(0.9426605744279632, 0.599) [b] 
(0.94266161609463, 0.607) [b] 
(0.9426640466501855, 0.611) [b] 
(0.9426661299835188, 0.612) [b] 
(0.9426671716501855, 0.623) [b] 
(0.94267099109463, 0.648) [b] 
(0.94299555549504, 0.653) [b] 
(0.9429962499394844, 0.682) [b] 
(0.9430102885738061, 0.697) [b] 
(0.9430243272081278, 0.699) [b] 
(0.9430253688747945, 0.704) [b] 
(0.9430305772081278, 0.709) [b] 
(0.9430316188747945, 0.713) [b] 
(0.9430326605414612, 0.715) [b] 
(0.9430378688747945, 0.717) [b] 
(0.9430519075091162, 0.72) [b] 
(0.9430529491757829, 0.722) [b] 
(0.9430539908424496, 0.727) [b] 
(0.9434383434218824, 0.728) [b] 
(0.9434498017552158, 0.732) [b] 
(0.943451885088549, 0.737) [b] 
(0.9434525795329934, 0.74) [b] 
(0.9434568501172094, 0.741) [b] 
(0.9434776834505427, 0.77) [b] 
(0.9435193501172094, 0.771) [b] 
(0.9438012086754602, 0.774) [b] 
(0.9438029447865713, 0.776) [b] 
(0.943803986453238, 0.777) [b] 
(0.9438139511497419, 0.779) [b] 
(0.9438224923181737, 0.781) [b] 
(0.9438266589848404, 0.782) [b] 
(0.9438270062070626, 0.783) [b] 
(0.9438405478737293, 0.787) [b] 
(0.943841589540396, 0.791) [b] 
(0.9438629424614756, 0.801) [b] 
(0.94386363690592, 0.811) [b] 
(0.9438723174614756, 0.821) [b] 
(0.9438726646836978, 0.832) [b] 
(0.9439908175136716, 0.836) [b] 
(0.9439918591803383, 0.87) [b] 
(0.9439946369581161, 0.874) [b] 
(0.9439953314025605, 0.877) [b] 
(0.944002275847005, 0.878) [b] 
(0.9440033175136717, 0.884) [b] 
(0.9440043591803384, 0.889) [b] 
(0.9440054008470051, 0.907) [b] 
(0.9440060952914495, 0.925) [b] 
(0.9445014830604963, 0.929) [b] 
(0.9445128712850721, 0.96) [b] 
(0.9445142948131441, 0.962) [b] 
(0.9445146420353663, 0.986) [b] 
(0.9445189126195822, 1.062) [b] 
(0.9446626889548515, 1.101) [b] 
(0.9446630361770737, 1.11) [b] 
(0.9446886596823693, 1.139) [b] 
(0.9447079082194805, 1.159) [b] 
(0.9447313956302283, 1.161) [b] 
(0.9447378118092654, 1.162) [b] 
(0.9447612992200133, 1.194) [b] 
(0.9447627227480853, 1.262) [b] 
(0.9447655698042292, 1.266) [b] 
(0.9447707781375625, 1.314) [b] 
(0.9447771943165996, 1.39) [b] 
(0.9447921248721551, 1.522) [b] 
(0.944802888761044, 1.567) [b] 
(0.9448035832054884, 1.74) [b] 
(0.9449372158760037, 1.796) [b] 
(0.9449565695731128, 1.84) [b] 
(0.9449574911777371, 1.849) [b] 
(0.9449579519800492, 1.851) [b] 
(0.9449584127823613, 1.856) [b] 
(0.9449593343869855, 1.864) [b] 
(0.9449597951892976, 1.866) [b] 
(0.9449644032124188, 1.876) [b] 
(0.9449648640147309, 1.878) [b] 
(0.9451989464117346, 1.887) [b] 
(0.9452058584464165, 1.891) [b] 
(0.9452063192487286, 1.893) [b] 
(0.9452067800510406, 1.896) [b] 
(0.9452123096787861, 1.903) [b] 
(0.9452162297394153, 1.909) [b] 
(0.9452176121463517, 1.912) [b] 
(0.9452194553556001, 1.917) [b] 
(0.9496318854095878, 1.933) [b] 
(0.9496323462118998, 1.951) [b] 
(0.9496611656563443, 1.972) [b] 
(0.949663930470217, 1.973) [b] 
(0.9496648520748413, 1.981) [b] 
(0.9496662344817777, 1.985) [b] 
(0.949667156086402, 1.99) [b] 
(0.9496699209002747, 1.993) [b] 
(0.949670842504899, 1.996) [b] 
(0.9496713033072111, 2.002) [b] 
(0.9496969601216609, 2.01) [b] 
(0.9497052545632791, 2.032) [b] 
(0.9497103233887125, 2.039) [b] 
(0.9497126274002732, 2.046) [b] 
(0.9497144706095216, 2.053) [b] 
(0.9497149314118337, 2.056) [b] 
(0.949715739436368, 2.061) [b] 
(0.9497162002386801, 2.064) [b] 
(0.9497175826456165, 2.066) [b] 
(0.9497432394600663, 2.098) [b] 
(0.9497847116681573, 2.129) [b] 
(0.9498010311126017, 2.206) [b] 
(0.949829503334824, 2.207) [b] 
(0.9498791561126018, 2.208) [b] 
(0.9498869897519079, 2.246) [b] 
(0.9498960175296857, 2.257) [b] 
(0.9499006042939322, 2.266) [b] 
(0.9499010438373696, 2.268) [b] 
(0.949901483380807, 2.27) [b] 
(0.9499037873923677, 2.274) [b] 
(0.9499153074501707, 2.285) [b] 
(0.949915600479129, 2.289) [b] 
(0.9499437254791291, 2.29) [b] 
(0.9499464902930018, 2.298) [b] 
(0.9499515591184352, 2.321) [b] 
(0.9499517056329143, 2.325) [b] 
(0.9499519986618726, 2.33) [b] 
(0.949952291690831, 2.331) [b] 
(0.9499524382053101, 2.333) [b] 
(0.9499534638066641, 2.335) [b] 
(0.9499539033501015, 2.337) [b] 
(0.9499540498645807, 2.34) [b] 
(0.9499541963790598, 2.343) [b] 
(0.9499544894080181, 2.345) [b] 
(0.9499546359224972, 2.346) [b] 
(0.9499560183294335, 2.365) [b] 
(0.9499574007363699, 2.367) [b] 
(0.9499578402798073, 2.386) [b] 
(0.9499581333087657, 2.392) [b] 
(0.949966466642099, 2.395) [b] 
(0.9499667596710574, 2.399) [b] 
(0.9499688959092641, 2.405) [b] 
(0.9499691889382225, 2.407) [b] 
(0.9499719537520952, 2.419) [b] 
(0.9499733361590316, 2.434) [b] 
(0.9499742365047811, 2.437) [b] 
(0.9499745295337394, 2.439) [b] 
(0.9499746760482185, 2.441) [b] 
(0.9499748225626976, 2.443) [b] 
(0.9499757441673219, 2.463) [b] 
(0.9499761837107593, 2.47) [b] 
(0.9499771053153836, 2.472) [b] 
(0.9499775661176957, 2.482) [b] 
(0.94997848772232, 2.485) [b] 
(0.9499794093269442, 2.488) [b] 
(0.9499807917338806, 2.495) [b] 
(0.949982174140817, 2.497) [b] 
(0.9499826349431291, 2.5) [b] 
(0.9499830957454412, 2.506) [b] 
(0.9499838283178369, 2.522) [b] 
(0.9500188977622813, 2.61) [b] 
(0.9500222759580546, 2.646) [b] 
(0.9500256541538278, 2.649) [b] 
(0.9500517774558132, 2.662) [b] 
(0.9500585338473596, 2.665) [b] 
(0.9500619120431328, 2.667) [b] 
(0.9500684428686292, 2.669) [b] 
(0.9500726095352959, 2.768) [b] 
(0.950074345646407, 2.775) [b] 
(0.9500757208293064, 2.782) [b] 
(0.950076408420756, 2.783) [b] 
(0.9500771028652004, 2.863) [b] 
(0.9500777904566501, 2.868) [b] 
(0.9500784780480998, 2.892) [b] 
(0.9500919908311926, 2.956) [b] 
(0.9500953690269658, 2.957) [b] 
(0.950098747222739, 2.958) [b] 
(0.9501055036142854, 2.967) [b] 
(0.9502225448127872, 3.035) [b] 
(0.9503667759364813, 3.283) [b] 
(0.9504876656850306, 3.304) [b] 
(0.9504890545739195, 3.416) [b] 
(0.9504894017961417, 3.647) [b] 
(0.9504939156850306, 3.662) [b] 
(0.9504972938808038, 3.76) [b] 
(0.950524420616915, 3.837) [b] 
(0.9507662001140137, 4.18) [b] 
(0.9507744945556319, 4.427) [b] 
(0.9507763377648804, 4.43) [b] 
(0.9507767985671924, 4.433) [b] 
(0.9507781809741288, 4.436) [b] 
(0.9507846322064986, 4.444) [b] 
(0.9507897010319319, 4.452) [b] 
(0.950833451031932, 4.535) [b] 
(0.9508468142989834, 4.575) [b] 
(0.9508481967059198, 4.578) [b] 
(0.9508606383683471, 4.688) [b] 
(0.9508772272515835, 4.781) [b] 
(0.9508809136700805, 4.787) [b] 
(0.9509080404061917, 4.804) [b] 
(0.9509351671423029, 4.808) [b] 
(0.9509457655954816, 4.94) [b] 
(0.9509462263977937, 4.949) [b] 
(0.9509480696070421, 4.955) [b] 
(0.9509485304093542, 4.957) [b] 
(0.9509489912116663, 4.96) [b] 
(0.9509507273227774, 5.004) [b] 
(0.9509562569505229, 5.115) [b] 
(0.9509702955848446, 5.378) [b] 
(0.950972138794093, 5.436) [b] 
(0.9509730603987173, 5.439) [b] 
(0.9510853694732906, 5.74) [b] 
(0.9511274853762556, 5.741) [b] 
(0.9511555626448989, 5.745) [b] 
(0.9512257558165071, 5.847) [b] 
(0.9512397944508288, 5.848) [b] 
(0.9512538330851505, 5.857) [b] 
(0.9512819103537938, 6.125) [b] 
(0.9512959489881155, 6.14) [b] 
(0.9512966434325599, 6.179) [b] 
(0.9512976850992266, 6.195) [b] 
(0.9514226850992267, 6.452) [b] 
(0.9514231459015388, 6.916) [b] 
(0.9514248820126499, 7.006) [b] 
(0.9514505388270997, 7.053) [b] 
(0.9514761956415496, 7.213) [b] 
(0.9515018524559995, 7.367) [b] 
(0.9515275092704494, 7.564) [b] 
(0.9515531660848993, 7.601) [b] 
(0.9515577741080204, 7.879) [b] 
(0.9515633037357659, 7.883) [b] 
(0.951563764538078, 7.888) [b] 
(0.9515642253403901, 7.895) [b] 
(0.95158988215484, 7.9) [b] 
(0.9515912645617763, 7.908) [b] 
(0.9515958725848975, 7.913) [b] 
(0.9515981765964582, 7.917) [b] 
(0.9515990982010825, 7.921) [b] 
(0.9516000198057067, 7.929) [b] 
(0.9516617673155311, 7.958) [b] 
(0.9516705225594614, 7.983) [b] 
(0.9516737481756462, 7.988) [b] 
(0.9516746697802705, 7.993) [b] 
(0.9516751305825826, 7.995) [b] 
(0.9516765129895189, 8.002) [b] 
(0.9516774345941432, 8.009) [b] 
(0.9516783561987675, 8.012) [b] 
(0.9516788170010796, 8.016) [b] 
(0.9516792778033917, 8.018) [b] 
(0.951680199408016, 8.031) [b] 
(0.9516811210126402, 8.069) [b] 
(0.9516857290357614, 8.083) [b] 
(0.9516884938496342, 8.086) [b] 
(0.9516894154542584, 8.09) [b] 
(0.9516917194658191, 8.093) [b] 
(0.9516935626750676, 8.096) [b] 
(0.9516940234773796, 8.099) [b] 
(0.9517009355120615, 8.114) [b] 
(0.9517018571166858, 8.119) [b] 
(0.9517050827328706, 8.123) [b] 
(0.9517073867444312, 8.131) [b] 
(0.9517078475467433, 8.137) [b] 
(0.9517083083490554, 8.14) [b] 
(0.9517087691513675, 8.142) [b] 
(0.9517092299536796, 8.161) [b] 
(0.9517101515583039, 8.165) [b] 
(0.9517119947675523, 8.168) [b] 
(0.9517124555698644, 8.172) [b] 
(0.9517152203837371, 8.176) [b] 
(0.9517156811860492, 8.179) [b] 
(0.9517161419883613, 8.189) [b] 
(0.9517166027906734, 8.192) [b] 
(0.951717290382123, 8.953) [b] 
(0.951719353156472, 8.973) [b] 
(0.9517245614898053, 9.161) [b] 
(0.9517259503786942, 9.19) [b] 
(0.951728728156472, 9.197) [b] 
(0.9517294226009164, 9.456) [b] 
(0.9517301101923661, 9.579) [b] 
(0.9517572369284772, 10.491) [b] 
(0.9517843636645884, 10.497) [b] 
(0.9518114904006996, 10.631) [b] 
(0.951919997345144, 10.632) [b] 
(0.9519471240812551, 10.633) [b] 
(0.9519742508173663, 10.634) [b] 
(0.9520013775534775, 10.636) [b] 
(0.9520285042895886, 10.652) [b] 
(0.952109884497922, 10.653) [b] 
(0.9521370112340332, 10.654) [b] 
(0.9522455181784776, 10.656) [b] 
(0.9522726449145887, 10.661) [b] 
(0.9522997716506999, 10.662) [b] 
(0.9523268983868111, 10.663) [b] 
(0.9523279400534778, 12.023) [b] 
(0.9523514274642256, 12.126) [b] 
(0.9523749148749735, 12.162) [b] 
(0.9523756024664232, 12.192) [b] 
(0.9523762900578728, 12.201) [b] 
(0.9523769776493225, 13.036) [b] 
(0.9523776720937669, 13.336) [b] 
(0.9523839220937669, 13.598) [b] 
(0.9523904406586396, 13.836) [b] 
(0.9523911282500893, 14.889) [b] 
(0.9523914754723115, 15.362) [b] 
(0.9523998088056449, 15.727) [b] 
(0.9524005032500893, 15.774) [b] 
(0.952419009115025, 17.814) [b] 
(0.9524197035594694, 17.858) [b] 
(0.952484889208197, 18.253) [b] 
(0.9524914077730697, 18.254) [b] 
(0.952493491106403, 20.385) [b] 
(0.9525000219318993, 20.803) [b] 
(0.9525620259001533, 20.852) [b] 
(0.9525703592334867, 24.194) [b] 
(0.9525776509001533, 24.241) [b] 
(0.9525783453445977, 24.507) [b] 
(0.9526018327553456, 28.243) [b] 
(0.9526021799775678, 28.836) [b] 
(0.9526230133109012, 29.008) [b] 
(0.9526438466442345, 29.12) [b] 
(0.9526502628232716, 30.077) [b] 
(0.9526506100454938, 31.437) [b] 
(0.9526634424035679, 31.731) [b] 
(0.952669858582605, 31.735) [b] 
(0.9526762747616421, 31.828) [b] 
(0.9526826909406791, 32.453) [b] 
(0.9526891071197162, 32.638) [b] 
(0.9526932737863829, 37.045) [b] 
(0.9526961208425269, 37.062) [b] 
(0.9526964680647491, 37.093) [b] 
(0.9527037597314157, 38.366) [b] 
(0.9527054958425268, 38.415) [b] 
(0.9527061902869712, 38.525) [b] 
(0.9527082736203045, 38.989) [b] 
(0.9527148044458009, 40.429) [b] 
(0.952721220624838, 42.278) [b] 
(0.9527225958077373, 42.615) [b] 
(0.9527239709906367, 42.813) [b] 
(0.9527260337649857, 42.828) [b] 
(0.95272672820943, 43.119) [b] 
(0.9527274226538744, 43.404) [b] 
(0.95273297820943, 48.601) [b] 
(0.9527364504316522, 48.65) [b] 
(0.9527371448760966, 48.747) [b] 
(0.9527472794634162, 56.591) [b] 
(0.9527607922465091, 56.598) [b] 
(0.9527675486380556, 56.751) [b] 
(0.9527709268338288, 56.757) [b] 
(0.9528108071379092, 57.94) [b] 
(0.9528114947293589, 57.941) [b] 
(0.9528142450951576, 57.942) [b] 
(0.9528156202780569, 57.943) [b] 
(0.9528197458267549, 57.951) [b] 
(0.9528259341498019, 57.985) [b] 
(0.9528266217412515, 57.989) [b] 
(0.9528273093327012, 58.008) [b] 
(0.9528279969241509, 58.014) [b] 
(0.9528313751199241, 58.705) [b] 
(0.9528716717027739, 59.524) [b] 
(0.9528757972514719, 59.873) [b] 
(0.9528764848429215, 60.291) [b] 
(0.9528778600258209, 60.314) [b] 
(0.9528785476172705, 60.834) [b] 
(0.9528792352087202, 60.84) [b] 
(0.9528799228001699, 61) [b] 
(0.9528806103916195, 61.537) [b] 
(0.9528812979830692, 61.544) [b] 
(0.9528826166133815, 64.291) [b] 
(0.9528829638356037, 68.253) [b] 
(0.9528833110578259, 69.982) [b] 
(0.952885047168937, 70.048) [b] 
(0.9528874777244926, 70.093) [b] 
(0.9528878249467148, 70.56) [b] 
(0.9529113123574626, 70.848) [b] 
(0.952913656589129, 71.078) [b] 
(0.9529140038113512, 71.14) [b] 
(0.9529591427002401, 81.399) [b] 
(0.9529598371446845, 81.458) [b] 
(0.9529601843669067, 81.774) [b] 
(0.9529605315891289, 81.829) [b] 
(0.9529643510335734, 82.129) [b] 
(0.9529900078480232, 84.584) [b] 
(0.9529920911813565, 86.107) [b] 
(0.9530177479958064, 87.3) [b] 
(0.9530180952180286, 87.619) [b] 
(0.953038928551362, 87.795) [b] 
(0.9530597618846953, 87.796) [b] 
(0.9530805952180287, 87.797) [b] 
(0.953101428551362, 87.798) [b] 
(0.9531222618846954, 87.799) [b] 
(0.9531430952180288, 87.8) [b] 
(0.9531639285513621, 87.801) [b] 
(0.9531847618846955, 87.802) [b] 
(0.9533305952180288, 87.804) [b] 
(0.9533514285513621, 87.805) [b] 
(0.9533722618846955, 87.808) [b] 
(0.9533930952180288, 87.81) [b] 
(0.9534139285513622, 87.827) [b] 
(0.9534347618846956, 87.83) [b] 
(0.9534555952180289, 87.865) [b] 
(0.9537472618846955, 87.938) [b] 
(0.9537680952180289, 88.006) [b] 
(0.9537889285513622, 89.247) [b] 
(0.9538097618846956, 101.249) [b] 
(0.9538305952180289, 101.252) [b] 
(0.9538514285513623, 101.255) [b] 
(0.9538722618846956, 101.26) [b] 
(0.953893095218029, 101.457) [b] 
(0.9539347618846957, 101.474) [b] 
(0.9539361854127677, 104.536) [b] 
(0.9539570187461011, 109.807) [b] 
(0.9539778520794344, 109.812) [b] 
(0.9539986854127678, 109.823) [b] 
(0.9541861854127678, 109.825) [b] 
(0.9542486854127677, 109.827) [b] 
(0.9542695187461011, 109.842) [b] 
(0.9542903520794345, 109.856) [b] 
(0.9543111854127678, 109.949) [b] 
(0.9543528520794345, 110.269) [b] 
(0.9543736854127679, 110.274) [b] 
(0.9544067541958366, 110.435) [b] 
(0.9544398229789054, 110.436) [b] 
(0.9544606563122388, 114.117) [b] 
(0.954461003534461, 119.639) [b] 
(0.9544616979789053, 119.684) [b] 
(0.9544620452011275, 119.728) [b] 
(0.9544623924233497, 119.774) [b] 
(0.9544682952011275, 119.896) [b] 
(0.9544759340900164, 119.947) [b] 
(0.9544776702011275, 120.2) [b] 
(0.9544985035344609, 122.464) [b] 
(0.9546443368677942, 122.465) [b] 
(0.9546651702011275, 122.469) [b] 
(0.9547068368677942, 122.471) [b] 
(0.9547276702011276, 122.472) [b] 
(0.9547901702011276, 122.473) [b] 
(0.9548318368677943, 122.474) [b] 
(0.954873503534461, 122.475) [b] 
(0.9549151702011277, 122.476) [b] 
(0.9549360035344611, 122.488) [b] 
(0.9549568368677944, 122.489) [b] 
(0.9549776702011278, 122.49) [b] 
(0.9549790590900167, 124.958) [b] 
(0.9549794063122389, 125.326) [b] 
(0.9549801007566833, 125.598) [b] 
(0.9549804479789055, 125.892) [b] 
(0.9550974891774073, 132.035) [b] 
(0.9552145303759092, 146.82) [b] 
(0.9552179085716824, 150.037) [b] 
(0.9552413959824303, 165.179) [b] 
(0.9552462570935414, 165.304) [b] 
(0.9552476459824303, 165.35) [b] 
(0.9552711333931782, 165.885) [b] 
(0.9552714806154003, 166.131) [b] 
(0.9552728695042892, 169.724) [b] 
(0.9552963569150371, 173.523) [b] 
(0.955319844325785, 184.653) [b] 
(0.9553433317365329, 189.014) [b] 
(0.9553668191472807, 191.868) [b] 
(0.9553675067387304, 201.003) [b] 
(0.95536819433018, 201.023) [b] 
(0.9553817359968467, 201.492) [b] 
(0.9553824235882964, 201.513) [b] 
(0.9553862430327409, 201.567) [b] 
(0.9554018680327409, 201.67) [b] 
(0.9554025556241905, 201.729) [b] 
(0.9554140139575239, 202.038) [b] 
(0.9554153891404232, 203.764) [b] 
(0.9554160767318729, 203.945) [b] 
(0.9554167643233226, 203.947) [b] 
(0.9554174519147722, 203.952) [b] 
(0.9554314905490939, 208.854) [b] 
(0.9554328657319933, 209.868) [b] 
(0.9554342409148926, 209.93) [b] 
(0.9554359770260037, 210.351) [b] 
(0.9554373659148926, 210.443) [b] 
(0.9554514045492143, 232.341) [b] 
(0.9554935204521793, 257.666) [b] 
(0.955507559086501, 257.667) [b] 
(0.9555215977208227, 257.671) [b] 
(0.9555356363551444, 257.678) [b] 
(0.9555612931695943, 257.73) [b] 
(0.9556174477068808, 257.732) [b] 
(0.9556314863412025, 257.733) [b] 
(0.9556455249755242, 258.105) [b] 
(0.9556595636098459, 258.109) [b] 
(0.9556736022441676, 259.645) [b] 
(0.9557157181471326, 260.206) [b] 
(0.9557297567814543, 260.21) [b] 
(0.955743795415776, 260.225) [b] 
(0.9557578340500977, 260.25) [b] 
(0.9557718726844194, 260.597) [b] 
(0.9557859113187411, 266.178) [b] 
(0.955811568133191, 266.495) [b] 
(0.955814693133191, 272.281) [b] 
(0.9558160820220799, 272.338) [b] 
(0.9558171236887466, 272.646) [b] 
(0.9558174709109688, 273.2) [b] 
(0.955817818133191, 273.536) [b] 
(0.9558188597998577, 273.736) [b] 
(0.9558195542443021, 274.093) [b] 
(0.9558522083717839, 285.109) [b] 
(0.9558587391972801, 285.115) [b] 
(0.9558590864195023, 307.832) [b] 
(0.955873125053824, 316.906) [b] 
(0.9558871636881457, 317.503) [b] 
(0.9561693886470949, 325.265) [b] 
(0.9561950454615448, 325.317) [b] 
(0.9562207022759946, 325.323) [b] 
(0.9562210494982168, 325.697) [b] 
(0.956221396720439, 327.433) [b] 
(0.9562470535348889, 327.678) [b] 
(0.9562727103493388, 327.838) [b] 
(0.9562737520160055, 328.278) [b] 
(0.9562994088304554, 328.297) [b] 
(0.9563001032748998, 328.396) [b] 
(0.9565823282338489, 335.648) [b] 
(0.9566079850482988, 335.703) [b] 
(0.9566336418627487, 335.709) [b] 
(0.9566592986771986, 338.136) [b] 
(0.9566849554916484, 338.303) [b] 
(0.9567106123060983, 338.786) [b] 
(0.9567362691205482, 349.649) [b] 
(0.9567619259349981, 357.715) [b] 
(0.956787582749448, 361.01) [b] 
(0.9568132395638979, 369.815) [b] 
(0.95681358678612, 458.756) [b] 
(0.9568142812305644, 459.368) [b] 
(0.9568187951194533, 466.082) [b] 
(0.95681983678612, 466.127) [b] 
(0.9568201840083422, 466.22) [b] 
(0.9568205312305644, 467.169) [b] 
(0.9568239094263377, 482.056) [b] 
(0.9568272876221109, 482.061) [b] 
(0.9568306658178841, 482.123) [b] 
(0.9568340440136573, 482.811) [b] 
(0.9568405748391537, 497.581) [b] 
(0.9568471056646501, 497.589) [b] 
(0.9568481473313168, 509.878) [b] 
(0.9568491889979835, 509.882) [b] 
(0.9568502306646502, 509.887) [b] 
(0.9568533556646502, 509.893) [b] 
(0.9568575223313169, 509.897) [b] 
(0.9568585639979836, 509.907) [b] 
(0.9568596056646503, 509.912) [b] 
(0.956860647331317, 509.926) [b] 
(0.9568616889979837, 510.079) [b] 
(0.9568627306646504, 510.094) [b] 
(0.9568637723313171, 511.454) [b] 
(0.9568648139979838, 511.46) [b] 
(0.9568658556646505, 514.767) [b] 
(0.9568762723313172, 571.715) [b] 
(0.9568825223313172, 571.775) [b] 
(0.9568832167757616, 571.888) [b] 
(0.956883911220206, 573.924) [b] 
(0.9568846056646504, 574.038) [b] 
(0.9568849528868726, 576.43) [b] 
(0.9568884251090948, 577.386) [b] 
(0.9568905084424281, 577.434) [b] 
(0.9568908556646503, 579.484) [b] 
(0.9568995362202058, 580.096) [b] 
(0.956899682734685, 588.943) [b] 
(0.9569010716235739, 695.815) [b] 
(0.9569017660680182, 701.632) [b] 
(0.956916050939694, 704.171) [b] 
(0.9569188157535667, 720.42) [b] 
(0.9569444725680166, 804.239) [b] 
(0.9569701293824665, 831.701) [b] 
(0.9569704766046887, 876.35) [b] 
(0.9569715182713554, 887.597) [b] 
(0.9569756849380221, 913.388) [b] 
(0.9569791571602443, 926.333) [b] 
(0.9569795043824665, 926.379) [b] 
(0.9569798516046887, 926.474) [b] 
(0.9569801988269109, 927.123) [b] 
(0.9569836710491331, 953.981) [b] 
(0.9569902018746295, 992.555) [b] 
(0.9569966180536665, 1053.16) [b] 
(0.9569973124981109, 1055.72) [b] 
(0.9570005381142958, 1060.41) [b] 
(0.9570040103365179, 1066.82) [b] 
(0.9570054338645899, 1073.23) [b] 
(0.9570231421979233, 1108.86) [b] 
(0.9570300866423678, 1108.92) [b] 
(0.9570314755312567, 1108.98) [b] 
(0.9570321699757011, 1109.04) [b] 
(0.9570328644201455, 1109.23) [b] 
(0.9570339060868122, 1109.29) [b] 
(0.9570342533090344, 1118.92) [b] 
(0.9570356768371063, 1134.34) [b] 
(0.957039843503773, 1321.81) [b] 
(0.9570412323926619, 1321.82) [b] 
(0.9570440101704397, 1321.83) [b] 
(0.9570447046148841, 1322) [b] 
(0.9570453990593285, 1322.34) [b] 
(0.9570488712815507, 1322.77) [b] 
(0.957050954614884, 1322.78) [b] 
(0.9570516490593284, 1323.05) [b] 
(0.9570530379482173, 1327.78) [b] 
(0.957067076582539, 1331.69) [b] 
(0.9570691599158723, 1334.79) [b] 
(0.9570705488047612, 1334.85) [b] 
(0.9570932393226, 1359.28) [b] 
(0.9570946145054994, 1359.3) [b] 
(0.9571028656028954, 1359.33) [b] 
(0.957103553194345, 1359.44) [b] 
(0.9571049283772444, 1360.27) [b] 
(0.957105615968694, 1361.25) [b] 
(0.9571063035601437, 1361.26) [b] 
(0.957107678743043, 1364.03) [b] 
(0.9571145546575397, 1364.06) [b] 
(0.9571152422489894, 1364.07) [b] 
(0.957115929840439, 1364.32) [b] 
(0.9571299684747607, 1365.56) [b] 
(0.9571306560662104, 1366.01) [b] 
(0.9571313436576601, 1366.36) [b] 
(0.9571327188405594, 1385.81) [b] 
(0.9571334064320091, 1441.69) [b] 
(0.9571347816149084, 1442.51) [b] 
(0.9571361567978077, 1442.68) [b] 
(0.9571368443892574, 1442.72) [b] 
(0.9571375319807071, 1462.47) [b] 
(0.9571382195721567, 1467.59) [b] 
(0.9571389071636064, 1467.73) [b] 
(0.957141657529405, 1473.43) [b] 
(0.9571423451208547, 1473.98) [b] 
(0.9571430327123044, 1474.51) [b] 
(0.9571450954866534, 1499.51) [b] 
(0.9571464706695527, 1503.83) [b] 
(0.9571471582610024, 1507.17) [b] 
(0.9571485334439017, 1634.51) [b] 
(0.9571523528883462, 1688.57) [b] 
(0.9571527001105684, 1688.66) [b] 
(0.957161380666124, 1690.34) [b] 
(0.9571652001105685, 1690.41) [b] 
(0.9571693667772352, 1690.47) [b] 
(0.9571704084439019, 1690.59) [b] 
(0.9571735334439019, 1691.22) [b] 
(0.9571756167772352, 1692.36) [b] 
(0.957178394555013, 1692.61) [b] 
(0.9571794362216797, 1696.16) [b] 
(0.9571795827361588, 1773.74) [b] 
(0.9571823605139366, 1838.63) [b] 
(0.9571834021806033, 1838.69) [b] 
(0.9571840966250477, 1838.92) [b] 
(0.9571854855139366, 1839.04) [b] 
(0.9571868744028255, 1840.3) [b] 
(0.9571872216250477, 1847.43) [b] 
(0.9572107090357955, 1856.78) [b] 
(0.9572247476701172, 1864.48) [b] 
(0.9572387863044389, 1911.35) [b] 
(0.9572453048693117, 2085.82) [b] 
(0.9572454513837908, 2103.26) [b] 
(0.957245798606013, 2308.91) [b] 
(0.9572461458282352, 2324.64) [b] 
(0.9572601844625569, 2442.39) [b] 
(0.9573841923990648, 2604.76) [b] 
(0.9575082003355727, 2604.77) [b] 
(0.9575702043038267, 2605.27) [b] 
(0.9576322082720806, 2612.14) [b] 
(0.9577492494705825, 2618.8) [b] 
(0.9578662906690844, 2619.22) [b] 
(0.957867332335751, 2855.77) [b] 
(0.9579213834681223, 2933.36) [b] 
(0.9579247616638955, 2933.39) [b] 
(0.957948409034308, 2942.29) [b] 
(0.9579551654258545, 2942.32) [b] 
(0.9579585436216277, 2942.45) [b] 
(0.9579619218174009, 2942.47) [b] 
(0.9579653000131741, 2942.77) [b] 
(0.9579793386474958, 3112.9) [b] 
(0.9580074159161391, 3113.11) [b] 
(0.9580105409161391, 3212.87) [b] 
(0.9580112353605835, 3224.26) [b] 
(0.9580129714716946, 3235.74) [b] 
(0.9580133186939168, 3246.64) [b] 
(0.9580147011008532, 3246.77) [b] 
(0.9580153955452976, 3247.03) [b] 
(0.958016089989742, 3247.7) [b] 
(0.9580167844341864, 3366.83) [b] 
},{(0.9534035625000002, 0.001) [c] 
(0.9534035625000002, 4.812143979166668) [c] 
(0.9534035625000002, 3600) [c] 
}}}{legend pos=north west}}
% 	\caption{\label{fig:cactus}Accuracy over time}
% \end{figure}

\subsection{Factor analysis}

Finally, we report results of three variants of \budalg, in order to analyse the relative contributions of the factors described in Section~\ref{sec:ext}. For each variant, we report the average error (error), the ratio of optimality proofs (opt.) and the cpu time ratio with respect to the default setting on data sets for which an optimal tree has been found.

In the variant ``No heuristic'', the Gini impurity heuristic described in Section~\ref{sec:heuristic} is disabled, and replaced by simply selecting first the feature with minimum error. For shallow trees (depth 3 or 4), since in many cases the search space is completely exhausted, not computing the slightly more costly Gini impurity score may actually be a good choice and we observe run time reduction of about 15\% to 20\%. However, the accuracy of the trees decreases extremely rapidly for larger maximum depth. As a results, many less optimality proofs are obtained, and they take much longer to compute.

In the variant ``No preprocessing'', both data set and feature preprocessings described in Section~\ref{sec:preprocessing} are disabled. The feature ordering is impacted by the removal of inconsistent datapoints, and therefore it may happen that, by luck, a more acurate tree is found for the non-preprocessed data set than for the preprocessed one. However, in most cases, the preprocessing does pay of, yielding more optimality proofs, better accuracy, and shorter runtimes (on that last aspect, we estimate that most of the gain is due to the removal of redundant features, and of inconsistent datapoints, whereas the fusion of duplicated datapoints accounts for only a few percent speed-up).

In the variant ``No lower bound'', the lower bound technique described in Section~\ref{sec:lb} is disabled. In this case we observe a small increase in computational time (from 1\% up to 40\%). However, the search space is explored in the same order, and it only slightly negatively affect accuracy and the number of optimality proofs.

\begin{table}[htbp]
\begin{center}
\begin{footnotesize}
\tabcolsep=2pt
\begin{tabular}{lrrrrrrrrrrrr}
\toprule
\multirow{2}{*}{$\mdepth$}&  \multicolumn{3}{c}{\budalg} & \multicolumn{3}{c}{\noheuristic} & \multicolumn{3}{c}{\nopreprocessing} & \multicolumn{3}{c}{\nolb}\\
\cmidrule(rr){2-4}\cmidrule(rr){5-7}\cmidrule(rr){8-10}\cmidrule(rr){11-13}
& \multicolumn{1}{c}{error} & \multicolumn{1}{c}{opt.} & \multicolumn{1}{c}{cpu} & \multicolumn{1}{c}{error} & \multicolumn{1}{c}{opt.} & \multicolumn{1}{c}{cpu$^*$} & \multicolumn{1}{c}{error} & \multicolumn{1}{c}{opt.} & \multicolumn{1}{c}{cpu$^*$} & \multicolumn{1}{c}{error} & \multicolumn{1}{c}{opt.} & \multicolumn{1}{c}{cpu$^*$} \\
\midrule

\texttt{3} & 793 & 0.93 & 67 & 817 & 0.93 & -1.6 & 793 & 0.93 & $\mathsmaller{+}$4.3 & 793 & 0.93 & -2.8\\
\texttt{4} & 696 & 0.78 & 388 & 789 & 0.79 & -61 & 696 & 0.72 & $\mathsmaller{+}$47 & 696 & 0.78 & $\mathsmaller{+}$13\\
\texttt{5} & 624 & 0.64 & 479 & 752 & 0.67 & -30 & 625 & 0.55 & $\mathsmaller{+}$276 & 624 & 0.64 & $\mathsmaller{+}$39\\
\texttt{7} & 508 & 0.48 & 1045 & 689 & 0.48 & $\mathsmaller{+}$17 & 509 & 0.43 & $\mathsmaller{+}$163 & 509 & 0.48 & $\mathsmaller{+}$30\\
\texttt{10} & 410 & 0.62 & 570 & 657 & 0.53 & $\mathsmaller{+}$66 & 410 & 0.43 & $\mathsmaller{+}$63 & 410 & 0.60 & $\mathsmaller{+}$8.0\\
\bottomrule
\end{tabular}

\end{footnotesize}
\end{center}
\caption{\label{tab:factor} Factor analysis}
\end{table}

\begin{table}[htbp]
\begin{center}
\begin{footnotesize}
\tabcolsep=5pt
\begin{tabular}{lcrrrrrrrr}
\toprule
\multirow{2}{*}{}& & \multicolumn{3}{c}{\iti} & \multicolumn{5}{c}{\bfsh}\\
\cmidrule(rr){3-5}\cmidrule(rr){6-10}
&\multirow{1}{*}{\#} &  \multicolumn{1}{c}{error} & \multicolumn{1}{c}{size} & \multicolumn{1}{c}{depth} & \multicolumn{1}{c}{error} & \multicolumn{1}{c}{init e.} & \multicolumn{1}{c}{size} & \multicolumn{1}{c}{init s.} & \multicolumn{1}{c}{depth} \\
\midrule

\texttt{$\mdepth \in [0,5]$} & \multicolumn{1}{r}{9}  & 7.3 & \textbf{13.4} & 3.8 & \textbf{6.8} & 1.9 & 14.5 & 22.9 & 3.8\\
\texttt{$\mdepth \in [6,10]$} & \multicolumn{1}{r}{13}  & 38.1 & \textbf{41.0} & 7.5 & \textbf{32.2} & 2.5 & 45.2 & 104.9 & \textbf{7.2}\\
\texttt{$\mdepth \in [11,15]$} & \multicolumn{1}{r}{13}  & 93.1 & 129.8 & 13.1 & \textbf{89.8} & 20.1 & \textbf{110.8} & 198.8 & \textbf{12.5}\\
\texttt{$\mdepth \in [16,20]$} & \multicolumn{1}{r}{11}  & 1101.5 & 1036.6 & 17.7 & \textbf{907.3} & 406.0 & \textbf{993.5} & 1840.4 & 17.7\\
\bottomrule
\end{tabular}

\end{footnotesize}
\end{center}
\caption{\label{tab:iti} ITI}
\end{table}


\begin{table}[htbp]
\begin{center}
\begin{footnotesize}
\tabcolsep=5pt
\begin{tabular}{lrrrrrr}
\toprule
\multirow{2}{*}{}&  \multicolumn{2}{c}{\iti} & \multicolumn{4}{c}{\bfss}\\
\cmidrule(rr){2-3}\cmidrule(rr){4-7}
& \multicolumn{1}{c}{error} & \multicolumn{1}{c}{size} & \multicolumn{1}{c}{error} & \multicolumn{1}{c}{init e.} & \multicolumn{1}{c}{size} & \multicolumn{1}{c}{init s.} \\
\midrule

\texttt{HTRU\_2} & 329 & 195 & \textbf{306} & 235 & \textbf{163} & 609\\
\texttt{IndiansDiabetes} & 116 & \textbf{99} & \textbf{95} & 8 & 133 & 417\\
\texttt{Statlog\_satellite} & 62 & 101 & \textbf{57} & 8 & \textbf{77} & 181\\
\texttt{Statlog\_shuttle} & 0 & \textbf{17} & 0 & 0 & 21 & 21\\
\texttt{adult\_discretized} & 3801 & 1693 & \textbf{3631} & 3300 & \textbf{1041} & 2535\\
\texttt{anneal} & 66 & 75 & \textbf{62} & 34 & \textbf{61} & 157\\
\texttt{audiology} & 4 & 13 & \textbf{3} & 0 & 13 & 21\\
\texttt{australian-credit} & 57 & \textbf{37} & \textbf{47} & 2 & 43 & 161\\
\texttt{balance-scale} & 49 & \textbf{21} & \textbf{48} & 48 & 27 & 27\\
\texttt{bank} & 2931 & \textbf{1201} & \textbf{1921} & 810 & 2341 & 5089\\
\texttt{bank\_conv} & 276 & 223 & \textbf{233} & 90 & 223 & 575\\
\texttt{banknote} & 6 & \textbf{41} & \textbf{4} & 2 & 43 & 49\\
\texttt{biodeg} & 84 & \textbf{85} & \textbf{75} & 25 & 95 & 215\\
\texttt{breast-cancer} & 23 & 21 & \textbf{21} & 5 & 21 & 57\\
\texttt{breast-wisconsin} & 13 & \textbf{21} & 13 & 1 & 23 & 53\\
\texttt{car} & 10 & \textbf{69} & 10 & 0 & 71 & 93\\
\texttt{car\_evaluation} & 80 & 51 & 80 & 80 & \textbf{21} & 21\\
\texttt{chess} & 0 & 3 & 0 & 0 & 3 & 3\\
\texttt{compas\_discretized} & 1863 & 489 & \textbf{1858} & 1828 & \textbf{167} & 285\\
\texttt{default\_credit} & 3723 & \textbf{2353} & \textbf{2798} & 1609 & 3583 & 7723\\
\texttt{diabetes} & 106 & \textbf{73} & \textbf{90} & 13 & 95 & 271\\
\texttt{forest-fires} & 134 & 75 & \textbf{130} & 112 & \textbf{67} & 103\\
\texttt{german-credit} & 146 & \textbf{105} & \textbf{115} & 14 & 113 & 341\\
\texttt{hand\_posture} & 226 & 523 & \textbf{208} & 156 & \textbf{189} & 385\\
\texttt{heart-cleveland} & 39 & \textbf{19} & \textbf{37} & 7 & 25 & 63\\
\texttt{hepatitis} & 14 & \textbf{13} & \textbf{13} & 3 & 17 & 29\\
\texttt{hypothyroid} & 49 & \textbf{33} & \textbf{45} & 22 & 43 & 109\\
\texttt{ionosphere} & 22 & 17 & \textbf{21} & 4 & \textbf{15} & 43\\
\texttt{iris} & 1 & 7 & 1 & 1 & \textbf{5} & 5\\
\texttt{kr-vs-kp} & 7 & \textbf{77} & \textbf{6} & 0 & 87 & 111\\
\texttt{letter} & 69 & 165 & \textbf{57} & 0 & \textbf{161} & 363\\
\texttt{lymph} & 11 & 17 & \textbf{10} & 3 & 17 & 31\\
\texttt{magic04} & 1786 & \textbf{1415} & \textbf{1484} & 770 & 1741 & 4077\\
\texttt{messidor} & 212 & \textbf{171} & \textbf{184} & 39 & 189 & 555\\
\texttt{mnist\_0} & 332 & \textbf{499} & \textbf{268} & 30 & 549 & 1157\\
\texttt{monk1} & 9 & 13 & \textbf{5} & 0 & 13 & 15\\
\texttt{monk2} & 24 & 41 & \textbf{23} & 0 & \textbf{29} & 67\\
\texttt{monk3} & 5 & \textbf{13} & \textbf{4} & 4 & 21 & 23\\
\texttt{mushroom} & 0 & 21 & 0 & 0 & \textbf{17} & 17\\
\texttt{pendigits} & 20 & 43 & 20 & 0 & \textbf{23} & 45\\
\texttt{primary-tumor} & 39 & \textbf{31} & \textbf{37} & 16 & 37 & 113\\
\texttt{segment} & 3 & 11 & \textbf{2} & 0 & \textbf{9} & 11\\
\texttt{seismic\_bumps} & 143 & \textbf{73} & \textbf{125} & 71 & 77 & 289\\
\texttt{soybean} & 21 & \textbf{35} & \textbf{20} & 3 & 51 & 85\\
\texttt{spambase} & 338 & 307 & \textbf{299} & 178 & \textbf{285} & 595\\
\texttt{splice-1} & 71 & 63 & \textbf{65} & 9 & 63 & 197\\
\texttt{surgical-deepnet} & 962 & \textbf{455} & \textbf{124} & 124 & 2281 & 2281\\
\texttt{taiwan\_binarised} & 3845 & \textbf{2379} & \textbf{3108} & 2022 & 3281 & 7243\\
\texttt{tic-tac-toe} & 29 & 75 & \textbf{24} & 0 & \textbf{59} & 97\\
\texttt{titanic} & 107 & 65 & \textbf{106} & 75 & \textbf{43} & 151\\
\texttt{vehicle} & 25 & \textbf{25} & \textbf{21} & 0 & 33 & 73\\
\texttt{vote} & 12 & \textbf{13} & 12 & 5 & 19 & 29\\
\texttt{weather-aus} & \textbf{1160} & 643 & 1371 & 1371 & \textbf{511} & 511\\
\texttt{wine1} & 31 & 27 & \textbf{29} & 17 & \textbf{19} & 35\\
\texttt{wine2} & 24 & 31 & \textbf{22} & 14 & 31 & 47\\
\texttt{wine3} & 26 & 23 & \textbf{25} & 15 & \textbf{19} & 35\\
\texttt{winequality-red} & 10 & \textbf{5} & \textbf{9} & 9 & 7 & 7\\
\texttt{yeast} & 232 & \textbf{177} & \textbf{175} & 13 & 271 & 705\\
\bottomrule
\end{tabular}

\end{footnotesize}
\end{center}
\caption{\label{tab:iti} ITI}
\end{table}



\begin{table}[htbp]
\begin{center}
\begin{footnotesize}
\tabcolsep=5pt
\begin{tabular}{lrrr}
\toprule
\multirow{2}{*}{}&  \multicolumn{3}{c}{\gosdtmed}\\
\cmidrule(rr){2-4}
& \multicolumn{1}{c}{error} & \multicolumn{1}{c}{size} & \multicolumn{1}{c}{depth} \\
\midrule

\texttt{pendigits} & - & - & -\\
\texttt{vote} & - & - & -\\
\texttt{zoo-1} & - & - & -\\
\texttt{audiology} & 11.0 & 5.0 & 2.0\\
\texttt{german-credit} & 275.0 & 9.0 & 3.0\\
\texttt{weather-aus-un} & - & - & -\\
\texttt{adult\_discretized} & - & - & -\\
\texttt{soybean} & 92.0 & 1.0 & 0.0\\
\texttt{mnist\_4} & - & - & -\\
\texttt{ionosphere} & - & - & -\\
\texttt{hypothyroid} & 70.0 & 5.0 & 2.0\\
\texttt{heart-cleveland} & 60.0 & 7.0 & 2.0\\
\texttt{taiwan\_binarised} & - & - & -\\
\texttt{breast-cancer-un} & 35.0 & 7.0 & 2.0\\
\texttt{wine2-un} & - & - & -\\
\texttt{hepatitis} & 17.0 & 5.0 & 2.0\\
\texttt{tic-tac-toe} & - & - & -\\
\texttt{vehicle} & - & - & -\\
\texttt{anneal} & 119.0 & 13.0 & 5.0\\
\texttt{titanic-un} & - & - & -\\
\texttt{mushroom} & - & - & -\\
\texttt{yeast} & 440.0 & 5.0 & 2.0\\
\texttt{car-un} & 178.0 & 9.0 & 4.0\\
\texttt{letter} & - & - & -\\
\texttt{lymph} & 22.0 & 7.0 & 2.0\\
\texttt{surgical-deepnet-un} & - & - & -\\
\texttt{segment} & 5.0 & 7.0 & 3.0\\
\texttt{wine3-un} & - & - & -\\
\texttt{diabetes} & - & - & -\\
\texttt{compas\_discretized} & 2045.0 & 9.0 & 4.0\\
\texttt{primary-tumor} & - & - & -\\
\texttt{kr-vs-kp} & 494.0 & 7.0 & 2.0\\
\texttt{forest-fires-un} & - & - & -\\
\texttt{splice-1} & - & - & -\\
\texttt{bank-un} & - & - & -\\
\texttt{wine1-un} & - & - & -\\
\texttt{australian-credit} & - & - & -\\
\texttt{mnist\_9} & - & - & -\\
\texttt{mnist\_8} & - & - & -\\
\texttt{mnist\_5} & - & - & -\\
\texttt{breast-wisconsin} & 31.0 & 5.0 & 2.0\\
\texttt{mnist\_7} & - & - & -\\
\texttt{mnist\_6} & - & - & -\\
\texttt{mnist\_1} & - & - & -\\
\texttt{mnist\_0} & - & - & -\\
\texttt{mnist\_3} & - & - & -\\
\texttt{mnist\_2} & - & - & -\\
\bottomrule
\end{tabular}

\end{footnotesize}
\end{center}
\caption{\label{tab:gosdt} GOSDT}
\end{table}



\section{Conclusion}

We have introduced a simple, exact, iterative, memory-efficient and anytime algorithm for computing optimaly-accurate tree classifiers of bounded depth.
This algorithm is considerably more efficient than state-of-the-art exact algorithms. Moreover, it has no significant time nor memory overhead with respect to greedy heuristic methods.


\bibliographystyle{plain}
\bibliography{src/references}


% \end{document}

\clearpage

% \newgeometry{bottom=2cm,top=2cm,margin=1cm}

\section*{Appendix}

The benchmark of classification data set we used is described in Table~\ref{tab:info}. It consists of 30 data sets 
commonly used in related work articles, to which we added some large data sets from Kaggle: \texttt{bank}, \texttt{titanic}, \texttt{surgical-deepnet} and \texttt{weather-aus}, as well as the \texttt{mnist} data sets, \texttt{adult\_discretized} and \texttt{compas\_discretized}. We report the number of data points ($|\allex|$), the number of features ($|\features|$), the same parameters after preprocessing (respectively $|\allex|^*$ and $|\features|^*$), and the ``noise'' ratio, that is: $2|\posex \cap \negex|/(|\posex|+|\negex|)$.

\medskip

Then we report the raw data from our experimental comparison with the state of the art for $\mdepth=3,4,5,6,7,8,9,10$
in Tables~\ref{tab:all3},
\ref{tab:all4},
\ref{tab:all5},
\ref{tab:all6},
\ref{tab:all7},
\ref{tab:all8},
\ref{tab:all9} and \ref{tab:all10}, respectively.
For every instance, we give the classification error of the best tree found within a time limit of 1h for every method. Moreover, we give the CPU time taken by each method to prove optimality when optimality is proven (in which case we mark it by a ``$^*$''), and to find the best solution otherwise. Notice that \cp\ and \dleight\ are not anytime and hence only report a solution at the end of the time limit when optimality is not proven. In this case, we write $\geq1h$. 

\medskip

Every process was first run with a memory limit of 3.5GB. Many runs of \dleight, \cp\ and \binoct\ went well over that limit and were rerun with a limit of 50GB. Still, 138 runs of \binoct and 164 runs of \dleight (out of 460) went over the limit. As \binoct can output trees anytime, the data for these runs (up until the memory blow-out) are in the tables. For \dleight, however, this is marked as a ``-'' since there was no output.
%\binoct 138
%\dleight 164 %5+10+20+23+30+27+25+24


\renewcommand{\arraystretch}{.8}

\begin{table}[htbp]%
\begin{center}%
\begin{scriptsize}%
\tabcolsep=10pt%
\begin{tabular}{lrrrrr}
\toprule
set & $|\allex|$ & $|\features|$ & $|\allex|^*$ & $|\features|^*$ & noise \\
\midrule
\texttt{hepatitis}& 137& 68& 136& 34& 0.0000\\
\texttt{lymph}& 148& 68& 148& 47& 0.0000\\
\texttt{wine1}& 178& 1276& 178& 646& 0.0000\\
\texttt{wine2}& 178& 1276& 178& 646& 0.0000\\
\texttt{wine3}& 178& 1276& 178& 646& 0.0000\\
\texttt{audiology}& 216& 148& 186& 84& 0.0000\\
\texttt{heart-cleveland}& 296& 95& 296& 54& 0.0000\\
\texttt{primary-tumor}& 336& 31& 240& 17& 0.0893\\
\texttt{ionosphere}& 351& 445& 350& 222& 0.0000\\
\texttt{vote}& 435& 48& 342& 48& 0.0000\\
\texttt{forest-fires}& 517& 989& 504& 656& 0.0155\\
\texttt{soybean}& 630& 50& 502& 43& 0.0063\\
\texttt{australian-credit}& 653& 125& 653& 74& 0.0000\\
\texttt{breast-cancer}& 683& 89& 449& 89& 0.0000\\
\texttt{breast-wisconsin}& 683& 120& 449& 60& 0.0000\\
\texttt{diabetes}& 768& 112& 768& 56& 0.0000\\
\texttt{anneal}& 812& 93& 495& 49& 0.0837\\
\texttt{vehicle}& 846& 252& 846& 126& 0.0000\\
\texttt{titanic}& 887& 333& 803& 333& 0.0361\\
\texttt{tic-tac-toe}& 958& 27& 958& 27& 0.0000\\
\texttt{german-credit}& 1000& 112& 998& 86& 0.0000\\
\texttt{yeast}& 1484& 89& 1418& 46& 0.0067\\
\texttt{car}& 1728& 21& 1728& 21& 0.0000\\
\texttt{segment}& 2310& 235& 2027& 114& 0.0000\\
\texttt{splice-1}& 3190& 287& 3005& 255& 0.0006\\
\texttt{kr-vs-kp}& 3196& 73& 3196& 38& 0.0000\\
\texttt{hypothyroid}& 3247& 88& 2527& 44& 0.0105\\
\texttt{compas\_discretized}& 6167& 25& 4181& 20& 0.5928\\
\texttt{pendigits}& 7494& 216& 7415& 108& 0.0000\\
\texttt{mushroom}& 8124& 119& 8124& 100& 0.0000\\
\texttt{surgical-deepnet}& 14635& 6047& 11733& 6046& 0.0000\\
\texttt{letter}& 20000& 224& 18200& 112& 0.0000\\
\texttt{taiwan\_binarised}& 30000& 205& 29112& 198& 0.0253\\
\texttt{adult\_discretized}& 30299& 59& 17804& 56& 0.2149\\
\texttt{bank}& 45211& 9531& 45211& 9530& 0.0000\\
\texttt{mnist\_8}& 60000& 784& 59987& 671& 0.0000\\
\texttt{mnist\_9}& 60000& 784& 59987& 671& 0.0000\\
\texttt{mnist\_0}& 60000& 784& 59987& 671& 0.0000\\
\texttt{mnist\_6}& 60000& 784& 59987& 671& 0.0000\\
\texttt{mnist\_5}& 60000& 784& 59987& 671& 0.0000\\
\texttt{mnist\_3}& 60000& 784& 59987& 671& 0.0000\\
\texttt{mnist\_2}& 60000& 784& 59987& 671& 0.0000\\
\texttt{mnist\_4}& 60000& 784& 59987& 671& 0.0000\\
\texttt{mnist\_7}& 60000& 784& 59987& 671& 0.0000\\
\texttt{mnist\_1}& 60000& 784& 59987& 671& 0.0000\\
\texttt{weather-aus}& 142193& 4759& 142151& 4756& 0.0000\\
\bottomrule
\end{tabular}
%
\end{scriptsize}%
\end{center}%
\caption{\label{tab:info} Benchmark and preprocessing data}%
\end{table}%

\begin{table}[htbp]%
\begin{center}%
\begin{scriptsize}%
\tabcolsep=2pt%
\begin{tabular}{lccrrrrrrrrrrrr}
\toprule
\multirow{2}{*}{}& && \multicolumn{2}{c}{\budalg} & \multicolumn{2}{c}{\murtree} & \multicolumn{2}{c}{\dleight} & \multicolumn{2}{c}{\cp} & \multicolumn{2}{c}{binoct} & \multicolumn{2}{c}{\cart}\\
\cmidrule(rr){4-5}\cmidrule(rr){6-7}\cmidrule(rr){8-9}\cmidrule(rr){10-11}\cmidrule(rr){12-13}\cmidrule(rr){14-15}
&\multirow{1}{*}{$\#ex.$} & \multirow{1}{*}{\#feat.} &  \multicolumn{1}{c}{error} & \multicolumn{1}{c}{cpu} & \multicolumn{1}{c}{error} & \multicolumn{1}{c}{cpu} & \multicolumn{1}{c}{error} & \multicolumn{1}{c}{cpu} & \multicolumn{1}{c}{error} & \multicolumn{1}{c}{cpu} & \multicolumn{1}{c}{error} & \multicolumn{1}{c}{cpu} & \multicolumn{1}{c}{error} & \multicolumn{1}{c}{cpu} \\
\midrule

\texttt{adult\_discretized} & \multicolumn{1}{r}{30299} & \multicolumn{1}{r}{59}  & 5020 & 0.43$^*$ & 5020 & 0.84$^*$ & 5020 & 10$^*$ & 5020 & 6.4$^*$ & 5600 & $\mathsmaller{\geq}1$h & 5758 & 0.05\\
\texttt{anneal} & \multicolumn{1}{r}{812} & \multicolumn{1}{r}{93}  & 112 & 0.03$^*$ & 112 & 0.14$^*$ & 112 & 2.4$^*$ & 112 & 6.0$^*$ & 123 & $\mathsmaller{\geq}1$h & 149 & 0.00\\
\texttt{audiology} & \multicolumn{1}{r}{216} & \multicolumn{1}{r}{148}  & 5 & 0.06$^*$ & 5 & 0.13$^*$ & 5 & 4.5$^*$ & 5 & 9.1$^*$ & 6 & $\mathsmaller{\geq}1$h & 6 & 0.00\\
\texttt{australian-credit} & \multicolumn{1}{r}{653} & \multicolumn{1}{r}{125}  & 73 & 0.14$^*$ & 73 & 0.35$^*$ & 73 & 9.6$^*$ & 73 & 14$^*$ & 87 & $\mathsmaller{\geq}1$h & 87 & 0.00\\
\texttt{bank} & \multicolumn{1}{r}{45211} & \multicolumn{1}{r}{9531}  & 4453 & 259 & 5289 & 0.84 & 4805 & $\mathsmaller{\geq}1$h & 4453 & $\mathsmaller{\geq}1$h & - & - & 4462 & 33\\
\texttt{breast-cancer} & \multicolumn{1}{r}{683} & \multicolumn{1}{r}{89}  & 24 & 0.16$^*$ & 24 & 0.07$^*$ & 24 & 0.98$^*$ & 24 & 5.7$^*$ & 25 & $\mathsmaller{\geq}1$h & 28 & 0.00\\
\texttt{breast-wisconsin} & \multicolumn{1}{r}{683} & \multicolumn{1}{r}{120}  & 15 & 0.05$^*$ & 15 & 0.20$^*$ & 15 & 6.4$^*$ & 15 & 11$^*$ & 18 & $\mathsmaller{\geq}1$h & 26 & 0.00\\
\texttt{car} & \multicolumn{1}{r}{1728} & \multicolumn{1}{r}{21}  & 192 & 0.01$^*$ & 192 & 0.01$^*$ & 192 & 0.04$^*$ & 192 & 1.7$^*$ & 192 & $\mathsmaller{\geq}1$h & 202 & 0.00\\
\texttt{compas\_discretized} & \multicolumn{1}{r}{6167} & \multicolumn{1}{r}{25}  & 2004 & 0.00$^*$ & 2004 & 0.06$^*$ & 2004 & 0.23$^*$ & 2004 & 1.8$^*$ & 2032 & $\mathsmaller{\geq}1$h & 2072 & 0.01\\
\texttt{diabetes} & \multicolumn{1}{r}{768} & \multicolumn{1}{r}{112}  & 162 & 0.09$^*$ & 162 & 0.37$^*$ & 162 & 11$^*$ & 162 & 12$^*$ & 165 & $\mathsmaller{\geq}1$h & 177 & 0.00\\
\texttt{forest-fires} & \multicolumn{1}{r}{517} & \multicolumn{1}{r}{989}  & 193 & 20$^*$ & 193 & 9.6$^*$ & - & - & 193 & 2836$^*$ & 198 & $\mathsmaller{\geq}1$h & 198 & 0.01\\
\texttt{german-credit} & \multicolumn{1}{r}{1000} & \multicolumn{1}{r}{112}  & 236 & 0.26$^*$ & 236 & 0.38$^*$ & 236 & 7.7$^*$ & 236 & 13$^*$ & 244 & $\mathsmaller{\geq}1$h & 251 & 0.00\\
\texttt{heart-cleveland} & \multicolumn{1}{r}{296} & \multicolumn{1}{r}{95}  & 41 & 0.05$^*$ & 41 & 0.12$^*$ & 41 & 3.5$^*$ & 41 & 6.8$^*$ & 42 & $\mathsmaller{\geq}1$h & 43 & 0.00\\
\texttt{hepatitis} & \multicolumn{1}{r}{137} & \multicolumn{1}{r}{68}  & 10 & 0.00$^*$ & 10 & 0.03$^*$ & 10 & 1.2$^*$ & 10 & 3.9$^*$ & 10 & $\mathsmaller{\geq}1$h & 16 & 0.00\\
\texttt{hypothyroid} & \multicolumn{1}{r}{3247} & \multicolumn{1}{r}{88}  & 61 & 0.07$^*$ & 61 & 0.41$^*$ & 61 & 4.4$^*$ & 61 & 6.6$^*$ & 62 & $\mathsmaller{\geq}1$h & 62 & 0.01\\
\texttt{ionosphere} & \multicolumn{1}{r}{351} & \multicolumn{1}{r}{445}  & 22 & 3.8$^*$ & 22 & 12$^*$ & 22 & 410$^*$ & 22 & 460$^*$ & 27 & $\mathsmaller{\geq}1$h & 29 & 0.01\\
\texttt{kr-vs-kp} & \multicolumn{1}{r}{3196} & \multicolumn{1}{r}{73}  & 198 & 0.09$^*$ & 198 & 0.22$^*$ & 198 & 2.4$^*$ & 198 & 4.8$^*$ & 375 & $\mathsmaller{\geq}1$h & 306 & 0.01\\
\texttt{letter} & \multicolumn{1}{r}{20000} & \multicolumn{1}{r}{224}  & 369 & 10$^*$ & 369 & 34$^*$ & 369 & 443$^*$ & 369 & 158$^*$ & 813 & 1251 & 677 & 0.17\\
\texttt{lymph} & \multicolumn{1}{r}{148} & \multicolumn{1}{r}{68}  & 12 & 0.01$^*$ & 12 & 0.03$^*$ & 12 & 0.76$^*$ & 12 & 3.7$^*$ & 14 & $\mathsmaller{\geq}1$h & 17 & 0.00\\
\texttt{mnist\_0} & \multicolumn{1}{r}{60000} & \multicolumn{1}{r}{784}  & 2557 & 1994$^*$ & 2557 & 568$^*$ & 3319 & $\mathsmaller{\geq}1$h & 2557 & $\mathsmaller{\geq}1$h & - & - & 3329 & 2.5\\
\texttt{mnist\_1} & \multicolumn{1}{r}{60000} & \multicolumn{1}{r}{784}  & 3462 & 1896$^*$ & 3462 & 538$^*$ & 4552 & $\mathsmaller{\geq}1$h & 3462 & $\mathsmaller{\geq}1$h & - & - & 3534 & 2.5\\
\texttt{mnist\_2} & \multicolumn{1}{r}{60000} & \multicolumn{1}{r}{784}  & 3938 & 1946$^*$ & 3938 & 672$^*$ & 4289 & $\mathsmaller{\geq}1$h & 3938 & $\mathsmaller{\geq}1$h & - & - & 4530 & 2.6\\
\texttt{mnist\_3} & \multicolumn{1}{r}{60000} & \multicolumn{1}{r}{784}  & 4354 & 2054$^*$ & 4354 & 644$^*$ & 4974 & $\mathsmaller{\geq}1$h & 4354 & $\mathsmaller{\geq}1$h & - & - & 6131 & 2.5\\
\texttt{mnist\_4} & \multicolumn{1}{r}{60000} & \multicolumn{1}{r}{784}  & 4729 & 2070$^*$ & 4729 & 700$^*$ & 5580 & $\mathsmaller{\geq}1$h & 4729 & $\mathsmaller{\geq}1$h & - & - & 5037 & 2.6\\
\texttt{mnist\_5} & \multicolumn{1}{r}{60000} & \multicolumn{1}{r}{784}  & 3539 & 2095$^*$ & 3539 & 715$^*$ & 4379 & $\mathsmaller{\geq}1$h & 3539 & $\mathsmaller{\geq}1$h & - & - & 4032 & 2.6\\
\texttt{mnist\_6} & \multicolumn{1}{r}{60000} & \multicolumn{1}{r}{784}  & 2756 & 1916$^*$ & 2756 & 664$^*$ & 2756 & $\mathsmaller{\geq}1$h & 2756 & $\mathsmaller{\geq}1$h & - & - & 2893 & 2.6\\
\texttt{mnist\_7} & \multicolumn{1}{r}{60000} & \multicolumn{1}{r}{784}  & 3483 & 1928$^*$ & 3483 & 570$^*$ & 4546 & $\mathsmaller{\geq}1$h & 3483 & $\mathsmaller{\geq}1$h & - & - & 3788 & 2.5\\
\texttt{mnist\_8} & \multicolumn{1}{r}{60000} & \multicolumn{1}{r}{784}  & 3583 & 2061$^*$ & 3583 & 593$^*$ & 4609 & $\mathsmaller{\geq}1$h & 3583 & $\mathsmaller{\geq}1$h & - & - & 4250 & 2.6\\
\texttt{mnist\_9} & \multicolumn{1}{r}{60000} & \multicolumn{1}{r}{784}  & 4590 & 2039$^*$ & 4590 & 746$^*$ & 5253 & $\mathsmaller{\geq}1$h & 4590 & $\mathsmaller{\geq}1$h & - & - & 5355 & 2.6\\
\texttt{mushroom} & \multicolumn{1}{r}{8124} & \multicolumn{1}{r}{119}  & 8 & 0.79$^*$ & 8 & 0.53$^*$ & 8 & 6.3$^*$ & 8 & 8.4$^*$ & 180 & $\mathsmaller{\geq}1$h & 280 & 0.02\\
\texttt{pendigits} & \multicolumn{1}{r}{7494} & \multicolumn{1}{r}{216}  & 47 & 3.3$^*$ & 47 & 11$^*$ & 47 & 134$^*$ & 47 & 70$^*$ & 477 & $\mathsmaller{\geq}1$h & 51 & 0.05\\
\texttt{primary-tumor} & \multicolumn{1}{r}{336} & \multicolumn{1}{r}{31}  & 46 & 0.00$^*$ & 46 & 0.01$^*$ & 46 & 0.14$^*$ & 46 & 2.0$^*$ & 46 & $\mathsmaller{\geq}1$h & 53 & 0.00\\
\texttt{segment} & \multicolumn{1}{r}{2310} & \multicolumn{1}{r}{235}  & 0 & 0.03$^*$ & 0 & 0.13$^*$ & 0 & 2.3$^*$ & 0 & 4.1$^*$ & 4 & $\mathsmaller{\geq}1$h & 5 & 0.01\\
\texttt{soybean} & \multicolumn{1}{r}{630} & \multicolumn{1}{r}{50}  & 29 & 0.01$^*$ & 29 & 0.02$^*$ & 29 & 0.29$^*$ & 29 & 2.3$^*$ & 31 & $\mathsmaller{\geq}1$h & 47 & 0.00\\
\texttt{splice-1} & \multicolumn{1}{r}{3190} & \multicolumn{1}{r}{287}  & 224 & 9.8$^*$ & 224 & 5.3$^*$ & 224 & 114$^*$ & 224 & 173$^*$ & 453 & $\mathsmaller{\geq}1$h & 279 & 0.03\\
\texttt{surgical-deepnet} & \multicolumn{1}{r}{14635} & \multicolumn{1}{r}{6047}  & 2512 & 953 & 2512 & 3523 & - & - & 2512 & $\mathsmaller{\geq}1$h & - & - & 2924 & 5.7\\
\texttt{taiwan\_binarised} & \multicolumn{1}{r}{30000} & \multicolumn{1}{r}{205}  & 5326 & 48$^*$ & 5326 & 45$^*$ & 5326 & 526$^*$ & 5326 & 190$^*$ & 6636 & 1639 & 5346 & 0.26\\
\texttt{tic-tac-toe} & \multicolumn{1}{r}{958} & \multicolumn{1}{r}{27}  & 216 & 0.01$^*$ & 216 & 0.02$^*$ & 216 & 0.13$^*$ & 216 & 1.8$^*$ & 232 & $\mathsmaller{\geq}1$h & 236 & 0.00\\
\texttt{titanic} & \multicolumn{1}{r}{887} & \multicolumn{1}{r}{333}  & 143 & 6.7$^*$ & 143 & 11$^*$ & 143 & 167$^*$ & 143 & 173$^*$ & 150 & $\mathsmaller{\geq}1$h & 148 & 0.01\\
\texttt{vehicle} & \multicolumn{1}{r}{846} & \multicolumn{1}{r}{252}  & 26 & 0.93$^*$ & 26 & 2.2$^*$ & 26 & 64$^*$ & 26 & 66$^*$ & 42 & $\mathsmaller{\geq}1$h & 66 & 0.01\\
\texttt{vote} & \multicolumn{1}{r}{435} & \multicolumn{1}{r}{48}  & 12 & 0.02$^*$ & 12 & 0.02$^*$ & 12 & 0.34$^*$ & 12 & 2.6$^*$ & 13 & $\mathsmaller{\geq}1$h & 14 & 0.00\\
\texttt{weather-aus} & \multicolumn{1}{r}{142193} & \multicolumn{1}{r}{4759}  & 1756 & 14 & 1756 & 611 & - & - & 1756 & $\mathsmaller{\geq}1$h & - & - & 1761 & 20\\
\texttt{wine1} & \multicolumn{1}{r}{178} & \multicolumn{1}{r}{1276}  & 43 & 16$^*$ & 43 & 9.0$^*$ & - & - & 43 & $\mathsmaller{\geq}1$h & 44 & $\mathsmaller{\geq}1$h & 45 & 0.00\\
\texttt{wine2} & \multicolumn{1}{r}{178} & \multicolumn{1}{r}{1276}  & 49 & 17$^*$ & 49 & 5.8$^*$ & - & - & 49 & $\mathsmaller{\geq}1$h & 57 & $\mathsmaller{\geq}1$h & 52 & 0.00\\
\texttt{wine3} & \multicolumn{1}{r}{178} & \multicolumn{1}{r}{1276}  & 33 & 16$^*$ & 33 & 8.4$^*$ & - & - & 33 & $\mathsmaller{\geq}1$h & 35 & $\mathsmaller{\geq}1$h & 35 & 0.00\\
\texttt{yeast} & \multicolumn{1}{r}{1484} & \multicolumn{1}{r}{89}  & 403 & 0.07$^*$ & 403 & 0.34$^*$ & 403 & 6.1$^*$ & 403 & 7.7$^*$ & 434 & $\mathsmaller{\geq}1$h & 418 & 0.00\\
\bottomrule
\end{tabular}
%
\end{scriptsize}%
\end{center}%
\caption{\label{tab:all3} Comparison with state of the art: depth 3}%
\end{table}%

\begin{table}[htbp]
\begin{center}
\begin{scriptsize}
\tabcolsep=2pt
\begin{tabular}{lrrrrrrrrrrrr}
\toprule
\multirow{2}{*}{}&  \multicolumn{2}{c}{\budalg} & \multicolumn{2}{c}{\murtree} & \multicolumn{2}{c}{\dleight} & \multicolumn{2}{c}{\cp} & \multicolumn{2}{c}{binoct} & \multicolumn{2}{c}{\cart}\\
\cmidrule(rr){2-3}\cmidrule(rr){4-5}\cmidrule(rr){6-7}\cmidrule(rr){8-9}\cmidrule(rr){10-11}\cmidrule(rr){12-13}
& \multicolumn{1}{c}{error} & \multicolumn{1}{c}{cpu} & \multicolumn{1}{c}{error} & \multicolumn{1}{c}{cpu} & \multicolumn{1}{c}{error} & \multicolumn{1}{c}{cpu} & \multicolumn{1}{c}{error} & \multicolumn{1}{c}{cpu} & \multicolumn{1}{c}{error} & \multicolumn{1}{c}{cpu} & \multicolumn{1}{c}{error} & \multicolumn{1}{c}{cpu} \\
\midrule

\texttt{monk3-bin} & 4 & 0.00$^*$ & 4 & 0.01$^*$ & 4 & 0.01$^*$ & 4 & 1.0$^*$ & - & - & 5 & 0.00\\
\texttt{monk1-bin} & 2 & 0.00$^*$ & 2 & 0.00$^*$ & 2 & 0.01$^*$ & 2 & 1.5$^*$ & - & - & 11 & 0.00\\
\texttt{hepatitis} & 3 & 0.32$^*$ & 3 & 0.47$^*$ & 3 & 28$^*$ & 3 & 70$^*$ & 11 & 510 & 12 & 0.00\\
\texttt{lymph} & 3 & 0.74$^*$ & 3 & 0.46$^*$ & 3 & 14$^*$ & 3 & 64$^*$ & 7 & 2987 & 10 & 0.00\\
\texttt{iris-bin} & 1 & 0.00$^*$ & 1 & 0.00$^*$ & 1 & 0.00$^*$ & 1 & 0.92$^*$ & - & - & 1 & 0.00\\
\texttt{monk2-bin} & 31 & 0.01$^*$ & 31 & 0.01$^*$ & 31 & 0.04$^*$ & 31 & 2.1$^*$ & - & - & 50 & 0.00\\
\texttt{wine3} & 28 & 33 & 28 & 2008$^*$ & - & - & 30 & $\mathsmaller{\geq}1$h & 32 & 3388 & 32 & 0.01\\
\texttt{wine1} & 37 & 1674 & 37 & 1226$^*$ & - & - & 39 & $\mathsmaller{\geq}1$h & 45 & 3506 & 42 & 0.01\\
\texttt{wine2} & 43 & 17 & 43 & 763$^*$ & - & - & 46 & $\mathsmaller{\geq}1$h & 57 & 3232 & 47 & 0.01\\
\texttt{wine-bin} & 0 & 0.00$^*$ & 0 & 0.00$^*$ & 0 & 0.05$^*$ & 0 & 0.06$^*$ & - & - & 1 & 0.00\\
\texttt{audiology} & 1 & 4.0$^*$ & 1 & 3.9$^*$ & 1 & 128$^*$ & 1 & 773$^*$ & 2 & 2687 & 3 & 0.00\\
\texttt{heart-cleveland} & 25 & 3.1$^*$ & 25 & 3.3$^*$ & 25 & 154$^*$ & 25 & 391$^*$ & 37 & 2750 & 38 & 0.00\\
\texttt{primary-tumor} & 34 & 0.03$^*$ & 34 & 0.09$^*$ & 34 & 2.0$^*$ & 34 & 5.6$^*$ & 38 & 3132 & 44 & 0.00\\
\texttt{Ionosphere-bin} & 12 & 137$^*$ & 12 & 49$^*$ & - & - & 12 & 2264$^*$ & - & - & 30 & 0.00\\
\texttt{ionosphere} & 7 & 730$^*$ & 7 & 748$^*$ & - & - & 8 & $\mathsmaller{\geq}1$h & 24 & 751 & 27 & 0.01\\
\texttt{vote} & 5 & 1.2$^*$ & 5 & 0.36$^*$ & 5 & 7.6$^*$ & 5 & 21$^*$ & 12 & 3311 & 8 & 0.00\\
\texttt{forest-fires} & 173 & 15 & 171 & 1850$^*$ & - & - & 179 & $\mathsmaller{\geq}1$h & 196 & 3356 & 186 & 0.01\\
\texttt{balance-scale-bin} & 48 & 0.04$^*$ & 48 & 0.04$^*$ & 48 & 0.22$^*$ & 48 & 1.8$^*$ & - & - & 49 & 0.00\\
\texttt{soybean} & 14 & 0.62$^*$ & 14 & 0.24$^*$ & 14 & 5.1$^*$ & 14 & 22$^*$ & 22 & 2906 & 32 & 0.00\\
\texttt{australian-credit} & 56 & 10$^*$ & 56 & 12$^*$ & 56 & 470$^*$ & 56 & 1170$^*$ & 83 & 3258 & 74 & 0.00\\
\texttt{breast-wisconsin} & 7 & 3.1$^*$ & 7 & 4.2$^*$ & 7 & 245$^*$ & 7 & 662$^*$ & 15 & 3460 & 16 & 0.00\\
\texttt{breast-cancer} & 16 & 9.6$^*$ & 16 & 2.3$^*$ & 16 & 28$^*$ & 16 & 219$^*$ & 22 & 2746 & 21 & 0.00\\
\texttt{IndiansDiabetes-bin} & 149 & 0.90$^*$ & 149 & 1.1$^*$ & 149 & 7.3$^*$ & 149 & 16$^*$ & - & - & 166 & 0.00\\
\texttt{diabetes} & 137 & 5.7$^*$ & 137 & 15$^*$ & 137 & 550$^*$ & 137 & 1001$^*$ & 180 & 2663 & 166 & 0.00\\
\texttt{anneal} & 91 & 1.5$^*$ & 91 & 1.8$^*$ & 91 & 102$^*$ & 91 & 193$^*$ & 108 & 2954 & 135 & 0.00\\
\texttt{vehicle} & 12 & 71$^*$ & 12 & 72$^*$ & - & - & 12 & $\mathsmaller{\geq}1$h & 30 & 3410 & 28 & 0.01\\
\texttt{titanic} & 119 & 1604$^*$ & 119 & 311$^*$ & - & - & 119 & $\mathsmaller{\geq}1$h & 135 & 3501 & 134 & 0.01\\
\texttt{tic-tac-toe} & 137 & 0.38$^*$ & 137 & 0.21$^*$ & 137 & 1.8$^*$ & 137 & 7.2$^*$ & 162 & 2511 & 150 & 0.00\\
\texttt{tic-tac-toe-bin} & 137 & 0.09$^*$ & 137 & 0.09$^*$ & 137 & 0.36$^*$ & 137 & 3.7$^*$ & - & - & 150 & 0.00\\
\texttt{german-credit} & 204 & 28$^*$ & 204 & 17$^*$ & 204 & 423$^*$ & 204 & 1008$^*$ & 236 & 3306 & 231 & 0.00\\
\texttt{biodeg-bin} & 128 & 1511$^*$ & 128 & 500$^*$ & - & - & 129 & $\mathsmaller{\geq}1$h & - & - & 148 & 0.01\\
\texttt{messidor-bin} & 332 & 21$^*$ & 332 & 19$^*$ & 332 & 245$^*$ & 332 & 269$^*$ & - & - & 364 & 0.00\\
\texttt{banknote-bin} & 13 & 0.08$^*$ & 13 & 0.21$^*$ & 13 & 0.78$^*$ & 13 & 4.2$^*$ & - & - & 38 & 0.00\\
\texttt{yeast} & 366 & 3.4$^*$ & 366 & 10.0$^*$ & 366 & 257$^*$ & 366 & 386$^*$ & 438 & 888 & 394 & 0.01\\
\texttt{winequality-red-bin} & 4 & 0.62$^*$ & 4 & 0.79$^*$ & 4 & 4.3$^*$ & 4 & 12$^*$ & - & - & 8 & 0.00\\
\texttt{car\_evaluation-bin} & 130 & 0.02$^*$ & 130 & 0.08$^*$ & 130 & 0.13$^*$ & 130 & 1.3$^*$ & - & - & 130 & 0.00\\
\texttt{car} & 136 & 0.19$^*$ & 136 & 0.15$^*$ & 136 & 0.36$^*$ & 136 & 2.8$^*$ & 178 & 871 & 178 & 0.00\\
\texttt{segment} & 0 & 0.00$^*$ & 0 & 0.26$^*$ & 0 & 1.6$^*$ & 0 & 2.5$^*$ & 1 & 3501 & 1 & 0.01\\
\texttt{seismic\_bumps-bin} & 148 & 22$^*$ & 148 & 30$^*$ & 148 & 290$^*$ & 148 & 303$^*$ & - & - & 158 & 0.01\\
\texttt{splice-1} & 141 & 3241$^*$ & 141 & 949$^*$ & - & - & 141 & $\mathsmaller{\geq}1$h & 568 & 3416 & 141 & 0.03\\
\texttt{chess-bin} & 0 & 0.00$^*$ & 0 & 0.03$^*$ & 0 & 0.00$^*$ & 0 & 0.07$^*$ & - & - & 0 & 0.00\\
\texttt{kr-vs-kp} & 144 & 2.8$^*$ & 144 & 6.7$^*$ & 144 & 88$^*$ & 144 & 141$^*$ & 189 & 2850 & 189 & 0.01\\
\texttt{hypothyroid} & 53 & 2.9$^*$ & 53 & 11$^*$ & 53 & 181$^*$ & 53 & 254$^*$ & 55 & 3071 & 53 & 0.01\\
\texttt{Statlog\_satellite-bin} & 111 & 3571 & 111 & 3170 & - & - & 136 & $\mathsmaller{\geq}1$h & - & - & 204 & 0.08\\
\texttt{bank\_conv-bin} & 392 & 1963$^*$ & 392 & 759$^*$ & - & - & 392 & $\mathsmaller{\geq}1$h & - & - & 408 & 0.04\\
\texttt{spambase-bin} & 590 & 7.7 & 590 & 937$^*$ & - & - & 590 & $\mathsmaller{\geq}1$h & - & - & 624 & 0.06\\
\texttt{compas\_discretized} & 1954 & 0.07$^*$ & 1954 & 0.76$^*$ & 1954 & 3.5$^*$ & 1954 & 6.3$^*$ & 1991 & 3390 & 1997 & 0.01\\
\texttt{pendigits} & 13 & 230$^*$ & 13 & 395$^*$ & - & - & 14 & $\mathsmaller{\geq}1$h & 780 & 0.00 & 25 & 0.07\\
\texttt{mushroom} & 0 & 0.00$^*$ & 0 & 3.2$^*$ & 0 & 41$^*$ & 0 & 0.07$^*$ & 192 & 3354 & 4 & 0.02\\
\texttt{surgical-deepnet} & 2269 & 49 & 2664 & 3593 & - & - & 3690 & $\mathsmaller{\geq}1$h & - & - & 2704 & 6.2\\
\texttt{HTRU\_2-bin} & 385 & 74$^*$ & 385 & 87$^*$ & 385 & 450$^*$ & 385 & 295$^*$ & - & - & 409 & 0.05\\
\texttt{magic04-bin} & 3112 & 232$^*$ & 3112 & 202$^*$ & 3112 & 1296$^*$ & 3112 & 800$^*$ & - & - & 3350 & 0.07\\
\texttt{letter\_recognition-bin} & 105 & 405 & 105 & 2956$^*$ & 122 & $\mathsmaller{\geq}1$h & 105 & $\mathsmaller{\geq}1$h & - & - & 133 & 0.38\\
\texttt{letter} & 261 & 1185$^*$ & 261 & 1624$^*$ & 335 & $\mathsmaller{\geq}1$h & 261 & $\mathsmaller{\geq}1$h & 813 & 0.00 & 462 & 0.20\\
\texttt{taiwan\_binarised} & 5273 & 6.2 & 5273 & 3075$^*$ & 5307 & $\mathsmaller{\geq}1$h & 5273 & $\mathsmaller{\geq}1$h & 6521 & 75 & 5306 & 0.27\\
\texttt{default\_credit-bin} & 5270 & 209 & 5291 & 943 & 5306 & $\mathsmaller{\geq}1$h & 5270 & $\mathsmaller{\geq}1$h & - & - & 5306 & 0.69\\
\texttt{adult\_discretized} & 4609 & 14$^*$ & 4609 & 18$^*$ & 4609 & 271$^*$ & 4609 & 246$^*$ & 5659 & 3392 & 5022 & 0.06\\
\texttt{Statlog\_shuttle-bin} & 0 & 0.64$^*$ & 0 & 267$^*$ & 1 & $\mathsmaller{\geq}1$h & 0 & 42$^*$ & - & - & 36 & 2.4\\
\texttt{bank} & 4314 & 290 & 5287 & 0.01 & 4808 & $\mathsmaller{\geq}1$h & 5289 & $\mathsmaller{\geq}1$h & - & - & 4420 & 32\\
\texttt{mnist\_7} & 2793 & 52 & 3483 & 262 & 4546 & $\mathsmaller{\geq}1$h & 6265 & $\mathsmaller{\geq}1$h & - & - & 3218 & 3.9\\
\texttt{mnist\_5} & 3312 & 219 & 3539 & 79 & 4373 & $\mathsmaller{\geq}1$h & 5421 & $\mathsmaller{\geq}1$h & - & - & 3648 & 3.8\\
\texttt{mnist\_1} & 2332 & 2248 & 3460 & 1936 & 4551 & $\mathsmaller{\geq}1$h & 6742 & $\mathsmaller{\geq}1$h & - & - & 2501 & 3.6\\
\texttt{mnist\_2} & 3358 & 169 & 3927 & 3307 & 4289 & $\mathsmaller{\geq}1$h & 5958 & $\mathsmaller{\geq}1$h & - & - & 4326 & 3.1\\
\texttt{mnist\_3} & 3485 & 2225 & 4353 & 838 & 4900 & $\mathsmaller{\geq}1$h & 6131 & $\mathsmaller{\geq}1$h & - & - & 4367 & 4.9\\
\texttt{mnist\_9} & 3977 & 2061 & 4590 & 268 & 5252 & $\mathsmaller{\geq}1$h & 5949 & $\mathsmaller{\geq}1$h & - & - & 4231 & 3.1\\
\texttt{mnist\_8} & 3165 & 1206 & 3583 & 203 & 4609 & $\mathsmaller{\geq}1$h & 5851 & $\mathsmaller{\geq}1$h & - & - & 3987 & 4.5\\
\texttt{mnist\_4} & 3670 & 2476 & 4721 & 2368 & 5580 & $\mathsmaller{\geq}1$h & 5842 & $\mathsmaller{\geq}1$h & - & - & 4129 & 3.2\\
\texttt{mnist\_6} & 1940 & 2752 & 2720 & 1793 & 2755 & $\mathsmaller{\geq}1$h & 5918 & $\mathsmaller{\geq}1$h & - & - & 2251 & 4.1\\
\texttt{mnist\_0} & 2173 & 2158 & 2556 & 2059 & 3319 & $\mathsmaller{\geq}1$h & 5923 & $\mathsmaller{\geq}1$h & - & - & 2311 & 3.8\\
\texttt{hand\_posture-bin} & 4896 & 976 & 9839 & 1688 & 11021 & $\mathsmaller{\geq}1$h & 16265 & $\mathsmaller{\geq}1$h & - & - & 6098 & 27\\
\texttt{weather-aus} & 1749 & 2525 & 2404 & 3147 & - & - & 1752 & $\mathsmaller{\geq}1$h & - & - & 1761 & 20\\
\bottomrule
\end{tabular}

\end{scriptsize}
\end{center}
\caption{\label{tab:all4} Comparison with state of the art: depth 4}
\end{table}

\begin{table}[htbp]
\begin{center}
\begin{scriptsize}
\tabcolsep=2pt
\begin{tabular}{lccrrrrrrrrrrr}
\toprule
\multirow{2}{*}{}& && \multicolumn{3}{c}{\budalg} & \multicolumn{3}{c}{\murtree} & \multicolumn{3}{c}{\dleight} & \multicolumn{2}{c}{\cart}\\
\cmidrule(rr){4-6}\cmidrule(rr){7-9}\cmidrule(rr){10-12}\cmidrule(rr){13-14}
&\multirow{1}{*}{$\#ex.$} & \multirow{1}{*}{\#feat.} &  \multicolumn{1}{c}{error} & \multicolumn{1}{c}{cpu} & \multicolumn{1}{c}{opt.} & \multicolumn{1}{c}{error} & \multicolumn{1}{c}{cpu} & \multicolumn{1}{c}{opt.} & \multicolumn{1}{c}{error} & \multicolumn{1}{c}{cpu} & \multicolumn{1}{c}{opt.} & \multicolumn{1}{c}{error} & \multicolumn{1}{c}{cpu} \\
\midrule

\texttt{adult\_discretized} & \multicolumn{1}{r}{30299} & \multicolumn{1}{r}{59}  & 4423 & 725.1 & 1 & 4423 & 794.1 & 1 & 4442 & 3600.0 & 0 & 4728 & \textbf{0.1}\\
\texttt{anneal} & \multicolumn{1}{r}{812} & \multicolumn{1}{r}{93}  & 70 & 43.9 & 1 & 70 & 148.1 & 1 & - & - & 0 & 123 & \textbf{0.0}\\
\texttt{audiology} & \multicolumn{1}{r}{216} & \multicolumn{1}{r}{148}  & 0 & \textbf{0.0} & 1 & 0 & 0.0 & 1 & 0 & 0.0 & 1 & 2 & 0.0\\
\texttt{australian-credit} & \multicolumn{1}{r}{653} & \multicolumn{1}{r}{125}  & 39 & 657.5 & 1 & 39 & 871.9 & 1 & - & - & 0 & 64 & \textbf{0.0}\\
\texttt{bank} & \multicolumn{1}{r}{45211} & \multicolumn{1}{r}{9531}  & \textbf{4187} & 1151.9 & 0 & 4365 & 2092.6 & 0 & 4809 & 3603.0 & 0 & 4358 & \textbf{47.1}\\
\texttt{breast-cancer} & \multicolumn{1}{r}{683} & \multicolumn{1}{r}{89}  & 6 & 725.0 & 1 & 6 & 72.1 & 1 & 6 & 438.0 & 1 & 16 & \textbf{0.0}\\
\texttt{breast-wisconsin} & \multicolumn{1}{r}{683} & \multicolumn{1}{r}{120}  & 0 & 19.9 & 1 & 0 & 72.0 & 1 & - & - & 0 & 13 & \textbf{0.0}\\
\texttt{car} & \multicolumn{1}{r}{1728} & \multicolumn{1}{r}{21}  & 86 & 2.4 & 1 & 86 & 1.2 & 1 & 86 & 2.7 & 1 & 106 & \textbf{0.0}\\
\texttt{compas\_discretized} & \multicolumn{1}{r}{6167} & \multicolumn{1}{r}{25}  & 1919 & 1.1 & 1 & 1919 & 10.8 & 1 & 1919 & 26.4 & 1 & 1968 & \textbf{0.0}\\
\texttt{diabetes} & \multicolumn{1}{r}{768} & \multicolumn{1}{r}{112}  & 106 & 312.4 & 1 & 106 & 919.8 & 1 & - & - & 0 & 141 & \textbf{0.0}\\
\texttt{forest-fires} & \multicolumn{1}{r}{517} & \multicolumn{1}{r}{989}  & 156 & 777.0 & 0 & \textbf{149} & 2977.4 & 0 & - & - & 0 & 177 & \textbf{0.0}\\
\texttt{german-credit} & \multicolumn{1}{r}{1000} & \multicolumn{1}{r}{112}  & 161 & 2741.0 & 1 & 161 & 973.2 & 1 & - & - & 0 & 209 & \textbf{0.0}\\
\texttt{heart-cleveland} & \multicolumn{1}{r}{296} & \multicolumn{1}{r}{95}  & 7 & 93.5 & 1 & 7 & 100.7 & 1 & - & - & 0 & 26 & \textbf{0.0}\\
\texttt{hepatitis} & \multicolumn{1}{r}{137} & \multicolumn{1}{r}{68}  & 0 & 0.1 & 1 & 0 & 0.2 & 1 & 0 & 71.4 & 1 & 8 & \textbf{0.0}\\
\texttt{hypothyroid} & \multicolumn{1}{r}{3247} & \multicolumn{1}{r}{88}  & 44 & 87.4 & 1 & 44 & 343.4 & 1 & - & - & 0 & 50 & \textbf{0.0}\\
\texttt{ionosphere} & \multicolumn{1}{r}{351} & \multicolumn{1}{r}{445}  & 0 & 506.0 & 1 & 0 & 1340.0 & 1 & - & - & 0 & 17 & \textbf{0.0}\\
\texttt{kr-vs-kp} & \multicolumn{1}{r}{3196} & \multicolumn{1}{r}{73}  & 81 & 64.6 & 1 & 81 & 150.5 & 1 & - & - & 0 & 189 & \textbf{0.0}\\
\texttt{letter} & \multicolumn{1}{r}{20000} & \multicolumn{1}{r}{224}  & \textbf{168} & 3082.5 & 0 & 190 & 549.0 & 0 & 352 & 3600.0 & 0 & 335 & \textbf{0.3}\\
\texttt{lymph} & \multicolumn{1}{r}{148} & \multicolumn{1}{r}{68}  & 0 & \textbf{0.0} & 1 & 0 & 0.0 & 1 & 0 & 14.0 & 1 & 4 & 0.0\\
\texttt{mnist\_0} & \multicolumn{1}{r}{60000} & \multicolumn{1}{r}{784}  & \textbf{1714} & 283.8 & 0 & 2066 & 2148.9 & 0 & 3319 & 3600.2 & 0 & 2021 & \textbf{4.5}\\
\texttt{mnist\_1} & \multicolumn{1}{r}{60000} & \multicolumn{1}{r}{784}  & \textbf{1585} & 3111.0 & 0 & 1790 & 993.0 & 0 & 4029 & 3600.2 & 0 & 1965 & \textbf{3.6}\\
\texttt{mnist\_2} & \multicolumn{1}{r}{60000} & \multicolumn{1}{r}{784}  & 3118 & 3229.5 & 0 & \textbf{2963} & 2670.8 & 0 & 4026 & 3600.2 & 0 & 3676 & \textbf{3.9}\\
\texttt{mnist\_3} & \multicolumn{1}{r}{60000} & \multicolumn{1}{r}{784}  & \textbf{2893} & 1935.6 & 0 & 3184 & 398.0 & 0 & 4900 & 3600.3 & 0 & 3768 & \textbf{6.0}\\
\texttt{mnist\_4} & \multicolumn{1}{r}{60000} & \multicolumn{1}{r}{784}  & \textbf{2864} & 707.9 & 0 & 3164 & 107.1 & 0 & 5580 & 3600.2 & 0 & 3619 & \textbf{4.5}\\
\texttt{mnist\_5} & \multicolumn{1}{r}{60000} & \multicolumn{1}{r}{784}  & \textbf{3138} & 2411.4 & 0 & 3163 & 2007.3 & 0 & 4376 & 3600.2 & 0 & 3479 & \textbf{5.8}\\
\texttt{mnist\_6} & \multicolumn{1}{r}{60000} & \multicolumn{1}{r}{784}  & \textbf{1485} & 2097.4 & 0 & 1653 & 645.5 & 0 & 2753 & 3600.2 & 0 & 1900 & \textbf{4.4}\\
\texttt{mnist\_7} & \multicolumn{1}{r}{60000} & \multicolumn{1}{r}{784}  & 2532 & 1792.6 & 0 & \textbf{2464} & 2363.1 & 0 & 4542 & 3600.2 & 0 & 2848 & \textbf{6.7}\\
\texttt{mnist\_8} & \multicolumn{1}{r}{60000} & \multicolumn{1}{r}{784}  & \textbf{2547} & 2846.6 & 0 & 2818 & 1149.5 & 0 & 4609 & 3600.2 & 0 & 3172 & \textbf{6.3}\\
\texttt{mnist\_9} & \multicolumn{1}{r}{60000} & \multicolumn{1}{r}{784}  & \textbf{3352} & 1695.0 & 0 & 3521 & 1368.5 & 0 & 5252 & 3600.2 & 0 & 3830 & \textbf{6.8}\\
\texttt{mushroom} & \multicolumn{1}{r}{8124} & \multicolumn{1}{r}{119}  & 0 & \textbf{0.0} & 1 & 0 & 0.0 & 1 & 0 & 35.6 & 1 & 3 & 0.0\\
\texttt{pendigits} & \multicolumn{1}{r}{7494} & \multicolumn{1}{r}{216}  & 0 & 283.5 & 1 & 0 & 1294.7 & 1 & - & - & 0 & 11 & \textbf{0.1}\\
\texttt{primary-tumor} & \multicolumn{1}{r}{336} & \multicolumn{1}{r}{31}  & 26 & 0.4 & 1 & 26 & 1.5 & 1 & 26 & 24.0 & 1 & 35 & \textbf{0.0}\\
\texttt{segment} & \multicolumn{1}{r}{2310} & \multicolumn{1}{r}{235}  & 0 & \textbf{0.0} & 1 & 0 & 0.0 & 1 & 0 & 1.0 & 1 & 1 & 0.0\\
\texttt{soybean} & \multicolumn{1}{r}{630} & \multicolumn{1}{r}{50}  & 8 & 19.6 & 1 & 8 & 7.6 & 1 & 8 & 63.1 & 1 & 23 & \textbf{0.0}\\
\texttt{splice-1} & \multicolumn{1}{r}{3190} & \multicolumn{1}{r}{287}  & 101 & 23.8 & 0 & \textbf{100} & 3308.2 & 0 & - & - & 0 & 117 & \textbf{0.0}\\
\texttt{surgical-deepnet} & \multicolumn{1}{r}{14635} & \multicolumn{1}{r}{6047}  & \textbf{2131} & 2167.6 & 0 & 2337 & 400.4 & 0 & - & - & 0 & 2245 & \textbf{8.4}\\
\texttt{taiwan\_binarised} & \multicolumn{1}{r}{30000} & \multicolumn{1}{r}{205}  & \textbf{5200} & 104.6 & 0 & 5261 & 37.8 & 0 & 5412 & 3600.0 & 0 & 5280 & \textbf{0.4}\\
\texttt{tic-tac-toe} & \multicolumn{1}{r}{958} & \multicolumn{1}{r}{27}  & 63 & 10.2 & 1 & 63 & 2.3 & 1 & 63 & 14.0 & 1 & 78 & \textbf{0.0}\\
\texttt{titanic} & \multicolumn{1}{r}{887} & \multicolumn{1}{r}{333}  & 95 & 1427.7 & 0 & 95 & 1370.9 & 0 & - & - & 0 & 130 & \textbf{0.0}\\
\texttt{vehicle} & \multicolumn{1}{r}{846} & \multicolumn{1}{r}{252}  & 1 & 690.2 & 0 & 1 & 1540.1 & 0 & - & - & 0 & 23 & \textbf{0.0}\\
\texttt{vote} & \multicolumn{1}{r}{435} & \multicolumn{1}{r}{48}  & 1 & 23.9 & 1 & 1 & 6.1 & 1 & 1 & 45.0 & 1 & 6 & \textbf{0.0}\\
\texttt{weather-aus} & \multicolumn{1}{r}{142193} & \multicolumn{1}{r}{4759}  & 1735 & 419.4 & 0 & 1735 & 1907.5 & 0 & - & - & 0 & 1751 & \textbf{25.6}\\
\texttt{wine1} & \multicolumn{1}{r}{178} & \multicolumn{1}{r}{1276}  & 33 & 1154.5 & 0 & 33 & 287.2 & 0 & - & - & 0 & 39 & \textbf{0.0}\\
\texttt{wine2} & \multicolumn{1}{r}{178} & \multicolumn{1}{r}{1276}  & 39 & 410.5 & 0 & \textbf{37} & 3399.8 & 0 & - & - & 0 & 44 & \textbf{0.0}\\
\texttt{wine3} & \multicolumn{1}{r}{178} & \multicolumn{1}{r}{1276}  & 25 & 16.7 & 0 & 25 & 25.2 & 0 & - & - & 0 & 30 & \textbf{0.0}\\
\texttt{yeast} & \multicolumn{1}{r}{1484} & \multicolumn{1}{r}{89}  & 313 & 139.2 & 1 & 313 & 557.7 & 1 & - & - & 0 & 367 & \textbf{0.0}\\
\bottomrule
\end{tabular}

\end{scriptsize}
\end{center}
\caption{\label{tab:all5} Comparison with state of the art: depth 5}
\end{table}


\begin{table}[htbp]
\begin{center}
\begin{scriptsize}
\tabcolsep=2pt
\begin{tabular}{lccrrrrrrrrrrrr}
\toprule
\multirow{2}{*}{}& && \multicolumn{2}{c}{\budalg} & \multicolumn{2}{c}{\murtree} & \multicolumn{2}{c}{\dleight} & \multicolumn{2}{c}{\cp} & \multicolumn{2}{c}{binoct} & \multicolumn{2}{c}{\cart}\\
\cmidrule(rr){4-5}\cmidrule(rr){6-7}\cmidrule(rr){8-9}\cmidrule(rr){10-11}\cmidrule(rr){12-13}\cmidrule(rr){14-15}
&\multirow{1}{*}{$\#ex.$} & \multirow{1}{*}{\#feat.} &  \multicolumn{1}{c}{error} & \multicolumn{1}{c}{cpu} & \multicolumn{1}{c}{error} & \multicolumn{1}{c}{cpu} & \multicolumn{1}{c}{error} & \multicolumn{1}{c}{cpu} & \multicolumn{1}{c}{error} & \multicolumn{1}{c}{cpu} & \multicolumn{1}{c}{error} & \multicolumn{1}{c}{cpu} & \multicolumn{1}{c}{error} & \multicolumn{1}{c}{cpu} \\
\midrule

\texttt{adult\_discretized} & \multicolumn{1}{r}{30299} & \multicolumn{1}{r}{59}  & 4281 & 1326 & 4281 & 276 & - & - & 7511 & $\mathsmaller{\geq}1$h & - & - & 4532 & 0.08\\
\texttt{anneal} & \multicolumn{1}{r}{812} & \multicolumn{1}{r}{93}  & 51 & 1330$^*$ & 51 & 3115$^*$ & - & - & 187 & $\mathsmaller{\geq}1$h & 129 & 405 & 106 & 0.00\\
\texttt{audiology} & \multicolumn{1}{r}{216} & \multicolumn{1}{r}{148}  & 0 & 0.00$^*$ & 0 & 0.01$^*$ & 0 & 0.02$^*$ & 0 & 0.12$^*$ & 0 & $\mathsmaller{\geq}1$h$^*$ & 1 & 0.00\\
\texttt{australian-credit} & \multicolumn{1}{r}{653} & \multicolumn{1}{r}{125}  & 15 & 342 & 15 & 413 & - & - & 296 & $\mathsmaller{\geq}1$h & 286 & 476 & 56 & 0.00\\
\texttt{bank} & \multicolumn{1}{r}{45211} & \multicolumn{1}{r}{9531}  & \textbf{4046} & 339 & 4270 & 2917 & 4810 & $\mathsmaller{\geq}1$h & 5289 & $\mathsmaller{\geq}1$h & - & - & 4245 & 43\\
\texttt{breast-cancer} & \multicolumn{1}{r}{683} & \multicolumn{1}{r}{89}  & 1 & 3328 & 1 & 1451$^*$ & - & - & 3 & $\mathsmaller{\geq}1$h & 40 & 695 & 13 & 0.00\\
\texttt{breast-wisconsin} & \multicolumn{1}{r}{683} & \multicolumn{1}{r}{120}  & 0 & 5.9$^*$ & 0 & 27$^*$ & - & - & 1 & $\mathsmaller{\geq}1$h & 80 & 354 & 7 & 0.00\\
\texttt{car} & \multicolumn{1}{r}{1728} & \multicolumn{1}{r}{21}  & 36 & 27$^*$ & 36 & 5.3$^*$ & 36 & 7.9$^*$ & 36 & 222$^*$ & 444 & 696 & 90 & 0.00\\
\texttt{compas\_discretized} & \multicolumn{1}{r}{6167} & \multicolumn{1}{r}{25}  & 1887 & 17$^*$ & 1887 & 86$^*$ & 1887 & 161$^*$ & 1887 & 1049$^*$ & 2043 & 735 & 1955 & 0.01\\
\texttt{diabetes} & \multicolumn{1}{r}{768} & \multicolumn{1}{r}{112}  & \textbf{60} & 2706 & 62 & 425 & - & - & 268 & $\mathsmaller{\geq}1$h & 268 & 467 & 130 & 0.01\\
\texttt{forest-fires} & \multicolumn{1}{r}{517} & \multicolumn{1}{r}{989}  & \textbf{132} & 1934 & 137 & 2505 & - & - & 247 & $\mathsmaller{\geq}1$h & 270 & 402 & 171 & 0.02\\
\texttt{german-credit} & \multicolumn{1}{r}{1000} & \multicolumn{1}{r}{112}  & 101 & 2883 & 101 & 1720 & - & - & 300 & $\mathsmaller{\geq}1$h & 293 & 131 & 171 & 0.01\\
\texttt{heart-cleveland} & \multicolumn{1}{r}{296} & \multicolumn{1}{r}{95}  & 0 & 0.03$^*$ & 0 & 0.12$^*$ & - & - & 0 & 9.1$^*$ & 32 & 3399 & 15 & 0.00\\
\texttt{hepatitis} & \multicolumn{1}{r}{137} & \multicolumn{1}{r}{68}  & 0 & 0.00$^*$ & 0 & 0.01$^*$ & 0 & 30$^*$ & 0 & 1.8$^*$ & 7 & $\mathsmaller{\geq}1$h & 3 & 0.00\\
\texttt{hypothyroid} & \multicolumn{1}{r}{3247} & \multicolumn{1}{r}{88}  & 32 & 2391$^*$ & 32 & 1327 & - & - & 277 & $\mathsmaller{\geq}1$h & 2970 & 468 & 47 & 0.01\\
\texttt{ionosphere} & \multicolumn{1}{r}{351} & \multicolumn{1}{r}{445}  & 0 & 4.4$^*$ & 0 & 11$^*$ & - & - & 0 & 1204$^*$ & 61 & 210 & 11 & 0.01\\
\texttt{kr-vs-kp} & \multicolumn{1}{r}{3196} & \multicolumn{1}{r}{73}  & 45 & 1694$^*$ & 45 & 2782$^*$ & - & - & 76 & $\mathsmaller{\geq}1$h & 1669 & 474 & 184 & 0.01\\
\texttt{letter} & \multicolumn{1}{r}{20000} & \multicolumn{1}{r}{224}  & \textbf{118} & 2186 & 275 & 116 & 387 & $\mathsmaller{\geq}1$h & 813 & $\mathsmaller{\geq}1$h & - & - & 217 & 0.34\\
\texttt{lymph} & \multicolumn{1}{r}{148} & \multicolumn{1}{r}{68}  & 0 & 0.00$^*$ & 0 & 0.01$^*$ & 0 & 0.59$^*$ & 0 & 0.35$^*$ & 2 & $\mathsmaller{\geq}1$h & 1 & 0.00\\
\texttt{mnist\_0} & \multicolumn{1}{r}{60000} & \multicolumn{1}{r}{784}  & \textbf{1468} & 2513 & 1930 & 2810 & 3319 & $\mathsmaller{\geq}1$h & 5923 & $\mathsmaller{\geq}1$h & - & - & 1781 & 5.4\\
\texttt{mnist\_1} & \multicolumn{1}{r}{60000} & \multicolumn{1}{r}{784}  & \textbf{1167} & 1875 & 1778 & 2866 & 4551 & $\mathsmaller{\geq}1$h & 6742 & $\mathsmaller{\geq}1$h & - & - & 1542 & 5.1\\
\texttt{mnist\_2} & \multicolumn{1}{r}{60000} & \multicolumn{1}{r}{784}  & \textbf{2519} & 230 & 2687 & 1177 & 4232 & $\mathsmaller{\geq}1$h & 5958 & $\mathsmaller{\geq}1$h & - & - & 2818 & 5.6\\
\texttt{mnist\_3} & \multicolumn{1}{r}{60000} & \multicolumn{1}{r}{784}  & \textbf{2486} & 2793 & 2923 & 1740 & 4900 & $\mathsmaller{\geq}1$h & 6131 & $\mathsmaller{\geq}1$h & - & - & 2902 & 7.8\\
\texttt{mnist\_4} & \multicolumn{1}{r}{60000} & \multicolumn{1}{r}{784}  & \textbf{2180} & 3375 & 2973 & 1936 & 5580 & $\mathsmaller{\geq}1$h & 5842 & $\mathsmaller{\geq}1$h & - & - & 2543 & 4.4\\
\texttt{mnist\_5} & \multicolumn{1}{r}{60000} & \multicolumn{1}{r}{784}  & \textbf{2930} & 1759 & 3060 & 1259 & 4376 & $\mathsmaller{\geq}1$h & 5421 & $\mathsmaller{\geq}1$h & - & - & 3402 & 7.2\\
\texttt{mnist\_6} & \multicolumn{1}{r}{60000} & \multicolumn{1}{r}{784}  & \textbf{1278} & 2111 & 1474 & 2795 & 2750 & $\mathsmaller{\geq}1$h & 5918 & $\mathsmaller{\geq}1$h & - & - & 1686 & 5.5\\
\texttt{mnist\_7} & \multicolumn{1}{r}{60000} & \multicolumn{1}{r}{784}  & \textbf{2074} & 2012 & 2304 & 553 & 4543 & $\mathsmaller{\geq}1$h & 6265 & $\mathsmaller{\geq}1$h & - & - & 2163 & 5.2\\
\texttt{mnist\_8} & \multicolumn{1}{r}{60000} & \multicolumn{1}{r}{784}  & \textbf{2060} & 806 & 3228 & 84 & 4656 & $\mathsmaller{\geq}1$h & 5851 & $\mathsmaller{\geq}1$h & - & - & 2633 & 6.1\\
\texttt{mnist\_9} & \multicolumn{1}{r}{60000} & \multicolumn{1}{r}{784}  & \textbf{2879} & 2229 & 3327 & 1778 & 5252 & $\mathsmaller{\geq}1$h & 5949 & $\mathsmaller{\geq}1$h & - & - & 3366 & 6.6\\
\texttt{mushroom} & \multicolumn{1}{r}{8124} & \multicolumn{1}{r}{119}  & 0 & 0.00$^*$ & 0 & 0.02$^*$ & 0 & 32$^*$ & 0 & 0.10$^*$ & - & - & 3 & 0.03\\
\texttt{pendigits} & \multicolumn{1}{r}{7494} & \multicolumn{1}{r}{216}  & 0 & 0.01$^*$ & 0 & 0.31$^*$ & - & - & 0 & 24$^*$ & - & - & 5 & 0.07\\
\texttt{primary-tumor} & \multicolumn{1}{r}{336} & \multicolumn{1}{r}{31}  & 18 & 3.1$^*$ & 18 & 15$^*$ & 18 & 138$^*$ & 18 & 1726$^*$ & 30 & $\mathsmaller{\geq}1$h & 28 & 0.00\\
\texttt{segment} & \multicolumn{1}{r}{2310} & \multicolumn{1}{r}{235}  & 0 & 0.00$^*$ & 0 & 0.02$^*$ & 0 & 0.39$^*$ & 0 & 0.63$^*$ & 330 & 296 & 0 & 0.01\\
\texttt{soybean} & \multicolumn{1}{r}{630} & \multicolumn{1}{r}{50}  & 3 & 354$^*$ & 3 & 86$^*$ & 3 & 513$^*$ & 3 & $\mathsmaller{\geq}1$h & 13 & 3501 & 15 & 0.00\\
\texttt{splice-1} & \multicolumn{1}{r}{3190} & \multicolumn{1}{r}{287}  & \textbf{68} & $\mathsmaller{\geq}1$h & 80 & 1695 & - & - & 1535 & $\mathsmaller{\geq}1$h & 1655 & 627 & 87 & 0.04\\
\texttt{surgical-deepnet} & \multicolumn{1}{r}{14635} & \multicolumn{1}{r}{6047}  & \textbf{1767} & 2343 & 2110 & 231 & - & - & 3690 & $\mathsmaller{\geq}1$h & - & - & 1969 & 7.4\\
\texttt{taiwan\_binarised} & \multicolumn{1}{r}{30000} & \multicolumn{1}{r}{205}  & \textbf{5073} & 1473 & 5169 & 3580 & - & - & 6636 & $\mathsmaller{\geq}1$h & - & - & 5250 & 0.48\\
\texttt{tic-tac-toe} & \multicolumn{1}{r}{958} & \multicolumn{1}{r}{27}  & 12 & 126$^*$ & 12 & 17$^*$ & 12 & 47$^*$ & 12 & 1297$^*$ & 195 & 1838 & 49 & 0.00\\
\texttt{titanic} & \multicolumn{1}{r}{887} & \multicolumn{1}{r}{333}  & \textbf{78} & 1234 & 102 & 3598 & - & - & 342 & $\mathsmaller{\geq}1$h & 342 & 534 & 119 & 0.01\\
\texttt{vehicle} & \multicolumn{1}{r}{846} & \multicolumn{1}{r}{252}  & 0 & 0.08$^*$ & 0 & 0.50$^*$ & - & - & 218 & $\mathsmaller{\geq}1$h & 218 & 43 & 9 & 0.01\\
\texttt{vote} & \multicolumn{1}{r}{435} & \multicolumn{1}{r}{48}  & 0 & 0.00$^*$ & 0 & 0.01$^*$ & 0 & 0.55$^*$ & 0 & 4.0$^*$ & 7 & $\mathsmaller{\geq}1$h & 2 & 0.00\\
\texttt{weather-aus} & \multicolumn{1}{r}{142193} & \multicolumn{1}{r}{4759}  & \textbf{1713} & 418 & 1736 & 820 & - & - & 1761 & $\mathsmaller{\geq}1$h & - & - & 1734 & 22\\
\texttt{wine1} & \multicolumn{1}{r}{178} & \multicolumn{1}{r}{1276}  & 31 & 2113 & \textbf{30} & 2482 & - & - & 38 & $\mathsmaller{\geq}1$h & 59 & 341 & 36 & 0.01\\
\texttt{wine2} & \multicolumn{1}{r}{178} & \multicolumn{1}{r}{1276}  & 34 & 44 & 34 & 29 & - & - & 37 & $\mathsmaller{\geq}1$h & 71 & 305 & 41 & 0.01\\
\texttt{wine3} & \multicolumn{1}{r}{178} & \multicolumn{1}{r}{1276}  & 22 & 93 & 22 & 87 & - & - & 25 & $\mathsmaller{\geq}1$h & 48 & 283 & 27 & 0.01\\
\texttt{yeast} & \multicolumn{1}{r}{1484} & \multicolumn{1}{r}{89}  & 245 & 388 & 245 & 1668 & - & - & 463 & $\mathsmaller{\geq}1$h & 444 & 87 & 346 & 0.01\\
\bottomrule
\end{tabular}

\end{scriptsize}
\end{center}
\caption{\label{tab:all6} Comparison with state of the art: depth 6}
\end{table}

\begin{table}[htbp]
\begin{center}
\begin{scriptsize}
\tabcolsep=2pt
\begin{tabular}{lrrrrrrrrrr}
\toprule
\multirow{2}{*}{}&  \multicolumn{2}{c}{\budalg} & \multicolumn{2}{c}{\murtree} & \multicolumn{2}{c}{\dleight} & \multicolumn{2}{c}{\cp} & \multicolumn{2}{c}{\cart}\\
\cmidrule(rr){2-3}\cmidrule(rr){4-5}\cmidrule(rr){6-7}\cmidrule(rr){8-9}\cmidrule(rr){10-11}
& \multicolumn{1}{c}{error} & \multicolumn{1}{c}{cpu} & \multicolumn{1}{c}{error} & \multicolumn{1}{c}{cpu} & \multicolumn{1}{c}{error} & \multicolumn{1}{c}{cpu} & \multicolumn{1}{c}{error} & \multicolumn{1}{c}{cpu} & \multicolumn{1}{c}{error} & \multicolumn{1}{c}{cpu} \\
\midrule

\texttt{monk3-bin} & 0 & 0.00$^*$ & 0 & 0.00$^*$ & 0 & 0.00$^*$ & 0 & 0.45$^*$ & 2 & 0.00\\
\texttt{monk1-bin} & 0 & 0.00$^*$ & 0 & 0.00$^*$ & 0 & 0.00$^*$ & 0 & 0.17$^*$ & 8 & 0.00\\
\texttt{hepatitis} & 0 & 0.00$^*$ & 0 & 0.29$^*$ & 0 & 8.9$^*$ & 0 & 0.49$^*$ & 1 & 0.00\\
\texttt{lymph} & 0 & 0.00$^*$ & 0 & 0.01$^*$ & 0 & 0.01$^*$ & 0 & 0.24$^*$ & 0 & 0.00\\
\texttt{iris-bin} & 1 & 0.00$^*$ & 1 & 0.00$^*$ & 1 & 0.01$^*$ & 1 & 2.6$^*$ & 1 & 0.00\\
\texttt{monk2-bin} & 0 & 0.00$^*$ & 0 & 0.00$^*$ & 0 & 0.00$^*$ & 0 & 0.78$^*$ & 5 & 0.00\\
\texttt{wine-bin} & 0 & 0.00$^*$ & 0 & 0.00$^*$ & 0 & 0.00$^*$ & 0 & 0.26$^*$ & 0 & 0.00\\
\texttt{wine3} & 21 & 524 & 27 & 2011 & - & - & 24 & $\mathsmaller{\geq}1$h & 24 & 0.01\\
\texttt{wine2} & 31 & 28 & 43 & 680 & - & - & 35 & $\mathsmaller{\geq}1$h & 38 & 0.01\\
\texttt{wine1} & 28 & 892 & 36 & 1764 & - & - & 36 & $\mathsmaller{\geq}1$h & 33 & 0.01\\
\texttt{audiology} & 0 & 0.00$^*$ & 0 & 0.01$^*$ & 0 & 0.00$^*$ & 0 & 0.18$^*$ & 0 & 0.00\\
\texttt{heart-cleveland} & 0 & 0.00$^*$ & 0 & 42$^*$ & - & - & 0 & 3.0$^*$ & 6 & 0.01\\
\texttt{primary-tumor} & 16 & 18$^*$ & 16 & 54$^*$ & 16 & 458$^*$ & 16 & $\mathsmaller{\geq}1$h & 26 & 0.00\\
\texttt{ionosphere} & 0 & 0.07$^*$ & 0 & 13$^*$ & - & - & 0 & 566$^*$ & 7 & 0.01\\
\texttt{Ionosphere-bin} & 0 & 1.4$^*$ & 0 & 1.4$^*$ & 0 & 29$^*$ & 0 & 198$^*$ & 9 & 0.00\\
\texttt{vote} & 0 & 0.00$^*$ & 0 & 0.01$^*$ & 0 & 0.17$^*$ & 0 & 3.2$^*$ & 2 & 0.00\\
\texttt{forest-fires} & 146 & 125 & 147 & 2336 & - & - & 247 & $\mathsmaller{\geq}1$h & 161 & 0.02\\
\texttt{balance-scale-bin} & 29 & 37$^*$ & 29 & 6.9$^*$ & 29 & 10$^*$ & 29 & 228$^*$ & 49 & 0.00\\
\texttt{soybean} & 2 & 19$^*$ & 2 & 230$^*$ & - & - & 3 & $\mathsmaller{\geq}1$h & 11 & 0.00\\
\texttt{australian-credit} & 0 & 101$^*$ & 3 & 1584 & - & - & 296 & $\mathsmaller{\geq}1$h & 43 & 0.01\\
\texttt{breast-wisconsin} & 0 & 0.02$^*$ & 0 & 6.2$^*$ & - & - & 0 & 2805$^*$ & 4 & 0.00\\
\texttt{breast-cancer} & 0 & 1007$^*$ & 0 & 44$^*$ & 0 & 450$^*$ & 1 & $\mathsmaller{\geq}1$h & 8 & 0.00\\
\texttt{diabetes} & 21 & 827 & 85 & 2580 & - & - & 268 & $\mathsmaller{\geq}1$h & 100 & 0.01\\
\texttt{IndiansDiabetes-bin} & 44 & 3343 & 44 & 356$^*$ & - & - & 268 & $\mathsmaller{\geq}1$h & 113 & 0.00\\
\texttt{anneal} & 41 & 3036 & 36 & 3555$^*$ & - & - & 187 & $\mathsmaller{\geq}1$h & 96 & 0.00\\
\texttt{vehicle} & 0 & 0.09$^*$ & 0 & 17$^*$ & - & - & 0 & 1178$^*$ & 4 & 0.01\\
\texttt{titanic} & 72 & 442 & 211 & 1791 & - & - & 342 & $\mathsmaller{\geq}1$h & 111 & 0.01\\
\texttt{tic-tac-toe-bin} & 0 & 2.1$^*$ & 0 & 1.7$^*$ & 0 & 2.5$^*$ & 0 & 76$^*$ & 15 & 0.00\\
\texttt{tic-tac-toe} & 0 & 32$^*$ & 0 & 15$^*$ & 0 & 29$^*$ & 0 & 764$^*$ & 22 & 0.00\\
\texttt{german-credit} & 56 & 1192 & 171 & 363 & - & - & 300 & $\mathsmaller{\geq}1$h & 150 & 0.01\\
\texttt{biodeg-bin} & 26 & 2775 & 126 & 1288 & - & - & 356 & $\mathsmaller{\geq}1$h & 86 & 0.02\\
\texttt{messidor-bin} & 179 & 2456 & 210 & 798 & - & - & 540 & $\mathsmaller{\geq}1$h & 305 & 0.01\\
\texttt{banknote-bin} & 2 & 0.00$^*$ & 2 & 25$^*$ & 2 & 80$^*$ & 2 & $\mathsmaller{\geq}1$h & 5 & 0.00\\
\texttt{yeast} & 182 & 3558 & 260 & 3038 & - & - & 463 & $\mathsmaller{\geq}1$h & 306 & 0.02\\
\texttt{winequality-red-bin} & 2 & 0.01$^*$ & 2 & 435$^*$ & - & - & 10 & $\mathsmaller{\geq}1$h & 4 & 0.00\\
\texttt{car\_evaluation-bin} & 80 & 0.00$^*$ & 80 & 4.1$^*$ & 80 & 4.2$^*$ & 80 & 123$^*$ & 80 & 0.00\\
\texttt{car} & 11 & 231$^*$ & 11 & 19$^*$ & 11 & 16$^*$ & 11 & 1678$^*$ & 50 & 0.00\\
\texttt{segment} & 0 & 0.00$^*$ & 0 & 0.26$^*$ & 0 & 0.23$^*$ & 0 & 0.28$^*$ & 0 & 0.01\\
\texttt{seismic\_bumps-bin} & 76 & 2389 & 71 & 3541 & - & - & 170 & $\mathsmaller{\geq}1$h & 137 & 0.01\\
\texttt{splice-1} & 29 & 3484 & 786 & 3397 & - & - & 1535 & $\mathsmaller{\geq}1$h & 58 & 0.05\\
\texttt{kr-vs-kp} & 18 & 2550 & 26 & 1003 & - & - & 37 & $\mathsmaller{\geq}1$h & 103 & 0.01\\
\texttt{chess-bin} & 0 & 0.00$^*$ & 0 & 0.02$^*$ & 0 & 0.00$^*$ & 0 & 0.13$^*$ & 0 & 0.00\\
\texttt{hypothyroid} & 22 & 3478 & 30 & 3125 & - & - & 277 & $\mathsmaller{\geq}1$h & 42 & 0.01\\
\texttt{Statlog\_satellite-bin} & 14 & 2428 & 111 & 2949 & - & - & - & - & 41 & 0.12\\
\texttt{bank\_conv-bin} & 220 & 1642 & 375 & 3501 & - & - & 521 & $\mathsmaller{\geq}1$h & 303 & 0.06\\
\texttt{spambase-bin} & 352 & 3562 & 640 & 3492 & - & - & - & - & 462 & 0.08\\
\texttt{compas\_discretized} & 1852 & 198$^*$ & 1852 & 323$^*$ & 1852 & 575$^*$ & 1857 & $\mathsmaller{\geq}1$h & 1941 & 0.01\\
\texttt{pendigits} & 0 & 0.00$^*$ & 0 & 474$^*$ & - & - & 0 & 8.1$^*$ & 1 & 0.07\\
\texttt{mushroom} & 0 & 0.00$^*$ & 0 & 0.52$^*$ & 0 & 10$^*$ & 0 & 0.15$^*$ & 0 & 0.03\\
\texttt{surgical-deepnet} & 1647 & 1248 & 2658 & $\mathsmaller{\geq}1$h & - & - & 3690 & $\mathsmaller{\geq}1$h & 1871 & 9.9\\
\texttt{HTRU\_2-bin} & 297 & 3334 & 334 & 3481 & - & - & 1639 & $\mathsmaller{\geq}1$h & 352 & 0.08\\
\texttt{magic04-bin} & 2488 & 2773 & 3121 & 655 & - & - & 6688 & $\mathsmaller{\geq}1$h & 2768 & 0.11\\
\texttt{letter} & 68 & 177 & 280 & 3390 & 488 & $\mathsmaller{\geq}1$h & 813 & $\mathsmaller{\geq}1$h & 153 & 0.31\\
\texttt{letter\_recognition-bin} & 44 & 704 & 143 & 3562 & - & - & - & - & 68 & 0.58\\
\texttt{taiwan\_binarised} & 4896 & 1958 & 5382 & 2325 & 5412 & $\mathsmaller{\geq}1$h & 6636 & $\mathsmaller{\geq}1$h & 5161 & 0.58\\
\texttt{default\_credit-bin} & 4935 & 222 & 5496 & 826 & - & - & - & - & 5153 & 1.0\\
\texttt{adult\_discretized} & 4191 & 534 & 4470 & 3238 & 4998 & $\mathsmaller{\geq}1$h & 7511 & $\mathsmaller{\geq}1$h & 4481 & 0.09\\
\texttt{Statlog\_shuttle-bin} & 0 & 0.04$^*$ & 0 & 65$^*$ & - & - & 0 & 14$^*$ & 4 & 2.8\\
\texttt{bank} & 3844 & 2369 & 5287 & 0.01 & 4807 & $\mathsmaller{\geq}1$h & 5289 & $\mathsmaller{\geq}1$h & 4038 & 77\\
\texttt{mnist\_0} & 1107 & 2895 & 2556 & 1397 & - & - & 5923 & $\mathsmaller{\geq}1$h & 1323 & 8.5\\
\texttt{mnist\_1} & 810 & 510 & 3460 & 1283 & - & - & 6742 & $\mathsmaller{\geq}1$h & 1129 & 6.0\\
\texttt{mnist\_2} & 2133 & 2575 & 3927 & 2584 & - & - & 5958 & $\mathsmaller{\geq}1$h & 2502 & 5.2\\
\texttt{mnist\_3} & 1843 & 3188 & 4353 & 121 & 5172 & $\mathsmaller{\geq}1$h & 6131 & $\mathsmaller{\geq}1$h & 2274 & 4.9\\
\texttt{mnist\_4} & 1727 & 3234 & 4716 & 1937 & - & - & 5842 & $\mathsmaller{\geq}1$h & 2072 & 7.1\\
\texttt{mnist\_5} & 2830 & 2556 & 3539 & 75 & - & - & 5421 & $\mathsmaller{\geq}1$h & 3117 & 6.0\\
\texttt{mnist\_7} & 1659 & 2853 & 3483 & 245 & - & - & 6265 & $\mathsmaller{\geq}1$h & 1864 & 5.2\\
\texttt{mnist\_6} & 1208 & 2902 & 2715 & 1167 & - & - & 5918 & $\mathsmaller{\geq}1$h & 1483 & 7.8\\
\texttt{mnist\_8} & 1566 & 3230 & 3583 & 191 & - & - & 5851 & $\mathsmaller{\geq}1$h & 2101 & 5.8\\
\texttt{mnist\_9} & 2550 & 1863 & 4590 & 254 & - & - & 5949 & $\mathsmaller{\geq}1$h & 2811 & 5.4\\
\texttt{hand\_posture-bin} & 749 & 2684 & 12375 & 2337 & - & - & - & - & 962 & 78\\
\texttt{weather-aus} & 1685 & 2048 & 4072 & 2385 & - & - & 1761 & $\mathsmaller{\geq}1$h & 1721 & 27\\
\bottomrule
\end{tabular}

\end{scriptsize}
\end{center}
\caption{\label{tab:all7} Comparison with state of the art: depth 7}
\end{table}

\begin{table}[htbp]
\begin{center}
\begin{scriptsize}
\tabcolsep=2pt
\begin{tabular}{lccrrrrrrrrrrr}
\toprule
& && \multicolumn{3}{c}{\budalg} & \multicolumn{3}{c}{\murtree} & \multicolumn{3}{c}{\dleight} & \multicolumn{2}{c}{\cart}\\
\cmidrule(rr){4-6}\cmidrule(rr){7-9}\cmidrule(rr){10-12}\cmidrule(rr){13-14}
&\multirow{1}{*}{$\#ex.$} & \multirow{1}{*}{\#feat.} &  \multicolumn{1}{c}{error} & \multicolumn{1}{c}{cpu} & \multicolumn{1}{c}{opt.} & \multicolumn{1}{c}{error} & \multicolumn{1}{c}{cpu} & \multicolumn{1}{c}{opt.} & \multicolumn{1}{c}{error} & \multicolumn{1}{c}{cpu} & \multicolumn{1}{c}{opt.} & \multicolumn{1}{c}{error} & \multicolumn{1}{c}{cpu} \\
\midrule

\texttt{adult\_discretized} & \multicolumn{1}{r}{30299} & \multicolumn{1}{r}{59}  & \textbf{4148} & 449.9 & 0 & 4190 & 2840.6 & 0 & 4957 & 3600.0 & 0 & 4399 & \textbf{0.1}\\
\texttt{anneal} & \multicolumn{1}{r}{812} & \multicolumn{1}{r}{93}  & \textbf{36} & 2221.2 & 0 & 40 & 2467.5 & 0 & - & - & 0 & 88 & \textbf{0.0}\\
\texttt{audiology} & \multicolumn{1}{r}{216} & \multicolumn{1}{r}{148}  & 0 & \textbf{0.0} & 1 & 0 & 0.0 & 1 & 0 & 0.0 & 1 & 0 & 0.0\\
\texttt{australian-credit} & \multicolumn{1}{r}{653} & \multicolumn{1}{r}{125}  & 0 & 12.9 & 1 & 0 & 84.8 & 1 & - & - & 0 & 33 & \textbf{0.0}\\
\texttt{bank} & \multicolumn{1}{r}{45211} & \multicolumn{1}{r}{9531}  & \textbf{3709} & 1815.2 & 0 & 4127 & 1040.9 & 0 & 4810 & 3604.9 & 0 & 3814 & \textbf{72.9}\\
\texttt{breast-cancer} & \multicolumn{1}{r}{683} & \multicolumn{1}{r}{89}  & 0 & 25.4 & 1 & 0 & 13.1 & 1 & 0 & 13.5 & 1 & 4 & \textbf{0.0}\\
\texttt{breast-wisconsin} & \multicolumn{1}{r}{683} & \multicolumn{1}{r}{120}  & 0 & \textbf{0.0} & 1 & 0 & 0.0 & 1 & 0 & 402.0 & 1 & 0 & 0.0\\
\texttt{car} & \multicolumn{1}{r}{1728} & \multicolumn{1}{r}{21}  & 0 & 404.4 & 1 & 0 & 58.5 & 1 & 0 & 13.3 & 1 & 36 & \textbf{0.0}\\
\texttt{compas\_discretized} & \multicolumn{1}{r}{6167} & \multicolumn{1}{r}{25}  & 1832 & 1461.5 & 1 & 1832 & 2878.1 & 1 & 1832 & 1637.8 & 1 & 1904 & \textbf{0.0}\\
\texttt{diabetes} & \multicolumn{1}{r}{768} & \multicolumn{1}{r}{112}  & 0 & 219.5 & 1 & 0 & 2895.4 & 1 & - & - & 0 & 79 & \textbf{0.0}\\
\texttt{forest-fires} & \multicolumn{1}{r}{517} & \multicolumn{1}{r}{989}  & \textbf{137} & 189.5 & 0 & 145 & 140.1 & 0 & - & - & 0 & 157 & \textbf{0.0}\\
\texttt{german-credit} & \multicolumn{1}{r}{1000} & \multicolumn{1}{r}{112}  & \textbf{23} & 2235.2 & 0 & 40 & 2979.3 & 0 & - & - & 0 & 117 & \textbf{0.0}\\
\texttt{heart-cleveland} & \multicolumn{1}{r}{296} & \multicolumn{1}{r}{95}  & 0 & \textbf{0.0} & 1 & 0 & 0.0 & 1 & - & - & 0 & 2 & 0.0\\
\texttt{hepatitis} & \multicolumn{1}{r}{137} & \multicolumn{1}{r}{68}  & 0 & \textbf{0.0} & 1 & 0 & 0.0 & 1 & 0 & 0.6 & 1 & 0 & 0.0\\
\texttt{hypothyroid} & \multicolumn{1}{r}{3247} & \multicolumn{1}{r}{88}  & \textbf{17} & 99.2 & 1 & 18 & 1133.0 & 0 & - & - & 0 & 38 & \textbf{0.0}\\
\texttt{ionosphere} & \multicolumn{1}{r}{351} & \multicolumn{1}{r}{445}  & 0 & \textbf{0.0} & 1 & 0 & 0.2 & 1 & - & - & 0 & 3 & 0.0\\
\texttt{kr-vs-kp} & \multicolumn{1}{r}{3196} & \multicolumn{1}{r}{73}  & \textbf{13} & 2736.5 & 0 & 16 & 1617.7 & 0 & - & - & 0 & 48 & \textbf{0.0}\\
\texttt{letter} & \multicolumn{1}{r}{20000} & \multicolumn{1}{r}{224}  & \textbf{24} & 297.5 & 0 & 111 & 1907.9 & 0 & 601 & 3600.0 & 0 & 94 & \textbf{0.4}\\
\texttt{lymph} & \multicolumn{1}{r}{148} & \multicolumn{1}{r}{68}  & 0 & \textbf{0.0} & 1 & 0 & 0.0 & 1 & 0 & 0.0 & 1 & 0 & 0.0\\
\texttt{mnist\_0} & \multicolumn{1}{r}{60000} & \multicolumn{1}{r}{784}  & \textbf{788} & 1983.0 & 0 & 1242 & 572.0 & 0 & - & - & 0 & 991 & \textbf{7.0}\\
\texttt{mnist\_1} & \multicolumn{1}{r}{60000} & \multicolumn{1}{r}{784}  & \textbf{565} & 1177.0 & 0 & 1190 & 1964.1 & 0 & 4548 & 3600.3 & 0 & 781 & \textbf{6.5}\\
\texttt{mnist\_2} & \multicolumn{1}{r}{60000} & \multicolumn{1}{r}{784}  & \textbf{1857} & 291.0 & 0 & 2496 & 2398.4 & 0 & - & - & 0 & 2234 & \textbf{6.8}\\
\texttt{mnist\_3} & \multicolumn{1}{r}{60000} & \multicolumn{1}{r}{784}  & \textbf{1436} & 1679.9 & 0 & 2341 & 48.2 & 0 & - & - & 0 & 1692 & \textbf{5.5}\\
\texttt{mnist\_4} & \multicolumn{1}{r}{60000} & \multicolumn{1}{r}{784}  & \textbf{1279} & 1521.5 & 0 & 2228 & 2908.7 & 0 & - & - & 0 & 1662 & \textbf{6.2}\\
\texttt{mnist\_5} & \multicolumn{1}{r}{60000} & \multicolumn{1}{r}{784}  & \textbf{2519} & 1518.1 & 0 & 2610 & 1963.4 & 0 & - & - & 0 & 2726 & \textbf{7.2}\\
\texttt{mnist\_6} & \multicolumn{1}{r}{60000} & \multicolumn{1}{r}{784}  & 1203 & 109.0 & 0 & \textbf{1141} & 1113.9 & 0 & - & - & 0 & 1356 & \textbf{7.2}\\
\texttt{mnist\_7} & \multicolumn{1}{r}{60000} & \multicolumn{1}{r}{784}  & \textbf{1429} & 346.7 & 0 & 1753 & 1700.8 & 0 & 4544 & 3600.3 & 0 & 1538 & \textbf{6.7}\\
\texttt{mnist\_8} & \multicolumn{1}{r}{60000} & \multicolumn{1}{r}{784}  & \textbf{1192} & 3086.6 & 0 & 2535 & 2521.7 & 0 & - & - & 0 & 1705 & \textbf{5.3}\\
\texttt{mnist\_9} & \multicolumn{1}{r}{60000} & \multicolumn{1}{r}{784}  & \textbf{2186} & 2934.0 & 0 & 2639 & 576.5 & 0 & 5254 & 3600.3 & 0 & 2379 & \textbf{5.9}\\
\texttt{mushroom} & \multicolumn{1}{r}{8124} & \multicolumn{1}{r}{119}  & 0 & \textbf{0.0} & 1 & 0 & 0.0 & 1 & 0 & 5.4 & 1 & 0 & 0.0\\
\texttt{pendigits} & \multicolumn{1}{r}{7494} & \multicolumn{1}{r}{216}  & 0 & \textbf{0.0} & 1 & 0 & 0.1 & 1 & - & - & 0 & 0 & 0.1\\
\texttt{primary-tumor} & \multicolumn{1}{r}{336} & \multicolumn{1}{r}{31}  & 15 & \textbf{0.0} & 1 & 15 & 1814.7 & 1 & - & - & 0 & 22 & 0.0\\
\texttt{segment} & \multicolumn{1}{r}{2310} & \multicolumn{1}{r}{235}  & 0 & \textbf{0.0} & 1 & 0 & 0.0 & 1 & 0 & 0.3 & 1 & 0 & 0.0\\
\texttt{soybean} & \multicolumn{1}{r}{630} & \multicolumn{1}{r}{50}  & 2 & 0.1 & 1 & 2 & 88.8 & 0 & - & - & 0 & 8 & \textbf{0.0}\\
\texttt{splice-1} & \multicolumn{1}{r}{3190} & \multicolumn{1}{r}{287}  & \textbf{24} & 0.5 & 0 & 31 & 1670.3 & 0 & - & - & 0 & 34 & \textbf{0.0}\\
\texttt{surgical-deepnet} & \multicolumn{1}{r}{14635} & \multicolumn{1}{r}{6047}  & \textbf{1297} & 65.8 & 0 & 1609 & 147.5 & 0 & - & - & 0 & 1400 & \textbf{8.5}\\
\texttt{taiwan\_binarised} & \multicolumn{1}{r}{30000} & \multicolumn{1}{r}{205}  & \textbf{4727} & 3246.2 & 0 & 5172 & 2489.3 & 0 & 5412 & 3600.0 & 0 & 5043 & \textbf{0.7}\\
\texttt{tic-tac-toe} & \multicolumn{1}{r}{958} & \multicolumn{1}{r}{27}  & 0 & \textbf{0.0} & 1 & 0 & 0.0 & 1 & 0 & 1.5 & 1 & 13 & 0.0\\
\texttt{titanic} & \multicolumn{1}{r}{887} & \multicolumn{1}{r}{333}  & \textbf{64} & 1499.8 & 0 & 88 & 2490.9 & 0 & - & - & 0 & 105 & \textbf{0.0}\\
\texttt{vehicle} & \multicolumn{1}{r}{846} & \multicolumn{1}{r}{252}  & 0 & \textbf{0.0} & 1 & 0 & 0.1 & 1 & 0 & 23.9 & 1 & 3 & 0.0\\
\texttt{vote} & \multicolumn{1}{r}{435} & \multicolumn{1}{r}{48}  & 0 & \textbf{0.0} & 1 & 0 & 0.0 & 1 & 0 & 0.0 & 1 & 1 & 0.0\\
\texttt{weather-aus} & \multicolumn{1}{r}{142193} & \multicolumn{1}{r}{4759}  & \textbf{1657} & 2060.5 & 0 & 1709 & 1451.9 & 0 & - & - & 0 & 1703 & \textbf{21.3}\\
\texttt{wine1} & \multicolumn{1}{r}{178} & \multicolumn{1}{r}{1276}  & 27 & 42.9 & 0 & 27 & 406.2 & 0 & - & - & 0 & 30 & \textbf{0.0}\\
\texttt{wine2} & \multicolumn{1}{r}{178} & \multicolumn{1}{r}{1276}  & \textbf{30} & 604.9 & 0 & 31 & 1003.5 & 0 & - & - & 0 & 35 & \textbf{0.0}\\
\texttt{wine3} & \multicolumn{1}{r}{178} & \multicolumn{1}{r}{1276}  & \textbf{20} & 406.5 & 0 & 21 & 26.0 & 0 & - & - & 0 & 24 & \textbf{0.0}\\
\texttt{yeast} & \multicolumn{1}{r}{1484} & \multicolumn{1}{r}{89}  & \textbf{132} & 1681.5 & 0 & 222 & 2259.5 & 0 & - & - & 0 & 261 & \textbf{0.0}\\
\bottomrule
\end{tabular}

\end{scriptsize}
\end{center}
\caption{\label{tab:all8} Comparison with state of the art: depth 8}
\end{table}

\begin{table}[htbp]
\begin{center}
\begin{scriptsize}
\tabcolsep=2pt
\begin{tabular}{lrrrrrrrrrrrr}
\toprule
\multirow{2}{*}{}&  \multicolumn{2}{c}{\budalg} & \multicolumn{2}{c}{\murtree} & \multicolumn{2}{c}{\dleight} & \multicolumn{2}{c}{\cp} & \multicolumn{2}{c}{binoct} & \multicolumn{2}{c}{\cart}\\
\cmidrule(rr){2-3}\cmidrule(rr){4-5}\cmidrule(rr){6-7}\cmidrule(rr){8-9}\cmidrule(rr){10-11}\cmidrule(rr){12-13}
& \multicolumn{1}{c}{error} & \multicolumn{1}{c}{cpu} & \multicolumn{1}{c}{error} & \multicolumn{1}{c}{cpu} & \multicolumn{1}{c}{error} & \multicolumn{1}{c}{cpu} & \multicolumn{1}{c}{error} & \multicolumn{1}{c}{cpu} & \multicolumn{1}{c}{error} & \multicolumn{1}{c}{cpu} & \multicolumn{1}{c}{error} & \multicolumn{1}{c}{cpu} \\
\midrule

\texttt{hepatitis} & 0 & 0.00$^*$ & 0 & 0.01$^*$ & 0 & 0.00$^*$ & 0 & 0.73$^*$ & 0 & 3354$^*$ & 0 & 0.00\\
\texttt{lymph} & 0 & 0.00$^*$ & 0 & 0.00$^*$ & 0 & 0.00$^*$ & 0 & 0.66$^*$ & 3 & 3446 & 0 & 0.00\\
\texttt{wine1} & 24 & 1541 & 24 & 1299 & - & - & 30 & $\mathsmaller{\geq}1$h & 59 & 0.00 & 27 & 0.01\\
\texttt{wine2} & 27 & 505 & 27 & 183 & - & - & 32 & $\mathsmaller{\geq}1$h & 71 & 0.00 & 32 & 0.01\\
\texttt{wine3} & 18 & 317 & \textbf{17} & 237 & - & - & 20 & $\mathsmaller{\geq}1$h & 48 & 0.00 & 18 & 0.01\\
\texttt{audiology} & 0 & 0.00$^*$ & 0 & 0.01$^*$ & 0 & 0.00$^*$ & 0 & 0.65$^*$ & 11 & 9.1 & 0 & 0.00\\
\texttt{heart-cleveland} & 0 & 0.00$^*$ & 0 & 0.02$^*$ & 0 & 130$^*$ & 0 & 0.75$^*$ & 136 & 87 & 0 & 0.00\\
\texttt{primary-tumor} & 15 & 0.00$^*$ & 15 & 2883 & - & - & 82 & $\mathsmaller{\geq}1$h & 24 & 3496 & 21 & 0.01\\
\texttt{ionosphere} & 0 & 0.00$^*$ & 0 & 0.18$^*$ & 0 & 317$^*$ & 0 & 11$^*$ & 225 & 0.00 & 0 & 0.01\\
\texttt{vote} & 0 & 0.00$^*$ & 0 & 0.00$^*$ & 0 & 0.00$^*$ & 0 & 2.5$^*$ & 4 & 3383 & 1 & 0.00\\
\texttt{forest-fires} & 133 & 7.7 & \textbf{125} & 3224 & - & - & 247 & $\mathsmaller{\geq}1$h & - & - & 152 & 0.02\\
\texttt{soybean} & 2 & 0.04$^*$ & 2 & 557 & - & - & 92 & $\mathsmaller{\geq}1$h & 92 & 0.00 & 5 & 0.00\\
\texttt{australian-credit} & 0 & 1.8$^*$ & 0 & 4.6$^*$ & - & - & 296 & $\mathsmaller{\geq}1$h & 357 & 0.00 & 19 & 0.01\\
\texttt{breast-cancer} & 0 & 10$^*$ & 0 & 5.1$^*$ & 0 & 0.00$^*$ & 0 & 134$^*$ & 226 & 158 & 1 & 0.00\\
\texttt{breast-wisconsin} & 0 & 0.00$^*$ & 0 & 0.02$^*$ & 0 & 42$^*$ & 0 & 14$^*$ & 225 & 3167 & 0 & 0.00\\
\texttt{diabetes} & 0 & 2.6$^*$ & 0 & 14$^*$ & - & - & 0 & 1897$^*$ & 500 & 0.00 & 55 & 0.01\\
\texttt{anneal} & \textbf{35} & 340 & 42 & 1289 & - & - & 187 & $\mathsmaller{\geq}1$h & 187 & 308 & 74 & 0.00\\
\texttt{vehicle} & 0 & 0.00$^*$ & 0 & 0.07$^*$ & 0 & 6.9$^*$ & 0 & 58$^*$ & 218 & 0.00 & 1 & 0.01\\
\texttt{titanic} & \textbf{47} & 2117 & 82 & 1571 & - & - & 342 & $\mathsmaller{\geq}1$h & 342 & 0.00 & 93 & 0.01\\
\texttt{tic-tac-toe} & 0 & 0.00$^*$ & 0 & 0.00$^*$ & 0 & 0.23$^*$ & 0 & 0.76$^*$ & 332 & 94 & 10 & 0.00\\
\texttt{german-credit} & \textbf{4} & 3513 & 19 & 1244 & - & - & 254 & $\mathsmaller{\geq}1$h & 700 & 0.00 & 97 & 0.01\\
\texttt{yeast} & \textbf{68} & 3461 & 205 & 1767 & - & - & 463 & $\mathsmaller{\geq}1$h & 463 & 0.00 & 232 & 0.01\\
\texttt{car} & 0 & 77$^*$ & 0 & 24$^*$ & 0 & 1.3$^*$ & 0 & 1227$^*$ & 518 & 0.00 & 15 & 0.00\\
\texttt{segment} & 0 & 0.00$^*$ & 0 & 0.01$^*$ & 0 & 0.16$^*$ & 0 & 0.93$^*$ & - & - & 0 & 0.01\\
\texttt{splice-1} & \textbf{12} & 212 & 24 & 301 & - & - & 1535 & $\mathsmaller{\geq}1$h & - & - & 18 & 0.05\\
\texttt{kr-vs-kp} & \textbf{5} & 376 & 31 & 1387 & - & - & 784 & $\mathsmaller{\geq}1$h & 1669 & 0.00 & 23 & 0.01\\
\texttt{hypothyroid} & \textbf{17} & 76$^*$ & 18 & 650 & - & - & 277 & $\mathsmaller{\geq}1$h & 2970 & 0.00 & 36 & 0.01\\
\texttt{compas\_discretized} & 1828 & 205$^*$ & 1828 & 3443 & - & - & 2809 & $\mathsmaller{\geq}1$h & 2809 & 0.00 & 1891 & 0.01\\
\texttt{pendigits} & 0 & 0.00$^*$ & 0 & 0.11$^*$ & - & - & 0 & 4.6$^*$ & - & - & 0 & 0.07\\
\texttt{mushroom} & 0 & 0.00$^*$ & 0 & 0.02$^*$ & 0 & 1.6$^*$ & 0 & 0.87$^*$ & - & - & 0 & 0.03\\
\texttt{surgical-deepnet} & \textbf{1127} & 53 & 1503 & 484 & - & - & 3690 & $\mathsmaller{\geq}1$h & - & - & 1193 & 11\\
\texttt{letter} & \textbf{2} & 366 & 4 & 1309 & 697 & $\mathsmaller{\geq}1$h & 813 & $\mathsmaller{\geq}1$h & - & - & 48 & 0.37\\
\texttt{taiwan\_binarised} & \textbf{4427} & 3484 & 5123 & 1990 & - & - & 6636 & $\mathsmaller{\geq}1$h & - & - & 4911 & 0.63\\
\texttt{adult\_discretized} & \textbf{3999} & 669 & 4094 & 2569 & 6200 & $\mathsmaller{\geq}1$h & 7511 & $\mathsmaller{\geq}1$h & - & - & 4252 & 0.12\\
\texttt{bank} & \textbf{3493} & 944 & 3955 & 695 & 4817 & $\mathsmaller{\geq}1$h & 5289 & $\mathsmaller{\geq}1$h & - & - & 3575 & 76\\
\texttt{mnist\_8} & \textbf{849} & 1304 & 2023 & 2162 & - & - & 5851 & $\mathsmaller{\geq}1$h & - & - & 1267 & 6.9\\
\texttt{mnist\_9} & \textbf{1829} & 2522 & 2479 & 549 & - & - & 5949 & $\mathsmaller{\geq}1$h & - & - & 2110 & 9.3\\
\texttt{mnist\_0} & \textbf{595} & 1511 & 1084 & 39 & - & - & 5923 & $\mathsmaller{\geq}1$h & - & - & 710 & 8.6\\
\texttt{mnist\_6} & 1138 & 1606 & \textbf{1070} & 1668 & 2718 & $\mathsmaller{\geq}1$h & 5918 & $\mathsmaller{\geq}1$h & - & - & 1245 & 6.2\\
\texttt{mnist\_5} & 2424 & 612 & \textbf{2327} & 1184 & 4379 & $\mathsmaller{\geq}1$h & 5421 & $\mathsmaller{\geq}1$h & - & - & 2553 & 9.1\\
\texttt{mnist\_3} & \textbf{1213} & 543 & 2022 & 546 & - & - & 6131 & $\mathsmaller{\geq}1$h & - & - & 1442 & 6.9\\
\texttt{mnist\_2} & \textbf{1682} & 2193 & 2158 & 3507 & - & - & 5958 & $\mathsmaller{\geq}1$h & - & - & 2058 & 7.2\\
\texttt{mnist\_4} & \textbf{908} & 3043 & 1870 & 912 & 5580 & $\mathsmaller{\geq}1$h & 5842 & $\mathsmaller{\geq}1$h & - & - & 1306 & 5.4\\
\texttt{mnist\_7} & \textbf{1261} & 1792 & 1576 & 577 & 4546 & $\mathsmaller{\geq}1$h & 6265 & $\mathsmaller{\geq}1$h & - & - & 1371 & 7.2\\
\texttt{mnist\_1} & \textbf{465} & 500 & 846 & 1045 & 4547 & $\mathsmaller{\geq}1$h & 6742 & $\mathsmaller{\geq}1$h & - & - & 573 & 6.5\\
\texttt{weather-aus} & \textbf{1638} & 2359 & 1694 & 674 & - & - & 1761 & $\mathsmaller{\geq}1$h & - & - & 1677 & 27\\
\bottomrule
\end{tabular}

\end{scriptsize}
\end{center}
\caption{\label{tab:all9} Comparison with state of the art: depth 9}
\end{table}

\begin{table}[htbp]
\begin{center}
\begin{scriptsize}
\tabcolsep=2pt
\begin{tabular}{lccrrrrrrrrrrrrrrr}
\toprule
& && \multicolumn{4}{c}{\budalg} & \multicolumn{4}{c}{\murtree} & \multicolumn{4}{c}{\dleight} & \multicolumn{3}{c}{\cart}\\
\cmidrule(rr){4-7}\cmidrule(rr){8-11}\cmidrule(rr){12-15}\cmidrule(rr){16-18}
&\multirow{1}{*}{$\#ex.$} & \multirow{1}{*}{\#feat.} &  \multicolumn{1}{c}{error} & \multicolumn{1}{c}{acc.} & \multicolumn{1}{c}{cpu} & \multicolumn{1}{c}{opt.} & \multicolumn{1}{c}{error} & \multicolumn{1}{c}{acc.} & \multicolumn{1}{c}{cpu} & \multicolumn{1}{c}{opt.} & \multicolumn{1}{c}{error} & \multicolumn{1}{c}{acc.} & \multicolumn{1}{c}{cpu} & \multicolumn{1}{c}{opt.} & \multicolumn{1}{c}{error} & \multicolumn{1}{c}{acc.} & \multicolumn{1}{c}{cpu} \\
\midrule

\texttt{adult\_discretized} & \multicolumn{1}{r}{30299} & \multicolumn{1}{r}{59}  & \textbf{3881} & 0.8719 & 1647.2 & 0.00 & 4052 & 0.8663 & 1634.5 & 0.00 & - & - & - & 0.00 & 4148 & 0.8631 & \textbf{0.1}\\
\texttt{anneal} & \multicolumn{1}{r}{812} & \multicolumn{1}{r}{93}  & \textbf{34} & 0.9581 & 21.8 & \textbf{1.00} & 39 & 0.9520 & 831.7 & 0.00 & - & - & - & 0.00 & 59 & 0.9273 & \textbf{0.0}\\
\texttt{audiology} & \multicolumn{1}{r}{216} & \multicolumn{1}{r}{148}  & 0 & 1.0000 & \textbf{0.0} & 1.00 & 0 & 1.0000 & 0.0 & 1.00 & 0 & 1.0000 & 0.0 & 1.00 & 0 & 1.0000 & 0.0\\
\texttt{australian-credit} & \multicolumn{1}{r}{653} & \multicolumn{1}{r}{125}  & 0 & 1.0000 & 0.0 & 1.00 & 0 & 1.0000 & 0.6 & 1.00 & - & - & - & 0.00 & 12 & 0.9816 & \textbf{0.0}\\
\texttt{bank} & \multicolumn{1}{r}{45211} & \multicolumn{1}{r}{9531}  & \textbf{3241} & 0.9283 & \textbf{21.2} & 0.00 & 3767 & 0.9167 & 3246.8 & 0.00 & 4826 & 0.8933 & 3607.2 & 0.00 & 3327 & 0.9264 & 101.7\\
\texttt{breast-cancer} & \multicolumn{1}{r}{683} & \multicolumn{1}{r}{89}  & 0 & 1.0000 & \textbf{0.0} & 1.00 & 0 & 1.0000 & 0.0 & 1.00 & 0 & 1.0000 & 0.0 & 1.00 & 0 & 1.0000 & 0.0\\
\texttt{breast-wisconsin} & \multicolumn{1}{r}{683} & \multicolumn{1}{r}{120}  & 0 & 1.0000 & \textbf{0.0} & 1.00 & 0 & 1.0000 & 0.0 & 1.00 & 0 & 1.0000 & 3.4 & 1.00 & 0 & 1.0000 & 0.0\\
\texttt{car} & \multicolumn{1}{r}{1728} & \multicolumn{1}{r}{21}  & 0 & 1.0000 & 0.2 & 1.00 & 0 & 1.0000 & 0.5 & 1.00 & 0 & 1.0000 & 0.0 & 1.00 & 11 & 0.9936 & \textbf{0.0}\\
\texttt{compas\_discretized} & \multicolumn{1}{r}{6167} & \multicolumn{1}{r}{25}  & \textbf{1828} & 0.7036 & 0.7 & \textbf{1.00} & 1843 & 0.7012 & 2942.8 & 0.00 & - & - & - & 0.00 & 1871 & 0.6966 & \textbf{0.0}\\
\texttt{diabetes} & \multicolumn{1}{r}{768} & \multicolumn{1}{r}{112}  & 0 & 1.0000 & 0.6 & 1.00 & 0 & 1.0000 & 10.7 & 1.00 & - & - & - & 0.00 & 35 & 0.9544 & \textbf{0.0}\\
\texttt{forest-fires} & \multicolumn{1}{r}{517} & \multicolumn{1}{r}{989}  & \textbf{113} & 0.7814 & 1023.8 & 0.00 & 131 & 0.7466 & 59.5 & 0.00 & - & - & - & 0.00 & 146 & 0.7176 & \textbf{0.0}\\
\texttt{german-credit} & \multicolumn{1}{r}{1000} & \multicolumn{1}{r}{112}  & 0 & 1.0000 & 70.6 & 1.00 & 0 & 1.0000 & 122.5 & 1.00 & - & - & - & 0.00 & 66 & 0.9340 & \textbf{0.0}\\
\texttt{heart-cleveland} & \multicolumn{1}{r}{296} & \multicolumn{1}{r}{95}  & 0 & 1.0000 & \textbf{0.0} & 1.00 & 0 & 1.0000 & 0.0 & 1.00 & 0 & 1.0000 & 0.1 & 1.00 & 0 & 1.0000 & 0.0\\
\texttt{hepatitis} & \multicolumn{1}{r}{137} & \multicolumn{1}{r}{68}  & 0 & 1.0000 & \textbf{0.0} & 1.00 & 0 & 1.0000 & 0.0 & 1.00 & 0 & 1.0000 & 0.0 & 1.00 & 0 & 1.0000 & 0.0\\
\texttt{hypothyroid} & \multicolumn{1}{r}{3247} & \multicolumn{1}{r}{88}  & 17 & 0.9948 & 1.0 & \textbf{1.00} & 17 & 0.9948 & 1053.2 & 0.00 & - & - & - & 0.00 & 31 & 0.9905 & \textbf{0.0}\\
\texttt{ionosphere} & \multicolumn{1}{r}{351} & \multicolumn{1}{r}{445}  & 0 & 1.0000 & \textbf{0.0} & 1.00 & 0 & 1.0000 & 0.1 & 1.00 & 0 & 1.0000 & 109.7 & 1.00 & 0 & 1.0000 & 0.0\\
\texttt{kr-vs-kp} & \multicolumn{1}{r}{3196} & \multicolumn{1}{r}{73}  & \textbf{0} & 1.0000 & 1719.0 & \textbf{1.00} & 24 & 0.9925 & 2085.8 & 0.00 & - & - & - & 0.00 & 12 & 0.9962 & \textbf{0.0}\\
\texttt{letter} & \multicolumn{1}{r}{20000} & \multicolumn{1}{r}{224}  & 0 & 1.0000 & 74.3 & 1.00 & 0 & 1.0000 & 888.0 & 1.00 & 725 & 0.9637 & 3600.0 & 0.00 & 21 & 0.9990 & \textbf{0.3}\\
\texttt{lymph} & \multicolumn{1}{r}{148} & \multicolumn{1}{r}{68}  & 0 & 1.0000 & \textbf{0.0} & 1.00 & 0 & 1.0000 & 0.0 & 1.00 & 0 & 1.0000 & 0.0 & 1.00 & 0 & 1.0000 & 0.0\\
\texttt{mnist\_0} & \multicolumn{1}{r}{60000} & \multicolumn{1}{r}{784}  & \textbf{412} & 0.9931 & 1172.0 & 0.00 & 880 & 0.9853 & 125.3 & 0.00 & 3314 & 0.9448 & 3600.4 & 0.00 & 477 & 0.9920 & \textbf{8.5}\\
\texttt{mnist\_1} & \multicolumn{1}{r}{60000} & \multicolumn{1}{r}{784}  & \textbf{405} & 0.9932 & 1383.9 & 0.00 & 610 & 0.9898 & 3247.7 & 0.00 & 4544 & 0.9243 & 3600.4 & 0.00 & 439 & 0.9927 & \textbf{7.8}\\
\texttt{mnist\_2} & \multicolumn{1}{r}{60000} & \multicolumn{1}{r}{784}  & \textbf{1801} & 0.9700 & 958.9 & 0.00 & 1899 & 0.9684 & 3235.7 & 0.00 & - & - & - & 0.00 & 1959 & 0.9674 & \textbf{8.7}\\
\texttt{mnist\_3} & \multicolumn{1}{r}{60000} & \multicolumn{1}{r}{784}  & \textbf{1050} & 0.9825 & 1258.0 & 0.00 & 1568 & 0.9739 & 327.4 & 0.00 & 5171 & 0.9138 & 3600.5 & 0.00 & 1169 & 0.9805 & \textbf{6.7}\\
\texttt{mnist\_4} & \multicolumn{1}{r}{60000} & \multicolumn{1}{r}{784}  & \textbf{908} & 0.9849 & 246.0 & 0.00 & 1517 & 0.9747 & 459.4 & 0.00 & 5580 & 0.9070 & 3600.5 & 0.00 & 1010 & 0.9832 & \textbf{10.3}\\
\texttt{mnist\_5} & \multicolumn{1}{r}{60000} & \multicolumn{1}{r}{784}  & 2142 & 0.9643 & 96.6 & 0.00 & \textbf{2088} & 0.9652 & 1066.8 & 0.00 & 4379 & 0.9270 & 3600.4 & 0.00 & 2266 & 0.9622 & \textbf{6.9}\\
\texttt{mnist\_6} & \multicolumn{1}{r}{60000} & \multicolumn{1}{r}{784}  & 1115 & 0.9814 & 2491.3 & 0.00 & \textbf{952} & 0.9841 & 1118.9 & 0.00 & 2699 & 0.9550 & 3600.3 & 0.00 & 1211 & 0.9798 & \textbf{7.4}\\
\texttt{mnist\_7} & \multicolumn{1}{r}{60000} & \multicolumn{1}{r}{784}  & \textbf{1126} & 0.9812 & 836.0 & 0.00 & 1264 & 0.9789 & 1696.2 & 0.00 & - & - & - & 0.00 & 1263 & 0.9789 & \textbf{10.7}\\
\texttt{mnist\_8} & \multicolumn{1}{r}{60000} & \multicolumn{1}{r}{784}  & \textbf{836} & 0.9861 & 720.5 & 0.00 & 1560 & 0.9740 & 2308.9 & 0.00 & - & - & - & 0.00 & 916 & 0.9847 & \textbf{7.9}\\
\texttt{mnist\_9} & \multicolumn{1}{r}{60000} & \multicolumn{1}{r}{784}  & \textbf{1597} & 0.9734 & 2446.4 & 0.00 & 2221 & 0.9630 & 3366.8 & 0.00 & - & - & - & 0.00 & 1722 & 0.9713 & \textbf{7.1}\\
\texttt{mushroom} & \multicolumn{1}{r}{8124} & \multicolumn{1}{r}{119}  & 0 & 1.0000 & \textbf{0.0} & 1.00 & 0 & 1.0000 & 0.0 & 1.00 & 0 & 1.0000 & 1.1 & 1.00 & 0 & 1.0000 & 0.0\\
\texttt{pendigits} & \multicolumn{1}{r}{7494} & \multicolumn{1}{r}{216}  & 0 & 1.0000 & \textbf{0.0} & 1.00 & 0 & 1.0000 & 0.1 & 1.00 & 0 & 1.0000 & 1246.6 & 1.00 & 0 & 1.0000 & 0.1\\
\texttt{primary-tumor} & \multicolumn{1}{r}{336} & \multicolumn{1}{r}{31}  & 15 & 0.9554 & \textbf{0.0} & \textbf{1.00} & 15 & 0.9554 & 2612.1 & 0.00 & - & - & - & 0.00 & 20 & 0.9405 & 0.0\\
\texttt{segment} & \multicolumn{1}{r}{2310} & \multicolumn{1}{r}{235}  & 0 & 1.0000 & \textbf{0.0} & 1.00 & 0 & 1.0000 & 0.0 & 1.00 & 0 & 1.0000 & 0.1 & 1.00 & 0 & 1.0000 & 0.0\\
\texttt{soybean} & \multicolumn{1}{r}{630} & \multicolumn{1}{r}{50}  & 2 & 0.9968 & \textbf{0.0} & \textbf{1.00} & 2 & 0.9968 & 380.2 & 0.00 & - & - & - & 0.00 & 2 & 0.9968 & 0.0\\
\texttt{splice-1} & \multicolumn{1}{r}{3190} & \multicolumn{1}{r}{287}  & \textbf{5} & 0.9984 & 1200.0 & 0.00 & 13 & 0.9959 & 992.6 & 0.00 & - & - & - & 0.00 & 12 & 0.9962 & \textbf{0.1}\\
\texttt{surgical-deepnet} & \multicolumn{1}{r}{14635} & \multicolumn{1}{r}{6047}  & \textbf{1040} & 0.9289 & 1737.2 & 0.00 & 1382 & 0.9056 & 1134.3 & 0.00 & - & - & - & 0.00 & 1089 & 0.9256 & \textbf{14.4}\\
\texttt{taiwan\_binarised} & \multicolumn{1}{r}{30000} & \multicolumn{1}{r}{205}  & \textbf{4410} & 0.8530 & 452.8 & 0.00 & 4993 & 0.8336 & 1334.8 & 0.00 & - & - & - & 0.00 & 4710 & 0.8430 & \textbf{0.5}\\
\texttt{tic-tac-toe} & \multicolumn{1}{r}{958} & \multicolumn{1}{r}{27}  & 0 & 1.0000 & \textbf{0.0} & 1.00 & 0 & 1.0000 & 0.0 & 1.00 & 0 & 1.0000 & 0.0 & 1.00 & 6 & 0.9937 & 0.0\\
\texttt{titanic} & \multicolumn{1}{r}{887} & \multicolumn{1}{r}{333}  & \textbf{53} & 0.9403 & 3388.2 & 0.00 & 80 & 0.9098 & 1856.8 & 0.00 & - & - & - & 0.00 & 78 & 0.9121 & \textbf{0.0}\\
\texttt{vehicle} & \multicolumn{1}{r}{846} & \multicolumn{1}{r}{252}  & 0 & 1.0000 & \textbf{0.0} & 1.00 & 0 & 1.0000 & 0.1 & 1.00 & 0 & 1.0000 & 0.4 & 1.00 & 0 & 1.0000 & 0.0\\
\texttt{vote} & \multicolumn{1}{r}{435} & \multicolumn{1}{r}{48}  & 0 & 1.0000 & \textbf{0.0} & 1.00 & 0 & 1.0000 & 0.0 & 1.00 & 0 & 1.0000 & 0.0 & 1.00 & 0 & 1.0000 & 0.0\\
\texttt{weather-aus} & \multicolumn{1}{r}{142193} & \multicolumn{1}{r}{4759}  & \textbf{1628} & 0.9886 & 38.0 & 0.00 & 1675 & 0.9882 & 2103.3 & 0.00 & - & - & - & 0.00 & 1642 & 0.9885 & \textbf{31.6}\\
\texttt{wine1} & \multicolumn{1}{r}{178} & \multicolumn{1}{r}{1276}  & \textbf{22} & 0.8764 & 541.5 & 0.00 & 23 & 0.8708 & 132.0 & 0.00 & - & - & - & 0.00 & 25 & 0.8596 & \textbf{0.0}\\
\texttt{wine2} & \multicolumn{1}{r}{178} & \multicolumn{1}{r}{1276}  & \textbf{24} & 0.8652 & 424.8 & 0.00 & 27 & 0.8483 & 0.6 & 0.00 & - & - & - & 0.00 & 29 & 0.8371 & \textbf{0.0}\\
\texttt{wine3} & \multicolumn{1}{r}{178} & \multicolumn{1}{r}{1276}  & 16 & 0.9101 & 266.0 & 0.00 & \textbf{15} & 0.9157 & 2619.2 & 0.00 & - & - & - & 0.00 & 19 & 0.8933 & \textbf{0.0}\\
\texttt{yeast} & \multicolumn{1}{r}{1484} & \multicolumn{1}{r}{89}  & \textbf{26} & 0.9825 & 1244.0 & 0.00 & 170 & 0.8854 & 3113.1 & 0.00 & - & - & - & 0.00 & 185 & 0.8753 & \textbf{0.0}\\
\bottomrule
\end{tabular}

\end{scriptsize}
\end{center}
\caption{\label{tab:all10} Comparison with state of the art: depth 10}
\end{table}


%
%
% \begin{table}[htbp]%
% \begin{center}%
% \begin{footnotesize}%
% \tabcolsep=2pt%
% \begin{tabular}{lccrrrrrrrr}
\toprule
\multirow{2}{*}{}& && \multicolumn{2}{c}{\budalg} & \multicolumn{2}{c}{\noheuristic} & \multicolumn{2}{c}{\nopreprocessing} & \multicolumn{2}{c}{\nolb}\\
\cmidrule(rr){4-5}\cmidrule(rr){6-7}\cmidrule(rr){8-9}\cmidrule(rr){10-11}
&\multirow{1}{*}{$|\allex|$} & \multirow{1}{*}{$|\features|$} &  \multicolumn{1}{c}{error} & \multicolumn{1}{c}{cpu} & \multicolumn{1}{c}{error} & \multicolumn{1}{c}{cpu} & \multicolumn{1}{c}{error} & \multicolumn{1}{c}{cpu} & \multicolumn{1}{c}{error} & \multicolumn{1}{c}{cpu} \\
\midrule

\texttt{hepatitis} & 10 & 0.00$^*$ & 10 & 0.00$^*$ & 10 & 0.07$^*$ & 10 & 0.01$^*$\\
\texttt{lymph} & 12 & 0.01$^*$ & 12 & 0.01$^*$ & 12 & 0.06$^*$ & 12 & 0.02$^*$\\
\texttt{wine1} & 43 & 16$^*$ & 43 & 14$^*$ & 43 & 120$^*$ & 43 & 17$^*$\\
\texttt{wine2} & 49 & 17$^*$ & 49 & 14$^*$ & 49 & 118$^*$ & 49 & 17$^*$\\
\texttt{wine3} & 33 & 16$^*$ & 33 & 13$^*$ & 33 & 118$^*$ & 33 & 16$^*$\\
\texttt{audiology} & 5 & 0.06$^*$ & 5 & 0.04$^*$ & 5 & 0.33$^*$ & 5 & 0.06$^*$\\
\texttt{heart-cleveland} & 41 & 0.05$^*$ & 41 & 0.03$^*$ & 41 & 0.20$^*$ & 41 & 0.05$^*$\\
\texttt{primary-tumor} & 46 & 0.00$^*$ & 46 & 0.00$^*$ & 46 & 0.01$^*$ & 46 & 0.00$^*$\\
\texttt{ionosphere} & 22 & 3.8$^*$ & 22 & 3.0$^*$ & 22 & 22$^*$ & 22 & 4.2$^*$\\
\texttt{vote} & 12 & 0.02$^*$ & 12 & 0.02$^*$ & 12 & 0.03$^*$ & 12 & 0.03$^*$\\
\texttt{forest-fires} & 193 & 20$^*$ & 193 & 16$^*$ & 193 & 65$^*$ & 193 & 20$^*$\\
\texttt{soybean} & 29 & 0.01$^*$ & 29 & 0.01$^*$ & 29 & 0.03$^*$ & 29 & 0.02$^*$\\
\texttt{australian-credit} & 73 & 0.14$^*$ & 73 & 0.11$^*$ & 73 & 0.55$^*$ & 73 & 0.13$^*$\\
\texttt{breast-cancer} & 24 & 0.16$^*$ & 24 & 0.08$^*$ & 24 & 0.10$^*$ & 24 & 0.10$^*$\\
\texttt{breast-wisconsin} & 15 & 0.05$^*$ & 15 & 0.04$^*$ & 15 & 0.28$^*$ & 15 & 0.06$^*$\\
\texttt{diabetes} & 162 & 0.09$^*$ & 162 & 0.08$^*$ & 162 & 0.50$^*$ & 162 & 0.09$^*$\\
\texttt{anneal} & 112 & 0.03$^*$ & 112 & 0.02$^*$ & 112 & 0.17$^*$ & 112 & 0.03$^*$\\
\texttt{vehicle} & 26 & 0.93$^*$ & 26 & 0.59$^*$ & 26 & 3.5$^*$ & 26 & 0.83$^*$\\
\texttt{titanic} & 143 & 6.7$^*$ & 143 & 5.5$^*$ & 143 & 6.6$^*$ & 143 & 6.7$^*$\\
\texttt{tic-tac-toe} & 216 & 0.01$^*$ & 216 & 0.01$^*$ & 216 & 0.01$^*$ & 216 & 0.01$^*$\\
\texttt{german-credit} & 236 & 0.26$^*$ & 236 & 0.20$^*$ & 236 & 0.54$^*$ & 236 & 0.26$^*$\\
\texttt{yeast} & 403 & 0.07$^*$ & 403 & 0.07$^*$ & 403 & 0.36$^*$ & 403 & 0.07$^*$\\
\texttt{car} & 192 & 0.01$^*$ & 192 & 0.00$^*$ & 192 & 0.00$^*$ & 192 & 0.01$^*$\\
\texttt{segment} & 0 & 0.03$^*$ & 0 & 0.03$^*$ & 0 & 0.20$^*$ & 0 & 0.03$^*$\\
\texttt{splice-1} & 224 & 9.8$^*$ & 224 & 8.2$^*$ & 224 & 11$^*$ & 224 & 9.8$^*$\\
\texttt{kr-vs-kp} & 198 & 0.09$^*$ & 198 & 0.06$^*$ & 198 & 0.22$^*$ & 198 & 0.07$^*$\\
\texttt{hypothyroid} & 61 & 0.07$^*$ & 61 & 0.07$^*$ & 61 & 0.33$^*$ & 61 & 0.08$^*$\\
\texttt{compas\_discretized} & 2004 & 0.00$^*$ & 2004 & 0.00$^*$ & 2004 & 0.03$^*$ & 2004 & 0.00$^*$\\
\texttt{pendigits} & 47 & 3.3$^*$ & 47 & 3.1$^*$ & 47 & 13$^*$ & 47 & 3.6$^*$\\
\texttt{mushroom} & 8 & 0.79$^*$ & 8 & 0.60$^*$ & 8 & 0.76$^*$ & 8 & 0.68$^*$\\
\texttt{surgical-deepnet} & 2512 & 953 & 2524 & 1304 & 2512 & 907 & 2512 & 918\\
\texttt{letter} & 369 & 10$^*$ & 369 & 8.4$^*$ & 369 & 45$^*$ & 369 & 8.2$^*$\\
\texttt{taiwan\_binarised} & 5326 & 48$^*$ & 5326 & 28$^*$ & 5326 & 45$^*$ & 5326 & 33$^*$\\
\texttt{adult\_discretized} & 5020 & 0.43$^*$ & 5020 & 0.25$^*$ & 5020 & 0.72$^*$ & 5020 & 0.27$^*$\\
\texttt{bank} & 4453 & 259 & \textbf{4383} & 84 & 4453 & 226 & 4453 & 257\\
\texttt{mnist\_6} & 2756 & 1916$^*$ & 2756 & 1785$^*$ & 2756 & 1804$^*$ & 2756 & 2002$^*$\\
\texttt{mnist\_2} & 3938 & 1946$^*$ & 3938 & 1254$^*$ & 3938 & 1844$^*$ & 3938 & 2009$^*$\\
\texttt{mnist\_3} & 4354 & 2054$^*$ & 4354 & 1906$^*$ & 4354 & 1821$^*$ & 4354 & 1848$^*$\\
\texttt{mnist\_8} & 3583 & 2061$^*$ & 3583 & 1922$^*$ & 3583 & 1802$^*$ & 3583 & 1524$^*$\\
\texttt{mnist\_5} & 3539 & 2095$^*$ & 3539 & 1825$^*$ & 3539 & 1844$^*$ & 3539 & 1706$^*$\\
\texttt{mnist\_7} & 3483 & 1928$^*$ & 3483 & 1966$^*$ & 3483 & 1792$^*$ & 3483 & 1653$^*$\\
\texttt{mnist\_4} & 4729 & 2070$^*$ & 4729 & 1837$^*$ & 4729 & 1784$^*$ & 4729 & 2195$^*$\\
\texttt{mnist\_9} & 4590 & 2039$^*$ & 4590 & 1895$^*$ & 4590 & 1833$^*$ & 4590 & 2285$^*$\\
\texttt{mnist\_0} & 2557 & 1994$^*$ & 2557 & 1832$^*$ & 2557 & 1792$^*$ & 2557 & 1867$^*$\\
\texttt{mnist\_1} & 3462 & 1896$^*$ & 3462 & 1290$^*$ & 3462 & 1774$^*$ & 3462 & 1676$^*$\\
\texttt{weather-aus} & 1756 & 14 & 1756 & 1.3 & 1756 & 12 & 1756 & 13\\
\bottomrule
\end{tabular}
%
% \end{footnotesize}%
% \end{center}%
% \caption{\label{tab:fact3} Factor analysis: depth 3}%
% \end{table}%
%
% \begin{table}[htbp]
% \begin{center}
% \begin{footnotesize}
% \tabcolsep=2pt
% \begin{tabular}{lccrrrrrrrrrrrr}
\toprule
\multirow{2}{*}{}& && \multicolumn{3}{c}{\budalg} & \multicolumn{3}{c}{\noheuristic} & \multicolumn{3}{c}{\nopreprocessing} & \multicolumn{3}{c}{\nolb}\\
\cmidrule(rr){4-6}\cmidrule(rr){7-9}\cmidrule(rr){10-12}\cmidrule(rr){13-15}
&\multirow{1}{*}{$\#ex.$} & \multirow{1}{*}{\#feat.} &  \multicolumn{1}{c}{error} & \multicolumn{1}{c}{cpu} & \multicolumn{1}{c}{opt.} & \multicolumn{1}{c}{error} & \multicolumn{1}{c}{cpu} & \multicolumn{1}{c}{opt.} & \multicolumn{1}{c}{error} & \multicolumn{1}{c}{cpu} & \multicolumn{1}{c}{opt.} & \multicolumn{1}{c}{error} & \multicolumn{1}{c}{cpu} & \multicolumn{1}{c}{opt.} \\
\midrule

\texttt{adult\_discretized} & \multicolumn{1}{r}{30299} & \multicolumn{1}{r}{59}  & 4609 & 14.3 & 1 & 4609 & \textbf{13.6} & 1 & 4609 & 42.8 & 1 & 4609 & 14.1 & 1\\
\texttt{anneal} & \multicolumn{1}{r}{812} & \multicolumn{1}{r}{93}  & 91 & 1.5 & 1 & 91 & \textbf{1.0} & 1 & 91 & 10.8 & 1 & 91 & 1.3 & 1\\
\texttt{audiology} & \multicolumn{1}{r}{216} & \multicolumn{1}{r}{148}  & 1 & 4.0 & 1 & 1 & \textbf{3.2} & 1 & 1 & 29.0 & 1 & 1 & 4.5 & 1\\
\texttt{australian-credit} & \multicolumn{1}{r}{653} & \multicolumn{1}{r}{125}  & 56 & 10.2 & 1 & 56 & \textbf{8.5} & 1 & 56 & 67.9 & 1 & 56 & 11.4 & 1\\
\texttt{bank} & \multicolumn{1}{r}{45211} & \multicolumn{1}{r}{9531}  & 4314 & 290.3 & 0 & 4326 & 1102.1 & 0 & 4314 & \textbf{257.9} & 0 & 4314 & 307.9 & 0\\
\texttt{breast-cancer} & \multicolumn{1}{r}{683} & \multicolumn{1}{r}{89}  & 16 & 9.6 & 1 & 16 & \textbf{7.6} & 1 & 16 & 9.1 & 1 & 16 & 8.9 & 1\\
\texttt{breast-wisconsin} & \multicolumn{1}{r}{683} & \multicolumn{1}{r}{120}  & 7 & 3.1 & 1 & 7 & \textbf{2.1} & 1 & 7 & 32.8 & 1 & 7 & 3.4 & 1\\
\texttt{car} & \multicolumn{1}{r}{1728} & \multicolumn{1}{r}{21}  & 136 & 0.2 & 1 & 136 & 0.2 & 1 & 136 & \textbf{0.1} & 1 & 136 & 0.2 & 1\\
\texttt{compas\_discretized} & \multicolumn{1}{r}{6167} & \multicolumn{1}{r}{25}  & 1954 & 0.1 & 1 & 1954 & \textbf{0.1} & 1 & 1954 & 0.7 & 1 & 1954 & 0.1 & 1\\
\texttt{diabetes} & \multicolumn{1}{r}{768} & \multicolumn{1}{r}{112}  & 137 & 5.7 & 1 & 137 & \textbf{4.8} & 1 & 137 & 59.2 & 1 & 137 & 6.0 & 1\\
\texttt{forest-fires} & \multicolumn{1}{r}{517} & \multicolumn{1}{r}{989}  & 173 & 14.7 & 0 & 173 & \textbf{10.8} & 0 & 173 & 47.5 & 0 & 173 & 14.7 & 0\\
\texttt{german-credit} & \multicolumn{1}{r}{1000} & \multicolumn{1}{r}{112}  & 204 & 28.2 & 1 & 204 & \textbf{21.6} & 1 & 204 & 65.8 & 1 & 204 & 29.0 & 1\\
\texttt{heart-cleveland} & \multicolumn{1}{r}{296} & \multicolumn{1}{r}{95}  & 25 & 3.1 & 1 & 25 & \textbf{2.3} & 1 & 25 & 19.0 & 1 & 25 & 3.3 & 1\\
\texttt{hepatitis} & \multicolumn{1}{r}{137} & \multicolumn{1}{r}{68}  & 3 & 0.3 & 1 & 3 & \textbf{0.2} & 1 & 3 & 3.0 & 1 & 3 & 0.3 & 1\\
\texttt{hypothyroid} & \multicolumn{1}{r}{3247} & \multicolumn{1}{r}{88}  & 53 & 2.9 & 1 & 53 & \textbf{2.5} & 1 & 53 & 23.4 & 1 & 53 & 3.1 & 1\\
\texttt{ionosphere} & \multicolumn{1}{r}{351} & \multicolumn{1}{r}{445}  & 7 & 729.7 & 1 & 7 & 548.1 & 1 & 8 & \textbf{54.6} & 0 & 7 & 1026.0 & 1\\
\texttt{kr-vs-kp} & \multicolumn{1}{r}{3196} & \multicolumn{1}{r}{73}  & 144 & 2.8 & 1 & 144 & \textbf{2.4} & 1 & 144 & 13.7 & 1 & 144 & 2.5 & 1\\
\texttt{letter} & \multicolumn{1}{r}{20000} & \multicolumn{1}{r}{224}  & 261 & 1185.2 & 1 & 261 & 813.3 & 1 & 261 & \textbf{292.3} & 0 & 261 & 1406.9 & 1\\
\texttt{lymph} & \multicolumn{1}{r}{148} & \multicolumn{1}{r}{68}  & 3 & 0.7 & 1 & 3 & \textbf{0.6} & 1 & 3 & 2.4 & 1 & 3 & 0.9 & 1\\
\texttt{mnist\_0} & \multicolumn{1}{r}{60000} & \multicolumn{1}{r}{784}  & 2173 & 2157.8 & 0 & 2229 & 3292.5 & 0 & 2173 & \textbf{1843.7} & 0 & 2173 & 2443.7 & 0\\
\texttt{mnist\_1} & \multicolumn{1}{r}{60000} & \multicolumn{1}{r}{784}  & 2332 & 2248.1 & 0 & 3462 & \textbf{563.4} & 0 & 2332 & 2211.4 & 0 & 2332 & 2698.5 & 0\\
\texttt{mnist\_2} & \multicolumn{1}{r}{60000} & \multicolumn{1}{r}{784}  & 3358 & \textbf{168.6} & 0 & 3647 & 2672.2 & 0 & 3143 & 3477.3 & 0 & 3143 & 3472.3 & 0\\
\texttt{mnist\_3} & \multicolumn{1}{r}{60000} & \multicolumn{1}{r}{784}  & 3485 & 2225.3 & 0 & 4230 & \textbf{1475.5} & 0 & 3485 & 2025.2 & 0 & 3485 & 2435.0 & 0\\
\texttt{mnist\_4} & \multicolumn{1}{r}{60000} & \multicolumn{1}{r}{784}  & 3670 & 2475.7 & 0 & 4466 & \textbf{500.1} & 0 & 3670 & 2258.4 & 0 & 3670 & 3062.4 & 0\\
\texttt{mnist\_5} & \multicolumn{1}{r}{60000} & \multicolumn{1}{r}{784}  & 3312 & 218.6 & 0 & 3312 & 1970.6 & 0 & 3312 & \textbf{198.2} & 0 & 3312 & 230.3 & 0\\
\texttt{mnist\_6} & \multicolumn{1}{r}{60000} & \multicolumn{1}{r}{784}  & 1940 & 2751.5 & 0 & 2009 & 2570.3 & 0 & 1940 & 2518.0 & 0 & 1940 & \textbf{2175.1} & 0\\
\texttt{mnist\_7} & \multicolumn{1}{r}{60000} & \multicolumn{1}{r}{784}  & 2793 & 51.7 & 0 & 3212 & 787.1 & 0 & 2793 & \textbf{50.4} & 0 & 2793 & 57.0 & 0\\
\texttt{mnist\_8} & \multicolumn{1}{r}{60000} & \multicolumn{1}{r}{784}  & 3165 & 1206.1 & 0 & 3572 & 2357.7 & 0 & 3165 & \textbf{1103.7} & 0 & 3165 & 1412.6 & 0\\
\texttt{mnist\_9} & \multicolumn{1}{r}{60000} & \multicolumn{1}{r}{784}  & 3977 & 2061.3 & 0 & 4182 & 3122.8 & 0 & 3977 & \textbf{1705.6} & 0 & 3977 & 1954.3 & 0\\
\texttt{mushroom} & \multicolumn{1}{r}{8124} & \multicolumn{1}{r}{119}  & 0 & 0.0 & 1 & 0 & 0.0 & 1 & 0 & 0.0 & 1 & 0 & 0.0 & 1\\
\texttt{pendigits} & \multicolumn{1}{r}{7494} & \multicolumn{1}{r}{216}  & 13 & \textbf{229.9} & 1 & 13 & 237.4 & 1 & 13 & 1871.4 & 1 & 13 & 341.0 & 1\\
\texttt{primary-tumor} & \multicolumn{1}{r}{336} & \multicolumn{1}{r}{31}  & 34 & 0.0 & 1 & 34 & \textbf{0.0} & 1 & 34 & 0.2 & 1 & 34 & 0.0 & 1\\
\texttt{segment} & \multicolumn{1}{r}{2310} & \multicolumn{1}{r}{235}  & 0 & 0.0 & 1 & 0 & 0.0 & 1 & 0 & 0.0 & 1 & 0 & 0.0 & 1\\
\texttt{soybean} & \multicolumn{1}{r}{630} & \multicolumn{1}{r}{50}  & 14 & 0.6 & 1 & 14 & \textbf{0.5} & 1 & 14 & 1.1 & 1 & 14 & 0.7 & 1\\
\texttt{splice-1} & \multicolumn{1}{r}{3190} & \multicolumn{1}{r}{287}  & 141 & 3241.2 & 1 & 141 & 2519.2 & 1 & 141 & \textbf{0.0} & 0 & 141 & 3563.1 & 1\\
\texttt{surgical-deepnet} & \multicolumn{1}{r}{14635} & \multicolumn{1}{r}{6047}  & 2269 & 48.9 & 0 & 2414 & 1478.7 & 0 & 2269 & \textbf{46.4} & 0 & 2269 & 50.9 & 0\\
\texttt{taiwan\_binarised} & \multicolumn{1}{r}{30000} & \multicolumn{1}{r}{205}  & 5273 & 6.2 & 0 & 5273 & 39.2 & 0 & 5273 & \textbf{6.2} & 0 & 5273 & 7.1 & 0\\
\texttt{tic-tac-toe} & \multicolumn{1}{r}{958} & \multicolumn{1}{r}{27}  & 137 & 0.4 & 1 & 137 & \textbf{0.3} & 1 & 137 & 0.4 & 1 & 137 & 0.4 & 1\\
\texttt{titanic} & \multicolumn{1}{r}{887} & \multicolumn{1}{r}{333}  & 119 & 1604.4 & 1 & 119 & \textbf{1317.8} & 1 & 119 & 1619.8 & 1 & 119 & 1721.5 & 1\\
\texttt{vehicle} & \multicolumn{1}{r}{846} & \multicolumn{1}{r}{252}  & 12 & 70.6 & 1 & 12 & \textbf{60.1} & 1 & 12 & 706.4 & 1 & 12 & 91.5 & 1\\
\texttt{vote} & \multicolumn{1}{r}{435} & \multicolumn{1}{r}{48}  & 5 & 1.2 & 1 & 5 & \textbf{0.9} & 1 & 5 & 1.2 & 1 & 5 & 1.4 & 1\\
\texttt{weather-aus} & \multicolumn{1}{r}{142193} & \multicolumn{1}{r}{4759}  & 1749 & 2525.2 & 0 & 1750 & 2646.2 & 0 & 1749 & \textbf{2142.0} & 0 & 1749 & 2638.1 & 0\\
\texttt{wine1} & \multicolumn{1}{r}{178} & \multicolumn{1}{r}{1276}  & 37 & 1674.0 & 0 & 37 & 1807.7 & 0 & 38 & 2247.6 & 0 & 37 & \textbf{1616.8} & 0\\
\texttt{wine2} & \multicolumn{1}{r}{178} & \multicolumn{1}{r}{1276}  & 43 & 16.6 & 0 & 43 & \textbf{0.0} & 0 & 43 & 109.7 & 0 & 43 & 16.2 & 0\\
\texttt{wine3} & \multicolumn{1}{r}{178} & \multicolumn{1}{r}{1276}  & 28 & 32.6 & 0 & 28 & 189.9 & 0 & 28 & 221.9 & 0 & 28 & \textbf{32.6} & 0\\
\texttt{yeast} & \multicolumn{1}{r}{1484} & \multicolumn{1}{r}{89}  & 366 & 3.4 & 1 & 366 & \textbf{3.0} & 1 & 366 & 29.5 & 1 & 366 & 3.4 & 1\\
\bottomrule
\end{tabular}

% \end{footnotesize}
% \end{center}
% \caption{\label{tab:fact4} Factor analysis: depth 4}
% \end{table}
%
% \begin{table}[htbp]
% \begin{center}
% \begin{footnotesize}
% \tabcolsep=2pt
% \begin{tabular}{lccrrrrrrrrrrrr}
\toprule
\multirow{2}{*}{}& && \multicolumn{3}{c}{\budalg} & \multicolumn{3}{c}{\noheuristic} & \multicolumn{3}{c}{\nopreprocessing} & \multicolumn{3}{c}{\nolb}\\
\cmidrule(rr){4-6}\cmidrule(rr){7-9}\cmidrule(rr){10-12}\cmidrule(rr){13-15}
&\multirow{1}{*}{$\#ex.$} & \multirow{1}{*}{\#feat.} &  \multicolumn{1}{c}{error} & \multicolumn{1}{c}{cpu} & \multicolumn{1}{c}{opt.} & \multicolumn{1}{c}{error} & \multicolumn{1}{c}{cpu} & \multicolumn{1}{c}{opt.} & \multicolumn{1}{c}{error} & \multicolumn{1}{c}{cpu} & \multicolumn{1}{c}{opt.} & \multicolumn{1}{c}{error} & \multicolumn{1}{c}{cpu} & \multicolumn{1}{c}{opt.} \\
\midrule

\texttt{adult\_discretized} & \multicolumn{1}{r}{30299} & \multicolumn{1}{r}{59}  & 4423 & 725.1 & 1 & 4423 & \textbf{693.3} & 1 & 4423 & 2388.0 & 1 & 4423 & 754.7 & 1\\
\texttt{anneal} & \multicolumn{1}{r}{812} & \multicolumn{1}{r}{93}  & 70 & 43.9 & 1 & 70 & \textbf{37.7} & 1 & 70 & 736.5 & 1 & 70 & 50.4 & 1\\
\texttt{audiology} & \multicolumn{1}{r}{216} & \multicolumn{1}{r}{148}  & 0 & 0.0 & 1 & 0 & 0.0 & 1 & 0 & 0.0 & 1 & 0 & 0.0 & 1\\
\texttt{australian-credit} & \multicolumn{1}{r}{653} & \multicolumn{1}{r}{125}  & 39 & 657.5 & 1 & 39 & 513.0 & 1 & 40 & \textbf{39.9} & 0 & 39 & 839.1 & 1\\
\texttt{bank} & \multicolumn{1}{r}{45211} & \multicolumn{1}{r}{9531}  & 4187 & 1151.9 & 0 & 4309 & 1112.8 & 0 & 4187 & \textbf{1073.2} & 0 & 4187 & 1205.4 & 0\\
\texttt{breast-cancer} & \multicolumn{1}{r}{683} & \multicolumn{1}{r}{89}  & 6 & 725.0 & 1 & 6 & \textbf{603.8} & 1 & 6 & 763.8 & 1 & 6 & 764.0 & 1\\
\texttt{breast-wisconsin} & \multicolumn{1}{r}{683} & \multicolumn{1}{r}{120}  & 0 & 19.9 & 1 & 0 & \textbf{15.8} & 1 & 0 & 477.7 & 1 & 0 & 31.2 & 1\\
\texttt{car} & \multicolumn{1}{r}{1728} & \multicolumn{1}{r}{21}  & 86 & \textbf{2.4} & 1 & 86 & 2.5 & 1 & 86 & 2.5 & 1 & 86 & 2.9 & 1\\
\texttt{compas\_discretized} & \multicolumn{1}{r}{6167} & \multicolumn{1}{r}{25}  & 1919 & \textbf{1.1} & 1 & 1919 & 1.1 & 1 & 1919 & 14.0 & 1 & 1919 & 1.3 & 1\\
\texttt{diabetes} & \multicolumn{1}{r}{768} & \multicolumn{1}{r}{112}  & 106 & 312.4 & 1 & 106 & \textbf{244.8} & 1 & 106 & 1425.1 & 0 & 106 & 357.1 & 1\\
\texttt{forest-fires} & \multicolumn{1}{r}{517} & \multicolumn{1}{r}{989}  & 156 & 777.0 & 0 & 157 & \textbf{60.7} & 0 & 156 & 2891.0 & 0 & 156 & 759.8 & 0\\
\texttt{german-credit} & \multicolumn{1}{r}{1000} & \multicolumn{1}{r}{112}  & 161 & 2741.0 & 1 & 161 & 2037.3 & 1 & 161 & \textbf{82.3} & 0 & 161 & 2885.0 & 1\\
\texttt{heart-cleveland} & \multicolumn{1}{r}{296} & \multicolumn{1}{r}{95}  & 7 & 93.5 & 1 & 7 & \textbf{77.8} & 1 & 7 & 1223.6 & 1 & 7 & 155.5 & 1\\
\texttt{hepatitis} & \multicolumn{1}{r}{137} & \multicolumn{1}{r}{68}  & 0 & 0.1 & 1 & 0 & 0.1 & 1 & 0 & 0.4 & 1 & 0 & 0.1 & 1\\
\texttt{hypothyroid} & \multicolumn{1}{r}{3247} & \multicolumn{1}{r}{88}  & 44 & 87.4 & 1 & 44 & \textbf{85.4} & 1 & 44 & 1538.8 & 1 & 44 & 103.2 & 1\\
\texttt{ionosphere} & \multicolumn{1}{r}{351} & \multicolumn{1}{r}{445}  & 0 & 506.0 & 1 & 0 & \textbf{444.4} & 1 & 2 & 1746.3 & 0 & 0 & 805.8 & 1\\
\texttt{kr-vs-kp} & \multicolumn{1}{r}{3196} & \multicolumn{1}{r}{73}  & 81 & \textbf{64.6} & 1 & 81 & 64.7 & 1 & 81 & 823.1 & 1 & 81 & 80.7 & 1\\
\texttt{letter} & \multicolumn{1}{r}{20000} & \multicolumn{1}{r}{224}  & \textbf{168} & 3082.5 & 0 & 172 & 2110.2 & 0 & 192 & \textbf{207.6} & 0 & 173 & 2313.2 & 0\\
\texttt{lymph} & \multicolumn{1}{r}{148} & \multicolumn{1}{r}{68}  & 0 & 0.0 & 1 & 0 & 0.0 & 1 & 0 & 0.0 & 1 & 0 & 0.0 & 1\\
\texttt{mnist\_0} & \multicolumn{1}{r}{60000} & \multicolumn{1}{r}{784}  & 1714 & 283.8 & 0 & 2075 & 1861.5 & 0 & 1714 & \textbf{240.5} & 0 & 1714 & 300.2 & 0\\
\texttt{mnist\_1} & \multicolumn{1}{r}{60000} & \multicolumn{1}{r}{784}  & 1585 & 3111.0 & 0 & 2852 & 3369.2 & 0 & 1585 & \textbf{2451.8} & 0 & 1585 & 2471.8 & 0\\
\texttt{mnist\_2} & \multicolumn{1}{r}{60000} & \multicolumn{1}{r}{784}  & 3118 & 3229.5 & 0 & 3440 & \textbf{1361.5} & 0 & 3118 & 3214.3 & 0 & 3118 & 3569.6 & 0\\
\texttt{mnist\_3} & \multicolumn{1}{r}{60000} & \multicolumn{1}{r}{784}  & 2893 & 1935.6 & 0 & 4187 & 2564.0 & 0 & 2893 & \textbf{1606.8} & 0 & 2893 & 2304.8 & 0\\
\texttt{mnist\_4} & \multicolumn{1}{r}{60000} & \multicolumn{1}{r}{784}  & 2864 & 707.9 & 0 & 3766 & 3356.9 & 0 & 2864 & \textbf{500.7} & 0 & 2864 & 1033.8 & 0\\
\texttt{mnist\_5} & \multicolumn{1}{r}{60000} & \multicolumn{1}{r}{784}  & 3138 & 2411.4 & 0 & 3241 & 2433.8 & 0 & 3138 & \textbf{2093.8} & 0 & 3138 & 2674.9 & 0\\
\texttt{mnist\_6} & \multicolumn{1}{r}{60000} & \multicolumn{1}{r}{784}  & 1485 & 2097.4 & 0 & 1989 & 2723.0 & 0 & 1485 & \textbf{1527.7} & 0 & 1485 & 2139.5 & 0\\
\texttt{mnist\_7} & \multicolumn{1}{r}{60000} & \multicolumn{1}{r}{784}  & 2532 & 1792.6 & 0 & 2919 & 2090.4 & 0 & 2532 & \textbf{1627.5} & 0 & 2532 & 1828.8 & 0\\
\texttt{mnist\_8} & \multicolumn{1}{r}{60000} & \multicolumn{1}{r}{784}  & 2547 & 2846.6 & 0 & 3022 & 2732.9 & 0 & 2547 & \textbf{2051.8} & 0 & 2547 & 3241.9 & 0\\
\texttt{mnist\_9} & \multicolumn{1}{r}{60000} & \multicolumn{1}{r}{784}  & 3352 & 1695.0 & 0 & 4066 & 2178.0 & 0 & 3352 & \textbf{1491.4} & 0 & 3352 & 1791.7 & 0\\
\texttt{mushroom} & \multicolumn{1}{r}{8124} & \multicolumn{1}{r}{119}  & 0 & 0.0 & 1 & 0 & 0.0 & 1 & 0 & 0.0 & 1 & 0 & 0.0 & 1\\
\texttt{pendigits} & \multicolumn{1}{r}{7494} & \multicolumn{1}{r}{216}  & 0 & 283.5 & 1 & 0 & 725.4 & 1 & 2 & \textbf{55.3} & 0 & 0 & 446.9 & 1\\
\texttt{primary-tumor} & \multicolumn{1}{r}{336} & \multicolumn{1}{r}{31}  & 26 & \textbf{0.4} & 1 & 26 & 0.4 & 1 & 26 & 6.7 & 1 & 26 & 0.5 & 1\\
\texttt{segment} & \multicolumn{1}{r}{2310} & \multicolumn{1}{r}{235}  & 0 & 0.0 & 1 & 0 & 0.0 & 1 & 0 & 0.0 & 1 & 0 & 0.0 & 1\\
\texttt{soybean} & \multicolumn{1}{r}{630} & \multicolumn{1}{r}{50}  & 8 & 19.6 & 1 & 8 & \textbf{15.7} & 1 & 8 & 39.8 & 1 & 8 & 25.8 & 1\\
\texttt{splice-1} & \multicolumn{1}{r}{3190} & \multicolumn{1}{r}{287}  & 101 & \textbf{23.8} & 0 & 101 & 1861.3 & 0 & 101 & 25.8 & 0 & 101 & 25.5 & 0\\
\texttt{surgical-deepnet} & \multicolumn{1}{r}{14635} & \multicolumn{1}{r}{6047}  & 2131 & 2167.6 & 0 & 2310 & 2836.1 & 0 & 2131 & \textbf{1932.0} & 0 & 2131 & 2286.5 & 0\\
\texttt{taiwan\_binarised} & \multicolumn{1}{r}{30000} & \multicolumn{1}{r}{205}  & 5200 & 104.6 & 0 & 5201 & 3306.3 & 0 & 5200 & \textbf{82.6} & 0 & 5200 & 115.3 & 0\\
\texttt{tic-tac-toe} & \multicolumn{1}{r}{958} & \multicolumn{1}{r}{27}  & 63 & 10.2 & 1 & 63 & \textbf{8.7} & 1 & 63 & 9.3 & 1 & 63 & 11.4 & 1\\
\texttt{titanic} & \multicolumn{1}{r}{887} & \multicolumn{1}{r}{333}  & 95 & 1427.7 & 0 & 95 & \textbf{1057.1} & 0 & 95 & 1464.4 & 0 & 95 & 1465.0 & 0\\
\texttt{vehicle} & \multicolumn{1}{r}{846} & \multicolumn{1}{r}{252}  & 1 & 690.2 & 0 & 1 & 3525.3 & 1 & 3 & \textbf{42.2} & 0 & 1 & 1142.4 & 0\\
\texttt{vote} & \multicolumn{1}{r}{435} & \multicolumn{1}{r}{48}  & 1 & 23.9 & 1 & 1 & \textbf{21.1} & 1 & 1 & 25.6 & 1 & 1 & 44.6 & 1\\
\texttt{weather-aus} & \multicolumn{1}{r}{142193} & \multicolumn{1}{r}{4759}  & 1735 & 419.4 & 0 & 1749 & 1835.3 & 0 & 1735 & \textbf{350.4} & 0 & 1735 & 400.5 & 0\\
\texttt{wine1} & \multicolumn{1}{r}{178} & \multicolumn{1}{r}{1276}  & 33 & 1154.5 & 0 & 33 & \textbf{950.1} & 0 & 34 & 1318.8 & 0 & 33 & 1158.5 & 0\\
\texttt{wine2} & \multicolumn{1}{r}{178} & \multicolumn{1}{r}{1276}  & 39 & 410.5 & 0 & \textbf{37} & \textbf{12.6} & 0 & 39 & 2755.6 & 0 & 39 & 409.1 & 0\\
\texttt{wine3} & \multicolumn{1}{r}{178} & \multicolumn{1}{r}{1276}  & 25 & 16.7 & 0 & 25 & 89.9 & 0 & 25 & 99.7 & 0 & 25 & \textbf{16.4} & 0\\
\texttt{yeast} & \multicolumn{1}{r}{1484} & \multicolumn{1}{r}{89}  & 313 & 139.2 & 1 & 313 & \textbf{122.6} & 1 & 313 & 2348.5 & 1 & 313 & 150.5 & 1\\
\bottomrule
\end{tabular}

% \end{footnotesize}
% \end{center}
% \caption{\label{tab:fact5} Factor analysis: depth 5}
% \end{table}
%
% \begin{table}[htbp]
% \begin{center}
% \begin{footnotesize}
% \tabcolsep=2pt
% \begin{tabular}{lccrrrrrrrr}
\toprule
\multirow{2}{*}{}& && \multicolumn{2}{c}{\budalg} & \multicolumn{2}{c}{\noheuristic} & \multicolumn{2}{c}{\nopreprocessing} & \multicolumn{2}{c}{\nolb}\\
\cmidrule(rr){4-5}\cmidrule(rr){6-7}\cmidrule(rr){8-9}\cmidrule(rr){10-11}
&\multirow{1}{*}{$\#ex.$} & \multirow{1}{*}{\#feat.} &  \multicolumn{1}{c}{error} & \multicolumn{1}{c}{cpu} & \multicolumn{1}{c}{error} & \multicolumn{1}{c}{cpu} & \multicolumn{1}{c}{error} & \multicolumn{1}{c}{cpu} & \multicolumn{1}{c}{error} & \multicolumn{1}{c}{cpu} \\
\midrule

\texttt{adult\_discretized} & \multicolumn{1}{r}{30299} & \multicolumn{1}{r}{59}  & 4281 & 1326 & 4286 & 3325 & 4281 & 458 & 4281 & 1433\\
\texttt{anneal} & \multicolumn{1}{r}{812} & \multicolumn{1}{r}{93}  & 51 & 1330$^*$ & 51 & 1114$^*$ & 68 & 396 & 51 & 1799$^*$\\
\texttt{audiology} & \multicolumn{1}{r}{216} & \multicolumn{1}{r}{148}  & 0 & 0.00$^*$ & 0 & 0.00$^*$ & 0 & 0.00$^*$ & 0 & 0.00$^*$\\
\texttt{australian-credit} & \multicolumn{1}{r}{653} & \multicolumn{1}{r}{125}  & 15 & 342 & 15 & 267 & 15 & 1207 & 15 & 511\\
\texttt{bank} & \multicolumn{1}{r}{45211} & \multicolumn{1}{r}{9531}  & 4046 & 339 & 4311 & 519 & 4046 & 308 & 4046 & 353\\
\texttt{breast-cancer} & \multicolumn{1}{r}{683} & \multicolumn{1}{r}{89}  & 1 & 3328 & 1 & 2193 & 1 & 3329 & 1 & 3557\\
\texttt{breast-wisconsin} & \multicolumn{1}{r}{683} & \multicolumn{1}{r}{120}  & 0 & 5.9$^*$ & 0 & 0.27$^*$ & 0 & 133$^*$ & 0 & 8.4$^*$\\
\texttt{car} & \multicolumn{1}{r}{1728} & \multicolumn{1}{r}{21}  & 36 & 27$^*$ & 36 & 28$^*$ & 36 & 28$^*$ & 36 & 47$^*$\\
\texttt{compas\_discretized} & \multicolumn{1}{r}{6167} & \multicolumn{1}{r}{25}  & 1887 & 17$^*$ & 1887 & 16$^*$ & 1887 & 263$^*$ & 1887 & 21$^*$\\
\texttt{diabetes} & \multicolumn{1}{r}{768} & \multicolumn{1}{r}{112}  & \textbf{60} & 2706 & 62 & 591 & 62 & 3190 & 62 & 306\\
\texttt{forest-fires} & \multicolumn{1}{r}{517} & \multicolumn{1}{r}{989}  & 132 & 1934 & 137 & 234 & 137 & 1775 & 132 & 1955\\
\texttt{german-credit} & \multicolumn{1}{r}{1000} & \multicolumn{1}{r}{112}  & 101 & 2883 & 143 & 2121 & 113 & 114 & 101 & 3356\\
\texttt{heart-cleveland} & \multicolumn{1}{r}{296} & \multicolumn{1}{r}{95}  & 0 & 0.03$^*$ & 0 & 7.2$^*$ & 0 & 0.22$^*$ & 0 & 0.03$^*$\\
\texttt{hepatitis} & \multicolumn{1}{r}{137} & \multicolumn{1}{r}{68}  & 0 & 0.00$^*$ & 0 & 0.00$^*$ & 0 & 0.00$^*$ & 0 & 0.00$^*$\\
\texttt{hypothyroid} & \multicolumn{1}{r}{3247} & \multicolumn{1}{r}{88}  & 32 & 2391$^*$ & 32 & 2216$^*$ & 33 & 616 & 32 & 3353$^*$\\
\texttt{ionosphere} & \multicolumn{1}{r}{351} & \multicolumn{1}{r}{445}  & 0 & 4.4$^*$ & 0 & 4.3$^*$ & 0 & 51$^*$ & 0 & 4.8$^*$\\
\texttt{kr-vs-kp} & \multicolumn{1}{r}{3196} & \multicolumn{1}{r}{73}  & 45 & 1694$^*$ & 45 & 1599$^*$ & 47 & 3002 & 45 & 2469$^*$\\
\texttt{letter} & \multicolumn{1}{r}{20000} & \multicolumn{1}{r}{224}  & 118 & 2186 & 173 & 2285 & 139 & 25 & 118 & 2601\\
\texttt{lymph} & \multicolumn{1}{r}{148} & \multicolumn{1}{r}{68}  & 0 & 0.00$^*$ & 0 & 0.00$^*$ & 0 & 0.00$^*$ & 0 & 0.00$^*$\\
\texttt{mnist\_0} & \multicolumn{1}{r}{60000} & \multicolumn{1}{r}{784}  & 1468 & 2513 & 1604 & 3454 & 1468 & 2094 & 1468 & 2858\\
\texttt{mnist\_1} & \multicolumn{1}{r}{60000} & \multicolumn{1}{r}{784}  & 1167 & 1875 & 2582 & 3574 & 1167 & 2132 & 1167 & 2224\\
\texttt{mnist\_2} & \multicolumn{1}{r}{60000} & \multicolumn{1}{r}{784}  & 2519 & 230 & 3337 & 2858 & 2519 & 229 & 2519 & 259\\
\texttt{mnist\_3} & \multicolumn{1}{r}{60000} & \multicolumn{1}{r}{784}  & 2486 & 2793 & 4145 & 1515 & 2486 & 2753 & 2486 & 2486\\
\texttt{mnist\_4} & \multicolumn{1}{r}{60000} & \multicolumn{1}{r}{784}  & 2180 & 3375 & 3600 & 2538 & 2180 & 3122 & 2180 & 3358\\
\texttt{mnist\_5} & \multicolumn{1}{r}{60000} & \multicolumn{1}{r}{784}  & 2930 & 1759 & 3178 & 1909 & 2930 & 1637 & 2930 & 1610\\
\texttt{mnist\_6} & \multicolumn{1}{r}{60000} & \multicolumn{1}{r}{784}  & 1278 & 2111 & 1854 & 2072 & 1278 & 2159 & 1278 & 1839\\
\texttt{mnist\_7} & \multicolumn{1}{r}{60000} & \multicolumn{1}{r}{784}  & 2074 & 2012 & 2806 & 2962 & 2074 & 1964 & 2074 & 1920\\
\texttt{mnist\_8} & \multicolumn{1}{r}{60000} & \multicolumn{1}{r}{784}  & 2060 & 806 & 3392 & 2590 & 2060 & 912 & 2060 & 797\\
\texttt{mnist\_9} & \multicolumn{1}{r}{60000} & \multicolumn{1}{r}{784}  & 2879 & 2229 & 3959 & 3209 & 2879 & 1878 & 2879 & 2447\\
\texttt{mushroom} & \multicolumn{1}{r}{8124} & \multicolumn{1}{r}{119}  & 0 & 0.00$^*$ & 0 & 0.00$^*$ & 0 & 0.00$^*$ & 0 & 0.00$^*$\\
\texttt{pendigits} & \multicolumn{1}{r}{7494} & \multicolumn{1}{r}{216}  & 0 & 0.01$^*$ & 0 & 57$^*$ & 0 & 0.06$^*$ & 0 & 0.01$^*$\\
\texttt{primary-tumor} & \multicolumn{1}{r}{336} & \multicolumn{1}{r}{31}  & 18 & 3.1$^*$ & 18 & 3.0$^*$ & 18 & 140$^*$ & 18 & 4.8$^*$\\
\texttt{segment} & \multicolumn{1}{r}{2310} & \multicolumn{1}{r}{235}  & 0 & 0.00$^*$ & 0 & 0.00$^*$ & 0 & 0.00$^*$ & 0 & 0.00$^*$\\
\texttt{soybean} & \multicolumn{1}{r}{630} & \multicolumn{1}{r}{50}  & 3 & 354$^*$ & 3 & 307$^*$ & 3 & 1186$^*$ & 3 & 753$^*$\\
\texttt{splice-1} & \multicolumn{1}{r}{3190} & \multicolumn{1}{r}{287}  & 68 & $\mathsmaller{\geq}1$h & 77 & 2384 & 68 & 3406 & 68 & 3584\\
\texttt{surgical-deepnet} & \multicolumn{1}{r}{14635} & \multicolumn{1}{r}{6047}  & 1767 & 2343 & 2314 & 2370 & 1767 & 2257 & 1767 & 2442\\
\texttt{taiwan\_binarised} & \multicolumn{1}{r}{30000} & \multicolumn{1}{r}{205}  & 5073 & 1473 & 5095 & 3416 & 5073 & 2210 & 5073 & 1459\\
\texttt{tic-tac-toe} & \multicolumn{1}{r}{958} & \multicolumn{1}{r}{27}  & 12 & 126$^*$ & 12 & 112$^*$ & 12 & 127$^*$ & 12 & 257$^*$\\
\texttt{titanic} & \multicolumn{1}{r}{887} & \multicolumn{1}{r}{333}  & 78 & 1234 & 88 & 3045 & 78 & 1299 & 78 & 1327\\
\texttt{vehicle} & \multicolumn{1}{r}{846} & \multicolumn{1}{r}{252}  & 0 & 0.08$^*$ & 0 & 2783$^*$ & 0 & 0.44$^*$ & 0 & 0.08$^*$\\
\texttt{vote} & \multicolumn{1}{r}{435} & \multicolumn{1}{r}{48}  & 0 & 0.00$^*$ & 0 & 0.05$^*$ & 0 & 0.00$^*$ & 0 & 0.00$^*$\\
\texttt{weather-aus} & \multicolumn{1}{r}{142193} & \multicolumn{1}{r}{4759}  & 1713 & 418 & 1748 & 1696 & 1713 & 384 & 1713 & 412\\
\texttt{wine1} & \multicolumn{1}{r}{178} & \multicolumn{1}{r}{1276}  & 31 & 2113 & \textbf{30} & 1137 & 32 & 1017 & 31 & 2178\\
\texttt{wine2} & \multicolumn{1}{r}{178} & \multicolumn{1}{r}{1276}  & 34 & 44 & 34 & 12 & 34 & 282 & 34 & 44\\
\texttt{wine3} & \multicolumn{1}{r}{178} & \multicolumn{1}{r}{1276}  & 22 & 93 & 22 & 714 & 22 & 604 & 22 & 91\\
\texttt{yeast} & \multicolumn{1}{r}{1484} & \multicolumn{1}{r}{89}  & 245 & 388 & 245 & 2099 & 272 & 407 & 245 & 455\\
\bottomrule
\end{tabular}

% \end{footnotesize}
% \end{center}
% \caption{\label{tab:fact6} Factor analysis: depth 6}
% \end{table}
%
% \begin{table}[htbp]
% \begin{center}
% \begin{footnotesize}
% \tabcolsep=2pt
% \begin{tabular}{lccrrrrrrrr}
\toprule
\multirow{2}{*}{}& && \multicolumn{2}{c}{\budalg} & \multicolumn{2}{c}{\noheuristic} & \multicolumn{2}{c}{\nopreprocessing} & \multicolumn{2}{c}{\nolb}\\
\cmidrule(rr){4-5}\cmidrule(rr){6-7}\cmidrule(rr){8-9}\cmidrule(rr){10-11}
&\multirow{1}{*}{$\#ex.$} & \multirow{1}{*}{\#feat.} &  \multicolumn{1}{c}{error} & \multicolumn{1}{c}{cpu} & \multicolumn{1}{c}{error} & \multicolumn{1}{c}{cpu} & \multicolumn{1}{c}{error} & \multicolumn{1}{c}{cpu} & \multicolumn{1}{c}{error} & \multicolumn{1}{c}{cpu} \\
\midrule

\texttt{adult\_discretized} & \multicolumn{1}{r}{30299} & \multicolumn{1}{r}{59}  & 4191 & 534 & 4203 & 686 & \textbf{4162} & 2418 & 4191 & 553\\
\texttt{anneal} & \multicolumn{1}{r}{812} & \multicolumn{1}{r}{93}  & \textbf{41} & 3036 & 49 & 2818 & 58 & 272 & 50 & 232\\
\texttt{audiology} & \multicolumn{1}{r}{216} & \multicolumn{1}{r}{148}  & 0 & 0.00$^*$ & 0 & 0.00$^*$ & 0 & 0.00$^*$ & 0 & 0.00$^*$\\
\texttt{australian-credit} & \multicolumn{1}{r}{653} & \multicolumn{1}{r}{125}  & 0 & 101$^*$ & 0 & 477$^*$ & 0 & 1002$^*$ & 0 & 153$^*$\\
\texttt{bank} & \multicolumn{1}{r}{45211} & \multicolumn{1}{r}{9531}  & 3844 & 2369 & 4303 & 252 & 3844 & 2351 & 3844 & 2460\\
\texttt{breast-cancer} & \multicolumn{1}{r}{683} & \multicolumn{1}{r}{89}  & 0 & 1007$^*$ & 0 & 824$^*$ & 0 & 1024$^*$ & 0 & 1194$^*$\\
\texttt{breast-wisconsin} & \multicolumn{1}{r}{683} & \multicolumn{1}{r}{120}  & 0 & 0.02$^*$ & 0 & 0.23$^*$ & 0 & 0.33$^*$ & 0 & 0.03$^*$\\
\texttt{car} & \multicolumn{1}{r}{1728} & \multicolumn{1}{r}{21}  & 11 & 231$^*$ & 11 & 256$^*$ & 11 & 233$^*$ & 11 & 627$^*$\\
\texttt{compas\_discretized} & \multicolumn{1}{r}{6167} & \multicolumn{1}{r}{25}  & 1852 & 198$^*$ & 1852 & 184$^*$ & 1852 & 2030 & 1852 & 299$^*$\\
\texttt{diabetes} & \multicolumn{1}{r}{768} & \multicolumn{1}{r}{112}  & 21 & 827 & 27 & 238 & 26 & 3164 & 21 & 1324\\
\texttt{forest-fires} & \multicolumn{1}{r}{517} & \multicolumn{1}{r}{989}  & 146 & 125 & 142 & 140 & \textbf{132} & 1346 & 146 & 124\\
\texttt{german-credit} & \multicolumn{1}{r}{1000} & \multicolumn{1}{r}{112}  & 56 & 1192 & 117 & 2789 & 56 & 2472 & 56 & 1446\\
\texttt{heart-cleveland} & \multicolumn{1}{r}{296} & \multicolumn{1}{r}{95}  & 0 & 0.00$^*$ & 0 & 3.0$^*$ & 0 & 0.03$^*$ & 0 & 0.00$^*$\\
\texttt{hepatitis} & \multicolumn{1}{r}{137} & \multicolumn{1}{r}{68}  & 0 & 0.00$^*$ & 0 & 0.00$^*$ & 0 & 0.00$^*$ & 0 & 0.00$^*$\\
\texttt{hypothyroid} & \multicolumn{1}{r}{3247} & \multicolumn{1}{r}{88}  & \textbf{22} & 3478 & 23 & 147 & 27 & 113 & 23 & 171\\
\texttt{ionosphere} & \multicolumn{1}{r}{351} & \multicolumn{1}{r}{445}  & 0 & 0.07$^*$ & 0 & 0.07$^*$ & 0 & 0.49$^*$ & 0 & 0.07$^*$\\
\texttt{kr-vs-kp} & \multicolumn{1}{r}{3196} & \multicolumn{1}{r}{73}  & 18 & 2550 & 18 & 1423 & 34 & 3090 & 21 & 1756\\
\texttt{letter} & \multicolumn{1}{r}{20000} & \multicolumn{1}{r}{224}  & 68 & 177 & 168 & 2143 & 70 & 3525 & 68 & 193\\
\texttt{lymph} & \multicolumn{1}{r}{148} & \multicolumn{1}{r}{68}  & 0 & 0.00$^*$ & 0 & 0.01$^*$ & 0 & 0.00$^*$ & 0 & 0.00$^*$\\
\texttt{mnist\_0} & \multicolumn{1}{r}{60000} & \multicolumn{1}{r}{784}  & 1107 & 2895 & 1556 & 1539 & 1107 & 2983 & 1107 & 2735\\
\texttt{mnist\_1} & \multicolumn{1}{r}{60000} & \multicolumn{1}{r}{784}  & 810 & 510 & 2657 & 1574 & 810 & 496 & 810 & 543\\
\texttt{mnist\_2} & \multicolumn{1}{r}{60000} & \multicolumn{1}{r}{784}  & 2133 & 2575 & 3288 & 3471 & 2133 & 2383 & 2133 & 2285\\
\texttt{mnist\_3} & \multicolumn{1}{r}{60000} & \multicolumn{1}{r}{784}  & 1843 & 3188 & 4120 & 1269 & 1843 & 2882 & 1843 & 3189\\
\texttt{mnist\_4} & \multicolumn{1}{r}{60000} & \multicolumn{1}{r}{784}  & 1727 & 3234 & 3423 & 2208 & 1727 & 3076 & 1727 & 3401\\
\texttt{mnist\_5} & \multicolumn{1}{r}{60000} & \multicolumn{1}{r}{784}  & 2830 & 2556 & 3112 & 2791 & 2830 & 2472 & 2830 & 2299\\
\texttt{mnist\_6} & \multicolumn{1}{r}{60000} & \multicolumn{1}{r}{784}  & 1208 & 2902 & 1787 & 3465 & 1208 & 2665 & 1208 & 2244\\
\texttt{mnist\_7} & \multicolumn{1}{r}{60000} & \multicolumn{1}{r}{784}  & 1659 & 2853 & 2666 & 3428 & 1659 & 2757 & 1659 & 2897\\
\texttt{mnist\_8} & \multicolumn{1}{r}{60000} & \multicolumn{1}{r}{784}  & 1566 & 3230 & 2928 & 2701 & 1566 & 2800 & 1566 & 3090\\
\texttt{mnist\_9} & \multicolumn{1}{r}{60000} & \multicolumn{1}{r}{784}  & 2550 & 1863 & 4420 & 1577 & 2550 & 1905 & 2550 & 2016\\
\texttt{mushroom} & \multicolumn{1}{r}{8124} & \multicolumn{1}{r}{119}  & 0 & 0.00$^*$ & 0 & 0.00$^*$ & 0 & 0.00$^*$ & 0 & 0.00$^*$\\
\texttt{pendigits} & \multicolumn{1}{r}{7494} & \multicolumn{1}{r}{216}  & 0 & 0.00$^*$ & 0 & 3.5$^*$ & 0 & 0.00$^*$ & 0 & 0.00$^*$\\
\texttt{primary-tumor} & \multicolumn{1}{r}{336} & \multicolumn{1}{r}{31}  & 16 & 18$^*$ & 16 & 17$^*$ & 16 & 2866$^*$ & 16 & 39$^*$\\
\texttt{segment} & \multicolumn{1}{r}{2310} & \multicolumn{1}{r}{235}  & 0 & 0.00$^*$ & 0 & 0.00$^*$ & 0 & 0.00$^*$ & 0 & 0.00$^*$\\
\texttt{soybean} & \multicolumn{1}{r}{630} & \multicolumn{1}{r}{50}  & 2 & 19$^*$ & 2 & 6.1$^*$ & 2 & 729 & 2 & 32$^*$\\
\texttt{splice-1} & \multicolumn{1}{r}{3190} & \multicolumn{1}{r}{287}  & 29 & 3484 & 46 & 3380 & 29 & 3575 & 29 & 3408\\
\texttt{surgical-deepnet} & \multicolumn{1}{r}{14635} & \multicolumn{1}{r}{6047}  & 1647 & 1248 & 2246 & 3102 & 1647 & 1086 & 1647 & 1288\\
\texttt{taiwan\_binarised} & \multicolumn{1}{r}{30000} & \multicolumn{1}{r}{205}  & 4896 & 1958 & 5016 & 2961 & 4909 & 1426 & 4896 & 2055\\
\texttt{tic-tac-toe} & \multicolumn{1}{r}{958} & \multicolumn{1}{r}{27}  & 0 & 32$^*$ & 0 & 83$^*$ & 0 & 31$^*$ & 0 & 100$^*$\\
\texttt{titanic} & \multicolumn{1}{r}{887} & \multicolumn{1}{r}{333}  & 72 & 442 & 78 & 2696 & 72 & 471 & 72 & 500\\
\texttt{vehicle} & \multicolumn{1}{r}{846} & \multicolumn{1}{r}{252}  & 0 & 0.09$^*$ & 0 & 196$^*$ & 0 & 0.66$^*$ & 0 & 0.10$^*$\\
\texttt{vote} & \multicolumn{1}{r}{435} & \multicolumn{1}{r}{48}  & 0 & 0.00$^*$ & 0 & 0.04$^*$ & 0 & 0.00$^*$ & 0 & 0.00$^*$\\
\texttt{weather-aus} & \multicolumn{1}{r}{142193} & \multicolumn{1}{r}{4759}  & 1685 & 2048 & 1747 & 1685 & 1685 & 1948 & 1685 & 2083\\
\texttt{wine1} & \multicolumn{1}{r}{178} & \multicolumn{1}{r}{1276}  & 28 & 892 & 28 & 2666 & 29 & 487 & 28 & 892\\
\texttt{wine2} & \multicolumn{1}{r}{178} & \multicolumn{1}{r}{1276}  & 31 & 28 & 31 & 23 & 31 & 168 & 31 & 28\\
\texttt{wine3} & \multicolumn{1}{r}{178} & \multicolumn{1}{r}{1276}  & 21 & 524 & 21 & 1062 & \textbf{20} & 296 & 21 & 531\\
\texttt{yeast} & \multicolumn{1}{r}{1484} & \multicolumn{1}{r}{89}  & \textbf{182} & 3558 & 234 & 1611 & 210 & 1191 & 203 & 410\\
\bottomrule
\end{tabular}

% \end{footnotesize}
% \end{center}
% \caption{\label{tab:fact7} Factor analysis: depth 7}
% \end{table}
%
% \begin{table}[htbp]
% \begin{center}
% \begin{footnotesize}
% \tabcolsep=2pt
% \begin{tabular}{lccrrrrrrrr}
\toprule
\multirow{2}{*}{}& && \multicolumn{2}{c}{\budalg} & \multicolumn{2}{c}{\noheuristic} & \multicolumn{2}{c}{\nopreprocessing} & \multicolumn{2}{c}{\nolb}\\
\cmidrule(rr){4-5}\cmidrule(rr){6-7}\cmidrule(rr){8-9}\cmidrule(rr){10-11}
&\multirow{1}{*}{$|\allex|$} & \multirow{1}{*}{$|\features|$} &  \multicolumn{1}{c}{error} & \multicolumn{1}{c}{cpu} & \multicolumn{1}{c}{error} & \multicolumn{1}{c}{cpu} & \multicolumn{1}{c}{error} & \multicolumn{1}{c}{cpu} & \multicolumn{1}{c}{error} & \multicolumn{1}{c}{cpu} \\
\midrule

\texttt{hepatitis} & 0 & 0.00$^*$ & 0 & 0.00$^*$ & 0 & 0.00$^*$ & 0 & 0.00$^*$\\
\texttt{lymph} & 0 & 0.00$^*$ & 0 & 0.00$^*$ & 0 & 0.00$^*$ & 0 & 0.00$^*$\\
\texttt{wine1} & 27 & 43 & \textbf{25} & 1520 & 27 & 189 & 27 & 44\\
\texttt{wine2} & 30 & 605 & \textbf{28} & 0.04 & 31 & 592 & 30 & 603\\
\texttt{wine3} & 20 & 406 & 19 & 951 & 19 & 2406 & 20 & 408\\
\texttt{audiology} & 0 & 0.00$^*$ & 0 & 0.00$^*$ & 0 & 0.00$^*$ & 0 & 0.00$^*$\\
\texttt{heart-cleveland} & 0 & 0.00$^*$ & 0 & 1.7$^*$ & 0 & 0.00$^*$ & 0 & 0.00$^*$\\
\texttt{primary-tumor} & 15 & 0.00$^*$ & 15 & 0.55$^*$ & 15 & 25 & 15 & 0.01$^*$\\
\texttt{ionosphere} & 0 & 0.00$^*$ & 0 & 0.08$^*$ & 0 & 0.02$^*$ & 0 & 0.00$^*$\\
\texttt{vote} & 0 & 0.00$^*$ & 0 & 0.03$^*$ & 0 & 0.00$^*$ & 0 & 0.00$^*$\\
\texttt{forest-fires} & 137 & 190 & 133 & 108 & 133 & 326 & 137 & 195\\
\texttt{soybean} & 2 & 0.13$^*$ & 2 & 0.20$^*$ & 2 & 1.8 & 2 & 0.18$^*$\\
\texttt{australian-credit} & 0 & 13$^*$ & 0 & 47$^*$ & 0 & 144$^*$ & 0 & 17$^*$\\
\texttt{breast-cancer} & 0 & 25$^*$ & 0 & 22$^*$ & 0 & 27$^*$ & 0 & 27$^*$\\
\texttt{breast-wisconsin} & 0 & 0.00$^*$ & 0 & 0.00$^*$ & 0 & 0.00$^*$ & 0 & 0.00$^*$\\
\texttt{diabetes} & 0 & 220$^*$ & 16 & 2998 & 4 & 2216 & 0 & 443$^*$\\
\texttt{anneal} & \textbf{36} & 2221 & 47 & 1669 & 41 & 560 & 40 & 296\\
\texttt{vehicle} & 0 & 0.00$^*$ & 0 & 151$^*$ & 0 & 0.04$^*$ & 0 & 0.00$^*$\\
\texttt{titanic} & 64 & 1500 & 77 & 1409 & 75 & 18 & 64 & 1756\\
\texttt{tic-tac-toe} & 0 & 0.00$^*$ & 0 & 0.00$^*$ & 0 & 0.00$^*$ & 0 & 0.00$^*$\\
\texttt{german-credit} & 23 & 2235 & 107 & 3373 & 34 & 1315 & 23 & 2836\\
\texttt{yeast} & 132 & 1682 & 159 & 3450 & 166 & 969 & 132 & 2194\\
\texttt{car} & 0 & 404$^*$ & 0 & 536$^*$ & 0 & 407$^*$ & 0 & 1806$^*$\\
\texttt{segment} & 0 & 0.00$^*$ & 0 & 0.00$^*$ & 0 & 0.00$^*$ & 0 & 0.00$^*$\\
\texttt{splice-1} & 24 & 0.47 & 38 & 1642 & 24 & 0.47 & 24 & 0.50\\
\texttt{kr-vs-kp} & 13 & 2736 & \textbf{8} & 456 & 16 & 1725 & 14 & 1384\\
\texttt{hypothyroid} & 17 & 99$^*$ & 17 & 300$^*$ & 23 & $\mathsmaller{\geq}1$h & 17 & 181$^*$\\
\texttt{compas\_discretized} & 1832 & 1462$^*$ & 1832 & 1255$^*$ & 1835 & 2508 & 1832 & 3290$^*$\\
\texttt{pendigits} & 0 & 0.00$^*$ & 0 & 0.59$^*$ & 0 & 0.00$^*$ & 0 & 0.00$^*$\\
\texttt{mushroom} & 0 & 0.00$^*$ & 0 & 0.00$^*$ & 0 & 0.00$^*$ & 0 & 0.00$^*$\\
\texttt{surgical-deepnet} & 1297 & 66 & 1979 & 2061 & 1297 & 62 & 1297 & 68\\
\texttt{letter} & 24 & 297 & 158 & 3358 & 25 & 962 & 24 & 362\\
\texttt{taiwan\_binarised} & 4727 & 3246 & 4990 & 2579 & 4727 & 3521 & 4727 & 3349\\
\texttt{adult\_discretized} & 4148 & 450 & 4219 & 475 & \textbf{4058} & 2826 & 4148 & 477\\
\texttt{bank} & 3709 & 1815 & 4246 & 3454 & 3709 & 1704 & 3709 & 1814\\
\texttt{mnist\_8} & 1192 & 3087 & 2764 & 886 & \textbf{1183} & 3445 & 1192 & 3142\\
\texttt{mnist\_9} & 2186 & 2934 & 4681 & 1054 & 2187 & 2909 & 2186 & 3034\\
\texttt{mnist\_0} & 788 & 1983 & 1913 & 1368 & 788 & 1795 & 788 & 2158\\
\texttt{mnist\_6} & 1203 & 109 & 1563 & 1368 & 1203 & 104 & 1203 & 109\\
\texttt{mnist\_5} & 2519 & 1518 & 3082 & 2519 & 2519 & 1007 & 2519 & 1455\\
\texttt{mnist\_3} & 1436 & 1680 & 4112 & 829 & 1436 & 1675 & 1436 & 1822\\
\texttt{mnist\_2} & 1857 & 291 & 3131 & 1828 & 1856 & 3536 & 1856 & 3582\\
\texttt{mnist\_4} & 1279 & 1522 & 3251 & 1375 & 1279 & 1222 & 1279 & 1324\\
\texttt{mnist\_7} & 1429 & 347 & 2520 & 2431 & 1429 & 343 & 1429 & 365\\
\texttt{mnist\_1} & 565 & 1177 & 2282 & 1805 & 565 & 1189 & 565 & 1327\\
\texttt{weather-aus} & 1657 & 2060 & 1746 & 1698 & \textbf{1656} & 2025 & 1657 & 2162\\
\bottomrule
\end{tabular}

% \end{footnotesize}
% \end{center}
% \caption{\label{tab:fact8} Factor analysis: depth 8}
% \end{table}
%
% \begin{table}[htbp]
% \begin{center}
% \begin{footnotesize}
% \tabcolsep=2pt
% \begin{tabular}{lccrrrrrrrrrrrr}
\toprule
\multirow{2}{*}{}& && \multicolumn{3}{c}{\budalg} & \multicolumn{3}{c}{\noheuristic} & \multicolumn{3}{c}{\nopreprocessing} & \multicolumn{3}{c}{\nolb}\\
\cmidrule(rr){4-6}\cmidrule(rr){7-9}\cmidrule(rr){10-12}\cmidrule(rr){13-15}
&\multirow{1}{*}{$\#ex.$} & \multirow{1}{*}{\#feat.} &  \multicolumn{1}{c}{error} & \multicolumn{1}{c}{cpu} & \multicolumn{1}{c}{opt.} & \multicolumn{1}{c}{error} & \multicolumn{1}{c}{cpu} & \multicolumn{1}{c}{opt.} & \multicolumn{1}{c}{error} & \multicolumn{1}{c}{cpu} & \multicolumn{1}{c}{opt.} & \multicolumn{1}{c}{error} & \multicolumn{1}{c}{cpu} & \multicolumn{1}{c}{opt.} \\
\midrule

\texttt{adult\_discretized} & \multicolumn{1}{r}{30299} & \multicolumn{1}{r}{59}  & 3999 & 668.9 & 0 & 4108 & 1911.3 & 0 & \textbf{3969} & \textbf{109.2} & 0 & 3999 & 703.8 & 0\\
\texttt{anneal} & \multicolumn{1}{r}{812} & \multicolumn{1}{r}{93}  & 35 & 339.9 & 0 & 36 & 1433.0 & 0 & 39 & \textbf{141.2} & 0 & 35 & 532.9 & 0\\
\texttt{audiology} & \multicolumn{1}{r}{216} & \multicolumn{1}{r}{148}  & 0 & 0.0 & 1 & 0 & 0.0 & 1 & 0 & 0.0 & 1 & 0 & 0.0 & 1\\
\texttt{australian-credit} & \multicolumn{1}{r}{653} & \multicolumn{1}{r}{125}  & 0 & \textbf{1.8} & 1 & 0 & 8.5 & 1 & 0 & 10.6 & 1 & 0 & 2.3 & 1\\
\texttt{bank} & \multicolumn{1}{r}{45211} & \multicolumn{1}{r}{9531}  & 3493 & 944.3 & 0 & 4194 & 3395.9 & 0 & 3493 & \textbf{923.4} & 0 & 3493 & 948.8 & 0\\
\texttt{breast-cancer} & \multicolumn{1}{r}{683} & \multicolumn{1}{r}{89}  & 0 & 10.1 & 1 & 0 & 10.9 & 1 & 0 & 9.9 & 1 & 0 & \textbf{9.9} & 1\\
\texttt{breast-wisconsin} & \multicolumn{1}{r}{683} & \multicolumn{1}{r}{120}  & 0 & 0.0 & 1 & 0 & 0.0 & 1 & 0 & 0.0 & 1 & 0 & 0.0 & 1\\
\texttt{car} & \multicolumn{1}{r}{1728} & \multicolumn{1}{r}{21}  & 0 & \textbf{76.7} & 1 & 0 & 88.0 & 1 & 0 & 78.4 & 1 & 0 & 305.2 & 1\\
\texttt{compas\_discretized} & \multicolumn{1}{r}{6167} & \multicolumn{1}{r}{25}  & 1828 & \textbf{205.4} & 1 & 1828 & 310.0 & 1 & 1829 & 702.5 & 0 & 1828 & 664.1 & 1\\
\texttt{diabetes} & \multicolumn{1}{r}{768} & \multicolumn{1}{r}{112}  & 0 & \textbf{2.6} & 1 & 9 & 2204.9 & 0 & 0 & 60.8 & 1 & 0 & 4.0 & 1\\
\texttt{forest-fires} & \multicolumn{1}{r}{517} & \multicolumn{1}{r}{989}  & 133 & \textbf{7.7} & 0 & 137 & 87.2 & 0 & \textbf{132} & 18.5 & 0 & 133 & 7.9 & 0\\
\texttt{german-credit} & \multicolumn{1}{r}{1000} & \multicolumn{1}{r}{112}  & \textbf{4} & 3512.9 & 0 & 74 & 3281.4 & 0 & 15 & 3438.2 & 0 & 16 & \textbf{979.6} & 0\\
\texttt{heart-cleveland} & \multicolumn{1}{r}{296} & \multicolumn{1}{r}{95}  & 0 & 0.0 & 1 & 0 & 0.5 & 1 & 0 & 0.0 & 1 & 0 & 0.0 & 1\\
\texttt{hepatitis} & \multicolumn{1}{r}{137} & \multicolumn{1}{r}{68}  & 0 & 0.0 & 1 & 0 & 0.0 & 1 & 0 & 0.0 & 1 & 0 & 0.0 & 1\\
\texttt{hypothyroid} & \multicolumn{1}{r}{3247} & \multicolumn{1}{r}{88}  & 17 & \textbf{76.3} & 1 & 17 & 102.6 & 1 & 18 & 2583.0 & 0 & 17 & 139.1 & 1\\
\texttt{ionosphere} & \multicolumn{1}{r}{351} & \multicolumn{1}{r}{445}  & 0 & 0.0 & 1 & 0 & 0.0 & 1 & 0 & 0.0 & 1 & 0 & 0.0 & 1\\
\texttt{kr-vs-kp} & \multicolumn{1}{r}{3196} & \multicolumn{1}{r}{73}  & 5 & 376.5 & 0 & \textbf{1} & 1510.0 & 0 & 8 & \textbf{307.2} & 0 & 5 & 710.5 & 0\\
\texttt{letter} & \multicolumn{1}{r}{20000} & \multicolumn{1}{r}{224}  & 2 & \textbf{366.0} & 0 & 139 & 2398.0 & 0 & 10 & 810.2 & 0 & 2 & 494.7 & 0\\
\texttt{lymph} & \multicolumn{1}{r}{148} & \multicolumn{1}{r}{68}  & 0 & 0.0 & 1 & 0 & 0.0 & 1 & 0 & 0.0 & 1 & 0 & 0.0 & 1\\
\texttt{mnist\_0} & \multicolumn{1}{r}{60000} & \multicolumn{1}{r}{784}  & 595 & 1511.1 & 0 & 1798 & \textbf{786.3} & 0 & 595 & 1304.6 & 0 & 595 & 1744.2 & 0\\
\texttt{mnist\_1} & \multicolumn{1}{r}{60000} & \multicolumn{1}{r}{784}  & 465 & 500.3 & 0 & 1735 & 1440.9 & 0 & 466 & \textbf{428.7} & 0 & 465 & 472.6 & 0\\
\texttt{mnist\_2} & \multicolumn{1}{r}{60000} & \multicolumn{1}{r}{784}  & 1682 & 2193.2 & 0 & 3028 & \textbf{1636.4} & 0 & 1682 & 1728.0 & 0 & 1682 & 2076.5 & 0\\
\texttt{mnist\_3} & \multicolumn{1}{r}{60000} & \multicolumn{1}{r}{784}  & 1213 & \textbf{543.0} & 0 & 4105 & 2453.7 & 0 & 1374 & 3470.7 & 0 & 1213 & 543.3 & 0\\
\texttt{mnist\_4} & \multicolumn{1}{r}{60000} & \multicolumn{1}{r}{784}  & 908 & 3042.7 & 0 & 4478 & \textbf{1696.2} & 0 & 908 & 3029.3 & 0 & 908 & 3481.9 & 0\\
\texttt{mnist\_5} & \multicolumn{1}{r}{60000} & \multicolumn{1}{r}{784}  & 2424 & \textbf{611.7} & 0 & 3419 & 697.0 & 0 & 2424 & 645.5 & 0 & 2424 & 899.4 & 0\\
\texttt{mnist\_6} & \multicolumn{1}{r}{60000} & \multicolumn{1}{r}{784}  & 1138 & 1606.0 & 0 & 1444 & 3144.8 & 0 & 1138 & \textbf{1482.9} & 0 & 1138 & 1791.3 & 0\\
\texttt{mnist\_7} & \multicolumn{1}{r}{60000} & \multicolumn{1}{r}{784}  & 1261 & 1792.2 & 0 & 2331 & 2675.2 & 0 & 1261 & \textbf{1761.1} & 0 & 1261 & 1880.2 & 0\\
\texttt{mnist\_8} & \multicolumn{1}{r}{60000} & \multicolumn{1}{r}{784}  & 849 & 1303.7 & 0 & 2602 & \textbf{941.4} & 0 & 849 & 1268.2 & 0 & 849 & 1347.0 & 0\\
\texttt{mnist\_9} & \multicolumn{1}{r}{60000} & \multicolumn{1}{r}{784}  & 1829 & 2522.5 & 0 & 4291 & \textbf{1571.8} & 0 & 1829 & 2809.8 & 0 & 1829 & 2979.0 & 0\\
\texttt{mushroom} & \multicolumn{1}{r}{8124} & \multicolumn{1}{r}{119}  & 0 & 0.0 & 1 & 0 & 0.0 & 1 & 0 & 0.0 & 1 & 0 & 0.0 & 1\\
\texttt{pendigits} & \multicolumn{1}{r}{7494} & \multicolumn{1}{r}{216}  & 0 & 0.0 & 1 & 0 & 0.0 & 1 & 0 & 0.0 & 1 & 0 & 0.0 & 1\\
\texttt{primary-tumor} & \multicolumn{1}{r}{336} & \multicolumn{1}{r}{31}  & 15 & 0.0 & 1 & 15 & 0.0 & 1 & 15 & 0.3 & 0 & 15 & 0.0 & 1\\
\texttt{segment} & \multicolumn{1}{r}{2310} & \multicolumn{1}{r}{235}  & 0 & 0.0 & 1 & 0 & 0.0 & 1 & 0 & 0.0 & 1 & 0 & 0.0 & 1\\
\texttt{soybean} & \multicolumn{1}{r}{630} & \multicolumn{1}{r}{50}  & 2 & \textbf{0.0} & 1 & 2 & 0.4 & 1 & 2 & 0.2 & 0 & 2 & 0.1 & 1\\
\texttt{splice-1} & \multicolumn{1}{r}{3190} & \multicolumn{1}{r}{287}  & 12 & \textbf{212.3} & 0 & 20 & 1695.6 & 0 & 12 & 212.6 & 0 & 12 & 215.5 & 0\\
\texttt{surgical-deepnet} & \multicolumn{1}{r}{14635} & \multicolumn{1}{r}{6047}  & 1127 & 53.0 & 0 & 1956 & 2313.6 & 0 & 1127 & \textbf{52.3} & 0 & 1127 & 53.2 & 0\\
\texttt{taiwan\_binarised} & \multicolumn{1}{r}{30000} & \multicolumn{1}{r}{205}  & 4427 & 3484.1 & 0 & 4941 & \textbf{1856.1} & 0 & \textbf{4423} & 3557.2 & 0 & 4432 & 3185.4 & 0\\
\texttt{tic-tac-toe} & \multicolumn{1}{r}{958} & \multicolumn{1}{r}{27}  & 0 & 0.0 & 1 & 0 & 0.0 & 1 & 0 & 0.0 & 1 & 0 & 0.0 & 1\\
\texttt{titanic} & \multicolumn{1}{r}{887} & \multicolumn{1}{r}{333}  & 47 & 2117.0 & 0 & 66 & 2898.1 & 0 & 50 & \textbf{1175.7} & 0 & 47 & 2568.1 & 0\\
\texttt{vehicle} & \multicolumn{1}{r}{846} & \multicolumn{1}{r}{252}  & 0 & 0.0 & 1 & 0 & 69.5 & 1 & 0 & 0.0 & 1 & 0 & 0.0 & 1\\
\texttt{vote} & \multicolumn{1}{r}{435} & \multicolumn{1}{r}{48}  & 0 & 0.0 & 1 & 0 & 0.0 & 1 & 0 & 0.0 & 1 & 0 & 0.0 & 1\\
\texttt{weather-aus} & \multicolumn{1}{r}{142193} & \multicolumn{1}{r}{4759}  & 1638 & 2359.2 & 0 & 1739 & \textbf{1599.4} & 0 & \textbf{1637} & 2730.0 & 0 & 1638 & 2264.0 & 0\\
\texttt{wine1} & \multicolumn{1}{r}{178} & \multicolumn{1}{r}{1276}  & 24 & 1540.8 & 0 & \textbf{22} & 1439.6 & 0 & 25 & \textbf{84.7} & 0 & 24 & 1536.2 & 0\\
\texttt{wine2} & \multicolumn{1}{r}{178} & \multicolumn{1}{r}{1276}  & 27 & 505.1 & 0 & \textbf{24} & \textbf{27.7} & 0 & 27 & 3366.7 & 0 & 27 & 513.7 & 0\\
\texttt{wine3} & \multicolumn{1}{r}{178} & \multicolumn{1}{r}{1276}  & 18 & \textbf{316.8} & 0 & \textbf{17} & 418.6 & 0 & 18 & 2032.2 & 0 & 18 & 319.6 & 0\\
\texttt{yeast} & \multicolumn{1}{r}{1484} & \multicolumn{1}{r}{89}  & \textbf{68} & 3460.9 & 0 & 72 & 2721.6 & 0 & 119 & \textbf{810.3} & 0 & 75 & 2825.7 & 0\\
\bottomrule
\end{tabular}

% \end{footnotesize}
% \end{center}
% \caption{\label{tab:fact9} Factor analysis: depth 9}
% \end{table}
%
% \begin{table}[htbp]
% \begin{center}
% \begin{footnotesize}
% \tabcolsep=2pt
% \begin{tabular}{lccrrrrrrrrrrrr}
\toprule
\multirow{2}{*}{}& && \multicolumn{3}{c}{\budalg} & \multicolumn{3}{c}{\noheuristic} & \multicolumn{3}{c}{\nopreprocessing} & \multicolumn{3}{c}{\nolb}\\
\cmidrule(rr){4-6}\cmidrule(rr){7-9}\cmidrule(rr){10-12}\cmidrule(rr){13-15}
&\multirow{1}{*}{$\#ex.$} & \multirow{1}{*}{\#feat.} &  \multicolumn{1}{c}{error} & \multicolumn{1}{c}{cpu} & \multicolumn{1}{c}{opt.} & \multicolumn{1}{c}{error} & \multicolumn{1}{c}{cpu} & \multicolumn{1}{c}{opt.} & \multicolumn{1}{c}{error} & \multicolumn{1}{c}{cpu} & \multicolumn{1}{c}{opt.} & \multicolumn{1}{c}{error} & \multicolumn{1}{c}{cpu} & \multicolumn{1}{c}{opt.} \\
\midrule

\texttt{adult\_discretized} & \multicolumn{1}{r}{30299} & \multicolumn{1}{r}{59}  & 3841 & \textbf{2632.1} & 0 & 4119 & 3074.9 & 0 & \textbf{3775} & 2993.9 & 0 & 3841 & 2988.0 & 0\\
\texttt{anneal} & \multicolumn{1}{r}{812} & \multicolumn{1}{r}{93}  & 34 & \textbf{22.6} & 1 & 36 & 1985.8 & 0 & 36 & 661.2 & 0 & 34 & 32.0 & 1\\
\texttt{audiology} & \multicolumn{1}{r}{216} & \multicolumn{1}{r}{148}  & 0 & 0.0 & 1 & 0 & 0.0 & 1 & 0 & 0.0 & 1 & 0 & 0.0 & 1\\
\texttt{australian-credit} & \multicolumn{1}{r}{653} & \multicolumn{1}{r}{125}  & 0 & 0.0 & 1 & 0 & 0.1 & 1 & 0 & 0.3 & 1 & 0 & 0.0 & 1\\
\texttt{bank} & \multicolumn{1}{r}{45211} & \multicolumn{1}{r}{9531}  & 3242 & 800.3 & 0 & 4200 & \textbf{20.3} & 0 & 3245 & 851.0 & 0 & 3242 & 844.9 & 0\\
\texttt{breast-cancer} & \multicolumn{1}{r}{683} & \multicolumn{1}{r}{89}  & 0 & 0.0 & 1 & 0 & 0.3 & 1 & 0 & 0.0 & 1 & 0 & 0.0 & 1\\
\texttt{breast-wisconsin} & \multicolumn{1}{r}{683} & \multicolumn{1}{r}{120}  & 0 & 0.0 & 1 & 0 & 0.0 & 1 & 0 & 0.0 & 1 & 0 & 0.0 & 1\\
\texttt{car} & \multicolumn{1}{r}{1728} & \multicolumn{1}{r}{21}  & 0 & \textbf{0.3} & 1 & 0 & 21.1 & 1 & 0 & 0.3 & 1 & 0 & 0.4 & 1\\
\texttt{compas\_discretized} & \multicolumn{1}{r}{6167} & \multicolumn{1}{r}{25}  & 1828 & \textbf{0.7} & 1 & 1828 & 9.1 & 1 & 1828 & 323.1 & 0 & 1828 & 1.4 & 1\\
\texttt{diabetes} & \multicolumn{1}{r}{768} & \multicolumn{1}{r}{112}  & 0 & 0.7 & 1 & 0 & 3026.4 & 1 & 0 & 11.4 & 1 & 0 & \textbf{0.6} & 1\\
\texttt{forest-fires} & \multicolumn{1}{r}{517} & \multicolumn{1}{r}{989}  & 113 & \textbf{941.8} & 0 & 114 & 3067.9 & 0 & 118 & 3166.7 & 0 & 113 & 1003.4 & 0\\
\texttt{german-credit} & \multicolumn{1}{r}{1000} & \multicolumn{1}{r}{112}  & 0 & \textbf{69.2} & 1 & 62 & 2594.4 & 0 & 0 & 172.8 & 1 & 0 & 95.9 & 1\\
\texttt{heart-cleveland} & \multicolumn{1}{r}{296} & \multicolumn{1}{r}{95}  & 0 & 0.0 & 1 & 0 & 0.0 & 1 & 0 & 0.0 & 1 & 0 & 0.0 & 1\\
\texttt{hepatitis} & \multicolumn{1}{r}{137} & \multicolumn{1}{r}{68}  & 0 & 0.0 & 1 & 0 & 0.0 & 1 & 0 & 0.0 & 1 & 0 & 0.0 & 1\\
\texttt{hypothyroid} & \multicolumn{1}{r}{3247} & \multicolumn{1}{r}{88}  & 17 & \textbf{1.0} & 1 & 17 & 39.6 & 1 & 17 & 71.6 & 0 & 17 & 1.5 & 1\\
\texttt{ionosphere} & \multicolumn{1}{r}{351} & \multicolumn{1}{r}{445}  & 0 & 0.0 & 1 & 0 & 0.0 & 1 & 0 & 0.0 & 1 & 0 & 0.0 & 1\\
\texttt{kr-vs-kp} & \multicolumn{1}{r}{3196} & \multicolumn{1}{r}{73}  & 0 & 1896.7 & 1 & 0 & 752.2 & 1 & 5 & \textbf{85.8} & 0 & 1 & 400.0 & 0\\
\texttt{letter} & \multicolumn{1}{r}{20000} & \multicolumn{1}{r}{224}  & 0 & \textbf{79.2} & 1 & 88 & 1824.6 & 0 & 0 & 1534.7 & 1 & 0 & 104.3 & 1\\
\texttt{lymph} & \multicolumn{1}{r}{148} & \multicolumn{1}{r}{68}  & 0 & 0.0 & 1 & 0 & 0.0 & 1 & 0 & 0.0 & 1 & 0 & 0.0 & 1\\
\texttt{mnist\_0} & \multicolumn{1}{r}{60000} & \multicolumn{1}{r}{784}  & 383 & 413.3 & 0 & 1721 & 3234.8 & 0 & 383 & \textbf{404.3} & 0 & 383 & 449.9 & 0\\
\texttt{mnist\_1} & \multicolumn{1}{r}{60000} & \multicolumn{1}{r}{784}  & 331 & 359.8 & 0 & 1493 & 1782.8 & 0 & \textbf{330} & \textbf{354.7} & 0 & 331 & 381.9 & 0\\
\texttt{mnist\_2} & \multicolumn{1}{r}{60000} & \multicolumn{1}{r}{784}  & 1522 & 2519.6 & 0 & 2968 & \textbf{1804.3} & 0 & 1522 & 2465.5 & 0 & 1522 & 3022.5 & 0\\
\texttt{mnist\_3} & \multicolumn{1}{r}{60000} & \multicolumn{1}{r}{784}  & 1079 & \textbf{376.1} & 0 & 4098 & 2365.0 & 0 & \textbf{1065} & 3406.5 & 0 & 1079 & 405.8 & 0\\
\texttt{mnist\_4} & \multicolumn{1}{r}{60000} & \multicolumn{1}{r}{784}  & 801 & \textbf{397.9} & 0 & 4289 & 1461.4 & 0 & 801 & 423.8 & 0 & 801 & 446.8 & 0\\
\texttt{mnist\_5} & \multicolumn{1}{r}{60000} & \multicolumn{1}{r}{784}  & 1973 & \textbf{490.5} & 0 & 2866 & 3419.6 & 0 & 1974 & 508.8 & 0 & 1973 & 949.5 & 0\\
\texttt{mnist\_6} & \multicolumn{1}{r}{60000} & \multicolumn{1}{r}{784}  & 965 & \textbf{1742.0} & 0 & 1370 & 2484.2 & 0 & 965 & 2227.8 & 0 & 965 & 2418.4 & 0\\
\texttt{mnist\_7} & \multicolumn{1}{r}{60000} & \multicolumn{1}{r}{784}  & 1082 & \textbf{1276.8} & 0 & 2228 & 2846.1 & 0 & 1082 & 1543.9 & 0 & 1082 & 1797.6 & 0\\
\texttt{mnist\_8} & \multicolumn{1}{r}{60000} & \multicolumn{1}{r}{784}  & 696 & 2141.2 & 0 & 2450 & \textbf{1144.2} & 0 & 801 & 1935.8 & 0 & 696 & 2465.5 & 0\\
\texttt{mnist\_9} & \multicolumn{1}{r}{60000} & \multicolumn{1}{r}{784}  & 1594 & \textbf{2124.1} & 0 & 3598 & 2205.4 & 0 & 1596 & 2234.5 & 0 & 1594 & 2299.2 & 0\\
\texttt{mushroom} & \multicolumn{1}{r}{8124} & \multicolumn{1}{r}{119}  & 0 & 0.0 & 1 & 0 & 0.0 & 1 & 0 & 0.0 & 1 & 0 & 0.0 & 1\\
\texttt{pendigits} & \multicolumn{1}{r}{7494} & \multicolumn{1}{r}{216}  & 0 & 0.0 & 1 & 0 & 0.0 & 1 & 0 & 0.0 & 1 & 0 & 0.0 & 1\\
\texttt{primary-tumor} & \multicolumn{1}{r}{336} & \multicolumn{1}{r}{31}  & 15 & 0.0 & 1 & 15 & 0.0 & 1 & 15 & 0.3 & 0 & 15 & 0.0 & 1\\
\texttt{segment} & \multicolumn{1}{r}{2310} & \multicolumn{1}{r}{235}  & 0 & 0.0 & 1 & 0 & 0.0 & 1 & 0 & 0.0 & 1 & 0 & 0.0 & 1\\
\texttt{soybean} & \multicolumn{1}{r}{630} & \multicolumn{1}{r}{50}  & 2 & 0.0 & 1 & 2 & 0.4 & 1 & 2 & 0.0 & 0 & 2 & 0.0 & 1\\
\texttt{splice-1} & \multicolumn{1}{r}{3190} & \multicolumn{1}{r}{287}  & 5 & \textbf{1159.5} & 0 & 12 & 1675.9 & 0 & \textbf{4} & 3505.7 & 0 & 5 & 1205.0 & 0\\
\texttt{surgical-deepnet} & \multicolumn{1}{r}{14635} & \multicolumn{1}{r}{6047}  & 965 & \textbf{2865.0} & 0 & 1849 & 3204.4 & 0 & 965 & 3133.3 & 0 & 965 & 3191.7 & 0\\
\texttt{taiwan\_binarised} & \multicolumn{1}{r}{30000} & \multicolumn{1}{r}{205}  & 4217 & \textbf{1001.1} & 0 & 4896 & 2890.2 & 0 & \textbf{4189} & 1045.5 & 0 & 4217 & 1041.2 & 0\\
\texttt{tic-tac-toe} & \multicolumn{1}{r}{958} & \multicolumn{1}{r}{27}  & 0 & 0.0 & 1 & 0 & 0.0 & 1 & 0 & 0.0 & 1 & 0 & 0.0 & 1\\
\texttt{titanic} & \multicolumn{1}{r}{887} & \multicolumn{1}{r}{333}  & \textbf{35} & 3059.0 & 0 & 52 & 942.9 & 0 & 45 & 1076.6 & 0 & 42 & \textbf{179.5} & 0\\
\texttt{vehicle} & \multicolumn{1}{r}{846} & \multicolumn{1}{r}{252}  & 0 & 0.0 & 1 & 0 & 60.1 & 1 & 0 & 0.0 & 1 & 0 & 0.0 & 1\\
\texttt{vote} & \multicolumn{1}{r}{435} & \multicolumn{1}{r}{48}  & 0 & 0.0 & 1 & 0 & 0.0 & 1 & 0 & 0.0 & 1 & 0 & 0.0 & 1\\
\texttt{weather-aus} & \multicolumn{1}{r}{142193} & \multicolumn{1}{r}{4759}  & 1601 & 2590.7 & 0 & 1734 & 2391.1 & 0 & 1603 & \textbf{1987.5} & 0 & 1601 & 2758.1 & 0\\
\texttt{wine1} & \multicolumn{1}{r}{178} & \multicolumn{1}{r}{1276}  & 22 & 544.9 & 0 & \textbf{20} & 1468.7 & 0 & 22 & 3226.6 & 0 & 22 & \textbf{539.1} & 0\\
\texttt{wine2} & \multicolumn{1}{r}{178} & \multicolumn{1}{r}{1276}  & 24 & 398.8 & 0 & \textbf{21} & \textbf{19.9} & 0 & 24 & 2831.9 & 0 & 24 & 415.0 & 0\\
\texttt{wine3} & \multicolumn{1}{r}{178} & \multicolumn{1}{r}{1276}  & 16 & 271.7 & 0 & 17 & 690.4 & 0 & 18 & 1801.5 & 0 & 16 & \textbf{269.7} & 0\\
\texttt{yeast} & \multicolumn{1}{r}{1484} & \multicolumn{1}{r}{89}  & 28 & 1008.5 & 0 & 68 & 2610.5 & 0 & 67 & \textbf{465.7} & 0 & 28 & 1633.3 & 0\\
\bottomrule
\end{tabular}

% \end{footnotesize}
% \end{center}
% \caption{\label{tab:fact10} Factor analysis: depth 10}
% \end{table}


\end{document}

\begin{table}[htbp]
\begin{center}
\begin{normalsize}
\tabcolsep=4pt
\begin{tabular}{lrrrrrrrr}
\toprule
&  \multicolumn{2}{c}{\budalg} & \multicolumn{2}{c}{\murtree} & \multicolumn{4}{c}{\dleight}\\
\cmidrule(rr){2-3}\cmidrule(rr){4-5}\cmidrule(rr){6-9}
& \multicolumn{1}{c}{acc.} & \multicolumn{1}{c}{opt.} & \multicolumn{1}{c}{acc.} & \multicolumn{1}{c}{opt.} & \multicolumn{1}{c}{acc.} & \multicolumn{1}{c}{acc. (r)} & \multicolumn{1}{c}{cpu (r)} & \multicolumn{1}{c}{opt.} \\
\midrule

\texttt{adult\_discretized} & \textbf{0.8669} & 0.00 & 0.8289 & \textbf{1.00} & 0.7954 & -0.0825 & - & 0.00\\
\texttt{anneal} & \textbf{0.9569} & 0.00 & 0.8855 & \textbf{1.00} & - & - & - & 0.00\\
\texttt{audiology} & 1.0000 & 1.00 & 1.0000 & 1.00 & 1.0000 & 0.0000 & +1.01 & 1.00\\
\texttt{australian-credit} & 1.0000 & 1.00 & 1.0000 & 1.00 & - & - & - & 0.00\\
\texttt{bank} & \textbf{0.9230} & 0.00 & 0.8833 & 0.00 & 0.8935 & -0.0320 & - & 0.00\\
\texttt{breast-cancer} & 1.0000 & 1.00 & 1.0000 & 1.00 & 1.0000 & 0.0000 & -0.98 & 1.00\\
\texttt{breast-wisconsin} & 1.0000 & 1.00 & 1.0000 & 1.00 & 1.0000 & 0.0000 & +424.02 & 1.00\\
\texttt{car} & 1.0000 & 1.00 & 1.0000 & 1.00 & 1.0000 & 0.0000 & -0.98 & 1.00\\
\texttt{compas\_discretized} & 0.7036 & 1.00 & \textbf{0.7036} & 1.00 & - & - & - & 0.00\\
\texttt{diabetes} & \textbf{1.0000} & 1.00 & 0.8737 & 1.00 & - & - & - & 0.00\\
\texttt{forest-fires} & 0.7099 & 0.00 & \textbf{0.7563} & 0.00 & - & - & - & 0.00\\
\texttt{german-credit} & 0.9850 & 0.00 & \textbf{1.0000} & \textbf{1.00} & - & - & - & 0.00\\
\texttt{heart-cleveland} & 1.0000 & 1.00 & 1.0000 & 1.00 & 1.0000 & 0.0000 & +1304.02 & 1.00\\
\texttt{hepatitis} & 1.0000 & 1.00 & 1.0000 & 1.00 & 1.0000 & 0.0000 & +1.03 & 1.00\\
\texttt{hypothyroid} & \textbf{0.9948} & 1.00 & 0.9769 & 1.00 & - & - & - & 0.00\\
\texttt{ionosphere} & 1.0000 & 1.00 & 1.0000 & 1.00 & 1.0000 & 0.0000 & +3171.02 & 1.00\\
\texttt{kr-vs-kp} & \textbf{0.9984} & 0.00 & 0.9671 & \textbf{1.00} & - & - & - & 0.00\\
\texttt{letter} & \textbf{0.9988} & 0.00 & 0.9730 & 0.00 & 0.9651 & -0.0337 & - & 0.00\\
\texttt{lymph} & 1.0000 & 1.00 & 1.0000 & 1.00 & 1.0000 & 0.0000 & +1.01 & 1.00\\
\texttt{mnist\_0} & \textbf{0.9904} & 0.00 & 0.9574 & 0.00 & - & - & - & 0.00\\
\texttt{mnist\_1} & \textbf{0.9908} & 0.00 & 0.9425 & \textbf{1.00} & 0.9242 & -0.0672 & - & 0.00\\
\texttt{mnist\_2} & \textbf{0.9671} & 0.00 & 0.9346 & 0.00 & - & - & - & 0.00\\
\texttt{mnist\_3} & \textbf{0.9780} & 0.00 & 0.9275 & 0.00 & - & - & - & 0.00\\
\texttt{mnist\_4} & \textbf{0.9799} & 0.00 & 0.9214 & \textbf{1.00} & 0.9070 & -0.0744 & - & 0.00\\
\texttt{mnist\_5} & \textbf{0.9591} & 0.00 & 0.9410 & 0.00 & 0.9270 & -0.0335 & - & 0.00\\
\texttt{mnist\_6} & \textbf{0.9802} & 0.00 & 0.9550 & 0.00 & 0.9547 & -0.0260 & - & 0.00\\
\texttt{mnist\_7} & \textbf{0.9784} & 0.00 & 0.9419 & 0.00 & 0.9242 & -0.0554 & - & 0.00\\
\texttt{mnist\_8} & \textbf{0.9806} & 0.00 & 0.9403 & \textbf{1.00} & - & - & - & 0.00\\
\texttt{mnist\_9} & \textbf{0.9672} & 0.00 & 0.9235 & 0.00 & - & - & - & 0.00\\
\texttt{mushroom} & 1.0000 & 1.00 & 1.0000 & 1.00 & 1.0000 & 0.0000 & +16.77 & 1.00\\
\texttt{pendigits} & 1.0000 & 1.00 & 1.0000 & 1.00 & - & - & - & 0.00\\
\texttt{primary-tumor} & \textbf{0.9554} & 1.00 & 0.9554 & 1.00 & - & - & - & 0.00\\
\texttt{segment} & 1.0000 & 1.00 & 1.0000 & 1.00 & 1.0000 & 0.0000 & +2.58 & 1.00\\
\texttt{soybean} & \textbf{0.9968} & 1.00 & 0.9968 & 1.00 & - & - & - & 0.00\\
\texttt{splice-1} & \textbf{0.9962} & 0.00 & 0.6821 & 0.00 & - & - & - & 0.00\\
\texttt{surgical-deepnet} & \textbf{0.9215} & 0.00 & 0.8194 & 0.00 & - & - & - & 0.00\\
\texttt{taiwan\_binarised} & \textbf{0.8444} & 0.00 & 0.8032 & 0.00 & - & - & - & 0.00\\
\texttt{tic-tac-toe} & 1.0000 & 1.00 & 1.0000 & 1.00 & 1.0000 & 0.0000 & +3.34 & 1.00\\
\texttt{titanic} & \textbf{0.9245} & 0.00 & 0.9019 & \textbf{1.00} & - & - & - & 0.00\\
\texttt{vehicle} & 1.0000 & 1.00 & 1.0000 & 1.00 & 1.0000 & 0.0000 & +69.76 & 1.00\\
\texttt{vote} & 1.0000 & 1.00 & 1.0000 & 1.00 & 1.0000 & 0.0000 & +1.02 & 1.00\\
\texttt{weather-aus} & \textbf{0.9883} & 0.00 & 0.9625 & 0.00 & - & - & - & 0.00\\
\texttt{wine1} & \textbf{0.8652} & 0.00 & 0.8146 & \textbf{1.00} & - & - & - & 0.00\\
\texttt{wine2} & \textbf{0.8483} & 0.00 & 0.7584 & \textbf{1.00} & - & - & - & 0.00\\
\texttt{wine3} & \textbf{0.8989} & 0.00 & 0.8483 & \textbf{1.00} & - & - & - & 0.00\\
\texttt{yeast} & \textbf{0.9589} & 0.00 & 0.8059 & \textbf{1.00} & - & - & - & 0.00\\
\bottomrule
\end{tabular}

\end{normalsize}
\end{center}
\caption{\label{tab:all} Comparison with state of the art: depth 9}
\end{table}


\begin{table}[htbp]
\begin{center}
\begin{normalsize}
\tabcolsep=4pt
\begin{tabular}{lrrrrrrrrrrrr}
\toprule
&  \multicolumn{5}{c}{\budalg} & \multicolumn{5}{c}{\murtree} & \multicolumn{2}{c}{\cart}\\
\cmidrule(rr){2-6}\cmidrule(rr){7-11}\cmidrule(rr){12-13}
& \multicolumn{1}{c}{cpu} & \multicolumn{1}{c}{first} & \multicolumn{1}{c}{$\leq$3s} & \multicolumn{1}{c}{$\leq$10s} & \multicolumn{1}{c}{$\leq$1m} & \multicolumn{1}{c}{cpu} & \multicolumn{1}{c}{first} & \multicolumn{1}{c}{$\leq$3s} & \multicolumn{1}{c}{$\leq$10s} & \multicolumn{1}{c}{$\leq$1m} & \multicolumn{1}{c}{cpu} & \multicolumn{1}{c}{first} \\
\midrule

\texttt{adult\_discretized} & 0.00 & \textbf{5554} & 5020 & 5020 & 5020 & 0.00 & 7511 & 5020 & 5020 & 5020 & 0.05 & 5758\\
\texttt{anneal} & 0.00 & \textbf{137} & 112 & 112 & 112 & 0.00 & 187 & 112 & 112 & 112 & 0.00 & 149\\
\texttt{audiology} & 0.00 & 6 & 5 & 5 & 5 & 0.00 & 57 & 5 & 5 & 5 & 0.00 & 6\\
\texttt{australian-credit} & 0.00 & \textbf{82} & 73 & 73 & 73 & 0.00 & 296 & 73 & 73 & 73 & 0.00 & 87\\
\texttt{bank} & 0.82 & \textbf{4456} & \textbf{4453} & \textbf{4453} & \textbf{4453} & \textbf{0.00} & 5289 & 5287 & 5287 & 5287 & 32.54 & 4462\\
\texttt{breast-cancer} & 0.00 & 28 & 24 & 24 & 24 & 0.00 & 239 & 24 & 24 & 24 & 0.00 & 28\\
\texttt{breast-wisconsin} & 0.00 & 26 & 15 & 15 & 15 & 0.00 & 239 & 15 & 15 & 15 & 0.00 & 26\\
\texttt{car} & 0.00 & 202 & 192 & 192 & 192 & 0.00 & 518 & 192 & 192 & 192 & 0.00 & 202\\
\texttt{compas\_discretized} & 0.00 & \textbf{2067} & 2004 & 2004 & 2004 & 0.00 & 2809 & 2004 & 2004 & 2004 & 0.01 & 2072\\
\texttt{diabetes} & 0.00 & \textbf{169} & 162 & 162 & 162 & 0.00 & 268 & 162 & 162 & 162 & 0.00 & 177\\
\texttt{forest-fires} & 0.00 & 198 & 194 & 194 & 193 & 0.00 & 247 & \textbf{193} & \textbf{193} & 193 & 0.01 & 198\\
\texttt{german-credit} & 0.00 & \textbf{249} & 236 & 236 & 236 & 0.00 & 300 & 236 & 236 & 236 & 0.00 & 251\\
\texttt{heart-cleveland} & 0.00 & 43 & 41 & 41 & 41 & 0.00 & 136 & 41 & 41 & 41 & 0.00 & 43\\
\texttt{hepatitis} & 0.00 & \textbf{14} & 10 & 10 & 10 & 0.00 & 26 & 10 & 10 & 10 & 0.00 & 16\\
\texttt{hypothyroid} & 0.00 & 62 & 61 & 61 & 61 & 0.00 & 277 & 61 & 61 & 61 & 0.01 & 62\\
\texttt{ionosphere} & 0.00 & 29 & 22 & 22 & 22 & 0.00 & 126 & 22 & 22 & 22 & 0.01 & 29\\
\texttt{kr-vs-kp} & 0.00 & 306 & 198 & 198 & 198 & 0.00 & 1527 & 198 & 198 & 198 & 0.01 & 306\\
\texttt{letter} & 0.00 & \textbf{657} & \textbf{531} & \textbf{369} & 369 & 0.00 & 813 & 587 & 532 & 369 & 0.17 & 677\\
\texttt{lymph} & 0.00 & \textbf{16} & 12 & 12 & 12 & 0.00 & 67 & 12 & 12 & 12 & 0.00 & 17\\
\texttt{mnist\_0} & 0.01 & \textbf{3110} & \textbf{2862} & \textbf{2822} & \textbf{2568} & \textbf{0.00} & 5923 & 5923 & 3366 & 2717 & 2.48 & 3329\\
\texttt{mnist\_1} & 0.02 & \textbf{3533} & \textbf{3465} & \textbf{3464} & \textbf{3462} & \textbf{0.00} & 6742 & 6742 & 4725 & 3590 & 2.45 & 3534\\
\texttt{mnist\_2} & 0.01 & \textbf{4338} & \textbf{3994} & \textbf{3994} & \textbf{3938} & \textbf{0.00} & 5958 & 5958 & 4289 & 4026 & 2.57 & 4530\\
\texttt{mnist\_3} & 0.01 & \textbf{5024} & \textbf{4563} & \textbf{4354} & \textbf{4354} & \textbf{0.00} & 6131 & 6131 & 5172 & 4364 & 2.48 & 6131\\
\texttt{mnist\_4} & 0.02 & \textbf{4900} & \textbf{4891} & \textbf{4754} & \textbf{4742} & \textbf{0.00} & 5842 & 5842 & 5580 & 4751 & 2.59 & 5037\\
\texttt{mnist\_5} & 0.01 & 4032 & \textbf{3982} & \textbf{3701} & \textbf{3607} & \textbf{0.00} & 5421 & 5421 & 4400 & 3636 & 2.64 & 4032\\
\texttt{mnist\_6} & 0.01 & 2893 & \textbf{2885} & \textbf{2774} & 2774 & \textbf{0.00} & 5918 & 5918 & 2999 & \textbf{2756} & 2.64 & 2893\\
\texttt{mnist\_7} & 0.01 & \textbf{3660} & \textbf{3617} & \textbf{3483} & \textbf{3483} & \textbf{0.00} & 6265 & 6265 & 4546 & 3978 & 2.54 & 3788\\
\texttt{mnist\_8} & 0.02 & \textbf{4247} & \textbf{4003} & \textbf{4003} & \textbf{3583} & \textbf{0.00} & 5851 & 5851 & 4755 & 4437 & 2.58 & 4250\\
\texttt{mnist\_9} & 0.01 & \textbf{4874} & \textbf{4845} & \textbf{4704} & \textbf{4624} & \textbf{0.00} & 5949 & 5949 & 5254 & 4708 & 2.62 & 5355\\
\texttt{mushroom} & 0.00 & 280 & 8 & 8 & 8 & 0.00 & 3916 & 8 & 8 & 8 & 0.02 & 280\\
\texttt{pendigits} & 0.00 & 51 & \textbf{47} & 47 & 47 & 0.00 & 780 & 72 & 47 & 47 & 0.05 & 51\\
\texttt{primary-tumor} & 0.00 & \textbf{51} & 46 & 46 & 46 & 0.00 & 82 & 46 & 46 & 46 & 0.00 & 53\\
\texttt{segment} & 0.00 & 5 & 0 & 0 & 0 & 0.00 & 330 & 0 & 0 & 0 & 0.01 & 5\\
\texttt{soybean} & 0.00 & 48 & 29 & 29 & 29 & 0.00 & 92 & 29 & 29 & 29 & 0.00 & \textbf{47}\\
\texttt{splice-1} & 0.00 & 279 & \textbf{224} & 224 & 224 & 0.00 & 1535 & 444 & 224 & 224 & 0.03 & 279\\
\texttt{surgical-deepnet} & 0.25 & \textbf{2794} & \textbf{2758} & \textbf{2709} & \textbf{2530} & \textbf{0.00} & 3690 & 3690 & 3690 & 3690 & 5.68 & 2924\\
\texttt{taiwan\_binarised} & 0.00 & \textbf{5333} & \textbf{5326} & 5326 & 5326 & 0.00 & 6636 & 5369 & 5326 & 5326 & 0.26 & 5346\\
\texttt{tic-tac-toe} & 0.00 & 236 & 216 & 216 & 216 & 0.00 & 332 & 216 & 216 & 216 & 0.00 & 236\\
\texttt{titanic} & 0.00 & \textbf{146} & 145 & 143 & 143 & 0.00 & 342 & \textbf{143} & 143 & 143 & 0.01 & 148\\
\texttt{vehicle} & 0.00 & \textbf{55} & 26 & 26 & 26 & 0.00 & 218 & 26 & 26 & 26 & 0.01 & 66\\
\texttt{vote} & 0.00 & 14 & 12 & 12 & 12 & 0.00 & 168 & 12 & 12 & 12 & 0.00 & 14\\
\texttt{weather-aus} & 0.44 & \textbf{1758} & \textbf{1757} & \textbf{1757} & \textbf{1756} & \textbf{0.00} & 31877 & 31877 & 31877 & 31877 & 19.57 & 1761\\
\texttt{wine1} & 0.00 & 45 & \textbf{43} & 43 & 43 & 0.00 & 59 & 45 & 43 & 43 & 0.00 & 45\\
\texttt{wine2} & 0.00 & 52 & \textbf{49} & 49 & 49 & 0.00 & 71 & 57 & 49 & 49 & 0.00 & 52\\
\texttt{wine3} & 0.00 & 35 & \textbf{33} & 33 & 33 & 0.00 & 48 & 37 & 33 & 33 & 0.00 & 35\\
\texttt{yeast} & 0.00 & \textbf{417} & 403 & 403 & 403 & 0.00 & 463 & 403 & 403 & 403 & 0.00 & 418\\
\bottomrule
\end{tabular}

\end{normalsize}
\end{center}
\caption{\label{tab:all} Comparison with state of the art: depth 3}
\end{table}

\medskip

\begin{table}[htbp]
\begin{center}
\begin{normalsize}
\tabcolsep=4pt
\begin{tabular}{lrrrrrrrrrrrr}
\toprule
&  \multicolumn{5}{c}{\budalg} & \multicolumn{5}{c}{\murtree} & \multicolumn{2}{c}{\cart}\\
\cmidrule(rr){2-6}\cmidrule(rr){7-11}\cmidrule(rr){12-13}
& \multicolumn{1}{c}{cpu} & \multicolumn{1}{c}{first} & \multicolumn{1}{c}{$\leq$3s} & \multicolumn{1}{c}{$\leq$10s} & \multicolumn{1}{c}{$\leq$1m} & \multicolumn{1}{c}{cpu} & \multicolumn{1}{c}{first} & \multicolumn{1}{c}{$\leq$3s} & \multicolumn{1}{c}{$\leq$10s} & \multicolumn{1}{c}{$\leq$1m} & \multicolumn{1}{c}{cpu} & \multicolumn{1}{c}{first} \\
\midrule

\texttt{adult\_discretized} & 0.00 & 5149 & \textbf{4609} & \textbf{4609} & 4609 & 0.00 & 7511 & 4985 & 4613 & 4609 & 0.06 & \textbf{5022}\\
\texttt{anneal} & 0.00 & 135 & 91 & 91 & 91 & 0.00 & 187 & 91 & 91 & 91 & 0.00 & 135\\
\texttt{audiology} & 0.00 & 3 & 1 & 1 & 1 & 0.00 & 57 & 1 & 1 & 1 & 0.00 & 3\\
\texttt{australian-credit} & 0.00 & \textbf{73} & \textbf{60} & 57 & 56 & 0.00 & 296 & 66 & 57 & 56 & 0.00 & 74\\
\texttt{bank} & 0.81 & \textbf{4351} & \textbf{4343} & \textbf{4343} & \textbf{4326} & \textbf{0.00} & 5289 & 5287 & 5287 & 5287 & 32.03 & 4420\\
\texttt{breast-cancer} & 0.00 & 21 & 16 & 16 & 16 & 0.00 & 239 & 16 & 16 & 16 & 0.00 & 21\\
\texttt{breast-wisconsin} & 0.00 & 16 & \textbf{7} & 7 & 7 & 0.00 & 239 & 8 & 7 & 7 & 0.00 & 16\\
\texttt{car} & 0.00 & 178 & 136 & 136 & 136 & 0.00 & 518 & 136 & 136 & 136 & 0.00 & 178\\
\texttt{compas\_discretized} & 0.00 & 2023 & 1954 & 1954 & 1954 & 0.00 & 2809 & 1954 & 1954 & 1954 & 0.01 & \textbf{1997}\\
\texttt{diabetes} & 0.00 & \textbf{159} & \textbf{137} & 137 & 137 & 0.00 & 268 & 142 & 137 & 137 & 0.00 & 166\\
\texttt{forest-fires} & 0.00 & 191 & \textbf{179} & 179 & 173 & 0.00 & 247 & 188 & \textbf{175} & \textbf{172} & 0.01 & \textbf{186}\\
\texttt{german-credit} & 0.00 & \textbf{224} & 204 & 204 & 204 & 0.00 & 300 & 204 & 204 & 204 & 0.00 & 231\\
\texttt{heart-cleveland} & 0.00 & \textbf{36} & 25 & 25 & 25 & 0.00 & 136 & 25 & 25 & 25 & 0.00 & 38\\
\texttt{hepatitis} & 0.00 & 12 & 3 & 3 & 3 & 0.00 & 26 & 3 & 3 & 3 & 0.00 & 12\\
\texttt{hypothyroid} & 0.00 & 53 & \textbf{53} & 53 & 53 & 0.00 & 277 & 57 & 53 & 53 & 0.01 & 53\\
\texttt{ionosphere} & 0.00 & \textbf{25} & \textbf{13} & \textbf{8} & \textbf{8} & 0.00 & 126 & 20 & 16 & 9 & 0.01 & 27\\
\texttt{kr-vs-kp} & 0.00 & \textbf{188} & \textbf{144} & 144 & 144 & 0.00 & 1527 & 172 & 144 & 144 & 0.01 & 189\\
\texttt{letter} & 0.00 & \textbf{443} & \textbf{431} & \textbf{276} & \textbf{261} & 0.00 & 813 & 596 & 550 & 338 & 0.20 & 462\\
\texttt{lymph} & 0.00 & \textbf{9} & 3 & 3 & 3 & 0.00 & 67 & 3 & 3 & 3 & 0.00 & 10\\
\texttt{mnist\_0} & 0.01 & \textbf{2310} & \textbf{2285} & \textbf{2265} & \textbf{2245} & \textbf{0.00} & 5923 & 5923 & 3319 & 2717 & 3.80 & 2311\\
\texttt{mnist\_1} & 0.02 & \textbf{2478} & \textbf{2440} & \textbf{2433} & \textbf{2433} & \textbf{0.00} & 6742 & 6742 & 4583 & 3589 & 3.57 & 2501\\
\texttt{mnist\_2} & 0.01 & \textbf{3989} & \textbf{3982} & \textbf{3963} & \textbf{3963} & \textbf{0.00} & 5958 & 5957 & 4304 & 4025 & 3.09 & 4326\\
\texttt{mnist\_3} & 0.02 & \textbf{4263} & \textbf{4181} & \textbf{4180} & \textbf{4180} & \textbf{0.00} & 6131 & 6131 & 4900 & 4364 & 4.85 & 4367\\
\texttt{mnist\_4} & 0.03 & \textbf{4101} & \textbf{4037} & \textbf{4037} & \textbf{4037} & \textbf{0.00} & 5842 & 5842 & 5580 & 4751 & 3.22 & 4129\\
\texttt{mnist\_5} & 0.01 & \textbf{3605} & \textbf{3595} & \textbf{3595} & \textbf{3371} & \textbf{0.00} & 5421 & 5421 & 4401 & 3636 & 3.84 & 3648\\
\texttt{mnist\_6} & 0.01 & \textbf{2248} & \textbf{2196} & \textbf{2191} & \textbf{2124} & \textbf{0.00} & 5918 & 5915 & 2798 & 2754 & 4.12 & 2251\\
\texttt{mnist\_7} & 0.01 & \textbf{3072} & \textbf{3028} & \textbf{3028} & \textbf{2793} & \textbf{0.00} & 6265 & 6265 & 4547 & 3978 & 3.94 & 3218\\
\texttt{mnist\_8} & 0.03 & \textbf{3905} & \textbf{3785} & \textbf{3760} & \textbf{3749} & \textbf{0.00} & 5851 & 5851 & 4753 & 4422 & 4.53 & 3987\\
\texttt{mnist\_9} & 0.01 & \textbf{4226} & \textbf{4199} & \textbf{4199} & \textbf{4117} & \textbf{0.00} & 5949 & 5949 & 5254 & 4708 & 3.15 & 4231\\
\texttt{mushroom} & 0.00 & 4 & \textbf{0} & 0 & 0 & 0.00 & 3916 & 6 & 0 & 0 & 0.02 & 4\\
\texttt{pendigits} & 0.00 & \textbf{22} & \textbf{17} & \textbf{14} & \textbf{13} & 0.00 & 780 & 70 & 32 & 24 & 0.07 & 25\\
\texttt{primary-tumor} & 0.00 & \textbf{43} & 34 & 34 & 34 & 0.00 & 82 & 34 & 34 & 34 & 0.00 & 44\\
\texttt{segment} & 0.00 & 1 & 0 & 0 & 0 & 0.00 & 330 & 0 & 0 & 0 & 0.01 & 1\\
\texttt{soybean} & 0.00 & 33 & 14 & 14 & 14 & 0.00 & 92 & 14 & 14 & 14 & 0.00 & \textbf{32}\\
\texttt{splice-1} & 0.00 & 141 & \textbf{141} & \textbf{141} & \textbf{141} & 0.00 & 1535 & 701 & 224 & 213 & 0.03 & 141\\
\texttt{surgical-deepnet} & 0.21 & \textbf{2605} & \textbf{2604} & \textbf{2590} & \textbf{2485} & \textbf{0.00} & 3690 & 3690 & 3690 & 3690 & 6.16 & 2704\\
\texttt{taiwan\_binarised} & 0.00 & \textbf{5293} & \textbf{5284} & \textbf{5273} & \textbf{5273} & 0.00 & 6636 & 5522 & 5489 & 5309 & 0.27 & 5306\\
\texttt{tic-tac-toe} & 0.00 & 150 & 137 & 137 & 137 & 0.00 & 332 & 137 & 137 & 137 & 0.00 & 150\\
\texttt{titanic} & 0.00 & 134 & \textbf{122} & \textbf{119} & 119 & 0.00 & 342 & 134 & 131 & 119 & 0.01 & 134\\
\texttt{vehicle} & 0.00 & 28 & \textbf{13} & \textbf{12} & 12 & 0.00 & 218 & 22 & 16 & 12 & 0.01 & 28\\
\texttt{vote} & 0.00 & 8 & 5 & 5 & 5 & 0.00 & 168 & 5 & 5 & 5 & 0.00 & 8\\
\texttt{weather-aus} & 0.46 & \textbf{1757} & \textbf{1753} & \textbf{1752} & \textbf{1752} & \textbf{0.00} & 31877 & 31877 & 31877 & 31877 & 19.96 & 1761\\
\texttt{wine1} & 0.00 & 42 & \textbf{39} & \textbf{39} & \textbf{39} & 0.00 & 59 & 45 & 43 & 42 & 0.01 & 42\\
\texttt{wine2} & 0.01 & 47 & \textbf{46} & \textbf{46} & \textbf{43} & \textbf{0.00} & 71 & 57 & 49 & 49 & 0.01 & 47\\
\texttt{wine3} & 0.01 & 32 & \textbf{30} & \textbf{30} & \textbf{28} & \textbf{0.00} & 48 & 37 & 33 & 32 & 0.01 & 32\\
\texttt{yeast} & 0.00 & \textbf{391} & 367 & 366 & 366 & 0.00 & 463 & \textbf{366} & 366 & 366 & 0.01 & 394\\
\bottomrule
\end{tabular}

\end{normalsize}
\end{center}
\caption{\label{tab:all} Comparison with state of the art: depth 4}
\end{table}

\medskip

\begin{table}[htbp]
\begin{center}
\begin{normalsize}
\tabcolsep=4pt
\begin{tabular}{lrrrrrrrrrrrr}
\toprule
&  \multicolumn{5}{c}{\budalg} & \multicolumn{5}{c}{\murtree} & \multicolumn{2}{c}{\cart}\\
\cmidrule(rr){2-6}\cmidrule(rr){7-11}\cmidrule(rr){12-13}
& \multicolumn{1}{c}{cpu} & \multicolumn{1}{c}{first} & \multicolumn{1}{c}{$\leq$3s} & \multicolumn{1}{c}{$\leq$10s} & \multicolumn{1}{c}{$\leq$1m} & \multicolumn{1}{c}{cpu} & \multicolumn{1}{c}{first} & \multicolumn{1}{c}{$\leq$3s} & \multicolumn{1}{c}{$\leq$10s} & \multicolumn{1}{c}{$\leq$1m} & \multicolumn{1}{c}{cpu} & \multicolumn{1}{c}{first} \\
\midrule

\texttt{adult\_discretized} & 0.00 & \textbf{4183} & \textbf{4174} & \textbf{4174} & \textbf{4173} & 0.00 & 7511 & 5773 & 5773 & 5374 & 0.12 & 4252\\
\texttt{anneal} & 0.00 & 77 & \textbf{53} & \textbf{39} & \textbf{39} & 0.00 & 187 & 117 & 101 & 101 & 0.00 & \textbf{74}\\
\texttt{audiology} & 0.00 & 0 & 0 & 0 & 0 & 0.00 & 57 & 0 & 0 & 0 & 0.00 & 0\\
\texttt{australian-credit} & 0.00 & 20 & \textbf{1} & \textbf{0} & \textbf{0} & 0.00 & 296 & 152 & 122 & 120 & 0.01 & \textbf{19}\\
\texttt{bank} & 1.05 & \textbf{3522} & \textbf{3520} & \textbf{3513} & \textbf{3508} & \textbf{0.00} & 5289 & 5278 & 5278 & 5278 & 75.74 & 3575\\
\texttt{breast-cancer} & 0.00 & 1 & 1 & 0 & 0 & 0.00 & 239 & \textbf{0} & 0 & 0 & 0.00 & 1\\
\texttt{breast-wisconsin} & 0.00 & 0 & 0 & 0 & 0 & 0.00 & 239 & 0 & 0 & 0 & 0.00 & 0\\
\texttt{car} & 0.00 & 15 & \textbf{6} & 2 & 2 & 0.00 & 518 & 181 & \textbf{0} & \textbf{0} & 0.00 & 15\\
\texttt{compas\_discretized} & 0.00 & \textbf{1880} & \textbf{1836} & \textbf{1832} & \textbf{1829} & 0.00 & 2809 & 2052 & 1965 & 1938 & 0.01 & 1891\\
\texttt{diabetes} & 0.00 & \textbf{54} & \textbf{3} & \textbf{0} & \textbf{0} & 0.00 & 268 & 172 & 150 & 142 & 0.01 & 55\\
\texttt{forest-fires} & 0.00 & \textbf{151} & \textbf{150} & \textbf{150} & \textbf{150} & 0.00 & 247 & 172 & 153 & 151 & 0.02 & 152\\
\texttt{german-credit} & 0.00 & \textbf{88} & \textbf{42} & \textbf{39} & \textbf{18} & 0.00 & 300 & 196 & 181 & 171 & 0.01 & 97\\
\texttt{heart-cleveland} & 0.00 & 0 & 0 & 0 & 0 & 0.00 & 136 & 0 & 0 & 0 & 0.00 & 0\\
\texttt{hepatitis} & 0.00 & 0 & 0 & 0 & 0 & 0.00 & 26 & 0 & 0 & 0 & 0.00 & 0\\
\texttt{hypothyroid} & 0.00 & \textbf{35} & \textbf{22} & \textbf{19} & \textbf{17} & 0.00 & 277 & 185 & 151 & 151 & 0.01 & 36\\
\texttt{ionosphere} & 0.00 & 0 & \textbf{0} & 0 & 0 & 0.00 & 126 & 23 & 0 & 0 & 0.01 & 0\\
\texttt{kr-vs-kp} & 0.00 & 23 & \textbf{13} & \textbf{13} & \textbf{13} & 0.00 & 1527 & 575 & 230 & 143 & 0.01 & 23\\
\texttt{letter} & 0.00 & \textbf{47} & \textbf{28} & \textbf{27} & \textbf{26} & 0.00 & 813 & 747 & 697 & 697 & 0.37 & 48\\
\texttt{lymph} & 0.00 & 0 & 0 & 0 & 0 & 0.00 & 67 & 0 & 0 & 0 & 0.00 & 0\\
\texttt{mnist\_0} & 0.04 & \textbf{694} & \textbf{692} & \textbf{692} & \textbf{683} & \textbf{0.00} & 5923 & 5923 & 3358 & 2717 & 8.62 & 710\\
\texttt{mnist\_1} & 0.03 & \textbf{570} & \textbf{563} & \textbf{563} & \textbf{561} & \textbf{0.00} & 6742 & 6742 & 4702 & 3584 & 6.47 & 573\\
\texttt{mnist\_2} & 0.02 & \textbf{2035} & \textbf{2013} & \textbf{2012} & \textbf{2011} & \textbf{0.00} & 5958 & 5955 & 4282 & 4024 & 7.24 & 2058\\
\texttt{mnist\_3} & 0.04 & \textbf{1399} & \textbf{1381} & \textbf{1381} & \textbf{1376} & \textbf{0.00} & 6131 & 6131 & 5188 & 4363 & 6.89 & 1442\\
\texttt{mnist\_4} & 0.04 & \textbf{1247} & \textbf{1224} & \textbf{1224} & \textbf{1219} & \textbf{0.00} & 5842 & 5842 & 5580 & 4750 & 5.36 & 1306\\
\texttt{mnist\_5} & 0.03 & \textbf{2476} & \textbf{2466} & \textbf{2465} & \textbf{2458} & \textbf{0.00} & 5421 & 5421 & 4396 & 3636 & 9.13 & 2553\\
\texttt{mnist\_6} & 0.01 & \textbf{1240} & \textbf{1227} & \textbf{1226} & \textbf{1193} & \textbf{0.00} & 5918 & 5861 & 2734 & 2715 & 6.17 & 1245\\
\texttt{mnist\_7} & 0.02 & \textbf{1326} & \textbf{1321} & \textbf{1320} & \textbf{1306} & \textbf{0.00} & 6265 & 6265 & 4546 & 3978 & 7.24 & 1371\\
\texttt{mnist\_8} & 0.04 & \textbf{1229} & \textbf{1207} & \textbf{1207} & \textbf{1197} & \textbf{0.00} & 5851 & 5851 & 4754 & 4436 & 6.87 & 1267\\
\texttt{mnist\_9} & 0.02 & \textbf{2038} & \textbf{2023} & \textbf{1997} & \textbf{1970} & \textbf{0.00} & 5949 & 5949 & 5253 & 4708 & 9.32 & 2110\\
\texttt{mushroom} & 0.00 & 0 & 0 & 0 & 0 & 0.00 & 3916 & 0 & 0 & 0 & 0.03 & 0\\
\texttt{pendigits} & 0.00 & 0 & \textbf{0} & \textbf{0} & \textbf{0} & 0.00 & 780 & 465 & 465 & 460 & 0.07 & 0\\
\texttt{primary-tumor} & 0.00 & \textbf{19} & \textbf{15} & \textbf{15} & \textbf{15} & 0.00 & 82 & 37 & 26 & 26 & 0.01 & 21\\
\texttt{segment} & 0.00 & 0 & 0 & 0 & 0 & 0.00 & 330 & 0 & 0 & 0 & 0.01 & 0\\
\texttt{soybean} & 0.00 & 9 & \textbf{2} & \textbf{2} & \textbf{2} & 0.00 & 92 & 29 & 29 & 17 & 0.00 & \textbf{5}\\
\texttt{splice-1} & 0.00 & 18 & \textbf{14} & \textbf{14} & \textbf{14} & 0.00 & 1535 & 1219 & 1153 & 1145 & 0.05 & 18\\
\texttt{surgical-deepnet} & 0.29 & \textbf{1169} & \textbf{1166} & \textbf{1159} & \textbf{1159} & \textbf{0.00} & 3690 & 3687 & 3687 & 3687 & 10.93 & 1193\\
\texttt{taiwan\_binarised} & 0.01 & \textbf{4858} & \textbf{4819} & \textbf{4805} & \textbf{4805} & \textbf{0.00} & 6636 & 6316 & 5963 & 5962 & 0.63 & 4911\\
\texttt{tic-tac-toe} & 0.00 & 10 & 0 & 0 & 0 & 0.00 & 332 & 0 & 0 & 0 & 0.00 & 10\\
\texttt{titanic} & 0.00 & \textbf{87} & \textbf{72} & \textbf{71} & \textbf{68} & 0.00 & 342 & 124 & 120 & 117 & 0.01 & 93\\
\texttt{vehicle} & 0.00 & 1 & 0 & 0 & 0 & 0.00 & 218 & 0 & 0 & 0 & 0.01 & 1\\
\texttt{vote} & 0.00 & 1 & 0 & 0 & 0 & 0.00 & 168 & 0 & 0 & 0 & 0.00 & 1\\
\texttt{weather-aus} & 0.46 & \textbf{1672} & \textbf{1669} & \textbf{1667} & \textbf{1667} & \textbf{0.00} & 31877 & 31874 & 31874 & 31874 & 27.11 & 1677\\
\texttt{wine1} & 0.01 & 27 & \textbf{26} & \textbf{26} & \textbf{25} & \textbf{0.00} & 59 & 39 & 37 & 36 & 0.01 & 27\\
\texttt{wine2} & 0.00 & 32 & \textbf{29} & \textbf{29} & \textbf{29} & 0.00 & 71 & 57 & 49 & 49 & 0.01 & 32\\
\texttt{wine3} & 0.01 & 22 & \textbf{20} & \textbf{20} & \textbf{20} & \textbf{0.00} & 48 & 36 & 32 & 32 & 0.01 & \textbf{18}\\
\texttt{yeast} & 0.00 & \textbf{224} & \textbf{167} & \textbf{142} & \textbf{121} & 0.00 & 463 & 397 & 361 & 360 & 0.01 & 232\\
\bottomrule
\end{tabular}

\end{normalsize}
\end{center}
\caption{\label{tab:all} Comparison with state of the art: depth 9}
\end{table}

\medskip




\begin{table}[htbp]
\begin{center}
\begin{normalsize}
\tabcolsep=4pt
\begin{tabular}{lccrrrrrrrrrrrr}
\toprule
\multirow{2}{*}{}& && \multicolumn{2}{c}{\budalg} & \multicolumn{2}{c}{\murtree} & \multicolumn{2}{c}{\dleight} & \multicolumn{2}{c}{\cp} & \multicolumn{2}{c}{binoct} & \multicolumn{2}{c}{\cart}\\
\cmidrule(rr){4-5}\cmidrule(rr){6-7}\cmidrule(rr){8-9}\cmidrule(rr){10-11}\cmidrule(rr){12-13}\cmidrule(rr){14-15}
&\multirow{1}{*}{$\#ex.$} & \multirow{1}{*}{\#feat.} &  \multicolumn{1}{c}{error} & \multicolumn{1}{c}{cpu} & \multicolumn{1}{c}{error} & \multicolumn{1}{c}{cpu} & \multicolumn{1}{c}{error} & \multicolumn{1}{c}{cpu} & \multicolumn{1}{c}{error} & \multicolumn{1}{c}{cpu} & \multicolumn{1}{c}{error} & \multicolumn{1}{c}{cpu} & \multicolumn{1}{c}{error} & \multicolumn{1}{c}{cpu} \\
\midrule

\texttt{adult\_discretized} & \multicolumn{1}{r}{30299} & \multicolumn{1}{r}{59}  & 5020 & 0.43$^*$ & 5020 & 0.84$^*$ & 5020 & 10$^*$ & 5020 & 6.4$^*$ & 5600 & $\mathsmaller{\geq}1$h & 5758 & 0.05\\
\texttt{anneal} & \multicolumn{1}{r}{812} & \multicolumn{1}{r}{93}  & 112 & 0.03$^*$ & 112 & 0.14$^*$ & 112 & 2.4$^*$ & 112 & 6.0$^*$ & 123 & $\mathsmaller{\geq}1$h & 149 & 0.00\\
\texttt{audiology} & \multicolumn{1}{r}{216} & \multicolumn{1}{r}{148}  & 5 & 0.06$^*$ & 5 & 0.13$^*$ & 5 & 4.5$^*$ & 5 & 9.1$^*$ & 6 & $\mathsmaller{\geq}1$h & 6 & 0.00\\
\texttt{australian-credit} & \multicolumn{1}{r}{653} & \multicolumn{1}{r}{125}  & 73 & 0.14$^*$ & 73 & 0.35$^*$ & 73 & 9.6$^*$ & 73 & 14$^*$ & 87 & $\mathsmaller{\geq}1$h & 87 & 0.00\\
\texttt{bank} & \multicolumn{1}{r}{45211} & \multicolumn{1}{r}{9531}  & 4453 & 259 & 5289 & 0.84 & 4805 & $\mathsmaller{\geq}1$h & 4453 & $\mathsmaller{\geq}1$h & - & - & 4462 & 33\\
\texttt{breast-cancer} & \multicolumn{1}{r}{683} & \multicolumn{1}{r}{89}  & 24 & 0.16$^*$ & 24 & 0.07$^*$ & 24 & 0.98$^*$ & 24 & 5.7$^*$ & 25 & $\mathsmaller{\geq}1$h & 28 & 0.00\\
\texttt{breast-wisconsin} & \multicolumn{1}{r}{683} & \multicolumn{1}{r}{120}  & 15 & 0.05$^*$ & 15 & 0.20$^*$ & 15 & 6.4$^*$ & 15 & 11$^*$ & 18 & $\mathsmaller{\geq}1$h & 26 & 0.00\\
\texttt{car} & \multicolumn{1}{r}{1728} & \multicolumn{1}{r}{21}  & 192 & 0.01$^*$ & 192 & 0.01$^*$ & 192 & 0.04$^*$ & 192 & 1.7$^*$ & 192 & $\mathsmaller{\geq}1$h & 202 & 0.00\\
\texttt{compas\_discretized} & \multicolumn{1}{r}{6167} & \multicolumn{1}{r}{25}  & 2004 & 0.00$^*$ & 2004 & 0.06$^*$ & 2004 & 0.23$^*$ & 2004 & 1.8$^*$ & 2032 & $\mathsmaller{\geq}1$h & 2072 & 0.01\\
\texttt{diabetes} & \multicolumn{1}{r}{768} & \multicolumn{1}{r}{112}  & 162 & 0.09$^*$ & 162 & 0.37$^*$ & 162 & 11$^*$ & 162 & 12$^*$ & 165 & $\mathsmaller{\geq}1$h & 177 & 0.00\\
\texttt{forest-fires} & \multicolumn{1}{r}{517} & \multicolumn{1}{r}{989}  & 193 & 20$^*$ & 193 & 9.6$^*$ & - & - & 193 & 2836$^*$ & 198 & $\mathsmaller{\geq}1$h & 198 & 0.01\\
\texttt{german-credit} & \multicolumn{1}{r}{1000} & \multicolumn{1}{r}{112}  & 236 & 0.26$^*$ & 236 & 0.38$^*$ & 236 & 7.7$^*$ & 236 & 13$^*$ & 244 & $\mathsmaller{\geq}1$h & 251 & 0.00\\
\texttt{heart-cleveland} & \multicolumn{1}{r}{296} & \multicolumn{1}{r}{95}  & 41 & 0.05$^*$ & 41 & 0.12$^*$ & 41 & 3.5$^*$ & 41 & 6.8$^*$ & 42 & $\mathsmaller{\geq}1$h & 43 & 0.00\\
\texttt{hepatitis} & \multicolumn{1}{r}{137} & \multicolumn{1}{r}{68}  & 10 & 0.00$^*$ & 10 & 0.03$^*$ & 10 & 1.2$^*$ & 10 & 3.9$^*$ & 10 & $\mathsmaller{\geq}1$h & 16 & 0.00\\
\texttt{hypothyroid} & \multicolumn{1}{r}{3247} & \multicolumn{1}{r}{88}  & 61 & 0.07$^*$ & 61 & 0.41$^*$ & 61 & 4.4$^*$ & 61 & 6.6$^*$ & 62 & $\mathsmaller{\geq}1$h & 62 & 0.01\\
\texttt{ionosphere} & \multicolumn{1}{r}{351} & \multicolumn{1}{r}{445}  & 22 & 3.8$^*$ & 22 & 12$^*$ & 22 & 410$^*$ & 22 & 460$^*$ & 27 & $\mathsmaller{\geq}1$h & 29 & 0.01\\
\texttt{kr-vs-kp} & \multicolumn{1}{r}{3196} & \multicolumn{1}{r}{73}  & 198 & 0.09$^*$ & 198 & 0.22$^*$ & 198 & 2.4$^*$ & 198 & 4.8$^*$ & 375 & $\mathsmaller{\geq}1$h & 306 & 0.01\\
\texttt{letter} & \multicolumn{1}{r}{20000} & \multicolumn{1}{r}{224}  & 369 & 10$^*$ & 369 & 34$^*$ & 369 & 443$^*$ & 369 & 158$^*$ & 813 & 1251 & 677 & 0.17\\
\texttt{lymph} & \multicolumn{1}{r}{148} & \multicolumn{1}{r}{68}  & 12 & 0.01$^*$ & 12 & 0.03$^*$ & 12 & 0.76$^*$ & 12 & 3.7$^*$ & 14 & $\mathsmaller{\geq}1$h & 17 & 0.00\\
\texttt{mnist\_0} & \multicolumn{1}{r}{60000} & \multicolumn{1}{r}{784}  & 2557 & 1994$^*$ & 2557 & 568$^*$ & 3319 & $\mathsmaller{\geq}1$h & 2557 & $\mathsmaller{\geq}1$h & - & - & 3329 & 2.5\\
\texttt{mnist\_1} & \multicolumn{1}{r}{60000} & \multicolumn{1}{r}{784}  & 3462 & 1896$^*$ & 3462 & 538$^*$ & 4552 & $\mathsmaller{\geq}1$h & 3462 & $\mathsmaller{\geq}1$h & - & - & 3534 & 2.5\\
\texttt{mnist\_2} & \multicolumn{1}{r}{60000} & \multicolumn{1}{r}{784}  & 3938 & 1946$^*$ & 3938 & 672$^*$ & 4289 & $\mathsmaller{\geq}1$h & 3938 & $\mathsmaller{\geq}1$h & - & - & 4530 & 2.6\\
\texttt{mnist\_3} & \multicolumn{1}{r}{60000} & \multicolumn{1}{r}{784}  & 4354 & 2054$^*$ & 4354 & 644$^*$ & 4974 & $\mathsmaller{\geq}1$h & 4354 & $\mathsmaller{\geq}1$h & - & - & 6131 & 2.5\\
\texttt{mnist\_4} & \multicolumn{1}{r}{60000} & \multicolumn{1}{r}{784}  & 4729 & 2070$^*$ & 4729 & 700$^*$ & 5580 & $\mathsmaller{\geq}1$h & 4729 & $\mathsmaller{\geq}1$h & - & - & 5037 & 2.6\\
\texttt{mnist\_5} & \multicolumn{1}{r}{60000} & \multicolumn{1}{r}{784}  & 3539 & 2095$^*$ & 3539 & 715$^*$ & 4379 & $\mathsmaller{\geq}1$h & 3539 & $\mathsmaller{\geq}1$h & - & - & 4032 & 2.6\\
\texttt{mnist\_6} & \multicolumn{1}{r}{60000} & \multicolumn{1}{r}{784}  & 2756 & 1916$^*$ & 2756 & 664$^*$ & 2756 & $\mathsmaller{\geq}1$h & 2756 & $\mathsmaller{\geq}1$h & - & - & 2893 & 2.6\\
\texttt{mnist\_7} & \multicolumn{1}{r}{60000} & \multicolumn{1}{r}{784}  & 3483 & 1928$^*$ & 3483 & 570$^*$ & 4546 & $\mathsmaller{\geq}1$h & 3483 & $\mathsmaller{\geq}1$h & - & - & 3788 & 2.5\\
\texttt{mnist\_8} & \multicolumn{1}{r}{60000} & \multicolumn{1}{r}{784}  & 3583 & 2061$^*$ & 3583 & 593$^*$ & 4609 & $\mathsmaller{\geq}1$h & 3583 & $\mathsmaller{\geq}1$h & - & - & 4250 & 2.6\\
\texttt{mnist\_9} & \multicolumn{1}{r}{60000} & \multicolumn{1}{r}{784}  & 4590 & 2039$^*$ & 4590 & 746$^*$ & 5253 & $\mathsmaller{\geq}1$h & 4590 & $\mathsmaller{\geq}1$h & - & - & 5355 & 2.6\\
\texttt{mushroom} & \multicolumn{1}{r}{8124} & \multicolumn{1}{r}{119}  & 8 & 0.79$^*$ & 8 & 0.53$^*$ & 8 & 6.3$^*$ & 8 & 8.4$^*$ & 180 & $\mathsmaller{\geq}1$h & 280 & 0.02\\
\texttt{pendigits} & \multicolumn{1}{r}{7494} & \multicolumn{1}{r}{216}  & 47 & 3.3$^*$ & 47 & 11$^*$ & 47 & 134$^*$ & 47 & 70$^*$ & 477 & $\mathsmaller{\geq}1$h & 51 & 0.05\\
\texttt{primary-tumor} & \multicolumn{1}{r}{336} & \multicolumn{1}{r}{31}  & 46 & 0.00$^*$ & 46 & 0.01$^*$ & 46 & 0.14$^*$ & 46 & 2.0$^*$ & 46 & $\mathsmaller{\geq}1$h & 53 & 0.00\\
\texttt{segment} & \multicolumn{1}{r}{2310} & \multicolumn{1}{r}{235}  & 0 & 0.03$^*$ & 0 & 0.13$^*$ & 0 & 2.3$^*$ & 0 & 4.1$^*$ & 4 & $\mathsmaller{\geq}1$h & 5 & 0.01\\
\texttt{soybean} & \multicolumn{1}{r}{630} & \multicolumn{1}{r}{50}  & 29 & 0.01$^*$ & 29 & 0.02$^*$ & 29 & 0.29$^*$ & 29 & 2.3$^*$ & 31 & $\mathsmaller{\geq}1$h & 47 & 0.00\\
\texttt{splice-1} & \multicolumn{1}{r}{3190} & \multicolumn{1}{r}{287}  & 224 & 9.8$^*$ & 224 & 5.3$^*$ & 224 & 114$^*$ & 224 & 173$^*$ & 453 & $\mathsmaller{\geq}1$h & 279 & 0.03\\
\texttt{surgical-deepnet} & \multicolumn{1}{r}{14635} & \multicolumn{1}{r}{6047}  & 2512 & 953 & 2512 & 3523 & - & - & 2512 & $\mathsmaller{\geq}1$h & - & - & 2924 & 5.7\\
\texttt{taiwan\_binarised} & \multicolumn{1}{r}{30000} & \multicolumn{1}{r}{205}  & 5326 & 48$^*$ & 5326 & 45$^*$ & 5326 & 526$^*$ & 5326 & 190$^*$ & 6636 & 1639 & 5346 & 0.26\\
\texttt{tic-tac-toe} & \multicolumn{1}{r}{958} & \multicolumn{1}{r}{27}  & 216 & 0.01$^*$ & 216 & 0.02$^*$ & 216 & 0.13$^*$ & 216 & 1.8$^*$ & 232 & $\mathsmaller{\geq}1$h & 236 & 0.00\\
\texttt{titanic} & \multicolumn{1}{r}{887} & \multicolumn{1}{r}{333}  & 143 & 6.7$^*$ & 143 & 11$^*$ & 143 & 167$^*$ & 143 & 173$^*$ & 150 & $\mathsmaller{\geq}1$h & 148 & 0.01\\
\texttt{vehicle} & \multicolumn{1}{r}{846} & \multicolumn{1}{r}{252}  & 26 & 0.93$^*$ & 26 & 2.2$^*$ & 26 & 64$^*$ & 26 & 66$^*$ & 42 & $\mathsmaller{\geq}1$h & 66 & 0.01\\
\texttt{vote} & \multicolumn{1}{r}{435} & \multicolumn{1}{r}{48}  & 12 & 0.02$^*$ & 12 & 0.02$^*$ & 12 & 0.34$^*$ & 12 & 2.6$^*$ & 13 & $\mathsmaller{\geq}1$h & 14 & 0.00\\
\texttt{weather-aus} & \multicolumn{1}{r}{142193} & \multicolumn{1}{r}{4759}  & 1756 & 14 & 1756 & 611 & - & - & 1756 & $\mathsmaller{\geq}1$h & - & - & 1761 & 20\\
\texttt{wine1} & \multicolumn{1}{r}{178} & \multicolumn{1}{r}{1276}  & 43 & 16$^*$ & 43 & 9.0$^*$ & - & - & 43 & $\mathsmaller{\geq}1$h & 44 & $\mathsmaller{\geq}1$h & 45 & 0.00\\
\texttt{wine2} & \multicolumn{1}{r}{178} & \multicolumn{1}{r}{1276}  & 49 & 17$^*$ & 49 & 5.8$^*$ & - & - & 49 & $\mathsmaller{\geq}1$h & 57 & $\mathsmaller{\geq}1$h & 52 & 0.00\\
\texttt{wine3} & \multicolumn{1}{r}{178} & \multicolumn{1}{r}{1276}  & 33 & 16$^*$ & 33 & 8.4$^*$ & - & - & 33 & $\mathsmaller{\geq}1$h & 35 & $\mathsmaller{\geq}1$h & 35 & 0.00\\
\texttt{yeast} & \multicolumn{1}{r}{1484} & \multicolumn{1}{r}{89}  & 403 & 0.07$^*$ & 403 & 0.34$^*$ & 403 & 6.1$^*$ & 403 & 7.7$^*$ & 434 & $\mathsmaller{\geq}1$h & 418 & 0.00\\
\bottomrule
\end{tabular}

\end{normalsize}
\end{center}
\caption{\label{tab:all} Comparison with state of the art: depth 3}
\end{table}

\medskip

\begin{table}[htbp]
\begin{center}
\begin{normalsize}
\tabcolsep=4pt
\begin{tabular}{lrrrrrrrrrrrr}
\toprule
\multirow{2}{*}{}&  \multicolumn{2}{c}{\budalg} & \multicolumn{2}{c}{\murtree} & \multicolumn{2}{c}{\dleight} & \multicolumn{2}{c}{\cp} & \multicolumn{2}{c}{binoct} & \multicolumn{2}{c}{\cart}\\
\cmidrule(rr){2-3}\cmidrule(rr){4-5}\cmidrule(rr){6-7}\cmidrule(rr){8-9}\cmidrule(rr){10-11}\cmidrule(rr){12-13}
& \multicolumn{1}{c}{error} & \multicolumn{1}{c}{cpu} & \multicolumn{1}{c}{error} & \multicolumn{1}{c}{cpu} & \multicolumn{1}{c}{error} & \multicolumn{1}{c}{cpu} & \multicolumn{1}{c}{error} & \multicolumn{1}{c}{cpu} & \multicolumn{1}{c}{error} & \multicolumn{1}{c}{cpu} & \multicolumn{1}{c}{error} & \multicolumn{1}{c}{cpu} \\
\midrule

\texttt{monk3-bin} & 4 & 0.00$^*$ & 4 & 0.01$^*$ & 4 & 0.01$^*$ & 4 & 1.0$^*$ & - & - & 5 & 0.00\\
\texttt{monk1-bin} & 2 & 0.00$^*$ & 2 & 0.00$^*$ & 2 & 0.01$^*$ & 2 & 1.5$^*$ & - & - & 11 & 0.00\\
\texttt{hepatitis} & 3 & 0.32$^*$ & 3 & 0.47$^*$ & 3 & 28$^*$ & 3 & 70$^*$ & 11 & 510 & 12 & 0.00\\
\texttt{lymph} & 3 & 0.74$^*$ & 3 & 0.46$^*$ & 3 & 14$^*$ & 3 & 64$^*$ & 7 & 2987 & 10 & 0.00\\
\texttt{iris-bin} & 1 & 0.00$^*$ & 1 & 0.00$^*$ & 1 & 0.00$^*$ & 1 & 0.92$^*$ & - & - & 1 & 0.00\\
\texttt{monk2-bin} & 31 & 0.01$^*$ & 31 & 0.01$^*$ & 31 & 0.04$^*$ & 31 & 2.1$^*$ & - & - & 50 & 0.00\\
\texttt{wine3} & 28 & 33 & 28 & 2008$^*$ & - & - & 30 & $\mathsmaller{\geq}1$h & 32 & 3388 & 32 & 0.01\\
\texttt{wine1} & 37 & 1674 & 37 & 1226$^*$ & - & - & 39 & $\mathsmaller{\geq}1$h & 45 & 3506 & 42 & 0.01\\
\texttt{wine2} & 43 & 17 & 43 & 763$^*$ & - & - & 46 & $\mathsmaller{\geq}1$h & 57 & 3232 & 47 & 0.01\\
\texttt{wine-bin} & 0 & 0.00$^*$ & 0 & 0.00$^*$ & 0 & 0.05$^*$ & 0 & 0.06$^*$ & - & - & 1 & 0.00\\
\texttt{audiology} & 1 & 4.0$^*$ & 1 & 3.9$^*$ & 1 & 128$^*$ & 1 & 773$^*$ & 2 & 2687 & 3 & 0.00\\
\texttt{heart-cleveland} & 25 & 3.1$^*$ & 25 & 3.3$^*$ & 25 & 154$^*$ & 25 & 391$^*$ & 37 & 2750 & 38 & 0.00\\
\texttt{primary-tumor} & 34 & 0.03$^*$ & 34 & 0.09$^*$ & 34 & 2.0$^*$ & 34 & 5.6$^*$ & 38 & 3132 & 44 & 0.00\\
\texttt{Ionosphere-bin} & 12 & 137$^*$ & 12 & 49$^*$ & - & - & 12 & 2264$^*$ & - & - & 30 & 0.00\\
\texttt{ionosphere} & 7 & 730$^*$ & 7 & 748$^*$ & - & - & 8 & $\mathsmaller{\geq}1$h & 24 & 751 & 27 & 0.01\\
\texttt{vote} & 5 & 1.2$^*$ & 5 & 0.36$^*$ & 5 & 7.6$^*$ & 5 & 21$^*$ & 12 & 3311 & 8 & 0.00\\
\texttt{forest-fires} & 173 & 15 & 171 & 1850$^*$ & - & - & 179 & $\mathsmaller{\geq}1$h & 196 & 3356 & 186 & 0.01\\
\texttt{balance-scale-bin} & 48 & 0.04$^*$ & 48 & 0.04$^*$ & 48 & 0.22$^*$ & 48 & 1.8$^*$ & - & - & 49 & 0.00\\
\texttt{soybean} & 14 & 0.62$^*$ & 14 & 0.24$^*$ & 14 & 5.1$^*$ & 14 & 22$^*$ & 22 & 2906 & 32 & 0.00\\
\texttt{australian-credit} & 56 & 10$^*$ & 56 & 12$^*$ & 56 & 470$^*$ & 56 & 1170$^*$ & 83 & 3258 & 74 & 0.00\\
\texttt{breast-wisconsin} & 7 & 3.1$^*$ & 7 & 4.2$^*$ & 7 & 245$^*$ & 7 & 662$^*$ & 15 & 3460 & 16 & 0.00\\
\texttt{breast-cancer} & 16 & 9.6$^*$ & 16 & 2.3$^*$ & 16 & 28$^*$ & 16 & 219$^*$ & 22 & 2746 & 21 & 0.00\\
\texttt{IndiansDiabetes-bin} & 149 & 0.90$^*$ & 149 & 1.1$^*$ & 149 & 7.3$^*$ & 149 & 16$^*$ & - & - & 166 & 0.00\\
\texttt{diabetes} & 137 & 5.7$^*$ & 137 & 15$^*$ & 137 & 550$^*$ & 137 & 1001$^*$ & 180 & 2663 & 166 & 0.00\\
\texttt{anneal} & 91 & 1.5$^*$ & 91 & 1.8$^*$ & 91 & 102$^*$ & 91 & 193$^*$ & 108 & 2954 & 135 & 0.00\\
\texttt{vehicle} & 12 & 71$^*$ & 12 & 72$^*$ & - & - & 12 & $\mathsmaller{\geq}1$h & 30 & 3410 & 28 & 0.01\\
\texttt{titanic} & 119 & 1604$^*$ & 119 & 311$^*$ & - & - & 119 & $\mathsmaller{\geq}1$h & 135 & 3501 & 134 & 0.01\\
\texttt{tic-tac-toe} & 137 & 0.38$^*$ & 137 & 0.21$^*$ & 137 & 1.8$^*$ & 137 & 7.2$^*$ & 162 & 2511 & 150 & 0.00\\
\texttt{tic-tac-toe-bin} & 137 & 0.09$^*$ & 137 & 0.09$^*$ & 137 & 0.36$^*$ & 137 & 3.7$^*$ & - & - & 150 & 0.00\\
\texttt{german-credit} & 204 & 28$^*$ & 204 & 17$^*$ & 204 & 423$^*$ & 204 & 1008$^*$ & 236 & 3306 & 231 & 0.00\\
\texttt{biodeg-bin} & 128 & 1511$^*$ & 128 & 500$^*$ & - & - & 129 & $\mathsmaller{\geq}1$h & - & - & 148 & 0.01\\
\texttt{messidor-bin} & 332 & 21$^*$ & 332 & 19$^*$ & 332 & 245$^*$ & 332 & 269$^*$ & - & - & 364 & 0.00\\
\texttt{banknote-bin} & 13 & 0.08$^*$ & 13 & 0.21$^*$ & 13 & 0.78$^*$ & 13 & 4.2$^*$ & - & - & 38 & 0.00\\
\texttt{yeast} & 366 & 3.4$^*$ & 366 & 10.0$^*$ & 366 & 257$^*$ & 366 & 386$^*$ & 438 & 888 & 394 & 0.01\\
\texttt{winequality-red-bin} & 4 & 0.62$^*$ & 4 & 0.79$^*$ & 4 & 4.3$^*$ & 4 & 12$^*$ & - & - & 8 & 0.00\\
\texttt{car\_evaluation-bin} & 130 & 0.02$^*$ & 130 & 0.08$^*$ & 130 & 0.13$^*$ & 130 & 1.3$^*$ & - & - & 130 & 0.00\\
\texttt{car} & 136 & 0.19$^*$ & 136 & 0.15$^*$ & 136 & 0.36$^*$ & 136 & 2.8$^*$ & 178 & 871 & 178 & 0.00\\
\texttt{segment} & 0 & 0.00$^*$ & 0 & 0.26$^*$ & 0 & 1.6$^*$ & 0 & 2.5$^*$ & 1 & 3501 & 1 & 0.01\\
\texttt{seismic\_bumps-bin} & 148 & 22$^*$ & 148 & 30$^*$ & 148 & 290$^*$ & 148 & 303$^*$ & - & - & 158 & 0.01\\
\texttt{splice-1} & 141 & 3241$^*$ & 141 & 949$^*$ & - & - & 141 & $\mathsmaller{\geq}1$h & 568 & 3416 & 141 & 0.03\\
\texttt{chess-bin} & 0 & 0.00$^*$ & 0 & 0.03$^*$ & 0 & 0.00$^*$ & 0 & 0.07$^*$ & - & - & 0 & 0.00\\
\texttt{kr-vs-kp} & 144 & 2.8$^*$ & 144 & 6.7$^*$ & 144 & 88$^*$ & 144 & 141$^*$ & 189 & 2850 & 189 & 0.01\\
\texttt{hypothyroid} & 53 & 2.9$^*$ & 53 & 11$^*$ & 53 & 181$^*$ & 53 & 254$^*$ & 55 & 3071 & 53 & 0.01\\
\texttt{Statlog\_satellite-bin} & 111 & 3571 & 111 & 3170 & - & - & 136 & $\mathsmaller{\geq}1$h & - & - & 204 & 0.08\\
\texttt{bank\_conv-bin} & 392 & 1963$^*$ & 392 & 759$^*$ & - & - & 392 & $\mathsmaller{\geq}1$h & - & - & 408 & 0.04\\
\texttt{spambase-bin} & 590 & 7.7 & 590 & 937$^*$ & - & - & 590 & $\mathsmaller{\geq}1$h & - & - & 624 & 0.06\\
\texttt{compas\_discretized} & 1954 & 0.07$^*$ & 1954 & 0.76$^*$ & 1954 & 3.5$^*$ & 1954 & 6.3$^*$ & 1991 & 3390 & 1997 & 0.01\\
\texttt{pendigits} & 13 & 230$^*$ & 13 & 395$^*$ & - & - & 14 & $\mathsmaller{\geq}1$h & 780 & 0.00 & 25 & 0.07\\
\texttt{mushroom} & 0 & 0.00$^*$ & 0 & 3.2$^*$ & 0 & 41$^*$ & 0 & 0.07$^*$ & 192 & 3354 & 4 & 0.02\\
\texttt{surgical-deepnet} & 2269 & 49 & 2664 & 3593 & - & - & 3690 & $\mathsmaller{\geq}1$h & - & - & 2704 & 6.2\\
\texttt{HTRU\_2-bin} & 385 & 74$^*$ & 385 & 87$^*$ & 385 & 450$^*$ & 385 & 295$^*$ & - & - & 409 & 0.05\\
\texttt{magic04-bin} & 3112 & 232$^*$ & 3112 & 202$^*$ & 3112 & 1296$^*$ & 3112 & 800$^*$ & - & - & 3350 & 0.07\\
\texttt{letter\_recognition-bin} & 105 & 405 & 105 & 2956$^*$ & 122 & $\mathsmaller{\geq}1$h & 105 & $\mathsmaller{\geq}1$h & - & - & 133 & 0.38\\
\texttt{letter} & 261 & 1185$^*$ & 261 & 1624$^*$ & 335 & $\mathsmaller{\geq}1$h & 261 & $\mathsmaller{\geq}1$h & 813 & 0.00 & 462 & 0.20\\
\texttt{taiwan\_binarised} & 5273 & 6.2 & 5273 & 3075$^*$ & 5307 & $\mathsmaller{\geq}1$h & 5273 & $\mathsmaller{\geq}1$h & 6521 & 75 & 5306 & 0.27\\
\texttt{default\_credit-bin} & 5270 & 209 & 5291 & 943 & 5306 & $\mathsmaller{\geq}1$h & 5270 & $\mathsmaller{\geq}1$h & - & - & 5306 & 0.69\\
\texttt{adult\_discretized} & 4609 & 14$^*$ & 4609 & 18$^*$ & 4609 & 271$^*$ & 4609 & 246$^*$ & 5659 & 3392 & 5022 & 0.06\\
\texttt{Statlog\_shuttle-bin} & 0 & 0.64$^*$ & 0 & 267$^*$ & 1 & $\mathsmaller{\geq}1$h & 0 & 42$^*$ & - & - & 36 & 2.4\\
\texttt{bank} & 4314 & 290 & 5287 & 0.01 & 4808 & $\mathsmaller{\geq}1$h & 5289 & $\mathsmaller{\geq}1$h & - & - & 4420 & 32\\
\texttt{mnist\_7} & 2793 & 52 & 3483 & 262 & 4546 & $\mathsmaller{\geq}1$h & 6265 & $\mathsmaller{\geq}1$h & - & - & 3218 & 3.9\\
\texttt{mnist\_5} & 3312 & 219 & 3539 & 79 & 4373 & $\mathsmaller{\geq}1$h & 5421 & $\mathsmaller{\geq}1$h & - & - & 3648 & 3.8\\
\texttt{mnist\_1} & 2332 & 2248 & 3460 & 1936 & 4551 & $\mathsmaller{\geq}1$h & 6742 & $\mathsmaller{\geq}1$h & - & - & 2501 & 3.6\\
\texttt{mnist\_2} & 3358 & 169 & 3927 & 3307 & 4289 & $\mathsmaller{\geq}1$h & 5958 & $\mathsmaller{\geq}1$h & - & - & 4326 & 3.1\\
\texttt{mnist\_3} & 3485 & 2225 & 4353 & 838 & 4900 & $\mathsmaller{\geq}1$h & 6131 & $\mathsmaller{\geq}1$h & - & - & 4367 & 4.9\\
\texttt{mnist\_9} & 3977 & 2061 & 4590 & 268 & 5252 & $\mathsmaller{\geq}1$h & 5949 & $\mathsmaller{\geq}1$h & - & - & 4231 & 3.1\\
\texttt{mnist\_8} & 3165 & 1206 & 3583 & 203 & 4609 & $\mathsmaller{\geq}1$h & 5851 & $\mathsmaller{\geq}1$h & - & - & 3987 & 4.5\\
\texttt{mnist\_4} & 3670 & 2476 & 4721 & 2368 & 5580 & $\mathsmaller{\geq}1$h & 5842 & $\mathsmaller{\geq}1$h & - & - & 4129 & 3.2\\
\texttt{mnist\_6} & 1940 & 2752 & 2720 & 1793 & 2755 & $\mathsmaller{\geq}1$h & 5918 & $\mathsmaller{\geq}1$h & - & - & 2251 & 4.1\\
\texttt{mnist\_0} & 2173 & 2158 & 2556 & 2059 & 3319 & $\mathsmaller{\geq}1$h & 5923 & $\mathsmaller{\geq}1$h & - & - & 2311 & 3.8\\
\texttt{hand\_posture-bin} & 4896 & 976 & 9839 & 1688 & 11021 & $\mathsmaller{\geq}1$h & 16265 & $\mathsmaller{\geq}1$h & - & - & 6098 & 27\\
\texttt{weather-aus} & 1749 & 2525 & 2404 & 3147 & - & - & 1752 & $\mathsmaller{\geq}1$h & - & - & 1761 & 20\\
\bottomrule
\end{tabular}

\end{normalsize}
\end{center}
\caption{\label{tab:all} Comparison with state of the art: depth 4}
\end{table}

\medskip

\begin{table}[htbp]
\begin{center}
\begin{normalsize}
\tabcolsep=4pt
\begin{tabular}{lccrrrrrrrrrrr}
\toprule
\multirow{2}{*}{}& && \multicolumn{3}{c}{\budalg} & \multicolumn{3}{c}{\murtree} & \multicolumn{3}{c}{\dleight} & \multicolumn{2}{c}{\cart}\\
\cmidrule(rr){4-6}\cmidrule(rr){7-9}\cmidrule(rr){10-12}\cmidrule(rr){13-14}
&\multirow{1}{*}{$\#ex.$} & \multirow{1}{*}{\#feat.} &  \multicolumn{1}{c}{error} & \multicolumn{1}{c}{cpu} & \multicolumn{1}{c}{opt.} & \multicolumn{1}{c}{error} & \multicolumn{1}{c}{cpu} & \multicolumn{1}{c}{opt.} & \multicolumn{1}{c}{error} & \multicolumn{1}{c}{cpu} & \multicolumn{1}{c}{opt.} & \multicolumn{1}{c}{error} & \multicolumn{1}{c}{cpu} \\
\midrule

\texttt{adult\_discretized} & \multicolumn{1}{r}{30299} & \multicolumn{1}{r}{59}  & 4423 & 725.1 & 1 & 4423 & 794.1 & 1 & 4442 & 3600.0 & 0 & 4728 & \textbf{0.1}\\
\texttt{anneal} & \multicolumn{1}{r}{812} & \multicolumn{1}{r}{93}  & 70 & 43.9 & 1 & 70 & 148.1 & 1 & - & - & 0 & 123 & \textbf{0.0}\\
\texttt{audiology} & \multicolumn{1}{r}{216} & \multicolumn{1}{r}{148}  & 0 & \textbf{0.0} & 1 & 0 & 0.0 & 1 & 0 & 0.0 & 1 & 2 & 0.0\\
\texttt{australian-credit} & \multicolumn{1}{r}{653} & \multicolumn{1}{r}{125}  & 39 & 657.5 & 1 & 39 & 871.9 & 1 & - & - & 0 & 64 & \textbf{0.0}\\
\texttt{bank} & \multicolumn{1}{r}{45211} & \multicolumn{1}{r}{9531}  & \textbf{4187} & 1151.9 & 0 & 4365 & 2092.6 & 0 & 4809 & 3603.0 & 0 & 4358 & \textbf{47.1}\\
\texttt{breast-cancer} & \multicolumn{1}{r}{683} & \multicolumn{1}{r}{89}  & 6 & 725.0 & 1 & 6 & 72.1 & 1 & 6 & 438.0 & 1 & 16 & \textbf{0.0}\\
\texttt{breast-wisconsin} & \multicolumn{1}{r}{683} & \multicolumn{1}{r}{120}  & 0 & 19.9 & 1 & 0 & 72.0 & 1 & - & - & 0 & 13 & \textbf{0.0}\\
\texttt{car} & \multicolumn{1}{r}{1728} & \multicolumn{1}{r}{21}  & 86 & 2.4 & 1 & 86 & 1.2 & 1 & 86 & 2.7 & 1 & 106 & \textbf{0.0}\\
\texttt{compas\_discretized} & \multicolumn{1}{r}{6167} & \multicolumn{1}{r}{25}  & 1919 & 1.1 & 1 & 1919 & 10.8 & 1 & 1919 & 26.4 & 1 & 1968 & \textbf{0.0}\\
\texttt{diabetes} & \multicolumn{1}{r}{768} & \multicolumn{1}{r}{112}  & 106 & 312.4 & 1 & 106 & 919.8 & 1 & - & - & 0 & 141 & \textbf{0.0}\\
\texttt{forest-fires} & \multicolumn{1}{r}{517} & \multicolumn{1}{r}{989}  & 156 & 777.0 & 0 & \textbf{149} & 2977.4 & 0 & - & - & 0 & 177 & \textbf{0.0}\\
\texttt{german-credit} & \multicolumn{1}{r}{1000} & \multicolumn{1}{r}{112}  & 161 & 2741.0 & 1 & 161 & 973.2 & 1 & - & - & 0 & 209 & \textbf{0.0}\\
\texttt{heart-cleveland} & \multicolumn{1}{r}{296} & \multicolumn{1}{r}{95}  & 7 & 93.5 & 1 & 7 & 100.7 & 1 & - & - & 0 & 26 & \textbf{0.0}\\
\texttt{hepatitis} & \multicolumn{1}{r}{137} & \multicolumn{1}{r}{68}  & 0 & 0.1 & 1 & 0 & 0.2 & 1 & 0 & 71.4 & 1 & 8 & \textbf{0.0}\\
\texttt{hypothyroid} & \multicolumn{1}{r}{3247} & \multicolumn{1}{r}{88}  & 44 & 87.4 & 1 & 44 & 343.4 & 1 & - & - & 0 & 50 & \textbf{0.0}\\
\texttt{ionosphere} & \multicolumn{1}{r}{351} & \multicolumn{1}{r}{445}  & 0 & 506.0 & 1 & 0 & 1340.0 & 1 & - & - & 0 & 17 & \textbf{0.0}\\
\texttt{kr-vs-kp} & \multicolumn{1}{r}{3196} & \multicolumn{1}{r}{73}  & 81 & 64.6 & 1 & 81 & 150.5 & 1 & - & - & 0 & 189 & \textbf{0.0}\\
\texttt{letter} & \multicolumn{1}{r}{20000} & \multicolumn{1}{r}{224}  & \textbf{168} & 3082.5 & 0 & 190 & 549.0 & 0 & 352 & 3600.0 & 0 & 335 & \textbf{0.3}\\
\texttt{lymph} & \multicolumn{1}{r}{148} & \multicolumn{1}{r}{68}  & 0 & \textbf{0.0} & 1 & 0 & 0.0 & 1 & 0 & 14.0 & 1 & 4 & 0.0\\
\texttt{mnist\_0} & \multicolumn{1}{r}{60000} & \multicolumn{1}{r}{784}  & \textbf{1714} & 283.8 & 0 & 2066 & 2148.9 & 0 & 3319 & 3600.2 & 0 & 2021 & \textbf{4.5}\\
\texttt{mnist\_1} & \multicolumn{1}{r}{60000} & \multicolumn{1}{r}{784}  & \textbf{1585} & 3111.0 & 0 & 1790 & 993.0 & 0 & 4029 & 3600.2 & 0 & 1965 & \textbf{3.6}\\
\texttt{mnist\_2} & \multicolumn{1}{r}{60000} & \multicolumn{1}{r}{784}  & 3118 & 3229.5 & 0 & \textbf{2963} & 2670.8 & 0 & 4026 & 3600.2 & 0 & 3676 & \textbf{3.9}\\
\texttt{mnist\_3} & \multicolumn{1}{r}{60000} & \multicolumn{1}{r}{784}  & \textbf{2893} & 1935.6 & 0 & 3184 & 398.0 & 0 & 4900 & 3600.3 & 0 & 3768 & \textbf{6.0}\\
\texttt{mnist\_4} & \multicolumn{1}{r}{60000} & \multicolumn{1}{r}{784}  & \textbf{2864} & 707.9 & 0 & 3164 & 107.1 & 0 & 5580 & 3600.2 & 0 & 3619 & \textbf{4.5}\\
\texttt{mnist\_5} & \multicolumn{1}{r}{60000} & \multicolumn{1}{r}{784}  & \textbf{3138} & 2411.4 & 0 & 3163 & 2007.3 & 0 & 4376 & 3600.2 & 0 & 3479 & \textbf{5.8}\\
\texttt{mnist\_6} & \multicolumn{1}{r}{60000} & \multicolumn{1}{r}{784}  & \textbf{1485} & 2097.4 & 0 & 1653 & 645.5 & 0 & 2753 & 3600.2 & 0 & 1900 & \textbf{4.4}\\
\texttt{mnist\_7} & \multicolumn{1}{r}{60000} & \multicolumn{1}{r}{784}  & 2532 & 1792.6 & 0 & \textbf{2464} & 2363.1 & 0 & 4542 & 3600.2 & 0 & 2848 & \textbf{6.7}\\
\texttt{mnist\_8} & \multicolumn{1}{r}{60000} & \multicolumn{1}{r}{784}  & \textbf{2547} & 2846.6 & 0 & 2818 & 1149.5 & 0 & 4609 & 3600.2 & 0 & 3172 & \textbf{6.3}\\
\texttt{mnist\_9} & \multicolumn{1}{r}{60000} & \multicolumn{1}{r}{784}  & \textbf{3352} & 1695.0 & 0 & 3521 & 1368.5 & 0 & 5252 & 3600.2 & 0 & 3830 & \textbf{6.8}\\
\texttt{mushroom} & \multicolumn{1}{r}{8124} & \multicolumn{1}{r}{119}  & 0 & \textbf{0.0} & 1 & 0 & 0.0 & 1 & 0 & 35.6 & 1 & 3 & 0.0\\
\texttt{pendigits} & \multicolumn{1}{r}{7494} & \multicolumn{1}{r}{216}  & 0 & 283.5 & 1 & 0 & 1294.7 & 1 & - & - & 0 & 11 & \textbf{0.1}\\
\texttt{primary-tumor} & \multicolumn{1}{r}{336} & \multicolumn{1}{r}{31}  & 26 & 0.4 & 1 & 26 & 1.5 & 1 & 26 & 24.0 & 1 & 35 & \textbf{0.0}\\
\texttt{segment} & \multicolumn{1}{r}{2310} & \multicolumn{1}{r}{235}  & 0 & \textbf{0.0} & 1 & 0 & 0.0 & 1 & 0 & 1.0 & 1 & 1 & 0.0\\
\texttt{soybean} & \multicolumn{1}{r}{630} & \multicolumn{1}{r}{50}  & 8 & 19.6 & 1 & 8 & 7.6 & 1 & 8 & 63.1 & 1 & 23 & \textbf{0.0}\\
\texttt{splice-1} & \multicolumn{1}{r}{3190} & \multicolumn{1}{r}{287}  & 101 & 23.8 & 0 & \textbf{100} & 3308.2 & 0 & - & - & 0 & 117 & \textbf{0.0}\\
\texttt{surgical-deepnet} & \multicolumn{1}{r}{14635} & \multicolumn{1}{r}{6047}  & \textbf{2131} & 2167.6 & 0 & 2337 & 400.4 & 0 & - & - & 0 & 2245 & \textbf{8.4}\\
\texttt{taiwan\_binarised} & \multicolumn{1}{r}{30000} & \multicolumn{1}{r}{205}  & \textbf{5200} & 104.6 & 0 & 5261 & 37.8 & 0 & 5412 & 3600.0 & 0 & 5280 & \textbf{0.4}\\
\texttt{tic-tac-toe} & \multicolumn{1}{r}{958} & \multicolumn{1}{r}{27}  & 63 & 10.2 & 1 & 63 & 2.3 & 1 & 63 & 14.0 & 1 & 78 & \textbf{0.0}\\
\texttt{titanic} & \multicolumn{1}{r}{887} & \multicolumn{1}{r}{333}  & 95 & 1427.7 & 0 & 95 & 1370.9 & 0 & - & - & 0 & 130 & \textbf{0.0}\\
\texttt{vehicle} & \multicolumn{1}{r}{846} & \multicolumn{1}{r}{252}  & 1 & 690.2 & 0 & 1 & 1540.1 & 0 & - & - & 0 & 23 & \textbf{0.0}\\
\texttt{vote} & \multicolumn{1}{r}{435} & \multicolumn{1}{r}{48}  & 1 & 23.9 & 1 & 1 & 6.1 & 1 & 1 & 45.0 & 1 & 6 & \textbf{0.0}\\
\texttt{weather-aus} & \multicolumn{1}{r}{142193} & \multicolumn{1}{r}{4759}  & 1735 & 419.4 & 0 & 1735 & 1907.5 & 0 & - & - & 0 & 1751 & \textbf{25.6}\\
\texttt{wine1} & \multicolumn{1}{r}{178} & \multicolumn{1}{r}{1276}  & 33 & 1154.5 & 0 & 33 & 287.2 & 0 & - & - & 0 & 39 & \textbf{0.0}\\
\texttt{wine2} & \multicolumn{1}{r}{178} & \multicolumn{1}{r}{1276}  & 39 & 410.5 & 0 & \textbf{37} & 3399.8 & 0 & - & - & 0 & 44 & \textbf{0.0}\\
\texttt{wine3} & \multicolumn{1}{r}{178} & \multicolumn{1}{r}{1276}  & 25 & 16.7 & 0 & 25 & 25.2 & 0 & - & - & 0 & 30 & \textbf{0.0}\\
\texttt{yeast} & \multicolumn{1}{r}{1484} & \multicolumn{1}{r}{89}  & 313 & 139.2 & 1 & 313 & 557.7 & 1 & - & - & 0 & 367 & \textbf{0.0}\\
\bottomrule
\end{tabular}

\end{normalsize}
\end{center}
\caption{\label{tab:all} Comparison with state of the art: depth 5}
\end{table}

\medskip

\begin{table}[htbp]
\begin{center}
\begin{normalsize}
\tabcolsep=4pt
\begin{tabular}{lccrrrrrrrrrrrr}
\toprule
\multirow{2}{*}{}& && \multicolumn{2}{c}{\budalg} & \multicolumn{2}{c}{\murtree} & \multicolumn{2}{c}{\dleight} & \multicolumn{2}{c}{\cp} & \multicolumn{2}{c}{binoct} & \multicolumn{2}{c}{\cart}\\
\cmidrule(rr){4-5}\cmidrule(rr){6-7}\cmidrule(rr){8-9}\cmidrule(rr){10-11}\cmidrule(rr){12-13}\cmidrule(rr){14-15}
&\multirow{1}{*}{$\#ex.$} & \multirow{1}{*}{\#feat.} &  \multicolumn{1}{c}{error} & \multicolumn{1}{c}{cpu} & \multicolumn{1}{c}{error} & \multicolumn{1}{c}{cpu} & \multicolumn{1}{c}{error} & \multicolumn{1}{c}{cpu} & \multicolumn{1}{c}{error} & \multicolumn{1}{c}{cpu} & \multicolumn{1}{c}{error} & \multicolumn{1}{c}{cpu} & \multicolumn{1}{c}{error} & \multicolumn{1}{c}{cpu} \\
\midrule

\texttt{adult\_discretized} & \multicolumn{1}{r}{30299} & \multicolumn{1}{r}{59}  & 4281 & 1326 & 4281 & 276 & - & - & 7511 & $\mathsmaller{\geq}1$h & - & - & 4532 & 0.08\\
\texttt{anneal} & \multicolumn{1}{r}{812} & \multicolumn{1}{r}{93}  & 51 & 1330$^*$ & 51 & 3115$^*$ & - & - & 187 & $\mathsmaller{\geq}1$h & 129 & 405 & 106 & 0.00\\
\texttt{audiology} & \multicolumn{1}{r}{216} & \multicolumn{1}{r}{148}  & 0 & 0.00$^*$ & 0 & 0.01$^*$ & 0 & 0.02$^*$ & 0 & 0.12$^*$ & 0 & $\mathsmaller{\geq}1$h$^*$ & 1 & 0.00\\
\texttt{australian-credit} & \multicolumn{1}{r}{653} & \multicolumn{1}{r}{125}  & 15 & 342 & 15 & 413 & - & - & 296 & $\mathsmaller{\geq}1$h & 286 & 476 & 56 & 0.00\\
\texttt{bank} & \multicolumn{1}{r}{45211} & \multicolumn{1}{r}{9531}  & \textbf{4046} & 339 & 4270 & 2917 & 4810 & $\mathsmaller{\geq}1$h & 5289 & $\mathsmaller{\geq}1$h & - & - & 4245 & 43\\
\texttt{breast-cancer} & \multicolumn{1}{r}{683} & \multicolumn{1}{r}{89}  & 1 & 3328 & 1 & 1451$^*$ & - & - & 3 & $\mathsmaller{\geq}1$h & 40 & 695 & 13 & 0.00\\
\texttt{breast-wisconsin} & \multicolumn{1}{r}{683} & \multicolumn{1}{r}{120}  & 0 & 5.9$^*$ & 0 & 27$^*$ & - & - & 1 & $\mathsmaller{\geq}1$h & 80 & 354 & 7 & 0.00\\
\texttt{car} & \multicolumn{1}{r}{1728} & \multicolumn{1}{r}{21}  & 36 & 27$^*$ & 36 & 5.3$^*$ & 36 & 7.9$^*$ & 36 & 222$^*$ & 444 & 696 & 90 & 0.00\\
\texttt{compas\_discretized} & \multicolumn{1}{r}{6167} & \multicolumn{1}{r}{25}  & 1887 & 17$^*$ & 1887 & 86$^*$ & 1887 & 161$^*$ & 1887 & 1049$^*$ & 2043 & 735 & 1955 & 0.01\\
\texttt{diabetes} & \multicolumn{1}{r}{768} & \multicolumn{1}{r}{112}  & \textbf{60} & 2706 & 62 & 425 & - & - & 268 & $\mathsmaller{\geq}1$h & 268 & 467 & 130 & 0.01\\
\texttt{forest-fires} & \multicolumn{1}{r}{517} & \multicolumn{1}{r}{989}  & \textbf{132} & 1934 & 137 & 2505 & - & - & 247 & $\mathsmaller{\geq}1$h & 270 & 402 & 171 & 0.02\\
\texttt{german-credit} & \multicolumn{1}{r}{1000} & \multicolumn{1}{r}{112}  & 101 & 2883 & 101 & 1720 & - & - & 300 & $\mathsmaller{\geq}1$h & 293 & 131 & 171 & 0.01\\
\texttt{heart-cleveland} & \multicolumn{1}{r}{296} & \multicolumn{1}{r}{95}  & 0 & 0.03$^*$ & 0 & 0.12$^*$ & - & - & 0 & 9.1$^*$ & 32 & 3399 & 15 & 0.00\\
\texttt{hepatitis} & \multicolumn{1}{r}{137} & \multicolumn{1}{r}{68}  & 0 & 0.00$^*$ & 0 & 0.01$^*$ & 0 & 30$^*$ & 0 & 1.8$^*$ & 7 & $\mathsmaller{\geq}1$h & 3 & 0.00\\
\texttt{hypothyroid} & \multicolumn{1}{r}{3247} & \multicolumn{1}{r}{88}  & 32 & 2391$^*$ & 32 & 1327 & - & - & 277 & $\mathsmaller{\geq}1$h & 2970 & 468 & 47 & 0.01\\
\texttt{ionosphere} & \multicolumn{1}{r}{351} & \multicolumn{1}{r}{445}  & 0 & 4.4$^*$ & 0 & 11$^*$ & - & - & 0 & 1204$^*$ & 61 & 210 & 11 & 0.01\\
\texttt{kr-vs-kp} & \multicolumn{1}{r}{3196} & \multicolumn{1}{r}{73}  & 45 & 1694$^*$ & 45 & 2782$^*$ & - & - & 76 & $\mathsmaller{\geq}1$h & 1669 & 474 & 184 & 0.01\\
\texttt{letter} & \multicolumn{1}{r}{20000} & \multicolumn{1}{r}{224}  & \textbf{118} & 2186 & 275 & 116 & 387 & $\mathsmaller{\geq}1$h & 813 & $\mathsmaller{\geq}1$h & - & - & 217 & 0.34\\
\texttt{lymph} & \multicolumn{1}{r}{148} & \multicolumn{1}{r}{68}  & 0 & 0.00$^*$ & 0 & 0.01$^*$ & 0 & 0.59$^*$ & 0 & 0.35$^*$ & 2 & $\mathsmaller{\geq}1$h & 1 & 0.00\\
\texttt{mnist\_0} & \multicolumn{1}{r}{60000} & \multicolumn{1}{r}{784}  & \textbf{1468} & 2513 & 1930 & 2810 & 3319 & $\mathsmaller{\geq}1$h & 5923 & $\mathsmaller{\geq}1$h & - & - & 1781 & 5.4\\
\texttt{mnist\_1} & \multicolumn{1}{r}{60000} & \multicolumn{1}{r}{784}  & \textbf{1167} & 1875 & 1778 & 2866 & 4551 & $\mathsmaller{\geq}1$h & 6742 & $\mathsmaller{\geq}1$h & - & - & 1542 & 5.1\\
\texttt{mnist\_2} & \multicolumn{1}{r}{60000} & \multicolumn{1}{r}{784}  & \textbf{2519} & 230 & 2687 & 1177 & 4232 & $\mathsmaller{\geq}1$h & 5958 & $\mathsmaller{\geq}1$h & - & - & 2818 & 5.6\\
\texttt{mnist\_3} & \multicolumn{1}{r}{60000} & \multicolumn{1}{r}{784}  & \textbf{2486} & 2793 & 2923 & 1740 & 4900 & $\mathsmaller{\geq}1$h & 6131 & $\mathsmaller{\geq}1$h & - & - & 2902 & 7.8\\
\texttt{mnist\_4} & \multicolumn{1}{r}{60000} & \multicolumn{1}{r}{784}  & \textbf{2180} & 3375 & 2973 & 1936 & 5580 & $\mathsmaller{\geq}1$h & 5842 & $\mathsmaller{\geq}1$h & - & - & 2543 & 4.4\\
\texttt{mnist\_5} & \multicolumn{1}{r}{60000} & \multicolumn{1}{r}{784}  & \textbf{2930} & 1759 & 3060 & 1259 & 4376 & $\mathsmaller{\geq}1$h & 5421 & $\mathsmaller{\geq}1$h & - & - & 3402 & 7.2\\
\texttt{mnist\_6} & \multicolumn{1}{r}{60000} & \multicolumn{1}{r}{784}  & \textbf{1278} & 2111 & 1474 & 2795 & 2750 & $\mathsmaller{\geq}1$h & 5918 & $\mathsmaller{\geq}1$h & - & - & 1686 & 5.5\\
\texttt{mnist\_7} & \multicolumn{1}{r}{60000} & \multicolumn{1}{r}{784}  & \textbf{2074} & 2012 & 2304 & 553 & 4543 & $\mathsmaller{\geq}1$h & 6265 & $\mathsmaller{\geq}1$h & - & - & 2163 & 5.2\\
\texttt{mnist\_8} & \multicolumn{1}{r}{60000} & \multicolumn{1}{r}{784}  & \textbf{2060} & 806 & 3228 & 84 & 4656 & $\mathsmaller{\geq}1$h & 5851 & $\mathsmaller{\geq}1$h & - & - & 2633 & 6.1\\
\texttt{mnist\_9} & \multicolumn{1}{r}{60000} & \multicolumn{1}{r}{784}  & \textbf{2879} & 2229 & 3327 & 1778 & 5252 & $\mathsmaller{\geq}1$h & 5949 & $\mathsmaller{\geq}1$h & - & - & 3366 & 6.6\\
\texttt{mushroom} & \multicolumn{1}{r}{8124} & \multicolumn{1}{r}{119}  & 0 & 0.00$^*$ & 0 & 0.02$^*$ & 0 & 32$^*$ & 0 & 0.10$^*$ & - & - & 3 & 0.03\\
\texttt{pendigits} & \multicolumn{1}{r}{7494} & \multicolumn{1}{r}{216}  & 0 & 0.01$^*$ & 0 & 0.31$^*$ & - & - & 0 & 24$^*$ & - & - & 5 & 0.07\\
\texttt{primary-tumor} & \multicolumn{1}{r}{336} & \multicolumn{1}{r}{31}  & 18 & 3.1$^*$ & 18 & 15$^*$ & 18 & 138$^*$ & 18 & 1726$^*$ & 30 & $\mathsmaller{\geq}1$h & 28 & 0.00\\
\texttt{segment} & \multicolumn{1}{r}{2310} & \multicolumn{1}{r}{235}  & 0 & 0.00$^*$ & 0 & 0.02$^*$ & 0 & 0.39$^*$ & 0 & 0.63$^*$ & 330 & 296 & 0 & 0.01\\
\texttt{soybean} & \multicolumn{1}{r}{630} & \multicolumn{1}{r}{50}  & 3 & 354$^*$ & 3 & 86$^*$ & 3 & 513$^*$ & 3 & $\mathsmaller{\geq}1$h & 13 & 3501 & 15 & 0.00\\
\texttt{splice-1} & \multicolumn{1}{r}{3190} & \multicolumn{1}{r}{287}  & \textbf{68} & $\mathsmaller{\geq}1$h & 80 & 1695 & - & - & 1535 & $\mathsmaller{\geq}1$h & 1655 & 627 & 87 & 0.04\\
\texttt{surgical-deepnet} & \multicolumn{1}{r}{14635} & \multicolumn{1}{r}{6047}  & \textbf{1767} & 2343 & 2110 & 231 & - & - & 3690 & $\mathsmaller{\geq}1$h & - & - & 1969 & 7.4\\
\texttt{taiwan\_binarised} & \multicolumn{1}{r}{30000} & \multicolumn{1}{r}{205}  & \textbf{5073} & 1473 & 5169 & 3580 & - & - & 6636 & $\mathsmaller{\geq}1$h & - & - & 5250 & 0.48\\
\texttt{tic-tac-toe} & \multicolumn{1}{r}{958} & \multicolumn{1}{r}{27}  & 12 & 126$^*$ & 12 & 17$^*$ & 12 & 47$^*$ & 12 & 1297$^*$ & 195 & 1838 & 49 & 0.00\\
\texttt{titanic} & \multicolumn{1}{r}{887} & \multicolumn{1}{r}{333}  & \textbf{78} & 1234 & 102 & 3598 & - & - & 342 & $\mathsmaller{\geq}1$h & 342 & 534 & 119 & 0.01\\
\texttt{vehicle} & \multicolumn{1}{r}{846} & \multicolumn{1}{r}{252}  & 0 & 0.08$^*$ & 0 & 0.50$^*$ & - & - & 218 & $\mathsmaller{\geq}1$h & 218 & 43 & 9 & 0.01\\
\texttt{vote} & \multicolumn{1}{r}{435} & \multicolumn{1}{r}{48}  & 0 & 0.00$^*$ & 0 & 0.01$^*$ & 0 & 0.55$^*$ & 0 & 4.0$^*$ & 7 & $\mathsmaller{\geq}1$h & 2 & 0.00\\
\texttt{weather-aus} & \multicolumn{1}{r}{142193} & \multicolumn{1}{r}{4759}  & \textbf{1713} & 418 & 1736 & 820 & - & - & 1761 & $\mathsmaller{\geq}1$h & - & - & 1734 & 22\\
\texttt{wine1} & \multicolumn{1}{r}{178} & \multicolumn{1}{r}{1276}  & 31 & 2113 & \textbf{30} & 2482 & - & - & 38 & $\mathsmaller{\geq}1$h & 59 & 341 & 36 & 0.01\\
\texttt{wine2} & \multicolumn{1}{r}{178} & \multicolumn{1}{r}{1276}  & 34 & 44 & 34 & 29 & - & - & 37 & $\mathsmaller{\geq}1$h & 71 & 305 & 41 & 0.01\\
\texttt{wine3} & \multicolumn{1}{r}{178} & \multicolumn{1}{r}{1276}  & 22 & 93 & 22 & 87 & - & - & 25 & $\mathsmaller{\geq}1$h & 48 & 283 & 27 & 0.01\\
\texttt{yeast} & \multicolumn{1}{r}{1484} & \multicolumn{1}{r}{89}  & 245 & 388 & 245 & 1668 & - & - & 463 & $\mathsmaller{\geq}1$h & 444 & 87 & 346 & 0.01\\
\bottomrule
\end{tabular}

\end{normalsize}
\end{center}
\caption{\label{tab:all} Comparison with state of the art: depth 6}
\end{table}

\medskip


When the maximum depth and number of feature is not too large, both algorithms are comparable, although \budalg is systematically faster. However, when the depth or the number of features grows, the best solution found by \dleight is often of much lower quality. In fact, in most cases, it reaches the time or memory limit without outputing a solution (the missing entries corresponds to \dleight reaching the 50GB memory limit). Notice that \budalg uses a tiny memory space (much lower than the size of the data set).



\begin{table}[htbp]
\begin{center}
\begin{normalsize}
\tabcolsep=5pt
\begin{tabular}{lccrrrrrrrrr}
\toprule
& && \multicolumn{3}{c}{\budalg} & \multicolumn{3}{c}{\murtree} & \multicolumn{3}{c}{\dleight}\\
\cmidrule(rr){4-6}\cmidrule(rr){7-9}\cmidrule(rr){10-12}
&\multirow{1}{*}{$\#ex.$} & \multirow{1}{*}{\#feat.} &  \multicolumn{1}{c}{error} & \multicolumn{1}{c}{time} & \multicolumn{1}{c}{opt.} & \multicolumn{1}{c}{error} & \multicolumn{1}{c}{time} & \multicolumn{1}{c}{opt.} & \multicolumn{1}{c}{error} & \multicolumn{1}{c}{time} & \multicolumn{1}{c}{opt.} \\
\midrule

\texttt{adult\_discretized} & \multicolumn{1}{r}{30299} & \multicolumn{1}{r}{59}  & \cellcolor{TealBlue!30}{5020} & \cellcolor{TealBlue!30}{\textbf{0.3}} & \cellcolor{TealBlue!30}{1.00} & \cellcolor{TealBlue!30}{5020} & 0.5 & \cellcolor{TealBlue!30}{1.00} & \cellcolor{TealBlue!30}{5020} & 10.1 & \cellcolor{TealBlue!30}{1.00}\\
\texttt{anneal} & \multicolumn{1}{r}{812} & \multicolumn{1}{r}{93}  & \cellcolor{TealBlue!30}{112} & \cellcolor{TealBlue!30}{\textbf{0.0}} & \cellcolor{TealBlue!30}{1.00} & \cellcolor{TealBlue!30}{112} & 0.1 & \cellcolor{TealBlue!30}{1.00} & \cellcolor{TealBlue!30}{112} & 2.4 & \cellcolor{TealBlue!30}{1.00}\\
\texttt{audiology} & \multicolumn{1}{r}{216} & \multicolumn{1}{r}{148}  & \cellcolor{TealBlue!30}{5} & \cellcolor{TealBlue!30}{\textbf{0.1}} & \cellcolor{TealBlue!30}{1.00} & \cellcolor{TealBlue!30}{5} & 0.2 & \cellcolor{TealBlue!30}{1.00} & \cellcolor{TealBlue!30}{5} & 4.5 & \cellcolor{TealBlue!30}{1.00}\\
\texttt{australian-credit} & \multicolumn{1}{r}{653} & \multicolumn{1}{r}{125}  & \cellcolor{TealBlue!30}{73} & \cellcolor{TealBlue!30}{\textbf{0.1}} & \cellcolor{TealBlue!30}{1.00} & \cellcolor{TealBlue!30}{73} & 0.4 & \cellcolor{TealBlue!30}{1.00} & \cellcolor{TealBlue!30}{73} & 9.6 & \cellcolor{TealBlue!30}{1.00}\\
\texttt{bank-un} & \multicolumn{1}{r}{45211} & \multicolumn{1}{r}{9531}  & \cellcolor{TealBlue!30}{\textbf{4453}} & 0.9 & \cellcolor{TealBlue!30}{0.00} & 5289 & \cellcolor{TealBlue!30}{\textbf{0.8}} & \cellcolor{TealBlue!30}{0.00} & 4805 & 3603.1 & \cellcolor{TealBlue!30}{0.00}\\
\texttt{breast-cancer-un} & \multicolumn{1}{r}{683} & \multicolumn{1}{r}{89}  & \cellcolor{TealBlue!30}{24} & 0.1 & \cellcolor{TealBlue!30}{1.00} & \cellcolor{TealBlue!30}{24} & \cellcolor{TealBlue!30}{\textbf{0.1}} & \cellcolor{TealBlue!30}{1.00} & \cellcolor{TealBlue!30}{24} & 1.0 & \cellcolor{TealBlue!30}{1.00}\\
\texttt{breast-wisconsin} & \multicolumn{1}{r}{683} & \multicolumn{1}{r}{120}  & \cellcolor{TealBlue!30}{15} & \cellcolor{TealBlue!30}{\textbf{0.1}} & \cellcolor{TealBlue!30}{1.00} & \cellcolor{TealBlue!30}{15} & 0.2 & \cellcolor{TealBlue!30}{1.00} & \cellcolor{TealBlue!30}{15} & 6.4 & \cellcolor{TealBlue!30}{1.00}\\
\texttt{car-un} & \multicolumn{1}{r}{1728} & \multicolumn{1}{r}{21}  & \cellcolor{TealBlue!30}{192} & 0.0 & \cellcolor{TealBlue!30}{1.00} & \cellcolor{TealBlue!30}{192} & \cellcolor{TealBlue!30}{\textbf{0.0}} & \cellcolor{TealBlue!30}{1.00} & \cellcolor{TealBlue!30}{192} & 0.0 & \cellcolor{TealBlue!30}{1.00}\\
\texttt{compas\_discretized} & \multicolumn{1}{r}{6167} & \multicolumn{1}{r}{25}  & \cellcolor{TealBlue!30}{2004} & \cellcolor{TealBlue!30}{\textbf{0.0}} & \cellcolor{TealBlue!30}{1.00} & \cellcolor{TealBlue!30}{2004} & 0.0 & \cellcolor{TealBlue!30}{1.00} & \cellcolor{TealBlue!30}{2004} & 0.2 & \cellcolor{TealBlue!30}{1.00}\\
\texttt{diabetes} & \multicolumn{1}{r}{768} & \multicolumn{1}{r}{112}  & \cellcolor{TealBlue!30}{162} & \cellcolor{TealBlue!30}{\textbf{0.1}} & \cellcolor{TealBlue!30}{1.00} & \cellcolor{TealBlue!30}{162} & 0.4 & \cellcolor{TealBlue!30}{1.00} & \cellcolor{TealBlue!30}{162} & 10.6 & \cellcolor{TealBlue!30}{1.00}\\
\texttt{forest-fires-un} & \multicolumn{1}{r}{517} & \multicolumn{1}{r}{989}  & \cellcolor{TealBlue!30}{193} & \cellcolor{TealBlue!30}{\textbf{19.9}} & \cellcolor{TealBlue!30}{1.00} & \cellcolor{TealBlue!30}{193} & 67.1 & \cellcolor{TealBlue!30}{1.00} & - & - & -\\
\texttt{german-credit} & \multicolumn{1}{r}{1000} & \multicolumn{1}{r}{112}  & \cellcolor{TealBlue!30}{236} & \cellcolor{TealBlue!30}{\textbf{0.3}} & \cellcolor{TealBlue!30}{1.00} & \cellcolor{TealBlue!30}{236} & 0.4 & \cellcolor{TealBlue!30}{1.00} & \cellcolor{TealBlue!30}{236} & 7.7 & \cellcolor{TealBlue!30}{1.00}\\
\texttt{heart-cleveland} & \multicolumn{1}{r}{296} & \multicolumn{1}{r}{95}  & \cellcolor{TealBlue!30}{41} & \cellcolor{TealBlue!30}{\textbf{0.1}} & \cellcolor{TealBlue!30}{1.00} & \cellcolor{TealBlue!30}{41} & 0.1 & \cellcolor{TealBlue!30}{1.00} & \cellcolor{TealBlue!30}{41} & 3.5 & \cellcolor{TealBlue!30}{1.00}\\
\texttt{hepatitis} & \multicolumn{1}{r}{137} & \multicolumn{1}{r}{68}  & \cellcolor{TealBlue!30}{10} & \cellcolor{TealBlue!30}{\textbf{0.0}} & \cellcolor{TealBlue!30}{1.00} & \cellcolor{TealBlue!30}{10} & 0.0 & \cellcolor{TealBlue!30}{1.00} & \cellcolor{TealBlue!30}{10} & 1.2 & \cellcolor{TealBlue!30}{1.00}\\
\texttt{hypothyroid} & \multicolumn{1}{r}{3247} & \multicolumn{1}{r}{88}  & \cellcolor{TealBlue!30}{61} & \cellcolor{TealBlue!30}{\textbf{0.1}} & \cellcolor{TealBlue!30}{1.00} & \cellcolor{TealBlue!30}{61} & 0.4 & \cellcolor{TealBlue!30}{1.00} & \cellcolor{TealBlue!30}{61} & 4.4 & \cellcolor{TealBlue!30}{1.00}\\
\texttt{ionosphere} & \multicolumn{1}{r}{351} & \multicolumn{1}{r}{445}  & \cellcolor{TealBlue!30}{22} & \cellcolor{TealBlue!30}{\textbf{4.3}} & \cellcolor{TealBlue!30}{1.00} & \cellcolor{TealBlue!30}{22} & 11.8 & \cellcolor{TealBlue!30}{1.00} & \cellcolor{TealBlue!30}{22} & 409.6 & \cellcolor{TealBlue!30}{1.00}\\
\texttt{kr-vs-kp} & \multicolumn{1}{r}{3196} & \multicolumn{1}{r}{73}  & \cellcolor{TealBlue!30}{198} & \cellcolor{TealBlue!30}{\textbf{0.1}} & \cellcolor{TealBlue!30}{1.00} & \cellcolor{TealBlue!30}{198} & 0.2 & \cellcolor{TealBlue!30}{1.00} & \cellcolor{TealBlue!30}{198} & 2.4 & \cellcolor{TealBlue!30}{1.00}\\
\texttt{letter} & \multicolumn{1}{r}{20000} & \multicolumn{1}{r}{224}  & \cellcolor{TealBlue!30}{369} & \cellcolor{TealBlue!30}{\textbf{10.3}} & \cellcolor{TealBlue!30}{1.00} & \cellcolor{TealBlue!30}{369} & 36.2 & \cellcolor{TealBlue!30}{1.00} & \cellcolor{TealBlue!30}{369} & 443.1 & \cellcolor{TealBlue!30}{1.00}\\
\texttt{lymph} & \multicolumn{1}{r}{148} & \multicolumn{1}{r}{68}  & \cellcolor{TealBlue!30}{12} & \cellcolor{TealBlue!30}{\textbf{0.0}} & \cellcolor{TealBlue!30}{1.00} & \cellcolor{TealBlue!30}{12} & 0.0 & \cellcolor{TealBlue!30}{1.00} & \cellcolor{TealBlue!30}{12} & 0.8 & \cellcolor{TealBlue!30}{1.00}\\
\texttt{mnist\_0} & \multicolumn{1}{r}{60000} & \multicolumn{1}{r}{784}  & \cellcolor{TealBlue!30}{2557} & 1722.8 & \cellcolor{TealBlue!30}{1.00} & \cellcolor{TealBlue!30}{2557} & \cellcolor{TealBlue!30}{\textbf{576.8}} & \cellcolor{TealBlue!30}{1.00} & 3319 & 3600.2 & 0.00\\
\texttt{mnist\_1} & \multicolumn{1}{r}{60000} & \multicolumn{1}{r}{784}  & \cellcolor{TealBlue!30}{3462} & 2043.6 & \cellcolor{TealBlue!30}{1.00} & \cellcolor{TealBlue!30}{3462} & \cellcolor{TealBlue!30}{\textbf{532.0}} & \cellcolor{TealBlue!30}{1.00} & 4552 & 3600.2 & 0.00\\
\texttt{mnist\_2} & \multicolumn{1}{r}{60000} & \multicolumn{1}{r}{784}  & \cellcolor{TealBlue!30}{3938} & 1751.6 & \cellcolor{TealBlue!30}{1.00} & \cellcolor{TealBlue!30}{3938} & \cellcolor{TealBlue!30}{\textbf{663.5}} & \cellcolor{TealBlue!30}{1.00} & 4289 & 3600.2 & 0.00\\
\texttt{mnist\_3} & \multicolumn{1}{r}{60000} & \multicolumn{1}{r}{784}  & \cellcolor{TealBlue!30}{4354} & 1733.0 & \cellcolor{TealBlue!30}{1.00} & \cellcolor{TealBlue!30}{4354} & \cellcolor{TealBlue!30}{\textbf{630.0}} & \cellcolor{TealBlue!30}{1.00} & 4974 & 3600.2 & 0.00\\
\texttt{mnist\_4} & \multicolumn{1}{r}{60000} & \multicolumn{1}{r}{784}  & \cellcolor{TealBlue!30}{4729} & 2044.4 & \cellcolor{TealBlue!30}{1.00} & \cellcolor{TealBlue!30}{4729} & \cellcolor{TealBlue!30}{\textbf{645.1}} & \cellcolor{TealBlue!30}{1.00} & 5580 & 3600.2 & 0.00\\
\texttt{mnist\_5} & \multicolumn{1}{r}{60000} & \multicolumn{1}{r}{784}  & \cellcolor{TealBlue!30}{3539} & 1985.7 & \cellcolor{TealBlue!30}{1.00} & \cellcolor{TealBlue!30}{3539} & \cellcolor{TealBlue!30}{\textbf{660.6}} & \cellcolor{TealBlue!30}{1.00} & 4379 & 3600.2 & 0.00\\
\texttt{mnist\_6} & \multicolumn{1}{r}{60000} & \multicolumn{1}{r}{784}  & \cellcolor{TealBlue!30}{2756} & 1960.2 & \cellcolor{TealBlue!30}{1.00} & \cellcolor{TealBlue!30}{2756} & \cellcolor{TealBlue!30}{\textbf{606.0}} & \cellcolor{TealBlue!30}{1.00} & \cellcolor{TealBlue!30}{2756} & 3600.2 & 0.00\\
\texttt{mnist\_7} & \multicolumn{1}{r}{60000} & \multicolumn{1}{r}{784}  & \cellcolor{TealBlue!30}{3483} & 1955.1 & \cellcolor{TealBlue!30}{1.00} & \cellcolor{TealBlue!30}{3483} & \cellcolor{TealBlue!30}{\textbf{559.8}} & \cellcolor{TealBlue!30}{1.00} & 4546 & 3600.2 & 0.00\\
\texttt{mnist\_8} & \multicolumn{1}{r}{60000} & \multicolumn{1}{r}{784}  & \cellcolor{TealBlue!30}{3583} & 2050.2 & \cellcolor{TealBlue!30}{1.00} & \cellcolor{TealBlue!30}{3583} & \cellcolor{TealBlue!30}{\textbf{580.5}} & \cellcolor{TealBlue!30}{1.00} & 4609 & 3600.2 & 0.00\\
\texttt{mnist\_9} & \multicolumn{1}{r}{60000} & \multicolumn{1}{r}{784}  & \cellcolor{TealBlue!30}{4590} & 2056.1 & \cellcolor{TealBlue!30}{1.00} & \cellcolor{TealBlue!30}{4590} & \cellcolor{TealBlue!30}{\textbf{613.5}} & \cellcolor{TealBlue!30}{1.00} & 5253 & 3600.2 & 0.00\\
\texttt{mushroom} & \multicolumn{1}{r}{8124} & \multicolumn{1}{r}{119}  & \cellcolor{TealBlue!30}{8} & 0.8 & \cellcolor{TealBlue!30}{1.00} & \cellcolor{TealBlue!30}{8} & \cellcolor{TealBlue!30}{\textbf{0.5}} & \cellcolor{TealBlue!30}{1.00} & \cellcolor{TealBlue!30}{8} & 6.3 & \cellcolor{TealBlue!30}{1.00}\\
\texttt{pendigits} & \multicolumn{1}{r}{7494} & \multicolumn{1}{r}{216}  & \cellcolor{TealBlue!30}{47} & \cellcolor{TealBlue!30}{\textbf{3.7}} & \cellcolor{TealBlue!30}{1.00} & \cellcolor{TealBlue!30}{47} & 11.8 & \cellcolor{TealBlue!30}{1.00} & \cellcolor{TealBlue!30}{47} & 134.2 & \cellcolor{TealBlue!30}{1.00}\\
\texttt{primary-tumor} & \multicolumn{1}{r}{336} & \multicolumn{1}{r}{31}  & \cellcolor{TealBlue!30}{46} & \cellcolor{TealBlue!30}{\textbf{0.0}} & \cellcolor{TealBlue!30}{1.00} & \cellcolor{TealBlue!30}{46} & 0.0 & \cellcolor{TealBlue!30}{1.00} & \cellcolor{TealBlue!30}{46} & 0.1 & \cellcolor{TealBlue!30}{1.00}\\
\texttt{segment} & \multicolumn{1}{r}{2310} & \multicolumn{1}{r}{235}  & \cellcolor{TealBlue!30}{0} & \cellcolor{TealBlue!30}{\textbf{0.0}} & \cellcolor{TealBlue!30}{1.00} & \cellcolor{TealBlue!30}{0} & 0.1 & \cellcolor{TealBlue!30}{1.00} & \cellcolor{TealBlue!30}{0} & 2.3 & \cellcolor{TealBlue!30}{1.00}\\
\texttt{soybean} & \multicolumn{1}{r}{630} & \multicolumn{1}{r}{50}  & \cellcolor{TealBlue!30}{29} & \cellcolor{TealBlue!30}{\textbf{0.0}} & \cellcolor{TealBlue!30}{1.00} & \cellcolor{TealBlue!30}{29} & 0.0 & \cellcolor{TealBlue!30}{1.00} & \cellcolor{TealBlue!30}{29} & 0.3 & \cellcolor{TealBlue!30}{1.00}\\
\texttt{splice-1} & \multicolumn{1}{r}{3190} & \multicolumn{1}{r}{287}  & \cellcolor{TealBlue!30}{224} & 10.2 & \cellcolor{TealBlue!30}{1.00} & \cellcolor{TealBlue!30}{224} & \cellcolor{TealBlue!30}{\textbf{5.3}} & \cellcolor{TealBlue!30}{1.00} & \cellcolor{TealBlue!30}{224} & 113.8 & \cellcolor{TealBlue!30}{1.00}\\
\texttt{surgical-deepnet-un} & \multicolumn{1}{r}{14635} & \multicolumn{1}{r}{6047}  & \cellcolor{TealBlue!30}{2512} & \cellcolor{TealBlue!30}{\textbf{909.4}} & \cellcolor{TealBlue!30}{0.00} & \cellcolor{TealBlue!30}{2512} & 3426.9 & \cellcolor{TealBlue!30}{0.00} & - & - & -\\
\texttt{taiwan\_binarised} & \multicolumn{1}{r}{30000} & \multicolumn{1}{r}{205}  & \cellcolor{TealBlue!30}{5326} & \cellcolor{TealBlue!30}{\textbf{45.2}} & \cellcolor{TealBlue!30}{1.00} & \cellcolor{TealBlue!30}{5326} & 46.6 & \cellcolor{TealBlue!30}{1.00} & \cellcolor{TealBlue!30}{5326} & 526.2 & \cellcolor{TealBlue!30}{1.00}\\
\texttt{tic-tac-toe} & \multicolumn{1}{r}{958} & \multicolumn{1}{r}{27}  & \cellcolor{TealBlue!30}{216} & \cellcolor{TealBlue!30}{\textbf{0.0}} & \cellcolor{TealBlue!30}{1.00} & \cellcolor{TealBlue!30}{216} & 0.0 & \cellcolor{TealBlue!30}{1.00} & \cellcolor{TealBlue!30}{216} & 0.1 & \cellcolor{TealBlue!30}{1.00}\\
\texttt{titanic-un} & \multicolumn{1}{r}{887} & \multicolumn{1}{r}{333}  & \cellcolor{TealBlue!30}{143} & \cellcolor{TealBlue!30}{\textbf{7.4}} & \cellcolor{TealBlue!30}{1.00} & \cellcolor{TealBlue!30}{143} & 11.8 & \cellcolor{TealBlue!30}{1.00} & \cellcolor{TealBlue!30}{143} & 166.7 & \cellcolor{TealBlue!30}{1.00}\\
\texttt{vehicle} & \multicolumn{1}{r}{846} & \multicolumn{1}{r}{252}  & \cellcolor{TealBlue!30}{26} & \cellcolor{TealBlue!30}{\textbf{1.1}} & \cellcolor{TealBlue!30}{1.00} & \cellcolor{TealBlue!30}{26} & 2.4 & \cellcolor{TealBlue!30}{1.00} & \cellcolor{TealBlue!30}{26} & 64.1 & \cellcolor{TealBlue!30}{1.00}\\
\texttt{vote} & \multicolumn{1}{r}{435} & \multicolumn{1}{r}{48}  & \cellcolor{TealBlue!30}{12} & 0.0 & \cellcolor{TealBlue!30}{1.00} & \cellcolor{TealBlue!30}{12} & \cellcolor{TealBlue!30}{\textbf{0.0}} & \cellcolor{TealBlue!30}{1.00} & \cellcolor{TealBlue!30}{12} & 0.3 & \cellcolor{TealBlue!30}{1.00}\\
\texttt{weather-aus-un} & \multicolumn{1}{r}{142193} & \multicolumn{1}{r}{4759}  & \cellcolor{TealBlue!30}{1756} & \cellcolor{TealBlue!30}{\textbf{13.8}} & \cellcolor{TealBlue!30}{0.00} & \cellcolor{TealBlue!30}{1756} & 613.1 & \cellcolor{TealBlue!30}{0.00} & - & - & -\\
\texttt{wine1-un} & \multicolumn{1}{r}{178} & \multicolumn{1}{r}{1276}  & \cellcolor{TealBlue!30}{43} & \cellcolor{TealBlue!30}{\textbf{16.1}} & \cellcolor{TealBlue!30}{1.00} & \cellcolor{TealBlue!30}{43} & 136.1 & \cellcolor{TealBlue!30}{1.00} & - & - & -\\
\texttt{wine2-un} & \multicolumn{1}{r}{178} & \multicolumn{1}{r}{1276}  & \cellcolor{TealBlue!30}{49} & \cellcolor{TealBlue!30}{\textbf{17.1}} & \cellcolor{TealBlue!30}{1.00} & \cellcolor{TealBlue!30}{49} & 162.8 & \cellcolor{TealBlue!30}{1.00} & - & - & -\\
\texttt{wine3-un} & \multicolumn{1}{r}{178} & \multicolumn{1}{r}{1276}  & \cellcolor{TealBlue!30}{33} & \cellcolor{TealBlue!30}{\textbf{15.9}} & \cellcolor{TealBlue!30}{1.00} & \cellcolor{TealBlue!30}{33} & 141.6 & \cellcolor{TealBlue!30}{1.00} & - & - & -\\
\texttt{yeast} & \multicolumn{1}{r}{1484} & \multicolumn{1}{r}{89}  & \cellcolor{TealBlue!30}{403} & \cellcolor{TealBlue!30}{\textbf{0.1}} & \cellcolor{TealBlue!30}{1.00} & \cellcolor{TealBlue!30}{403} & 0.3 & \cellcolor{TealBlue!30}{1.00} & \cellcolor{TealBlue!30}{403} & 6.1 & \cellcolor{TealBlue!30}{1.00}\\
\texttt{zoo-1} & \multicolumn{1}{r}{101} & \multicolumn{1}{r}{36}  & \cellcolor{TealBlue!30}{0} & \cellcolor{TealBlue!30}{\textbf{0.0}} & \cellcolor{TealBlue!30}{1.00} & \cellcolor{TealBlue!30}{0} & 0.0 & \cellcolor{TealBlue!30}{1.00} & \cellcolor{TealBlue!30}{0} & 0.0 & \cellcolor{TealBlue!30}{1.00}\\
\bottomrule
\end{tabular}

\end{normalsize}
\end{center}
\caption{\label{tab:d3} Comparison with state of the art on shallow trees (max depth=3)}
\end{table}

\begin{table}[htbp]
\begin{center}
\begin{normalsize}
\tabcolsep=5pt
\begin{tabular}{lccrrrrrrrrr}
\toprule
& && \multicolumn{3}{c}{\budalg} & \multicolumn{3}{c}{\murtree} & \multicolumn{3}{c}{\dleight}\\
\cmidrule(rr){4-6}\cmidrule(rr){7-9}\cmidrule(rr){10-12}
&\multirow{1}{*}{$\#ex.$} & \multirow{1}{*}{\#feat.} &  \multicolumn{1}{c}{error} & \multicolumn{1}{c}{time} & \multicolumn{1}{c}{opt.} & \multicolumn{1}{c}{error} & \multicolumn{1}{c}{time} & \multicolumn{1}{c}{opt.} & \multicolumn{1}{c}{error} & \multicolumn{1}{c}{time} & \multicolumn{1}{c}{opt.} \\
\midrule

\texttt{adult\_discretized} & \multicolumn{1}{r}{30299} & \multicolumn{1}{r}{59}  & \cellcolor{TealBlue!30}{4609} & \cellcolor{TealBlue!30}{\textbf{14.3}} & \cellcolor{TealBlue!30}{1.00} & \cellcolor{TealBlue!30}{4609} & 16.9 & \cellcolor{TealBlue!30}{1.00} & \cellcolor{TealBlue!30}{4609} & 271.4 & \cellcolor{TealBlue!30}{1.00}\\
\texttt{anneal} & \multicolumn{1}{r}{812} & \multicolumn{1}{r}{93}  & \cellcolor{TealBlue!30}{91} & \cellcolor{TealBlue!30}{\textbf{1.3}} & \cellcolor{TealBlue!30}{1.00} & \cellcolor{TealBlue!30}{91} & 7.3 & \cellcolor{TealBlue!30}{1.00} & \cellcolor{TealBlue!30}{91} & 101.5 & \cellcolor{TealBlue!30}{1.00}\\
\texttt{audiology} & \multicolumn{1}{r}{216} & \multicolumn{1}{r}{148}  & \cellcolor{TealBlue!30}{1} & \cellcolor{TealBlue!30}{\textbf{4.0}} & \cellcolor{TealBlue!30}{1.00} & \cellcolor{TealBlue!30}{1} & 14.0 & \cellcolor{TealBlue!30}{1.00} & \cellcolor{TealBlue!30}{1} & 128.1 & \cellcolor{TealBlue!30}{1.00}\\
\texttt{australian-credit} & \multicolumn{1}{r}{653} & \multicolumn{1}{r}{125}  & \cellcolor{TealBlue!30}{56} & \cellcolor{TealBlue!30}{\textbf{11.6}} & \cellcolor{TealBlue!30}{1.00} & \cellcolor{TealBlue!30}{56} & 27.0 & \cellcolor{TealBlue!30}{1.00} & \cellcolor{TealBlue!30}{56} & 470.4 & \cellcolor{TealBlue!30}{1.00}\\
\texttt{bank-un} & \multicolumn{1}{r}{45211} & \multicolumn{1}{r}{9531}  & \cellcolor{TealBlue!30}{\textbf{4326}} & 56.1 & \cellcolor{TealBlue!30}{0.00} & 4686 & \cellcolor{TealBlue!30}{\textbf{2.8}} & \cellcolor{TealBlue!30}{0.00} & 4808 & 3603.5 & \cellcolor{TealBlue!30}{0.00}\\
\texttt{breast-cancer-un} & \multicolumn{1}{r}{683} & \multicolumn{1}{r}{89}  & \cellcolor{TealBlue!30}{16} & 8.9 & \cellcolor{TealBlue!30}{1.00} & \cellcolor{TealBlue!30}{16} & \cellcolor{TealBlue!30}{\textbf{3.4}} & \cellcolor{TealBlue!30}{1.00} & \cellcolor{TealBlue!30}{16} & 27.6 & \cellcolor{TealBlue!30}{1.00}\\
\texttt{breast-wisconsin} & \multicolumn{1}{r}{683} & \multicolumn{1}{r}{120}  & \cellcolor{TealBlue!30}{7} & \cellcolor{TealBlue!30}{\textbf{3.0}} & \cellcolor{TealBlue!30}{1.00} & \cellcolor{TealBlue!30}{7} & 13.5 & \cellcolor{TealBlue!30}{1.00} & \cellcolor{TealBlue!30}{7} & 245.1 & \cellcolor{TealBlue!30}{1.00}\\
\texttt{car-un} & \multicolumn{1}{r}{1728} & \multicolumn{1}{r}{21}  & \cellcolor{TealBlue!30}{136} & 0.1 & \cellcolor{TealBlue!30}{1.00} & \cellcolor{TealBlue!30}{136} & \cellcolor{TealBlue!30}{\textbf{0.1}} & \cellcolor{TealBlue!30}{1.00} & \cellcolor{TealBlue!30}{136} & 0.4 & \cellcolor{TealBlue!30}{1.00}\\
\texttt{compas\_discretized} & \multicolumn{1}{r}{6167} & \multicolumn{1}{r}{25}  & \cellcolor{TealBlue!30}{1954} & \cellcolor{TealBlue!30}{\textbf{0.1}} & \cellcolor{TealBlue!30}{1.00} & \cellcolor{TealBlue!30}{1954} & 0.7 & \cellcolor{TealBlue!30}{1.00} & \cellcolor{TealBlue!30}{1954} & 3.5 & \cellcolor{TealBlue!30}{1.00}\\
\texttt{diabetes} & \multicolumn{1}{r}{768} & \multicolumn{1}{r}{112}  & \cellcolor{TealBlue!30}{137} & \cellcolor{TealBlue!30}{\textbf{6.2}} & \cellcolor{TealBlue!30}{1.00} & \cellcolor{TealBlue!30}{137} & 27.1 & \cellcolor{TealBlue!30}{1.00} & \cellcolor{TealBlue!30}{137} & 550.2 & \cellcolor{TealBlue!30}{1.00}\\
\texttt{forest-fires-un} & \multicolumn{1}{r}{517} & \multicolumn{1}{r}{989}  & \cellcolor{TealBlue!30}{173} & \cellcolor{TealBlue!30}{\textbf{14.1}} & \cellcolor{TealBlue!30}{0.00} & \cellcolor{TealBlue!30}{173} & 74.2 & \cellcolor{TealBlue!30}{0.00} & - & - & -\\
\texttt{german-credit} & \multicolumn{1}{r}{1000} & \multicolumn{1}{r}{112}  & \cellcolor{TealBlue!30}{204} & 29.8 & \cellcolor{TealBlue!30}{1.00} & \cellcolor{TealBlue!30}{204} & \cellcolor{TealBlue!30}{\textbf{25.8}} & \cellcolor{TealBlue!30}{1.00} & \cellcolor{TealBlue!30}{204} & 422.8 & \cellcolor{TealBlue!30}{1.00}\\
\texttt{heart-cleveland} & \multicolumn{1}{r}{296} & \multicolumn{1}{r}{95}  & \cellcolor{TealBlue!30}{25} & \cellcolor{TealBlue!30}{\textbf{3.0}} & \cellcolor{TealBlue!30}{1.00} & \cellcolor{TealBlue!30}{25} & 6.5 & \cellcolor{TealBlue!30}{1.00} & \cellcolor{TealBlue!30}{25} & 154.3 & \cellcolor{TealBlue!30}{1.00}\\
\texttt{hepatitis} & \multicolumn{1}{r}{137} & \multicolumn{1}{r}{68}  & \cellcolor{TealBlue!30}{3} & \cellcolor{TealBlue!30}{\textbf{0.4}} & \cellcolor{TealBlue!30}{1.00} & \cellcolor{TealBlue!30}{3} & 1.0 & \cellcolor{TealBlue!30}{1.00} & \cellcolor{TealBlue!30}{3} & 28.0 & \cellcolor{TealBlue!30}{1.00}\\
\texttt{hypothyroid} & \multicolumn{1}{r}{3247} & \multicolumn{1}{r}{88}  & \cellcolor{TealBlue!30}{53} & \cellcolor{TealBlue!30}{\textbf{2.8}} & \cellcolor{TealBlue!30}{1.00} & \cellcolor{TealBlue!30}{53} & 18.4 & \cellcolor{TealBlue!30}{1.00} & \cellcolor{TealBlue!30}{53} & 181.0 & \cellcolor{TealBlue!30}{1.00}\\
\texttt{ionosphere} & \multicolumn{1}{r}{351} & \multicolumn{1}{r}{445}  & \cellcolor{TealBlue!30}{7} & \cellcolor{TealBlue!30}{\textbf{850.9}} & \cellcolor{TealBlue!30}{1.00} & \cellcolor{TealBlue!30}{7} & 1931.7 & \cellcolor{TealBlue!30}{1.00} & - & - & -\\
\texttt{kr-vs-kp} & \multicolumn{1}{r}{3196} & \multicolumn{1}{r}{73}  & \cellcolor{TealBlue!30}{144} & \cellcolor{TealBlue!30}{\textbf{2.3}} & \cellcolor{TealBlue!30}{1.00} & \cellcolor{TealBlue!30}{144} & 8.4 & \cellcolor{TealBlue!30}{1.00} & \cellcolor{TealBlue!30}{144} & 88.3 & \cellcolor{TealBlue!30}{1.00}\\
\texttt{letter} & \multicolumn{1}{r}{20000} & \multicolumn{1}{r}{224}  & \cellcolor{TealBlue!30}{261} & \cellcolor{TealBlue!30}{\textbf{1107.5}} & \cellcolor{TealBlue!30}{1.00} & \cellcolor{TealBlue!30}{261} & 3372.4 & \cellcolor{TealBlue!30}{1.00} & 335 & 3600.0 & 0.00\\
\texttt{lymph} & \multicolumn{1}{r}{148} & \multicolumn{1}{r}{68}  & \cellcolor{TealBlue!30}{3} & \cellcolor{TealBlue!30}{\textbf{0.9}} & \cellcolor{TealBlue!30}{1.00} & \cellcolor{TealBlue!30}{3} & 1.0 & \cellcolor{TealBlue!30}{1.00} & \cellcolor{TealBlue!30}{3} & 14.5 & \cellcolor{TealBlue!30}{1.00}\\
\texttt{mnist\_0} & \multicolumn{1}{r}{60000} & \multicolumn{1}{r}{784}  & 2173 & \cellcolor{TealBlue!30}{\textbf{1807.7}} & \cellcolor{TealBlue!30}{0.00} & \cellcolor{TealBlue!30}{\textbf{2040}} & 3512.8 & \cellcolor{TealBlue!30}{0.00} & 3319 & 3600.3 & \cellcolor{TealBlue!30}{0.00}\\
\texttt{mnist\_1} & \multicolumn{1}{r}{60000} & \multicolumn{1}{r}{784}  & \cellcolor{TealBlue!30}{2332} & 3207.5 & \cellcolor{TealBlue!30}{0.00} & \cellcolor{TealBlue!30}{2332} & \cellcolor{TealBlue!30}{\textbf{702.2}} & \cellcolor{TealBlue!30}{0.00} & 4551 & 3600.2 & \cellcolor{TealBlue!30}{0.00}\\
\texttt{mnist\_2} & \multicolumn{1}{r}{60000} & \multicolumn{1}{r}{784}  & 3143 & 3403.3 & \cellcolor{TealBlue!30}{0.00} & \cellcolor{TealBlue!30}{\textbf{3116}} & \cellcolor{TealBlue!30}{\textbf{2582.4}} & \cellcolor{TealBlue!30}{0.00} & 4289 & 3600.2 & \cellcolor{TealBlue!30}{0.00}\\
\texttt{mnist\_3} & \multicolumn{1}{r}{60000} & \multicolumn{1}{r}{784}  & \cellcolor{TealBlue!30}{3485} & 2563.8 & \cellcolor{TealBlue!30}{0.00} & \cellcolor{TealBlue!30}{3485} & \cellcolor{TealBlue!30}{\textbf{1157.8}} & \cellcolor{TealBlue!30}{0.00} & 4900 & 3600.2 & \cellcolor{TealBlue!30}{0.00}\\
\texttt{mnist\_4} & \multicolumn{1}{r}{60000} & \multicolumn{1}{r}{784}  & 3670 & 2785.2 & \cellcolor{TealBlue!30}{0.00} & \cellcolor{TealBlue!30}{\textbf{3615}} & \cellcolor{TealBlue!30}{\textbf{1702.3}} & \cellcolor{TealBlue!30}{0.00} & 5580 & 3600.2 & \cellcolor{TealBlue!30}{0.00}\\
\texttt{mnist\_5} & \multicolumn{1}{r}{60000} & \multicolumn{1}{r}{784}  & 3312 & \cellcolor{TealBlue!30}{\textbf{222.1}} & \cellcolor{TealBlue!30}{0.00} & \cellcolor{TealBlue!30}{\textbf{3085}} & 1897.5 & \cellcolor{TealBlue!30}{0.00} & 4373 & 3600.2 & \cellcolor{TealBlue!30}{0.00}\\
\texttt{mnist\_6} & \multicolumn{1}{r}{60000} & \multicolumn{1}{r}{784}  & \cellcolor{TealBlue!30}{1940} & 2723.7 & \cellcolor{TealBlue!30}{0.00} & \cellcolor{TealBlue!30}{1940} & \cellcolor{TealBlue!30}{\textbf{1236.8}} & \cellcolor{TealBlue!30}{0.00} & 2755 & 3600.2 & \cellcolor{TealBlue!30}{0.00}\\
\texttt{mnist\_7} & \multicolumn{1}{r}{60000} & \multicolumn{1}{r}{784}  & 2793 & \cellcolor{TealBlue!30}{\textbf{47.5}} & \cellcolor{TealBlue!30}{0.00} & \cellcolor{TealBlue!30}{\textbf{2773}} & 3379.8 & \cellcolor{TealBlue!30}{0.00} & 4546 & 3600.2 & \cellcolor{TealBlue!30}{0.00}\\
\texttt{mnist\_8} & \multicolumn{1}{r}{60000} & \multicolumn{1}{r}{784}  & \cellcolor{TealBlue!30}{3165} & 1597.9 & \cellcolor{TealBlue!30}{0.00} & \cellcolor{TealBlue!30}{3165} & \cellcolor{TealBlue!30}{\textbf{501.8}} & \cellcolor{TealBlue!30}{0.00} & 4609 & 3600.2 & \cellcolor{TealBlue!30}{0.00}\\
\texttt{mnist\_9} & \multicolumn{1}{r}{60000} & \multicolumn{1}{r}{784}  & \cellcolor{TealBlue!30}{3977} & 1643.9 & \cellcolor{TealBlue!30}{0.00} & \cellcolor{TealBlue!30}{3977} & \cellcolor{TealBlue!30}{\textbf{1236.2}} & \cellcolor{TealBlue!30}{0.00} & 5252 & 3600.2 & \cellcolor{TealBlue!30}{0.00}\\
\texttt{mushroom} & \multicolumn{1}{r}{8124} & \multicolumn{1}{r}{119}  & \cellcolor{TealBlue!30}{0} & \cellcolor{TealBlue!30}{\textbf{0.0}} & \cellcolor{TealBlue!30}{1.00} & \cellcolor{TealBlue!30}{0} & 0.0 & \cellcolor{TealBlue!30}{1.00} & \cellcolor{TealBlue!30}{0} & 40.8 & \cellcolor{TealBlue!30}{1.00}\\
\texttt{pendigits} & \multicolumn{1}{r}{7494} & \multicolumn{1}{r}{216}  & \cellcolor{TealBlue!30}{13} & \cellcolor{TealBlue!30}{\textbf{250.6}} & \cellcolor{TealBlue!30}{1.00} & \cellcolor{TealBlue!30}{13} & 1021.6 & \cellcolor{TealBlue!30}{1.00} & - & - & -\\
\texttt{primary-tumor} & \multicolumn{1}{r}{336} & \multicolumn{1}{r}{31}  & \cellcolor{TealBlue!30}{34} & \cellcolor{TealBlue!30}{\textbf{0.0}} & \cellcolor{TealBlue!30}{1.00} & \cellcolor{TealBlue!30}{34} & 0.1 & \cellcolor{TealBlue!30}{1.00} & \cellcolor{TealBlue!30}{34} & 2.0 & \cellcolor{TealBlue!30}{1.00}\\
\texttt{segment} & \multicolumn{1}{r}{2310} & \multicolumn{1}{r}{235}  & \cellcolor{TealBlue!30}{0} & \cellcolor{TealBlue!30}{\textbf{0.0}} & \cellcolor{TealBlue!30}{1.00} & \cellcolor{TealBlue!30}{0} & 0.0 & \cellcolor{TealBlue!30}{1.00} & \cellcolor{TealBlue!30}{0} & 1.6 & \cellcolor{TealBlue!30}{1.00}\\
\texttt{soybean} & \multicolumn{1}{r}{630} & \multicolumn{1}{r}{50}  & \cellcolor{TealBlue!30}{14} & 0.9 & \cellcolor{TealBlue!30}{1.00} & \cellcolor{TealBlue!30}{14} & \cellcolor{TealBlue!30}{\textbf{0.6}} & \cellcolor{TealBlue!30}{1.00} & \cellcolor{TealBlue!30}{14} & 5.1 & \cellcolor{TealBlue!30}{1.00}\\
\texttt{splice-1} & \multicolumn{1}{r}{3190} & \multicolumn{1}{r}{287}  & \cellcolor{TealBlue!30}{141} & 3145.4 & \cellcolor{TealBlue!30}{1.00} & \cellcolor{TealBlue!30}{141} & \cellcolor{TealBlue!30}{\textbf{690.9}} & \cellcolor{TealBlue!30}{1.00} & - & - & -\\
\texttt{surgical-deepnet-un} & \multicolumn{1}{r}{14635} & \multicolumn{1}{r}{6047}  & \cellcolor{TealBlue!30}{\textbf{2475}} & 2886.9 & \cellcolor{TealBlue!30}{0.00} & 2506 & \cellcolor{TealBlue!30}{\textbf{530.1}} & \cellcolor{TealBlue!30}{0.00} & - & - & -\\
\texttt{taiwan\_binarised} & \multicolumn{1}{r}{30000} & \multicolumn{1}{r}{205}  & \cellcolor{TealBlue!30}{5273} & \cellcolor{TealBlue!30}{\textbf{7.1}} & \cellcolor{TealBlue!30}{0.00} & \cellcolor{TealBlue!30}{5273} & 34.0 & \cellcolor{TealBlue!30}{0.00} & 5307 & 3600.0 & \cellcolor{TealBlue!30}{0.00}\\
\texttt{tic-tac-toe} & \multicolumn{1}{r}{958} & \multicolumn{1}{r}{27}  & \cellcolor{TealBlue!30}{137} & 0.5 & \cellcolor{TealBlue!30}{1.00} & \cellcolor{TealBlue!30}{137} & \cellcolor{TealBlue!30}{\textbf{0.2}} & \cellcolor{TealBlue!30}{1.00} & \cellcolor{TealBlue!30}{137} & 1.8 & \cellcolor{TealBlue!30}{1.00}\\
\texttt{titanic-un} & \multicolumn{1}{r}{887} & \multicolumn{1}{r}{333}  & \cellcolor{TealBlue!30}{119} & \cellcolor{TealBlue!30}{\textbf{1669.6}} & \cellcolor{TealBlue!30}{1.00} & \cellcolor{TealBlue!30}{119} & 2575.5 & \cellcolor{TealBlue!30}{1.00} & - & - & -\\
\texttt{vehicle} & \multicolumn{1}{r}{846} & \multicolumn{1}{r}{252}  & \cellcolor{TealBlue!30}{12} & \cellcolor{TealBlue!30}{\textbf{81.1}} & \cellcolor{TealBlue!30}{1.00} & \cellcolor{TealBlue!30}{12} & 256.8 & \cellcolor{TealBlue!30}{1.00} & - & - & -\\
\texttt{vote} & \multicolumn{1}{r}{435} & \multicolumn{1}{r}{48}  & \cellcolor{TealBlue!30}{5} & 1.2 & \cellcolor{TealBlue!30}{1.00} & \cellcolor{TealBlue!30}{5} & \cellcolor{TealBlue!30}{\textbf{0.5}} & \cellcolor{TealBlue!30}{1.00} & \cellcolor{TealBlue!30}{5} & 7.6 & \cellcolor{TealBlue!30}{1.00}\\
\texttt{weather-aus-un} & \multicolumn{1}{r}{142193} & \multicolumn{1}{r}{4759}  & \cellcolor{TealBlue!30}{\textbf{1749}} & 2458.8 & \cellcolor{TealBlue!30}{0.00} & 1750 & \cellcolor{TealBlue!30}{\textbf{1484.8}} & \cellcolor{TealBlue!30}{0.00} & - & - & -\\
\texttt{wine1-un} & \multicolumn{1}{r}{178} & \multicolumn{1}{r}{1276}  & \cellcolor{TealBlue!30}{\textbf{37}} & \cellcolor{TealBlue!30}{\textbf{1654.9}} & \cellcolor{TealBlue!30}{0.00} & 38 & 2852.9 & \cellcolor{TealBlue!30}{0.00} & - & - & -\\
\texttt{wine2-un} & \multicolumn{1}{r}{178} & \multicolumn{1}{r}{1276}  & \cellcolor{TealBlue!30}{43} & \cellcolor{TealBlue!30}{\textbf{16.7}} & \cellcolor{TealBlue!30}{0.00} & \cellcolor{TealBlue!30}{43} & 274.6 & \cellcolor{TealBlue!30}{0.00} & - & - & -\\
\texttt{wine3-un} & \multicolumn{1}{r}{178} & \multicolumn{1}{r}{1276}  & \cellcolor{TealBlue!30}{28} & \cellcolor{TealBlue!30}{\textbf{32.7}} & \cellcolor{TealBlue!30}{0.00} & \cellcolor{TealBlue!30}{28} & 421.1 & \cellcolor{TealBlue!30}{0.00} & - & - & -\\
\texttt{yeast} & \multicolumn{1}{r}{1484} & \multicolumn{1}{r}{89}  & \cellcolor{TealBlue!30}{366} & \cellcolor{TealBlue!30}{\textbf{3.7}} & \cellcolor{TealBlue!30}{1.00} & \cellcolor{TealBlue!30}{366} & 20.3 & \cellcolor{TealBlue!30}{1.00} & \cellcolor{TealBlue!30}{366} & 257.1 & \cellcolor{TealBlue!30}{1.00}\\
\texttt{zoo-1} & \multicolumn{1}{r}{101} & \multicolumn{1}{r}{36}  & \cellcolor{TealBlue!30}{0} & \cellcolor{TealBlue!30}{\textbf{0.0}} & \cellcolor{TealBlue!30}{1.00} & \cellcolor{TealBlue!30}{0} & 0.0 & \cellcolor{TealBlue!30}{1.00} & \cellcolor{TealBlue!30}{0} & 0.0 & \cellcolor{TealBlue!30}{1.00}\\
\bottomrule
\end{tabular}

\end{normalsize}
\end{center}
\caption{\label{tab:d4} Comparison with state of the art on shallow trees (max depth=4)}
\end{table}

\begin{table}[htbp]
\begin{center}
\begin{normalsize}
\tabcolsep=5pt
\begin{tabular}{lccrrrrrrrrr}
\toprule
& && \multicolumn{3}{c}{\budalg} & \multicolumn{3}{c}{\murtree} & \multicolumn{3}{c}{\dleight}\\
\cmidrule(rr){4-6}\cmidrule(rr){7-9}\cmidrule(rr){10-12}
&\multirow{1}{*}{$\#ex.$} & \multirow{1}{*}{\#feat.} &  \multicolumn{1}{c}{error} & \multicolumn{1}{c}{time} & \multicolumn{1}{c}{opt.} & \multicolumn{1}{c}{error} & \multicolumn{1}{c}{time} & \multicolumn{1}{c}{opt.} & \multicolumn{1}{c}{error} & \multicolumn{1}{c}{time} & \multicolumn{1}{c}{opt.} \\
\midrule

\texttt{adult\_discretized} & \multicolumn{1}{r}{30299} & \multicolumn{1}{r}{59}  & \cellcolor{TealBlue!30}{4423} & 805.3 & \cellcolor{TealBlue!30}{1.00} & \cellcolor{TealBlue!30}{4423} & \cellcolor{TealBlue!30}{\textbf{490.1}} & \cellcolor{TealBlue!30}{1.00} & 4442 & 3600.0 & 0.00\\
\texttt{anneal} & \multicolumn{1}{r}{812} & \multicolumn{1}{r}{93}  & \cellcolor{TealBlue!30}{70} & \cellcolor{TealBlue!30}{\textbf{49.1}} & \cellcolor{TealBlue!30}{1.00} & \cellcolor{TealBlue!30}{70} & 253.7 & \cellcolor{TealBlue!30}{1.00} & - & - & -\\
\texttt{audiology} & \multicolumn{1}{r}{216} & \multicolumn{1}{r}{148}  & \cellcolor{TealBlue!30}{0} & \cellcolor{TealBlue!30}{\textbf{0.0}} & \cellcolor{TealBlue!30}{1.00} & \cellcolor{TealBlue!30}{0} & 0.0 & \cellcolor{TealBlue!30}{1.00} & \cellcolor{TealBlue!30}{0} & 0.0 & \cellcolor{TealBlue!30}{1.00}\\
\texttt{australian-credit} & \multicolumn{1}{r}{653} & \multicolumn{1}{r}{125}  & \cellcolor{TealBlue!30}{39} & \cellcolor{TealBlue!30}{\textbf{794.2}} & \cellcolor{TealBlue!30}{1.00} & \cellcolor{TealBlue!30}{39} & 1181.3 & \cellcolor{TealBlue!30}{1.00} & - & - & -\\
\texttt{bank-un} & \multicolumn{1}{r}{45211} & \multicolumn{1}{r}{9531}  & \cellcolor{TealBlue!30}{\textbf{4241}} & \cellcolor{TealBlue!30}{\textbf{906.9}} & \cellcolor{TealBlue!30}{0.00} & 4365 & 2094.4 & \cellcolor{TealBlue!30}{0.00} & 4809 & 3603.0 & \cellcolor{TealBlue!30}{0.00}\\
\texttt{breast-cancer-un} & \multicolumn{1}{r}{683} & \multicolumn{1}{r}{89}  & \cellcolor{TealBlue!30}{6} & 743.3 & \cellcolor{TealBlue!30}{1.00} & \cellcolor{TealBlue!30}{6} & \cellcolor{TealBlue!30}{\textbf{97.6}} & \cellcolor{TealBlue!30}{1.00} & \cellcolor{TealBlue!30}{6} & 438.0 & \cellcolor{TealBlue!30}{1.00}\\
\texttt{breast-wisconsin} & \multicolumn{1}{r}{683} & \multicolumn{1}{r}{120}  & \cellcolor{TealBlue!30}{0} & \cellcolor{TealBlue!30}{\textbf{20.6}} & \cellcolor{TealBlue!30}{1.00} & \cellcolor{TealBlue!30}{0} & 183.8 & \cellcolor{TealBlue!30}{1.00} & - & - & -\\
\texttt{car-un} & \multicolumn{1}{r}{1728} & \multicolumn{1}{r}{21}  & \cellcolor{TealBlue!30}{86} & 2.5 & \cellcolor{TealBlue!30}{1.00} & \cellcolor{TealBlue!30}{86} & \cellcolor{TealBlue!30}{\textbf{0.9}} & \cellcolor{TealBlue!30}{1.00} & \cellcolor{TealBlue!30}{86} & 2.7 & \cellcolor{TealBlue!30}{1.00}\\
\texttt{compas\_discretized} & \multicolumn{1}{r}{6167} & \multicolumn{1}{r}{25}  & \cellcolor{TealBlue!30}{1919} & \cellcolor{TealBlue!30}{\textbf{1.2}} & \cellcolor{TealBlue!30}{1.00} & \cellcolor{TealBlue!30}{1919} & 7.6 & \cellcolor{TealBlue!30}{1.00} & \cellcolor{TealBlue!30}{1919} & 26.4 & \cellcolor{TealBlue!30}{1.00}\\
\texttt{diabetes} & \multicolumn{1}{r}{768} & \multicolumn{1}{r}{112}  & \cellcolor{TealBlue!30}{106} & \cellcolor{TealBlue!30}{\textbf{349.6}} & \cellcolor{TealBlue!30}{1.00} & \cellcolor{TealBlue!30}{106} & 1231.1 & \cellcolor{TealBlue!30}{1.00} & - & - & -\\
\texttt{forest-fires-un} & \multicolumn{1}{r}{517} & \multicolumn{1}{r}{989}  & \cellcolor{TealBlue!30}{\textbf{156}} & \cellcolor{TealBlue!30}{\textbf{788.6}} & \cellcolor{TealBlue!30}{0.00} & 163 & 2622.0 & \cellcolor{TealBlue!30}{0.00} & - & - & -\\
\texttt{german-credit} & \multicolumn{1}{r}{1000} & \multicolumn{1}{r}{112}  & \cellcolor{TealBlue!30}{161} & 2822.9 & \cellcolor{TealBlue!30}{1.00} & \cellcolor{TealBlue!30}{161} & \cellcolor{TealBlue!30}{\textbf{1139.2}} & \cellcolor{TealBlue!30}{1.00} & - & - & -\\
\texttt{heart-cleveland} & \multicolumn{1}{r}{296} & \multicolumn{1}{r}{95}  & \cellcolor{TealBlue!30}{7} & \cellcolor{TealBlue!30}{\textbf{112.4}} & \cellcolor{TealBlue!30}{1.00} & \cellcolor{TealBlue!30}{7} & 165.3 & \cellcolor{TealBlue!30}{1.00} & - & - & -\\
\texttt{hepatitis} & \multicolumn{1}{r}{137} & \multicolumn{1}{r}{68}  & \cellcolor{TealBlue!30}{0} & \cellcolor{TealBlue!30}{\textbf{0.0}} & \cellcolor{TealBlue!30}{1.00} & \cellcolor{TealBlue!30}{0} & 0.3 & \cellcolor{TealBlue!30}{1.00} & \cellcolor{TealBlue!30}{0} & 71.4 & \cellcolor{TealBlue!30}{1.00}\\
\texttt{hypothyroid} & \multicolumn{1}{r}{3247} & \multicolumn{1}{r}{88}  & \cellcolor{TealBlue!30}{44} & \cellcolor{TealBlue!30}{\textbf{101.0}} & \cellcolor{TealBlue!30}{1.00} & \cellcolor{TealBlue!30}{44} & 505.2 & \cellcolor{TealBlue!30}{1.00} & - & - & -\\
\texttt{ionosphere} & \multicolumn{1}{r}{351} & \multicolumn{1}{r}{445}  & \cellcolor{TealBlue!30}{0} & \cellcolor{TealBlue!30}{\textbf{570.7}} & \cellcolor{TealBlue!30}{1.00} & \cellcolor{TealBlue!30}{0} & 3399.2 & \cellcolor{TealBlue!30}{1.00} & - & - & -\\
\texttt{kr-vs-kp} & \multicolumn{1}{r}{3196} & \multicolumn{1}{r}{73}  & \cellcolor{TealBlue!30}{81} & \cellcolor{TealBlue!30}{\textbf{67.6}} & \cellcolor{TealBlue!30}{1.00} & \cellcolor{TealBlue!30}{81} & 197.3 & \cellcolor{TealBlue!30}{1.00} & - & - & -\\
\texttt{letter} & \multicolumn{1}{r}{20000} & \multicolumn{1}{r}{224}  & \cellcolor{TealBlue!30}{\textbf{168}} & 3141.4 & \cellcolor{TealBlue!30}{0.00} & 190 & \cellcolor{TealBlue!30}{\textbf{795.4}} & \cellcolor{TealBlue!30}{0.00} & 352 & 3600.0 & \cellcolor{TealBlue!30}{0.00}\\
\texttt{lymph} & \multicolumn{1}{r}{148} & \multicolumn{1}{r}{68}  & \cellcolor{TealBlue!30}{0} & \cellcolor{TealBlue!30}{\textbf{0.0}} & \cellcolor{TealBlue!30}{1.00} & \cellcolor{TealBlue!30}{0} & 0.0 & \cellcolor{TealBlue!30}{1.00} & \cellcolor{TealBlue!30}{0} & 14.0 & \cellcolor{TealBlue!30}{1.00}\\
\texttt{mnist\_0} & \multicolumn{1}{r}{60000} & \multicolumn{1}{r}{784}  & \cellcolor{TealBlue!30}{\textbf{1714}} & \cellcolor{TealBlue!30}{\textbf{262.3}} & \cellcolor{TealBlue!30}{0.00} & 2066 & 2054.0 & \cellcolor{TealBlue!30}{0.00} & 3319 & 3600.2 & \cellcolor{TealBlue!30}{0.00}\\
\texttt{mnist\_1} & \multicolumn{1}{r}{60000} & \multicolumn{1}{r}{784}  & \cellcolor{TealBlue!30}{\textbf{1737}} & \cellcolor{TealBlue!30}{\textbf{823.2}} & \cellcolor{TealBlue!30}{0.00} & 1790 & 1110.4 & \cellcolor{TealBlue!30}{0.00} & 4029 & 3600.2 & \cellcolor{TealBlue!30}{0.00}\\
\texttt{mnist\_2} & \multicolumn{1}{r}{60000} & \multicolumn{1}{r}{784}  & 3365 & \cellcolor{TealBlue!30}{\textbf{2237.4}} & \cellcolor{TealBlue!30}{0.00} & \cellcolor{TealBlue!30}{\textbf{2963}} & 2544.7 & \cellcolor{TealBlue!30}{0.00} & 4026 & 3600.2 & \cellcolor{TealBlue!30}{0.00}\\
\texttt{mnist\_3} & \multicolumn{1}{r}{60000} & \multicolumn{1}{r}{784}  & 3316 & 1175.4 & \cellcolor{TealBlue!30}{0.00} & \cellcolor{TealBlue!30}{\textbf{3184}} & \cellcolor{TealBlue!30}{\textbf{369.2}} & \cellcolor{TealBlue!30}{0.00} & 4900 & 3600.3 & \cellcolor{TealBlue!30}{0.00}\\
\texttt{mnist\_4} & \multicolumn{1}{r}{60000} & \multicolumn{1}{r}{784}  & 3212 & 1789.8 & \cellcolor{TealBlue!30}{0.00} & \cellcolor{TealBlue!30}{\textbf{3164}} & \cellcolor{TealBlue!30}{\textbf{109.5}} & \cellcolor{TealBlue!30}{0.00} & 5580 & 3600.2 & \cellcolor{TealBlue!30}{0.00}\\
\texttt{mnist\_5} & \multicolumn{1}{r}{60000} & \multicolumn{1}{r}{784}  & 3399 & \cellcolor{TealBlue!30}{\textbf{478.8}} & \cellcolor{TealBlue!30}{0.00} & \cellcolor{TealBlue!30}{\textbf{3163}} & 2038.8 & \cellcolor{TealBlue!30}{0.00} & 4376 & 3600.2 & \cellcolor{TealBlue!30}{0.00}\\
\texttt{mnist\_6} & \multicolumn{1}{r}{60000} & \multicolumn{1}{r}{784}  & 1828 & 3581.8 & \cellcolor{TealBlue!30}{0.00} & \cellcolor{TealBlue!30}{\textbf{1653}} & \cellcolor{TealBlue!30}{\textbf{624.8}} & \cellcolor{TealBlue!30}{0.00} & 2753 & 3600.2 & \cellcolor{TealBlue!30}{0.00}\\
\texttt{mnist\_7} & \multicolumn{1}{r}{60000} & \multicolumn{1}{r}{784}  & 2699 & \cellcolor{TealBlue!30}{\textbf{254.8}} & \cellcolor{TealBlue!30}{0.00} & \cellcolor{TealBlue!30}{\textbf{2464}} & 2300.9 & \cellcolor{TealBlue!30}{0.00} & 4542 & 3600.2 & \cellcolor{TealBlue!30}{0.00}\\
\texttt{mnist\_8} & \multicolumn{1}{r}{60000} & \multicolumn{1}{r}{784}  & 2843 & 3174.8 & \cellcolor{TealBlue!30}{0.00} & \cellcolor{TealBlue!30}{\textbf{2818}} & \cellcolor{TealBlue!30}{\textbf{1150.3}} & \cellcolor{TealBlue!30}{0.00} & 4609 & 3600.2 & \cellcolor{TealBlue!30}{0.00}\\
\texttt{mnist\_9} & \multicolumn{1}{r}{60000} & \multicolumn{1}{r}{784}  & 3682 & \cellcolor{TealBlue!30}{\textbf{56.4}} & \cellcolor{TealBlue!30}{0.00} & \cellcolor{TealBlue!30}{\textbf{3521}} & 1159.3 & \cellcolor{TealBlue!30}{0.00} & 5252 & 3600.2 & \cellcolor{TealBlue!30}{0.00}\\
\texttt{mushroom} & \multicolumn{1}{r}{8124} & \multicolumn{1}{r}{119}  & \cellcolor{TealBlue!30}{0} & \cellcolor{TealBlue!30}{\textbf{0.0}} & \cellcolor{TealBlue!30}{1.00} & \cellcolor{TealBlue!30}{0} & 0.0 & \cellcolor{TealBlue!30}{1.00} & \cellcolor{TealBlue!30}{0} & 35.6 & \cellcolor{TealBlue!30}{1.00}\\
\texttt{pendigits} & \multicolumn{1}{r}{7494} & \multicolumn{1}{r}{216}  & \cellcolor{TealBlue!30}{0} & \cellcolor{TealBlue!30}{\textbf{313.6}} & \cellcolor{TealBlue!30}{1.00} & \cellcolor{TealBlue!30}{0} & 2860.1 & \cellcolor{TealBlue!30}{1.00} & - & - & -\\
\texttt{primary-tumor} & \multicolumn{1}{r}{336} & \multicolumn{1}{r}{31}  & \cellcolor{TealBlue!30}{26} & \cellcolor{TealBlue!30}{\textbf{0.4}} & \cellcolor{TealBlue!30}{1.00} & \cellcolor{TealBlue!30}{26} & 2.2 & \cellcolor{TealBlue!30}{1.00} & \cellcolor{TealBlue!30}{26} & 24.0 & \cellcolor{TealBlue!30}{1.00}\\
\texttt{segment} & \multicolumn{1}{r}{2310} & \multicolumn{1}{r}{235}  & \cellcolor{TealBlue!30}{0} & \cellcolor{TealBlue!30}{\textbf{0.0}} & \cellcolor{TealBlue!30}{1.00} & \cellcolor{TealBlue!30}{0} & 0.0 & \cellcolor{TealBlue!30}{1.00} & \cellcolor{TealBlue!30}{0} & 1.0 & \cellcolor{TealBlue!30}{1.00}\\
\texttt{soybean} & \multicolumn{1}{r}{630} & \multicolumn{1}{r}{50}  & \cellcolor{TealBlue!30}{8} & 21.1 & \cellcolor{TealBlue!30}{1.00} & \cellcolor{TealBlue!30}{8} & \cellcolor{TealBlue!30}{\textbf{9.4}} & \cellcolor{TealBlue!30}{1.00} & \cellcolor{TealBlue!30}{8} & 63.1 & \cellcolor{TealBlue!30}{1.00}\\
\texttt{splice-1} & \multicolumn{1}{r}{3190} & \multicolumn{1}{r}{287}  & 101 & \cellcolor{TealBlue!30}{\textbf{1728.2}} & 0.00 & \cellcolor{TealBlue!30}{\textbf{100}} & 3121.5 & \cellcolor{TealBlue!30}{\textbf{1.00}} & - & - & -\\
\texttt{surgical-deepnet-un} & \multicolumn{1}{r}{14635} & \multicolumn{1}{r}{6047}  & \cellcolor{TealBlue!30}{\textbf{2188}} & 2139.2 & \cellcolor{TealBlue!30}{0.00} & 2337 & \cellcolor{TealBlue!30}{\textbf{456.7}} & \cellcolor{TealBlue!30}{0.00} & - & - & -\\
\texttt{taiwan\_binarised} & \multicolumn{1}{r}{30000} & \multicolumn{1}{r}{205}  & \cellcolor{TealBlue!30}{\textbf{5200}} & 886.0 & \cellcolor{TealBlue!30}{0.00} & 5261 & \cellcolor{TealBlue!30}{\textbf{33.8}} & \cellcolor{TealBlue!30}{0.00} & 5412 & 3600.0 & \cellcolor{TealBlue!30}{0.00}\\
\texttt{tic-tac-toe} & \multicolumn{1}{r}{958} & \multicolumn{1}{r}{27}  & \cellcolor{TealBlue!30}{63} & 9.7 & \cellcolor{TealBlue!30}{1.00} & \cellcolor{TealBlue!30}{63} & \cellcolor{TealBlue!30}{\textbf{2.0}} & \cellcolor{TealBlue!30}{1.00} & \cellcolor{TealBlue!30}{63} & 14.0 & \cellcolor{TealBlue!30}{1.00}\\
\texttt{titanic-un} & \multicolumn{1}{r}{887} & \multicolumn{1}{r}{333}  & \cellcolor{TealBlue!30}{95} & 2149.8 & \cellcolor{TealBlue!30}{0.00} & \cellcolor{TealBlue!30}{95} & \cellcolor{TealBlue!30}{\textbf{2119.2}} & \cellcolor{TealBlue!30}{0.00} & - & - & -\\
\texttt{vehicle} & \multicolumn{1}{r}{846} & \multicolumn{1}{r}{252}  & \cellcolor{TealBlue!30}{\textbf{1}} & \cellcolor{TealBlue!30}{\textbf{751.5}} & \cellcolor{TealBlue!30}{0.00} & 2 & 2251.8 & \cellcolor{TealBlue!30}{0.00} & - & - & -\\
\texttt{vote} & \multicolumn{1}{r}{435} & \multicolumn{1}{r}{48}  & \cellcolor{TealBlue!30}{1} & 26.0 & \cellcolor{TealBlue!30}{1.00} & \cellcolor{TealBlue!30}{1} & \cellcolor{TealBlue!30}{\textbf{7.3}} & \cellcolor{TealBlue!30}{1.00} & \cellcolor{TealBlue!30}{1} & 45.0 & \cellcolor{TealBlue!30}{1.00}\\
\texttt{weather-aus-un} & \multicolumn{1}{r}{142193} & \multicolumn{1}{r}{4759}  & \cellcolor{TealBlue!30}{1735} & \cellcolor{TealBlue!30}{\textbf{390.9}} & \cellcolor{TealBlue!30}{0.00} & \cellcolor{TealBlue!30}{1735} & 1923.7 & \cellcolor{TealBlue!30}{0.00} & - & - & -\\
\texttt{wine1-un} & \multicolumn{1}{r}{178} & \multicolumn{1}{r}{1276}  & \cellcolor{TealBlue!30}{\textbf{33}} & \cellcolor{TealBlue!30}{\textbf{1133.2}} & \cellcolor{TealBlue!30}{0.00} & 34 & 1977.9 & \cellcolor{TealBlue!30}{0.00} & - & - & -\\
\texttt{wine2-un} & \multicolumn{1}{r}{178} & \multicolumn{1}{r}{1276}  & \cellcolor{TealBlue!30}{\textbf{39}} & 416.4 & \cellcolor{TealBlue!30}{0.00} & 40 & \cellcolor{TealBlue!30}{\textbf{133.3}} & \cellcolor{TealBlue!30}{0.00} & - & - & -\\
\texttt{wine3-un} & \multicolumn{1}{r}{178} & \multicolumn{1}{r}{1276}  & \cellcolor{TealBlue!30}{25} & \cellcolor{TealBlue!30}{\textbf{15.7}} & 0.00 & \cellcolor{TealBlue!30}{25} & 289.6 & \cellcolor{TealBlue!30}{\textbf{1.00}} & - & - & -\\
\texttt{yeast} & \multicolumn{1}{r}{1484} & \multicolumn{1}{r}{89}  & \cellcolor{TealBlue!30}{313} & \cellcolor{TealBlue!30}{\textbf{147.4}} & \cellcolor{TealBlue!30}{1.00} & \cellcolor{TealBlue!30}{313} & 749.4 & \cellcolor{TealBlue!30}{1.00} & - & - & -\\
\texttt{zoo-1} & \multicolumn{1}{r}{101} & \multicolumn{1}{r}{36}  & \cellcolor{TealBlue!30}{0} & \cellcolor{TealBlue!30}{\textbf{0.0}} & \cellcolor{TealBlue!30}{1.00} & \cellcolor{TealBlue!30}{0} & 0.0 & \cellcolor{TealBlue!30}{1.00} & \cellcolor{TealBlue!30}{0} & 0.0 & \cellcolor{TealBlue!30}{1.00}\\
\bottomrule
\end{tabular}

\end{normalsize}
\end{center}
\caption{\label{tab:d5} Comparison with state of the art on shallow trees (max depth=5)}
\end{table}







\begin{table}[htbp]
\begin{center}
\begin{normalsize}
\tabcolsep=5pt
\begin{tabular}{lccrrrrrrrrr}
\toprule
& && \multicolumn{3}{c}{\cart} & \multicolumn{3}{c}{\greedy} & \multicolumn{3}{c}{\budalg}\\
\cmidrule(rr){4-6}\cmidrule(rr){7-9}\cmidrule(rr){10-12}
&\multirow{1}{*}{$\#ex.$} & \multirow{1}{*}{\#feat.} &  \multicolumn{1}{c}{acc.} & \multicolumn{1}{c}{error} & \multicolumn{1}{c}{time} & \multicolumn{1}{c}{acc.} & \multicolumn{1}{c}{error} & \multicolumn{1}{c}{time} & \multicolumn{1}{c}{acc.} & \multicolumn{1}{c}{error} & \multicolumn{1}{c}{time} \\
\midrule

\texttt{anneal} & \multicolumn{1}{r}{812} & \multicolumn{1}{r}{47}  & 0.834 & 135 & 0.0 & 0.830 & 138 & \cellcolor{TealBlue!30}{\textbf{0.0}} & \cellcolor{TealBlue!30}{\textbf{0.879}} & \cellcolor{TealBlue!30}{\textbf{98}} & 3.0\\
\texttt{audiology} & \multicolumn{1}{r}{216} & \multicolumn{1}{r}{79}  & 0.986 & 3 & 0.0 & 0.986 & 3 & \cellcolor{TealBlue!30}{\textbf{0.0}} & \cellcolor{TealBlue!30}{\textbf{0.995}} & \cellcolor{TealBlue!30}{\textbf{1}} & 3.0\\
\texttt{australian-credit} & \multicolumn{1}{r}{653} & \multicolumn{1}{r}{73}  & 0.887 & 74 & 0.0 & 0.885 & 75 & \cellcolor{TealBlue!30}{\textbf{0.0}} & \cellcolor{TealBlue!30}{\textbf{0.908}} & \cellcolor{TealBlue!30}{\textbf{60}} & 3.0\\
\texttt{breast-cancer-un} & \multicolumn{1}{r}{683} & \multicolumn{1}{r}{89}  & 0.969 & 21 & 0.0 & 0.958 & 29 & \cellcolor{TealBlue!30}{\textbf{0.0}} & \cellcolor{TealBlue!30}{\textbf{0.977}} & \cellcolor{TealBlue!30}{\textbf{16}} & 3.0\\
\texttt{breast-wisconsin} & \multicolumn{1}{r}{683} & \multicolumn{1}{r}{120}  & 0.977 & 16 & 0.0 & 0.974 & 18 & \cellcolor{TealBlue!30}{\textbf{0.0}} & \cellcolor{TealBlue!30}{\textbf{0.988}} & \cellcolor{TealBlue!30}{\textbf{8}} & 3.0\\
\texttt{car-un} & \multicolumn{1}{r}{1728} & \multicolumn{1}{r}{21}  & 0.883 & 202 & 0.0 & 0.897 & 178 & \cellcolor{TealBlue!30}{\textbf{0.0}} & \cellcolor{TealBlue!30}{\textbf{0.921}} & \cellcolor{TealBlue!30}{\textbf{136}} & 0.2\\
\texttt{diabetes} & \multicolumn{1}{r}{768} & \multicolumn{1}{r}{112}  & 0.784 & 166 & 0.0 & 0.790 & 161 & \cellcolor{TealBlue!30}{\textbf{0.0}} & \cellcolor{TealBlue!30}{\textbf{0.822}} & \cellcolor{TealBlue!30}{\textbf{137}} & 3.1\\
\texttt{forest-fires-un} & \multicolumn{1}{r}{517} & \multicolumn{1}{r}{989}  & \cellcolor{TealBlue!30}{\textbf{0.640}} & \cellcolor{TealBlue!30}{\textbf{186}} & 0.0 & 0.615 & 199 & \cellcolor{TealBlue!30}{\textbf{0.0}} & 0.627 & 193 & 3.1\\
\texttt{german-credit} & \multicolumn{1}{r}{1000} & \multicolumn{1}{r}{110}  & 0.769 & 231 & 0.0 & 0.775 & 225 & \cellcolor{TealBlue!30}{\textbf{0.0}} & \cellcolor{TealBlue!30}{\textbf{0.796}} & \cellcolor{TealBlue!30}{\textbf{204}} & 3.0\\
\texttt{heart-cleveland} & \multicolumn{1}{r}{296} & \multicolumn{1}{r}{50}  & 0.872 & 38 & 0.0 & 0.868 & 39 & \cellcolor{TealBlue!30}{\textbf{0.0}} & \cellcolor{TealBlue!30}{\textbf{0.912}} & \cellcolor{TealBlue!30}{\textbf{26}} & 3.0\\
\texttt{hepatitis} & \multicolumn{1}{r}{137} & \multicolumn{1}{r}{68}  & 0.920 & 11 & 0.0 & 0.905 & 13 & \cellcolor{TealBlue!30}{\textbf{0.0}} & \cellcolor{TealBlue!30}{\textbf{0.978}} & \cellcolor{TealBlue!30}{\textbf{3}} & 2.3\\
\texttt{hypothyroid} & \multicolumn{1}{r}{3247} & \multicolumn{1}{r}{43}  & \cellcolor{TealBlue!30}{0.984} & \cellcolor{TealBlue!30}{53} & 0.0 & 0.983 & 54 & \cellcolor{TealBlue!30}{\textbf{0.0}} & \cellcolor{TealBlue!30}{0.984} & \cellcolor{TealBlue!30}{53} & 3.2\\
\texttt{ionosphere} & \multicolumn{1}{r}{351} & \multicolumn{1}{r}{444}  & 0.923 & 27 & 0.0 & 0.937 & 22 & \cellcolor{TealBlue!30}{\textbf{0.0}} & \cellcolor{TealBlue!30}{\textbf{0.963}} & \cellcolor{TealBlue!30}{\textbf{13}} & 3.2\\
\texttt{kr-vs-kp} & \multicolumn{1}{r}{3196} & \multicolumn{1}{r}{37}  & 0.941 & 189 & 0.0 & 0.941 & 189 & \cellcolor{TealBlue!30}{\textbf{0.0}} & \cellcolor{TealBlue!30}{\textbf{0.952}} & \cellcolor{TealBlue!30}{\textbf{154}} & 3.1\\
\texttt{letter} & \multicolumn{1}{r}{20000} & \multicolumn{1}{r}{224}  & \cellcolor{TealBlue!30}{\textbf{0.977}} & \cellcolor{TealBlue!30}{\textbf{462}} & 0.3 & 0.963 & 732 & \cellcolor{TealBlue!30}{\textbf{0.0}} & 0.964 & 724 & 3.4\\
\texttt{lymph} & \multicolumn{1}{r}{148} & \multicolumn{1}{r}{41}  & 0.932 & 10 & 0.0 & 0.939 & 9 & \cellcolor{TealBlue!30}{\textbf{0.0}} & \cellcolor{TealBlue!30}{\textbf{0.980}} & \cellcolor{TealBlue!30}{\textbf{3}} & 1.8\\
\texttt{mushroom} & \multicolumn{1}{r}{8124} & \multicolumn{1}{r}{91}  & 1.000 & 4 & 0.0 & 1.000 & 4 & \cellcolor{TealBlue!30}{\textbf{0.0}} & \cellcolor{TealBlue!30}{\textbf{1.000}} & \cellcolor{TealBlue!30}{\textbf{0}} & 0.0\\
\texttt{pendigits} & \multicolumn{1}{r}{7494} & \multicolumn{1}{r}{216}  & \cellcolor{TealBlue!30}{\textbf{0.997}} & \cellcolor{TealBlue!30}{\textbf{25}} & 0.1 & 0.996 & 27 & \cellcolor{TealBlue!30}{\textbf{0.0}} & 0.996 & 27 & 3.5\\
\texttt{primary-tumor} & \multicolumn{1}{r}{336} & \multicolumn{1}{r}{16}  & 0.869 & 44 & 0.0 & 0.866 & 45 & \cellcolor{TealBlue!30}{\textbf{0.0}} & \cellcolor{TealBlue!30}{\textbf{0.899}} & \cellcolor{TealBlue!30}{\textbf{34}} & 0.3\\
\texttt{segment} & \multicolumn{1}{r}{2310} & \multicolumn{1}{r}{234}  & 1.000 & 1 & 0.0 & 1.000 & 1 & \cellcolor{TealBlue!30}{\textbf{0.0}} & \cellcolor{TealBlue!30}{\textbf{1.000}} & \cellcolor{TealBlue!30}{\textbf{0}} & 0.0\\
\texttt{soybean} & \multicolumn{1}{r}{630} & \multicolumn{1}{r}{34}  & 0.949 & 32 & 0.0 & 0.887 & 71 & \cellcolor{TealBlue!30}{\textbf{0.0}} & \cellcolor{TealBlue!30}{\textbf{0.978}} & \cellcolor{TealBlue!30}{\textbf{14}} & 1.2\\
\texttt{splice-1} & \multicolumn{1}{r}{3190} & \multicolumn{1}{r}{227}  & \cellcolor{TealBlue!30}{0.956} & \cellcolor{TealBlue!30}{141} & 0.0 & \cellcolor{TealBlue!30}{0.956} & \cellcolor{TealBlue!30}{141} & \cellcolor{TealBlue!30}{\textbf{0.0}} & \cellcolor{TealBlue!30}{0.956} & \cellcolor{TealBlue!30}{141} & 3.2\\
\texttt{taiwan\_binarised} & \multicolumn{1}{r}{30000} & \multicolumn{1}{r}{198}  & 0.823 & 5306 & 0.4 & 0.823 & 5305 & \cellcolor{TealBlue!30}{\textbf{0.0}} & \cellcolor{TealBlue!30}{\textbf{0.824}} & \cellcolor{TealBlue!30}{\textbf{5284}} & 3.4\\
\texttt{tic-tac-toe} & \multicolumn{1}{r}{958} & \multicolumn{1}{r}{18}  & 0.843 & 150 & 0.0 & 0.843 & 150 & \cellcolor{TealBlue!30}{\textbf{0.0}} & \cellcolor{TealBlue!30}{\textbf{0.857}} & \cellcolor{TealBlue!30}{\textbf{137}} & 0.4\\
\texttt{vehicle} & \multicolumn{1}{r}{846} & \multicolumn{1}{r}{252}  & 0.967 & 28 & 0.0 & 0.952 & 41 & \cellcolor{TealBlue!30}{\textbf{0.0}} & \cellcolor{TealBlue!30}{\textbf{0.978}} & \cellcolor{TealBlue!30}{\textbf{19}} & 3.1\\
\texttt{vote} & \multicolumn{1}{r}{435} & \multicolumn{1}{r}{32}  & 0.982 & 8 & 0.0 & 0.982 & 8 & \cellcolor{TealBlue!30}{\textbf{0.0}} & \cellcolor{TealBlue!30}{\textbf{0.989}} & \cellcolor{TealBlue!30}{\textbf{5}} & 1.1\\
\texttt{wine1-un} & \multicolumn{1}{r}{178} & \multicolumn{1}{r}{1276}  & 0.764 & 42 & 0.0 & 0.747 & 45 & \cellcolor{TealBlue!30}{\textbf{0.0}} & \cellcolor{TealBlue!30}{\textbf{0.781}} & \cellcolor{TealBlue!30}{\textbf{39}} & 3.1\\
\texttt{wine2-un} & \multicolumn{1}{r}{178} & \multicolumn{1}{r}{1276}  & 0.736 & 47 & 0.0 & 0.730 & 48 & \cellcolor{TealBlue!30}{\textbf{0.0}} & \cellcolor{TealBlue!30}{\textbf{0.742}} & \cellcolor{TealBlue!30}{\textbf{46}} & 3.2\\
\texttt{wine3-un} & \multicolumn{1}{r}{178} & \multicolumn{1}{r}{1276}  & 0.820 & 32 & 0.0 & 0.820 & 32 & \cellcolor{TealBlue!30}{\textbf{0.0}} & \cellcolor{TealBlue!30}{\textbf{0.831}} & \cellcolor{TealBlue!30}{\textbf{30}} & 3.1\\
\texttt{yeast} & \multicolumn{1}{r}{1484} & \multicolumn{1}{r}{89}  & 0.735 & 394 & 0.0 & 0.737 & 391 & \cellcolor{TealBlue!30}{\textbf{0.0}} & \cellcolor{TealBlue!30}{\textbf{0.751}} & \cellcolor{TealBlue!30}{\textbf{369}} & 3.1\\
\texttt{zoo-1} & \multicolumn{1}{r}{101} & \multicolumn{1}{r}{20}  & \cellcolor{TealBlue!30}{1.000} & \cellcolor{TealBlue!30}{0} & 0.0 & \cellcolor{TealBlue!30}{1.000} & \cellcolor{TealBlue!30}{0} & \cellcolor{TealBlue!30}{\textbf{0.0}} & \cellcolor{TealBlue!30}{1.000} & \cellcolor{TealBlue!30}{0} & 0.0\\
\bottomrule
\end{tabular}

\end{normalsize}
\end{center}
\caption{\label{tab:f4} Comparison with state of the art heuristics (max depth=4)}
\end{table}

\begin{table}[htbp]
\begin{center}
\begin{normalsize}
\tabcolsep=5pt
\begin{tabular}{lccrrrrrr}
\toprule
& && \multicolumn{2}{c}{\cart} & \multicolumn{2}{c}{first sol.} & \multicolumn{2}{c}{$\leq 3$s}\\
\cmidrule(rr){4-5}\cmidrule(rr){6-7}\cmidrule(rr){8-9}
&\multirow{1}{*}{$\#ex.$} & \multirow{1}{*}{\#feat.} &  \multicolumn{1}{c}{error} & \multicolumn{1}{c}{time} & \multicolumn{1}{c}{error} & \multicolumn{1}{c}{time} & \multicolumn{1}{c}{error} & \multicolumn{1}{c}{time} \\
\midrule

\texttt{anneal} & \multicolumn{1}{r}{812} & \multicolumn{1}{r}{47}  & 96 & 0.00 & 94 & \cellcolor{TealBlue!30}{\textbf{0.00}} & \cellcolor{TealBlue!30}{\textbf{91}} & 2.87\\
\texttt{audiology} & \multicolumn{1}{r}{216} & \multicolumn{1}{r}{79}  & \cellcolor{TealBlue!30}{0} & 0.00 & \cellcolor{TealBlue!30}{0} & \cellcolor{TealBlue!30}{\textbf{0.00}} & \cellcolor{TealBlue!30}{0} & 0.00\\
\texttt{australian-credit} & \multicolumn{1}{r}{653} & \multicolumn{1}{r}{73}  & 43 & 0.01 & 43 & \cellcolor{TealBlue!30}{\textbf{0.00}} & \cellcolor{TealBlue!30}{\textbf{30}} & 0.55\\
\texttt{breast-cancer-un} & \multicolumn{1}{r}{683} & \multicolumn{1}{r}{89}  & 8 & 0.00 & 8 & \cellcolor{TealBlue!30}{\textbf{0.00}} & \cellcolor{TealBlue!30}{\textbf{2}} & 0.39\\
\texttt{breast-wisconsin} & \multicolumn{1}{r}{683} & \multicolumn{1}{r}{120}  & 4 & 0.00 & 4 & \cellcolor{TealBlue!30}{\textbf{0.00}} & \cellcolor{TealBlue!30}{\textbf{0}} & 0.33\\
\texttt{car-un} & \multicolumn{1}{r}{1728} & \multicolumn{1}{r}{21}  & 50 & 0.00 & 50 & \cellcolor{TealBlue!30}{\textbf{0.00}} & \cellcolor{TealBlue!30}{\textbf{34}} & 0.27\\
\texttt{diabetes} & \multicolumn{1}{r}{768} & \multicolumn{1}{r}{112}  & 100 & 0.01 & 99 & \cellcolor{TealBlue!30}{\textbf{0.00}} & \cellcolor{TealBlue!30}{\textbf{77}} & 2.05\\
\texttt{forest-fires-un} & \multicolumn{1}{r}{517} & \multicolumn{1}{r}{989}  & 162 & 0.02 & 161 & \cellcolor{TealBlue!30}{\textbf{0.00}} & \cellcolor{TealBlue!30}{\textbf{159}} & 0.22\\
\texttt{german-credit} & \multicolumn{1}{r}{1000} & \multicolumn{1}{r}{110}  & 149 & 0.03 & 141 & \cellcolor{TealBlue!30}{\textbf{0.00}} & \cellcolor{TealBlue!30}{\textbf{114}} & 0.63\\
\texttt{heart-cleveland} & \multicolumn{1}{r}{296} & \multicolumn{1}{r}{50}  & 6 & 0.00 & 7 & \cellcolor{TealBlue!30}{\textbf{0.00}} & \cellcolor{TealBlue!30}{\textbf{0}} & 0.03\\
\texttt{hepatitis} & \multicolumn{1}{r}{137} & \multicolumn{1}{r}{68}  & \cellcolor{TealBlue!30}{0} & 0.00 & \cellcolor{TealBlue!30}{0} & \cellcolor{TealBlue!30}{\textbf{0.00}} & \cellcolor{TealBlue!30}{0} & 0.00\\
\texttt{hypothyroid} & \multicolumn{1}{r}{3247} & \multicolumn{1}{r}{43}  & \cellcolor{TealBlue!30}{\textbf{42}} & 0.01 & 43 & \cellcolor{TealBlue!30}{\textbf{0.00}} & 43 & 1.33\\
\texttt{ionosphere} & \multicolumn{1}{r}{351} & \multicolumn{1}{r}{444}  & 7 & 0.01 & 7 & \cellcolor{TealBlue!30}{\textbf{0.00}} & \cellcolor{TealBlue!30}{\textbf{0}} & 0.52\\
\texttt{kr-vs-kp} & \multicolumn{1}{r}{3196} & \multicolumn{1}{r}{37}  & 103 & 0.01 & 102 & \cellcolor{TealBlue!30}{\textbf{0.00}} & \cellcolor{TealBlue!30}{\textbf{101}} & 2.17\\
\texttt{letter} & \multicolumn{1}{r}{20000} & \multicolumn{1}{r}{224}  & 153 & 0.62 & 143 & \cellcolor{TealBlue!30}{\textbf{0.04}} & \cellcolor{TealBlue!30}{\textbf{142}} & 0.50\\
\texttt{lymph} & \multicolumn{1}{r}{148} & \multicolumn{1}{r}{41}  & \cellcolor{TealBlue!30}{0} & 0.00 & \cellcolor{TealBlue!30}{0} & \cellcolor{TealBlue!30}{\textbf{0.00}} & \cellcolor{TealBlue!30}{0} & 0.00\\
\texttt{mushroom} & \multicolumn{1}{r}{8124} & \multicolumn{1}{r}{91}  & \cellcolor{TealBlue!30}{0} & 0.03 & \cellcolor{TealBlue!30}{0} & \cellcolor{TealBlue!30}{\textbf{0.00}} & \cellcolor{TealBlue!30}{0} & 0.02\\
\texttt{pendigits} & \multicolumn{1}{r}{7494} & \multicolumn{1}{r}{216}  & 1 & 0.09 & 1 & \cellcolor{TealBlue!30}{\textbf{0.01}} & \cellcolor{TealBlue!30}{\textbf{0}} & 0.11\\
\texttt{primary-tumor} & \multicolumn{1}{r}{336} & \multicolumn{1}{r}{16}  & 26 & 0.00 & 26 & \cellcolor{TealBlue!30}{\textbf{0.00}} & \cellcolor{TealBlue!30}{\textbf{17}} & 1.86\\
\texttt{segment} & \multicolumn{1}{r}{2310} & \multicolumn{1}{r}{234}  & \cellcolor{TealBlue!30}{0} & 0.01 & \cellcolor{TealBlue!30}{0} & \cellcolor{TealBlue!30}{\textbf{0.00}} & \cellcolor{TealBlue!30}{0} & 0.03\\
\texttt{soybean} & \multicolumn{1}{r}{630} & \multicolumn{1}{r}{34}  & 11 & 0.00 & 11 & \cellcolor{TealBlue!30}{\textbf{0.00}} & \cellcolor{TealBlue!30}{\textbf{4}} & 0.70\\
\texttt{splice-1} & \multicolumn{1}{r}{3190} & \multicolumn{1}{r}{227}  & 58 & 0.05 & 58 & \cellcolor{TealBlue!30}{\textbf{0.00}} & \cellcolor{TealBlue!30}{\textbf{55}} & 2.24\\
\texttt{taiwan\_binarised} & \multicolumn{1}{r}{30000} & \multicolumn{1}{r}{198}  & 5161 & 0.66 & 5121 & \cellcolor{TealBlue!30}{\textbf{0.04}} & \cellcolor{TealBlue!30}{\textbf{5087}} & 2.06\\
\texttt{tic-tac-toe} & \multicolumn{1}{r}{958} & \multicolumn{1}{r}{18}  & 22 & 0.00 & 21 & \cellcolor{TealBlue!30}{\textbf{0.00}} & \cellcolor{TealBlue!30}{\textbf{1}} & 1.14\\
\texttt{vehicle} & \multicolumn{1}{r}{846} & \multicolumn{1}{r}{252}  & 4 & 0.01 & 4 & \cellcolor{TealBlue!30}{\textbf{0.00}} & \cellcolor{TealBlue!30}{\textbf{0}} & 0.59\\
\texttt{vote} & \multicolumn{1}{r}{435} & \multicolumn{1}{r}{32}  & 2 & 0.00 & 2 & \cellcolor{TealBlue!30}{\textbf{0.00}} & \cellcolor{TealBlue!30}{\textbf{0}} & 0.00\\
\texttt{wine1-un} & \multicolumn{1}{r}{178} & \multicolumn{1}{r}{1276}  & 33 & 0.01 & 33 & \cellcolor{TealBlue!30}{\textbf{0.00}} & \cellcolor{TealBlue!30}{\textbf{30}} & 0.17\\
\texttt{wine2-un} & \multicolumn{1}{r}{178} & \multicolumn{1}{r}{1276}  & 38 & 0.01 & 38 & \cellcolor{TealBlue!30}{\textbf{0.00}} & \cellcolor{TealBlue!30}{\textbf{35}} & 0.49\\
\texttt{wine3-un} & \multicolumn{1}{r}{178} & \multicolumn{1}{r}{1276}  & 26 & 0.01 & 26 & \cellcolor{TealBlue!30}{\textbf{0.00}} & \cellcolor{TealBlue!30}{\textbf{24}} & 2.12\\
\texttt{yeast} & \multicolumn{1}{r}{1484} & \multicolumn{1}{r}{89}  & 306 & 0.01 & 305 & \cellcolor{TealBlue!30}{\textbf{0.00}} & \cellcolor{TealBlue!30}{\textbf{278}} & 1.30\\
\texttt{zoo-1} & \multicolumn{1}{r}{101} & \multicolumn{1}{r}{20}  & \cellcolor{TealBlue!30}{0} & 0.00 & \cellcolor{TealBlue!30}{0} & \cellcolor{TealBlue!30}{\textbf{0.00}} & \cellcolor{TealBlue!30}{0} & 0.00\\
\bottomrule
\end{tabular}

\end{normalsize}
\end{center}
\caption{\label{tab:f7} Comparison with state of the art heuristics (max depth=7)}
\end{table}

\begin{table}[htbp]
\begin{center}
\begin{normalsize}
\tabcolsep=5pt
\begin{tabular}{lccrrrrrr}
\toprule
& && \multicolumn{2}{c}{\cart} & \multicolumn{2}{c}{first sol.} & \multicolumn{2}{c}{$\leq 3$s}\\
\cmidrule(rr){4-5}\cmidrule(rr){6-7}\cmidrule(rr){8-9}
&\multirow{1}{*}{$\#ex.$} & \multirow{1}{*}{\#feat.} &  \multicolumn{1}{c}{error} & \multicolumn{1}{c}{time} & \multicolumn{1}{c}{error} & \multicolumn{1}{c}{time} & \multicolumn{1}{c}{error} & \multicolumn{1}{c}{time} \\
\midrule

\texttt{anneal} & \multicolumn{1}{r}{812} & \multicolumn{1}{r}{47}  & 59 & 0.00 & \cellcolor{TealBlue!30}{58} & \cellcolor{TealBlue!30}{\textbf{0.00}} & \cellcolor{TealBlue!30}{58} & 0.47\\
\texttt{audiology} & \multicolumn{1}{r}{216} & \multicolumn{1}{r}{79}  & \cellcolor{TealBlue!30}{0} & 0.00 & \cellcolor{TealBlue!30}{0} & \cellcolor{TealBlue!30}{\textbf{0.00}} & \cellcolor{TealBlue!30}{0} & 0.00\\
\texttt{australian-credit} & \multicolumn{1}{r}{653} & \multicolumn{1}{r}{73}  & 13 & 0.01 & 13 & \cellcolor{TealBlue!30}{\textbf{0.00}} & \cellcolor{TealBlue!30}{\textbf{0}} & 0.30\\
\texttt{breast-cancer-un} & \multicolumn{1}{r}{683} & \multicolumn{1}{r}{89}  & \cellcolor{TealBlue!30}{0} & 0.00 & \cellcolor{TealBlue!30}{0} & \cellcolor{TealBlue!30}{\textbf{0.00}} & \cellcolor{TealBlue!30}{0} & 0.00\\
\texttt{breast-wisconsin} & \multicolumn{1}{r}{683} & \multicolumn{1}{r}{120}  & \cellcolor{TealBlue!30}{0} & 0.00 & \cellcolor{TealBlue!30}{0} & \cellcolor{TealBlue!30}{\textbf{0.00}} & \cellcolor{TealBlue!30}{0} & 0.00\\
\texttt{car-un} & \multicolumn{1}{r}{1728} & \multicolumn{1}{r}{21}  & 11 & 0.00 & 11 & \cellcolor{TealBlue!30}{\textbf{0.00}} & \cellcolor{TealBlue!30}{\textbf{0}} & 0.30\\
\texttt{diabetes} & \multicolumn{1}{r}{768} & \multicolumn{1}{r}{112}  & 35 & 0.01 & 39 & \cellcolor{TealBlue!30}{\textbf{0.00}} & \cellcolor{TealBlue!30}{\textbf{4}} & 1.28\\
\texttt{forest-fires-un} & \multicolumn{1}{r}{517} & \multicolumn{1}{r}{989}  & 145 & 0.02 & 145 & \cellcolor{TealBlue!30}{\textbf{0.00}} & \cellcolor{TealBlue!30}{\textbf{131}} & 2.59\\
\texttt{german-credit} & \multicolumn{1}{r}{1000} & \multicolumn{1}{r}{110}  & 66 & 0.01 & 64 & \cellcolor{TealBlue!30}{\textbf{0.00}} & \cellcolor{TealBlue!30}{\textbf{21}} & 0.73\\
\texttt{heart-cleveland} & \multicolumn{1}{r}{296} & \multicolumn{1}{r}{50}  & \cellcolor{TealBlue!30}{0} & 0.00 & \cellcolor{TealBlue!30}{0} & \cellcolor{TealBlue!30}{\textbf{0.00}} & \cellcolor{TealBlue!30}{0} & 0.00\\
\texttt{hepatitis} & \multicolumn{1}{r}{137} & \multicolumn{1}{r}{68}  & \cellcolor{TealBlue!30}{0} & 0.00 & \cellcolor{TealBlue!30}{0} & \cellcolor{TealBlue!30}{\textbf{0.00}} & \cellcolor{TealBlue!30}{0} & 0.00\\
\texttt{hypothyroid} & \multicolumn{1}{r}{3247} & \multicolumn{1}{r}{43}  & \cellcolor{TealBlue!30}{\textbf{31}} & 0.01 & 32 & \cellcolor{TealBlue!30}{0.00} & 32 & \cellcolor{TealBlue!30}{0.00}\\
\texttt{ionosphere} & \multicolumn{1}{r}{351} & \multicolumn{1}{r}{444}  & \cellcolor{TealBlue!30}{0} & 0.01 & \cellcolor{TealBlue!30}{0} & \cellcolor{TealBlue!30}{\textbf{0.00}} & \cellcolor{TealBlue!30}{0} & 0.03\\
\texttt{kr-vs-kp} & \multicolumn{1}{r}{3196} & \multicolumn{1}{r}{37}  & 12 & 0.01 & 12 & \cellcolor{TealBlue!30}{\textbf{0.00}} & \cellcolor{TealBlue!30}{\textbf{10}} & 1.35\\
\texttt{letter} & \multicolumn{1}{r}{20000} & \multicolumn{1}{r}{224}  & 20 & 0.40 & 20 & \cellcolor{TealBlue!30}{\textbf{0.03}} & \cellcolor{TealBlue!30}{\textbf{13}} & 2.05\\
\texttt{lymph} & \multicolumn{1}{r}{148} & \multicolumn{1}{r}{41}  & \cellcolor{TealBlue!30}{0} & 0.00 & \cellcolor{TealBlue!30}{0} & \cellcolor{TealBlue!30}{\textbf{0.00}} & \cellcolor{TealBlue!30}{0} & 0.00\\
\texttt{mushroom} & \multicolumn{1}{r}{8124} & \multicolumn{1}{r}{91}  & \cellcolor{TealBlue!30}{0} & 0.03 & \cellcolor{TealBlue!30}{0} & \cellcolor{TealBlue!30}{\textbf{0.01}} & \cellcolor{TealBlue!30}{0} & 0.03\\
\texttt{pendigits} & \multicolumn{1}{r}{7494} & \multicolumn{1}{r}{216}  & \cellcolor{TealBlue!30}{0} & 0.08 & \cellcolor{TealBlue!30}{0} & \cellcolor{TealBlue!30}{\textbf{0.01}} & \cellcolor{TealBlue!30}{0} & 0.11\\
\texttt{primary-tumor} & \multicolumn{1}{r}{336} & \multicolumn{1}{r}{16}  & 20 & 0.00 & 20 & \cellcolor{TealBlue!30}{\textbf{0.00}} & \cellcolor{TealBlue!30}{\textbf{19}} & 2.88\\
\texttt{segment} & \multicolumn{1}{r}{2310} & \multicolumn{1}{r}{234}  & \cellcolor{TealBlue!30}{0} & 0.01 & \cellcolor{TealBlue!30}{0} & \cellcolor{TealBlue!30}{\textbf{0.00}} & \cellcolor{TealBlue!30}{0} & 0.03\\
\texttt{soybean} & \multicolumn{1}{r}{630} & \multicolumn{1}{r}{34}  & \cellcolor{TealBlue!30}{2} & 0.00 & \cellcolor{TealBlue!30}{2} & \cellcolor{TealBlue!30}{\textbf{0.00}} & \cellcolor{TealBlue!30}{2} & 0.01\\
\texttt{splice-1} & \multicolumn{1}{r}{3190} & \multicolumn{1}{r}{227}  & 12 & 0.05 & 12 & \cellcolor{TealBlue!30}{\textbf{0.01}} & \cellcolor{TealBlue!30}{\textbf{9}} & 0.81\\
\texttt{taiwan\_binarised} & \multicolumn{1}{r}{30000} & \multicolumn{1}{r}{198}  & 4707 & 0.80 & 4668 & \cellcolor{TealBlue!30}{\textbf{0.05}} & \cellcolor{TealBlue!30}{\textbf{4607}} & 1.77\\
\texttt{tic-tac-toe} & \multicolumn{1}{r}{958} & \multicolumn{1}{r}{18}  & 6 & 0.00 & 6 & \cellcolor{TealBlue!30}{\textbf{0.00}} & \cellcolor{TealBlue!30}{\textbf{0}} & 0.00\\
\texttt{vehicle} & \multicolumn{1}{r}{846} & \multicolumn{1}{r}{252}  & \cellcolor{TealBlue!30}{0} & 0.01 & \cellcolor{TealBlue!30}{0} & \cellcolor{TealBlue!30}{\textbf{0.00}} & \cellcolor{TealBlue!30}{0} & 0.02\\
\texttt{vote} & \multicolumn{1}{r}{435} & \multicolumn{1}{r}{32}  & \cellcolor{TealBlue!30}{0} & 0.00 & \cellcolor{TealBlue!30}{0} & \cellcolor{TealBlue!30}{\textbf{0.00}} & \cellcolor{TealBlue!30}{0} & 0.00\\
\texttt{wine1-un} & \multicolumn{1}{r}{178} & \multicolumn{1}{r}{1276}  & 25 & 0.01 & 25 & \cellcolor{TealBlue!30}{\textbf{0.00}} & \cellcolor{TealBlue!30}{\textbf{23}} & 1.70\\
\texttt{wine2-un} & \multicolumn{1}{r}{178} & \multicolumn{1}{r}{1276}  & 29 & 0.01 & 29 & \cellcolor{TealBlue!30}{\textbf{0.00}} & \cellcolor{TealBlue!30}{\textbf{27}} & 0.08\\
\texttt{wine3-un} & \multicolumn{1}{r}{178} & \multicolumn{1}{r}{1276}  & \cellcolor{TealBlue!30}{\textbf{15}} & 0.01 & 20 & \cellcolor{TealBlue!30}{\textbf{0.00}} & 19 & 1.07\\
\texttt{yeast} & \multicolumn{1}{r}{1484} & \multicolumn{1}{r}{89}  & 185 & 0.01 & 180 & \cellcolor{TealBlue!30}{\textbf{0.00}} & \cellcolor{TealBlue!30}{\textbf{115}} & 2.39\\
\texttt{zoo-1} & \multicolumn{1}{r}{101} & \multicolumn{1}{r}{20}  & \cellcolor{TealBlue!30}{0} & 0.01 & \cellcolor{TealBlue!30}{0} & \cellcolor{TealBlue!30}{\textbf{0.00}} & \cellcolor{TealBlue!30}{0} & 0.00\\
\bottomrule
\end{tabular}

\end{normalsize}
\end{center}
\caption{\label{tab:f10} Comparison with state of the art heuristics (max depth=10)}
\end{table}


\subsection{Size stats}


\begin{table}[htbp]
\begin{center}
\begin{normalsize}
\tabcolsep=3pt
\begin{tabular}{lccrrrrrrrr}
\toprule
& && \multicolumn{8}{c}{\budalg}\\
\cmidrule(rr){4-11}
&\multirow{1}{*}{$\#ex.$} & \multirow{1}{*}{\#feat.} &  \multicolumn{1}{c}{error (f)} & \multicolumn{1}{c}{error (b)} & \multicolumn{1}{c}{depth (b)} & \multicolumn{1}{c}{size (b)} & \multicolumn{1}{c}{time (b)} & \multicolumn{1}{c}{opt} & \multicolumn{1}{c}{time (a)} & \multicolumn{1}{c}{search (a)} \\
\midrule

\texttt{anneal} & \multicolumn{1}{r}{812} & \multicolumn{1}{r}{47}  & \cellcolor{TealBlue!30}{\textbf{137}} & \cellcolor{TealBlue!30}{\textbf{112}} & \cellcolor{TealBlue!30}{\textbf{3}} & \cellcolor{TealBlue!30}{\textbf{15}} & \cellcolor{TealBlue!30}{\textbf{0.03}} & \cellcolor{TealBlue!30}{\textbf{1}} & \cellcolor{TealBlue!30}{\textbf{0.05}} & \cellcolor{TealBlue!30}{\textbf{6332}}\\
\texttt{audiology} & \multicolumn{1}{r}{216} & \multicolumn{1}{r}{79}  & \cellcolor{TealBlue!30}{\textbf{6}} & \cellcolor{TealBlue!30}{\textbf{5}} & \cellcolor{TealBlue!30}{\textbf{3}} & \cellcolor{TealBlue!30}{\textbf{11}} & \cellcolor{TealBlue!30}{\textbf{0.01}} & \cellcolor{TealBlue!30}{\textbf{1}} & \cellcolor{TealBlue!30}{\textbf{0.08}} & \cellcolor{TealBlue!30}{\textbf{14684}}\\
\texttt{australian-credit} & \multicolumn{1}{r}{653} & \multicolumn{1}{r}{73}  & \cellcolor{TealBlue!30}{\textbf{82}} & \cellcolor{TealBlue!30}{\textbf{73}} & \cellcolor{TealBlue!30}{\textbf{3}} & \cellcolor{TealBlue!30}{\textbf{15}} & \cellcolor{TealBlue!30}{\textbf{0.10}} & \cellcolor{TealBlue!30}{\textbf{1}} & \cellcolor{TealBlue!30}{\textbf{0.15}} & \cellcolor{TealBlue!30}{\textbf{18557}}\\
\texttt{breast-cancer-un} & \multicolumn{1}{r}{683} & \multicolumn{1}{r}{89}  & \cellcolor{TealBlue!30}{\textbf{28}} & \cellcolor{TealBlue!30}{\textbf{24}} & \cellcolor{TealBlue!30}{\textbf{3}} & \cellcolor{TealBlue!30}{\textbf{15}} & \cellcolor{TealBlue!30}{\textbf{0.12}} & \cellcolor{TealBlue!30}{\textbf{1}} & \cellcolor{TealBlue!30}{\textbf{0.13}} & \cellcolor{TealBlue!30}{\textbf{20272}}\\
\texttt{breast-wisconsin} & \multicolumn{1}{r}{683} & \multicolumn{1}{r}{120}  & \cellcolor{TealBlue!30}{\textbf{26}} & \cellcolor{TealBlue!30}{\textbf{15}} & \cellcolor{TealBlue!30}{\textbf{3}} & \cellcolor{TealBlue!30}{\textbf{15}} & \cellcolor{TealBlue!30}{\textbf{0.00}} & \cellcolor{TealBlue!30}{\textbf{1}} & \cellcolor{TealBlue!30}{\textbf{0.07}} & \cellcolor{TealBlue!30}{\textbf{10936}}\\
\texttt{car-un} & \multicolumn{1}{r}{1728} & \multicolumn{1}{r}{21}  & \cellcolor{TealBlue!30}{\textbf{202}} & \cellcolor{TealBlue!30}{\textbf{192}} & \cellcolor{TealBlue!30}{\textbf{3}} & \cellcolor{TealBlue!30}{\textbf{9}} & \cellcolor{TealBlue!30}{\textbf{0.00}} & \cellcolor{TealBlue!30}{\textbf{1}} & \cellcolor{TealBlue!30}{\textbf{0.01}} & \cellcolor{TealBlue!30}{\textbf{1502}}\\
\texttt{diabetes} & \multicolumn{1}{r}{768} & \multicolumn{1}{r}{112}  & \cellcolor{TealBlue!30}{\textbf{169}} & \cellcolor{TealBlue!30}{\textbf{162}} & \cellcolor{TealBlue!30}{\textbf{3}} & \cellcolor{TealBlue!30}{\textbf{15}} & \cellcolor{TealBlue!30}{\textbf{0.01}} & \cellcolor{TealBlue!30}{\textbf{1}} & \cellcolor{TealBlue!30}{\textbf{0.09}} & \cellcolor{TealBlue!30}{\textbf{11564}}\\
\texttt{forest-fires-un} & \multicolumn{1}{r}{517} & \multicolumn{1}{r}{989}  & \cellcolor{TealBlue!30}{\textbf{198}} & \cellcolor{TealBlue!30}{\textbf{193}} & \cellcolor{TealBlue!30}{\textbf{3}} & \cellcolor{TealBlue!30}{\textbf{15}} & \cellcolor{TealBlue!30}{\textbf{20.90}} & \cellcolor{TealBlue!30}{\textbf{1}} & \cellcolor{TealBlue!30}{\textbf{24.10}} & \cellcolor{TealBlue!30}{\textbf{888423}}\\
\texttt{german-credit} & \multicolumn{1}{r}{1000} & \multicolumn{1}{r}{110}  & \cellcolor{TealBlue!30}{\textbf{249}} & \cellcolor{TealBlue!30}{\textbf{236}} & \cellcolor{TealBlue!30}{\textbf{3}} & \cellcolor{TealBlue!30}{\textbf{15}} & \cellcolor{TealBlue!30}{\textbf{0.04}} & \cellcolor{TealBlue!30}{\textbf{1}} & \cellcolor{TealBlue!30}{\textbf{0.28}} & \cellcolor{TealBlue!30}{\textbf{27568}}\\
\texttt{heart-cleveland} & \multicolumn{1}{r}{296} & \multicolumn{1}{r}{50}  & \cellcolor{TealBlue!30}{\textbf{43}} & \cellcolor{TealBlue!30}{\textbf{41}} & \cellcolor{TealBlue!30}{\textbf{3}} & \cellcolor{TealBlue!30}{\textbf{15}} & \cellcolor{TealBlue!30}{\textbf{0.00}} & \cellcolor{TealBlue!30}{\textbf{1}} & \cellcolor{TealBlue!30}{\textbf{0.06}} & \cellcolor{TealBlue!30}{\textbf{10515}}\\
\texttt{hepatitis} & \multicolumn{1}{r}{137} & \multicolumn{1}{r}{68}  & \cellcolor{TealBlue!30}{\textbf{14}} & \cellcolor{TealBlue!30}{\textbf{10}} & \cellcolor{TealBlue!30}{\textbf{3}} & \cellcolor{TealBlue!30}{\textbf{15}} & \cellcolor{TealBlue!30}{\textbf{0.01}} & \cellcolor{TealBlue!30}{\textbf{1}} & \cellcolor{TealBlue!30}{\textbf{0.02}} & \cellcolor{TealBlue!30}{\textbf{3814}}\\
\texttt{hypothyroid} & \multicolumn{1}{r}{3247} & \multicolumn{1}{r}{43}  & \cellcolor{TealBlue!30}{\textbf{62}} & \cellcolor{TealBlue!30}{\textbf{61}} & \cellcolor{TealBlue!30}{\textbf{3}} & \cellcolor{TealBlue!30}{\textbf{13}} & \cellcolor{TealBlue!30}{\textbf{0.04}} & \cellcolor{TealBlue!30}{\textbf{1}} & \cellcolor{TealBlue!30}{\textbf{0.10}} & \cellcolor{TealBlue!30}{\textbf{6339}}\\
\texttt{ionosphere} & \multicolumn{1}{r}{351} & \multicolumn{1}{r}{444}  & \cellcolor{TealBlue!30}{\textbf{29}} & \cellcolor{TealBlue!30}{\textbf{22}} & \cellcolor{TealBlue!30}{\textbf{3}} & \cellcolor{TealBlue!30}{\textbf{15}} & \cellcolor{TealBlue!30}{\textbf{0.57}} & \cellcolor{TealBlue!30}{\textbf{1}} & \cellcolor{TealBlue!30}{\textbf{4.27}} & \cellcolor{TealBlue!30}{\textbf{182760}}\\
\texttt{kr-vs-kp} & \multicolumn{1}{r}{3196} & \multicolumn{1}{r}{37}  & \cellcolor{TealBlue!30}{\textbf{306}} & \cellcolor{TealBlue!30}{\textbf{198}} & \cellcolor{TealBlue!30}{\textbf{3}} & \cellcolor{TealBlue!30}{\textbf{11}} & \cellcolor{TealBlue!30}{\textbf{0.00}} & \cellcolor{TealBlue!30}{\textbf{1}} & \cellcolor{TealBlue!30}{\textbf{0.07}} & \cellcolor{TealBlue!30}{\textbf{4646}}\\
\texttt{letter} & \multicolumn{1}{r}{20000} & \multicolumn{1}{r}{224}  & \cellcolor{TealBlue!30}{\textbf{657}} & \cellcolor{TealBlue!30}{\textbf{369}} & \cellcolor{TealBlue!30}{\textbf{3}} & \cellcolor{TealBlue!30}{\textbf{15}} & \cellcolor{TealBlue!30}{\textbf{6.31}} & \cellcolor{TealBlue!30}{\textbf{1}} & \cellcolor{TealBlue!30}{\textbf{11.00}} & \cellcolor{TealBlue!30}{\textbf{44370}}\\
\texttt{lymph} & \multicolumn{1}{r}{148} & \multicolumn{1}{r}{41}  & \cellcolor{TealBlue!30}{\textbf{16}} & \cellcolor{TealBlue!30}{\textbf{12}} & \cellcolor{TealBlue!30}{\textbf{3}} & \cellcolor{TealBlue!30}{\textbf{15}} & \cellcolor{TealBlue!30}{\textbf{0.00}} & \cellcolor{TealBlue!30}{\textbf{1}} & \cellcolor{TealBlue!30}{\textbf{0.02}} & \cellcolor{TealBlue!30}{\textbf{6294}}\\
\texttt{mushroom} & \multicolumn{1}{r}{8124} & \multicolumn{1}{r}{91}  & \cellcolor{TealBlue!30}{\textbf{280}} & \cellcolor{TealBlue!30}{\textbf{8}} & \cellcolor{TealBlue!30}{\textbf{3}} & \cellcolor{TealBlue!30}{\textbf{13}} & \cellcolor{TealBlue!30}{\textbf{0.00}} & \cellcolor{TealBlue!30}{\textbf{1}} & \cellcolor{TealBlue!30}{\textbf{0.62}} & \cellcolor{TealBlue!30}{\textbf{19409}}\\
\texttt{pendigits} & \multicolumn{1}{r}{7494} & \multicolumn{1}{r}{216}  & \cellcolor{TealBlue!30}{\textbf{51}} & \cellcolor{TealBlue!30}{\textbf{47}} & \cellcolor{TealBlue!30}{\textbf{3}} & \cellcolor{TealBlue!30}{\textbf{13}} & \cellcolor{TealBlue!30}{\textbf{0.68}} & \cellcolor{TealBlue!30}{\textbf{1}} & \cellcolor{TealBlue!30}{\textbf{3.29}} & \cellcolor{TealBlue!30}{\textbf{38424}}\\
\texttt{primary-tumor} & \multicolumn{1}{r}{336} & \multicolumn{1}{r}{16}  & \cellcolor{TealBlue!30}{\textbf{51}} & \cellcolor{TealBlue!30}{\textbf{46}} & \cellcolor{TealBlue!30}{\textbf{3}} & \cellcolor{TealBlue!30}{\textbf{15}} & \cellcolor{TealBlue!30}{\textbf{0.00}} & \cellcolor{TealBlue!30}{\textbf{1}} & \cellcolor{TealBlue!30}{\textbf{0.01}} & \cellcolor{TealBlue!30}{\textbf{1036}}\\
\texttt{segment} & \multicolumn{1}{r}{2310} & \multicolumn{1}{r}{234}  & \cellcolor{TealBlue!30}{\textbf{5}} & \cellcolor{TealBlue!30}{\textbf{0}} & \cellcolor{TealBlue!30}{\textbf{3}} & \cellcolor{TealBlue!30}{\textbf{11}} & \cellcolor{TealBlue!30}{\textbf{0.05}} & \cellcolor{TealBlue!30}{\textbf{1}} & \cellcolor{TealBlue!30}{\textbf{1.12}} & \cellcolor{TealBlue!30}{\textbf{33184}}\\
\texttt{soybean} & \multicolumn{1}{r}{630} & \multicolumn{1}{r}{34}  & \cellcolor{TealBlue!30}{\textbf{47}} & \cellcolor{TealBlue!30}{\textbf{29}} & \cellcolor{TealBlue!30}{\textbf{3}} & \cellcolor{TealBlue!30}{\textbf{15}} & \cellcolor{TealBlue!30}{\textbf{0.01}} & \cellcolor{TealBlue!30}{\textbf{1}} & \cellcolor{TealBlue!30}{\textbf{0.03}} & \cellcolor{TealBlue!30}{\textbf{5227}}\\
\texttt{splice-1} & \multicolumn{1}{r}{3190} & \multicolumn{1}{r}{227}  & \cellcolor{TealBlue!30}{\textbf{279}} & \cellcolor{TealBlue!30}{\textbf{224}} & \cellcolor{TealBlue!30}{\textbf{3}} & \cellcolor{TealBlue!30}{\textbf{15}} & \cellcolor{TealBlue!30}{\textbf{0.12}} & \cellcolor{TealBlue!30}{\textbf{1}} & \cellcolor{TealBlue!30}{\textbf{9.99}} & \cellcolor{TealBlue!30}{\textbf{244223}}\\
\texttt{taiwan\_binarised} & \multicolumn{1}{r}{30000} & \multicolumn{1}{r}{198}  & \cellcolor{TealBlue!30}{\textbf{5333}} & \cellcolor{TealBlue!30}{\textbf{5326}} & \cellcolor{TealBlue!30}{\textbf{3}} & \cellcolor{TealBlue!30}{\textbf{15}} & \cellcolor{TealBlue!30}{\textbf{1.65}} & \cellcolor{TealBlue!30}{\textbf{1}} & \cellcolor{TealBlue!30}{\textbf{29.90}} & \cellcolor{TealBlue!30}{\textbf{143222}}\\
\texttt{tic-tac-toe} & \multicolumn{1}{r}{958} & \multicolumn{1}{r}{18}  & \cellcolor{TealBlue!30}{\textbf{236}} & \cellcolor{TealBlue!30}{\textbf{216}} & \cellcolor{TealBlue!30}{\textbf{3}} & \cellcolor{TealBlue!30}{\textbf{15}} & \cellcolor{TealBlue!30}{\textbf{0.01}} & \cellcolor{TealBlue!30}{\textbf{1}} & \cellcolor{TealBlue!30}{\textbf{0.01}} & \cellcolor{TealBlue!30}{\textbf{2700}}\\
\texttt{vehicle} & \multicolumn{1}{r}{846} & \multicolumn{1}{r}{252}  & \cellcolor{TealBlue!30}{\textbf{55}} & \cellcolor{TealBlue!30}{\textbf{26}} & \cellcolor{TealBlue!30}{\textbf{3}} & \cellcolor{TealBlue!30}{\textbf{15}} & \cellcolor{TealBlue!30}{\textbf{0.03}} & \cellcolor{TealBlue!30}{\textbf{1}} & \cellcolor{TealBlue!30}{\textbf{0.89}} & \cellcolor{TealBlue!30}{\textbf{48196}}\\
\texttt{vote} & \multicolumn{1}{r}{435} & \multicolumn{1}{r}{32}  & \cellcolor{TealBlue!30}{\textbf{14}} & \cellcolor{TealBlue!30}{\textbf{12}} & \cellcolor{TealBlue!30}{\textbf{3}} & \cellcolor{TealBlue!30}{\textbf{13}} & \cellcolor{TealBlue!30}{\textbf{0.01}} & \cellcolor{TealBlue!30}{\textbf{1}} & \cellcolor{TealBlue!30}{\textbf{0.04}} & \cellcolor{TealBlue!30}{\textbf{7608}}\\
\texttt{wine1-un} & \multicolumn{1}{r}{178} & \multicolumn{1}{r}{1276}  & \cellcolor{TealBlue!30}{\textbf{45}} & \cellcolor{TealBlue!30}{\textbf{43}} & \cellcolor{TealBlue!30}{\textbf{3}} & \cellcolor{TealBlue!30}{\textbf{13}} & \cellcolor{TealBlue!30}{\textbf{0.64}} & \cellcolor{TealBlue!30}{\textbf{1}} & \cellcolor{TealBlue!30}{\textbf{20.60}} & \cellcolor{TealBlue!30}{\textbf{832394}}\\
\texttt{wine2-un} & \multicolumn{1}{r}{178} & \multicolumn{1}{r}{1276}  & \cellcolor{TealBlue!30}{\textbf{52}} & \cellcolor{TealBlue!30}{\textbf{49}} & \cellcolor{TealBlue!30}{\textbf{3}} & \cellcolor{TealBlue!30}{\textbf{15}} & \cellcolor{TealBlue!30}{\textbf{0.20}} & \cellcolor{TealBlue!30}{\textbf{1}} & \cellcolor{TealBlue!30}{\textbf{20.00}} & \cellcolor{TealBlue!30}{\textbf{833674}}\\
\texttt{wine3-un} & \multicolumn{1}{r}{178} & \multicolumn{1}{r}{1276}  & \cellcolor{TealBlue!30}{\textbf{35}} & \cellcolor{TealBlue!30}{\textbf{33}} & \cellcolor{TealBlue!30}{\textbf{3}} & \cellcolor{TealBlue!30}{\textbf{13}} & \cellcolor{TealBlue!30}{\textbf{0.10}} & \cellcolor{TealBlue!30}{\textbf{1}} & \cellcolor{TealBlue!30}{\textbf{21.30}} & \cellcolor{TealBlue!30}{\textbf{832097}}\\
\texttt{yeast} & \multicolumn{1}{r}{1484} & \multicolumn{1}{r}{89}  & \cellcolor{TealBlue!30}{\textbf{417}} & \cellcolor{TealBlue!30}{\textbf{403}} & \cellcolor{TealBlue!30}{\textbf{3}} & \cellcolor{TealBlue!30}{\textbf{15}} & \cellcolor{TealBlue!30}{\textbf{0.01}} & \cellcolor{TealBlue!30}{\textbf{1}} & \cellcolor{TealBlue!30}{\textbf{0.08}} & \cellcolor{TealBlue!30}{\textbf{7281}}\\
\texttt{zoo-1} & \multicolumn{1}{r}{101} & \multicolumn{1}{r}{20}  & \cellcolor{TealBlue!30}{\textbf{0}} & \cellcolor{TealBlue!30}{\textbf{0}} & \cellcolor{TealBlue!30}{\textbf{1}} & \cellcolor{TealBlue!30}{\textbf{3}} & \cellcolor{TealBlue!30}{\textbf{0.00}} & \cellcolor{TealBlue!30}{\textbf{1}} & \cellcolor{TealBlue!30}{\textbf{0.00}} & \cellcolor{TealBlue!30}{\textbf{1}}\\
\bottomrule
\end{tabular}

\end{normalsize}
\end{center}
\caption{\label{tab:s3} max depth=3}
\end{table}

\begin{table}[htbp]
\begin{center}
\begin{normalsize}
\tabcolsep=3pt
\begin{tabular}{lccrrrrrrrr}
\toprule
& && \multicolumn{8}{c}{\budalg}\\
\cmidrule(rr){4-11}
&\multirow{1}{*}{$\#ex.$} & \multirow{1}{*}{\#feat.} &  \multicolumn{1}{c}{error (f)} & \multicolumn{1}{c}{error (b)} & \multicolumn{1}{c}{depth (b)} & \multicolumn{1}{c}{size (b)} & \multicolumn{1}{c}{time (b)} & \multicolumn{1}{c}{opt} & \multicolumn{1}{c}{time (a)} & \multicolumn{1}{c}{search (a)} \\
\midrule

\texttt{anneal} & \multicolumn{1}{r}{812} & \multicolumn{1}{r}{47}  & \cellcolor{TealBlue!30}{\textbf{135}} & \cellcolor{TealBlue!30}{\textbf{91}} & \cellcolor{TealBlue!30}{\textbf{4}} & \cellcolor{TealBlue!30}{\textbf{29}} & \cellcolor{TealBlue!30}{\textbf{0.94}} & \cellcolor{TealBlue!30}{\textbf{1}} & \cellcolor{TealBlue!30}{\textbf{1.69}} & \cellcolor{TealBlue!30}{\textbf{341104}}\\
\texttt{audiology} & \multicolumn{1}{r}{216} & \multicolumn{1}{r}{79}  & \cellcolor{TealBlue!30}{\textbf{3}} & \cellcolor{TealBlue!30}{\textbf{1}} & \cellcolor{TealBlue!30}{\textbf{4}} & \cellcolor{TealBlue!30}{\textbf{23}} & \cellcolor{TealBlue!30}{\textbf{0.03}} & \cellcolor{TealBlue!30}{\textbf{1}} & \cellcolor{TealBlue!30}{\textbf{4.62}} & \cellcolor{TealBlue!30}{\textbf{1000235}}\\
\texttt{australian-credit} & \multicolumn{1}{r}{653} & \multicolumn{1}{r}{73}  & \cellcolor{TealBlue!30}{\textbf{73}} & \cellcolor{TealBlue!30}{\textbf{56}} & \cellcolor{TealBlue!30}{\textbf{4}} & \cellcolor{TealBlue!30}{\textbf{31}} & \cellcolor{TealBlue!30}{\textbf{12.20}} & \cellcolor{TealBlue!30}{\textbf{1}} & \cellcolor{TealBlue!30}{\textbf{13.10}} & \cellcolor{TealBlue!30}{\textbf{1891214}}\\
\texttt{breast-cancer-un} & \multicolumn{1}{r}{683} & \multicolumn{1}{r}{89}  & \cellcolor{TealBlue!30}{\textbf{21}} & \cellcolor{TealBlue!30}{\textbf{16}} & \cellcolor{TealBlue!30}{\textbf{4}} & \cellcolor{TealBlue!30}{\textbf{29}} & \cellcolor{TealBlue!30}{\textbf{0.40}} & \cellcolor{TealBlue!30}{\textbf{1}} & \cellcolor{TealBlue!30}{\textbf{10.80}} & \cellcolor{TealBlue!30}{\textbf{1860878}}\\
\texttt{breast-wisconsin} & \multicolumn{1}{r}{683} & \multicolumn{1}{r}{120}  & \cellcolor{TealBlue!30}{\textbf{16}} & \cellcolor{TealBlue!30}{\textbf{7}} & \cellcolor{TealBlue!30}{\textbf{4}} & \cellcolor{TealBlue!30}{\textbf{31}} & \cellcolor{TealBlue!30}{\textbf{1.59}} & \cellcolor{TealBlue!30}{\textbf{1}} & \cellcolor{TealBlue!30}{\textbf{3.82}} & \cellcolor{TealBlue!30}{\textbf{688584}}\\
\texttt{car-un} & \multicolumn{1}{r}{1728} & \multicolumn{1}{r}{21}  & \cellcolor{TealBlue!30}{\textbf{178}} & \cellcolor{TealBlue!30}{\textbf{136}} & \cellcolor{TealBlue!30}{\textbf{4}} & \cellcolor{TealBlue!30}{\textbf{25}} & \cellcolor{TealBlue!30}{\textbf{0.05}} & \cellcolor{TealBlue!30}{\textbf{1}} & \cellcolor{TealBlue!30}{\textbf{0.16}} & \cellcolor{TealBlue!30}{\textbf{46288}}\\
\texttt{diabetes} & \multicolumn{1}{r}{768} & \multicolumn{1}{r}{112}  & \cellcolor{TealBlue!30}{\textbf{159}} & \cellcolor{TealBlue!30}{\textbf{137}} & \cellcolor{TealBlue!30}{\textbf{4}} & \cellcolor{TealBlue!30}{\textbf{31}} & \cellcolor{TealBlue!30}{\textbf{0.07}} & \cellcolor{TealBlue!30}{\textbf{1}} & \cellcolor{TealBlue!30}{\textbf{6.24}} & \cellcolor{TealBlue!30}{\textbf{1052519}}\\
\texttt{forest-fires-un} & \multicolumn{1}{r}{517} & \multicolumn{1}{r}{989}  & \cellcolor{TealBlue!30}{\textbf{186}} & \cellcolor{TealBlue!30}{\textbf{173}} & \cellcolor{TealBlue!30}{\textbf{4}} & \cellcolor{TealBlue!30}{\textbf{29}} & \cellcolor{TealBlue!30}{\textbf{17.90}} & \cellcolor{TealBlue!30}{\textbf{0}} & - & -\\
\texttt{german-credit} & \multicolumn{1}{r}{1000} & \multicolumn{1}{r}{110}  & \cellcolor{TealBlue!30}{\textbf{224}} & \cellcolor{TealBlue!30}{\textbf{204}} & \cellcolor{TealBlue!30}{\textbf{4}} & \cellcolor{TealBlue!30}{\textbf{31}} & \cellcolor{TealBlue!30}{\textbf{0.41}} & \cellcolor{TealBlue!30}{\textbf{1}} & \cellcolor{TealBlue!30}{\textbf{31.30}} & \cellcolor{TealBlue!30}{\textbf{3777150}}\\
\texttt{heart-cleveland} & \multicolumn{1}{r}{296} & \multicolumn{1}{r}{50}  & \cellcolor{TealBlue!30}{\textbf{36}} & \cellcolor{TealBlue!30}{\textbf{25}} & \cellcolor{TealBlue!30}{\textbf{4}} & \cellcolor{TealBlue!30}{\textbf{31}} & \cellcolor{TealBlue!30}{\textbf{1.60}} & \cellcolor{TealBlue!30}{\textbf{1}} & \cellcolor{TealBlue!30}{\textbf{3.82}} & \cellcolor{TealBlue!30}{\textbf{778085}}\\
\texttt{hepatitis} & \multicolumn{1}{r}{137} & \multicolumn{1}{r}{68}  & \cellcolor{TealBlue!30}{\textbf{12}} & \cellcolor{TealBlue!30}{\textbf{3}} & \cellcolor{TealBlue!30}{\textbf{4}} & \cellcolor{TealBlue!30}{\textbf{29}} & \cellcolor{TealBlue!30}{\textbf{0.27}} & \cellcolor{TealBlue!30}{\textbf{1}} & \cellcolor{TealBlue!30}{\textbf{0.34}} & \cellcolor{TealBlue!30}{\textbf{120490}}\\
\texttt{hypothyroid} & \multicolumn{1}{r}{3247} & \multicolumn{1}{r}{43}  & \cellcolor{TealBlue!30}{\textbf{53}} & \cellcolor{TealBlue!30}{\textbf{53}} & \cellcolor{TealBlue!30}{\textbf{4}} & \cellcolor{TealBlue!30}{\textbf{23}} & \cellcolor{TealBlue!30}{\textbf{0.00}} & \cellcolor{TealBlue!30}{\textbf{1}} & \cellcolor{TealBlue!30}{\textbf{3.70}} & \cellcolor{TealBlue!30}{\textbf{329869}}\\
\texttt{ionosphere} & \multicolumn{1}{r}{351} & \multicolumn{1}{r}{444}  & \cellcolor{TealBlue!30}{\textbf{25}} & \cellcolor{TealBlue!30}{\textbf{7}} & \cellcolor{TealBlue!30}{\textbf{4}} & \cellcolor{TealBlue!30}{\textbf{31}} & \cellcolor{TealBlue!30}{\textbf{442.00}} & \cellcolor{TealBlue!30}{\textbf{1}} & \cellcolor{TealBlue!30}{\textbf{966.00}} & \cellcolor{TealBlue!30}{\textbf{44451199}}\\
\texttt{kr-vs-kp} & \multicolumn{1}{r}{3196} & \multicolumn{1}{r}{37}  & \cellcolor{TealBlue!30}{\textbf{188}} & \cellcolor{TealBlue!30}{\textbf{144}} & \cellcolor{TealBlue!30}{\textbf{4}} & \cellcolor{TealBlue!30}{\textbf{27}} & \cellcolor{TealBlue!30}{\textbf{0.51}} & \cellcolor{TealBlue!30}{\textbf{1}} & \cellcolor{TealBlue!30}{\textbf{2.24}} & \cellcolor{TealBlue!30}{\textbf{230292}}\\
\texttt{letter} & \multicolumn{1}{r}{20000} & \multicolumn{1}{r}{224}  & \cellcolor{TealBlue!30}{\textbf{443}} & \cellcolor{TealBlue!30}{\textbf{261}} & \cellcolor{TealBlue!30}{\textbf{4}} & \cellcolor{TealBlue!30}{\textbf{29}} & \cellcolor{TealBlue!30}{\textbf{55.90}} & \cellcolor{TealBlue!30}{\textbf{1}} & \cellcolor{TealBlue!30}{\textbf{979.00}} & \cellcolor{TealBlue!30}{\textbf{7294586}}\\
\texttt{lymph} & \multicolumn{1}{r}{148} & \multicolumn{1}{r}{41}  & \cellcolor{TealBlue!30}{\textbf{9}} & \cellcolor{TealBlue!30}{\textbf{3}} & \cellcolor{TealBlue!30}{\textbf{4}} & \cellcolor{TealBlue!30}{\textbf{31}} & \cellcolor{TealBlue!30}{\textbf{0.00}} & \cellcolor{TealBlue!30}{\textbf{1}} & \cellcolor{TealBlue!30}{\textbf{0.79}} & \cellcolor{TealBlue!30}{\textbf{242060}}\\
\texttt{mushroom} & \multicolumn{1}{r}{8124} & \multicolumn{1}{r}{91}  & \cellcolor{TealBlue!30}{\textbf{4}} & \cellcolor{TealBlue!30}{\textbf{0}} & \cellcolor{TealBlue!30}{\textbf{4}} & \cellcolor{TealBlue!30}{\textbf{15}} & \cellcolor{TealBlue!30}{\textbf{0.27}} & \cellcolor{TealBlue!30}{\textbf{1}} & \cellcolor{TealBlue!30}{\textbf{55.00}} & \cellcolor{TealBlue!30}{\textbf{2017043}}\\
\texttt{pendigits} & \multicolumn{1}{r}{7494} & \multicolumn{1}{r}{216}  & \cellcolor{TealBlue!30}{\textbf{22}} & \cellcolor{TealBlue!30}{\textbf{13}} & \cellcolor{TealBlue!30}{\textbf{4}} & \cellcolor{TealBlue!30}{\textbf{27}} & \cellcolor{TealBlue!30}{\textbf{57.90}} & \cellcolor{TealBlue!30}{\textbf{1}} & \cellcolor{TealBlue!30}{\textbf{246.00}} & \cellcolor{TealBlue!30}{\textbf{4746647}}\\
\texttt{primary-tumor} & \multicolumn{1}{r}{336} & \multicolumn{1}{r}{16}  & \cellcolor{TealBlue!30}{\textbf{43}} & \cellcolor{TealBlue!30}{\textbf{34}} & \cellcolor{TealBlue!30}{\textbf{4}} & \cellcolor{TealBlue!30}{\textbf{31}} & \cellcolor{TealBlue!30}{\textbf{0.00}} & \cellcolor{TealBlue!30}{\textbf{1}} & \cellcolor{TealBlue!30}{\textbf{0.04}} & \cellcolor{TealBlue!30}{\textbf{24241}}\\
\texttt{segment} & \multicolumn{1}{r}{2310} & \multicolumn{1}{r}{234}  & \cellcolor{TealBlue!30}{\textbf{1}} & \cellcolor{TealBlue!30}{\textbf{0}} & \cellcolor{TealBlue!30}{\textbf{4}} & \cellcolor{TealBlue!30}{\textbf{11}} & \cellcolor{TealBlue!30}{\textbf{0.00}} & \cellcolor{TealBlue!30}{\textbf{1}} & \cellcolor{TealBlue!30}{\textbf{69.80}} & \cellcolor{TealBlue!30}{\textbf{3955297}}\\
\texttt{soybean} & \multicolumn{1}{r}{630} & \multicolumn{1}{r}{34}  & \cellcolor{TealBlue!30}{\textbf{32}} & \cellcolor{TealBlue!30}{\textbf{14}} & \cellcolor{TealBlue!30}{\textbf{4}} & \cellcolor{TealBlue!30}{\textbf{27}} & \cellcolor{TealBlue!30}{\textbf{0.13}} & \cellcolor{TealBlue!30}{\textbf{1}} & \cellcolor{TealBlue!30}{\textbf{0.88}} & \cellcolor{TealBlue!30}{\textbf{253314}}\\
\texttt{splice-1} & \multicolumn{1}{r}{3190} & \multicolumn{1}{r}{227}  & \cellcolor{TealBlue!30}{\textbf{141}} & \cellcolor{TealBlue!30}{\textbf{141}} & \cellcolor{TealBlue!30}{\textbf{4}} & \cellcolor{TealBlue!30}{\textbf{29}} & \cellcolor{TealBlue!30}{\textbf{0.00}} & \cellcolor{TealBlue!30}{\textbf{1}} & \cellcolor{TealBlue!30}{\textbf{3520.00}} & \cellcolor{TealBlue!30}{\textbf{108716836}}\\
\texttt{taiwan\_binarised} & \multicolumn{1}{r}{30000} & \multicolumn{1}{r}{198}  & \cellcolor{TealBlue!30}{\textbf{5293}} & \cellcolor{TealBlue!30}{\textbf{5273}} & \cellcolor{TealBlue!30}{\textbf{4}} & \cellcolor{TealBlue!30}{\textbf{31}} & \cellcolor{TealBlue!30}{\textbf{6.53}} & \cellcolor{TealBlue!30}{\textbf{0}} & - & -\\
\texttt{tic-tac-toe} & \multicolumn{1}{r}{958} & \multicolumn{1}{r}{18}  & \cellcolor{TealBlue!30}{\textbf{150}} & \cellcolor{TealBlue!30}{\textbf{137}} & \cellcolor{TealBlue!30}{\textbf{4}} & \cellcolor{TealBlue!30}{\textbf{27}} & \cellcolor{TealBlue!30}{\textbf{0.16}} & \cellcolor{TealBlue!30}{\textbf{1}} & \cellcolor{TealBlue!30}{\textbf{0.41}} & \cellcolor{TealBlue!30}{\textbf{123924}}\\
\texttt{vehicle} & \multicolumn{1}{r}{846} & \multicolumn{1}{r}{252}  & \cellcolor{TealBlue!30}{\textbf{28}} & \cellcolor{TealBlue!30}{\textbf{12}} & \cellcolor{TealBlue!30}{\textbf{4}} & \cellcolor{TealBlue!30}{\textbf{29}} & \cellcolor{TealBlue!30}{\textbf{4.93}} & \cellcolor{TealBlue!30}{\textbf{1}} & \cellcolor{TealBlue!30}{\textbf{86.40}} & \cellcolor{TealBlue!30}{\textbf{6395374}}\\
\texttt{vote} & \multicolumn{1}{r}{435} & \multicolumn{1}{r}{32}  & \cellcolor{TealBlue!30}{\textbf{8}} & \cellcolor{TealBlue!30}{\textbf{5}} & \cellcolor{TealBlue!30}{\textbf{4}} & \cellcolor{TealBlue!30}{\textbf{29}} & \cellcolor{TealBlue!30}{\textbf{0.08}} & \cellcolor{TealBlue!30}{\textbf{1}} & \cellcolor{TealBlue!30}{\textbf{1.55}} & \cellcolor{TealBlue!30}{\textbf{397257}}\\
\texttt{wine1-un} & \multicolumn{1}{r}{178} & \multicolumn{1}{r}{1276}  & \cellcolor{TealBlue!30}{\textbf{42}} & \cellcolor{TealBlue!30}{\textbf{37}} & \cellcolor{TealBlue!30}{\textbf{4}} & \cellcolor{TealBlue!30}{\textbf{23}} & \cellcolor{TealBlue!30}{\textbf{1980.00}} & \cellcolor{TealBlue!30}{\textbf{0}} & - & -\\
\texttt{wine2-un} & \multicolumn{1}{r}{178} & \multicolumn{1}{r}{1276}  & \cellcolor{TealBlue!30}{\textbf{47}} & \cellcolor{TealBlue!30}{\textbf{43}} & \cellcolor{TealBlue!30}{\textbf{4}} & \cellcolor{TealBlue!30}{\textbf{21}} & \cellcolor{TealBlue!30}{\textbf{19.30}} & \cellcolor{TealBlue!30}{\textbf{0}} & - & -\\
\texttt{wine3-un} & \multicolumn{1}{r}{178} & \multicolumn{1}{r}{1276}  & \cellcolor{TealBlue!30}{\textbf{32}} & \cellcolor{TealBlue!30}{\textbf{28}} & \cellcolor{TealBlue!30}{\textbf{4}} & \cellcolor{TealBlue!30}{\textbf{21}} & \cellcolor{TealBlue!30}{\textbf{39.70}} & \cellcolor{TealBlue!30}{\textbf{0}} & - & -\\
\texttt{yeast} & \multicolumn{1}{r}{1484} & \multicolumn{1}{r}{89}  & \cellcolor{TealBlue!30}{\textbf{392}} & \cellcolor{TealBlue!30}{\textbf{366}} & \cellcolor{TealBlue!30}{\textbf{4}} & \cellcolor{TealBlue!30}{\textbf{31}} & \cellcolor{TealBlue!30}{\textbf{2.69}} & \cellcolor{TealBlue!30}{\textbf{1}} & \cellcolor{TealBlue!30}{\textbf{3.56}} & \cellcolor{TealBlue!30}{\textbf{508819}}\\
\texttt{zoo-1} & \multicolumn{1}{r}{101} & \multicolumn{1}{r}{20}  & \cellcolor{TealBlue!30}{\textbf{0}} & \cellcolor{TealBlue!30}{\textbf{0}} & \cellcolor{TealBlue!30}{\textbf{1}} & \cellcolor{TealBlue!30}{\textbf{3}} & \cellcolor{TealBlue!30}{\textbf{0.00}} & \cellcolor{TealBlue!30}{\textbf{1}} & \cellcolor{TealBlue!30}{\textbf{0.00}} & \cellcolor{TealBlue!30}{\textbf{1}}\\
\bottomrule
\end{tabular}

\end{normalsize}
\end{center}
\caption{\label{tab:s4} max depth=4}
\end{table}

\begin{table}[htbp]
\begin{center}
\begin{normalsize}
\tabcolsep=3pt
\begin{tabular}{lccrrrrrrrr}
\toprule
& && \multicolumn{8}{c}{\budalg}\\
\cmidrule(rr){4-11}
&\multirow{1}{*}{$\#ex.$} & \multirow{1}{*}{\#feat.} &  \multicolumn{1}{c}{error (f)} & \multicolumn{1}{c}{error (b)} & \multicolumn{1}{c}{depth (b)} & \multicolumn{1}{c}{size (b)} & \multicolumn{1}{c}{time (b)} & \multicolumn{1}{c}{opt} & \multicolumn{1}{c}{time (a)} & \multicolumn{1}{c}{search (a)} \\
\midrule

\texttt{anneal} & \multicolumn{1}{r}{812} & \multicolumn{1}{r}{47}  & \cellcolor{TealBlue!30}{\textbf{114}} & \cellcolor{TealBlue!30}{\textbf{70}} & \cellcolor{TealBlue!30}{\textbf{5}} & \cellcolor{TealBlue!30}{\textbf{53}} & \cellcolor{TealBlue!30}{\textbf{44.20}} & \cellcolor{TealBlue!30}{\textbf{1}} & \cellcolor{TealBlue!30}{\textbf{69.60}} & \cellcolor{TealBlue!30}{\textbf{15945623}}\\
\texttt{audiology} & \multicolumn{1}{r}{216} & \multicolumn{1}{r}{79}  & \cellcolor{TealBlue!30}{\textbf{2}} & \cellcolor{TealBlue!30}{\textbf{0}} & \cellcolor{TealBlue!30}{\textbf{5}} & \cellcolor{TealBlue!30}{\textbf{21}} & \cellcolor{TealBlue!30}{\textbf{0.53}} & \cellcolor{TealBlue!30}{\textbf{1}} & \cellcolor{TealBlue!30}{\textbf{362.00}} & \cellcolor{TealBlue!30}{\textbf{85985117}}\\
\texttt{australian-credit} & \multicolumn{1}{r}{653} & \multicolumn{1}{r}{73}  & \cellcolor{TealBlue!30}{\textbf{63}} & \cellcolor{TealBlue!30}{\textbf{39}} & \cellcolor{TealBlue!30}{\textbf{5}} & \cellcolor{TealBlue!30}{\textbf{63}} & \cellcolor{TealBlue!30}{\textbf{431.00}} & \cellcolor{TealBlue!30}{\textbf{1}} & \cellcolor{TealBlue!30}{\textbf{923.00}} & \cellcolor{TealBlue!30}{\textbf{150943152}}\\
\texttt{breast-cancer-un} & \multicolumn{1}{r}{683} & \multicolumn{1}{r}{89}  & \cellcolor{TealBlue!30}{\textbf{16}} & \cellcolor{TealBlue!30}{\textbf{6}} & \cellcolor{TealBlue!30}{\textbf{5}} & \cellcolor{TealBlue!30}{\textbf{47}} & \cellcolor{TealBlue!30}{\textbf{38.40}} & \cellcolor{TealBlue!30}{\textbf{1}} & \cellcolor{TealBlue!30}{\textbf{891.00}} & \cellcolor{TealBlue!30}{\textbf{151765942}}\\
\texttt{breast-wisconsin} & \multicolumn{1}{r}{683} & \multicolumn{1}{r}{120}  & \cellcolor{TealBlue!30}{\textbf{13}} & \cellcolor{TealBlue!30}{\textbf{0}} & \cellcolor{TealBlue!30}{\textbf{5}} & \cellcolor{TealBlue!30}{\textbf{49}} & \cellcolor{TealBlue!30}{\textbf{99.60}} & \cellcolor{TealBlue!30}{\textbf{1}} & \cellcolor{TealBlue!30}{\textbf{211.00}} & \cellcolor{TealBlue!30}{\textbf{50820345}}\\
\texttt{car-un} & \multicolumn{1}{r}{1728} & \multicolumn{1}{r}{21}  & \cellcolor{TealBlue!30}{\textbf{106}} & \cellcolor{TealBlue!30}{\textbf{86}} & \cellcolor{TealBlue!30}{\textbf{5}} & \cellcolor{TealBlue!30}{\textbf{37}} & \cellcolor{TealBlue!30}{\textbf{0.82}} & \cellcolor{TealBlue!30}{\textbf{1}} & \cellcolor{TealBlue!30}{\textbf{2.70}} & \cellcolor{TealBlue!30}{\textbf{1100690}}\\
\texttt{diabetes} & \multicolumn{1}{r}{768} & \multicolumn{1}{r}{112}  & \cellcolor{TealBlue!30}{\textbf{141}} & \cellcolor{TealBlue!30}{\textbf{106}} & \cellcolor{TealBlue!30}{\textbf{5}} & \cellcolor{TealBlue!30}{\textbf{63}} & \cellcolor{TealBlue!30}{\textbf{83.70}} & \cellcolor{TealBlue!30}{\textbf{1}} & \cellcolor{TealBlue!30}{\textbf{393.00}} & \cellcolor{TealBlue!30}{\textbf{78408884}}\\
\texttt{forest-fires-un} & \multicolumn{1}{r}{517} & \multicolumn{1}{r}{989}  & \cellcolor{TealBlue!30}{\textbf{176}} & \cellcolor{TealBlue!30}{\textbf{156}} & \cellcolor{TealBlue!30}{\textbf{5}} & \cellcolor{TealBlue!30}{\textbf{49}} & \cellcolor{TealBlue!30}{\textbf{1030.00}} & \cellcolor{TealBlue!30}{\textbf{0}} & - & -\\
\texttt{german-credit} & \multicolumn{1}{r}{1000} & \multicolumn{1}{r}{110}  & \cellcolor{TealBlue!30}{\textbf{201}} & \cellcolor{TealBlue!30}{\textbf{161}} & \cellcolor{TealBlue!30}{\textbf{5}} & \cellcolor{TealBlue!30}{\textbf{63}} & \cellcolor{TealBlue!30}{\textbf{44.50}} & \cellcolor{TealBlue!30}{\textbf{1}} & \cellcolor{TealBlue!30}{\textbf{3330.00}} & \cellcolor{TealBlue!30}{\textbf{431148502}}\\
\texttt{heart-cleveland} & \multicolumn{1}{r}{296} & \multicolumn{1}{r}{50}  & \cellcolor{TealBlue!30}{\textbf{26}} & \cellcolor{TealBlue!30}{\textbf{7}} & \cellcolor{TealBlue!30}{\textbf{5}} & \cellcolor{TealBlue!30}{\textbf{63}} & \cellcolor{TealBlue!30}{\textbf{9.84}} & \cellcolor{TealBlue!30}{\textbf{1}} & \cellcolor{TealBlue!30}{\textbf{127.00}} & \cellcolor{TealBlue!30}{\textbf{30306334}}\\
\texttt{hepatitis} & \multicolumn{1}{r}{137} & \multicolumn{1}{r}{68}  & \cellcolor{TealBlue!30}{\textbf{8}} & \cellcolor{TealBlue!30}{\textbf{0}} & \cellcolor{TealBlue!30}{\textbf{5}} & \cellcolor{TealBlue!30}{\textbf{35}} & \cellcolor{TealBlue!30}{\textbf{0.22}} & \cellcolor{TealBlue!30}{\textbf{1}} & \cellcolor{TealBlue!30}{\textbf{10.10}} & \cellcolor{TealBlue!30}{\textbf{4817264}}\\
\texttt{hypothyroid} & \multicolumn{1}{r}{3247} & \multicolumn{1}{r}{43}  & \cellcolor{TealBlue!30}{\textbf{50}} & \cellcolor{TealBlue!30}{\textbf{44}} & \cellcolor{TealBlue!30}{\textbf{5}} & \cellcolor{TealBlue!30}{\textbf{45}} & \cellcolor{TealBlue!30}{\textbf{40.50}} & \cellcolor{TealBlue!30}{\textbf{1}} & \cellcolor{TealBlue!30}{\textbf{142.00}} & \cellcolor{TealBlue!30}{\textbf{15022732}}\\
\texttt{ionosphere} & \multicolumn{1}{r}{351} & \multicolumn{1}{r}{444}  & \cellcolor{TealBlue!30}{\textbf{16}} & \cellcolor{TealBlue!30}{\textbf{0}} & \cellcolor{TealBlue!30}{\textbf{5}} & \cellcolor{TealBlue!30}{\textbf{43}} & \cellcolor{TealBlue!30}{\textbf{3180.00}} & \cellcolor{TealBlue!30}{\textbf{0}} & - & -\\
\texttt{kr-vs-kp} & \multicolumn{1}{r}{3196} & \multicolumn{1}{r}{37}  & \cellcolor{TealBlue!30}{\textbf{179}} & \cellcolor{TealBlue!30}{\textbf{81}} & \cellcolor{TealBlue!30}{\textbf{5}} & \cellcolor{TealBlue!30}{\textbf{47}} & \cellcolor{TealBlue!30}{\textbf{5.00}} & \cellcolor{TealBlue!30}{\textbf{1}} & \cellcolor{TealBlue!30}{\textbf{65.90}} & \cellcolor{TealBlue!30}{\textbf{8883948}}\\
\texttt{letter} & \multicolumn{1}{r}{20000} & \multicolumn{1}{r}{224}  & \cellcolor{TealBlue!30}{\textbf{335}} & \cellcolor{TealBlue!30}{\textbf{168}} & \cellcolor{TealBlue!30}{\textbf{5}} & \cellcolor{TealBlue!30}{\textbf{55}} & \cellcolor{TealBlue!30}{\textbf{3460.00}} & \cellcolor{TealBlue!30}{\textbf{0}} & - & -\\
\texttt{lymph} & \multicolumn{1}{r}{148} & \multicolumn{1}{r}{41}  & \cellcolor{TealBlue!30}{\textbf{4}} & \cellcolor{TealBlue!30}{\textbf{0}} & \cellcolor{TealBlue!30}{\textbf{5}} & \cellcolor{TealBlue!30}{\textbf{35}} & \cellcolor{TealBlue!30}{\textbf{1.11}} & \cellcolor{TealBlue!30}{\textbf{1}} & \cellcolor{TealBlue!30}{\textbf{33.80}} & \cellcolor{TealBlue!30}{\textbf{12447288}}\\
\texttt{mushroom} & \multicolumn{1}{r}{8124} & \multicolumn{1}{r}{91}  & \cellcolor{TealBlue!30}{\textbf{3}} & \cellcolor{TealBlue!30}{\textbf{0}} & \cellcolor{TealBlue!30}{\textbf{4}} & \cellcolor{TealBlue!30}{\textbf{15}} & \cellcolor{TealBlue!30}{\textbf{0.26}} & \cellcolor{TealBlue!30}{\textbf{1}} & \cellcolor{TealBlue!30}{\textbf{55.70}} & \cellcolor{TealBlue!30}{\textbf{2017055}}\\
\texttt{pendigits} & \multicolumn{1}{r}{7494} & \multicolumn{1}{r}{216}  & \cellcolor{TealBlue!30}{\textbf{11}} & \cellcolor{TealBlue!30}{\textbf{0}} & \cellcolor{TealBlue!30}{\textbf{5}} & \cellcolor{TealBlue!30}{\textbf{43}} & \cellcolor{TealBlue!30}{\textbf{3340.00}} & \cellcolor{TealBlue!30}{\textbf{0}} & - & -\\
\texttt{primary-tumor} & \multicolumn{1}{r}{336} & \multicolumn{1}{r}{16}  & \cellcolor{TealBlue!30}{\textbf{34}} & \cellcolor{TealBlue!30}{\textbf{26}} & \cellcolor{TealBlue!30}{\textbf{5}} & \cellcolor{TealBlue!30}{\textbf{61}} & \cellcolor{TealBlue!30}{\textbf{0.12}} & \cellcolor{TealBlue!30}{\textbf{1}} & \cellcolor{TealBlue!30}{\textbf{0.63}} & \cellcolor{TealBlue!30}{\textbf{434809}}\\
\texttt{segment} & \multicolumn{1}{r}{2310} & \multicolumn{1}{r}{234}  & \cellcolor{TealBlue!30}{\textbf{1}} & \cellcolor{TealBlue!30}{\textbf{0}} & \cellcolor{TealBlue!30}{\textbf{4}} & \cellcolor{TealBlue!30}{\textbf{11}} & \cellcolor{TealBlue!30}{\textbf{0.00}} & \cellcolor{TealBlue!30}{\textbf{1}} & \cellcolor{TealBlue!30}{\textbf{69.30}} & \cellcolor{TealBlue!30}{\textbf{3955315}}\\
\texttt{soybean} & \multicolumn{1}{r}{630} & \multicolumn{1}{r}{34}  & \cellcolor{TealBlue!30}{\textbf{23}} & \cellcolor{TealBlue!30}{\textbf{8}} & \cellcolor{TealBlue!30}{\textbf{5}} & \cellcolor{TealBlue!30}{\textbf{49}} & \cellcolor{TealBlue!30}{\textbf{3.82}} & \cellcolor{TealBlue!30}{\textbf{1}} & \cellcolor{TealBlue!30}{\textbf{30.00}} & \cellcolor{TealBlue!30}{\textbf{9732632}}\\
\texttt{splice-1} & \multicolumn{1}{r}{3190} & \multicolumn{1}{r}{227}  & \cellcolor{TealBlue!30}{\textbf{111}} & \cellcolor{TealBlue!30}{\textbf{101}} & \cellcolor{TealBlue!30}{\textbf{5}} & \cellcolor{TealBlue!30}{\textbf{49}} & \cellcolor{TealBlue!30}{\textbf{1870.00}} & \cellcolor{TealBlue!30}{\textbf{0}} & - & -\\
\texttt{taiwan\_binarised} & \multicolumn{1}{r}{30000} & \multicolumn{1}{r}{198}  & \cellcolor{TealBlue!30}{\textbf{5257}} & \cellcolor{TealBlue!30}{\textbf{5200}} & \cellcolor{TealBlue!30}{\textbf{5}} & \cellcolor{TealBlue!30}{\textbf{63}} & \cellcolor{TealBlue!30}{\textbf{936.00}} & \cellcolor{TealBlue!30}{\textbf{0}} & - & -\\
\texttt{tic-tac-toe} & \multicolumn{1}{r}{958} & \multicolumn{1}{r}{18}  & \cellcolor{TealBlue!30}{\textbf{78}} & \cellcolor{TealBlue!30}{\textbf{63}} & \cellcolor{TealBlue!30}{\textbf{5}} & \cellcolor{TealBlue!30}{\textbf{47}} & \cellcolor{TealBlue!30}{\textbf{0.00}} & \cellcolor{TealBlue!30}{\textbf{1}} & \cellcolor{TealBlue!30}{\textbf{10.90}} & \cellcolor{TealBlue!30}{\textbf{4387432}}\\
\texttt{vehicle} & \multicolumn{1}{r}{846} & \multicolumn{1}{r}{252}  & \cellcolor{TealBlue!30}{\textbf{21}} & \cellcolor{TealBlue!30}{\textbf{1}} & \cellcolor{TealBlue!30}{\textbf{5}} & \cellcolor{TealBlue!30}{\textbf{51}} & \cellcolor{TealBlue!30}{\textbf{763.00}} & \cellcolor{TealBlue!30}{\textbf{0}} & - & -\\
\texttt{vote} & \multicolumn{1}{r}{435} & \multicolumn{1}{r}{32}  & \cellcolor{TealBlue!30}{\textbf{6}} & \cellcolor{TealBlue!30}{\textbf{1}} & \cellcolor{TealBlue!30}{\textbf{5}} & \cellcolor{TealBlue!30}{\textbf{43}} & \cellcolor{TealBlue!30}{\textbf{0.02}} & \cellcolor{TealBlue!30}{\textbf{1}} & \cellcolor{TealBlue!30}{\textbf{30.40}} & \cellcolor{TealBlue!30}{\textbf{8831574}}\\
\texttt{wine1-un} & \multicolumn{1}{r}{178} & \multicolumn{1}{r}{1276}  & \cellcolor{TealBlue!30}{\textbf{39}} & \cellcolor{TealBlue!30}{\textbf{33}} & \cellcolor{TealBlue!30}{\textbf{5}} & \cellcolor{TealBlue!30}{\textbf{25}} & \cellcolor{TealBlue!30}{\textbf{1400.00}} & \cellcolor{TealBlue!30}{\textbf{0}} & - & -\\
\texttt{wine2-un} & \multicolumn{1}{r}{178} & \multicolumn{1}{r}{1276}  & \cellcolor{TealBlue!30}{\textbf{44}} & \cellcolor{TealBlue!30}{\textbf{39}} & \cellcolor{TealBlue!30}{\textbf{5}} & \cellcolor{TealBlue!30}{\textbf{23}} & \cellcolor{TealBlue!30}{\textbf{493.00}} & \cellcolor{TealBlue!30}{\textbf{0}} & - & -\\
\texttt{wine3-un} & \multicolumn{1}{r}{178} & \multicolumn{1}{r}{1276}  & \cellcolor{TealBlue!30}{\textbf{30}} & \cellcolor{TealBlue!30}{\textbf{25}} & \cellcolor{TealBlue!30}{\textbf{5}} & \cellcolor{TealBlue!30}{\textbf{23}} & \cellcolor{TealBlue!30}{\textbf{20.80}} & \cellcolor{TealBlue!30}{\textbf{0}} & - & -\\
\texttt{yeast} & \multicolumn{1}{r}{1484} & \multicolumn{1}{r}{89}  & \cellcolor{TealBlue!30}{\textbf{365}} & \cellcolor{TealBlue!30}{\textbf{313}} & \cellcolor{TealBlue!30}{\textbf{5}} & \cellcolor{TealBlue!30}{\textbf{63}} & \cellcolor{TealBlue!30}{\textbf{36.90}} & \cellcolor{TealBlue!30}{\textbf{1}} & \cellcolor{TealBlue!30}{\textbf{161.00}} & \cellcolor{TealBlue!30}{\textbf{30774291}}\\
\texttt{zoo-1} & \multicolumn{1}{r}{101} & \multicolumn{1}{r}{20}  & \cellcolor{TealBlue!30}{\textbf{0}} & \cellcolor{TealBlue!30}{\textbf{0}} & \cellcolor{TealBlue!30}{\textbf{1}} & \cellcolor{TealBlue!30}{\textbf{3}} & \cellcolor{TealBlue!30}{\textbf{0.00}} & \cellcolor{TealBlue!30}{\textbf{1}} & \cellcolor{TealBlue!30}{\textbf{0.00}} & \cellcolor{TealBlue!30}{\textbf{1}}\\
\bottomrule
\end{tabular}

\end{normalsize}
\end{center}
\caption{\label{tab:s5} max depth=5}
\end{table}

\begin{table}[htbp]
\begin{center}
\begin{normalsize}
\tabcolsep=3pt
\begin{tabular}{lccrrrrrrrr}
\toprule
& && \multicolumn{8}{c}{\budalg}\\
\cmidrule(rr){4-11}
&\multirow{1}{*}{$\#ex.$} & \multirow{1}{*}{\#feat.} &  \multicolumn{1}{c}{error (f)} & \multicolumn{1}{c}{error (b)} & \multicolumn{1}{c}{depth (b)} & \multicolumn{1}{c}{size (b)} & \multicolumn{1}{c}{time (b)} & \multicolumn{1}{c}{opt} & \multicolumn{1}{c}{time (a)} & \multicolumn{1}{c}{search (a)} \\
\midrule

\texttt{anneal} & \multicolumn{1}{r}{812} & \multicolumn{1}{r}{47}  & \cellcolor{TealBlue!30}{\textbf{94}} & \cellcolor{TealBlue!30}{\textbf{50}} & \cellcolor{TealBlue!30}{\textbf{7}} & \cellcolor{TealBlue!30}{\textbf{111}} & \cellcolor{TealBlue!30}{\textbf{383.00}} & \cellcolor{TealBlue!30}{\textbf{0}} & - & -\\
\texttt{audiology} & \multicolumn{1}{r}{216} & \multicolumn{1}{r}{79}  & \cellcolor{TealBlue!30}{\textbf{0}} & \cellcolor{TealBlue!30}{\textbf{0}} & \cellcolor{TealBlue!30}{\textbf{5}} & \cellcolor{TealBlue!30}{\textbf{21}} & \cellcolor{TealBlue!30}{\textbf{0.52}} & \cellcolor{TealBlue!30}{\textbf{1}} & \cellcolor{TealBlue!30}{\textbf{373.00}} & \cellcolor{TealBlue!30}{\textbf{85985144}}\\
\texttt{australian-credit} & \multicolumn{1}{r}{653} & \multicolumn{1}{r}{73}  & \cellcolor{TealBlue!30}{\textbf{43}} & \cellcolor{TealBlue!30}{\textbf{0}} & \cellcolor{TealBlue!30}{\textbf{7}} & \cellcolor{TealBlue!30}{\textbf{173}} & \cellcolor{TealBlue!30}{\textbf{123.00}} & \cellcolor{TealBlue!30}{\textbf{0}} & - & -\\
\texttt{breast-cancer-un} & \multicolumn{1}{r}{683} & \multicolumn{1}{r}{89}  & \cellcolor{TealBlue!30}{\textbf{8}} & \cellcolor{TealBlue!30}{\textbf{0}} & \cellcolor{TealBlue!30}{\textbf{7}} & \cellcolor{TealBlue!30}{\textbf{71}} & \cellcolor{TealBlue!30}{\textbf{1270.00}} & \cellcolor{TealBlue!30}{\textbf{0}} & - & -\\
\texttt{breast-wisconsin} & \multicolumn{1}{r}{683} & \multicolumn{1}{r}{120}  & \cellcolor{TealBlue!30}{\textbf{4}} & \cellcolor{TealBlue!30}{\textbf{0}} & \cellcolor{TealBlue!30}{\textbf{6}} & \cellcolor{TealBlue!30}{\textbf{41}} & \cellcolor{TealBlue!30}{\textbf{3280.00}} & \cellcolor{TealBlue!30}{\textbf{0}} & - & -\\
\texttt{car-un} & \multicolumn{1}{r}{1728} & \multicolumn{1}{r}{21}  & \cellcolor{TealBlue!30}{\textbf{50}} & \cellcolor{TealBlue!30}{\textbf{11}} & \cellcolor{TealBlue!30}{\textbf{7}} & \cellcolor{TealBlue!30}{\textbf{107}} & \cellcolor{TealBlue!30}{\textbf{151.00}} & \cellcolor{TealBlue!30}{\textbf{1}} & \cellcolor{TealBlue!30}{\textbf{275.00}} & \cellcolor{TealBlue!30}{\textbf{182560348}}\\
\texttt{diabetes} & \multicolumn{1}{r}{768} & \multicolumn{1}{r}{112}  & \cellcolor{TealBlue!30}{\textbf{99}} & \cellcolor{TealBlue!30}{\textbf{21}} & \cellcolor{TealBlue!30}{\textbf{7}} & \cellcolor{TealBlue!30}{\textbf{223}} & \cellcolor{TealBlue!30}{\textbf{1150.00}} & \cellcolor{TealBlue!30}{\textbf{0}} & - & -\\
\texttt{forest-fires-un} & \multicolumn{1}{r}{517} & \multicolumn{1}{r}{989}  & \cellcolor{TealBlue!30}{\textbf{162}} & \cellcolor{TealBlue!30}{\textbf{156}} & \cellcolor{TealBlue!30}{\textbf{7}} & \cellcolor{TealBlue!30}{\textbf{73}} & \cellcolor{TealBlue!30}{\textbf{3.15}} & \cellcolor{TealBlue!30}{\textbf{0}} & - & -\\
\texttt{german-credit} & \multicolumn{1}{r}{1000} & \multicolumn{1}{r}{110}  & \cellcolor{TealBlue!30}{\textbf{141}} & \cellcolor{TealBlue!30}{\textbf{56}} & \cellcolor{TealBlue!30}{\textbf{7}} & \cellcolor{TealBlue!30}{\textbf{215}} & \cellcolor{TealBlue!30}{\textbf{1680.00}} & \cellcolor{TealBlue!30}{\textbf{0}} & - & -\\
\texttt{heart-cleveland} & \multicolumn{1}{r}{296} & \multicolumn{1}{r}{50}  & \cellcolor{TealBlue!30}{\textbf{6}} & \cellcolor{TealBlue!30}{\textbf{0}} & \cellcolor{TealBlue!30}{\textbf{6}} & \cellcolor{TealBlue!30}{\textbf{75}} & \cellcolor{TealBlue!30}{\textbf{686.00}} & \cellcolor{TealBlue!30}{\textbf{0}} & - & -\\
\texttt{hepatitis} & \multicolumn{1}{r}{137} & \multicolumn{1}{r}{68}  & \cellcolor{TealBlue!30}{\textbf{0}} & \cellcolor{TealBlue!30}{\textbf{0}} & \cellcolor{TealBlue!30}{\textbf{5}} & \cellcolor{TealBlue!30}{\textbf{35}} & \cellcolor{TealBlue!30}{\textbf{0.24}} & \cellcolor{TealBlue!30}{\textbf{1}} & \cellcolor{TealBlue!30}{\textbf{10.50}} & \cellcolor{TealBlue!30}{\textbf{4823276}}\\
\texttt{hypothyroid} & \multicolumn{1}{r}{3247} & \multicolumn{1}{r}{43}  & \cellcolor{TealBlue!30}{\textbf{42}} & \cellcolor{TealBlue!30}{\textbf{23}} & \cellcolor{TealBlue!30}{\textbf{7}} & \cellcolor{TealBlue!30}{\textbf{111}} & \cellcolor{TealBlue!30}{\textbf{1840.00}} & \cellcolor{TealBlue!30}{\textbf{0}} & - & -\\
\texttt{ionosphere} & \multicolumn{1}{r}{351} & \multicolumn{1}{r}{444}  & \cellcolor{TealBlue!30}{\textbf{7}} & \cellcolor{TealBlue!30}{\textbf{0}} & \cellcolor{TealBlue!30}{\textbf{6}} & \cellcolor{TealBlue!30}{\textbf{55}} & \cellcolor{TealBlue!30}{\textbf{5.42}} & \cellcolor{TealBlue!30}{\textbf{0}} & - & -\\
\texttt{kr-vs-kp} & \multicolumn{1}{r}{3196} & \multicolumn{1}{r}{37}  & \cellcolor{TealBlue!30}{\textbf{102}} & \cellcolor{TealBlue!30}{\textbf{18}} & \cellcolor{TealBlue!30}{\textbf{7}} & \cellcolor{TealBlue!30}{\textbf{131}} & \cellcolor{TealBlue!30}{\textbf{2460.00}} & \cellcolor{TealBlue!30}{\textbf{0}} & - & -\\
\texttt{letter} & \multicolumn{1}{r}{20000} & \multicolumn{1}{r}{224}  & \cellcolor{TealBlue!30}{\textbf{143}} & \cellcolor{TealBlue!30}{\textbf{70}} & \cellcolor{TealBlue!30}{\textbf{7}} & \cellcolor{TealBlue!30}{\textbf{173}} & \cellcolor{TealBlue!30}{\textbf{434.00}} & \cellcolor{TealBlue!30}{\textbf{0}} & - & -\\
\texttt{lymph} & \multicolumn{1}{r}{148} & \multicolumn{1}{r}{41}  & \cellcolor{TealBlue!30}{\textbf{0}} & \cellcolor{TealBlue!30}{\textbf{0}} & \cellcolor{TealBlue!30}{\textbf{5}} & \cellcolor{TealBlue!30}{\textbf{35}} & \cellcolor{TealBlue!30}{\textbf{1.16}} & \cellcolor{TealBlue!30}{\textbf{1}} & \cellcolor{TealBlue!30}{\textbf{35.00}} & \cellcolor{TealBlue!30}{\textbf{12447337}}\\
\texttt{mushroom} & \multicolumn{1}{r}{8124} & \multicolumn{1}{r}{91}  & \cellcolor{TealBlue!30}{\textbf{0}} & \cellcolor{TealBlue!30}{\textbf{0}} & \cellcolor{TealBlue!30}{\textbf{4}} & \cellcolor{TealBlue!30}{\textbf{15}} & \cellcolor{TealBlue!30}{\textbf{0.29}} & \cellcolor{TealBlue!30}{\textbf{1}} & \cellcolor{TealBlue!30}{\textbf{57.60}} & \cellcolor{TealBlue!30}{\textbf{2017076}}\\
\texttt{pendigits} & \multicolumn{1}{r}{7494} & \multicolumn{1}{r}{216}  & \cellcolor{TealBlue!30}{\textbf{1}} & \cellcolor{TealBlue!30}{\textbf{0}} & \cellcolor{TealBlue!30}{\textbf{6}} & \cellcolor{TealBlue!30}{\textbf{41}} & \cellcolor{TealBlue!30}{\textbf{2200.00}} & \cellcolor{TealBlue!30}{\textbf{0}} & - & -\\
\texttt{primary-tumor} & \multicolumn{1}{r}{336} & \multicolumn{1}{r}{16}  & \cellcolor{TealBlue!30}{\textbf{26}} & \cellcolor{TealBlue!30}{\textbf{16}} & \cellcolor{TealBlue!30}{\textbf{7}} & \cellcolor{TealBlue!30}{\textbf{115}} & \cellcolor{TealBlue!30}{\textbf{1.15}} & \cellcolor{TealBlue!30}{\textbf{1}} & \cellcolor{TealBlue!30}{\textbf{88.40}} & \cellcolor{TealBlue!30}{\textbf{75229131}}\\
\texttt{segment} & \multicolumn{1}{r}{2310} & \multicolumn{1}{r}{234}  & \cellcolor{TealBlue!30}{\textbf{0}} & \cellcolor{TealBlue!30}{\textbf{0}} & \cellcolor{TealBlue!30}{\textbf{4}} & \cellcolor{TealBlue!30}{\textbf{11}} & \cellcolor{TealBlue!30}{\textbf{0.00}} & \cellcolor{TealBlue!30}{\textbf{1}} & \cellcolor{TealBlue!30}{\textbf{76.30}} & \cellcolor{TealBlue!30}{\textbf{3955322}}\\
\texttt{soybean} & \multicolumn{1}{r}{630} & \multicolumn{1}{r}{34}  & \cellcolor{TealBlue!30}{\textbf{11}} & \cellcolor{TealBlue!30}{\textbf{2}} & \cellcolor{TealBlue!30}{\textbf{7}} & \cellcolor{TealBlue!30}{\textbf{107}} & \cellcolor{TealBlue!30}{\textbf{482.00}} & \cellcolor{TealBlue!30}{\textbf{0}} & - & -\\
\texttt{splice-1} & \multicolumn{1}{r}{3190} & \multicolumn{1}{r}{227}  & \cellcolor{TealBlue!30}{\textbf{58}} & \cellcolor{TealBlue!30}{\textbf{32}} & \cellcolor{TealBlue!30}{\textbf{7}} & \cellcolor{TealBlue!30}{\textbf{153}} & \cellcolor{TealBlue!30}{\textbf{2430.00}} & \cellcolor{TealBlue!30}{\textbf{0}} & - & -\\
\texttt{taiwan\_binarised} & \multicolumn{1}{r}{30000} & \multicolumn{1}{r}{198}  & \cellcolor{TealBlue!30}{\textbf{5121}} & \cellcolor{TealBlue!30}{\textbf{5065}} & \cellcolor{TealBlue!30}{\textbf{7}} & \cellcolor{TealBlue!30}{\textbf{247}} & \cellcolor{TealBlue!30}{\textbf{85.70}} & \cellcolor{TealBlue!30}{\textbf{0}} & - & -\\
\texttt{tic-tac-toe} & \multicolumn{1}{r}{958} & \multicolumn{1}{r}{18}  & \cellcolor{TealBlue!30}{\textbf{21}} & \cellcolor{TealBlue!30}{\textbf{0}} & \cellcolor{TealBlue!30}{\textbf{7}} & \cellcolor{TealBlue!30}{\textbf{89}} & \cellcolor{TealBlue!30}{\textbf{527.00}} & \cellcolor{TealBlue!30}{\textbf{1}} & \cellcolor{TealBlue!30}{\textbf{1740.00}} & \cellcolor{TealBlue!30}{\textbf{1046729418}}\\
\texttt{vehicle} & \multicolumn{1}{r}{846} & \multicolumn{1}{r}{252}  & \cellcolor{TealBlue!30}{\textbf{4}} & \cellcolor{TealBlue!30}{\textbf{0}} & \cellcolor{TealBlue!30}{\textbf{6}} & \cellcolor{TealBlue!30}{\textbf{73}} & \cellcolor{TealBlue!30}{\textbf{0.19}} & \cellcolor{TealBlue!30}{\textbf{0}} & - & -\\
\texttt{vote} & \multicolumn{1}{r}{435} & \multicolumn{1}{r}{32}  & \cellcolor{TealBlue!30}{\textbf{2}} & \cellcolor{TealBlue!30}{\textbf{0}} & \cellcolor{TealBlue!30}{\textbf{6}} & \cellcolor{TealBlue!30}{\textbf{39}} & \cellcolor{TealBlue!30}{\textbf{10.30}} & \cellcolor{TealBlue!30}{\textbf{1}} & \cellcolor{TealBlue!30}{\textbf{1870.00}} & \cellcolor{TealBlue!30}{\textbf{632757986}}\\
\texttt{wine1-un} & \multicolumn{1}{r}{178} & \multicolumn{1}{r}{1276}  & \cellcolor{TealBlue!30}{\textbf{33}} & \cellcolor{TealBlue!30}{\textbf{28}} & \cellcolor{TealBlue!30}{\textbf{7}} & \cellcolor{TealBlue!30}{\textbf{27}} & \cellcolor{TealBlue!30}{\textbf{1090.00}} & \cellcolor{TealBlue!30}{\textbf{0}} & - & -\\
\texttt{wine2-un} & \multicolumn{1}{r}{178} & \multicolumn{1}{r}{1276}  & \cellcolor{TealBlue!30}{\textbf{38}} & \cellcolor{TealBlue!30}{\textbf{31}} & \cellcolor{TealBlue!30}{\textbf{7}} & \cellcolor{TealBlue!30}{\textbf{31}} & \cellcolor{TealBlue!30}{\textbf{34.90}} & \cellcolor{TealBlue!30}{\textbf{0}} & - & -\\
\texttt{wine3-un} & \multicolumn{1}{r}{178} & \multicolumn{1}{r}{1276}  & \cellcolor{TealBlue!30}{\textbf{26}} & \cellcolor{TealBlue!30}{\textbf{21}} & \cellcolor{TealBlue!30}{\textbf{7}} & \cellcolor{TealBlue!30}{\textbf{31}} & \cellcolor{TealBlue!30}{\textbf{676.00}} & \cellcolor{TealBlue!30}{\textbf{0}} & - & -\\
\texttt{yeast} & \multicolumn{1}{r}{1484} & \multicolumn{1}{r}{89}  & \cellcolor{TealBlue!30}{\textbf{305}} & \cellcolor{TealBlue!30}{\textbf{203}} & \cellcolor{TealBlue!30}{\textbf{7}} & \cellcolor{TealBlue!30}{\textbf{195}} & \cellcolor{TealBlue!30}{\textbf{563.00}} & \cellcolor{TealBlue!30}{\textbf{0}} & - & -\\
\texttt{zoo-1} & \multicolumn{1}{r}{101} & \multicolumn{1}{r}{20}  & \cellcolor{TealBlue!30}{\textbf{0}} & \cellcolor{TealBlue!30}{\textbf{0}} & \cellcolor{TealBlue!30}{\textbf{1}} & \cellcolor{TealBlue!30}{\textbf{3}} & \cellcolor{TealBlue!30}{\textbf{0.00}} & \cellcolor{TealBlue!30}{\textbf{1}} & \cellcolor{TealBlue!30}{\textbf{0.00}} & \cellcolor{TealBlue!30}{\textbf{1}}\\
\bottomrule
\end{tabular}

\end{normalsize}
\end{center}
\caption{\label{tab:s7} max depth=7}
\end{table}

\begin{table}[htbp]
\begin{center}
\begin{normalsize}
\tabcolsep=3pt
\begin{tabular}{lccrrrrrrrr}
\toprule
& && \multicolumn{8}{c}{\budalg}\\
\cmidrule(rr){4-11}
&\multirow{1}{*}{$\#ex.$} & \multirow{1}{*}{\#feat.} &  \multicolumn{1}{c}{error (f)} & \multicolumn{1}{c}{error (b)} & \multicolumn{1}{c}{depth (b)} & \multicolumn{1}{c}{size (b)} & \multicolumn{1}{c}{time (b)} & \multicolumn{1}{c}{opt} & \multicolumn{1}{c}{time (a)} & \multicolumn{1}{c}{search (a)} \\
\midrule

\texttt{anneal} & \multicolumn{1}{r}{812} & \multicolumn{1}{r}{47}  & \cellcolor{TealBlue!30}{\textbf{59}} & \cellcolor{TealBlue!30}{\textbf{53}} & \cellcolor{TealBlue!30}{\textbf{10}} & \cellcolor{TealBlue!30}{\textbf{131}} & \cellcolor{TealBlue!30}{\textbf{610.00}} & \cellcolor{TealBlue!30}{\textbf{0}} & - & -\\
\texttt{audiology} & \multicolumn{1}{r}{216} & \multicolumn{1}{r}{79}  & \cellcolor{TealBlue!30}{\textbf{0}} & \cellcolor{TealBlue!30}{\textbf{0}} & \cellcolor{TealBlue!30}{\textbf{5}} & \cellcolor{TealBlue!30}{\textbf{21}} & \cellcolor{TealBlue!30}{\textbf{0.54}} & \cellcolor{TealBlue!30}{\textbf{1}} & \cellcolor{TealBlue!30}{\textbf{379.00}} & \cellcolor{TealBlue!30}{\textbf{85985144}}\\
\texttt{australian-credit} & \multicolumn{1}{r}{653} & \multicolumn{1}{r}{73}  & \cellcolor{TealBlue!30}{\textbf{12}} & \cellcolor{TealBlue!30}{\textbf{0}} & \cellcolor{TealBlue!30}{\textbf{7}} & \cellcolor{TealBlue!30}{\textbf{173}} & \cellcolor{TealBlue!30}{\textbf{205.00}} & \cellcolor{TealBlue!30}{\textbf{0}} & - & -\\
\texttt{breast-cancer-un} & \multicolumn{1}{r}{683} & \multicolumn{1}{r}{89}  & \cellcolor{TealBlue!30}{\textbf{0}} & \cellcolor{TealBlue!30}{\textbf{0}} & \cellcolor{TealBlue!30}{\textbf{7}} & \cellcolor{TealBlue!30}{\textbf{71}} & \cellcolor{TealBlue!30}{\textbf{1450.00}} & \cellcolor{TealBlue!30}{\textbf{0}} & - & -\\
\texttt{breast-wisconsin} & \multicolumn{1}{r}{683} & \multicolumn{1}{r}{120}  & \cellcolor{TealBlue!30}{\textbf{0}} & \cellcolor{TealBlue!30}{\textbf{0}} & \cellcolor{TealBlue!30}{\textbf{6}} & \cellcolor{TealBlue!30}{\textbf{41}} & \cellcolor{TealBlue!30}{\textbf{3170.00}} & \cellcolor{TealBlue!30}{\textbf{0}} & - & -\\
\texttt{car-un} & \multicolumn{1}{r}{1728} & \multicolumn{1}{r}{21}  & \cellcolor{TealBlue!30}{\textbf{11}} & \cellcolor{TealBlue!30}{\textbf{0}} & \cellcolor{TealBlue!30}{\textbf{9}} & \cellcolor{TealBlue!30}{\textbf{91}} & \cellcolor{TealBlue!30}{\textbf{1910.00}} & \cellcolor{TealBlue!30}{\textbf{0}} & - & -\\
\texttt{diabetes} & \multicolumn{1}{r}{768} & \multicolumn{1}{r}{112}  & \cellcolor{TealBlue!30}{\textbf{38}} & \cellcolor{TealBlue!30}{\textbf{0}} & \cellcolor{TealBlue!30}{\textbf{8}} & \cellcolor{TealBlue!30}{\textbf{323}} & \cellcolor{TealBlue!30}{\textbf{547.00}} & \cellcolor{TealBlue!30}{\textbf{0}} & - & -\\
\texttt{forest-fires-un} & \multicolumn{1}{r}{517} & \multicolumn{1}{r}{989}  & \cellcolor{TealBlue!30}{\textbf{145}} & \cellcolor{TealBlue!30}{\textbf{113}} & \cellcolor{TealBlue!30}{\textbf{10}} & \cellcolor{TealBlue!30}{\textbf{99}} & \cellcolor{TealBlue!30}{\textbf{1390.00}} & \cellcolor{TealBlue!30}{\textbf{0}} & - & -\\
\texttt{german-credit} & \multicolumn{1}{r}{1000} & \multicolumn{1}{r}{110}  & \cellcolor{TealBlue!30}{\textbf{66}} & \cellcolor{TealBlue!30}{\textbf{0}} & \cellcolor{TealBlue!30}{\textbf{10}} & \cellcolor{TealBlue!30}{\textbf{387}} & \cellcolor{TealBlue!30}{\textbf{87.70}} & \cellcolor{TealBlue!30}{\textbf{0}} & - & -\\
\texttt{heart-cleveland} & \multicolumn{1}{r}{296} & \multicolumn{1}{r}{50}  & \cellcolor{TealBlue!30}{\textbf{0}} & \cellcolor{TealBlue!30}{\textbf{0}} & \cellcolor{TealBlue!30}{\textbf{6}} & \cellcolor{TealBlue!30}{\textbf{75}} & \cellcolor{TealBlue!30}{\textbf{671.00}} & \cellcolor{TealBlue!30}{\textbf{0}} & - & -\\
\texttt{hepatitis} & \multicolumn{1}{r}{137} & \multicolumn{1}{r}{68}  & \cellcolor{TealBlue!30}{\textbf{0}} & \cellcolor{TealBlue!30}{\textbf{0}} & \cellcolor{TealBlue!30}{\textbf{5}} & \cellcolor{TealBlue!30}{\textbf{35}} & \cellcolor{TealBlue!30}{\textbf{0.24}} & \cellcolor{TealBlue!30}{\textbf{1}} & \cellcolor{TealBlue!30}{\textbf{10.60}} & \cellcolor{TealBlue!30}{\textbf{4823276}}\\
\texttt{hypothyroid} & \multicolumn{1}{r}{3247} & \multicolumn{1}{r}{43}  & \cellcolor{TealBlue!30}{\textbf{31}} & \cellcolor{TealBlue!30}{\textbf{31}} & \cellcolor{TealBlue!30}{\textbf{10}} & \cellcolor{TealBlue!30}{\textbf{131}} & \cellcolor{TealBlue!30}{\textbf{0.00}} & \cellcolor{TealBlue!30}{\textbf{0}} & - & -\\
\texttt{ionosphere} & \multicolumn{1}{r}{351} & \multicolumn{1}{r}{444}  & \cellcolor{TealBlue!30}{\textbf{0}} & \cellcolor{TealBlue!30}{\textbf{0}} & \cellcolor{TealBlue!30}{\textbf{6}} & \cellcolor{TealBlue!30}{\textbf{55}} & \cellcolor{TealBlue!30}{\textbf{5.47}} & \cellcolor{TealBlue!30}{\textbf{0}} & - & -\\
\texttt{kr-vs-kp} & \multicolumn{1}{r}{3196} & \multicolumn{1}{r}{37}  & \cellcolor{TealBlue!30}{\textbf{12}} & \cellcolor{TealBlue!30}{\textbf{0}} & \cellcolor{TealBlue!30}{\textbf{10}} & \cellcolor{TealBlue!30}{\textbf{151}} & \cellcolor{TealBlue!30}{\textbf{1900.00}} & \cellcolor{TealBlue!30}{\textbf{0}} & - & -\\
\texttt{letter} & \multicolumn{1}{r}{20000} & \multicolumn{1}{r}{224}  & \cellcolor{TealBlue!30}{\textbf{20}} & \cellcolor{TealBlue!30}{\textbf{0}} & \cellcolor{TealBlue!30}{\textbf{10}} & \cellcolor{TealBlue!30}{\textbf{347}} & \cellcolor{TealBlue!30}{\textbf{84.00}} & \cellcolor{TealBlue!30}{\textbf{0}} & - & -\\
\texttt{lymph} & \multicolumn{1}{r}{148} & \multicolumn{1}{r}{41}  & \cellcolor{TealBlue!30}{\textbf{0}} & \cellcolor{TealBlue!30}{\textbf{0}} & \cellcolor{TealBlue!30}{\textbf{5}} & \cellcolor{TealBlue!30}{\textbf{35}} & \cellcolor{TealBlue!30}{\textbf{1.16}} & \cellcolor{TealBlue!30}{\textbf{1}} & \cellcolor{TealBlue!30}{\textbf{34.60}} & \cellcolor{TealBlue!30}{\textbf{12447337}}\\
\texttt{mushroom} & \multicolumn{1}{r}{8124} & \multicolumn{1}{r}{91}  & \cellcolor{TealBlue!30}{\textbf{0}} & \cellcolor{TealBlue!30}{\textbf{0}} & \cellcolor{TealBlue!30}{\textbf{4}} & \cellcolor{TealBlue!30}{\textbf{15}} & \cellcolor{TealBlue!30}{\textbf{0.27}} & \cellcolor{TealBlue!30}{\textbf{1}} & \cellcolor{TealBlue!30}{\textbf{58.30}} & \cellcolor{TealBlue!30}{\textbf{2017076}}\\
\texttt{pendigits} & \multicolumn{1}{r}{7494} & \multicolumn{1}{r}{216}  & \cellcolor{TealBlue!30}{\textbf{0}} & \cellcolor{TealBlue!30}{\textbf{0}} & \cellcolor{TealBlue!30}{\textbf{6}} & \cellcolor{TealBlue!30}{\textbf{41}} & \cellcolor{TealBlue!30}{\textbf{2310.00}} & \cellcolor{TealBlue!30}{\textbf{0}} & - & -\\
\texttt{primary-tumor} & \multicolumn{1}{r}{336} & \multicolumn{1}{r}{16}  & \cellcolor{TealBlue!30}{\textbf{20}} & \cellcolor{TealBlue!30}{\textbf{15}} & \cellcolor{TealBlue!30}{\textbf{10}} & \cellcolor{TealBlue!30}{\textbf{153}} & \cellcolor{TealBlue!30}{\textbf{3.36}} & \cellcolor{TealBlue!30}{\textbf{0}} & - & -\\
\texttt{segment} & \multicolumn{1}{r}{2310} & \multicolumn{1}{r}{234}  & \cellcolor{TealBlue!30}{\textbf{0}} & \cellcolor{TealBlue!30}{\textbf{0}} & \cellcolor{TealBlue!30}{\textbf{4}} & \cellcolor{TealBlue!30}{\textbf{11}} & \cellcolor{TealBlue!30}{\textbf{0.00}} & \cellcolor{TealBlue!30}{\textbf{1}} & \cellcolor{TealBlue!30}{\textbf{72.80}} & \cellcolor{TealBlue!30}{\textbf{3955322}}\\
\texttt{soybean} & \multicolumn{1}{r}{630} & \multicolumn{1}{r}{34}  & \cellcolor{TealBlue!30}{\textbf{2}} & \cellcolor{TealBlue!30}{\textbf{2}} & \cellcolor{TealBlue!30}{\textbf{10}} & \cellcolor{TealBlue!30}{\textbf{87}} & \cellcolor{TealBlue!30}{\textbf{0.00}} & \cellcolor{TealBlue!30}{\textbf{0}} & - & -\\
\texttt{splice-1} & \multicolumn{1}{r}{3190} & \multicolumn{1}{r}{227}  & \cellcolor{TealBlue!30}{\textbf{12}} & \cellcolor{TealBlue!30}{\textbf{5}} & \cellcolor{TealBlue!30}{\textbf{10}} & \cellcolor{TealBlue!30}{\textbf{195}} & \cellcolor{TealBlue!30}{\textbf{1540.00}} & \cellcolor{TealBlue!30}{\textbf{0}} & - & -\\
\texttt{taiwan\_binarised} & \multicolumn{1}{r}{30000} & \multicolumn{1}{r}{198}  & \cellcolor{TealBlue!30}{\textbf{4666}} & \cellcolor{TealBlue!30}{\textbf{4564}} & \cellcolor{TealBlue!30}{\textbf{10}} & \cellcolor{TealBlue!30}{\textbf{999}} & \cellcolor{TealBlue!30}{\textbf{418.00}} & \cellcolor{TealBlue!30}{\textbf{0}} & - & -\\
\texttt{tic-tac-toe} & \multicolumn{1}{r}{958} & \multicolumn{1}{r}{18}  & \cellcolor{TealBlue!30}{\textbf{6}} & \cellcolor{TealBlue!30}{\textbf{0}} & \cellcolor{TealBlue!30}{\textbf{8}} & \cellcolor{TealBlue!30}{\textbf{85}} & \cellcolor{TealBlue!30}{\textbf{799.00}} & \cellcolor{TealBlue!30}{\textbf{0}} & - & -\\
\texttt{vehicle} & \multicolumn{1}{r}{846} & \multicolumn{1}{r}{252}  & \cellcolor{TealBlue!30}{\textbf{0}} & \cellcolor{TealBlue!30}{\textbf{0}} & \cellcolor{TealBlue!30}{\textbf{6}} & \cellcolor{TealBlue!30}{\textbf{73}} & \cellcolor{TealBlue!30}{\textbf{0.22}} & \cellcolor{TealBlue!30}{\textbf{0}} & - & -\\
\texttt{vote} & \multicolumn{1}{r}{435} & \multicolumn{1}{r}{32}  & \cellcolor{TealBlue!30}{\textbf{0}} & \cellcolor{TealBlue!30}{\textbf{0}} & \cellcolor{TealBlue!30}{\textbf{6}} & \cellcolor{TealBlue!30}{\textbf{39}} & \cellcolor{TealBlue!30}{\textbf{9.82}} & \cellcolor{TealBlue!30}{\textbf{1}} & \cellcolor{TealBlue!30}{\textbf{1800.00}} & \cellcolor{TealBlue!30}{\textbf{632823064}}\\
\texttt{wine1-un} & \multicolumn{1}{r}{178} & \multicolumn{1}{r}{1276}  & \cellcolor{TealBlue!30}{\textbf{25}} & \cellcolor{TealBlue!30}{\textbf{22}} & \cellcolor{TealBlue!30}{\textbf{10}} & \cellcolor{TealBlue!30}{\textbf{33}} & \cellcolor{TealBlue!30}{\textbf{682.00}} & \cellcolor{TealBlue!30}{\textbf{0}} & - & -\\
\texttt{wine2-un} & \multicolumn{1}{r}{178} & \multicolumn{1}{r}{1276}  & \cellcolor{TealBlue!30}{\textbf{29}} & \cellcolor{TealBlue!30}{\textbf{24}} & \cellcolor{TealBlue!30}{\textbf{10}} & \cellcolor{TealBlue!30}{\textbf{37}} & \cellcolor{TealBlue!30}{\textbf{526.00}} & \cellcolor{TealBlue!30}{\textbf{0}} & - & -\\
\texttt{wine3-un} & \multicolumn{1}{r}{178} & \multicolumn{1}{r}{1276}  & \cellcolor{TealBlue!30}{\textbf{19}} & \cellcolor{TealBlue!30}{\textbf{16}} & \cellcolor{TealBlue!30}{\textbf{10}} & \cellcolor{TealBlue!30}{\textbf{35}} & \cellcolor{TealBlue!30}{\textbf{342.00}} & \cellcolor{TealBlue!30}{\textbf{0}} & - & -\\
\texttt{yeast} & \multicolumn{1}{r}{1484} & \multicolumn{1}{r}{89}  & \cellcolor{TealBlue!30}{\textbf{180}} & \cellcolor{TealBlue!30}{\textbf{104}} & \cellcolor{TealBlue!30}{\textbf{10}} & \cellcolor{TealBlue!30}{\textbf{497}} & \cellcolor{TealBlue!30}{\textbf{110.00}} & \cellcolor{TealBlue!30}{\textbf{0}} & - & -\\
\texttt{zoo-1} & \multicolumn{1}{r}{101} & \multicolumn{1}{r}{20}  & \cellcolor{TealBlue!30}{\textbf{0}} & \cellcolor{TealBlue!30}{\textbf{0}} & \cellcolor{TealBlue!30}{\textbf{1}} & \cellcolor{TealBlue!30}{\textbf{3}} & \cellcolor{TealBlue!30}{\textbf{0.00}} & \cellcolor{TealBlue!30}{\textbf{1}} & \cellcolor{TealBlue!30}{\textbf{0.00}} & \cellcolor{TealBlue!30}{\textbf{1}}\\
\bottomrule
\end{tabular}

\end{normalsize}
\end{center}
\caption{\label{tab:s10} max depth=10}
\end{table}

%
%
% \begin{table}[htbp]
% \begin{center}
% \begin{normalsize}
% \tabcolsep=5pt
% \begin{tabular}{lccrrrrrrrrr}
\toprule
& && \multicolumn{3}{c}{entropy} & \multicolumn{3}{c}{\budalg} & \multicolumn{3}{c}{error}\\
\cmidrule(rr){4-6}\cmidrule(rr){7-9}\cmidrule(rr){10-12}
&\multirow{1}{*}{$\#ex.$} & \multirow{1}{*}{\#feat.} &  \multicolumn{1}{c}{opt} & \multicolumn{1}{c}{error} & \multicolumn{1}{c}{time} & \multicolumn{1}{c}{opt} & \multicolumn{1}{c}{error} & \multicolumn{1}{c}{time} & \multicolumn{1}{c}{opt} & \multicolumn{1}{c}{error} & \multicolumn{1}{c}{time} \\
\midrule

\texttt{anneal} & \multicolumn{1}{r}{812} & \multicolumn{1}{r}{47}  & \cellcolor{TealBlue!30}{1} & \cellcolor{TealBlue!30}{112} & 0.3 & \cellcolor{TealBlue!30}{1} & \cellcolor{TealBlue!30}{112} & \cellcolor{TealBlue!30}{\textbf{0.2}} & \cellcolor{TealBlue!30}{1} & \cellcolor{TealBlue!30}{112} & 0.2\\
\texttt{audiology} & \multicolumn{1}{r}{216} & \multicolumn{1}{r}{79}  & \cellcolor{TealBlue!30}{1} & \cellcolor{TealBlue!30}{5} & 0.6 & \cellcolor{TealBlue!30}{1} & \cellcolor{TealBlue!30}{5} & 0.3 & \cellcolor{TealBlue!30}{1} & \cellcolor{TealBlue!30}{5} & \cellcolor{TealBlue!30}{\textbf{0.2}}\\
\texttt{australian-credit} & \multicolumn{1}{r}{653} & \multicolumn{1}{r}{73}  & \cellcolor{TealBlue!30}{1} & \cellcolor{TealBlue!30}{73} & 1.1 & \cellcolor{TealBlue!30}{1} & \cellcolor{TealBlue!30}{73} & 0.6 & \cellcolor{TealBlue!30}{1} & \cellcolor{TealBlue!30}{73} & \cellcolor{TealBlue!30}{\textbf{0.6}}\\
\texttt{breast-cancer-un} & \multicolumn{1}{r}{683} & \multicolumn{1}{r}{89}  & \cellcolor{TealBlue!30}{1} & \cellcolor{TealBlue!30}{24} & 0.3 & \cellcolor{TealBlue!30}{1} & \cellcolor{TealBlue!30}{24} & 0.1 & \cellcolor{TealBlue!30}{1} & \cellcolor{TealBlue!30}{24} & \cellcolor{TealBlue!30}{\textbf{0.1}}\\
\texttt{breast-wisconsin} & \multicolumn{1}{r}{683} & \multicolumn{1}{r}{120}  & \cellcolor{TealBlue!30}{1} & \cellcolor{TealBlue!30}{15} & 0.8 & \cellcolor{TealBlue!30}{1} & \cellcolor{TealBlue!30}{15} & 0.4 & \cellcolor{TealBlue!30}{1} & \cellcolor{TealBlue!30}{15} & \cellcolor{TealBlue!30}{\textbf{0.4}}\\
\texttt{car-un} & \multicolumn{1}{r}{1728} & \multicolumn{1}{r}{21}  & \cellcolor{TealBlue!30}{1} & \cellcolor{TealBlue!30}{192} & 0.0 & \cellcolor{TealBlue!30}{1} & \cellcolor{TealBlue!30}{192} & \cellcolor{TealBlue!30}{\textbf{0.0}} & \cellcolor{TealBlue!30}{1} & \cellcolor{TealBlue!30}{192} & 0.0\\
\texttt{diabetes} & \multicolumn{1}{r}{768} & \multicolumn{1}{r}{112}  & \cellcolor{TealBlue!30}{1} & \cellcolor{TealBlue!30}{162} & 1.0 & \cellcolor{TealBlue!30}{1} & \cellcolor{TealBlue!30}{162} & 0.5 & \cellcolor{TealBlue!30}{1} & \cellcolor{TealBlue!30}{162} & \cellcolor{TealBlue!30}{\textbf{0.5}}\\
\texttt{forest-fires-un} & \multicolumn{1}{r}{517} & \multicolumn{1}{r}{989}  & \cellcolor{TealBlue!30}{1} & \cellcolor{TealBlue!30}{193} & 149.0 & \cellcolor{TealBlue!30}{1} & \cellcolor{TealBlue!30}{193} & 69.2 & \cellcolor{TealBlue!30}{1} & \cellcolor{TealBlue!30}{193} & \cellcolor{TealBlue!30}{\textbf{59.9}}\\
\texttt{german-credit} & \multicolumn{1}{r}{1000} & \multicolumn{1}{r}{110}  & \cellcolor{TealBlue!30}{1} & \cellcolor{TealBlue!30}{236} & 1.2 & \cellcolor{TealBlue!30}{1} & \cellcolor{TealBlue!30}{236} & 0.6 & \cellcolor{TealBlue!30}{1} & \cellcolor{TealBlue!30}{236} & \cellcolor{TealBlue!30}{\textbf{0.5}}\\
\texttt{heart-cleveland} & \multicolumn{1}{r}{296} & \multicolumn{1}{r}{50}  & \cellcolor{TealBlue!30}{1} & \cellcolor{TealBlue!30}{41} & 0.5 & \cellcolor{TealBlue!30}{1} & \cellcolor{TealBlue!30}{41} & 0.2 & \cellcolor{TealBlue!30}{1} & \cellcolor{TealBlue!30}{41} & \cellcolor{TealBlue!30}{\textbf{0.2}}\\
\texttt{hepatitis} & \multicolumn{1}{r}{137} & \multicolumn{1}{r}{68}  & \cellcolor{TealBlue!30}{1} & \cellcolor{TealBlue!30}{10} & 0.1 & \cellcolor{TealBlue!30}{1} & \cellcolor{TealBlue!30}{10} & 0.1 & \cellcolor{TealBlue!30}{1} & \cellcolor{TealBlue!30}{10} & \cellcolor{TealBlue!30}{\textbf{0.1}}\\
\texttt{hypothyroid} & \multicolumn{1}{r}{3247} & \multicolumn{1}{r}{43}  & \cellcolor{TealBlue!30}{1} & \cellcolor{TealBlue!30}{61} & 0.9 & \cellcolor{TealBlue!30}{1} & \cellcolor{TealBlue!30}{61} & \cellcolor{TealBlue!30}{\textbf{0.6}} & \cellcolor{TealBlue!30}{1} & \cellcolor{TealBlue!30}{61} & 0.7\\
\texttt{ionosphere} & \multicolumn{1}{r}{351} & \multicolumn{1}{r}{444}  & \cellcolor{TealBlue!30}{1} & \cellcolor{TealBlue!30}{22} & 53.3 & \cellcolor{TealBlue!30}{1} & \cellcolor{TealBlue!30}{22} & 24.6 & \cellcolor{TealBlue!30}{1} & \cellcolor{TealBlue!30}{22} & \cellcolor{TealBlue!30}{\textbf{22.5}}\\
\texttt{kr-vs-kp} & \multicolumn{1}{r}{3196} & \multicolumn{1}{r}{37}  & \cellcolor{TealBlue!30}{1} & \cellcolor{TealBlue!30}{198} & 0.5 & \cellcolor{TealBlue!30}{1} & \cellcolor{TealBlue!30}{198} & \cellcolor{TealBlue!30}{\textbf{0.4}} & \cellcolor{TealBlue!30}{1} & \cellcolor{TealBlue!30}{198} & 0.4\\
\texttt{letter} & \multicolumn{1}{r}{20000} & \multicolumn{1}{r}{224}  & \cellcolor{TealBlue!30}{1} & \cellcolor{TealBlue!30}{369} & 137.0 & \cellcolor{TealBlue!30}{1} & \cellcolor{TealBlue!30}{369} & \cellcolor{TealBlue!30}{\textbf{57.1}} & \cellcolor{TealBlue!30}{1} & \cellcolor{TealBlue!30}{369} & 138.0\\
\texttt{lymph} & \multicolumn{1}{r}{148} & \multicolumn{1}{r}{41}  & \cellcolor{TealBlue!30}{1} & \cellcolor{TealBlue!30}{12} & 0.1 & \cellcolor{TealBlue!30}{1} & \cellcolor{TealBlue!30}{12} & 0.1 & \cellcolor{TealBlue!30}{1} & \cellcolor{TealBlue!30}{12} & \cellcolor{TealBlue!30}{\textbf{0.1}}\\
\texttt{mushroom} & \multicolumn{1}{r}{8124} & \multicolumn{1}{r}{91}  & \cellcolor{TealBlue!30}{1} & \cellcolor{TealBlue!30}{8} & 2.8 & \cellcolor{TealBlue!30}{1} & \cellcolor{TealBlue!30}{8} & \cellcolor{TealBlue!30}{\textbf{1.7}} & \cellcolor{TealBlue!30}{1} & \cellcolor{TealBlue!30}{8} & 2.4\\
\texttt{pendigits} & \multicolumn{1}{r}{7494} & \multicolumn{1}{r}{216}  & \cellcolor{TealBlue!30}{1} & \cellcolor{TealBlue!30}{47} & 32.6 & \cellcolor{TealBlue!30}{1} & \cellcolor{TealBlue!30}{47} & \cellcolor{TealBlue!30}{\textbf{18.7}} & \cellcolor{TealBlue!30}{1} & \cellcolor{TealBlue!30}{47} & 35.5\\
\texttt{primary-tumor} & \multicolumn{1}{r}{336} & \multicolumn{1}{r}{16}  & \cellcolor{TealBlue!30}{1} & \cellcolor{TealBlue!30}{46} & 0.0 & \cellcolor{TealBlue!30}{1} & \cellcolor{TealBlue!30}{46} & \cellcolor{TealBlue!30}{\textbf{0.0}} & \cellcolor{TealBlue!30}{1} & \cellcolor{TealBlue!30}{46} & 0.0\\
\texttt{segment} & \multicolumn{1}{r}{2310} & \multicolumn{1}{r}{234}  & \cellcolor{TealBlue!30}{1} & \cellcolor{TealBlue!30}{0} & 0.4 & \cellcolor{TealBlue!30}{1} & \cellcolor{TealBlue!30}{0} & \cellcolor{TealBlue!30}{\textbf{0.3}} & \cellcolor{TealBlue!30}{1} & \cellcolor{TealBlue!30}{0} & 0.3\\
\texttt{soybean} & \multicolumn{1}{r}{630} & \multicolumn{1}{r}{34}  & \cellcolor{TealBlue!30}{1} & \cellcolor{TealBlue!30}{29} & 0.1 & \cellcolor{TealBlue!30}{1} & \cellcolor{TealBlue!30}{29} & 0.0 & \cellcolor{TealBlue!30}{1} & \cellcolor{TealBlue!30}{29} & \cellcolor{TealBlue!30}{\textbf{0.0}}\\
\texttt{splice-1} & \multicolumn{1}{r}{3190} & \multicolumn{1}{r}{227}  & \cellcolor{TealBlue!30}{1} & \cellcolor{TealBlue!30}{224} & 23.6 & \cellcolor{TealBlue!30}{1} & \cellcolor{TealBlue!30}{224} & 14.4 & \cellcolor{TealBlue!30}{1} & \cellcolor{TealBlue!30}{224} & \cellcolor{TealBlue!30}{\textbf{14.2}}\\
\texttt{taiwan\_binarised} & \multicolumn{1}{r}{30000} & \multicolumn{1}{r}{198}  & \cellcolor{TealBlue!30}{1} & \cellcolor{TealBlue!30}{5326} & 138.0 & \cellcolor{TealBlue!30}{1} & \cellcolor{TealBlue!30}{5326} & \cellcolor{TealBlue!30}{\textbf{45.8}} & \cellcolor{TealBlue!30}{1} & \cellcolor{TealBlue!30}{5326} & 146.0\\
\texttt{tic-tac-toe} & \multicolumn{1}{r}{958} & \multicolumn{1}{r}{18}  & \cellcolor{TealBlue!30}{1} & \cellcolor{TealBlue!30}{216} & 0.0 & \cellcolor{TealBlue!30}{1} & \cellcolor{TealBlue!30}{216} & \cellcolor{TealBlue!30}{\textbf{0.0}} & \cellcolor{TealBlue!30}{1} & \cellcolor{TealBlue!30}{216} & 0.0\\
\texttt{vehicle} & \multicolumn{1}{r}{846} & \multicolumn{1}{r}{252}  & \cellcolor{TealBlue!30}{1} & \cellcolor{TealBlue!30}{26} & 8.2 & \cellcolor{TealBlue!30}{1} & \cellcolor{TealBlue!30}{26} & \cellcolor{TealBlue!30}{\textbf{4.4}} & \cellcolor{TealBlue!30}{1} & \cellcolor{TealBlue!30}{26} & 4.4\\
\texttt{vote} & \multicolumn{1}{r}{435} & \multicolumn{1}{r}{32}  & \cellcolor{TealBlue!30}{1} & \cellcolor{TealBlue!30}{12} & 0.1 & \cellcolor{TealBlue!30}{1} & \cellcolor{TealBlue!30}{12} & 0.0 & \cellcolor{TealBlue!30}{1} & \cellcolor{TealBlue!30}{12} & \cellcolor{TealBlue!30}{\textbf{0.0}}\\
\texttt{wine1-un} & \multicolumn{1}{r}{178} & \multicolumn{1}{r}{1276}  & \cellcolor{TealBlue!30}{1} & \cellcolor{TealBlue!30}{43} & 256.0 & \cellcolor{TealBlue!30}{1} & \cellcolor{TealBlue!30}{43} & 123.0 & \cellcolor{TealBlue!30}{1} & \cellcolor{TealBlue!30}{43} & \cellcolor{TealBlue!30}{\textbf{115.0}}\\
\texttt{wine2-un} & \multicolumn{1}{r}{178} & \multicolumn{1}{r}{1276}  & \cellcolor{TealBlue!30}{1} & \cellcolor{TealBlue!30}{49} & 256.0 & \cellcolor{TealBlue!30}{1} & \cellcolor{TealBlue!30}{49} & 122.0 & \cellcolor{TealBlue!30}{1} & \cellcolor{TealBlue!30}{49} & \cellcolor{TealBlue!30}{\textbf{110.0}}\\
\texttt{wine3-un} & \multicolumn{1}{r}{178} & \multicolumn{1}{r}{1276}  & \cellcolor{TealBlue!30}{1} & \cellcolor{TealBlue!30}{33} & 255.0 & \cellcolor{TealBlue!30}{1} & \cellcolor{TealBlue!30}{33} & 122.0 & \cellcolor{TealBlue!30}{1} & \cellcolor{TealBlue!30}{33} & \cellcolor{TealBlue!30}{\textbf{104.0}}\\
\texttt{yeast} & \multicolumn{1}{r}{1484} & \multicolumn{1}{r}{89}  & \cellcolor{TealBlue!30}{1} & \cellcolor{TealBlue!30}{403} & 0.7 & \cellcolor{TealBlue!30}{1} & \cellcolor{TealBlue!30}{403} & \cellcolor{TealBlue!30}{\textbf{0.4}} & \cellcolor{TealBlue!30}{1} & \cellcolor{TealBlue!30}{403} & 0.5\\
\texttt{zoo-1} & \multicolumn{1}{r}{101} & \multicolumn{1}{r}{20}  & \cellcolor{TealBlue!30}{1} & \cellcolor{TealBlue!30}{0} & 0.0 & \cellcolor{TealBlue!30}{1} & \cellcolor{TealBlue!30}{0} & \cellcolor{TealBlue!30}{\textbf{0.0}} & \cellcolor{TealBlue!30}{1} & \cellcolor{TealBlue!30}{0} & 0.0\\
\bottomrule
\end{tabular}

% \end{normalsize}
% \end{center}
% \caption{\label{tab:ha3} Comparison of heuristics (max depth=3)}
% \end{table}
%
% \begin{table}[htbp]
% \begin{center}
% \begin{normalsize}
% \tabcolsep=5pt
% \begin{tabular}{lccrrrrrrrrr}
\toprule
& && \multicolumn{3}{c}{entropy} & \multicolumn{3}{c}{\budalg} & \multicolumn{3}{c}{error}\\
\cmidrule(rr){4-6}\cmidrule(rr){7-9}\cmidrule(rr){10-12}
&\multirow{1}{*}{$\#ex.$} & \multirow{1}{*}{\#feat.} &  \multicolumn{1}{c}{opt} & \multicolumn{1}{c}{error} & \multicolumn{1}{c}{time} & \multicolumn{1}{c}{opt} & \multicolumn{1}{c}{error} & \multicolumn{1}{c}{time} & \multicolumn{1}{c}{opt} & \multicolumn{1}{c}{error} & \multicolumn{1}{c}{time} \\
\midrule

\texttt{anneal} & \multicolumn{1}{r}{812} & \multicolumn{1}{r}{47}  & \cellcolor{TealBlue!30}{1} & \cellcolor{TealBlue!30}{91} & 23.2 & \cellcolor{TealBlue!30}{1} & \cellcolor{TealBlue!30}{91} & \cellcolor{TealBlue!30}{14.1} & \cellcolor{TealBlue!30}{1} & \cellcolor{TealBlue!30}{91} & \cellcolor{TealBlue!30}{14.1}\\
\texttt{audiology} & \multicolumn{1}{r}{216} & \multicolumn{1}{r}{79}  & \cellcolor{TealBlue!30}{1} & \cellcolor{TealBlue!30}{1} & 63.6 & \cellcolor{TealBlue!30}{1} & \cellcolor{TealBlue!30}{1} & 31.2 & \cellcolor{TealBlue!30}{1} & \cellcolor{TealBlue!30}{1} & \cellcolor{TealBlue!30}{\textbf{26.4}}\\
\texttt{australian-credit} & \multicolumn{1}{r}{653} & \multicolumn{1}{r}{73}  & \cellcolor{TealBlue!30}{1} & \cellcolor{TealBlue!30}{56} & 166.0 & \cellcolor{TealBlue!30}{1} & \cellcolor{TealBlue!30}{56} & 83.4 & \cellcolor{TealBlue!30}{1} & \cellcolor{TealBlue!30}{56} & \cellcolor{TealBlue!30}{\textbf{73.8}}\\
\texttt{breast-cancer-un} & \multicolumn{1}{r}{683} & \multicolumn{1}{r}{89}  & \cellcolor{TealBlue!30}{1} & \cellcolor{TealBlue!30}{16} & 21.6 & \cellcolor{TealBlue!30}{1} & \cellcolor{TealBlue!30}{16} & 12.3 & \cellcolor{TealBlue!30}{1} & \cellcolor{TealBlue!30}{16} & \cellcolor{TealBlue!30}{\textbf{11.2}}\\
\texttt{breast-wisconsin} & \multicolumn{1}{r}{683} & \multicolumn{1}{r}{120}  & \cellcolor{TealBlue!30}{1} & \cellcolor{TealBlue!30}{7} & 80.9 & \cellcolor{TealBlue!30}{1} & \cellcolor{TealBlue!30}{7} & 42.5 & \cellcolor{TealBlue!30}{1} & \cellcolor{TealBlue!30}{7} & \cellcolor{TealBlue!30}{\textbf{37.8}}\\
\texttt{car-un} & \multicolumn{1}{r}{1728} & \multicolumn{1}{r}{21}  & \cellcolor{TealBlue!30}{1} & \cellcolor{TealBlue!30}{136} & 0.4 & \cellcolor{TealBlue!30}{1} & \cellcolor{TealBlue!30}{136} & \cellcolor{TealBlue!30}{\textbf{0.3}} & \cellcolor{TealBlue!30}{1} & \cellcolor{TealBlue!30}{136} & 0.3\\
\texttt{diabetes} & \multicolumn{1}{r}{768} & \multicolumn{1}{r}{112}  & \cellcolor{TealBlue!30}{1} & \cellcolor{TealBlue!30}{137} & 142.0 & \cellcolor{TealBlue!30}{1} & \cellcolor{TealBlue!30}{137} & 71.1 & \cellcolor{TealBlue!30}{1} & \cellcolor{TealBlue!30}{137} & \cellcolor{TealBlue!30}{\textbf{65.8}}\\
\texttt{forest-fires-un} & \multicolumn{1}{r}{517} & \multicolumn{1}{r}{989}  & \cellcolor{TealBlue!30}{0} & \cellcolor{TealBlue!30}{173} & 244.0 & \cellcolor{TealBlue!30}{0} & \cellcolor{TealBlue!30}{173} & 50.9 & \cellcolor{TealBlue!30}{0} & \cellcolor{TealBlue!30}{173} & \cellcolor{TealBlue!30}{\textbf{39.0}}\\
\texttt{german-credit} & \multicolumn{1}{r}{1000} & \multicolumn{1}{r}{110}  & \cellcolor{TealBlue!30}{1} & \cellcolor{TealBlue!30}{204} & 163.0 & \cellcolor{TealBlue!30}{1} & \cellcolor{TealBlue!30}{204} & 81.0 & \cellcolor{TealBlue!30}{1} & \cellcolor{TealBlue!30}{204} & \cellcolor{TealBlue!30}{\textbf{65.1}}\\
\texttt{heart-cleveland} & \multicolumn{1}{r}{296} & \multicolumn{1}{r}{50}  & \cellcolor{TealBlue!30}{1} & \cellcolor{TealBlue!30}{25} & 58.8 & \cellcolor{TealBlue!30}{1} & \cellcolor{TealBlue!30}{25} & 25.4 & \cellcolor{TealBlue!30}{1} & \cellcolor{TealBlue!30}{25} & \cellcolor{TealBlue!30}{\textbf{22.0}}\\
\texttt{hepatitis} & \multicolumn{1}{r}{137} & \multicolumn{1}{r}{68}  & \cellcolor{TealBlue!30}{1} & \cellcolor{TealBlue!30}{3} & 7.6 & \cellcolor{TealBlue!30}{1} & \cellcolor{TealBlue!30}{3} & 3.6 & \cellcolor{TealBlue!30}{1} & \cellcolor{TealBlue!30}{3} & \cellcolor{TealBlue!30}{\textbf{3.5}}\\
\texttt{hypothyroid} & \multicolumn{1}{r}{3247} & \multicolumn{1}{r}{43}  & \cellcolor{TealBlue!30}{1} & \cellcolor{TealBlue!30}{53} & 61.6 & \cellcolor{TealBlue!30}{1} & \cellcolor{TealBlue!30}{53} & \cellcolor{TealBlue!30}{\textbf{45.0}} & \cellcolor{TealBlue!30}{1} & \cellcolor{TealBlue!30}{53} & 48.5\\
\texttt{ionosphere} & \multicolumn{1}{r}{351} & \multicolumn{1}{r}{444}  & \cellcolor{TealBlue!30}{0} & 8 & 123.0 & \cellcolor{TealBlue!30}{0} & 8 & \cellcolor{TealBlue!30}{\textbf{58.4}} & \cellcolor{TealBlue!30}{0} & \cellcolor{TealBlue!30}{\textbf{7}} & 3410.0\\
\texttt{kr-vs-kp} & \multicolumn{1}{r}{3196} & \multicolumn{1}{r}{37}  & \cellcolor{TealBlue!30}{1} & \cellcolor{TealBlue!30}{144} & 38.4 & \cellcolor{TealBlue!30}{1} & \cellcolor{TealBlue!30}{144} & \cellcolor{TealBlue!30}{\textbf{27.7}} & \cellcolor{TealBlue!30}{1} & \cellcolor{TealBlue!30}{144} & 28.8\\
\texttt{letter} & \multicolumn{1}{r}{20000} & \multicolumn{1}{r}{224}  & \cellcolor{TealBlue!30}{0} & \cellcolor{TealBlue!30}{261} & 1190.0 & \cellcolor{TealBlue!30}{0} & \cellcolor{TealBlue!30}{261} & \cellcolor{TealBlue!30}{\textbf{410.0}} & \cellcolor{TealBlue!30}{0} & 263 & 3040.0\\
\texttt{lymph} & \multicolumn{1}{r}{148} & \multicolumn{1}{r}{41}  & \cellcolor{TealBlue!30}{1} & \cellcolor{TealBlue!30}{3} & 5.3 & \cellcolor{TealBlue!30}{1} & \cellcolor{TealBlue!30}{3} & 2.7 & \cellcolor{TealBlue!30}{1} & \cellcolor{TealBlue!30}{3} & \cellcolor{TealBlue!30}{\textbf{2.3}}\\
\texttt{mushroom} & \multicolumn{1}{r}{8124} & \multicolumn{1}{r}{91}  & \cellcolor{TealBlue!30}{1} & \cellcolor{TealBlue!30}{0} & 0.0 & \cellcolor{TealBlue!30}{1} & \cellcolor{TealBlue!30}{0} & \cellcolor{TealBlue!30}{\textbf{0.0}} & \cellcolor{TealBlue!30}{1} & \cellcolor{TealBlue!30}{0} & 0.0\\
\texttt{pendigits} & \multicolumn{1}{r}{7494} & \multicolumn{1}{r}{216}  & 0 & \cellcolor{TealBlue!30}{13} & \cellcolor{TealBlue!30}{\textbf{1250.0}} & \cellcolor{TealBlue!30}{1} & \cellcolor{TealBlue!30}{13} & 3040.0 & \cellcolor{TealBlue!30}{1} & \cellcolor{TealBlue!30}{13} & 3560.0\\
\texttt{primary-tumor} & \multicolumn{1}{r}{336} & \multicolumn{1}{r}{16}  & \cellcolor{TealBlue!30}{1} & \cellcolor{TealBlue!30}{34} & 0.6 & \cellcolor{TealBlue!30}{1} & \cellcolor{TealBlue!30}{34} & \cellcolor{TealBlue!30}{\textbf{0.3}} & \cellcolor{TealBlue!30}{1} & \cellcolor{TealBlue!30}{34} & 0.3\\
\texttt{segment} & \multicolumn{1}{r}{2310} & \multicolumn{1}{r}{234}  & \cellcolor{TealBlue!30}{1} & \cellcolor{TealBlue!30}{0} & 0.0 & \cellcolor{TealBlue!30}{1} & \cellcolor{TealBlue!30}{0} & \cellcolor{TealBlue!30}{\textbf{0.0}} & \cellcolor{TealBlue!30}{1} & \cellcolor{TealBlue!30}{0} & 0.0\\
\texttt{soybean} & \multicolumn{1}{r}{630} & \multicolumn{1}{r}{34}  & \cellcolor{TealBlue!30}{1} & \cellcolor{TealBlue!30}{14} & 2.9 & \cellcolor{TealBlue!30}{1} & \cellcolor{TealBlue!30}{14} & 1.7 & \cellcolor{TealBlue!30}{1} & \cellcolor{TealBlue!30}{14} & \cellcolor{TealBlue!30}{\textbf{1.6}}\\
\texttt{splice-1} & \multicolumn{1}{r}{3190} & \multicolumn{1}{r}{227}  & \cellcolor{TealBlue!30}{0} & \cellcolor{TealBlue!30}{141} & 13.5 & \cellcolor{TealBlue!30}{0} & \cellcolor{TealBlue!30}{141} & 7.7 & \cellcolor{TealBlue!30}{0} & \cellcolor{TealBlue!30}{141} & \cellcolor{TealBlue!30}{\textbf{6.2}}\\
\texttt{taiwan\_binarised} & \multicolumn{1}{r}{30000} & \multicolumn{1}{r}{198}  & \cellcolor{TealBlue!30}{0} & \cellcolor{TealBlue!30}{5273} & 16.3 & \cellcolor{TealBlue!30}{0} & \cellcolor{TealBlue!30}{5273} & \cellcolor{TealBlue!30}{\textbf{7.9}} & \cellcolor{TealBlue!30}{0} & \cellcolor{TealBlue!30}{5273} & 100.0\\
\texttt{tic-tac-toe} & \multicolumn{1}{r}{958} & \multicolumn{1}{r}{18}  & \cellcolor{TealBlue!30}{1} & \cellcolor{TealBlue!30}{137} & 0.8 & \cellcolor{TealBlue!30}{1} & \cellcolor{TealBlue!30}{137} & 0.5 & \cellcolor{TealBlue!30}{1} & \cellcolor{TealBlue!30}{137} & \cellcolor{TealBlue!30}{\textbf{0.5}}\\
\texttt{vehicle} & \multicolumn{1}{r}{846} & \multicolumn{1}{r}{252}  & \cellcolor{TealBlue!30}{1} & \cellcolor{TealBlue!30}{12} & 1800.0 & \cellcolor{TealBlue!30}{1} & \cellcolor{TealBlue!30}{12} & 944.0 & \cellcolor{TealBlue!30}{1} & \cellcolor{TealBlue!30}{12} & \cellcolor{TealBlue!30}{\textbf{902.0}}\\
\texttt{vote} & \multicolumn{1}{r}{435} & \multicolumn{1}{r}{32}  & \cellcolor{TealBlue!30}{1} & \cellcolor{TealBlue!30}{5} & 2.9 & \cellcolor{TealBlue!30}{1} & \cellcolor{TealBlue!30}{5} & 1.6 & \cellcolor{TealBlue!30}{1} & \cellcolor{TealBlue!30}{5} & \cellcolor{TealBlue!30}{\textbf{1.4}}\\
\texttt{wine1-un} & \multicolumn{1}{r}{178} & \multicolumn{1}{r}{1276}  & \cellcolor{TealBlue!30}{0} & 39 & \cellcolor{TealBlue!30}{\textbf{3.8}} & \cellcolor{TealBlue!30}{0} & \cellcolor{TealBlue!30}{38} & 2290.0 & \cellcolor{TealBlue!30}{0} & \cellcolor{TealBlue!30}{38} & 918.0\\
\texttt{wine2-un} & \multicolumn{1}{r}{178} & \multicolumn{1}{r}{1276}  & \cellcolor{TealBlue!30}{0} & \cellcolor{TealBlue!30}{43} & 556.0 & \cellcolor{TealBlue!30}{0} & \cellcolor{TealBlue!30}{43} & 115.0 & \cellcolor{TealBlue!30}{0} & \cellcolor{TealBlue!30}{43} & \cellcolor{TealBlue!30}{\textbf{0.1}}\\
\texttt{wine3-un} & \multicolumn{1}{r}{178} & \multicolumn{1}{r}{1276}  & \cellcolor{TealBlue!30}{0} & \cellcolor{TealBlue!30}{28} & 3600.0 & \cellcolor{TealBlue!30}{0} & \cellcolor{TealBlue!30}{28} & \cellcolor{TealBlue!30}{\textbf{230.0}} & \cellcolor{TealBlue!30}{0} & \cellcolor{TealBlue!30}{28} & 413.0\\
\texttt{yeast} & \multicolumn{1}{r}{1484} & \multicolumn{1}{r}{89}  & \cellcolor{TealBlue!30}{1} & \cellcolor{TealBlue!30}{366} & 70.8 & \cellcolor{TealBlue!30}{1} & \cellcolor{TealBlue!30}{366} & 39.2 & \cellcolor{TealBlue!30}{1} & \cellcolor{TealBlue!30}{366} & \cellcolor{TealBlue!30}{\textbf{38.7}}\\
\texttt{zoo-1} & \multicolumn{1}{r}{101} & \multicolumn{1}{r}{20}  & \cellcolor{TealBlue!30}{1} & \cellcolor{TealBlue!30}{0} & 0.0 & \cellcolor{TealBlue!30}{1} & \cellcolor{TealBlue!30}{0} & \cellcolor{TealBlue!30}{\textbf{0.0}} & \cellcolor{TealBlue!30}{1} & \cellcolor{TealBlue!30}{0} & 0.0\\
\bottomrule
\end{tabular}

% \end{normalsize}
% \end{center}
% \caption{\label{tab:ha4} Comparison of heuristics (max depth=4)}
% \end{table}
%
% \begin{table}[htbp]
% \begin{center}
% \begin{normalsize}
% \tabcolsep=5pt
% \begin{tabular}{lccrrrrrrrrr}
\toprule
& && \multicolumn{3}{c}{entropy} & \multicolumn{3}{c}{\budalg} & \multicolumn{3}{c}{error}\\
\cmidrule(rr){4-6}\cmidrule(rr){7-9}\cmidrule(rr){10-12}
&\multirow{1}{*}{$\#ex.$} & \multirow{1}{*}{\#feat.} &  \multicolumn{1}{c}{opt} & \multicolumn{1}{c}{error} & \multicolumn{1}{c}{time} & \multicolumn{1}{c}{opt} & \multicolumn{1}{c}{error} & \multicolumn{1}{c}{time} & \multicolumn{1}{c}{opt} & \multicolumn{1}{c}{error} & \multicolumn{1}{c}{time} \\
\midrule

\texttt{anneal} & \multicolumn{1}{r}{812} & \multicolumn{1}{r}{47}  & \cellcolor{TealBlue!30}{1} & \cellcolor{TealBlue!30}{70} & 1630.0 & \cellcolor{TealBlue!30}{1} & \cellcolor{TealBlue!30}{70} & 995.0 & \cellcolor{TealBlue!30}{1} & \cellcolor{TealBlue!30}{70} & \cellcolor{TealBlue!30}{\textbf{920.0}}\\
\texttt{audiology} & \multicolumn{1}{r}{216} & \multicolumn{1}{r}{79}  & \cellcolor{TealBlue!30}{1} & \cellcolor{TealBlue!30}{0} & 0.0 & \cellcolor{TealBlue!30}{1} & \cellcolor{TealBlue!30}{0} & 0.0 & \cellcolor{TealBlue!30}{1} & \cellcolor{TealBlue!30}{0} & \cellcolor{TealBlue!30}{\textbf{0.0}}\\
\texttt{australian-credit} & \multicolumn{1}{r}{653} & \multicolumn{1}{r}{73}  & \cellcolor{TealBlue!30}{0} & \cellcolor{TealBlue!30}{40} & 108.0 & \cellcolor{TealBlue!30}{0} & \cellcolor{TealBlue!30}{40} & \cellcolor{TealBlue!30}{\textbf{51.3}} & \cellcolor{TealBlue!30}{0} & \cellcolor{TealBlue!30}{40} & 51.8\\
\texttt{breast-cancer-un} & \multicolumn{1}{r}{683} & \multicolumn{1}{r}{89}  & \cellcolor{TealBlue!30}{1} & \cellcolor{TealBlue!30}{6} & 1800.0 & \cellcolor{TealBlue!30}{1} & \cellcolor{TealBlue!30}{6} & 973.0 & \cellcolor{TealBlue!30}{1} & \cellcolor{TealBlue!30}{6} & \cellcolor{TealBlue!30}{\textbf{892.0}}\\
\texttt{breast-wisconsin} & \multicolumn{1}{r}{683} & \multicolumn{1}{r}{120}  & \cellcolor{TealBlue!30}{1} & \cellcolor{TealBlue!30}{0} & 1180.0 & \cellcolor{TealBlue!30}{1} & \cellcolor{TealBlue!30}{0} & 509.0 & \cellcolor{TealBlue!30}{1} & \cellcolor{TealBlue!30}{0} & \cellcolor{TealBlue!30}{\textbf{381.0}}\\
\texttt{car-un} & \multicolumn{1}{r}{1728} & \multicolumn{1}{r}{21}  & \cellcolor{TealBlue!30}{1} & \cellcolor{TealBlue!30}{86} & 5.4 & \cellcolor{TealBlue!30}{1} & \cellcolor{TealBlue!30}{86} & \cellcolor{TealBlue!30}{\textbf{3.9}} & \cellcolor{TealBlue!30}{1} & \cellcolor{TealBlue!30}{86} & 4.2\\
\texttt{diabetes} & \multicolumn{1}{r}{768} & \multicolumn{1}{r}{112}  & \cellcolor{TealBlue!30}{0} & 107 & \cellcolor{TealBlue!30}{\textbf{131.0}} & \cellcolor{TealBlue!30}{0} & \cellcolor{TealBlue!30}{106} & 1910.0 & \cellcolor{TealBlue!30}{0} & \cellcolor{TealBlue!30}{106} & 3130.0\\
\texttt{forest-fires-un} & \multicolumn{1}{r}{517} & \multicolumn{1}{r}{989}  & \cellcolor{TealBlue!30}{0} & 172 & \cellcolor{TealBlue!30}{\textbf{105.0}} & \cellcolor{TealBlue!30}{0} & \cellcolor{TealBlue!30}{\textbf{156}} & 2980.0 & \cellcolor{TealBlue!30}{0} & 157 & 435.0\\
\texttt{german-credit} & \multicolumn{1}{r}{1000} & \multicolumn{1}{r}{110}  & \cellcolor{TealBlue!30}{0} & \cellcolor{TealBlue!30}{161} & 197.0 & \cellcolor{TealBlue!30}{0} & \cellcolor{TealBlue!30}{161} & \cellcolor{TealBlue!30}{\textbf{103.0}} & \cellcolor{TealBlue!30}{0} & 165 & 2600.0\\
\texttt{heart-cleveland} & \multicolumn{1}{r}{296} & \multicolumn{1}{r}{50}  & \cellcolor{TealBlue!30}{1} & \cellcolor{TealBlue!30}{7} & 3450.0 & \cellcolor{TealBlue!30}{1} & \cellcolor{TealBlue!30}{7} & 1520.0 & \cellcolor{TealBlue!30}{1} & \cellcolor{TealBlue!30}{7} & \cellcolor{TealBlue!30}{\textbf{1370.0}}\\
\texttt{hepatitis} & \multicolumn{1}{r}{137} & \multicolumn{1}{r}{68}  & \cellcolor{TealBlue!30}{1} & \cellcolor{TealBlue!30}{0} & 1.0 & \cellcolor{TealBlue!30}{1} & \cellcolor{TealBlue!30}{0} & \cellcolor{TealBlue!30}{\textbf{0.5}} & \cellcolor{TealBlue!30}{1} & \cellcolor{TealBlue!30}{0} & 1.6\\
\texttt{hypothyroid} & \multicolumn{1}{r}{3247} & \multicolumn{1}{r}{43}  & 0 & \cellcolor{TealBlue!30}{44} & \cellcolor{TealBlue!30}{\textbf{1070.0}} & \cellcolor{TealBlue!30}{1} & \cellcolor{TealBlue!30}{44} & 2850.0 & \cellcolor{TealBlue!30}{1} & \cellcolor{TealBlue!30}{44} & 2910.0\\
\texttt{ionosphere} & \multicolumn{1}{r}{351} & \multicolumn{1}{r}{444}  & \cellcolor{TealBlue!30}{0} & 3 & \cellcolor{TealBlue!30}{\textbf{258.0}} & \cellcolor{TealBlue!30}{0} & \cellcolor{TealBlue!30}{2} & 1980.0 & \cellcolor{TealBlue!30}{0} & \cellcolor{TealBlue!30}{2} & 707.0\\
\texttt{kr-vs-kp} & \multicolumn{1}{r}{3196} & \multicolumn{1}{r}{37}  & \cellcolor{TealBlue!30}{1} & \cellcolor{TealBlue!30}{81} & 1970.0 & \cellcolor{TealBlue!30}{1} & \cellcolor{TealBlue!30}{81} & 1400.0 & \cellcolor{TealBlue!30}{1} & \cellcolor{TealBlue!30}{81} & \cellcolor{TealBlue!30}{\textbf{1390.0}}\\
\texttt{letter} & \multicolumn{1}{r}{20000} & \multicolumn{1}{r}{224}  & \cellcolor{TealBlue!30}{0} & 447 & 2910.0 & \cellcolor{TealBlue!30}{0} & 280 & \cellcolor{TealBlue!30}{\textbf{373.0}} & \cellcolor{TealBlue!30}{0} & \cellcolor{TealBlue!30}{\textbf{251}} & 2970.0\\
\texttt{lymph} & \multicolumn{1}{r}{148} & \multicolumn{1}{r}{41}  & \cellcolor{TealBlue!30}{1} & \cellcolor{TealBlue!30}{0} & 0.0 & \cellcolor{TealBlue!30}{1} & \cellcolor{TealBlue!30}{0} & \cellcolor{TealBlue!30}{\textbf{0.0}} & \cellcolor{TealBlue!30}{1} & \cellcolor{TealBlue!30}{0} & 0.0\\
\texttt{mushroom} & \multicolumn{1}{r}{8124} & \multicolumn{1}{r}{91}  & \cellcolor{TealBlue!30}{1} & \cellcolor{TealBlue!30}{0} & 0.0 & \cellcolor{TealBlue!30}{1} & \cellcolor{TealBlue!30}{0} & \cellcolor{TealBlue!30}{\textbf{0.0}} & \cellcolor{TealBlue!30}{1} & \cellcolor{TealBlue!30}{0} & 0.0\\
\texttt{pendigits} & \multicolumn{1}{r}{7494} & \multicolumn{1}{r}{216}  & \cellcolor{TealBlue!30}{0} & \cellcolor{TealBlue!30}{2} & \cellcolor{TealBlue!30}{\textbf{90.6}} & \cellcolor{TealBlue!30}{0} & \cellcolor{TealBlue!30}{2} & 1780.0 & \cellcolor{TealBlue!30}{0} & \cellcolor{TealBlue!30}{2} & 338.0\\
\texttt{primary-tumor} & \multicolumn{1}{r}{336} & \multicolumn{1}{r}{16}  & \cellcolor{TealBlue!30}{1} & \cellcolor{TealBlue!30}{26} & 16.6 & \cellcolor{TealBlue!30}{1} & \cellcolor{TealBlue!30}{26} & \cellcolor{TealBlue!30}{\textbf{8.9}} & \cellcolor{TealBlue!30}{1} & \cellcolor{TealBlue!30}{26} & 9.1\\
\texttt{segment} & \multicolumn{1}{r}{2310} & \multicolumn{1}{r}{234}  & \cellcolor{TealBlue!30}{1} & \cellcolor{TealBlue!30}{0} & 0.0 & \cellcolor{TealBlue!30}{1} & \cellcolor{TealBlue!30}{0} & \cellcolor{TealBlue!30}{\textbf{0.0}} & \cellcolor{TealBlue!30}{1} & \cellcolor{TealBlue!30}{0} & 0.0\\
\texttt{soybean} & \multicolumn{1}{r}{630} & \multicolumn{1}{r}{34}  & \cellcolor{TealBlue!30}{1} & \cellcolor{TealBlue!30}{8} & 101.0 & \cellcolor{TealBlue!30}{1} & \cellcolor{TealBlue!30}{8} & 62.7 & \cellcolor{TealBlue!30}{1} & \cellcolor{TealBlue!30}{8} & \cellcolor{TealBlue!30}{\textbf{59.0}}\\
\texttt{splice-1} & \multicolumn{1}{r}{3190} & \multicolumn{1}{r}{227}  & \cellcolor{TealBlue!30}{0} & 103 & \cellcolor{TealBlue!30}{\textbf{41.8}} & \cellcolor{TealBlue!30}{0} & \cellcolor{TealBlue!30}{101} & 2260.0 & \cellcolor{TealBlue!30}{0} & \cellcolor{TealBlue!30}{101} & 2010.0\\
\texttt{taiwan\_binarised} & \multicolumn{1}{r}{30000} & \multicolumn{1}{r}{198}  & \cellcolor{TealBlue!30}{0} & \cellcolor{TealBlue!30}{5200} & 1840.0 & \cellcolor{TealBlue!30}{0} & \cellcolor{TealBlue!30}{5200} & \cellcolor{TealBlue!30}{\textbf{1290.0}} & \cellcolor{TealBlue!30}{0} & 5204 & 2210.0\\
\texttt{tic-tac-toe} & \multicolumn{1}{r}{958} & \multicolumn{1}{r}{18}  & \cellcolor{TealBlue!30}{1} & \cellcolor{TealBlue!30}{63} & 21.5 & \cellcolor{TealBlue!30}{1} & \cellcolor{TealBlue!30}{63} & 12.1 & \cellcolor{TealBlue!30}{1} & \cellcolor{TealBlue!30}{63} & \cellcolor{TealBlue!30}{\textbf{11.1}}\\
\texttt{vehicle} & \multicolumn{1}{r}{846} & \multicolumn{1}{r}{252}  & \cellcolor{TealBlue!30}{0} & \cellcolor{TealBlue!30}{3} & 2630.0 & \cellcolor{TealBlue!30}{0} & \cellcolor{TealBlue!30}{3} & \cellcolor{TealBlue!30}{\textbf{88.8}} & \cellcolor{TealBlue!30}{0} & 9 & 3550.0\\
\texttt{vote} & \multicolumn{1}{r}{435} & \multicolumn{1}{r}{32}  & \cellcolor{TealBlue!30}{1} & \cellcolor{TealBlue!30}{1} & 54.8 & \cellcolor{TealBlue!30}{1} & \cellcolor{TealBlue!30}{1} & 30.1 & \cellcolor{TealBlue!30}{1} & \cellcolor{TealBlue!30}{1} & \cellcolor{TealBlue!30}{\textbf{27.7}}\\
\texttt{wine1-un} & \multicolumn{1}{r}{178} & \multicolumn{1}{r}{1276}  & \cellcolor{TealBlue!30}{0} & 35 & 1010.0 & \cellcolor{TealBlue!30}{0} & \cellcolor{TealBlue!30}{34} & 1430.0 & \cellcolor{TealBlue!30}{0} & \cellcolor{TealBlue!30}{34} & \cellcolor{TealBlue!30}{\textbf{804.0}}\\
\texttt{wine2-un} & \multicolumn{1}{r}{178} & \multicolumn{1}{r}{1276}  & \cellcolor{TealBlue!30}{0} & 40 & 226.0 & \cellcolor{TealBlue!30}{0} & 39 & 2820.0 & \cellcolor{TealBlue!30}{0} & \cellcolor{TealBlue!30}{\textbf{37}} & \cellcolor{TealBlue!30}{\textbf{86.9}}\\
\texttt{wine3-un} & \multicolumn{1}{r}{178} & \multicolumn{1}{r}{1276}  & \cellcolor{TealBlue!30}{0} & \cellcolor{TealBlue!30}{25} & 254.0 & \cellcolor{TealBlue!30}{0} & \cellcolor{TealBlue!30}{25} & \cellcolor{TealBlue!30}{\textbf{110.0}} & \cellcolor{TealBlue!30}{0} & 26 & 914.0\\
\texttt{yeast} & \multicolumn{1}{r}{1484} & \multicolumn{1}{r}{89}  & 0 & \cellcolor{TealBlue!30}{313} & \cellcolor{TealBlue!30}{\textbf{1430.0}} & \cellcolor{TealBlue!30}{1} & \cellcolor{TealBlue!30}{313} & 3270.0 & \cellcolor{TealBlue!30}{1} & \cellcolor{TealBlue!30}{313} & 3290.0\\
\texttt{zoo-1} & \multicolumn{1}{r}{101} & \multicolumn{1}{r}{20}  & \cellcolor{TealBlue!30}{1} & \cellcolor{TealBlue!30}{0} & 0.0 & \cellcolor{TealBlue!30}{1} & \cellcolor{TealBlue!30}{0} & \cellcolor{TealBlue!30}{\textbf{0.0}} & \cellcolor{TealBlue!30}{1} & \cellcolor{TealBlue!30}{0} & 0.0\\
\bottomrule
\end{tabular}

% \end{normalsize}
% \end{center}
% \caption{\label{tab:ha5} Comparison of heuristics (max depth=5)}
% \end{table}
%
% \begin{table}[htbp]
% \begin{center}
% \begin{normalsize}
% \tabcolsep=5pt
% \begin{tabular}{lccrrrrrrrrr}
\toprule
& && \multicolumn{3}{c}{entropy} & \multicolumn{3}{c}{\budalg} & \multicolumn{3}{c}{error}\\
\cmidrule(rr){4-6}\cmidrule(rr){7-9}\cmidrule(rr){10-12}
&\multirow{1}{*}{$\#ex.$} & \multirow{1}{*}{\#feat.} &  \multicolumn{1}{c}{opt} & \multicolumn{1}{c}{error} & \multicolumn{1}{c}{time} & \multicolumn{1}{c}{opt} & \multicolumn{1}{c}{error} & \multicolumn{1}{c}{time} & \multicolumn{1}{c}{opt} & \multicolumn{1}{c}{error} & \multicolumn{1}{c}{time} \\
\midrule

\texttt{anneal} & \multicolumn{1}{r}{812} & \multicolumn{1}{r}{47}  & \cellcolor{TealBlue!30}{0} & \cellcolor{TealBlue!30}{58} & 430.0 & \cellcolor{TealBlue!30}{0} & \cellcolor{TealBlue!30}{58} & \cellcolor{TealBlue!30}{\textbf{209.0}} & \cellcolor{TealBlue!30}{0} & 64 & 3090.0\\
\texttt{audiology} & \multicolumn{1}{r}{216} & \multicolumn{1}{r}{79}  & \cellcolor{TealBlue!30}{1} & \cellcolor{TealBlue!30}{0} & 0.0 & \cellcolor{TealBlue!30}{1} & \cellcolor{TealBlue!30}{0} & 0.0 & \cellcolor{TealBlue!30}{1} & \cellcolor{TealBlue!30}{0} & \cellcolor{TealBlue!30}{\textbf{0.0}}\\
\texttt{australian-credit} & \multicolumn{1}{r}{653} & \multicolumn{1}{r}{73}  & 0 & 9 & \cellcolor{TealBlue!30}{\textbf{726.0}} & \cellcolor{TealBlue!30}{\textbf{1}} & \cellcolor{TealBlue!30}{\textbf{0}} & 1040.0 & 0 & 13 & 2020.0\\
\texttt{breast-cancer-un} & \multicolumn{1}{r}{683} & \multicolumn{1}{r}{89}  & \cellcolor{TealBlue!30}{1} & \cellcolor{TealBlue!30}{0} & 1340.0 & \cellcolor{TealBlue!30}{1} & \cellcolor{TealBlue!30}{0} & 1190.0 & \cellcolor{TealBlue!30}{1} & \cellcolor{TealBlue!30}{0} & \cellcolor{TealBlue!30}{\textbf{973.0}}\\
\texttt{breast-wisconsin} & \multicolumn{1}{r}{683} & \multicolumn{1}{r}{120}  & \cellcolor{TealBlue!30}{1} & \cellcolor{TealBlue!30}{0} & 0.3 & \cellcolor{TealBlue!30}{1} & \cellcolor{TealBlue!30}{0} & 0.3 & \cellcolor{TealBlue!30}{1} & \cellcolor{TealBlue!30}{0} & \cellcolor{TealBlue!30}{\textbf{0.1}}\\
\texttt{car-un} & \multicolumn{1}{r}{1728} & \multicolumn{1}{r}{21}  & \cellcolor{TealBlue!30}{1} & \cellcolor{TealBlue!30}{11} & 438.0 & \cellcolor{TealBlue!30}{1} & \cellcolor{TealBlue!30}{11} & 294.0 & \cellcolor{TealBlue!30}{1} & \cellcolor{TealBlue!30}{11} & \cellcolor{TealBlue!30}{\textbf{255.0}}\\
\texttt{diabetes} & \multicolumn{1}{r}{768} & \multicolumn{1}{r}{112}  & \cellcolor{TealBlue!30}{0} & 32 & 2080.0 & \cellcolor{TealBlue!30}{0} & \cellcolor{TealBlue!30}{\textbf{27}} & 3210.0 & \cellcolor{TealBlue!30}{0} & 61 & \cellcolor{TealBlue!30}{\textbf{481.0}}\\
\texttt{forest-fires-un} & \multicolumn{1}{r}{517} & \multicolumn{1}{r}{989}  & \cellcolor{TealBlue!30}{0} & 160 & 669.0 & \cellcolor{TealBlue!30}{0} & 155 & \cellcolor{TealBlue!30}{\textbf{325.0}} & \cellcolor{TealBlue!30}{0} & \cellcolor{TealBlue!30}{\textbf{150}} & 390.0\\
\texttt{german-credit} & \multicolumn{1}{r}{1000} & \multicolumn{1}{r}{110}  & \cellcolor{TealBlue!30}{0} & \cellcolor{TealBlue!30}{57} & 1360.0 & \cellcolor{TealBlue!30}{0} & \cellcolor{TealBlue!30}{57} & \cellcolor{TealBlue!30}{\textbf{706.0}} & \cellcolor{TealBlue!30}{0} & 133 & 3540.0\\
\texttt{heart-cleveland} & \multicolumn{1}{r}{296} & \multicolumn{1}{r}{50}  & \cellcolor{TealBlue!30}{1} & \cellcolor{TealBlue!30}{0} & 1.4 & \cellcolor{TealBlue!30}{1} & \cellcolor{TealBlue!30}{0} & \cellcolor{TealBlue!30}{\textbf{0.0}} & \cellcolor{TealBlue!30}{1} & \cellcolor{TealBlue!30}{0} & 46.6\\
\texttt{hepatitis} & \multicolumn{1}{r}{137} & \multicolumn{1}{r}{68}  & \cellcolor{TealBlue!30}{1} & \cellcolor{TealBlue!30}{0} & 0.0 & \cellcolor{TealBlue!30}{1} & \cellcolor{TealBlue!30}{0} & 0.0 & \cellcolor{TealBlue!30}{1} & \cellcolor{TealBlue!30}{0} & \cellcolor{TealBlue!30}{\textbf{0.0}}\\
\texttt{hypothyroid} & \multicolumn{1}{r}{3247} & \multicolumn{1}{r}{43}  & \cellcolor{TealBlue!30}{0} & 45 & 126.0 & \cellcolor{TealBlue!30}{0} & \cellcolor{TealBlue!30}{\textbf{43}} & \cellcolor{TealBlue!30}{\textbf{78.3}} & \cellcolor{TealBlue!30}{0} & 50 & 338.0\\
\texttt{ionosphere} & \multicolumn{1}{r}{351} & \multicolumn{1}{r}{444}  & \cellcolor{TealBlue!30}{1} & \cellcolor{TealBlue!30}{0} & \cellcolor{TealBlue!30}{\textbf{0.2}} & \cellcolor{TealBlue!30}{1} & \cellcolor{TealBlue!30}{0} & 0.5 & \cellcolor{TealBlue!30}{1} & \cellcolor{TealBlue!30}{0} & 1.1\\
\texttt{kr-vs-kp} & \multicolumn{1}{r}{3196} & \multicolumn{1}{r}{37}  & \cellcolor{TealBlue!30}{0} & 37 & \cellcolor{TealBlue!30}{\textbf{194.0}} & \cellcolor{TealBlue!30}{0} & 37 & 1460.0 & \cellcolor{TealBlue!30}{0} & \cellcolor{TealBlue!30}{\textbf{21}} & 222.0\\
\texttt{letter} & \multicolumn{1}{r}{20000} & \multicolumn{1}{r}{224}  & \cellcolor{TealBlue!30}{0} & \cellcolor{TealBlue!30}{\textbf{112}} & 3260.0 & \cellcolor{TealBlue!30}{0} & 118 & \cellcolor{TealBlue!30}{\textbf{219.0}} & \cellcolor{TealBlue!30}{0} & 184 & 1680.0\\
\texttt{lymph} & \multicolumn{1}{r}{148} & \multicolumn{1}{r}{41}  & \cellcolor{TealBlue!30}{1} & \cellcolor{TealBlue!30}{0} & \cellcolor{TealBlue!30}{\textbf{0.0}} & \cellcolor{TealBlue!30}{1} & \cellcolor{TealBlue!30}{0} & 0.0 & \cellcolor{TealBlue!30}{1} & \cellcolor{TealBlue!30}{0} & 0.0\\
\texttt{mushroom} & \multicolumn{1}{r}{8124} & \multicolumn{1}{r}{91}  & \cellcolor{TealBlue!30}{1} & \cellcolor{TealBlue!30}{0} & 0.0 & \cellcolor{TealBlue!30}{1} & \cellcolor{TealBlue!30}{0} & \cellcolor{TealBlue!30}{\textbf{0.0}} & \cellcolor{TealBlue!30}{1} & \cellcolor{TealBlue!30}{0} & 0.0\\
\texttt{pendigits} & \multicolumn{1}{r}{7494} & \multicolumn{1}{r}{216}  & \cellcolor{TealBlue!30}{1} & \cellcolor{TealBlue!30}{0} & 0.1 & \cellcolor{TealBlue!30}{1} & \cellcolor{TealBlue!30}{0} & \cellcolor{TealBlue!30}{\textbf{0.1}} & \cellcolor{TealBlue!30}{1} & \cellcolor{TealBlue!30}{0} & 55.9\\
\texttt{primary-tumor} & \multicolumn{1}{r}{336} & \multicolumn{1}{r}{16}  & \cellcolor{TealBlue!30}{0} & \cellcolor{TealBlue!30}{16} & 178.0 & \cellcolor{TealBlue!30}{0} & \cellcolor{TealBlue!30}{16} & 93.9 & \cellcolor{TealBlue!30}{0} & \cellcolor{TealBlue!30}{16} & \cellcolor{TealBlue!30}{\textbf{93.7}}\\
\texttt{segment} & \multicolumn{1}{r}{2310} & \multicolumn{1}{r}{234}  & \cellcolor{TealBlue!30}{1} & \cellcolor{TealBlue!30}{0} & 0.0 & \cellcolor{TealBlue!30}{1} & \cellcolor{TealBlue!30}{0} & 0.0 & \cellcolor{TealBlue!30}{1} & \cellcolor{TealBlue!30}{0} & \cellcolor{TealBlue!30}{\textbf{0.0}}\\
\texttt{soybean} & \multicolumn{1}{r}{630} & \multicolumn{1}{r}{34}  & \cellcolor{TealBlue!30}{0} & \cellcolor{TealBlue!30}{2} & 1960.0 & \cellcolor{TealBlue!30}{0} & \cellcolor{TealBlue!30}{2} & 3540.0 & \cellcolor{TealBlue!30}{0} & \cellcolor{TealBlue!30}{2} & \cellcolor{TealBlue!30}{\textbf{1730.0}}\\
\texttt{splice-1} & \multicolumn{1}{r}{3190} & \multicolumn{1}{r}{227}  & \cellcolor{TealBlue!30}{0} & 33 & 2570.0 & \cellcolor{TealBlue!30}{0} & \cellcolor{TealBlue!30}{\textbf{32}} & 2440.0 & \cellcolor{TealBlue!30}{0} & 45 & \cellcolor{TealBlue!30}{\textbf{1730.0}}\\
\texttt{taiwan\_binarised} & \multicolumn{1}{r}{30000} & \multicolumn{1}{r}{198}  & \cellcolor{TealBlue!30}{0} & \cellcolor{TealBlue!30}{\textbf{5017}} & 2250.0 & \cellcolor{TealBlue!30}{0} & 5065 & \cellcolor{TealBlue!30}{\textbf{70.3}} & \cellcolor{TealBlue!30}{0} & 5152 & 177.0\\
\texttt{tic-tac-toe} & \multicolumn{1}{r}{958} & \multicolumn{1}{r}{18}  & \cellcolor{TealBlue!30}{1} & \cellcolor{TealBlue!30}{0} & 48.7 & \cellcolor{TealBlue!30}{1} & \cellcolor{TealBlue!30}{0} & \cellcolor{TealBlue!30}{\textbf{26.4}} & \cellcolor{TealBlue!30}{1} & \cellcolor{TealBlue!30}{0} & 27.5\\
\texttt{vehicle} & \multicolumn{1}{r}{846} & \multicolumn{1}{r}{252}  & \cellcolor{TealBlue!30}{1} & \cellcolor{TealBlue!30}{0} & 1.7 & \cellcolor{TealBlue!30}{1} & \cellcolor{TealBlue!30}{0} & \cellcolor{TealBlue!30}{\textbf{0.6}} & \cellcolor{TealBlue!30}{1} & \cellcolor{TealBlue!30}{0} & 688.0\\
\texttt{vote} & \multicolumn{1}{r}{435} & \multicolumn{1}{r}{32}  & \cellcolor{TealBlue!30}{1} & \cellcolor{TealBlue!30}{0} & 0.0 & \cellcolor{TealBlue!30}{1} & \cellcolor{TealBlue!30}{0} & \cellcolor{TealBlue!30}{\textbf{0.0}} & \cellcolor{TealBlue!30}{1} & \cellcolor{TealBlue!30}{0} & 0.1\\
\texttt{wine1-un} & \multicolumn{1}{r}{178} & \multicolumn{1}{r}{1276}  & \cellcolor{TealBlue!30}{0} & 34 & 2850.0 & \cellcolor{TealBlue!30}{0} & 29 & 509.0 & \cellcolor{TealBlue!30}{0} & \cellcolor{TealBlue!30}{\textbf{28}} & \cellcolor{TealBlue!30}{\textbf{94.4}}\\
\texttt{wine2-un} & \multicolumn{1}{r}{178} & \multicolumn{1}{r}{1276}  & \cellcolor{TealBlue!30}{0} & \cellcolor{TealBlue!30}{31} & 2110.0 & \cellcolor{TealBlue!30}{0} & \cellcolor{TealBlue!30}{31} & 172.0 & \cellcolor{TealBlue!30}{0} & \cellcolor{TealBlue!30}{31} & \cellcolor{TealBlue!30}{\textbf{0.2}}\\
\texttt{wine3-un} & \multicolumn{1}{r}{178} & \multicolumn{1}{r}{1276}  & \cellcolor{TealBlue!30}{0} & \cellcolor{TealBlue!30}{20} & 644.0 & \cellcolor{TealBlue!30}{0} & \cellcolor{TealBlue!30}{20} & 307.0 & \cellcolor{TealBlue!30}{0} & 23 & \cellcolor{TealBlue!30}{\textbf{76.3}}\\
\texttt{yeast} & \multicolumn{1}{r}{1484} & \multicolumn{1}{r}{89}  & \cellcolor{TealBlue!30}{0} & 265 & \cellcolor{TealBlue!30}{\textbf{215.0}} & \cellcolor{TealBlue!30}{0} & 264 & 3130.0 & \cellcolor{TealBlue!30}{0} & \cellcolor{TealBlue!30}{\textbf{252}} & 3190.0\\
\texttt{zoo-1} & \multicolumn{1}{r}{101} & \multicolumn{1}{r}{20}  & \cellcolor{TealBlue!30}{1} & \cellcolor{TealBlue!30}{0} & 0.0 & \cellcolor{TealBlue!30}{1} & \cellcolor{TealBlue!30}{0} & \cellcolor{TealBlue!30}{\textbf{0.0}} & \cellcolor{TealBlue!30}{1} & \cellcolor{TealBlue!30}{0} & 0.0\\
\bottomrule
\end{tabular}

% \end{normalsize}
% \end{center}
% \caption{\label{tab:ha7} Comparison of heuristics (max depth=7)}
% \end{table}
%
% \begin{table}[htbp]
% \begin{center}
% \begin{normalsize}
% \tabcolsep=5pt
% \begin{tabular}{lccrrrrrrrrr}
\toprule
& && \multicolumn{3}{c}{entropy} & \multicolumn{3}{c}{\budalg} & \multicolumn{3}{c}{error}\\
\cmidrule(rr){4-6}\cmidrule(rr){7-9}\cmidrule(rr){10-12}
&\multirow{1}{*}{$\#ex.$} & \multirow{1}{*}{\#feat.} &  \multicolumn{1}{c}{opt} & \multicolumn{1}{c}{error} & \multicolumn{1}{c}{time} & \multicolumn{1}{c}{opt} & \multicolumn{1}{c}{error} & \multicolumn{1}{c}{time} & \multicolumn{1}{c}{opt} & \multicolumn{1}{c}{error} & \multicolumn{1}{c}{time} \\
\midrule

\texttt{anneal} & \multicolumn{1}{r}{812} & \multicolumn{1}{r}{47}  & \cellcolor{TealBlue!30}{0} & 76 & 1260.0 & \cellcolor{TealBlue!30}{0} & 58 & 773.0 & \cellcolor{TealBlue!30}{0} & \cellcolor{TealBlue!30}{\textbf{48}} & \cellcolor{TealBlue!30}{\textbf{678.0}}\\
\texttt{audiology} & \multicolumn{1}{r}{216} & \multicolumn{1}{r}{79}  & \cellcolor{TealBlue!30}{1} & \cellcolor{TealBlue!30}{0} & 0.0 & \cellcolor{TealBlue!30}{1} & \cellcolor{TealBlue!30}{0} & 0.0 & \cellcolor{TealBlue!30}{1} & \cellcolor{TealBlue!30}{0} & \cellcolor{TealBlue!30}{\textbf{0.0}}\\
\texttt{australian-credit} & \multicolumn{1}{r}{653} & \multicolumn{1}{r}{73}  & \cellcolor{TealBlue!30}{1} & \cellcolor{TealBlue!30}{0} & 0.8 & \cellcolor{TealBlue!30}{1} & \cellcolor{TealBlue!30}{0} & \cellcolor{TealBlue!30}{\textbf{0.3}} & 0 & 1 & 2160.0\\
\texttt{breast-cancer-un} & \multicolumn{1}{r}{683} & \multicolumn{1}{r}{89}  & \cellcolor{TealBlue!30}{1} & \cellcolor{TealBlue!30}{0} & 1.9 & \cellcolor{TealBlue!30}{1} & \cellcolor{TealBlue!30}{0} & \cellcolor{TealBlue!30}{\textbf{0.0}} & \cellcolor{TealBlue!30}{1} & \cellcolor{TealBlue!30}{0} & 0.4\\
\texttt{breast-wisconsin} & \multicolumn{1}{r}{683} & \multicolumn{1}{r}{120}  & \cellcolor{TealBlue!30}{1} & \cellcolor{TealBlue!30}{0} & 0.0 & \cellcolor{TealBlue!30}{1} & \cellcolor{TealBlue!30}{0} & \cellcolor{TealBlue!30}{\textbf{0.0}} & \cellcolor{TealBlue!30}{1} & \cellcolor{TealBlue!30}{0} & 0.0\\
\texttt{car-un} & \multicolumn{1}{r}{1728} & \multicolumn{1}{r}{21}  & \cellcolor{TealBlue!30}{1} & \cellcolor{TealBlue!30}{0} & 0.4 & \cellcolor{TealBlue!30}{1} & \cellcolor{TealBlue!30}{0} & \cellcolor{TealBlue!30}{\textbf{0.3}} & \cellcolor{TealBlue!30}{1} & \cellcolor{TealBlue!30}{0} & 25.3\\
\texttt{diabetes} & \multicolumn{1}{r}{768} & \multicolumn{1}{r}{112}  & \cellcolor{TealBlue!30}{1} & \cellcolor{TealBlue!30}{0} & 26.1 & \cellcolor{TealBlue!30}{1} & \cellcolor{TealBlue!30}{0} & \cellcolor{TealBlue!30}{\textbf{12.8}} & 0 & 7 & 3580.0\\
\texttt{forest-fires-un} & \multicolumn{1}{r}{517} & \multicolumn{1}{r}{989}  & \cellcolor{TealBlue!30}{0} & 123 & 1370.0 & \cellcolor{TealBlue!30}{0} & \cellcolor{TealBlue!30}{\textbf{118}} & 3370.0 & \cellcolor{TealBlue!30}{0} & 136 & \cellcolor{TealBlue!30}{\textbf{179.0}}\\
\texttt{german-credit} & \multicolumn{1}{r}{1000} & \multicolumn{1}{r}{110}  & \cellcolor{TealBlue!30}{1} & \cellcolor{TealBlue!30}{0} & 296.0 & \cellcolor{TealBlue!30}{1} & \cellcolor{TealBlue!30}{0} & \cellcolor{TealBlue!30}{\textbf{171.0}} & 0 & 100 & 661.0\\
\texttt{heart-cleveland} & \multicolumn{1}{r}{296} & \multicolumn{1}{r}{50}  & \cellcolor{TealBlue!30}{1} & \cellcolor{TealBlue!30}{0} & \cellcolor{TealBlue!30}{\textbf{0.0}} & \cellcolor{TealBlue!30}{1} & \cellcolor{TealBlue!30}{0} & 0.0 & \cellcolor{TealBlue!30}{1} & \cellcolor{TealBlue!30}{0} & 0.1\\
\texttt{hepatitis} & \multicolumn{1}{r}{137} & \multicolumn{1}{r}{68}  & \cellcolor{TealBlue!30}{1} & \cellcolor{TealBlue!30}{0} & 0.0 & \cellcolor{TealBlue!30}{1} & \cellcolor{TealBlue!30}{0} & \cellcolor{TealBlue!30}{\textbf{0.0}} & \cellcolor{TealBlue!30}{1} & \cellcolor{TealBlue!30}{0} & 0.0\\
\texttt{hypothyroid} & \multicolumn{1}{r}{3247} & \multicolumn{1}{r}{43}  & \cellcolor{TealBlue!30}{0} & \cellcolor{TealBlue!30}{32} & 486.0 & \cellcolor{TealBlue!30}{0} & \cellcolor{TealBlue!30}{32} & \cellcolor{TealBlue!30}{\textbf{34.8}} & \cellcolor{TealBlue!30}{0} & 44 & 126.0\\
\texttt{ionosphere} & \multicolumn{1}{r}{351} & \multicolumn{1}{r}{444}  & \cellcolor{TealBlue!30}{1} & \cellcolor{TealBlue!30}{0} & 0.1 & \cellcolor{TealBlue!30}{1} & \cellcolor{TealBlue!30}{0} & 0.0 & \cellcolor{TealBlue!30}{1} & \cellcolor{TealBlue!30}{0} & \cellcolor{TealBlue!30}{\textbf{0.0}}\\
\texttt{kr-vs-kp} & \multicolumn{1}{r}{3196} & \multicolumn{1}{r}{37}  & \cellcolor{TealBlue!30}{0} & \cellcolor{TealBlue!30}{2} & 1140.0 & \cellcolor{TealBlue!30}{0} & \cellcolor{TealBlue!30}{2} & \cellcolor{TealBlue!30}{\textbf{954.0}} & \cellcolor{TealBlue!30}{0} & 71 & 1520.0\\
\texttt{letter} & \multicolumn{1}{r}{20000} & \multicolumn{1}{r}{224}  & 0 & 20 & 2300.0 & \cellcolor{TealBlue!30}{\textbf{1}} & \cellcolor{TealBlue!30}{\textbf{0}} & \cellcolor{TealBlue!30}{\textbf{1600.0}} & 0 & 204 & 3560.0\\
\texttt{lymph} & \multicolumn{1}{r}{148} & \multicolumn{1}{r}{41}  & \cellcolor{TealBlue!30}{1} & \cellcolor{TealBlue!30}{0} & 0.0 & \cellcolor{TealBlue!30}{1} & \cellcolor{TealBlue!30}{0} & \cellcolor{TealBlue!30}{\textbf{0.0}} & \cellcolor{TealBlue!30}{1} & \cellcolor{TealBlue!30}{0} & 0.0\\
\texttt{mushroom} & \multicolumn{1}{r}{8124} & \multicolumn{1}{r}{91}  & \cellcolor{TealBlue!30}{1} & \cellcolor{TealBlue!30}{0} & 0.0 & \cellcolor{TealBlue!30}{1} & \cellcolor{TealBlue!30}{0} & \cellcolor{TealBlue!30}{\textbf{0.0}} & \cellcolor{TealBlue!30}{1} & \cellcolor{TealBlue!30}{0} & 0.0\\
\texttt{pendigits} & \multicolumn{1}{r}{7494} & \multicolumn{1}{r}{216}  & \cellcolor{TealBlue!30}{1} & \cellcolor{TealBlue!30}{0} & 0.1 & \cellcolor{TealBlue!30}{1} & \cellcolor{TealBlue!30}{0} & \cellcolor{TealBlue!30}{\textbf{0.1}} & \cellcolor{TealBlue!30}{1} & \cellcolor{TealBlue!30}{0} & 0.4\\
\texttt{primary-tumor} & \multicolumn{1}{r}{336} & \multicolumn{1}{r}{16}  & \cellcolor{TealBlue!30}{0} & \cellcolor{TealBlue!30}{15} & 311.0 & \cellcolor{TealBlue!30}{0} & \cellcolor{TealBlue!30}{15} & \cellcolor{TealBlue!30}{\textbf{126.0}} & \cellcolor{TealBlue!30}{0} & \cellcolor{TealBlue!30}{15} & 3180.0\\
\texttt{segment} & \multicolumn{1}{r}{2310} & \multicolumn{1}{r}{234}  & \cellcolor{TealBlue!30}{1} & \cellcolor{TealBlue!30}{0} & 0.0 & \cellcolor{TealBlue!30}{1} & \cellcolor{TealBlue!30}{0} & 0.0 & \cellcolor{TealBlue!30}{1} & \cellcolor{TealBlue!30}{0} & \cellcolor{TealBlue!30}{\textbf{0.0}}\\
\texttt{soybean} & \multicolumn{1}{r}{630} & \multicolumn{1}{r}{34}  & \cellcolor{TealBlue!30}{0} & \cellcolor{TealBlue!30}{2} & \cellcolor{TealBlue!30}{\textbf{11.7}} & \cellcolor{TealBlue!30}{0} & \cellcolor{TealBlue!30}{2} & 229.0 & \cellcolor{TealBlue!30}{0} & \cellcolor{TealBlue!30}{2} & 3470.0\\
\texttt{splice-1} & \multicolumn{1}{r}{3190} & \multicolumn{1}{r}{227}  & \cellcolor{TealBlue!30}{0} & \cellcolor{TealBlue!30}{5} & 359.0 & \cellcolor{TealBlue!30}{0} & \cellcolor{TealBlue!30}{5} & 1420.0 & \cellcolor{TealBlue!30}{0} & 26 & \cellcolor{TealBlue!30}{\textbf{290.0}}\\
\texttt{taiwan\_binarised} & \multicolumn{1}{r}{30000} & \multicolumn{1}{r}{198}  & \cellcolor{TealBlue!30}{0} & 4648 & \cellcolor{TealBlue!30}{\textbf{157.0}} & \cellcolor{TealBlue!30}{0} & \cellcolor{TealBlue!30}{\textbf{4566}} & 207.0 & \cellcolor{TealBlue!30}{0} & 5074 & 176.0\\
\texttt{tic-tac-toe} & \multicolumn{1}{r}{958} & \multicolumn{1}{r}{18}  & \cellcolor{TealBlue!30}{1} & \cellcolor{TealBlue!30}{0} & \cellcolor{TealBlue!30}{\textbf{0.0}} & \cellcolor{TealBlue!30}{1} & \cellcolor{TealBlue!30}{0} & 0.0 & \cellcolor{TealBlue!30}{1} & \cellcolor{TealBlue!30}{0} & 0.0\\
\texttt{vehicle} & \multicolumn{1}{r}{846} & \multicolumn{1}{r}{252}  & \cellcolor{TealBlue!30}{1} & \cellcolor{TealBlue!30}{0} & 0.0 & \cellcolor{TealBlue!30}{1} & \cellcolor{TealBlue!30}{0} & \cellcolor{TealBlue!30}{\textbf{0.0}} & \cellcolor{TealBlue!30}{1} & \cellcolor{TealBlue!30}{0} & 135.0\\
\texttt{vote} & \multicolumn{1}{r}{435} & \multicolumn{1}{r}{32}  & \cellcolor{TealBlue!30}{1} & \cellcolor{TealBlue!30}{0} & 0.0 & \cellcolor{TealBlue!30}{1} & \cellcolor{TealBlue!30}{0} & \cellcolor{TealBlue!30}{\textbf{0.0}} & \cellcolor{TealBlue!30}{1} & \cellcolor{TealBlue!30}{0} & 0.0\\
\texttt{wine1-un} & \multicolumn{1}{r}{178} & \multicolumn{1}{r}{1276}  & \cellcolor{TealBlue!30}{0} & 26 & 1950.0 & \cellcolor{TealBlue!30}{0} & \cellcolor{TealBlue!30}{22} & 3450.0 & \cellcolor{TealBlue!30}{0} & \cellcolor{TealBlue!30}{22} & \cellcolor{TealBlue!30}{\textbf{25.3}}\\
\texttt{wine2-un} & \multicolumn{1}{r}{178} & \multicolumn{1}{r}{1276}  & \cellcolor{TealBlue!30}{0} & 27 & \cellcolor{TealBlue!30}{\textbf{4.1}} & \cellcolor{TealBlue!30}{0} & 24 & 2700.0 & \cellcolor{TealBlue!30}{0} & \cellcolor{TealBlue!30}{\textbf{21}} & 54.8\\
\texttt{wine3-un} & \multicolumn{1}{r}{178} & \multicolumn{1}{r}{1276}  & \cellcolor{TealBlue!30}{0} & \cellcolor{TealBlue!30}{\textbf{10}} & 2590.0 & \cellcolor{TealBlue!30}{0} & 18 & 1900.0 & \cellcolor{TealBlue!30}{0} & 17 & \cellcolor{TealBlue!30}{\textbf{107.0}}\\
\texttt{yeast} & \multicolumn{1}{r}{1484} & \multicolumn{1}{r}{89}  & \cellcolor{TealBlue!30}{0} & 109 & 3100.0 & \cellcolor{TealBlue!30}{0} & \cellcolor{TealBlue!30}{\textbf{104}} & 2680.0 & \cellcolor{TealBlue!30}{0} & 196 & \cellcolor{TealBlue!30}{\textbf{131.0}}\\
\texttt{zoo-1} & \multicolumn{1}{r}{101} & \multicolumn{1}{r}{20}  & \cellcolor{TealBlue!30}{1} & \cellcolor{TealBlue!30}{0} & 0.0 & \cellcolor{TealBlue!30}{1} & \cellcolor{TealBlue!30}{0} & \cellcolor{TealBlue!30}{\textbf{0.0}} & \cellcolor{TealBlue!30}{1} & \cellcolor{TealBlue!30}{0} & 0.0\\
\bottomrule
\end{tabular}

% \end{normalsize}
% \end{center}
% \caption{\label{tab:ha10} Comparison of heuristics (max depth=10)}
% \end{table}
%
%
%
%
%
% \begin{table}[htbp]
% \begin{center}
% \begin{normalsize}
% \tabcolsep=5pt
% \begin{tabular}{lccrrrrrr}
\toprule
& && \multicolumn{2}{c}{entropy} & \multicolumn{2}{c}{\budalg} & \multicolumn{2}{c}{error}\\
\cmidrule(rr){4-5}\cmidrule(rr){6-7}\cmidrule(rr){8-9}
&\multirow{1}{*}{$\#ex.$} & \multirow{1}{*}{\#feat.} &  \multicolumn{1}{c}{error} & \multicolumn{1}{c}{time} & \multicolumn{1}{c}{error} & \multicolumn{1}{c}{time} & \multicolumn{1}{c}{error} & \multicolumn{1}{c}{time} \\
\midrule

\texttt{anneal} & \multicolumn{1}{r}{812} & \multicolumn{1}{r}{47}  & 140 & \cellcolor{TealBlue!30}{\textbf{0.00}} & 137 & 0.00 & \cellcolor{TealBlue!30}{\textbf{130}} & 0.00\\
\texttt{audiology} & \multicolumn{1}{r}{216} & \multicolumn{1}{r}{79}  & \cellcolor{TealBlue!30}{6} & 0.00 & \cellcolor{TealBlue!30}{6} & 0.00 & \cellcolor{TealBlue!30}{6} & \cellcolor{TealBlue!30}{\textbf{0.00}}\\
\texttt{australian-credit} & \multicolumn{1}{r}{653} & \multicolumn{1}{r}{73}  & 84 & 0.00 & \cellcolor{TealBlue!30}{\textbf{82}} & \cellcolor{TealBlue!30}{\textbf{0.00}} & 86 & 0.00\\
\texttt{breast-cancer-un} & \multicolumn{1}{r}{683} & \multicolumn{1}{r}{89}  & 32 & 0.00 & \cellcolor{TealBlue!30}{28} & \cellcolor{TealBlue!30}{\textbf{0.00}} & \cellcolor{TealBlue!30}{28} & 0.00\\
\texttt{breast-wisconsin} & \multicolumn{1}{r}{683} & \multicolumn{1}{r}{120}  & \cellcolor{TealBlue!30}{23} & 0.00 & 26 & \cellcolor{TealBlue!30}{0.00} & \cellcolor{TealBlue!30}{23} & \cellcolor{TealBlue!30}{0.00}\\
\texttt{car-un} & \multicolumn{1}{r}{1728} & \multicolumn{1}{r}{21}  & \cellcolor{TealBlue!30}{202} & 0.00 & \cellcolor{TealBlue!30}{202} & \cellcolor{TealBlue!30}{\textbf{0.00}} & 300 & 0.00\\
\texttt{diabetes} & \multicolumn{1}{r}{768} & \multicolumn{1}{r}{112}  & \cellcolor{TealBlue!30}{169} & 0.00 & \cellcolor{TealBlue!30}{169} & \cellcolor{TealBlue!30}{\textbf{0.00}} & 183 & 0.00\\
\texttt{forest-fires-un} & \multicolumn{1}{r}{517} & \multicolumn{1}{r}{989}  & 209 & 0.00 & \cellcolor{TealBlue!30}{198} & \cellcolor{TealBlue!30}{\textbf{0.00}} & \cellcolor{TealBlue!30}{198} & 0.00\\
\texttt{german-credit} & \multicolumn{1}{r}{1000} & \multicolumn{1}{r}{110}  & \cellcolor{TealBlue!30}{249} & 0.00 & \cellcolor{TealBlue!30}{249} & \cellcolor{TealBlue!30}{\textbf{0.00}} & 271 & 0.00\\
\texttt{heart-cleveland} & \multicolumn{1}{r}{296} & \multicolumn{1}{r}{50}  & \cellcolor{TealBlue!30}{43} & 0.00 & \cellcolor{TealBlue!30}{43} & \cellcolor{TealBlue!30}{\textbf{0.00}} & 54 & 0.00\\
\texttt{hepatitis} & \multicolumn{1}{r}{137} & \multicolumn{1}{r}{68}  & \cellcolor{TealBlue!30}{14} & \cellcolor{TealBlue!30}{\textbf{0.00}} & \cellcolor{TealBlue!30}{14} & 0.00 & 16 & 0.00\\
\texttt{hypothyroid} & \multicolumn{1}{r}{3247} & \multicolumn{1}{r}{43}  & \cellcolor{TealBlue!30}{62} & 0.00 & \cellcolor{TealBlue!30}{62} & 0.00 & \cellcolor{TealBlue!30}{62} & \cellcolor{TealBlue!30}{\textbf{0.00}}\\
\texttt{ionosphere} & \multicolumn{1}{r}{351} & \multicolumn{1}{r}{444}  & \cellcolor{TealBlue!30}{29} & 0.00 & \cellcolor{TealBlue!30}{29} & 0.00 & \cellcolor{TealBlue!30}{29} & \cellcolor{TealBlue!30}{\textbf{0.00}}\\
\texttt{kr-vs-kp} & \multicolumn{1}{r}{3196} & \multicolumn{1}{r}{37}  & 306 & \cellcolor{TealBlue!30}{0.00} & 306 & 0.00 & \cellcolor{TealBlue!30}{\textbf{198}} & \cellcolor{TealBlue!30}{0.00}\\
\texttt{letter} & \multicolumn{1}{r}{20000} & \multicolumn{1}{r}{224}  & 801 & 0.03 & 657 & \cellcolor{TealBlue!30}{\textbf{0.03}} & \cellcolor{TealBlue!30}{\textbf{598}} & 0.03\\
\texttt{lymph} & \multicolumn{1}{r}{148} & \multicolumn{1}{r}{41}  & \cellcolor{TealBlue!30}{16} & 0.00 & \cellcolor{TealBlue!30}{16} & \cellcolor{TealBlue!30}{\textbf{0.00}} & 20 & 0.00\\
\texttt{mushroom} & \multicolumn{1}{r}{8124} & \multicolumn{1}{r}{91}  & 280 & 0.01 & 280 & \cellcolor{TealBlue!30}{\textbf{0.00}} & \cellcolor{TealBlue!30}{\textbf{24}} & 0.01\\
\texttt{pendigits} & \multicolumn{1}{r}{7494} & \multicolumn{1}{r}{216}  & 79 & 0.01 & \cellcolor{TealBlue!30}{\textbf{51}} & \cellcolor{TealBlue!30}{\textbf{0.01}} & 189 & 0.01\\
\texttt{primary-tumor} & \multicolumn{1}{r}{336} & \multicolumn{1}{r}{16}  & \cellcolor{TealBlue!30}{51} & 0.00 & \cellcolor{TealBlue!30}{51} & \cellcolor{TealBlue!30}{\textbf{0.00}} & 52 & 0.00\\
\texttt{segment} & \multicolumn{1}{r}{2310} & \multicolumn{1}{r}{234}  & \cellcolor{TealBlue!30}{5} & 0.00 & \cellcolor{TealBlue!30}{5} & \cellcolor{TealBlue!30}{\textbf{0.00}} & \cellcolor{TealBlue!30}{5} & 0.00\\
\texttt{soybean} & \multicolumn{1}{r}{630} & \multicolumn{1}{r}{34}  & 77 & 0.00 & \cellcolor{TealBlue!30}{\textbf{47}} & \cellcolor{TealBlue!30}{\textbf{0.00}} & 55 & 0.00\\
\texttt{splice-1} & \multicolumn{1}{r}{3190} & \multicolumn{1}{r}{227}  & \cellcolor{TealBlue!30}{279} & \cellcolor{TealBlue!30}{\textbf{0.00}} & \cellcolor{TealBlue!30}{279} & 0.01 & 340 & 0.01\\
\texttt{taiwan\_binarised} & \multicolumn{1}{r}{30000} & \multicolumn{1}{r}{198}  & 5350 & 0.04 & \cellcolor{TealBlue!30}{\textbf{5333}} & \cellcolor{TealBlue!30}{\textbf{0.04}} & 5342 & 0.05\\
\texttt{tic-tac-toe} & \multicolumn{1}{r}{958} & \multicolumn{1}{r}{18}  & \cellcolor{TealBlue!30}{236} & 0.00 & \cellcolor{TealBlue!30}{236} & \cellcolor{TealBlue!30}{\textbf{0.00}} & 241 & 0.00\\
\texttt{vehicle} & \multicolumn{1}{r}{846} & \multicolumn{1}{r}{252}  & 92 & 0.00 & \cellcolor{TealBlue!30}{\textbf{55}} & \cellcolor{TealBlue!30}{\textbf{0.00}} & 86 & 0.00\\
\texttt{vote} & \multicolumn{1}{r}{435} & \multicolumn{1}{r}{32}  & \cellcolor{TealBlue!30}{14} & 0.00 & \cellcolor{TealBlue!30}{14} & \cellcolor{TealBlue!30}{\textbf{0.00}} & 16 & 0.00\\
\texttt{wine1-un} & \multicolumn{1}{r}{178} & \multicolumn{1}{r}{1276}  & 49 & 0.00 & \cellcolor{TealBlue!30}{45} & \cellcolor{TealBlue!30}{\textbf{0.00}} & \cellcolor{TealBlue!30}{45} & 0.00\\
\texttt{wine2-un} & \multicolumn{1}{r}{178} & \multicolumn{1}{r}{1276}  & 53 & 0.00 & 52 & \cellcolor{TealBlue!30}{\textbf{0.00}} & \cellcolor{TealBlue!30}{\textbf{51}} & 0.00\\
\texttt{wine3-un} & \multicolumn{1}{r}{178} & \multicolumn{1}{r}{1276}  & \cellcolor{TealBlue!30}{35} & 0.00 & \cellcolor{TealBlue!30}{35} & \cellcolor{TealBlue!30}{\textbf{0.00}} & \cellcolor{TealBlue!30}{35} & 0.00\\
\texttt{yeast} & \multicolumn{1}{r}{1484} & \multicolumn{1}{r}{89}  & \cellcolor{TealBlue!30}{417} & 0.00 & \cellcolor{TealBlue!30}{417} & \cellcolor{TealBlue!30}{\textbf{0.00}} & 435 & 0.00\\
\texttt{zoo-1} & \multicolumn{1}{r}{101} & \multicolumn{1}{r}{20}  & \cellcolor{TealBlue!30}{0} & 0.00 & \cellcolor{TealBlue!30}{0} & \cellcolor{TealBlue!30}{\textbf{0.00}} & \cellcolor{TealBlue!30}{0} & 0.00\\
\bottomrule
\end{tabular}

% \end{normalsize}
% \end{center}
% \caption{\label{tab:ha3} Comparison of heuristics (max depth=3)}
% \end{table}
%
% \begin{table}[htbp]
% \begin{center}
% \begin{normalsize}
% \tabcolsep=5pt
% \begin{tabular}{lccrrrrrr}
\toprule
& && \multicolumn{2}{c}{entropy} & \multicolumn{2}{c}{\budalg} & \multicolumn{2}{c}{error}\\
\cmidrule(rr){4-5}\cmidrule(rr){6-7}\cmidrule(rr){8-9}
&\multirow{1}{*}{$\#ex.$} & \multirow{1}{*}{\#feat.} &  \multicolumn{1}{c}{error} & \multicolumn{1}{c}{time} & \multicolumn{1}{c}{error} & \multicolumn{1}{c}{time} & \multicolumn{1}{c}{error} & \multicolumn{1}{c}{time} \\
\midrule

\texttt{anneal} & \multicolumn{1}{r}{812} & \multicolumn{1}{r}{47}  & 138 & 0.00 & 135 & \cellcolor{TealBlue!30}{\textbf{0.00}} & \cellcolor{TealBlue!30}{\textbf{122}} & 0.00\\
\texttt{audiology} & \multicolumn{1}{r}{216} & \multicolumn{1}{r}{79}  & \cellcolor{TealBlue!30}{3} & 0.00 & \cellcolor{TealBlue!30}{3} & \cellcolor{TealBlue!30}{\textbf{0.00}} & \cellcolor{TealBlue!30}{3} & 0.00\\
\texttt{australian-credit} & \multicolumn{1}{r}{653} & \multicolumn{1}{r}{73}  & 75 & \cellcolor{TealBlue!30}{\textbf{0.00}} & \cellcolor{TealBlue!30}{\textbf{73}} & 0.00 & 83 & 0.00\\
\texttt{breast-cancer-un} & \multicolumn{1}{r}{683} & \multicolumn{1}{r}{89}  & 29 & 0.00 & \cellcolor{TealBlue!30}{21} & \cellcolor{TealBlue!30}{\textbf{0.00}} & \cellcolor{TealBlue!30}{21} & 0.00\\
\texttt{breast-wisconsin} & \multicolumn{1}{r}{683} & \multicolumn{1}{r}{120}  & 18 & 0.00 & \cellcolor{TealBlue!30}{16} & 0.00 & \cellcolor{TealBlue!30}{16} & \cellcolor{TealBlue!30}{\textbf{0.00}}\\
\texttt{car-un} & \multicolumn{1}{r}{1728} & \multicolumn{1}{r}{21}  & \cellcolor{TealBlue!30}{178} & 0.00 & \cellcolor{TealBlue!30}{178} & \cellcolor{TealBlue!30}{\textbf{0.00}} & 278 & 0.00\\
\texttt{diabetes} & \multicolumn{1}{r}{768} & \multicolumn{1}{r}{112}  & 161 & 0.00 & \cellcolor{TealBlue!30}{\textbf{159}} & 0.00 & 163 & \cellcolor{TealBlue!30}{\textbf{0.00}}\\
\texttt{forest-fires-un} & \multicolumn{1}{r}{517} & \multicolumn{1}{r}{989}  & 201 & 0.00 & \cellcolor{TealBlue!30}{186} & 0.00 & \cellcolor{TealBlue!30}{186} & \cellcolor{TealBlue!30}{\textbf{0.00}}\\
\texttt{german-credit} & \multicolumn{1}{r}{1000} & \multicolumn{1}{r}{110}  & 225 & 0.00 & \cellcolor{TealBlue!30}{\textbf{224}} & 0.00 & 246 & \cellcolor{TealBlue!30}{\textbf{0.00}}\\
\texttt{heart-cleveland} & \multicolumn{1}{r}{296} & \multicolumn{1}{r}{50}  & 39 & 0.00 & \cellcolor{TealBlue!30}{\textbf{36}} & 0.00 & 45 & \cellcolor{TealBlue!30}{\textbf{0.00}}\\
\texttt{hepatitis} & \multicolumn{1}{r}{137} & \multicolumn{1}{r}{68}  & 13 & 0.00 & \cellcolor{TealBlue!30}{\textbf{12}} & 0.00 & 14 & \cellcolor{TealBlue!30}{\textbf{0.00}}\\
\texttt{hypothyroid} & \multicolumn{1}{r}{3247} & \multicolumn{1}{r}{43}  & 54 & 0.00 & \cellcolor{TealBlue!30}{\textbf{53}} & \cellcolor{TealBlue!30}{\textbf{0.00}} & 55 & 0.00\\
\texttt{ionosphere} & \multicolumn{1}{r}{351} & \multicolumn{1}{r}{444}  & \cellcolor{TealBlue!30}{\textbf{22}} & 0.00 & 25 & \cellcolor{TealBlue!30}{\textbf{0.00}} & 26 & 0.00\\
\texttt{kr-vs-kp} & \multicolumn{1}{r}{3196} & \multicolumn{1}{r}{37}  & 189 & 0.00 & 188 & \cellcolor{TealBlue!30}{\textbf{0.00}} & \cellcolor{TealBlue!30}{\textbf{181}} & 0.00\\
\texttt{letter} & \multicolumn{1}{r}{20000} & \multicolumn{1}{r}{224}  & 732 & 0.03 & \cellcolor{TealBlue!30}{\textbf{443}} & \cellcolor{TealBlue!30}{\textbf{0.03}} & 589 & 0.04\\
\texttt{lymph} & \multicolumn{1}{r}{148} & \multicolumn{1}{r}{41}  & \cellcolor{TealBlue!30}{9} & 0.00 & \cellcolor{TealBlue!30}{9} & 0.00 & 17 & \cellcolor{TealBlue!30}{\textbf{0.00}}\\
\texttt{mushroom} & \multicolumn{1}{r}{8124} & \multicolumn{1}{r}{91}  & \cellcolor{TealBlue!30}{4} & 0.01 & \cellcolor{TealBlue!30}{4} & 0.01 & 16 & \cellcolor{TealBlue!30}{\textbf{0.00}}\\
\texttt{pendigits} & \multicolumn{1}{r}{7494} & \multicolumn{1}{r}{216}  & 27 & \cellcolor{TealBlue!30}{\textbf{0.01}} & \cellcolor{TealBlue!30}{\textbf{22}} & 0.01 & 63 & 0.01\\
\texttt{primary-tumor} & \multicolumn{1}{r}{336} & \multicolumn{1}{r}{16}  & 45 & 0.00 & \cellcolor{TealBlue!30}{\textbf{43}} & 0.00 & 46 & \cellcolor{TealBlue!30}{\textbf{0.00}}\\
\texttt{segment} & \multicolumn{1}{r}{2310} & \multicolumn{1}{r}{234}  & \cellcolor{TealBlue!30}{1} & 0.00 & \cellcolor{TealBlue!30}{1} & \cellcolor{TealBlue!30}{\textbf{0.00}} & \cellcolor{TealBlue!30}{1} & 0.00\\
\texttt{soybean} & \multicolumn{1}{r}{630} & \multicolumn{1}{r}{34}  & 71 & 0.00 & \cellcolor{TealBlue!30}{\textbf{32}} & 0.00 & 42 & \cellcolor{TealBlue!30}{\textbf{0.00}}\\
\texttt{splice-1} & \multicolumn{1}{r}{3190} & \multicolumn{1}{r}{227}  & \cellcolor{TealBlue!30}{141} & 0.01 & \cellcolor{TealBlue!30}{141} & \cellcolor{TealBlue!30}{\textbf{0.01}} & 292 & 0.01\\
\texttt{taiwan\_binarised} & \multicolumn{1}{r}{30000} & \multicolumn{1}{r}{198}  & 5305 & 0.05 & \cellcolor{TealBlue!30}{\textbf{5293}} & 0.04 & 5308 & \cellcolor{TealBlue!30}{\textbf{0.04}}\\
\texttt{tic-tac-toe} & \multicolumn{1}{r}{958} & \multicolumn{1}{r}{18}  & \cellcolor{TealBlue!30}{150} & 0.00 & \cellcolor{TealBlue!30}{150} & 0.00 & 179 & \cellcolor{TealBlue!30}{\textbf{0.00}}\\
\texttt{vehicle} & \multicolumn{1}{r}{846} & \multicolumn{1}{r}{252}  & 41 & 0.00 & \cellcolor{TealBlue!30}{\textbf{28}} & \cellcolor{TealBlue!30}{\textbf{0.00}} & 64 & 0.00\\
\texttt{vote} & \multicolumn{1}{r}{435} & \multicolumn{1}{r}{32}  & \cellcolor{TealBlue!30}{8} & 0.00 & \cellcolor{TealBlue!30}{8} & 0.00 & 12 & \cellcolor{TealBlue!30}{\textbf{0.00}}\\
\texttt{wine1-un} & \multicolumn{1}{r}{178} & \multicolumn{1}{r}{1276}  & 49 & 0.00 & 42 & \cellcolor{TealBlue!30}{\textbf{0.00}} & \cellcolor{TealBlue!30}{\textbf{41}} & 0.00\\
\texttt{wine2-un} & \multicolumn{1}{r}{178} & \multicolumn{1}{r}{1276}  & 48 & 0.00 & \cellcolor{TealBlue!30}{47} & \cellcolor{TealBlue!30}{\textbf{0.00}} & \cellcolor{TealBlue!30}{47} & 0.00\\
\texttt{wine3-un} & \multicolumn{1}{r}{178} & \multicolumn{1}{r}{1276}  & \cellcolor{TealBlue!30}{32} & 0.00 & \cellcolor{TealBlue!30}{32} & \cellcolor{TealBlue!30}{\textbf{0.00}} & \cellcolor{TealBlue!30}{32} & 0.00\\
\texttt{yeast} & \multicolumn{1}{r}{1484} & \multicolumn{1}{r}{89}  & \cellcolor{TealBlue!30}{\textbf{391}} & 0.00 & 392 & 0.00 & 430 & \cellcolor{TealBlue!30}{\textbf{0.00}}\\
\texttt{zoo-1} & \multicolumn{1}{r}{101} & \multicolumn{1}{r}{20}  & \cellcolor{TealBlue!30}{0} & 0.00 & \cellcolor{TealBlue!30}{0} & \cellcolor{TealBlue!30}{\textbf{0.00}} & \cellcolor{TealBlue!30}{0} & 0.00\\
\bottomrule
\end{tabular}

% \end{normalsize}
% \end{center}
% \caption{\label{tab:ha4} Comparison of heuristics (max depth=4)}
% \end{table}
%
% \begin{table}[htbp]
% \begin{center}
% \begin{normalsize}
% \tabcolsep=5pt
% \begin{tabular}{lccrrrrrr}
\toprule
& && \multicolumn{2}{c}{entropy} & \multicolumn{2}{c}{\budalg} & \multicolumn{2}{c}{error}\\
\cmidrule(rr){4-5}\cmidrule(rr){6-7}\cmidrule(rr){8-9}
&\multirow{1}{*}{$\#ex.$} & \multirow{1}{*}{\#feat.} &  \multicolumn{1}{c}{error} & \multicolumn{1}{c}{time} & \multicolumn{1}{c}{error} & \multicolumn{1}{c}{time} & \multicolumn{1}{c}{error} & \multicolumn{1}{c}{time} \\
\midrule

\texttt{anneal} & \multicolumn{1}{r}{812} & \multicolumn{1}{r}{47}  & 125 & 0.00 & \cellcolor{TealBlue!30}{\textbf{114}} & \cellcolor{TealBlue!30}{\textbf{0.00}} & 117 & 0.00\\
\texttt{audiology} & \multicolumn{1}{r}{216} & \multicolumn{1}{r}{79}  & \cellcolor{TealBlue!30}{2} & 0.00 & \cellcolor{TealBlue!30}{2} & \cellcolor{TealBlue!30}{\textbf{0.00}} & \cellcolor{TealBlue!30}{2} & 0.00\\
\texttt{australian-credit} & \multicolumn{1}{r}{653} & \multicolumn{1}{r}{73}  & \cellcolor{TealBlue!30}{\textbf{62}} & 0.00 & 63 & 0.00 & 78 & \cellcolor{TealBlue!30}{\textbf{0.00}}\\
\texttt{breast-cancer-un} & \multicolumn{1}{r}{683} & \multicolumn{1}{r}{89}  & 22 & 0.00 & \cellcolor{TealBlue!30}{\textbf{16}} & 0.00 & 17 & \cellcolor{TealBlue!30}{\textbf{0.00}}\\
\texttt{breast-wisconsin} & \multicolumn{1}{r}{683} & \multicolumn{1}{r}{120}  & \cellcolor{TealBlue!30}{\textbf{10}} & 0.00 & 13 & \cellcolor{TealBlue!30}{\textbf{0.00}} & 15 & 0.00\\
\texttt{car-un} & \multicolumn{1}{r}{1728} & \multicolumn{1}{r}{21}  & \cellcolor{TealBlue!30}{106} & 0.00 & \cellcolor{TealBlue!30}{106} & \cellcolor{TealBlue!30}{\textbf{0.00}} & 202 & 0.00\\
\texttt{diabetes} & \multicolumn{1}{r}{768} & \multicolumn{1}{r}{112}  & 143 & 0.00 & \cellcolor{TealBlue!30}{\textbf{141}} & \cellcolor{TealBlue!30}{\textbf{0.00}} & 145 & 0.00\\
\texttt{forest-fires-un} & \multicolumn{1}{r}{517} & \multicolumn{1}{r}{989}  & 189 & 0.00 & 176 & \cellcolor{TealBlue!30}{\textbf{0.00}} & \cellcolor{TealBlue!30}{\textbf{175}} & 0.00\\
\texttt{german-credit} & \multicolumn{1}{r}{1000} & \multicolumn{1}{r}{110}  & 215 & 0.00 & \cellcolor{TealBlue!30}{\textbf{201}} & 0.00 & 236 & \cellcolor{TealBlue!30}{\textbf{0.00}}\\
\texttt{heart-cleveland} & \multicolumn{1}{r}{296} & \multicolumn{1}{r}{50}  & \cellcolor{TealBlue!30}{\textbf{26}} & 0.00 & 28 & 0.00 & 36 & \cellcolor{TealBlue!30}{\textbf{0.00}}\\
\texttt{hepatitis} & \multicolumn{1}{r}{137} & \multicolumn{1}{r}{68}  & 10 & 0.00 & \cellcolor{TealBlue!30}{8} & 0.00 & \cellcolor{TealBlue!30}{8} & \cellcolor{TealBlue!30}{\textbf{0.00}}\\
\texttt{hypothyroid} & \multicolumn{1}{r}{3247} & \multicolumn{1}{r}{43}  & \cellcolor{TealBlue!30}{51} & 0.00 & \cellcolor{TealBlue!30}{51} & \cellcolor{TealBlue!30}{\textbf{0.00}} & 55 & 0.00\\
\texttt{ionosphere} & \multicolumn{1}{r}{351} & \multicolumn{1}{r}{444}  & \cellcolor{TealBlue!30}{16} & 0.00 & \cellcolor{TealBlue!30}{16} & 0.00 & 24 & \cellcolor{TealBlue!30}{\textbf{0.00}}\\
\texttt{kr-vs-kp} & \multicolumn{1}{r}{3196} & \multicolumn{1}{r}{37}  & \cellcolor{TealBlue!30}{179} & 0.00 & \cellcolor{TealBlue!30}{179} & \cellcolor{TealBlue!30}{\textbf{0.00}} & 180 & 0.00\\
\texttt{letter} & \multicolumn{1}{r}{20000} & \multicolumn{1}{r}{224}  & 478 & 0.04 & \cellcolor{TealBlue!30}{\textbf{335}} & \cellcolor{TealBlue!30}{\textbf{0.03}} & 579 & 0.04\\
\texttt{lymph} & \multicolumn{1}{r}{148} & \multicolumn{1}{r}{41}  & \cellcolor{TealBlue!30}{4} & 0.00 & \cellcolor{TealBlue!30}{4} & \cellcolor{TealBlue!30}{\textbf{0.00}} & 11 & 0.00\\
\texttt{mushroom} & \multicolumn{1}{r}{8124} & \multicolumn{1}{r}{91}  & \cellcolor{TealBlue!30}{3} & 0.01 & \cellcolor{TealBlue!30}{3} & \cellcolor{TealBlue!30}{\textbf{0.01}} & 12 & 0.01\\
\texttt{pendigits} & \multicolumn{1}{r}{7494} & \multicolumn{1}{r}{216}  & 15 & 0.01 & \cellcolor{TealBlue!30}{\textbf{11}} & \cellcolor{TealBlue!30}{\textbf{0.01}} & 49 & 0.01\\
\texttt{primary-tumor} & \multicolumn{1}{r}{336} & \multicolumn{1}{r}{16}  & 37 & \cellcolor{TealBlue!30}{\textbf{0.00}} & \cellcolor{TealBlue!30}{\textbf{34}} & 0.00 & 41 & 0.00\\
\texttt{segment} & \multicolumn{1}{r}{2310} & \multicolumn{1}{r}{234}  & \cellcolor{TealBlue!30}{1} & 0.00 & \cellcolor{TealBlue!30}{1} & \cellcolor{TealBlue!30}{\textbf{0.00}} & \cellcolor{TealBlue!30}{1} & 0.00\\
\texttt{soybean} & \multicolumn{1}{r}{630} & \multicolumn{1}{r}{34}  & 54 & 0.00 & \cellcolor{TealBlue!30}{\textbf{23}} & \cellcolor{TealBlue!30}{\textbf{0.00}} & 31 & 0.00\\
\texttt{splice-1} & \multicolumn{1}{r}{3190} & \multicolumn{1}{r}{227}  & \cellcolor{TealBlue!30}{111} & 0.01 & \cellcolor{TealBlue!30}{111} & \cellcolor{TealBlue!30}{\textbf{0.00}} & 256 & 0.00\\
\texttt{taiwan\_binarised} & \multicolumn{1}{r}{30000} & \multicolumn{1}{r}{198}  & 5274 & 0.05 & \cellcolor{TealBlue!30}{\textbf{5257}} & \cellcolor{TealBlue!30}{\textbf{0.04}} & 5280 & 0.04\\
\texttt{tic-tac-toe} & \multicolumn{1}{r}{958} & \multicolumn{1}{r}{18}  & \cellcolor{TealBlue!30}{78} & \cellcolor{TealBlue!30}{\textbf{0.00}} & \cellcolor{TealBlue!30}{78} & 0.00 & 152 & 0.00\\
\texttt{vehicle} & \multicolumn{1}{r}{846} & \multicolumn{1}{r}{252}  & 22 & 0.00 & \cellcolor{TealBlue!30}{\textbf{21}} & 0.00 & 55 & \cellcolor{TealBlue!30}{\textbf{0.00}}\\
\texttt{vote} & \multicolumn{1}{r}{435} & \multicolumn{1}{r}{32}  & \cellcolor{TealBlue!30}{6} & 0.00 & \cellcolor{TealBlue!30}{6} & 0.00 & 7 & \cellcolor{TealBlue!30}{\textbf{0.00}}\\
\texttt{wine1-un} & \multicolumn{1}{r}{178} & \multicolumn{1}{r}{1276}  & 45 & \cellcolor{TealBlue!30}{\textbf{0.00}} & 39 & 0.00 & \cellcolor{TealBlue!30}{\textbf{38}} & 0.00\\
\texttt{wine2-un} & \multicolumn{1}{r}{178} & \multicolumn{1}{r}{1276}  & 44 & 0.00 & 44 & 0.00 & \cellcolor{TealBlue!30}{\textbf{42}} & \cellcolor{TealBlue!30}{\textbf{0.00}}\\
\texttt{wine3-un} & \multicolumn{1}{r}{178} & \multicolumn{1}{r}{1276}  & 30 & 0.00 & 30 & 0.00 & \cellcolor{TealBlue!30}{\textbf{29}} & \cellcolor{TealBlue!30}{\textbf{0.00}}\\
\texttt{yeast} & \multicolumn{1}{r}{1484} & \multicolumn{1}{r}{89}  & \cellcolor{TealBlue!30}{365} & 0.00 & \cellcolor{TealBlue!30}{365} & 0.00 & 429 & \cellcolor{TealBlue!30}{\textbf{0.00}}\\
\texttt{zoo-1} & \multicolumn{1}{r}{101} & \multicolumn{1}{r}{20}  & \cellcolor{TealBlue!30}{0} & 0.00 & \cellcolor{TealBlue!30}{0} & \cellcolor{TealBlue!30}{\textbf{0.00}} & \cellcolor{TealBlue!30}{0} & 0.00\\
\bottomrule
\end{tabular}

% \end{normalsize}
% \end{center}
% \caption{\label{tab:ha5} Comparison of heuristics (max depth=5)}
% \end{table}
%
% \begin{table}[htbp]
% \begin{center}
% \begin{normalsize}
% \tabcolsep=5pt
% \begin{tabular}{lccrrrrrr}
\toprule
& && \multicolumn{2}{c}{entropy} & \multicolumn{2}{c}{\budalg} & \multicolumn{2}{c}{error}\\
\cmidrule(rr){4-5}\cmidrule(rr){6-7}\cmidrule(rr){8-9}
&\multirow{1}{*}{$\#ex.$} & \multirow{1}{*}{\#feat.} &  \multicolumn{1}{c}{error} & \multicolumn{1}{c}{time} & \multicolumn{1}{c}{error} & \multicolumn{1}{c}{time} & \multicolumn{1}{c}{error} & \multicolumn{1}{c}{time} \\
\midrule

\texttt{anneal} & \multicolumn{1}{r}{812} & \multicolumn{1}{r}{47}  & 104 & 0.00 & \cellcolor{TealBlue!30}{\textbf{94}} & 0.00 & 113 & \cellcolor{TealBlue!30}{\textbf{0.00}}\\
\texttt{audiology} & \multicolumn{1}{r}{216} & \multicolumn{1}{r}{79}  & \cellcolor{TealBlue!30}{0} & 0.00 & \cellcolor{TealBlue!30}{0} & \cellcolor{TealBlue!30}{\textbf{0.00}} & 1 & 0.00\\
\texttt{australian-credit} & \multicolumn{1}{r}{653} & \multicolumn{1}{r}{73}  & \cellcolor{TealBlue!30}{43} & 0.00 & \cellcolor{TealBlue!30}{43} & 0.00 & 71 & \cellcolor{TealBlue!30}{\textbf{0.00}}\\
\texttt{breast-cancer-un} & \multicolumn{1}{r}{683} & \multicolumn{1}{r}{89}  & 12 & 0.00 & \cellcolor{TealBlue!30}{\textbf{8}} & \cellcolor{TealBlue!30}{\textbf{0.00}} & 11 & 0.00\\
\texttt{breast-wisconsin} & \multicolumn{1}{r}{683} & \multicolumn{1}{r}{120}  & \cellcolor{TealBlue!30}{4} & 0.00 & \cellcolor{TealBlue!30}{4} & \cellcolor{TealBlue!30}{\textbf{0.00}} & 13 & 0.00\\
\texttt{car-un} & \multicolumn{1}{r}{1728} & \multicolumn{1}{r}{21}  & \cellcolor{TealBlue!30}{50} & 0.00 & \cellcolor{TealBlue!30}{50} & \cellcolor{TealBlue!30}{\textbf{0.00}} & 87 & 0.00\\
\texttt{diabetes} & \multicolumn{1}{r}{768} & \multicolumn{1}{r}{112}  & 101 & 0.00 & \cellcolor{TealBlue!30}{\textbf{99}} & 0.00 & 115 & \cellcolor{TealBlue!30}{\textbf{0.00}}\\
\texttt{forest-fires-un} & \multicolumn{1}{r}{517} & \multicolumn{1}{r}{989}  & 169 & 0.00 & 161 & \cellcolor{TealBlue!30}{\textbf{0.00}} & \cellcolor{TealBlue!30}{\textbf{159}} & 0.00\\
\texttt{german-credit} & \multicolumn{1}{r}{1000} & \multicolumn{1}{r}{110}  & 152 & 0.00 & \cellcolor{TealBlue!30}{\textbf{141}} & 0.00 & 213 & \cellcolor{TealBlue!30}{\textbf{0.00}}\\
\texttt{heart-cleveland} & \multicolumn{1}{r}{296} & \multicolumn{1}{r}{50}  & 11 & 0.00 & \cellcolor{TealBlue!30}{\textbf{7}} & \cellcolor{TealBlue!30}{\textbf{0.00}} & 24 & 0.00\\
\texttt{hepatitis} & \multicolumn{1}{r}{137} & \multicolumn{1}{r}{68}  & \cellcolor{TealBlue!30}{0} & 0.00 & \cellcolor{TealBlue!30}{0} & 0.00 & 3 & \cellcolor{TealBlue!30}{\textbf{0.00}}\\
\texttt{hypothyroid} & \multicolumn{1}{r}{3247} & \multicolumn{1}{r}{43}  & 45 & 0.00 & \cellcolor{TealBlue!30}{\textbf{43}} & \cellcolor{TealBlue!30}{\textbf{0.00}} & 52 & 0.00\\
\texttt{ionosphere} & \multicolumn{1}{r}{351} & \multicolumn{1}{r}{444}  & \cellcolor{TealBlue!30}{\textbf{6}} & 0.00 & 7 & 0.00 & 9 & \cellcolor{TealBlue!30}{\textbf{0.00}}\\
\texttt{kr-vs-kp} & \multicolumn{1}{r}{3196} & \multicolumn{1}{r}{37}  & \cellcolor{TealBlue!30}{\textbf{101}} & 0.00 & 102 & \cellcolor{TealBlue!30}{\textbf{0.00}} & 166 & 0.00\\
\texttt{letter} & \multicolumn{1}{r}{20000} & \multicolumn{1}{r}{224}  & 211 & 0.03 & \cellcolor{TealBlue!30}{\textbf{143}} & 0.04 & 539 & \cellcolor{TealBlue!30}{\textbf{0.03}}\\
\texttt{lymph} & \multicolumn{1}{r}{148} & \multicolumn{1}{r}{41}  & \cellcolor{TealBlue!30}{0} & \cellcolor{TealBlue!30}{\textbf{0.00}} & \cellcolor{TealBlue!30}{0} & 0.00 & 8 & 0.00\\
\texttt{mushroom} & \multicolumn{1}{r}{8124} & \multicolumn{1}{r}{91}  & \cellcolor{TealBlue!30}{0} & 0.01 & \cellcolor{TealBlue!30}{0} & \cellcolor{TealBlue!30}{\textbf{0.00}} & 11 & 0.01\\
\texttt{pendigits} & \multicolumn{1}{r}{7494} & \multicolumn{1}{r}{216}  & \cellcolor{TealBlue!30}{1} & 0.01 & \cellcolor{TealBlue!30}{1} & \cellcolor{TealBlue!30}{\textbf{0.01}} & 14 & 0.01\\
\texttt{primary-tumor} & \multicolumn{1}{r}{336} & \multicolumn{1}{r}{16}  & 28 & 0.00 & \cellcolor{TealBlue!30}{26} & 0.00 & \cellcolor{TealBlue!30}{26} & \cellcolor{TealBlue!30}{\textbf{0.00}}\\
\texttt{segment} & \multicolumn{1}{r}{2310} & \multicolumn{1}{r}{234}  & \cellcolor{TealBlue!30}{0} & 0.00 & \cellcolor{TealBlue!30}{0} & 0.00 & \cellcolor{TealBlue!30}{0} & \cellcolor{TealBlue!30}{\textbf{0.00}}\\
\texttt{soybean} & \multicolumn{1}{r}{630} & \multicolumn{1}{r}{34}  & 33 & 0.00 & \cellcolor{TealBlue!30}{\textbf{11}} & 0.00 & 29 & \cellcolor{TealBlue!30}{\textbf{0.00}}\\
\texttt{splice-1} & \multicolumn{1}{r}{3190} & \multicolumn{1}{r}{227}  & 60 & 0.01 & \cellcolor{TealBlue!30}{\textbf{58}} & \cellcolor{TealBlue!30}{\textbf{0.00}} & 202 & 0.01\\
\texttt{taiwan\_binarised} & \multicolumn{1}{r}{30000} & \multicolumn{1}{r}{198}  & 5160 & 0.06 & \cellcolor{TealBlue!30}{\textbf{5121}} & \cellcolor{TealBlue!30}{\textbf{0.04}} & 5228 & 0.05\\
\texttt{tic-tac-toe} & \multicolumn{1}{r}{958} & \multicolumn{1}{r}{18}  & \cellcolor{TealBlue!30}{21} & \cellcolor{TealBlue!30}{\textbf{0.00}} & \cellcolor{TealBlue!30}{21} & 0.00 & 80 & 0.00\\
\texttt{vehicle} & \multicolumn{1}{r}{846} & \multicolumn{1}{r}{252}  & 7 & 0.00 & \cellcolor{TealBlue!30}{\textbf{4}} & \cellcolor{TealBlue!30}{0.00} & 44 & \cellcolor{TealBlue!30}{0.00}\\
\texttt{vote} & \multicolumn{1}{r}{435} & \multicolumn{1}{r}{32}  & \cellcolor{TealBlue!30}{2} & 0.00 & \cellcolor{TealBlue!30}{2} & \cellcolor{TealBlue!30}{\textbf{0.00}} & 4 & 0.00\\
\texttt{wine1-un} & \multicolumn{1}{r}{178} & \multicolumn{1}{r}{1276}  & 39 & 0.00 & 33 & \cellcolor{TealBlue!30}{\textbf{0.00}} & \cellcolor{TealBlue!30}{\textbf{30}} & 0.00\\
\texttt{wine2-un} & \multicolumn{1}{r}{178} & \multicolumn{1}{r}{1276}  & 38 & 0.00 & 38 & 0.00 & \cellcolor{TealBlue!30}{\textbf{32}} & \cellcolor{TealBlue!30}{\textbf{0.00}}\\
\texttt{wine3-un} & \multicolumn{1}{r}{178} & \multicolumn{1}{r}{1276}  & \cellcolor{TealBlue!30}{\textbf{24}} & 0.00 & 26 & 0.00 & 25 & \cellcolor{TealBlue!30}{\textbf{0.00}}\\
\texttt{yeast} & \multicolumn{1}{r}{1484} & \multicolumn{1}{r}{89}  & 309 & 0.00 & \cellcolor{TealBlue!30}{\textbf{305}} & 0.00 & 376 & \cellcolor{TealBlue!30}{\textbf{0.00}}\\
\texttt{zoo-1} & \multicolumn{1}{r}{101} & \multicolumn{1}{r}{20}  & \cellcolor{TealBlue!30}{0} & 0.00 & \cellcolor{TealBlue!30}{0} & \cellcolor{TealBlue!30}{\textbf{0.00}} & \cellcolor{TealBlue!30}{0} & 0.00\\
\bottomrule
\end{tabular}

% \end{normalsize}
% \end{center}
% \caption{\label{tab:ha7} Comparison of heuristics (max depth=7)}
% \end{table}
%
% \begin{table}[htbp]
% \begin{center}
% \begin{normalsize}
% \tabcolsep=5pt
% \begin{tabular}{lccrrrrrr}
\toprule
& && \multicolumn{2}{c}{entropy} & \multicolumn{2}{c}{\budalg} & \multicolumn{2}{c}{error}\\
\cmidrule(rr){4-5}\cmidrule(rr){6-7}\cmidrule(rr){8-9}
&\multirow{1}{*}{$\#ex.$} & \multirow{1}{*}{\#feat.} &  \multicolumn{1}{c}{error} & \multicolumn{1}{c}{time} & \multicolumn{1}{c}{error} & \multicolumn{1}{c}{time} & \multicolumn{1}{c}{error} & \multicolumn{1}{c}{time} \\
\midrule

\texttt{anneal} & \multicolumn{1}{r}{812} & \multicolumn{1}{r}{47}  & 76 & 0.00 & \cellcolor{TealBlue!30}{\textbf{58}} & \cellcolor{TealBlue!30}{\textbf{0.00}} & 102 & 0.00\\
\texttt{audiology} & \multicolumn{1}{r}{216} & \multicolumn{1}{r}{79}  & \cellcolor{TealBlue!30}{0} & 0.00 & \cellcolor{TealBlue!30}{0} & 0.00 & \cellcolor{TealBlue!30}{0} & \cellcolor{TealBlue!30}{\textbf{0.00}}\\
\texttt{australian-credit} & \multicolumn{1}{r}{653} & \multicolumn{1}{r}{73}  & \cellcolor{TealBlue!30}{\textbf{12}} & 0.00 & 13 & 0.00 & 47 & \cellcolor{TealBlue!30}{\textbf{0.00}}\\
\texttt{breast-cancer-un} & \multicolumn{1}{r}{683} & \multicolumn{1}{r}{89}  & 6 & 0.00 & \cellcolor{TealBlue!30}{\textbf{0}} & \cellcolor{TealBlue!30}{\textbf{0.00}} & 10 & 0.00\\
\texttt{breast-wisconsin} & \multicolumn{1}{r}{683} & \multicolumn{1}{r}{120}  & 1 & 0.00 & \cellcolor{TealBlue!30}{\textbf{0}} & \cellcolor{TealBlue!30}{\textbf{0.00}} & 8 & 0.00\\
\texttt{car-un} & \multicolumn{1}{r}{1728} & \multicolumn{1}{r}{21}  & \cellcolor{TealBlue!30}{11} & 0.00 & \cellcolor{TealBlue!30}{11} & \cellcolor{TealBlue!30}{\textbf{0.00}} & 43 & 0.00\\
\texttt{diabetes} & \multicolumn{1}{r}{768} & \multicolumn{1}{r}{112}  & \cellcolor{TealBlue!30}{\textbf{38}} & 0.00 & 39 & \cellcolor{TealBlue!30}{\textbf{0.00}} & 79 & 0.00\\
\texttt{forest-fires-un} & \multicolumn{1}{r}{517} & \multicolumn{1}{r}{989}  & 154 & 0.00 & 145 & 0.00 & \cellcolor{TealBlue!30}{\textbf{141}} & \cellcolor{TealBlue!30}{\textbf{0.00}}\\
\texttt{german-credit} & \multicolumn{1}{r}{1000} & \multicolumn{1}{r}{110}  & 85 & 0.00 & \cellcolor{TealBlue!30}{\textbf{64}} & 0.00 & 176 & \cellcolor{TealBlue!30}{\textbf{0.00}}\\
\texttt{heart-cleveland} & \multicolumn{1}{r}{296} & \multicolumn{1}{r}{50}  & 1 & \cellcolor{TealBlue!30}{\textbf{0.00}} & \cellcolor{TealBlue!30}{\textbf{0}} & 0.00 & 13 & 0.00\\
\texttt{hepatitis} & \multicolumn{1}{r}{137} & \multicolumn{1}{r}{68}  & \cellcolor{TealBlue!30}{0} & 0.00 & \cellcolor{TealBlue!30}{0} & \cellcolor{TealBlue!30}{\textbf{0.00}} & 1 & 0.00\\
\texttt{hypothyroid} & \multicolumn{1}{r}{3247} & \multicolumn{1}{r}{43}  & \cellcolor{TealBlue!30}{32} & 0.00 & \cellcolor{TealBlue!30}{32} & \cellcolor{TealBlue!30}{\textbf{0.00}} & 45 & 0.00\\
\texttt{ionosphere} & \multicolumn{1}{r}{351} & \multicolumn{1}{r}{444}  & \cellcolor{TealBlue!30}{0} & \cellcolor{TealBlue!30}{\textbf{0.00}} & \cellcolor{TealBlue!30}{0} & 0.00 & 3 & 0.00\\
\texttt{kr-vs-kp} & \multicolumn{1}{r}{3196} & \multicolumn{1}{r}{37}  & \cellcolor{TealBlue!30}{12} & 0.00 & \cellcolor{TealBlue!30}{12} & \cellcolor{TealBlue!30}{\textbf{0.00}} & 134 & 0.00\\
\texttt{letter} & \multicolumn{1}{r}{20000} & \multicolumn{1}{r}{224}  & 124 & 0.04 & \cellcolor{TealBlue!30}{\textbf{20}} & \cellcolor{TealBlue!30}{\textbf{0.03}} & 305 & 0.03\\
\texttt{lymph} & \multicolumn{1}{r}{148} & \multicolumn{1}{r}{41}  & \cellcolor{TealBlue!30}{0} & 0.00 & \cellcolor{TealBlue!30}{0} & \cellcolor{TealBlue!30}{\textbf{0.00}} & 1 & 0.00\\
\texttt{mushroom} & \multicolumn{1}{r}{8124} & \multicolumn{1}{r}{91}  & \cellcolor{TealBlue!30}{0} & 0.01 & \cellcolor{TealBlue!30}{0} & \cellcolor{TealBlue!30}{\textbf{0.01}} & 3 & 0.01\\
\texttt{pendigits} & \multicolumn{1}{r}{7494} & \multicolumn{1}{r}{216}  & \cellcolor{TealBlue!30}{0} & 0.01 & \cellcolor{TealBlue!30}{0} & \cellcolor{TealBlue!30}{\textbf{0.01}} & 13 & 0.01\\
\texttt{primary-tumor} & \multicolumn{1}{r}{336} & \multicolumn{1}{r}{16}  & \cellcolor{TealBlue!30}{\textbf{19}} & 0.00 & 20 & 0.00 & 20 & \cellcolor{TealBlue!30}{\textbf{0.00}}\\
\texttt{segment} & \multicolumn{1}{r}{2310} & \multicolumn{1}{r}{234}  & \cellcolor{TealBlue!30}{0} & 0.00 & \cellcolor{TealBlue!30}{0} & 0.00 & \cellcolor{TealBlue!30}{0} & \cellcolor{TealBlue!30}{\textbf{0.00}}\\
\texttt{soybean} & \multicolumn{1}{r}{630} & \multicolumn{1}{r}{34}  & 7 & 0.00 & \cellcolor{TealBlue!30}{\textbf{2}} & 0.00 & 13 & \cellcolor{TealBlue!30}{\textbf{0.00}}\\
\texttt{splice-1} & \multicolumn{1}{r}{3190} & \multicolumn{1}{r}{227}  & 13 & \cellcolor{TealBlue!30}{\textbf{0.01}} & \cellcolor{TealBlue!30}{\textbf{12}} & 0.01 & 128 & 0.01\\
\texttt{taiwan\_binarised} & \multicolumn{1}{r}{30000} & \multicolumn{1}{r}{198}  & 4779 & 0.07 & \cellcolor{TealBlue!30}{\textbf{4668}} & 0.05 & 5152 & \cellcolor{TealBlue!30}{\textbf{0.04}}\\
\texttt{tic-tac-toe} & \multicolumn{1}{r}{958} & \multicolumn{1}{r}{18}  & \cellcolor{TealBlue!30}{6} & 0.00 & \cellcolor{TealBlue!30}{6} & \cellcolor{TealBlue!30}{\textbf{0.00}} & 14 & 0.00\\
\texttt{vehicle} & \multicolumn{1}{r}{846} & \multicolumn{1}{r}{252}  & \cellcolor{TealBlue!30}{0} & 0.00 & \cellcolor{TealBlue!30}{0} & \cellcolor{TealBlue!30}{\textbf{0.00}} & 14 & 0.00\\
\texttt{vote} & \multicolumn{1}{r}{435} & \multicolumn{1}{r}{32}  & \cellcolor{TealBlue!30}{0} & 0.00 & \cellcolor{TealBlue!30}{0} & \cellcolor{TealBlue!30}{\textbf{0.00}} & 3 & 0.00\\
\texttt{wine1-un} & \multicolumn{1}{r}{178} & \multicolumn{1}{r}{1276}  & 30 & 0.00 & 25 & 0.00 & \cellcolor{TealBlue!30}{\textbf{24}} & \cellcolor{TealBlue!30}{\textbf{0.00}}\\
\texttt{wine2-un} & \multicolumn{1}{r}{178} & \multicolumn{1}{r}{1276}  & 29 & 0.00 & 29 & 0.00 & \cellcolor{TealBlue!30}{\textbf{23}} & \cellcolor{TealBlue!30}{\textbf{0.00}}\\
\texttt{wine3-un} & \multicolumn{1}{r}{178} & \multicolumn{1}{r}{1276}  & \cellcolor{TealBlue!30}{\textbf{15}} & \cellcolor{TealBlue!30}{\textbf{0.00}} & 20 & 0.00 & 19 & 0.00\\
\texttt{yeast} & \multicolumn{1}{r}{1484} & \multicolumn{1}{r}{89}  & 192 & 0.00 & \cellcolor{TealBlue!30}{\textbf{180}} & 0.00 & 274 & \cellcolor{TealBlue!30}{\textbf{0.00}}\\
\texttt{zoo-1} & \multicolumn{1}{r}{101} & \multicolumn{1}{r}{20}  & \cellcolor{TealBlue!30}{0} & 0.00 & \cellcolor{TealBlue!30}{0} & \cellcolor{TealBlue!30}{\textbf{0.00}} & \cellcolor{TealBlue!30}{0} & 0.00\\
\bottomrule
\end{tabular}

% \end{normalsize}
% \end{center}
% \caption{\label{tab:ha10} Comparison of heuristics (max depth=10)}
% \end{table}


% \begin{table}[htbp]
% \begin{center}
% \begin{footnotesize}
% \tabcolsep=5pt
% \begin{tabular}{lccrrrrrrrrrrrr}
\toprule
& && \multicolumn{6}{c}{dt no restart} & \multicolumn{6}{c}{dt restarts (1.1)}\\
\cmidrule(rr){4-9}\cmidrule(rr){10-15}
&\multirow{1}{*}{$\#ex.$} & \multirow{1}{*}{\#feat.} &  \multicolumn{1}{c}{opt} & \multicolumn{1}{c}{error} & \multicolumn{1}{c}{acc.} & \multicolumn{1}{c}{size} & \multicolumn{1}{c}{time} & \multicolumn{1}{c}{choices} & \multicolumn{1}{c}{opt} & \multicolumn{1}{c}{error} & \multicolumn{1}{c}{acc.} & \multicolumn{1}{c}{size} & \multicolumn{1}{c}{time} & \multicolumn{1}{c}{choices} \\
\midrule

\texttt{anneal} & \multicolumn{1}{r}{812} & \multicolumn{1}{r}{88}  & \cellcolor{TealBlue!30}{1.0} & \cellcolor{TealBlue!30}{70.0} & \cellcolor{TealBlue!30}{0.914} & \cellcolor{TealBlue!30}{9.0} & \cellcolor{TealBlue!30}{\textbf{1034.7}} & \cellcolor{TealBlue!30}{\textbf{170{\sc m}}} & \cellcolor{TealBlue!30}{1.0} & \cellcolor{TealBlue!30}{70.0} & \cellcolor{TealBlue!30}{0.914} & \cellcolor{TealBlue!30}{9.0} & 1324.0 & 216{\sc m}\\
\texttt{audiology} & \multicolumn{1}{r}{216} & \multicolumn{1}{r}{145}  & \cellcolor{TealBlue!30}{0.0} & \cellcolor{TealBlue!30}{0.0} & \cellcolor{TealBlue!30}{1.000} & \cellcolor{TealBlue!30}{6.0} & \cellcolor{TealBlue!30}{\textbf{7.4}} & \cellcolor{TealBlue!30}{\textbf{1510{\sc k}}} & \cellcolor{TealBlue!30}{0.0} & \cellcolor{TealBlue!30}{0.0} & \cellcolor{TealBlue!30}{1.000} & \cellcolor{TealBlue!30}{6.0} & 139.5 & 29{\sc m}\\
\texttt{australian-credit} & \multicolumn{1}{r}{653} & \multicolumn{1}{r}{124}  & \cellcolor{TealBlue!30}{0.0} & \cellcolor{TealBlue!30}{40.0} & \cellcolor{TealBlue!30}{0.939} & \cellcolor{TealBlue!30}{8.0} & \cellcolor{TealBlue!30}{\textbf{60.2}} & \cellcolor{TealBlue!30}{\textbf{10{\sc m}}} & \cellcolor{TealBlue!30}{0.0} & \cellcolor{TealBlue!30}{40.0} & \cellcolor{TealBlue!30}{0.939} & \cellcolor{TealBlue!30}{8.0} & 683.3 & 117{\sc m}\\
\texttt{breast-cancer} & \multicolumn{1}{r}{683} & \multicolumn{1}{r}{89}  & \cellcolor{TealBlue!30}{1.0} & \cellcolor{TealBlue!30}{6.0} & \cellcolor{TealBlue!30}{0.991} & \cellcolor{TealBlue!30}{9.0} & \cellcolor{TealBlue!30}{\textbf{815.8}} & \cellcolor{TealBlue!30}{\textbf{158{\sc m}}} & \cellcolor{TealBlue!30}{1.0} & \cellcolor{TealBlue!30}{6.0} & \cellcolor{TealBlue!30}{0.991} & \cellcolor{TealBlue!30}{9.0} & 1022.2 & 202{\sc m}\\
\texttt{car} & \multicolumn{1}{r}{1728} & \multicolumn{1}{r}{21}  & \cellcolor{TealBlue!30}{1.0} & \cellcolor{TealBlue!30}{86.0} & \cellcolor{TealBlue!30}{0.950} & \cellcolor{TealBlue!30}{9.0} & \cellcolor{TealBlue!30}{\textbf{4.7}} & \cellcolor{TealBlue!30}{\textbf{1255{\sc k}}} & \cellcolor{TealBlue!30}{1.0} & \cellcolor{TealBlue!30}{86.0} & \cellcolor{TealBlue!30}{0.950} & \cellcolor{TealBlue!30}{9.0} & 8.8 & 2286{\sc k}\\
\texttt{forest-fires} & \multicolumn{1}{r}{517} & \multicolumn{1}{r}{989}  & \cellcolor{TealBlue!30}{0.0} & \cellcolor{TealBlue!30}{\textbf{156.6}} & \cellcolor{TealBlue!30}{\textbf{0.697}} & \cellcolor{TealBlue!30}{\textbf{9.0}} & 933.6 & 45{\sc m} & \cellcolor{TealBlue!30}{0.0} & 162.1 & 0.686 & 13.6 & \cellcolor{TealBlue!30}{\textbf{533.4}} & \cellcolor{TealBlue!30}{\textbf{26{\sc m}}}\\
\texttt{heart-cleveland} & \multicolumn{1}{r}{296} & \multicolumn{1}{r}{95}  & \cellcolor{TealBlue!30}{\textbf{0.1}} & \cellcolor{TealBlue!30}{7.0} & \cellcolor{TealBlue!30}{0.976} & \cellcolor{TealBlue!30}{9.0} & \cellcolor{TealBlue!30}{\textbf{279.9}} & \cellcolor{TealBlue!30}{\textbf{63{\sc m}}} & 0.0 & \cellcolor{TealBlue!30}{7.0} & \cellcolor{TealBlue!30}{0.976} & \cellcolor{TealBlue!30}{9.0} & 389.5 & 89{\sc m}\\
\texttt{hypothyroid} & \multicolumn{1}{r}{3247} & \multicolumn{1}{r}{83}  & \cellcolor{TealBlue!30}{0.0} & \cellcolor{TealBlue!30}{\textbf{44.0}} & \cellcolor{TealBlue!30}{\textbf{0.986}} & \cellcolor{TealBlue!30}{9.0} & 1519.0 & 103{\sc m} & \cellcolor{TealBlue!30}{0.0} & 45.0 & 0.986 & \cellcolor{TealBlue!30}{9.0} & \cellcolor{TealBlue!30}{\textbf{315.6}} & \cellcolor{TealBlue!30}{\textbf{16{\sc m}}}\\
\texttt{kr-vs-kp} & \multicolumn{1}{r}{3196} & \multicolumn{1}{r}{73}  & \cellcolor{TealBlue!30}{\textbf{0.2}} & \cellcolor{TealBlue!30}{81.0} & \cellcolor{TealBlue!30}{0.975} & \cellcolor{TealBlue!30}{7.0} & 402.1 & 31{\sc m} & 0.0 & \cellcolor{TealBlue!30}{81.0} & \cellcolor{TealBlue!30}{0.975} & \cellcolor{TealBlue!30}{7.0} & \cellcolor{TealBlue!30}{\textbf{243.4}} & \cellcolor{TealBlue!30}{\textbf{18{\sc m}}}\\
\texttt{lymph} & \multicolumn{1}{r}{148} & \multicolumn{1}{r}{68}  & \cellcolor{TealBlue!30}{1.0} & \cellcolor{TealBlue!30}{0.0} & \cellcolor{TealBlue!30}{1.000} & \cellcolor{TealBlue!30}{6.0} & \cellcolor{TealBlue!30}{\textbf{237.9}} & \cellcolor{TealBlue!30}{\textbf{79{\sc m}}} & \cellcolor{TealBlue!30}{1.0} & \cellcolor{TealBlue!30}{0.0} & \cellcolor{TealBlue!30}{1.000} & \cellcolor{TealBlue!30}{6.0} & 302.4 & 102{\sc m}\\
\texttt{mushroom} & \multicolumn{1}{r}{8124} & \multicolumn{1}{r}{111}  & \cellcolor{TealBlue!30}{0.0} & \cellcolor{TealBlue!30}{0.0} & \cellcolor{TealBlue!30}{1.000} & \cellcolor{TealBlue!30}{4.0} & \cellcolor{TealBlue!30}{\textbf{76.0}} & \cellcolor{TealBlue!30}{\textbf{2059{\sc k}}} & \cellcolor{TealBlue!30}{0.0} & \cellcolor{TealBlue!30}{0.0} & \cellcolor{TealBlue!30}{1.000} & \cellcolor{TealBlue!30}{4.0} & 1344.4 & 30{\sc m}\\
\texttt{primary-tumor} & \multicolumn{1}{r}{336} & \multicolumn{1}{r}{31}  & \cellcolor{TealBlue!30}{1.0} & \cellcolor{TealBlue!30}{26.0} & \cellcolor{TealBlue!30}{0.923} & \cellcolor{TealBlue!30}{9.0} & \cellcolor{TealBlue!30}{\textbf{9.7}} & \cellcolor{TealBlue!30}{\textbf{4936{\sc k}}} & \cellcolor{TealBlue!30}{1.0} & \cellcolor{TealBlue!30}{26.0} & \cellcolor{TealBlue!30}{0.923} & \cellcolor{TealBlue!30}{9.0} & 15.1 & 7745{\sc k}\\
\texttt{soybean} & \multicolumn{1}{r}{630} & \multicolumn{1}{r}{50}  & \cellcolor{TealBlue!30}{1.0} & \cellcolor{TealBlue!30}{8.0} & \cellcolor{TealBlue!30}{0.987} & \cellcolor{TealBlue!30}{8.0} & \cellcolor{TealBlue!30}{\textbf{65.0}} & \cellcolor{TealBlue!30}{\textbf{19{\sc m}}} & \cellcolor{TealBlue!30}{1.0} & \cellcolor{TealBlue!30}{8.0} & \cellcolor{TealBlue!30}{0.987} & \cellcolor{TealBlue!30}{8.0} & 87.8 & 26{\sc m}\\
\texttt{splice-1} & \multicolumn{1}{r}{3190} & \multicolumn{1}{r}{287}  & \cellcolor{TealBlue!30}{0.0} & 103.8 & 0.967 & 12.5 & 1437.2 & 93{\sc m} & \cellcolor{TealBlue!30}{0.0} & \cellcolor{TealBlue!30}{\textbf{103.4}} & \cellcolor{TealBlue!30}{\textbf{0.968}} & \cellcolor{TealBlue!30}{\textbf{11.8}} & \cellcolor{TealBlue!30}{\textbf{424.3}} & \cellcolor{TealBlue!30}{\textbf{21{\sc m}}}\\
\texttt{tic-tac-toe} & \multicolumn{1}{r}{958} & \multicolumn{1}{r}{27}  & \cellcolor{TealBlue!30}{1.0} & \cellcolor{TealBlue!30}{63.0} & \cellcolor{TealBlue!30}{0.934} & 8.5 & \cellcolor{TealBlue!30}{\textbf{13.2}} & \cellcolor{TealBlue!30}{\textbf{4990{\sc k}}} & \cellcolor{TealBlue!30}{1.0} & \cellcolor{TealBlue!30}{63.0} & \cellcolor{TealBlue!30}{0.934} & \cellcolor{TealBlue!30}{\textbf{8.2}} & 20.5 & 7985{\sc k}\\
\texttt{vote} & \multicolumn{1}{r}{435} & \multicolumn{1}{r}{48}  & \cellcolor{TealBlue!30}{1.0} & \cellcolor{TealBlue!30}{1.0} & \cellcolor{TealBlue!30}{0.998} & \cellcolor{TealBlue!30}{8.0} & \cellcolor{TealBlue!30}{\textbf{44.0}} & \cellcolor{TealBlue!30}{\textbf{15{\sc m}}} & \cellcolor{TealBlue!30}{1.0} & \cellcolor{TealBlue!30}{1.0} & \cellcolor{TealBlue!30}{0.998} & \cellcolor{TealBlue!30}{8.0} & 58.8 & 21{\sc m}\\
\texttt{wine1} & \multicolumn{1}{r}{178} & \multicolumn{1}{r}{1276}  & \cellcolor{TealBlue!30}{0.0} & \cellcolor{TealBlue!30}{\textbf{34.0}} & \cellcolor{TealBlue!30}{\textbf{0.809}} & 9.6 & 375.3 & 16{\sc m} & \cellcolor{TealBlue!30}{0.0} & 35.0 & 0.803 & \cellcolor{TealBlue!30}{\textbf{8.0}} & \cellcolor{TealBlue!30}{\textbf{262.9}} & \cellcolor{TealBlue!30}{\textbf{11{\sc m}}}\\
\texttt{wine2} & \multicolumn{1}{r}{178} & \multicolumn{1}{r}{1276}  & \cellcolor{TealBlue!30}{0.0} & \cellcolor{TealBlue!30}{37.0} & \cellcolor{TealBlue!30}{0.792} & \cellcolor{TealBlue!30}{9.0} & 56.5 & 2314{\sc k} & \cellcolor{TealBlue!30}{0.0} & \cellcolor{TealBlue!30}{37.0} & \cellcolor{TealBlue!30}{0.792} & \cellcolor{TealBlue!30}{9.0} & \cellcolor{TealBlue!30}{\textbf{2.7}} & \cellcolor{TealBlue!30}{\textbf{102{\sc k}}}\\
\texttt{wine3} & \multicolumn{1}{r}{178} & \multicolumn{1}{r}{1276}  & \cellcolor{TealBlue!30}{0.0} & \cellcolor{TealBlue!30}{\textbf{25.9}} & \cellcolor{TealBlue!30}{\textbf{0.854}} & 8.9 & 251.2 & 11{\sc m} & \cellcolor{TealBlue!30}{0.0} & 26.7 & 0.850 & \cellcolor{TealBlue!30}{\textbf{7.1}} & \cellcolor{TealBlue!30}{\textbf{176.2}} & \cellcolor{TealBlue!30}{\textbf{8446{\sc k}}}\\
\texttt{zoo-1} & \multicolumn{1}{r}{101} & \multicolumn{1}{r}{36}  & \cellcolor{TealBlue!30}{1.0} & \cellcolor{TealBlue!30}{0.0} & \cellcolor{TealBlue!30}{1.000} & \cellcolor{TealBlue!30}{1.0} & 0.0 & \cellcolor{TealBlue!30}{1} & \cellcolor{TealBlue!30}{1.0} & \cellcolor{TealBlue!30}{0.0} & \cellcolor{TealBlue!30}{1.000} & \cellcolor{TealBlue!30}{1.0} & \cellcolor{TealBlue!30}{\textbf{0.0}} & \cellcolor{TealBlue!30}{1}\\
\bottomrule
\end{tabular}

% \end{footnotesize}
% \end{center}
% \caption{\label{tab:thetable} Restarts (max depth=5)}
% \end{table}

% \clearpage

% \begin{table}[htbp]
% \begin{center}
% \begin{footnotesize}
% \tabcolsep=5pt
% \begin{tabular}{lccrrrrrrrrrrrr}
\toprule
& && \multicolumn{6}{c}{dt no restart} & \multicolumn{6}{c}{dt restarts (1.1)}\\
\cmidrule(rr){4-9}\cmidrule(rr){10-15}
&\multirow{1}{*}{$\#ex.$} & \multirow{1}{*}{\#feat.} &  \multicolumn{1}{c}{opt} & \multicolumn{1}{c}{error} & \multicolumn{1}{c}{acc.} & \multicolumn{1}{c}{size} & \multicolumn{1}{c}{time} & \multicolumn{1}{c}{choices} & \multicolumn{1}{c}{opt} & \multicolumn{1}{c}{error} & \multicolumn{1}{c}{acc.} & \multicolumn{1}{c}{size} & \multicolumn{1}{c}{time} & \multicolumn{1}{c}{choices} \\
\midrule

\texttt{anneal} & \multicolumn{1}{r}{812} & \multicolumn{1}{r}{88}  & \cellcolor{TealBlue!30}{0.0} & \cellcolor{TealBlue!30}{\textbf{64.0}} & \cellcolor{TealBlue!30}{\textbf{0.921}} & 12.9 & 1435.5 & 258{\sc m} & \cellcolor{TealBlue!30}{0.0} & 65.0 & 0.920 & \cellcolor{TealBlue!30}{\textbf{12.7}} & \cellcolor{TealBlue!30}{\textbf{826.5}} & \cellcolor{TealBlue!30}{\textbf{136{\sc m}}}\\
\texttt{audiology} & \multicolumn{1}{r}{216} & \multicolumn{1}{r}{145}  & \cellcolor{TealBlue!30}{0.0} & \cellcolor{TealBlue!30}{0.0} & \cellcolor{TealBlue!30}{1.000} & 9.2 & \cellcolor{TealBlue!30}{\textbf{33.6}} & \cellcolor{TealBlue!30}{\textbf{8518{\sc k}}} & \cellcolor{TealBlue!30}{0.0} & \cellcolor{TealBlue!30}{0.0} & \cellcolor{TealBlue!30}{1.000} & \cellcolor{TealBlue!30}{\textbf{9.0}} & 410.9 & 107{\sc m}\\
\texttt{australian-credit} & \multicolumn{1}{r}{653} & \multicolumn{1}{r}{124}  & \cellcolor{TealBlue!30}{0.0} & \cellcolor{TealBlue!30}{\textbf{0.5}} & \cellcolor{TealBlue!30}{\textbf{0.999}} & \cellcolor{TealBlue!30}{\textbf{12.4}} & 1457.1 & 307{\sc m} & \cellcolor{TealBlue!30}{0.0} & 3.1 & 0.995 & 16.5 & \cellcolor{TealBlue!30}{\textbf{711.8}} & \cellcolor{TealBlue!30}{\textbf{139{\sc m}}}\\
\texttt{breast-cancer} & \multicolumn{1}{r}{683} & \multicolumn{1}{r}{89}  & \cellcolor{TealBlue!30}{0.0} & \cellcolor{TealBlue!30}{0.0} & \cellcolor{TealBlue!30}{1.000} & \cellcolor{TealBlue!30}{\textbf{13.1}} & 1084.6 & 330{\sc m} & \cellcolor{TealBlue!30}{0.0} & \cellcolor{TealBlue!30}{0.0} & \cellcolor{TealBlue!30}{1.000} & 15.5 & \cellcolor{TealBlue!30}{\textbf{489.4}} & \cellcolor{TealBlue!30}{\textbf{161{\sc m}}}\\
\texttt{car} & \multicolumn{1}{r}{1728} & \multicolumn{1}{r}{21}  & \cellcolor{TealBlue!30}{0.0} & \cellcolor{TealBlue!30}{\textbf{0.0}} & \cellcolor{TealBlue!30}{\textbf{1.000}} & \cellcolor{TealBlue!30}{\textbf{11.9}} & 934.2 & 542{\sc m} & \cellcolor{TealBlue!30}{0.0} & 2.5 & 0.999 & 13.9 & \cellcolor{TealBlue!30}{\textbf{343.9}} & \cellcolor{TealBlue!30}{\textbf{212{\sc m}}}\\
\texttt{forest-fires} & \multicolumn{1}{r}{517} & \multicolumn{1}{r}{989}  & \cellcolor{TealBlue!30}{0.0} & 145.7 & 0.718 & 26.3 & 742.3 & 38{\sc m} & \cellcolor{TealBlue!30}{0.0} & \cellcolor{TealBlue!30}{\textbf{127.7}} & \cellcolor{TealBlue!30}{\textbf{0.753}} & \cellcolor{TealBlue!30}{\textbf{25.7}} & \cellcolor{TealBlue!30}{\textbf{596.9}} & \cellcolor{TealBlue!30}{\textbf{30{\sc m}}}\\
\texttt{heart-cleveland} & \multicolumn{1}{r}{296} & \multicolumn{1}{r}{95}  & \cellcolor{TealBlue!30}{0.0} & \cellcolor{TealBlue!30}{0.0} & \cellcolor{TealBlue!30}{1.000} & \cellcolor{TealBlue!30}{\textbf{18.3}} & 1489.8 & 416{\sc m} & \cellcolor{TealBlue!30}{0.0} & \cellcolor{TealBlue!30}{0.0} & \cellcolor{TealBlue!30}{1.000} & 25.9 & \cellcolor{TealBlue!30}{\textbf{622.2}} & \cellcolor{TealBlue!30}{\textbf{211{\sc m}}}\\
\texttt{hypothyroid} & \multicolumn{1}{r}{3247} & \multicolumn{1}{r}{83}  & \cellcolor{TealBlue!30}{0.0} & 49.0 & 0.985 & \cellcolor{TealBlue!30}{\textbf{36.5}} & \cellcolor{TealBlue!30}{\textbf{91.5}} & \cellcolor{TealBlue!30}{\textbf{22{\sc m}}} & \cellcolor{TealBlue!30}{0.0} & \cellcolor{TealBlue!30}{\textbf{38.1}} & \cellcolor{TealBlue!30}{\textbf{0.988}} & 41.8 & 259.4 & 23{\sc m}\\
\texttt{kr-vs-kp} & \multicolumn{1}{r}{3196} & \multicolumn{1}{r}{73}  & \cellcolor{TealBlue!30}{0.0} & \cellcolor{TealBlue!30}{\textbf{29.7}} & \cellcolor{TealBlue!30}{\textbf{0.991}} & 17.8 & \cellcolor{TealBlue!30}{\textbf{237.2}} & \cellcolor{TealBlue!30}{\textbf{47{\sc m}}} & \cellcolor{TealBlue!30}{0.0} & 45.9 & 0.986 & \cellcolor{TealBlue!30}{\textbf{17.2}} & 1017.8 & 218{\sc m}\\
\texttt{lymph} & \multicolumn{1}{r}{148} & \multicolumn{1}{r}{68}  & \cellcolor{TealBlue!30}{0.0} & \cellcolor{TealBlue!30}{0.0} & \cellcolor{TealBlue!30}{1.000} & 11.6 & 1526.6 & 640{\sc m} & \cellcolor{TealBlue!30}{0.0} & \cellcolor{TealBlue!30}{0.0} & \cellcolor{TealBlue!30}{1.000} & \cellcolor{TealBlue!30}{\textbf{11.3}} & \cellcolor{TealBlue!30}{\textbf{517.0}} & \cellcolor{TealBlue!30}{\textbf{225{\sc m}}}\\
\texttt{mushroom} & \multicolumn{1}{r}{8124} & \multicolumn{1}{r}{111}  & \cellcolor{TealBlue!30}{0.0} & \cellcolor{TealBlue!30}{0.0} & \cellcolor{TealBlue!30}{1.000} & 8.1 & \cellcolor{TealBlue!30}{\textbf{293.2}} & \cellcolor{TealBlue!30}{\textbf{9653{\sc k}}} & \cellcolor{TealBlue!30}{0.0} & \cellcolor{TealBlue!30}{0.0} & \cellcolor{TealBlue!30}{1.000} & \cellcolor{TealBlue!30}{\textbf{8.0}} & 394.5 & 25{\sc m}\\
\texttt{primary-tumor} & \multicolumn{1}{r}{336} & \multicolumn{1}{r}{31}  & \cellcolor{TealBlue!30}{0.0} & \cellcolor{TealBlue!30}{15.0} & \cellcolor{TealBlue!30}{0.955} & 17.3 & 929.2 & 599{\sc m} & \cellcolor{TealBlue!30}{0.0} & \cellcolor{TealBlue!30}{15.0} & \cellcolor{TealBlue!30}{0.955} & \cellcolor{TealBlue!30}{\textbf{16.6}} & \cellcolor{TealBlue!30}{\textbf{738.1}} & \cellcolor{TealBlue!30}{\textbf{481{\sc m}}}\\
\texttt{soybean} & \multicolumn{1}{r}{630} & \multicolumn{1}{r}{50}  & \cellcolor{TealBlue!30}{0.0} & \cellcolor{TealBlue!30}{2.0} & \cellcolor{TealBlue!30}{0.997} & \cellcolor{TealBlue!30}{\textbf{11.1}} & 879.0 & \cellcolor{TealBlue!30}{\textbf{329{\sc m}}} & \cellcolor{TealBlue!30}{0.0} & \cellcolor{TealBlue!30}{2.0} & \cellcolor{TealBlue!30}{0.997} & 13.6 & \cellcolor{TealBlue!30}{\textbf{848.1}} & 329{\sc m}\\
\texttt{splice-1} & \multicolumn{1}{r}{3190} & \multicolumn{1}{r}{287}  & \cellcolor{TealBlue!30}{0.0} & 85.6 & 0.973 & \cellcolor{TealBlue!30}{\textbf{42.6}} & 1099.9 & 110{\sc m} & \cellcolor{TealBlue!30}{0.0} & \cellcolor{TealBlue!30}{\textbf{45.6}} & \cellcolor{TealBlue!30}{\textbf{0.986}} & 43.7 & \cellcolor{TealBlue!30}{\textbf{516.2}} & \cellcolor{TealBlue!30}{\textbf{48{\sc m}}}\\
\texttt{tic-tac-toe} & \multicolumn{1}{r}{958} & \multicolumn{1}{r}{27}  & \cellcolor{TealBlue!30}{0.0} & \cellcolor{TealBlue!30}{0.0} & \cellcolor{TealBlue!30}{1.000} & \cellcolor{TealBlue!30}{\textbf{19.1}} & \cellcolor{TealBlue!30}{\textbf{306.1}} & \cellcolor{TealBlue!30}{\textbf{219{\sc m}}} & \cellcolor{TealBlue!30}{0.0} & \cellcolor{TealBlue!30}{0.0} & \cellcolor{TealBlue!30}{1.000} & 25.4 & 329.7 & 253{\sc m}\\
\texttt{vote} & \multicolumn{1}{r}{435} & \multicolumn{1}{r}{48}  & \cellcolor{TealBlue!30}{0.0} & \cellcolor{TealBlue!30}{0.0} & \cellcolor{TealBlue!30}{1.000} & \cellcolor{TealBlue!30}{\textbf{12.9}} & \cellcolor{TealBlue!30}{\textbf{330.8}} & \cellcolor{TealBlue!30}{\textbf{202{\sc m}}} & \cellcolor{TealBlue!30}{0.0} & \cellcolor{TealBlue!30}{0.0} & \cellcolor{TealBlue!30}{1.000} & 14.2 & 574.8 & 376{\sc m}\\
\texttt{wine1} & \multicolumn{1}{r}{178} & \multicolumn{1}{r}{1276}  & \cellcolor{TealBlue!30}{0.0} & \cellcolor{TealBlue!30}{26.0} & \cellcolor{TealBlue!30}{0.854} & \cellcolor{TealBlue!30}{12.0} & \cellcolor{TealBlue!30}{\textbf{16.7}} & \cellcolor{TealBlue!30}{\textbf{695{\sc k}}} & \cellcolor{TealBlue!30}{0.0} & \cellcolor{TealBlue!30}{26.0} & \cellcolor{TealBlue!30}{0.854} & \cellcolor{TealBlue!30}{12.0} & 128.7 & 5508{\sc k}\\
\texttt{wine2} & \multicolumn{1}{r}{178} & \multicolumn{1}{r}{1276}  & \cellcolor{TealBlue!30}{0.0} & 27.3 & 0.847 & 13.6 & \cellcolor{TealBlue!30}{\textbf{259.2}} & \cellcolor{TealBlue!30}{\textbf{11{\sc m}}} & \cellcolor{TealBlue!30}{0.0} & \cellcolor{TealBlue!30}{\textbf{26.7}} & \cellcolor{TealBlue!30}{\textbf{0.850}} & \cellcolor{TealBlue!30}{\textbf{13.3}} & 347.5 & 15{\sc m}\\
\texttt{wine3} & \multicolumn{1}{r}{178} & \multicolumn{1}{r}{1276}  & \cellcolor{TealBlue!30}{0.0} & 20.8 & 0.883 & \cellcolor{TealBlue!30}{11.2} & \cellcolor{TealBlue!30}{\textbf{72.4}} & \cellcolor{TealBlue!30}{\textbf{3180{\sc k}}} & \cellcolor{TealBlue!30}{0.0} & \cellcolor{TealBlue!30}{\textbf{19.3}} & \cellcolor{TealBlue!30}{\textbf{0.892}} & \cellcolor{TealBlue!30}{11.2} & 631.9 & 28{\sc m}\\
\texttt{zoo-1} & \multicolumn{1}{r}{101} & \multicolumn{1}{r}{36}  & \cellcolor{TealBlue!30}{1.0} & \cellcolor{TealBlue!30}{0.0} & \cellcolor{TealBlue!30}{1.000} & \cellcolor{TealBlue!30}{1.0} & 0.0 & \cellcolor{TealBlue!30}{1} & \cellcolor{TealBlue!30}{1.0} & \cellcolor{TealBlue!30}{0.0} & \cellcolor{TealBlue!30}{1.000} & \cellcolor{TealBlue!30}{1.0} & \cellcolor{TealBlue!30}{\textbf{0.0}} & \cellcolor{TealBlue!30}{1}\\
\bottomrule
\end{tabular}

% \end{footnotesize}
% \end{center}
% \caption{\label{tab:thetable} Restarts (max depth=8)}
% \end{table}

% \begin{table}[htbp]
% \begin{center}
% \begin{footnotesize}
% \tabcolsep=5pt
% \begin{tabular}{lccrrrrrrrr}
\toprule
& && \multicolumn{4}{c}{\dleight} & \multicolumn{4}{c}{\budalg}\\
\cmidrule(rr){4-7}\cmidrule(rr){8-11}
&\multirow{1}{*}{$\#ex.$} & \multirow{1}{*}{\#feat.} &  \multicolumn{1}{c}{opt} & \multicolumn{1}{c}{error} & \multicolumn{1}{c}{acc.} & \multicolumn{1}{c}{time} & \multicolumn{1}{c}{opt} & \multicolumn{1}{c}{error} & \multicolumn{1}{c}{acc.} & \multicolumn{1}{c}{time} \\
\midrule

\texttt{anneal} & \multicolumn{1}{r}{812} & \multicolumn{1}{r}{47}  & - & - & - & - & \cellcolor{TealBlue!30}{\textbf{0}} & \cellcolor{TealBlue!30}{\textbf{53}} & \cellcolor{TealBlue!30}{\textbf{0.935}} & \cellcolor{TealBlue!30}{\textbf{310.0}}\\
\texttt{audiology} & \multicolumn{1}{r}{216} & \multicolumn{1}{r}{79}  & - & - & - & - & \cellcolor{TealBlue!30}{\textbf{1}} & \cellcolor{TealBlue!30}{\textbf{0}} & \cellcolor{TealBlue!30}{\textbf{1.000}} & \cellcolor{TealBlue!30}{\textbf{0.0}}\\
\texttt{australian-credit} & \multicolumn{1}{r}{653} & \multicolumn{1}{r}{73}  & - & - & - & - & \cellcolor{TealBlue!30}{\textbf{1}} & \cellcolor{TealBlue!30}{\textbf{0}} & \cellcolor{TealBlue!30}{\textbf{1.000}} & \cellcolor{TealBlue!30}{\textbf{0.0}}\\
\texttt{breast-cancer-un} & \multicolumn{1}{r}{683} & \multicolumn{1}{r}{89}  & - & - & - & - & \cellcolor{TealBlue!30}{\textbf{1}} & \cellcolor{TealBlue!30}{\textbf{0}} & \cellcolor{TealBlue!30}{\textbf{1.000}} & \cellcolor{TealBlue!30}{\textbf{0.0}}\\
\texttt{breast-wisconsin} & \multicolumn{1}{r}{683} & \multicolumn{1}{r}{120}  & - & - & - & - & \cellcolor{TealBlue!30}{\textbf{1}} & \cellcolor{TealBlue!30}{\textbf{0}} & \cellcolor{TealBlue!30}{\textbf{1.000}} & \cellcolor{TealBlue!30}{\textbf{0.0}}\\
\texttt{car-un} & \multicolumn{1}{r}{1728} & \multicolumn{1}{r}{21}  & - & - & - & - & \cellcolor{TealBlue!30}{\textbf{1}} & \cellcolor{TealBlue!30}{\textbf{0}} & \cellcolor{TealBlue!30}{\textbf{1.000}} & \cellcolor{TealBlue!30}{\textbf{0.2}}\\
\texttt{diabetes} & \multicolumn{1}{r}{768} & \multicolumn{1}{r}{112}  & - & - & - & - & \cellcolor{TealBlue!30}{\textbf{1}} & \cellcolor{TealBlue!30}{\textbf{0}} & \cellcolor{TealBlue!30}{\textbf{1.000}} & \cellcolor{TealBlue!30}{\textbf{0.5}}\\
\texttt{forest-fires-un} & \multicolumn{1}{r}{517} & \multicolumn{1}{r}{989}  & - & - & - & - & \cellcolor{TealBlue!30}{\textbf{0}} & \cellcolor{TealBlue!30}{\textbf{118}} & \cellcolor{TealBlue!30}{\textbf{0.772}} & \cellcolor{TealBlue!30}{\textbf{830.0}}\\
\texttt{german-credit} & \multicolumn{1}{r}{1000} & \multicolumn{1}{r}{110}  & - & - & - & - & \cellcolor{TealBlue!30}{\textbf{1}} & \cellcolor{TealBlue!30}{\textbf{0}} & \cellcolor{TealBlue!30}{\textbf{1.000}} & \cellcolor{TealBlue!30}{\textbf{67.0}}\\
\texttt{heart-cleveland} & \multicolumn{1}{r}{296} & \multicolumn{1}{r}{50}  & - & - & - & - & \cellcolor{TealBlue!30}{\textbf{1}} & \cellcolor{TealBlue!30}{\textbf{0}} & \cellcolor{TealBlue!30}{\textbf{1.000}} & \cellcolor{TealBlue!30}{\textbf{0.0}}\\
\texttt{hepatitis} & \multicolumn{1}{r}{137} & \multicolumn{1}{r}{68}  & - & - & - & - & \cellcolor{TealBlue!30}{\textbf{1}} & \cellcolor{TealBlue!30}{\textbf{0}} & \cellcolor{TealBlue!30}{\textbf{1.000}} & \cellcolor{TealBlue!30}{\textbf{0.0}}\\
\texttt{hypothyroid} & \multicolumn{1}{r}{3247} & \multicolumn{1}{r}{43}  & - & - & - & - & \cellcolor{TealBlue!30}{\textbf{0}} & \cellcolor{TealBlue!30}{\textbf{31}} & \cellcolor{TealBlue!30}{\textbf{0.990}} & \cellcolor{TealBlue!30}{\textbf{1.4}}\\
\texttt{ionosphere} & \multicolumn{1}{r}{351} & \multicolumn{1}{r}{444}  & - & - & - & - & \cellcolor{TealBlue!30}{\textbf{1}} & \cellcolor{TealBlue!30}{\textbf{0}} & \cellcolor{TealBlue!30}{\textbf{1.000}} & \cellcolor{TealBlue!30}{\textbf{0.0}}\\
\texttt{kr-vs-kp} & \multicolumn{1}{r}{3196} & \multicolumn{1}{r}{37}  & - & - & - & - & \cellcolor{TealBlue!30}{\textbf{0}} & \cellcolor{TealBlue!30}{\textbf{1}} & \cellcolor{TealBlue!30}{\textbf{1.000}} & \cellcolor{TealBlue!30}{\textbf{224.0}}\\
\texttt{letter} & \multicolumn{1}{r}{20000} & \multicolumn{1}{r}{224}  & - & - & - & - & \cellcolor{TealBlue!30}{\textbf{1}} & \cellcolor{TealBlue!30}{\textbf{0}} & \cellcolor{TealBlue!30}{\textbf{1.000}} & \cellcolor{TealBlue!30}{\textbf{58.4}}\\
\texttt{lymph} & \multicolumn{1}{r}{148} & \multicolumn{1}{r}{41}  & - & - & - & - & \cellcolor{TealBlue!30}{\textbf{1}} & \cellcolor{TealBlue!30}{\textbf{0}} & \cellcolor{TealBlue!30}{\textbf{1.000}} & \cellcolor{TealBlue!30}{\textbf{0.0}}\\
\texttt{mushroom} & \multicolumn{1}{r}{8124} & \multicolumn{1}{r}{91}  & - & - & - & - & \cellcolor{TealBlue!30}{\textbf{1}} & \cellcolor{TealBlue!30}{\textbf{0}} & \cellcolor{TealBlue!30}{\textbf{1.000}} & \cellcolor{TealBlue!30}{\textbf{0.0}}\\
\texttt{pendigits} & \multicolumn{1}{r}{7494} & \multicolumn{1}{r}{216}  & - & - & - & - & \cellcolor{TealBlue!30}{\textbf{1}} & \cellcolor{TealBlue!30}{\textbf{0}} & \cellcolor{TealBlue!30}{\textbf{1.000}} & \cellcolor{TealBlue!30}{\textbf{0.0}}\\
\texttt{primary-tumor} & \multicolumn{1}{r}{336} & \multicolumn{1}{r}{16}  & - & - & - & - & \cellcolor{TealBlue!30}{\textbf{0}} & \cellcolor{TealBlue!30}{\textbf{15}} & \cellcolor{TealBlue!30}{\textbf{0.955}} & \cellcolor{TealBlue!30}{\textbf{144.0}}\\
\texttt{segment} & \multicolumn{1}{r}{2310} & \multicolumn{1}{r}{234}  & - & - & - & - & \cellcolor{TealBlue!30}{\textbf{1}} & \cellcolor{TealBlue!30}{\textbf{0}} & \cellcolor{TealBlue!30}{\textbf{1.000}} & \cellcolor{TealBlue!30}{\textbf{0.0}}\\
\texttt{soybean} & \multicolumn{1}{r}{630} & \multicolumn{1}{r}{34}  & - & - & - & - & \cellcolor{TealBlue!30}{\textbf{0}} & \cellcolor{TealBlue!30}{\textbf{2}} & \cellcolor{TealBlue!30}{\textbf{0.997}} & \cellcolor{TealBlue!30}{\textbf{149.0}}\\
\texttt{splice-1} & \multicolumn{1}{r}{3190} & \multicolumn{1}{r}{227}  & - & - & - & - & \cellcolor{TealBlue!30}{\textbf{0}} & \cellcolor{TealBlue!30}{\textbf{7}} & \cellcolor{TealBlue!30}{\textbf{0.998}} & \cellcolor{TealBlue!30}{\textbf{56.1}}\\
\texttt{taiwan\_binarised} & \multicolumn{1}{r}{30000} & \multicolumn{1}{r}{198}  & - & - & - & - & \cellcolor{TealBlue!30}{\textbf{0}} & \cellcolor{TealBlue!30}{\textbf{4564}} & \cellcolor{TealBlue!30}{\textbf{0.848}} & \cellcolor{TealBlue!30}{\textbf{200.0}}\\
\texttt{tic-tac-toe} & \multicolumn{1}{r}{958} & \multicolumn{1}{r}{18}  & - & - & - & - & \cellcolor{TealBlue!30}{\textbf{1}} & \cellcolor{TealBlue!30}{\textbf{0}} & \cellcolor{TealBlue!30}{\textbf{1.000}} & \cellcolor{TealBlue!30}{\textbf{0.0}}\\
\texttt{vehicle} & \multicolumn{1}{r}{846} & \multicolumn{1}{r}{252}  & - & - & - & - & \cellcolor{TealBlue!30}{\textbf{1}} & \cellcolor{TealBlue!30}{\textbf{0}} & \cellcolor{TealBlue!30}{\textbf{1.000}} & \cellcolor{TealBlue!30}{\textbf{0.0}}\\
\texttt{vote} & \multicolumn{1}{r}{435} & \multicolumn{1}{r}{32}  & - & - & - & - & \cellcolor{TealBlue!30}{\textbf{1}} & \cellcolor{TealBlue!30}{\textbf{0}} & \cellcolor{TealBlue!30}{\textbf{1.000}} & \cellcolor{TealBlue!30}{\textbf{0.0}}\\
\texttt{wine1-un} & \multicolumn{1}{r}{178} & \multicolumn{1}{r}{1276}  & - & - & - & - & \cellcolor{TealBlue!30}{\textbf{0}} & \cellcolor{TealBlue!30}{\textbf{22}} & \cellcolor{TealBlue!30}{\textbf{0.876}} & \cellcolor{TealBlue!30}{\textbf{515.0}}\\
\texttt{wine2-un} & \multicolumn{1}{r}{178} & \multicolumn{1}{r}{1276}  & - & - & - & - & \cellcolor{TealBlue!30}{\textbf{0}} & \cellcolor{TealBlue!30}{\textbf{24}} & \cellcolor{TealBlue!30}{\textbf{0.865}} & \cellcolor{TealBlue!30}{\textbf{340.0}}\\
\texttt{wine3-un} & \multicolumn{1}{r}{178} & \multicolumn{1}{r}{1276}  & - & - & - & - & \cellcolor{TealBlue!30}{\textbf{0}} & \cellcolor{TealBlue!30}{\textbf{16}} & \cellcolor{TealBlue!30}{\textbf{0.910}} & \cellcolor{TealBlue!30}{\textbf{260.0}}\\
\texttt{yeast} & \multicolumn{1}{r}{1484} & \multicolumn{1}{r}{89}  & - & - & - & - & \cellcolor{TealBlue!30}{\textbf{0}} & \cellcolor{TealBlue!30}{\textbf{104}} & \cellcolor{TealBlue!30}{\textbf{0.930}} & \cellcolor{TealBlue!30}{\textbf{72.3}}\\
\texttt{zoo-1} & \multicolumn{1}{r}{101} & \multicolumn{1}{r}{20}  & - & - & - & - & \cellcolor{TealBlue!30}{\textbf{1}} & \cellcolor{TealBlue!30}{\textbf{0}} & \cellcolor{TealBlue!30}{\textbf{1.000}} & \cellcolor{TealBlue!30}{\textbf{0.0}}\\
\bottomrule
\end{tabular}

% \end{footnotesize}
% \end{center}
% \caption{\label{tab:thetable} Restarts (max depth=10)}
% \end{table}



\end{document}

