\documentclass{llncs}
\usepackage[usenames,dvipsnames,svgnames,table]{xcolor}%% http://ctan.org/pkg/xcolor
\usepackage[utf8]{inputenc}
\usepackage{xspace}
\usepackage{array}
%\usepackage{amsthm}
\usepackage{amsmath} 
\usepackage{amssymb} 
\usepackage[ruled,vlined]{algorithm2e}
\usepackage{booktabs}
\usepackage{multirow}
\usepackage{url}
\usepackage{tikz}
\usepackage{pgfplots}
\usepackage{fp}
\usepackage{subfig}
\usetikzlibrary{arrows,shadows,fit,calc,positioning,decorations.pathreplacing,matrix,shapes,petri,topaths,fadings,mindmap,backgrounds,shapes.geometric}
% \usepackage{geometry}
\usepackage{xifthen}
\usepackage{rotating}
\usepackage{forest}
\usepackage{relsize}


% \newgeometry{bottom=2cm,top=2cm}

% \newenvironment{proof}{\paragraph{Proof:}}{\hfill$\square$}

\DeclareMathOperator*{\argmax}{arg\,max}
\DeclareMathOperator*{\argmin}{arg\,min}


% \newtheorem{theorem}{Theorem}


\newcommand{\tru}[0]{\texttt{true}}
\newcommand{\fal}[0]{\texttt{false}}
% \newcommand{\tru}[0]{1}
% \newcommand{\fal}[0]{0}


\newcommand\cactus[5]{%%
\begin{tikzpicture}[scale=.46]
	\begin{axis}[scatter/classes={
		a={mark=square*,blue!50,draw=blue!50!black},%%
		b={mark=triangle*,red!50,draw=red!50!black},%%
		c={mark=diamond*,green!50,draw=green!50!black},%%
		d={mark=*,black!50!white,draw=black},%%
		e={mark=pentagon*,orange!50,draw=orange!50!black},%%
		f={mark=10-pointed star,draw=blue!60!black},%%
		g={mark=x,draw=black,thick},%%
		h={mark=+,draw=red!50!black},%%
		i={mark=Mercedes star,draw=green!50!black},%%
		j={mark=square,draw=orange!50!black},%%
		k={mark=diamond,draw=red!50!black},%%
		l={mark=o,draw=blue!50!black},%%
		m={mark=asterisk,draw=magenta!50!black},%%
		n={mark=triangle,draw=green!50!black},%%
		o={mark=pentagon,draw=red!50!black},%%
		p={mark=asterisk,draw=black},%%
		q={mark=Mercedes star,draw=magenta!50!black},%%
		r={mark=pentagon*,blue!50,draw=blue!50!black},%%
		s={mark=x,draw=green!50!black},%%
		t={mark=diamond*,red!50,draw=red!50!black}},%%
		mark size=1.5pt,
		ymode=log,
		%xmode=log,
		ylabel=#2,
		xlabel=#1,
		font=\large,
		{#5}
		]

		\foreach \bidule in #4 {

		\addplot[scatter, scatter src=explicit symbolic] coordinates {\bidule};

		}
		\legend{#3}
	\end{axis}
\end{tikzpicture}
}


\newcommand{\setex}[1]{\ensuremath{{\mathcal X}^{#1}}\xspace}
\newcommand{\posex}{{\setex{\top}}\xspace}
\newcommand{\negex}{{\setex{\bot}}\xspace}
\newcommand{\allex}{{\setex{}}\xspace}
\newcommand{\setcube}[1]{\ensuremath{{\mathcal C}^{#1}}\xspace}
\newcommand{\poscube}{{\setcube{1}}\xspace}
\newcommand{\negcube}{{\setcube{0}}\xspace}
\newcommand{\features}{\ensuremath{{\mathcal F}}\xspace}
\newcommand{\feat}{\ensuremath{f}}
\newcommand{\classifier}{\ensuremath{f}}
\newcommand{\lit}[1]{\ensuremath{l_{#1}}}
\newcommand{\var}{\ensuremath{x}}
\newcommand{\truelit}[1]{\ensuremath{\var_{#1}}}
\newcommand{\falselit}[1]{\ensuremath{\bar{\var_{#1}}}}
\newcommand{\ex}{\ensuremath{\var}}
\newcommand{\cube}{\ensuremath{c}}
\newcommand{\universe}{\ensuremath{{\mathcal U}}}
\newcommand{\entropy}[1]{\ensuremath{{H}({#1})}\xspace}
\newcommand{\probability}[1]{\ensuremath{{Pr}({#1})}\xspace}


\newcommand{\lb}[1]{\ensuremath{lb({#1})}}


\newcommand{\nodes}[0]{\ensuremath{{\cal N}}}
\newcommand{\bud}[0]{\ensuremath{{\cal B}}}
\newcommand{\sequence}[0]{\ensuremath{{\cal S}}}

\newcommand{\maxd}[0]{\ensuremath{k}}
\newcommand{\anode}[0]{\ensuremath{i}}
\newcommand{\bnode}[0]{\ensuremath{b}}
\newcommand{\afeat}[0]{\ensuremath{f}}
\newcommand{\aclass}[0]{\ensuremath{l}}
\newcommand{\ub}[0]{\ensuremath{ub}}
% \newcommand{\error}[0]{\ensuremath{error}}
% \newcommand{\mdepth}[1][]{\ensuremath{\ifthenelse{\equal{#1}{}}{depth}{depth({#1})}}}
\newcommand{\weight}[1][]{\ensuremath{\ifthenelse{\equal{#1}{}}{w}{w({#1})}}}
\newcommand{\test}[1][]{\ensuremath{\ifthenelse{\equal{#1}{}}{test}{test({#1})}}}
\newcommand{\dom}[1][]{\ensuremath{\ifthenelse{\equal{#1}{}}{dom}{dom({#1})}}}
\newcommand{\best}[1][]{\ensuremath{\ifthenelse{\equal{#1}{}}{{best}}{{best({#1})}}}}
\newcommand{\maxe}[1][]{\ensuremath{\ifthenelse{\equal{#1}{}}{{max}}{{max({#1})}}}}
\newcommand{\opt}[1][]{\ensuremath{\ifthenelse{\equal{#1}{}}{optimal}{optimal({#1})}}}
\newcommand{\child}[1][]{\ensuremath{\ifthenelse{\equal{#1}{}}{{child}}{{child({{#1}})}}}}
\newcommand{\error}[1][]{\ensuremath{\ifthenelse{\equal{#1}{}}{{error}}{{error({{#1}})}}}}
\newcommand{\classlabel}[1][]{\ensuremath{\ifthenelse{\equal{#1}{}}{{y}}{{y({{#1}})}}}}

\newcommand{\abranch}[0]{\ensuremath{b}}


\newcommand{\numex}[0]{\ensuremath{n}}
\newcommand{\numfeat}[0]{\ensuremath{m}}
\newcommand{\mdepth}[0]{\ensuremath{k}}



	\SetKwFunction{storebest}{storeBest}
	\SetKwFunction{solution}{solution}
	\SetKwFunction{deadend}{deadEnd}
	\SetKwFunction{expend}{expend}
	\SetKwFunction{backtrack}{backtrack}
	\SetKwFunction{dt}{BTDT}
	\SetKwFunction{pop}{pop}
	\SetKwFunction{push}{push}
	\SetKwFunction{prune}{prune}
	% \SetKwFunction{error}{error}
	\SetKwFunction{grow}{grow}
	\SetKwFunction{branch}{branch}
	\SetKwFunction{select}{select\&remove}
	\SetKwFunction{budalg}{Bud-first-search}
	\SetKwFunction{dynprog}{DynProg}
	\SetKwFunction{dleight}{DL8.5}
	\SetKwFunction{cart}{CART}
	\SetKwFunction{greedy}{BudFS (first sol.)}
  \SetKwFunction{murtree}{Murtree}
	\SetKwFunction{dominated}{Dominated}
	
	
	\newcommand{\nolb}[0]{No lower bound}
	\newcommand{\nopreprocessing}[0]{No preprocessing}
	\newcommand{\noheuristic}[0]{No heuristic}
	

\DontPrintSemicolon

\title{A Simple and Efficient Anytime Algorithm for Computing Optimal Decision Trees}

\author{Emir Demirovi\'c\inst{1} \and Emmanuel Hebrard\inst{2} \and Louis Jean\inst{2}}
\institute{TU DELFT, The Netherlands, email: e.demirovic@tudelft.nl
  \and LAAS-CNRS, Universit\'e de Toulouse, CNRS, France, email: \{hebrard,ljean\}@laas.fr}

\begin{document}


\maketitle






\begin{abstract}
	
\end{abstract}

\section{Introduction}

Ever since the pionneering work decision tree classifiers~\cite{breiman1984classification,10.1023/A:1022643204877}, the question of computing \emph{optimal} decision trees has been alive, and recently a number of algorithms have been introduced for that purpose.
Several variants and criteria have been proposed. %, e.g. computing the perfect classifier of minimum size~\cite{}, 
Among those, computing a depth-bounded decision tree of maximum accuracy is often the preferred criterion~\cite{bertsimas2017optimal,hu2019optimal,dl8,verhaeghe2019learning}, because while being straightforward, it captures many desired features: it is resilient to noisy data, and shallow trees are both easier to explain and less prone to overfitting.

\medskip

Despite the vast offer of exact methods that can potentially provide optimal decision trees, or at least should, given enough time, improve on heuristics, the latter are still vastly more commonly used in practice. Some methods are memory-hungry, some are not anytime, and as far as we know there is not a single algorithm that can provide optimal classifiers while scaling to large datasets, feature space, or tree depth as heuristics do.
% \begin{itemize}
% 	\item MinDT -> does not scale, problem with noise
% 	\item other SAT approaches -> do not scale in some way or another
% 	\item DL8 -> does not scale in memory, slower
% 	\item BinOCT -> ? (probably much slower)
%   \item Murtree -> by far the most efficient, however, does not scale as well on deep trees and not as anytime
% \end{itemize}



\medskip

In this paper we introduce a relatively \emph{simple} algorithm, that is as memory and time efficient as heuristics, and yet more efficient than most exact methods on most datasets. 
This algorithm can be seen as an instance of the more general framework introduced in \cite{DBLP:journals/corr/abs-2007-12652}, however tuned to have the best scalability to large trees and the best anytime behavior as possible.

In a nutshell, this algorithm (\budalg) emulates the dynamic programming algorithm \dleight~\cite{dl8}, while expending non-terminal branches (a.k.a ``buds'') first. As a result, this algorithm is in a sense strictly better than both the standard dynamic programming approach (because it is anytime and at least as fast) and than classic heuristics (because it emulates them during search, without significant overhead).
%but explores the search space so as to improve its anytime behaviour.
Our experimental results show that it outperforms the state of the art, to the exception of \murtree~\cite{DBLP:journals/corr/abs-2007-12652} on relatively shallow trees (typically for maximum depth up to 4), for which this its more sophisticated (albeit complex) algorithmic features can pay off.
In particular, on datasets that \dleight can tackle, \budalg can always find classifiers at least as accurate faster, and when the former can prove optimality, the latter does it orders of magnitude faster.

%
%
% \begin{itemize}
% 	\item Same worst-case complexity than DL8
% 	\item No memory usage
% 	\item Better anytime behaviour than DL8 (in fact as good as the state of the art heuristics)
% 	\item Therefore, strictly better than greedy heuristics and almost always better than DL8
% \end{itemize}
%
%
% We consider the problem of finding the bounded-depth decision tree of maximum accuracy.
% The state of the art includes MIP approaches (BinOCT), a MaxSAT approach based on the SAT encoding proposed by Narodytska et al, and DL8.5.
% The latter algorithm is by far the most efficient, however, it is not \emph{anytime}: the left branch must be optimally solved before a solution of the right branch can be found. Moreover the use of a cache structure means that it uses a lot of memory. This algorithm is practical for a maximum depth of 4 (although using gigabytes of memory) but often not much beyond. Therefore, in a number of cases, a greedy heuristic (such as CART) is still the best method in practice.
%
% \medskip

% In this note we introduce what is essentially an anytime version of DL8.5, without cache.
% This algorithm therefore uses linear (in the size of the tree) memory and anytime, hence in principle strictly better than CART. Moreover, on instance where DL8.5 can find a solution, the algorithm described in this note is significantly faster (by about a factor 10).

\section{Preliminaries}

A binary data set is a pair $\langle \negex,\posex \rangle$ where the negative datapoints $\negex$ (resp. the positive datapoints $\posex$) is a subset of $2^{\features}$. It is associated a label function $\classlabel : 2^{\features} \mapsto \{\top,\bot\}$ such that:
$$
\forall \aclass \in \{\top,\bot\}, \forall \ex \in \setex{\aclass}, \classlabel[\ex] = \aclass
$$
We refer the union of negative and positive datapoints by $\allex = \negex \cup \posex$.
%$\top$ and $\bot$ are class labels and 
A datapoint $\ex$ can equivalently be seen as a subset of $\features$, or as the conjunction:
$$
\bigwedge_{\afeat \in \ex}\afeat \wedge \bigwedge_{\afeat \in \features \setminus \ex}\bar{\afeat}
$$


A \emph{binary decision tree} is a tree whose leaves are labelled with either $\top$ or $\bot$, internal vertices are labelled with features and edges are labelled with either $0$ or $1$.

To a \emph{branch} of a decision tree we can associate the conjunction of features which are on the path from the root to a leaf.
If the feature vertex is exited by a $1$-edge, the feature is positive in the conjunction, otherwise it is negative.

 
%Moreover, if we also consider data points as conjunctions of features (where every feature appears either positively or negatively),
%given a branch $\abranch \subseteq \features$ 
Given a data set $\langle \negex,\posex \rangle$, we can associate a data set $\langle \negex(\abranch),\posex(\abranch) \rangle$ to a branch $\abranch$ where:
\begin{eqnarray*}
\negex(\abranch) = \{\ex \mid \ex \in \negex, \ex \models \abranch\}\\
\posex(\abranch) = \{\ex \mid \ex \in \posex, \ex \models \abranch\}
\end{eqnarray*}

We write $\error[\abranch]$ for $\min(|\negex(\abranch)|, |\posex(\abranch)|)$, and $\error[\abranch,\afeat]$ for $\error[\abranch \wedge \afeat] + \error[\abranch \wedge \bar{\afeat}]$.
A branch \abranch\ is said \emph{pure} iff $\error[\abranch]=0$.

\medskip

Given a binary dataset $\langle \negex,\posex \rangle$ with label function $\classlabel$,
the \emph{minimum error bounded depth decision tree problem} consists in finding the minimum value $\epsilon$ such
there exists a binary tree of depth at most $\maxd$ whose sum of error $\error[\abranch]$ for all branches $\abranch$ from the root to the leaves is equal to $\epsilon$.


\subsection{Dynamic Programming Algorithm}

The solver DL8.5 is a dynamic programming algorithm. It relies on the observation that given a feature test, the two resulting branches are independent subproblems. Algorithm~\ref{alg:dynprog} gives a high level view of DL8.5.


	% \begin{algorithm}
	% 	\caption{Dynamic Programming Algorithm\label{alg:dynprog}}
	% 	\TitleOfAlgo{\dynprog}
	% 	  \KwData{$\negex,\posex,\maxd,\abranch[=(\tru)]$}
	% 	  \KwResult{The minimum error on $\negex,\posex$ for decision trees of depth at most $\maxd$}
	% 		\lIf{$\maxd = 0$ or $\error[\abranch] = 0$} {
	% 		\Return $\error[\abranch]$
	% 		}
	% 		$\best \gets \error[\abranch]$\;
	% 		\ForEach{$\afeat \in \features \setminus \abranch$} {
	% 				$\best \gets \min(\best, \dynprog(\negex,\posex, \abranch \wedge \afeat, \maxd-1) + \dynprog(\negex,\posex, \abranch \wedge \bar{\afeat}, \maxd-1))$\;
	% 		}
	% 		\Return $\best$\;
	% \end{algorithm}
	
	\begin{algorithm}
		\caption{Dynamic Programming Algorithm\label{alg:dynprog}}
		\TitleOfAlgo{\dynprog}
		  \KwData{$\negex,\posex,\features,\maxd$}
		  \KwResult{The minimum error on $\negex,\posex,\features$ for decision trees of depth $\maxd$}
			$\error \gets \min(|\negex|,|\posex|)$\;
			
			\lIf{$\maxd = 0$ or $\error = 0$} {
			\Return $\error$
			}
			
			\ForEach{$\afeat \in \features$} {
					$\error \gets \min(\error, \dynprog(\negex(\afeat),\posex(\afeat),\features \setminus \{\afeat\},\maxd-1)$\;
					\ \ \ \ \ \ \ \ \ \ \ \ \ \ \ \ \ \ \ \ \ \ \ \  $ + \dynprog(\negex(\bar{\afeat}),\posex(\bar{\afeat}),\features \setminus \{\afeat\},\maxd-1))$\;
			}
			\Return $\error$\;
	\end{algorithm}
	
	
	Let $\numex = |\posex| + |\negex|$, $\numfeat = |\features|$, and let $\mdepth$ be the maximum depth.
	We can safely assume $\mdepth \leq \numfeat$ (otherwise all features can be tested on every branch) and $\mdepth \leq \log \numex$ (otherwise we could have one distinct branch per datapoint), and it is often assumed that
	 $\mdepth \ll \numfeat$ and $\mdepth \ll \log \numex$. 
	%The number of recursive calls for Algorithm~\ref{alg:dynprog} is $\Theta(2^{\mdepth-1}{\numfeat \choose \mdepth})$, to explore at most ${\numfeat \choose \mdepth}$ combinations of features for at most $2^{\mdepth-1}$ branches. 
	%The number of recursive calls for Algorithm~\ref{alg:dynprog} is $\Theta(2^{\mdepth}{\numfeat \choose \mdepth})$, to explore $\Theta({\numfeat \choose \mdepth})$ combinations of features for $\Theta(2^{\mdepth})$ branches. 
	The number of recursive calls for Algorithm~\ref{alg:dynprog} is $\Theta((2\numfeat)^{\mdepth})$. 
	Moreover, at each call, the data set must be split into two subsets. However, this splitting procedure can be done in time $\Theta(\numex)$ amortised over a given level of a particular decision tree, since the data sets associated to branches up to a given level form a partition of the original data set.
	Therefore, since Algorithm~\ref{alg:dynprog} independently explores $\Theta({\numfeat \choose \mdepth})$ sets of branches of depth $\mdepth$, it runs in
	$\Theta((\numex+2^\mdepth){\numfeat \choose \mdepth})$ time (hence $O(\numex\numfeat^\mdepth)$ time, since we suppose $\mdepth \ll \numfeat$ and $\mdepth \ll \log \numex$).
	
	Note that this is a significant improvement with respect to the $\Theta(\numfeat^{2^{\mdepth}})$ trees (with redundant branches) explored by a brute-force algorithm.
	 %
	 % a brute-force algorithm, since the total number of decision trees of depth $\mdepth$ with $\numfeat$ attributes is $\prod_{x=1}^{\mdepth}(1-x+\numfeat)^{2^{\mdepth}}$. If we assume $\mdepth \in O(1)$, this is $\Theta(\numfeat^{2^{\mdepth}})$ distinct trees.
	 %



\section{An Anytime Algorithm}

In a nutshell the algorithm we propose works as follows:

\medskip

We start from a singleton set \bud\ of open branches or \emph{buds}.
% (open branches are branches of length strictly less than $\maxd$ that are not pure). % The set \nodes\ contains all the nodes of the current tree, open or closed, it is initially equal to \bud. Finally,
Moreover, we use an initially empty stack of decisions $\sequence$.

\begin{itemize}
	\item As long as there is a bud ($\bud \neq \emptyset$), we pick any one $\abranch \in \bud$ and check if it can be expended in Line~\ref{line:leaves}. If its length is $\mdepth$ the error at this leaf is recorded in $\best[\abranch]$. Otherwise, we pick a feature $\afeat$ marked as \emph{available} for \abranch, add the pair $(\abranch,\afeat)$ to \sequence\ and expend the tree with the two branches $\abranch \wedge \afeat$ and $\abranch \wedge \bar{\afeat}$. 
	%
	%
	%
	%
	%  a feature $\afeat$ marked as \emph{available} for \abranch, add the pair $(\abranch,\afeat)$ to \sequence\ and expend the tree with the two branches $\abranch \wedge \afeat$ and $\abranch \wedge \bar{\afeat}$.
	% %These new nodes are added to $\nodes$.
	% They are added to $\bud$ if their depth is strictly less than $\maxd-1$ (the last feature test is chosen according to minimum error) and if they are not pure, otherwise they are terminal tests and we record the corresponding error.

\item If there is no bud ($\bud = \emptyset$), then the tree is complete: every branch is of size $\mdepth$ or is pure. In that case we pop the last assignment $(\abranch,\afeat)$ from \sequence, mark the feature $\afeat$ not available for branch $\abranch$ and update the recorded best error of its subtrees. If there is at least one available feature for branch $\abranch$, we add $\abranch$ to $\bud$. 
Otherwise, we consider it \emph{terminal} and its error is the sum of the error of its best subtrees. This branch will never be expended anymore since it is not added to $\bud$.
% and we 
%store the minimum error recorded for any of the possible features.

\end{itemize}


Notice that to simplify the pseudo-code, we use branches to directly index array-like data structures in Algorithm~\ref{alg:bud}. Actually, a set of \emph{indices} (at most $2^{\mdepth}$ of them are necessary) are used as proxy for branches in all contexts. At Line~\ref{line:storebest}, the indices for $\abranch \wedge \afeat$, $\abranch \wedge \bar{\afeat}$ are released, and a free index is marked as used when expending a branch at Line~\ref{line:branching}. Moreover, the pseudo-code in Algorithm~\ref{alg:bud} does show how the best subtrees of terminal branches are recorded, nor how the overall best error is updated.



%The classification error of a tree is equal to the sum of the error of its terminal banches, the algorithm returns the minimum value encountered when exploring the possible decision trees.
%In other words, this algorithm will first build a complete tree by expanding non-pure, non-maximal branches in any order. When all leaves are pure of a maximal depth, the last test of the last expanded branch w



% \medskip
		

		\begin{theorem}
			The time complexities of Algorithms~\ref{alg:dynprog} and Algorithm~\ref{alg:bud} are equal.
			\end{theorem}
			
			\begin{proof}[sketch]
				The proof follows from the fact that both algorithm explore exactly the same set of branches, albeit not in the same order.
				Both algorithm eventually explore every combination of (negated) feature of size $\mdepth$.
				
				We show that this is true for Algorithm~\ref{alg:bud} by recursion on the maximum depth $\mdepth$.
				For $\mdepth = 0$, both algorithm return $\error[\emptyset]=\min(|\negex|, |\posex|)$.
				
				Now suppose that for $\mdepth \leq d$ every branch $\abranch$ explored by \dynprog (a recursive call ends on this branch) is also explored by \budalg ($\best[\abranch]$ is set in Line~\ref{line:best}), and let $\mdepth=d+1$. Without loss of generality, we can take an arbitrary branch $\abranch$ of length $d$ and show that the same extentions of $\abranch$ are explored by both algorithms.
				If $\error[\abranch]=0$, then no extension of $\abranch$ is explored by either algorithm.
				Otherwise, \dynprog explores every branch $\abranch \wedge \afeat$ and $\abranch \wedge \bar{\afeat}$ for $\afeat \in \features \setminus \abranch$. Since $\abranch$ is explored when $\mdepth = d$ and since $\error[\abranch] \neq 0$, then $\abranch$ will fail the test on Line~\ref{line:leaves} and for a feature $\afeat \in \features \setminus \abranch$, the branches $\abranch \wedge \afeat$ and $\abranch \wedge \bar{\afeat}$ will be explored, and the pair $(\abranch, \afeat)$ will be added to $\sequence$. 
				Moreover, the algorithm will not stop until $\sequence$ is not empty, hence $(\abranch, \afeat)$ will be poped out of $\sequence$, $\afeat$ removed from $\dom[\abranch]$ and $\abranch$ reinserted into $\bud$. Therefore, for every $\afeat \in \features \setminus \abranch$, $\abranch \wedge \afeat$ and $\abranch \wedge \bar{\afeat}$ will eventually be explored exactly once.
				\hfill$\square$
			\end{proof}
			
			\medskip
			
			The key difference between Algorithms~\ref{alg:dynprog} and \ref{alg:bud} is the order in which branches are explored (see Figure~\ref{fig:searchtree}). In particular, \dynprog must complete the first recursive call before outputing a full tree. Therefore, the computation time before to find a first tree is $\Theta((\numex+2^{\mdepth-1}){\numfeat-1 \choose \mdepth-1})$, that is $O(\numex(\numfeat-1)^{\mdepth-1})$ time. On the other hand, \budalg finds a first tree in $\Theta(2^{\mdepth}+\numex\numfeat) = \Theta(\numex\numfeat)$ time.
			
			
		
			
			
			
% 			% First, notice that the task of splitting the data set on eevry branch of the search tree can be done exactly as in DL8, that is, in $\Theta(\numex)$ amortised time for each of the $\mdepth$ tree levels.
%
% 			% \begin{proof}
%
% 			First, all branches eventually reached. Consider an arbitrary branch $\abranch$.
%
%
% 			\medskip
%
%
% 			$\best[\abranch]$ is the minimal error of any subtree rooted at $\abranch$ with a feature in $\features \setminus \dom[\abranch]$ is $\best[\abranch]$.
% 			This is true for pure branches
%
%
% 			 and maximum-depth branches (code after condition in Line~\ref{line:leaves}). Now consider a branch $\abranch$ such that $|\abranch|<\mdepth$ and $\forall \afeat \in \features, \error[\abranch,\afeat] > 0$.
%
% 			\medskip
%
%
%
%
% 			Let a branch $\abranch$ be \emph{explored} iff it was put in the stack $\sequence$ and $\dom[\abranch] = \emptyset$.
% 			If a branch $\abranch$ is explored, then $\best[\abranch]$ is the minimal error of any subtree rooted at $\abranch$.
% 			This is true for pure branches and maximum-depth branches (code after condition in Line~\ref{line:leaves}). Now consider a branch $\abranch$ such that $|\abranch|<\mdepth$ and $\forall \afeat \in \features, \error[\abranch,\afeat] > 0$.
%
% 			\medskip
%
%
%
% 				First, we show that the algorithm is correct. In particular the following property holds:
% 				after Line~\ref{line:storebest}, the minimum error of any subtree rooted at $\abranch$ with a feature in $\features \setminus \dom[\abranch]$ is $\best[\abranch]$.
%
%
% 				Observe that $|\abranch| \leq \mdepth-2$. Indeed, no pair $(\abranch,\afeat)$ is put on $\sequence$ unless $|\abranch| \leq \mdepth-2$. Now suppose that $|\abranch|=\mdepth-2$. Then $\best[\abranch \wedge \afeat]$ is by definition the minimal error of any single-node tree rooted at $\abranch \wedge \afeat$ and likewise, $\best[\abranch \wedge \bar{\afeat}]$ is the minimal error of any single-node tree rooted at $\abranch \wedge \bar{\afeat}$. Therefore, $\best[\abranch \wedge \afeat] + \best[\abranch \wedge \bar{\afeat}]$ is the minimal error of a depth 2 tree rooted at $\abranch$ with a test on $\afeat$. Since all features in $\features \setminus \dom[\abranch]$ have been tried (or belong to $\abranch$) and the minimum was kept, the property holds.
%
% 				Suppose now that the property holds for $|\abranch|=\mdepth-x$ with $x>2$. Then by the same reasoning as above, the property will also hold for $|\abranch|=\mdepth-x-1$. Therefore is always hold.
%
%
%
% 			\medskip
%
%
%
%
%
%
% 			The key is to observe that given a branch $\abranch$ and a feature $\afeat$, just as in DL8, the complexity of computing
% 			$\error[\abranch,\afeat]$ is equal to the complexity of computing $\error[\abranch \wedge \afeat]$ plus the complexity of computing $\error[\abranch \wedge \bar{\afeat}]$.
% 			Indeed, wlog, let the branch $\abranch \wedge \afeat$ be chosen before $\abranch \wedge \bar{\afeat}$.
% 				Both branches will be completed up to the maximal depth, however,
% 			the test appended to $\abranch \wedge \afeat$ will no change until
%
%
%
%
% 			the best possible errors for $\abranch \wedge \afeat$ and $\abranch \wedge \bar{\afeat}$
%
%
% 			% \end{proof}
%
% 			We first show that the number of assignment of tests (Line~\ref{line:assignment}) is $2^{\mdepth-1}\numfeat^{\mdepth-1}$.
% 			For $\mdepth=1$ this is true since there is a single assigned test (with the feature $\argmin_{\afeat \in \features}(\error[\emptyset,\afeat])$).
%
% 			Now suppose that the property holds for depth $\mdepth-1$ and consider depth $\mdepth$.
%
%
%
%
% 			No proof is given, but independent subtrees are not explored in Algorithm~\ref{alg:bud} hence both algorithms do the same computation, except not in the same order. Notice that Algorithm~\ref{alg:bud} eagerly compute the conditional error $\error[\abranch,\afeat]$ for every feature $\afeat \in \features$, which incurs an extra factor $\numfeat$ for the data set partitionning task. However, in return the last test of each branch is chosen in $O(1)$ so we gain the same factor $\numfeat$.
%
% 			The big difference is that whereas the cost of finding a first solution is $\Theta((\numex + 2^{\mdepth-1})\numfeat^{\mdepth-1})$ for Algorithm~\ref{alg:dynprog}, it is equal to $\Theta(\mdepth\numex + 2^\mdepth)$ for Algorithm~\ref{alg:bud}, which in practice is very important, as shown in the experimental section.
%
% 	%T(m,k) = 2mT(m-1, k-1)
%
%
%
%
% % \clearpage





\begin{algorithm}
\begin{footnotesize}
		\caption{Blossom Algorithm\label{alg:bud}}
		\TitleOfAlgo{\budalg}
		  \KwData{$\negex,\posex, \maxd$}
		  \KwResult{The minimum error on $\negex,\posex$ for decision trees of depth $\maxd$}
		$\sequence \gets []$\;
		$\bud \gets \{\emptyset\}$\;
		$\error \gets \min(|\negex|,|\posex|)$\;
		$\dom \gets (\lambda : {2^{\features}} \mapsto \features)$\;
		$\best \gets (\lambda : {2^{\features}} \mapsto \infty)$\;
		
		\While{$|\sequence| + |\bud| > 0$}{
		\lnl{line:dive}\eIf{$\bud \neq \emptyset$}{
			%$\abranch \gets \select{\bud}$\;
			\lnl{line:budchoice}pick and remove $\abranch$ from $\bud$\;
			
			\lnl{line:leaves}\eIf{$|\abranch| = \maxd$ or $\error[\abranch] = 0$} {
				% $\error \gets \error + \error[\abranch,\afeat]$\;
				\lnl{line:best}$\best[\abranch] \gets \error[\abranch]$\;
			}{
			\lnl{line:assignment} pick and remove $\afeat$ from $\dom[\abranch]$\;
			% $\dom[\abranch] \gets \dom[\abranch] \setminus \{\afeat\}$\;
			push $(\abranch,\afeat)$ on $\sequence$\; 
			\lnl{line:branching}\lForEach{$v \in \{\afeat, \bar{\afeat}\}$}{
						$\bud \gets \bud \cup \{\abranch \wedge v\}$
					}
			}
		}{
			% $\best[\emptyset] \gets \min(\best[\emptyset], \error)$\;
			\lnl{line:backtrack}\Repeat{$\dom[\abranch] \neq \emptyset$ or $|\sequence|=0$}{
				pop $(\abranch,\afeat)$ from $\sequence$\;
				\lnl{line:storebest}$\best[\abranch] \gets \min(\best[\abranch], \best[\abranch \wedge \afeat] +  \best[\abranch \wedge \bar{\afeat}])$\;
				% $\error \gets \error - \best[\abranch \wedge \afeat] -  \best[\abranch \wedge \bar{\afeat}]$\;
				\lIf{$\dom[\abranch] \neq \emptyset$} {
					$\bud \gets \bud \cup \{\abranch\}$
					% \lIf{$\opt[\abranch]$}{$\error \gets \error - \best[\abranch]$}
				} 
				% \lElse {
				% 	$\error \gets \error + \best[\abranch]$%$\error[\abranch,\afeat]$
				% }
				% {
				% 	$\opt[\abranch] \gets \tru$\;
				% }
			}
		}
		}
		\Return $\best[\emptyset]$\;
	\end{footnotesize}
	\end{algorithm}
	
	
	% \begin{algorithm}
	% 	\caption{Anytime Algorithm\label{alg:bud}}
	% 	\TitleOfAlgo{\budalg}
	% 	  \KwData{$\negex,\posex, \maxd$}
	% 	  \KwResult{The minimum error on $\negex,\posex$ for decision trees of depth at most $\maxd$}
	% 	$\sequence \gets []$\;
	% 	$\bud \gets \{\emptyset\}$\;
	% 	$\error \gets \min(|\negex|,|\posex|)$\;
	% 	$\dom \gets (\lambda : {2^{\features}} \mapsto \features)$\;
	% 	$\best \gets (\lambda : {2^{\features}} \mapsto \infty)$\;
	% 	% $\opt \gets (\lambda : {2^{\features}} \mapsto \fal)$\;
	%
	% 	\While{$|\sequence| + |\bud| > 0$}{
	% 	\eIf{$\bud \neq \emptyset$}{
	% 		%$\abranch \gets \select{\bud}$\;
	% 		pick and remove $\abranch$ from $\bud$\;
	% 		\lnl{line:assignment}$\afeat \gets \argmin_{\afeat \in \dom[\abranch]}(\error[\abranch,\afeat])$\;
	% 		\lnl{line:leaves}\eIf{$\error[\abranch,\afeat] = 0$ or $|\abranch| = \maxd-1$} {
	% 			$\error \gets \error + \error[\abranch,\afeat]$\;
	% 			$\best[\abranch] \gets \error[\abranch,\afeat]$\;
	% 			$\dom[\abranch] \gets \emptyset$\;
	% 			% $\opt[\abranch] \gets \tru$\;
	% 		}{
	% 		$\dom[\abranch] \gets \dom[\abranch] \setminus \{\afeat\}$\;
	% 		push $(\abranch,\afeat)$ on $\sequence$\; % $ \gets \sequence \oplus (\abranch,\afeat)$\;
	% 		\ForEach{$v \in \{\afeat, \bar{\afeat}\}$}{
	% 				\lIf{$\error[\abranch,v] > 0$}{
	% 					$\bud \gets \bud \cup \{\abranch \wedge v\}$
	% 				}
	% 		}
	% 		}
	% 	}{
	% 		$\best[\emptyset] \gets \min(\best[\emptyset], \error)$\;
	% 		\Repeat{$\dom[\abranch] \neq \emptyset$ or $|\sequence|=0$}{
	% 			pop $(\abranch,\afeat)$ from $\sequence$\;
	% 			\lnl{line:storebest}$\best[\abranch] \gets \min(\best[\abranch], \best[\abranch \wedge \afeat] +  \best[\abranch \wedge \bar{\afeat}])$\;
	% 			$\error \gets \error - \best[\abranch \wedge \afeat] -  \best[\abranch \wedge \bar{\afeat}]$\;
	% 			\lIf{$\dom[\abranch] \neq \emptyset$} {
	% 				$\bud \gets \bud \cup \{\abranch\}$
	% 				% \lIf{$\opt[\abranch]$}{$\error \gets \error - \best[\abranch]$}
	% 			}
	% 			\lElse {
	% 				$\error \gets \error + \best[\abranch]$%$\error[\abranch,\afeat]$
	% 			}
	% 			% {
	% 			% 	$\opt[\abranch] \gets \tru$\;
	% 			% }
	% 		}
	% 	}
	% 	}
	% 	\Return $\best[\emptyset]$\;
	% \end{algorithm}
	
	
	
	\begin{figure}
	\begin{center}
		\tabcolsep=0pt
		\scalebox{1}{
			\begin{forest}
				for tree={%
					l sep=20pt,
					s sep=3.5pt,
					node options={shape=rectangle, minimum width=10pt, inner sep=0pt, font=\footnotesize},
		  		edge={thick, -latex, shorten >=1pt, shorten <=1pt},
				}
				[{$\emptyset$}
					[{$a$}
						[{\begin{tabular}{c}$b$\\1\end{tabular}}]
						[{\begin{tabular}{c}$\bar{b}$\\2\end{tabular}}]
						[{\begin{tabular}{c}$c$\\7\end{tabular}}]
						[{\begin{tabular}{c}$\bar{c}$\\8\end{tabular}}]
					]
					[{$\bar{a}$}
						[{\begin{tabular}{c}$b$\\3\end{tabular}}]
						[{\begin{tabular}{c}$\bar{b}$\\4\end{tabular}}]
						[{\begin{tabular}{c}$c$\\5\end{tabular}}]
						[{\begin{tabular}{c}$\bar{c}$\\6\end{tabular}}]
					]
					[{$b$}
						[{\begin{tabular}{c}$a$\\9\end{tabular}}]
						[{\begin{tabular}{c}$\bar{a}$\\10\end{tabular}}]
						[{\begin{tabular}{c}$c$\\15\end{tabular}}]
						[{\begin{tabular}{c}$\bar{c}$\\16\end{tabular}}]
					]
					[{$\bar{b}$}
						[{\begin{tabular}{c}$a$\\11\end{tabular}}]
						[{\begin{tabular}{c}$\bar{a}$\\12\end{tabular}}]
						[{\begin{tabular}{c}$c$\\13\end{tabular}}]
						[{\begin{tabular}{c}$\bar{c}$\\14\end{tabular}}]
					]
					[{$c$}
						[{\begin{tabular}{c}$a$\\17\end{tabular}}]
						[{\begin{tabular}{c}$\bar{a}$\\18\end{tabular}}]
						[{\begin{tabular}{c}$b$\\23\end{tabular}}]
						[{\begin{tabular}{c}$\bar{b}$\\24\end{tabular}}]
					]
					[{$\bar{c}$}
						[{\begin{tabular}{c}$a$\\19\end{tabular}}]
						[{\begin{tabular}{c}$\bar{a}$\\20\end{tabular}}]
						[{\begin{tabular}{c}$b$\\21\end{tabular}}]
						[{\begin{tabular}{c}$\bar{b}$\\22\end{tabular}}]
					]
				]
			\end{forest}
		}
	\end{center}
	\caption{\label{fig:searchtree} The search tree for decision trees. \texttt{DynProg} explores it depth first, whereas \texttt{Bud-first-search} explores branches in the order given below the leaves.}
	%\caption{\label{fig:searchtree} The search tree for decision trees. \dynprog explores it depth first, whereas \budalg explores branches in the order given below the leaves.}
	\end{figure}
	
	
	
	
	% \begin{figure}
	% \begin{center}
	% 	\tabcolsep=0pt
	% 	\scalebox{1}{
	% 		\begin{forest}
	% 			for tree={%
	% 				l sep=20pt,
	% 				s sep=3.5pt,
	% 				node options={shape=rectangle, minimum width=10pt, inner sep=0pt, font=\footnotesize},
	% 	  		edge={thick, -latex, shorten >=1pt, shorten <=1pt},
	% 			}
	% 			[{$\emptyset$}
	% 			 [{$a,\bar{a}$}
	% 			  [{$a \wedge b$,$a \wedge \bar{b}$}
	% 					[{$a \wedge b \wedge c$,$a \wedge b \wedge \bar{c}$}]
	% 					[{$a \wedge \bar{b} \wedge c$,$a \wedge \bar{b} \wedge \bar{c}$}]
	% 				]
	% 			  [{$\bar{a} \wedge b$,$\bar{a} \wedge \bar{b}$}]
	% 			 ]
	% 			 [{$b$}]
	% 			 [{$c$}]
	% 			]
	% 		\end{forest}
	% 	}
	% \end{center}
	% \caption{\label{fig:searchtree} The search tree for decision trees. \texttt{DynProg} explores it depth first, whereas \texttt{Bud-first-search} explores branches in the order given below the leaves.}
	% %\caption{\label{fig:searchtree} The search tree for decision trees. \dynprog explores it depth first, whereas \budalg explores branches in the order given below the leaves.}
	% \end{figure}
	
	
	% \begin{tabular}{c|ll}
	% 	\# & \bud & \sequence \\
	% 	\hline
	% 	1 & $\{\emptyset\}$ & $[]$ \\
	% 	2 & $\{a,\bar{a}\}$ & $[(\emptyset,a)]$ \\
	% 	3 & $\{a \wedge b,a \wedge \bar{b},\bar{a}\}$ & $[(\emptyset,a),(a,b)]$ \\
	% 	4 & $\{a \wedge \bar{b},\bar{a}\}$ & $[(\emptyset,a),(a,b)]$ \\
	% 	5 & $\{\bar{a}\}$ & $[(\emptyset,a),(a,b)]$ \\
	% 	6 & $\{\bar{a} \wedge b, \bar{a} \wedge \bar{b}\}$ & $[(\emptyset,a),(a,b),(\bar{a},b)]$ \\
	% 	7 & $\{\bar{a} \wedge \bar{b}\}$ & $[(\emptyset,a),(a,b),(\bar{a},b)]$ \\
	% 	8 & $\{\}$ & $[(\emptyset,a),(a,b),(\bar{a},b)]$ \\
	% 	9 & $\{\bar{a} \wedge c, \bar{a} \wedge \bar{c}\}$ & $[(\emptyset,a),(a,b),(\bar{a},c)]$ \\
	% 	10 & $\{\bar{a} \wedge \bar{c}\}$ & $[(\emptyset,a),(a,b),(\bar{a},c)]$ \\
	% 	11 & $\{\}$ & $[(\emptyset,a),(a,b),(\bar{a},c)]$ \\
	% 	12 & $\{a \wedge c, a \wedge \bar{c}\}$ & $[(\emptyset,a),(a,c)]$ \\
	% 	13 & $\{a \wedge \bar{c}\}$ & $[(\emptyset,a),(a,c)]$ \\
	% 	14 & $\{\}$ & $[(\emptyset,a),(a,c)]$ \\
	% \end{tabular}


% For readability, we cut the algorithm into four blocks. The initialisation procedure (Algorithm~\ref{alg:init}) set up the data structures used in all other procedures:
% \begin{itemize}
% 	\item \sequence\ is simply the list of nodes in the current tree, ordered as they are explored.
% 	\item \nodes\ is the set of integers used to index a node of the current tree
% 	\item \bud\ is the set of nodes which do no have an assigned test yet
% 	\item \mdepth\ stores the depth of a node
% 	\item \test\ stores the feature tested at a node
% 	\item \dom\ stores the set of possible features which have no yet been tried for this node
% 	\item \best\ stores the error of the best subtree rooted at a node
% 	\item \opt\ indicates whether the best subtree of a given node is optimal
% 	\item \child\ stores the children of a node (children can be nodes or $\{\top, \bot\}$)
% 	\item $\error{\anode}$ $\min(|\posex(\anode)|,|\negex(\anode)|)$
% 	\item $\error{\anode,\afeat}$ $\min(|\posex(\anode=\afeat)|,|\negex(\anode=\afeat)|)$
% \end{itemize}
%
% Algorithm~\ref{alg:search} is a bactracking procedure which expends a current decision tree
%
% 	\begin{algorithm}
% 		\caption{Data Structures\label{alg:init}}
% 		\TitleOfAlgo{Initialise}
% 		$\sequence \gets []$\;
% 		$\bud \gets \emptyset$\;
% 		$\nodes \gets \emptyset$\;
% 		$\ub \gets \min(|\negex|,|\posex|)$\;
% 		$\error \gets ub$\;
%
% 		$\child \gets (\lambda : \mathbb{N} \times \{\fal, \tru\} \mapsto \emptyset)$\;
% 		$\mdepth \gets (\lambda : \mathbb{N} \mapsto 0)$\;
%
% 		$\test \gets (\lambda : \mathbb{N} \mapsto \emptyset)$\;
% 		$\dom \gets (\lambda : \mathbb{N} \mapsto \features)$\;
%
% 		$\best \gets (\lambda : \mathbb{N} \mapsto \infty)$\;
% 		$\opt \gets (\lambda : \mathbb{N} \mapsto \fal)$\;
% 	\end{algorithm}
%
%
%
% 	\begin{algorithm}
% 		\caption{Create a new node after branching\label{alg:alloc}}
% 		\TitleOfAlgo{\grow}
% 	  \KwData{integer \anode}
%
% 		$\nodes \gets \nodes \cup \{\anode\}$\;
% 		$\dom[\anode] \gets \features$ sorted by decreasing conditional error $\min(|\posex(\anode=\afeat)|,|\negex(\anode=\afeat)|)$\;
% 		$\test[\anode] \gets \pop(\dom[\anode])$\;
%
%
% 		\eIf{$\mdepth[\anode]=k-1$ or $\error{\anode,\test[\anode]}$}
% 		{
% 			$\best[\anode] = \error{\anode,\test[\anode]}$\; %\min(|\posex(\anode=\test[\anode])|,|\negex(\anode=\test[\anode])|)$\;
% 			$\opt[\anode] = \tru$\;
% 			\ForEach{$branch \in \{\tru, \fal\}$}
% 			{
% 				% $\child[\anode,branch] \gets (|\posex(\anode=\test[\anode])| > |\negex(\anode=\test[\anode])|)$\;
% 				\lIf{$|\posex(\anode=\test[\anode])| > |\negex(\anode=\test[\anode])|$}{$\child[\anode,branch] \gets \top$}
% 				\lElse{$\child[\anode,branch] \gets \bot$}
% 			}
% 		}
% 		{
% 			$\bud \gets \bud \cup \{n\}$\;
% 			$\best[\anode] = \min(|\posex(\anode)|, |\negex(\anode)|)$\;
% 			$\opt[\anode] \gets \fal$\;
% 		}
%
%
% 	\end{algorithm}
%
%
% 	\begin{algorithm}
% 		\caption{Suppress a node and all its descendants\label{alg:free}}
% 		\TitleOfAlgo{\prune}
% 	  \KwData{integer \anode}
%
% 		$\bud \gets \bud \setminus \{\anode\}$\;
% 		$\nodes \gets \nodes \setminus \{\anode\}$\;
%
% 		\ForEach{$branch \in \{\tru, \fal\}$}
% 		{
% 		\lIf{$\child[\anode,branch] \not\in \{\top, \bot\}$}
% 		{
% 			$\prune{\child[\anode,branch]}$
% 		}
% 		}
%
% 		\lIf{$\mdepth[\anode] = k-1$ or $\opt[\anode]$}{$error \gets error - \best[\anode]$}
%
% 	\end{algorithm}
%
%
% \begin{algorithm}
% 	\caption{Search loop\label{alg:search}}
%   \TitleOfAlgo{\dt}
%   \KwData{$\negex,\posex, k$}
%   \KwResult{A decision tree}
%
% 	$\bnode \gets 0$\;
% 	$\posex(1),\negex(1) \gets \posex, \negex$\;
% 	$\grow{\bud, \sequence, 1}$\;
%
% 	\While{\textbf{true}}{
% 		\eIf{$\bud = \emptyset$} {
% 			$\ub \gets \min(\ub,\error)$\;
% 			$deadend \gets \fal$\;
% 			\Repeat{$deadend$}{
% 				\lIf{$\bnode > 0$}{$\opt[\bnode] \gets \tru$}
% 				\lIf{$\bnode = 1$}{\Return}
% 				$\bnode \gets \pop{\sequence}$\;
% 				$\best[\bnode] \gets \min(\best[\bnode], \best(\child[\bnode,\tru]) + \best(\child[\bnode,\fal]))$\;
% 				$\test[\bnode] \gets \pop{\dom[\bnode]}$\;
% 				$\prune(\child[\bnode,\tru])$\;
% 				$\prune(\child[\bnode,\fal])$\;
%
% 				$deadend \gets \best[\bnode] = 0 ~\vee~ \dom[\bnode] = \emptyset$\;
% 				\If{$deadend$}
% 				{
% 				$\opt[\bnode] \gets \tru$\;
% 				$\error \gets \error + \error{\bnode}$\; %$\best[\bnode]$\;
% 				}
% 			}
% 			$\bud \gets \bud \cup \{b\}$\;
% 			$\error \gets \error + \min(|\posex(\bnode)|, |\negex(\bnode)|)$\;
% 		}
% 		{
% 			\If{$b = 0$}{
% 				$b=\select{\bud}$\;
% 				% $\bud \gets \bud \setminus \{b\}$\;
% 				$\push(\bnode,\sequence)$\;
% 			}
% 			$c_{\tru}, c_{\fal} = \argmin_{x,y}(\mathbb{N} \setminus \nodes)$\;
% 			$\posex(c_{\tru}),\negex(c_{\tru}),\posex(c_{\fal}),\negex(c_{\fal}) \gets \branch(\posex(\bnode),\negex(\bnode),\test[\bnode])$\;
% 			\ForEach{$branch \in \{\tru, \fal\}$}{
% 				\eIf{$\min(|\posex(c_{branch})|,|\negex(c_{branch})|) = 0$}
% 				{
% 					\lIf{$|\posex(c_{branch})|>|\negex(c_{branch})|$}{$\child[\bnode,branch] \gets \top$}
% 					\lElse{$\child[\bnode,branch] \gets \bot$}
% 				}{
% 					$\child[\bnode,branch] \gets c_{branch}$\;
% 					$\mdepth[c_{branch}] \gets \mdepth[\bnode]+1$\;
% 					$\grow(\bud, \sequence, c_{branch})$\;
% 				}
% 			}
% 			$\bnode \gets 0$\;
% 		}
% 	}
%
% \end{algorithm}

%\clearpage

\section{Extensions of the Algorithm}

The pseudo-code given in Algorithm~\ref{alg:bud} shows the basic structure of the algorithm. 
As discussed earlier, some important (but rather tedious) parts of the algorithms have been omitted, such as how the best subtrees are stored in Line~\ref{line:storebest} when the best classification score is updated.
We discuss here only improvements that have an impact on the efficiency of the algorithm.




\subsection{Heuristic Ordering}

In order to quickly find highly accurate trees, it is important to follow a heuristic. We experimented with three heuristics based on scores to minimize: \emph{error}, \emph{entropy}~\cite{10.1023/A:1022643204877}, and \emph{Gini impurity}~\cite{breiman1984classification}. 
Each of these heuristics associate a score to a feature $\afeat$ at a branch $\abranch$:
\begin{eqnarray}
	\textrm{error}: & \error[\abranch,\afeat] \\
	%\textrm{minimum entropy:} & \sum_{v \in \{\afeat,\bar{\afeat}\}} \frac{|\allex(\abranch \wedge v)|}{|\allex(\abranch)|} \cdot -\sum_{c \in \{\bot,\top\}} \frac{|\setex{c}(\abranch \wedge v)|}{|\setex{c}(\abranch)} \log_{2} \frac{|\setex{c}(\abranch \wedge v)|}{|\setex{c}(\abranch)|} \\
	\textrm{entropy:} & \sum\limits_{v \in \{\afeat,\bar{\afeat}\}} -p(v,\allex) \sum\limits_{c \in \{\bot,\top\}} p(v,\setex{c}) \log_{2} p(v,\setex{c}) \\
	\textrm{Gini impurity:} & 2 - \sum\limits_{v \in \{\afeat,\bar{\afeat}\}}\sum\limits_{c \in \{\bot,\top\}} p(v,\setex{c})^2
\end{eqnarray}
With $p(v,{\cal S}) = \frac{|{\cal S}(\abranch \wedge v)|}{|{\cal S}(\abranch)|}$.
%The minimum error is simply defined as 

The feature tests at Line~\ref{line:assignment} of Algorithm~\ref{alg:bud} are explored in non-decreasing order with respect to one of the scores above.


In the datasets we used, the Gini impurity was significantly better, and hence all reported experiment results are using Gini impurity unless stated otherwise. For branches of length $\mdepth-1$, however, we use the error instead. Indeed, the optimal feature $\afeat$ for a branch $\abranch$ that cannot be extended further is the one minimizing $\error[\abranch,\afeat]$. This means that we actually do not have to try other features for that node, which means that we effectively restrict search to branches of length $\mdepth-1$.


%We order the possible features for branch $\abranch$ in non-decreasing order with respect to a score above and 
%explore the features in that order in Line~\ref{line:assignment}.
Computing this order costs $\Theta(\numfeat \log \numfeat)$ for each of the $2^{\mdepth}\numfeat^{\mdepth}$ branches added to $\bud$ at Line~\ref{line:branching}. However, since the depth is effectively reduced by one, the resulting complexity 
%(excluding the time for splitting the dataset) 
is $O((\numex + 2^{\mdepth} \log \numfeat) \numfeat^{\mdepth})$. This very slight increase is often inconsequencial, since we often have $\numex \geq 2^{\mdepth} \log \numfeat$.

The feature ordering has a very significant impact on how quickly the algorithm can improve the accuracy of the classifier. Moreover, it also has an impact (though much less significant) on the computational time necessary to explore the whole search space and prove optimality, because of the lower bound technique detailed in the next section.


\subsection{Lower Bound}

It is possible to fail early using a lower bound on the error given prior decisions in the same way as \murtree, following the idea introduced in \cite{dl8}. The idea is that once some subtrees along a branch $\abranch$ are optimal and the sum of their errors is larger than the current upper bound (the best solution found so far) then there is no need to continue exploring branch $\abranch$. Line~\ref{line:leaves} can be changed to ``\textbf{If} $\bud \neq \emptyset ~\& \not\exists \abranch \in \bud, \dominated{\abranch}$ \textbf{then}''. %Notice that when a branch is ``pruned'' in this way, its 


%In this case, we can fail by forcing 


First, observe that $\best[\abranch]$ is an upper bound on the classification error for any subtree rooted at $\abranch$, since this value comes from an actual tree (of depth $\mdepth - |\abranch|$ for the dataset $\langle \negex(\abranch),\posex(\abranch) \rangle$). It is possible to propagate this upper bound to parent nodes efficiently (in $O(|\branch|)$ time). Here we assume that this is done every time the value $\best[\abranch]$ is actually updated, by recursively applying the same update procedure to the parent.


Now, when the loop in Line~\ref{line:backtrack} makes two or more iterations, it means that for the penultimate branch $\abranch$ popped out of \sequence, all possible subtrees have been explored, and therefore, $\best[\abranch]$ is also a lower bound on the classification error for any subtree rooted at $\abranch$. Let $\opt[\abranch]$ be 1 if $\abranch$ has been ``backtracked over'' in this way and 0 otherwise.


Let $\abranch$ be a bud in $\bud$. Trivially, if $|\abranch|=\mdepth-1$ then this branch will entail $\min\{\error[\afeat] \mid \afeat \in \dom[\abranch]\}$ misclassification.
However, for any parent $\abranch'$ of $\abranch$, we can define a lower bound $\lb{\abranch',\abranch}$, \emph{given the feature tests} $\abranch \setminus \abranch'$ as follows:
$$
\lb{\abranch',\abranch} = \sum\limits_{\abranch' \subset \abranch'' \wedge \afeat \subseteq \abranch}\opt[\abranch'' \wedge \bar{\afeat}]\best[\abranch'' \wedge \bar{\afeat}]
$$
This lower bound is correct as long as the branch $\abranch$ belongs to the decision tree.
The procedure $\dominated{\abranch}$ can therefore simply check, for all parent $\abranch'$ of $\abranch$ up until the root ($\emptyset$), whether $\lb{\abranch',\abranch} \geq \best[\abranch']$. As a result, a branch $\abranch$ which is guaranteed, by this reasoning, to never belong to a non-dominated tree will not be explored further.

\medskip

This reasoning will be more effective when good upper bounds are found early, hence the feature ordering heuristic discussed in the previous section has an impact. Moreover, the order in which buds are expended (the choice of branch in Line\ref{line:budchoice}) has an impact as well. We found that the simplest branch selection strategy was also the one giving the best results: we expend first the branch that was inserted into \bud\ first (i.e., \bud\ is \emph{FIFO}). One possible explanation is that by avoiding to unnecessarily ``jump'' to different parts of the decision tree, this strategy promotes optimizing sibling subtrees first, and therefore, deeper tree earlier.

% intuitivelly, one want to optimize the branches of the decision trees with the largest error first, in order to benefit from larger lower bounds earlier. To this end, it



 %for any feature test $\afeat \in \abranch \setminus \abranch'$, if 








%is $O((\numex + 2^{\mdepth}\numfeat \log \numfeat) \numfeat^{\mdepth-1})$, which is often better than 




% $$
% \sum_{v \in \{\afeat,\bar{\afeat\}} \frac{|\allex(\abranch \wedge v)|}{|\allex(\abranch)|} \cdot -\sum_{c \in \{\bot,\top\}} \frac{|\setex{c}(\abranch \wedge v)}{|\setex{c}(\abranch)} \log_{2} \frac{|\setex{c}(\abranch \wedge v)}{|\setex{c}(\abranch)}
% $$


% $$
% 2 - \sum\limits_{v \in \{\afeat,\bar{\afeat}\}}\sum\limits_{c \in \{\bot,\top\}} p(v,\setex{c})^2
% $$



\subsection{Preprocessing}

Finally, we use two preprocessing techniques, one on the dataset and one on the features. Although extremely straightforward (and probably not novel), they both have a significant impact.

\paragraph{Dataset reduction.}
It is easy to adapt \budalg (or most decision tree classifiers, actually) to handle weighted datasets by redefining the error as follows, given a weight function $\weight$ on $\allex$:
$$
\error[\abranch] = \min\left( \sum\limits_{x \in \negex(\abranch)}\weight[x], \sum\limits_{x \in \posex(\abranch)}\weight[x] \right)
$$
We can use the weighted version to handle noisy data, by merging duplicated datapoints and suppressing inconsistent datapoints.

Let $\weight_{\bot}$ (resp. $\weight_{\top}$) denote the number of occurrences of $x$ in $\setex{\bot}$ (resp. $\setex{\top}$). We use the weight function $\weight[x] = |\weight_{\bot}(x) - \weight_{\top}(x)|$. Then, for any datapoint $x$, we remove all but one of its occurrences, in $\setex{\bot}$ if $\weight_{\bot}>\weight_{\top}$, in $\setex{\top}$ if $\weight_{\top}>\weight_{\bot}$, and suppress it completely if $\weight_{\top}=\weight_{\bot}$.
The reported error will then need to be offset by the number of pair of suppressed inconsistant datapoints, that is:
$
\sum_{x \in \allex}\min(\weight_{\bot}(x), \weight_{\top}(x))
$.
Reducing the number of datapoints in the dataset has a non-null, although tiny impact on efficiency. However, suppressing inconsistent datapoints is very important. In particular, proving optimality when the minimum error is positive basically requires to exhaust the search space and is therefore extremely costly. On the other hand, when there exists a perfect tree within the maximum depth, we can stop as soon as we find it. This preprocessing allows to benefit from that when we find a tree whose error is equal to the number of pair of inconsistent datapoints in the original dataset.
This preprocessing can be done in $O(\numfeat \numex \log \numex)$ by ordering the datapoints in lexicographic order and then processing them in sequence.

\paragraph{Feature reduction.}

A feature $\afeat$ is redundant if there exists a feature $\afeat'$ such that either: $\forall x \in \allex, \afeat \in x \iff \afeat' \in x$, or $\forall x \in \allex, \afeat \in x \iff \afeat' \in x$. We simply remove such redundant features. This can be done by comparing pairs of rows of the dataset via bitset operations, and therefore in time $O(\numex\numfeat^2)$.

This may appear very naive, however, it turns out that the binarization techniques (one-hot encoding) used to turn general datasets into binary dataset are often not optimized and many redundant features do exist. The number of features ($\numfeat$) has a huge impact on the complexity of the algorithm since the branching factor in the tree representing the search space is $2\numfeat$ (see Figure~\ref{fig:searchtree}).

Moreover, at every branch, ``informationless'' features (i.e., features $\afeat$ such that $(\forall x \in \posex(\abranch) \afeat \in x) \iff (\forall x \in \negex(\abranch) \afeat \in x)$) can be suppressed at no additional cost since this can be detected when computing the feature ordering criterion.




\section{Experimental Results}

All experiments were run
on 7 cluster nodes, each with 36 Intel Xeon CPU E5-2695 v4 2.10GHz cores
running Linux Ubuntu 16.04.4. Sources were compiled using g++8. 
Every algorithm was run ten times with the same set of distinct random seeds and a time limit of one hour. We imposed a memory limit of 3.5GB on every instance in the data sets, except for \dleight, where the limit was raised to 50GB.

We used 47 datasets [from...], and for each dataset, we set the maximum depth using integers in $[3,10]$. We organized the results by providing averaged data over the 47 datasets, for each value of maximum depth.
The raw experimental data from these experiments can be found in the Appendix.



\subsection{Optimizing trees}

We first compare our algorithm (\budalg) to the state-of-the-art algorithms \murtree and \dleight, for computing optimal trees and proving optimality. We therefore report in Table~\ref{tab:summaryacc} the average error (error), the average accuracy (acc.), 
the ratio of optimality proofs (opt.), as well as the average cpu time (cpu) to either prove optimality or find the best solution.
As \dleight does not provide a solution for every dataset (on some instance it goes over the memory limit of 50GB), we provide the number of datasets for which a solution was returned (sol.). Moreover, 
instead of absolute values, we provide the average relative difference in error and accuracy w.r.t. \budalg, however, and only for the datasets where a decision tree was found. Similarly, we report the average cpu time ratio w.r.t. \budalg, however, only for instances which were proven optimal by both algorithms\footnote{every instance proven optimal by \dleight is also proven optimal by \budalg and \murtree}.


% We report in Table~\ref{tab:summaryacc} data averaged over the 47 datasets described above, for
%
%
% the average accuracy found within the one hour time limit for \budalg and \murtree
%
% on relatively shallow trees (3,4 and 5) in tables~\ref{tab:d3}, \ref{tab:d4} and \ref{tab:d5}, respectively.
% We give the minimum \emph{error}, the cpu in seconds \emph{time} and size of the search space (\emph{choices}) required to prove optimality (when a proof is given, as markes by a 1 in the column \emph{opt}) or to find the best solution (otherwise).


% \begin{table}[htbp]
% \begin{center}
% \begin{normalsize}
% \tabcolsep=3.7pt
% \begin{tabular}{lrrrrrrrrr}
\toprule
&  \multicolumn{3}{c}{\budalg} & \multicolumn{3}{c}{\murtree} & \multicolumn{3}{c}{\dleight}\\
\cmidrule(rr){2-4}\cmidrule(rr){5-7}\cmidrule(rr){8-10}
& \multicolumn{1}{c}{error} & \multicolumn{1}{c}{cpu} & \multicolumn{1}{c}{opt.} & \multicolumn{1}{c}{error} & \multicolumn{1}{c}{cpu} & \multicolumn{1}{c}{opt.} & \multicolumn{1}{c}{error$^*$} & \multicolumn{1}{c}{cpu$^*$} & \multicolumn{1}{c}{opt.} \\
\midrule

\texttt{D = 3} & \textbf{1328} & 465 & 0.93 & 1346 & \textbf{234} & 0.93 & $\mathsmaller{+}$190 & $\times$44 & 0.63\\
\texttt{D = 4} & 1144 & \textbf{594} & 0.61 & \textbf{1143} & 731 & 0.61 & $\mathsmaller{+}$416 & $\times$229 & 0.48\\
\texttt{D = 5} & \textbf{1010} & \textbf{826} & 0.52 & 1054 & 951 & 0.52 & $\mathsmaller{+}$738 & $\times$529 & 0.26\\
\texttt{D = 6} & \textbf{889} & 1139 & \textbf{0.41} & 1009 & \textbf{1097} & 0.37 & $\mathsmaller{+}$1050 & $\times$576 & 0.24\\
\texttt{D = 7} & \textbf{789} & 1215 & 0.39 & 944 & \textbf{937} & 0.39 & $\mathsmaller{+}$377 & $\times$179 & 0.24\\
\texttt{D = 8} & \textbf{704} & \textbf{792} & \textbf{0.43} & 861 & 1080 & 0.39 & $\mathsmaller{+}$702 & $\times$3615 & 0.26\\
\texttt{D = 9} & \textbf{637} & \textbf{788} & \textbf{0.43} & 790 & 867 & 0.35 & $\mathsmaller{+}$943 & $\times$3835 & 0.28\\
\texttt{D = 10} & \textbf{575} & \textbf{678} & \textbf{0.52} & 714 & 1002 & 0.39 & $\mathsmaller{+}$1021 & $\times$9725 & 0.30\\
\bottomrule
\end{tabular}

% \end{normalsize}
% \end{center}
% \caption{\label{tab:summary} Comparison with the state of the art: computing optimal trees}
% \end{table}


\begin{table}[htbp]
\begin{center}
\begin{footnotesize}
\tabcolsep=1.5pt
\begin{tabular}{lrrrrrrrrrrrrr}
\toprule
&  \multicolumn{4}{c}{\budalg} & \multicolumn{4}{c}{\murtree} & \multicolumn{5}{c}{\dleight}\\
\cmidrule(rr){2-5}\cmidrule(rr){6-9}\cmidrule(rr){10-14}
& \multicolumn{1}{c}{error} & \multicolumn{1}{c}{acc.} & \multicolumn{1}{c}{cpu} & \multicolumn{1}{c}{opt.} & \multicolumn{1}{c}{error} & \multicolumn{1}{c}{acc.} & \multicolumn{1}{c}{cpu$^*$} & \multicolumn{1}{c}{opt.} & \multicolumn{1}{c}{error$^*$} & \multicolumn{1}{c}{acc.$^*$} & \multicolumn{1}{c}{cpu$^*$} & \multicolumn{1}{c}{sol.} & \multicolumn{1}{c}{opt.} \\
\midrule

\texttt{D = 3} & \textbf{1328} & \textbf{0.8921} & \textbf{465} & 0.93 & 1346 & 0.8917 & $\mathsmaller{\times}$2.44 & 0.93 & $\mathsmaller{+}$190 & -0.3\% & $\mathsmaller{\times}$44 & 0.87 & 0.63\\
\texttt{D = 4} & 1144 & \textbf{0.9133} & \textbf{594} & 0.61 & \textbf{1143} & 0.9129 & $\mathsmaller{\times}$2.87 & 0.61 & $\mathsmaller{+}$416 & -0.7\% & $\mathsmaller{\times}$229 & 0.76 & 0.48\\
\texttt{D = 5} & \textbf{1010} & \textbf{0.9298} & \textbf{826} & 0.52 & 1054 & 0.9282 & $\mathsmaller{\times}$3.55 & 0.52 & $\mathsmaller{+}$738 & -1.3\% & $\mathsmaller{\times}$529 & 0.57 & 0.26\\
\texttt{D = 6} & \textbf{889} & \textbf{0.9436} & \textbf{1139} & \textbf{0.41} & 1009 & 0.9390 & $\mathsmaller{\times}$4.53 & 0.37 & $\mathsmaller{+}$1050 & -1.9\% & $\mathsmaller{\times}$576 & 0.50 & 0.24\\
\texttt{D = 7} & \textbf{789} & \textbf{0.9510} & \textbf{1215} & 0.39 & 944 & 0.9438 & $\mathsmaller{\times}$9.21 & 0.39 & $\mathsmaller{+}$377 & -1.0\% & $\mathsmaller{\times}$179 & 0.35 & 0.24\\
\texttt{D = 8} & \textbf{704} & \textbf{0.9564} & \textbf{792} & \textbf{0.43} & 861 & 0.9501 & $\mathsmaller{\times}$10087 & 0.39 & $\mathsmaller{+}$702 & -1.5\% & $\mathsmaller{\times}$3615 & 0.41 & 0.26\\
\texttt{D = 9} & \textbf{637} & \textbf{0.9609} & \textbf{788} & \textbf{0.43} & 790 & 0.9534 & $\mathsmaller{\times}$4.91 & 0.35 & $\mathsmaller{+}$943 & -2.0\% & $\mathsmaller{\times}$3835 & 0.46 & 0.28\\
\texttt{D = 10} & \textbf{575} & \textbf{0.9650} & \textbf{678} & \textbf{0.52} & 714 & 0.9572 & $\mathsmaller{\times}$5.26 & 0.39 & $\mathsmaller{+}$1021 & -1.9\% & $\mathsmaller{\times}$9725 & 0.48 & 0.30\\
\bottomrule
\end{tabular}

\end{footnotesize}
\end{center}
\caption{\label{tab:summaryacc} Comparison with the state of the art: computing optimal trees}
\end{table}


\subsection{Computing deep classifiers}

Next, we shift our focus to how fast can we obtain accurate trees and how fast can we improve the accuracy over basic solutions found by heuristics.
Therefore, in order to have a baseline, we use a well known heuristic: \cart (we ran its implementation in scikit-learn).
Here we report the average error after a given period of time (3 seconds, 10 seconds, 1 minute or 5 minutes) in Table~\ref{tab:summaryspeed}.


\medskip

We can see that the first solution found by our algorithm has comparable accuracy (actually, often slightly better) to the one found by \cart. The implementation of \cart in scikit-learn does not seem to be very efficient computationally. However, this is not so relevant as it is clear that one greedy run of the heuristic can be implemented to be as fast as the first dive of \budalg. The point of this experiment is threefold: first, it shows that the first solution is very similar to that found by \cart (here we believe that the only difference is to use the feature yielding the minimum error instead of the feature with minimum Gini impurity for the deepest feature tests); second, this tree
is found extremely quickly, and there is no scaling issue with respect to the depth of the tree or with respect to the size of the dataset; third, even for large datasets and deep trees, the accuracy of the initial classifier can be significantly improved given a reasonable computation time.

%Moreover, although we would need larger data sets to be confident about that, it seems that our algorithm is faster than \cart to find this first decision tree. %One can conjecture that \cart uses more sophisticated heuristic choices to explain these two observations.

%Then, in most cases, it is possible to improve the first solution significantly within a few seconds. Notice that for larger depth, improving the initial solution is harder and the 3s time limit is comparatively tighter than for smaller trees, so the gain of \budalg over \cart is more sensible for small trees.


\begin{table}[htbp]
\begin{center}
\begin{footnotesize}
\tabcolsep=1.5pt
\begin{tabular}{lrrrrrrrrrrrrrr}
\toprule
\multirow{2}{*}{$\mdepth$}&  \multicolumn{6}{c}{\budalg} & \multicolumn{6}{c}{\murtree} & \multicolumn{2}{c}{\cart}\\
\cmidrule(rr){2-7}\cmidrule(rr){8-13}\cmidrule(rr){14-15}
& \multicolumn{1}{c}{cpu} & \multicolumn{1}{c}{first} & \multicolumn{1}{c}{$\leq$3s} & \multicolumn{1}{c}{$\leq$10s} & \multicolumn{1}{c}{$\leq$1m} & \multicolumn{1}{c}{$\leq$5m} & \multicolumn{1}{c}{cpu} & \multicolumn{1}{c}{first} & \multicolumn{1}{c}{$\leq$3s} & \multicolumn{1}{c}{$\leq$10s} & \multicolumn{1}{c}{$\leq$1m} & \multicolumn{1}{c}{$\leq$5m} & \multicolumn{1}{c}{cpu} & \multicolumn{1}{c}{first} \\
\midrule

\texttt{3} & 0.03 & 1235 & 1174 & 1154 & 1140 & 1139 & 0.00 & 2542 & 2213 & 1949 & 1739 & 1712 & 1.54 & 1275\\
\texttt{4} & 0.03 & 1067 & 1019 & 1002 & 989 & 976 & 0.00 & 2542 & 2197 & 1933 & 1716 & 1685 & 1.80 & 1084\\
\texttt{5} & 0.03 & 932 & 887 & 882 & 872 & 856 & 0.00 & 2542 & 2207 & 1939 & 1716 & 1679 & 2.56 & 951\\
\texttt{7} & 0.03 & 723 & 694 & 686 & 680 & 665 & 0.00 & 2542 & 2249 & 2001 & 1758 & 1713 & 3.66 & 738\\
\texttt{10} & 0.04 & 534 & 515 & 510 & 497 & 486 & 0.00 & 2542 & 2289 & 2053 & 1813 & 1759 & 4.58 & 546\\
\bottomrule
\end{tabular}

\end{footnotesize}
\end{center}
\caption{\label{tab:summaryspeed} Comparison with state the of the art: computing accurate trees}
\end{table}


Figure~\ref{fig:cactus} reports the evolution of accuracy over time, giving a good view of the difference between \murtree and \budalg during search. The accuracy of the tree returned by \cart is given for reference. 
We can see in those graphs that \murtree finds an initial tree extremely quickly, although with very low accuracy. This is because \murtree show progress even when the tree is not complete, i.e., the first solution is always a single node with the most promising feature. We can see in Table~\ref{tab:summaryspeed} that this is indeed always the same feature, irrespective of the maximum depth.




% \begin{figure}
% 	\subfloat[depth=3]{\cactus{Average Accuracy}{CPU time}{\budalg, \murtree, \cart}{{{(0.8727813560274735, 0) [a] 
(0.8823711531560628, 0.01) [a] 
(0.8842343476005071, 0.02) [a] 
(0.8847995559338404, 0.03) [a] 
(0.8850485142671738, 0.04) [a] 
(0.8852078892671738, 0.05) [a] 
(0.8853278892671738, 0.06) [a] 
(0.8854310142671736, 0.08) [a] 
(0.8854808059338403, 0.09) [a] 
(0.8856505976005069, 0.11) [a] 
(0.8857191392671736, 0.14) [a] 
(0.885836222600507, 0.16) [a] 
(0.8858418476005069, 0.2) [a] 
(0.8871232078527561, 0.22) [a] 
(0.8871967495194228, 0.23) [a] 
(0.8872290411860895, 0.24) [a] 
(0.8872375828527561, 0.25) [a] 
(0.8872421661860894, 0.26) [a] 
(0.8873004995194228, 0.32) [a] 
(0.8873390411860894, 0.33) [a] 
(0.8873504995194228, 0.34) [a] 
(0.8873761245194228, 0.36) [a] 
(0.8873959161860895, 0.41) [a] 
(0.8875129995194229, 0.51) [a] 
(0.8919259415711882, 0.54) [a] 
(0.8919311499045215, 0.55) [a] 
(0.8919419832378548, 0.56) [a] 
(0.8919467749045215, 0.57) [a] 
(0.8919551082378547, 0.62) [a] 
(0.8920738582378547, 0.64) [a] 
(0.8924577083333329, 0.83) [a] 
(0.8924581249999995, 0.86) [a] 
(0.8924583333333329, 0.96) [a] 
(0.8924704166666663, 1.17) [a] 
(0.8924706249999996, 1.24) [a] 
(0.8924714583333329, 1.25) [a] 
(0.8924729166666662, 1.26) [a] 
(0.8924770833333329, 1.39) [a] 
(0.8925260416666662, 1.43) [a] 
(0.8925262499999995, 1.44) [a] 
(0.8925281249999996, 1.65) [a] 
(0.8925283333333329, 1.78) [a] 
(0.8925291666666663, 1.8) [a] 
(0.8925295833333329, 2.07) [a] 
(0.8925329166666662, 2.37) [a] 
(0.8926022916666663, 2.39) [a] 
(0.892638333333333, 2.72) [a] 
(0.8926847916666664, 2.73) [a] 
(0.8926970833333331, 2.76) [a] 
(0.8926991666666664, 2.81) [a] 
(0.8927729166666664, 2.83) [a] 
(0.8927741666666663, 2.84) [a] 
(0.892797083333333, 2.85) [a] 
(0.8927999999999996, 3.35) [a] 
(0.892800208333333, 3.52) [a] 
(0.8928010416666663, 3.65) [a] 
(0.8928012499999997, 3.66) [a] 
(0.8928062499999997, 4.07) [a] 
(0.8929593749999997, 5.84) [a] 
(0.8930062499999997, 5.96) [a] 
(0.8930181249999997, 6.09) [a] 
(0.8930337499999996, 6.1) [a] 
(0.8930366666666663, 6.47) [a] 
(0.8930437499999996, 6.71) [a] 
(0.8930545833333329, 6.74) [a] 
(0.8930616666666662, 6.98) [a] 
(0.8930660416666663, 7.17) [a] 
(0.8930937499999996, 7.21) [a] 
(0.8931166666666662, 7.22) [a] 
(0.8931197916666662, 7.27) [a] 
(0.8931268749999995, 7.39) [a] 
(0.8931339583333329, 7.73) [a] 
(0.8931383333333329, 7.82) [a] 
(0.8931454166666662, 8.06) [a] 
(0.8931510416666663, 8.18) [a] 
(0.8931568749999996, 8.38) [a] 
(0.8931666666666663, 8.71) [a] 
(0.8931681249999996, 8.74) [a] 
(0.893169583333333, 8.81) [a] 
(0.893175208333333, 9) [a] 
(0.8931810416666663, 9.17) [a] 
(0.893185208333333, 9.32) [a] 
(0.8931933333333331, 9.54) [a] 
(0.893196458333333, 9.55) [a] 
(0.8931985416666663, 9.56) [a] 
(0.8932091666666664, 9.57) [a] 
(0.8932287499999997, 9.58) [a] 
(0.893242083333333, 9.61) [a] 
(0.8932479166666664, 9.86) [a] 
(0.8932549999999997, 10.04) [a] 
(0.893262083333333, 10.32) [a] 
(0.8932691666666663, 10.56) [a] 
(0.8932735416666663, 10.77) [a] 
(0.8932791666666663, 10.93) [a] 
(0.8932797916666663, 11.35) [a] 
(0.8932802083333329, 11.38) [a] 
(0.8933022916666662, 13.79) [a] 
(0.8933027083333328, 13.8) [a] 
(0.8933029166666662, 14.27) [a] 
(0.8933714583333329, 16.51) [a] 
(0.8934116666666662, 17.05) [a] 
(0.8934118749999995, 17.57) [a] 
(0.8934149999999995, 17.6) [a] 
(0.8934345833333328, 17.61) [a] 
(0.8934366666666661, 17.65) [a] 
(0.893437916666666, 17.68) [a] 
(0.8934574999999993, 22.11) [a] 
(0.8934606249999993, 24.38) [a] 
(0.8934612499999993, 28.2) [a] 
(0.8934706249999993, 28.55) [a] 
(0.8934939583333326, 36.23) [a] 
(0.8935277083333326, 36.84) [a] 
(0.8935449999999993, 37.77) [a] 
(0.8935493749999992, 39.14) [a] 
(0.8935849999999992, 40.23) [a] 
(0.8935856249999992, 40.39) [a] 
(0.8936012499999991, 41.84) [a] 
(0.8936054166666658, 42.39) [a] 
(0.8936083333333324, 42.4) [a] 
(0.8936097916666658, 42.41) [a] 
(0.8936127083333324, 42.42) [a] 
(0.8936210416666658, 48.04) [a] 
(0.8936245833333325, 48.05) [a] 
(0.8936302083333325, 48.51) [a] 
(0.8936316666666658, 48.62) [a] 
(0.8936331249999991, 48.93) [a] 
(0.8936387499999991, 48.99) [a] 
(0.8936458333333325, 49.44) [a] 
(0.8936502083333324, 49.95) [a] 
(0.8936829166666658, 50.22) [a] 
(0.8936835416666657, 72.07) [a] 
(0.8936841666666657, 72.18) [a] 
(0.8936927083333324, 102.49) [a] 
(0.893695624999999, 108.77) [a] 
(0.8936958333333324, 108.8) [a] 
(0.8937066666666657, 146.63) [a] 
(0.8937129166666656, 198.87) [a] 
(0.8937166666666656, 235.62) [a] 
(0.8937181249999989, 259.14) [a] 
(0.8937310416666656, 570.34) [a] 
(0.8937410416666656, 953.22) [a] 
(0.8937466666666656, 953.29) [a] 
(0.8937481249999989, 953.31) [a] 
},{(0.8116311278759438, 0) [b] 
(0.8361056048198612, 0.001) [b] 
(0.8371783536009039, 0.002) [b] 
(0.8411755236815099, 0.003) [b] 
(0.8464734847388493, 0.004) [b] 
(0.848048570166541, 0.005) [b] 
(0.848289273160767, 0.006) [b] 
(0.8483033117950887, 0.007) [b] 
(0.8484309279257676, 0.008) [b] 
(0.8509740005206241, 0.009) [b] 
(0.8510573338539574, 0.011) [b] 
(0.8577216193324813, 0.012) [b] 
(0.8586021220167175, 0.013) [b] 
(0.8588893368534153, 0.015) [b] 
(0.8589808449061239, 0.019) [b] 
(0.8591329130326446, 0.02) [b] 
(0.8591890675699312, 0.022) [b] 
(0.8638218058307201, 0.023) [b] 
(0.8642262974174529, 0.024) [b] 
(0.8657545105835971, 0.028) [b] 
(0.865805824212497, 0.032) [b] 
(0.8658571378413968, 0.033) [b] 
(0.8667356184036245, 0.039) [b] 
(0.8671982920147355, 0.047) [b] 
(0.8679399575929313, 0.049) [b] 
(0.8685650560956893, 0.05) [b] 
(0.8686393764026997, 0.051) [b] 
(0.8686473625138108, 0.052) [b] 
(0.8706526005242572, 0.053) [b] 
(0.8706831032084934, 0.06) [b] 
(0.8709140145385427, 0.064) [b] 
(0.8720020222754866, 0.066) [b] 
(0.8722329336055359, 0.067) [b] 
(0.8723516420575776, 0.077) [b] 
(0.8725018510439936, 0.082) [b] 
(0.8728265263686689, 0.084) [b] 
(0.8733667368158692, 0.085) [b] 
(0.8733986408485389, 0.097) [b] 
(0.8743349704365538, 0.108) [b] 
(0.8753883412230707, 0.109) [b] 
(0.8755114696706673, 0.112) [b] 
(0.8770330052511917, 0.114) [b] 
(0.8775497282876441, 0.123) [b] 
(0.877609082513665, 0.13) [b] 
(0.8776541763087587, 0.132) [b] 
(0.8778153626401578, 0.133) [b] 
(0.8778472666728275, 0.134) [b] 
(0.8778536828518646, 0.148) [b] 
(0.8780765995185312, 0.16) [b] 
(0.8782464009814361, 0.165) [b] 
(0.8783421130794453, 0.231) [b] 
(0.878725753324975, 0.259) [b] 
(0.8794916114276968, 0.282) [b] 
(0.8795555574325155, 0.297) [b] 
(0.8796725986310173, 0.326) [b] 
(0.8796753786327965, 0.334) [b] 
(0.8797156752156463, 0.335) [b] 
(0.87994975761265, 0.344) [b] 
(0.8799789242793167, 0.455) [b] 
(0.8800087853904278, 0.523) [b] 
(0.8801258265889297, 0.543) [b] 
(0.8801320765889297, 0.727) [b] 
(0.8806483960333741, 0.818) [b] 
(0.8817862432555963, 0.822) [b] 
(0.8821167988111518, 0.936) [b] 
(0.8822338400096537, 1.027) [b] 
(0.8828685622318759, 1.072) [b] 
(0.8828692566763203, 1.095) [b] 
(0.8829862978748222, 1.184) [b] 
(0.883105006326864, 1.408) [b] 
(0.883142506326864, 1.459) [b] 
(0.883148756326864, 1.792) [b] 
(0.8832935479935307, 1.876) [b] 
(0.8877059780475183, 1.936) [b] 
(0.8879993808252961, 2.06) [b] 
(0.8883702141586294, 2.109) [b] 
(0.8883785541639669, 2.154) [b] 
(0.8889424430528559, 2.247) [b] 
(0.8889493874973003, 2.263) [b] 
(0.8894979986084115, 2.279) [b] 
(0.8898865402750782, 2.28) [b] 
(0.8899146652750782, 2.301) [b] 
(0.8904226513861894, 2.313) [b] 
(0.8904229986084116, 2.32) [b] 
(0.8906323736084115, 2.353) [b] 
(0.8906354986084115, 2.407) [b] 
(0.8908198736084115, 2.471) [b] 
(0.8910709152750782, 2.482) [b] 
(0.8912393180528559, 2.497) [b] 
(0.8914462624973004, 2.603) [b] 
(0.8914927902750782, 2.611) [b] 
(0.8915125819417449, 2.726) [b] 
(0.8915584152750782, 2.732) [b] 
(0.8915966097195226, 2.823) [b] 
(0.891631679163967, 2.871) [b] 
(0.8917066791639671, 2.892) [b] 
(0.8917556374973004, 2.944) [b] 
(0.8917823736084115, 3.103) [b] 
(0.8917882763861893, 3.17) [b] 
(0.8918108458306337, 3.319) [b] 
(0.8918118874973004, 3.96) [b] 
(0.8918657069417449, 4.557) [b] 
(0.8918882763861894, 4.645) [b] 
(0.8919365402750783, 5.057) [b] 
(0.8919559847195228, 7.42) [b] 
(0.8919570263861895, 7.815) [b] 
(0.8919973229690392, 8.91) [b] 
(0.8920442977905348, 9.064) [b] 
(0.8921106172349793, 10.628) [b] 
(0.8921206866794238, 11.149) [b] 
(0.892149158901646, 11.604) [b] 
(0.892152283901646, 15.528) [b] 
(0.892305408901646, 21.544) [b] 
(0.8923061033460904, 22.093) [b] 
(0.8923217283460904, 23.043) [b] 
(0.892372770012757, 23.828) [b] 
(0.8923960339016459, 27.185) [b] 
(0.8924005477905348, 27.745) [b] 
(0.8924123533460904, 32.109) [b] 
(0.8924137422349793, 50.679) [b] 
(0.8924245061238681, 73.439) [b] 
(0.8924276311238681, 76.434) [b] 
(0.8924314505683126, 98.767) [b] 
(0.8924377005683126, 99.756) [b] 
(0.8924505477905348, 311.363) [b] 
(0.8930555472211236, 489.37) [b] 
(0.8933359822513024, 564.835) [b] 
(0.8933445234197342, 577.042) [b] 
(0.89334525599213, 611.124) [b] 
(0.8933623383289937, 3522.86) [b] 
},{(0.8784109791666664, 0.001) [c] 
(0.8784109791666664, 1.7516523333333336) [c] 
(0.8784109791666664, 3600) [c] 
}}}{legend pos=north west}}
% 	\subfloat[depth=5]{\cactus{Average Accuracy}{CPU time}{\budalg, \murtree, \cart}{{{(0.8976572034057506, 0) [a] 
(0.9060691392671738, 0.01) [a] 
(0.9085149031560626, 0.02) [a] 
(0.9130899726005072, 0.03) [a] 
(0.9134299726005072, 0.04) [a] 
(0.9139026809338406, 0.05) [a] 
(0.9142541392671738, 0.06) [a] 
(0.914311847600507, 0.07) [a] 
(0.9143391392671737, 0.08) [a] 
(0.9145468476005072, 0.09) [a] 
(0.9146047642671737, 0.1) [a] 
(0.9146778892671737, 0.11) [a] 
(0.9147153892671738, 0.12) [a] 
(0.9147568476005071, 0.13) [a] 
(0.9147970559338404, 0.14) [a] 
(0.9148520559338404, 0.15) [a] 
(0.9151424726005073, 0.16) [a] 
(0.9152299726005072, 0.17) [a] 
(0.9152641392671739, 0.18) [a] 
(0.9153966392671739, 0.19) [a] 
(0.9154995559338406, 0.2) [a] 
(0.9155283059338406, 0.21) [a] 
(0.917732999519423, 0.22) [a] 
(0.9177475828527564, 0.23) [a] 
(0.9177942495194231, 0.24) [a] 
(0.9179975828527563, 0.25) [a] 
(0.9179984161860897, 0.26) [a] 
(0.918052374519423, 0.27) [a] 
(0.9182340411860896, 0.28) [a] 
(0.918236749519423, 0.29) [a] 
(0.9183409161860896, 0.3) [a] 
(0.9184646661860896, 0.31) [a] 
(0.9184652911860895, 0.32) [a] 
(0.9184950828527562, 0.33) [a] 
(0.9185211245194228, 0.34) [a] 
(0.9185252911860895, 0.35) [a] 
(0.9185467495194228, 0.36) [a] 
(0.9185677911860896, 0.37) [a] 
(0.9185679995194229, 0.39) [a] 
(0.9186194578527562, 0.4) [a] 
(0.9186788328527563, 0.41) [a] 
(0.9187107078527562, 0.42) [a] 
(0.9187346661860895, 0.43) [a] 
(0.9187554995194228, 0.44) [a] 
(0.9187971661860895, 0.45) [a] 
(0.9232117749045216, 0.47) [a] 
(0.9232128165711881, 0.48) [a] 
(0.9232134415711881, 0.49) [a] 
(0.9232396915711879, 0.5) [a] 
(0.9232405249045211, 0.51) [a] 
(0.923243024904521, 0.52) [a] 
(0.923244899904521, 0.53) [a] 
(0.923248024904521, 0.55) [a] 
(0.9232573999045209, 0.56) [a] 
(0.9232601082378542, 0.57) [a] 
(0.9232665665711876, 0.58) [a] 
(0.9232676082378543, 0.6) [a] 
(0.9232971915711876, 0.61) [a] 
(0.9233003165711876, 0.63) [a] 
(0.9233007332378542, 0.65) [a] 
(0.9233251082378543, 0.66) [a] 
(0.9236878165711876, 0.68) [a] 
(0.9237296915711876, 0.69) [a] 
(0.9237767749045208, 0.71) [a] 
(0.9239569832378541, 0.72) [a] 
(0.9239973999045208, 0.73) [a] 
(0.9240455249045207, 0.74) [a] 
(0.9240726082378541, 0.75) [a] 
(0.9240730249045207, 0.77) [a] 
(0.9241434415711873, 0.78) [a] 
(0.9241503165711874, 0.8) [a] 
(0.9241509415711874, 0.81) [a] 
(0.9241530249045207, 0.82) [a] 
(0.9241555249045207, 0.83) [a] 
(0.924158233237854, 0.84) [a] 
(0.9241584415711873, 0.85) [a] 
(0.9241830249045206, 0.86) [a] 
(0.9246581249999989, 0.87) [a] 
(0.9246672916666655, 0.88) [a] 
(0.9246677083333321, 0.93) [a] 
(0.9246756249999988, 0.95) [a] 
(0.9247531249999987, 0.97) [a] 
(0.9247887499999988, 0.98) [a] 
(0.9248191666666655, 1.01) [a] 
(0.9248333333333322, 1.04) [a] 
(0.9248337499999988, 1.05) [a] 
(0.9248397916666654, 1.06) [a] 
(0.9248722916666654, 1.07) [a] 
(0.9249124999999987, 1.1) [a] 
(0.9249160416666654, 1.11) [a] 
(0.9249966666666654, 1.12) [a] 
(0.924997083333332, 1.16) [a] 
(0.9249974999999986, 1.18) [a] 
(0.925000208333332, 1.21) [a] 
(0.925002083333332, 1.25) [a] 
(0.9250135416666654, 1.26) [a] 
(0.9250187499999988, 1.32) [a] 
(0.9250197916666654, 1.34) [a] 
(0.9250479166666654, 1.45) [a] 
(0.9250797916666654, 1.46) [a] 
(0.9251760416666654, 1.47) [a] 
(0.9252464583333321, 1.5) [a] 
(0.925247083333332, 1.55) [a] 
(0.9254083333333321, 1.62) [a] 
(0.925499583333332, 1.73) [a] 
(0.9255060416666654, 1.74) [a] 
(0.9255974999999987, 1.81) [a] 
(0.9256281249999987, 1.82) [a] 
(0.925634583333332, 1.92) [a] 
(0.9256672916666654, 1.93) [a] 
(0.925667708333332, 1.94) [a] 
(0.9256806249999987, 2.06) [a] 
(0.9256872916666654, 2.09) [a] 
(0.9257654166666655, 2.1) [a] 
(0.9257666666666654, 2.16) [a] 
(0.9257681249999987, 2.19) [a] 
(0.925792708333332, 2.27) [a] 
(0.925852083333332, 2.29) [a] 
(0.925852708333332, 2.31) [a] 
(0.9258543749999987, 2.35) [a] 
(0.9258552083333319, 2.36) [a] 
(0.9258558333333319, 2.39) [a] 
(0.9259152083333319, 2.4) [a] 
(0.9259162499999984, 2.42) [a] 
(0.9259164583333318, 2.43) [a] 
(0.9259756249999985, 2.49) [a] 
(0.9259772916666652, 2.55) [a] 
(0.9259777083333318, 2.56) [a] 
(0.9259804166666652, 2.57) [a] 
(0.9259808333333318, 2.59) [a] 
(0.9259818749999985, 2.61) [a] 
(0.9259877083333319, 2.65) [a] 
(0.9259887499999986, 2.79) [a] 
(0.9259897916666653, 2.8) [a] 
(0.9259902083333319, 2.84) [a] 
(0.9259910416666651, 2.85) [a] 
(0.9259914583333317, 2.9) [a] 
(0.9260574999999984, 2.92) [a] 
(0.9260599999999984, 2.93) [a] 
(0.9261193749999984, 2.97) [a] 
(0.9261449999999984, 3.01) [a] 
(0.9261706249999985, 3.02) [a] 
(0.9261964583333318, 3.13) [a] 
(0.9261983333333318, 3.17) [a] 
(0.9262006249999984, 3.2) [a] 
(0.9262014583333317, 3.23) [a] 
(0.926201666666665, 3.25) [a] 
(0.926202291666665, 3.44) [a] 
(0.9262864583333317, 3.49) [a] 
(0.9262870833333317, 3.5) [a] 
(0.9263156249999982, 3.52) [a] 
(0.9263435416666648, 3.53) [a] 
(0.9263445833333315, 3.61) [a] 
(0.9263702083333315, 3.62) [a] 
(0.9263958333333315, 3.63) [a] 
(0.9263985416666649, 3.73) [a] 
(0.9264243749999982, 3.75) [a] 
(0.9265854166666648, 3.81) [a] 
(0.9266660416666648, 3.84) [a] 
(0.9266856249999981, 3.92) [a] 
(0.9267183333333314, 3.93) [a] 
(0.9267885416666648, 4.01) [a] 
(0.926835624999998, 4.28) [a] 
(0.9268589583333314, 4.29) [a] 
(0.926882499999998, 4.35) [a] 
(0.9268827083333314, 4.55) [a] 
(0.9268835416666648, 4.58) [a] 
(0.9268841666666647, 4.67) [a] 
(0.9268847916666647, 4.8) [a] 
(0.9269129166666646, 4.82) [a] 
(0.9269585416666647, 4.84) [a] 
(0.9269858333333314, 4.89) [a] 
(0.9269885416666648, 5.22) [a] 
(0.9270156249999981, 5.24) [a] 
(0.9270364583333315, 5.4) [a] 
(0.9271068749999981, 5.46) [a] 
(0.9271124999999981, 5.63) [a] 
(0.9271424999999981, 5.67) [a] 
(0.9272129166666647, 6) [a] 
(0.927214999999998, 6.01) [a] 
(0.9272158333333314, 6.2) [a] 
(0.9272170833333313, 6.22) [a] 
(0.9272406249999979, 6.47) [a] 
(0.9272433333333313, 6.55) [a] 
(0.9273070833333313, 6.69) [a] 
(0.9273389583333314, 6.71) [a] 
(0.9273708333333314, 6.72) [a] 
(0.9273718749999981, 6.94) [a] 
(0.9273729166666648, 6.99) [a] 
(0.9274433333333314, 7.33) [a] 
(0.9274477083333313, 7.46) [a] 
(0.9274497916666646, 7.63) [a] 
(0.927464374999998, 7.64) [a] 
(0.927470624999998, 7.71) [a] 
(0.927547499999998, 8.24) [a] 
(0.9275477083333313, 8.28) [a] 
(0.9276247916666647, 8.31) [a] 
(0.9276252083333313, 8.7) [a] 
(0.927628749999998, 9.05) [a] 
(0.927705624999998, 9.27) [a] 
(0.9277827083333313, 9.34) [a] 
(0.9277883333333313, 9.46) [a] 
(0.9277958333333314, 9.54) [a] 
(0.927796249999998, 9.66) [a] 
(0.927796874999998, 10.31) [a] 
(0.9278562499999979, 10.58) [a] 
(0.9278704166666646, 10.71) [a] 
(0.9279872916666646, 11.08) [a] 
(0.928104374999998, 15.76) [a] 
(0.928129999999998, 16.03) [a] 
(0.9281366666666647, 16.15) [a] 
(0.9282537499999981, 16.7) [a] 
(0.9282793749999981, 17.75) [a] 
(0.9282820833333315, 17.94) [a] 
(0.9282831249999982, 17.95) [a] 
(0.9282835416666648, 17.99) [a] 
(0.9283445833333315, 18.03) [a] 
(0.9283702083333315, 19.25) [a] 
(0.9284006249999982, 19.91) [a] 
(0.9284012499999982, 20.09) [a] 
(0.9284020833333315, 20.1) [a] 
(0.9284027083333315, 20.16) [a] 
(0.9284041666666648, 21.08) [a] 
(0.9284297916666648, 21.25) [a] 
(0.9284312499999982, 21.47) [a] 
(0.9284318749999981, 21.84) [a] 
(0.9284329166666648, 22.07) [a] 
(0.9284333333333314, 23.03) [a] 
(0.9284343749999981, 23.06) [a] 
(0.9284570833333314, 23.55) [a] 
(0.9284585416666647, 23.59) [a] 
(0.9284591666666647, 23.61) [a] 
(0.9284656249999981, 23.81) [a] 
(0.9284772916666647, 23.99) [a] 
(0.9284779166666647, 24.02) [a] 
(0.9284816666666647, 24.93) [a] 
(0.9285120833333314, 24.97) [a] 
(0.928513124999998, 24.99) [a] 
(0.928513749999998, 25.02) [a] 
(0.9285152083333313, 25.09) [a] 
(0.9285202083333314, 25.19) [a] 
(0.9285266666666647, 25.68) [a] 
(0.9285779166666647, 25.91) [a] 
(0.9286035416666647, 25.96) [a] 
(0.9286097916666647, 25.97) [a] 
(0.9286104166666647, 26.13) [a] 
(0.9286145833333314, 26.15) [a] 
(0.9286152083333313, 26.17) [a] 
(0.928618124999998, 26.41) [a] 
(0.9286222916666647, 27.9) [a] 
(0.9286735416666646, 28.47) [a] 
(0.9286991666666646, 28.52) [a] 
(0.9287131249999979, 28.54) [a] 
(0.9287270833333312, 28.55) [a] 
(0.9287343749999979, 28.66) [a] 
(0.9287349999999979, 29.71) [a] 
(0.9287656249999978, 31.92) [a] 
(0.9288827083333312, 32.84) [a] 
(0.9289072916666645, 33.38) [a] 
(0.9289906249999978, 36.02) [a] 
(0.9290322916666646, 36.13) [a] 
(0.9290947916666646, 36.14) [a] 
(0.9291364583333314, 36.16) [a] 
(0.9291572916666647, 36.22) [a] 
(0.9291956249999981, 38.47) [a] 
(0.9291979166666647, 38.72) [a] 
(0.9291995833333314, 38.94) [a] 
(0.929199999999998, 38.98) [a] 
(0.9292004166666646, 39.14) [a] 
(0.9292239583333313, 39.23) [a] 
(0.9292245833333312, 39.53) [a] 
(0.929225624999998, 39.57) [a] 
(0.9292285416666646, 40.1) [a] 
(0.9292295833333313, 40.99) [a] 
(0.9292322916666647, 42.88) [a] 
(0.9292335416666646, 44.54) [a] 
(0.9292349999999979, 44.78) [a] 
(0.9292720833333312, 46.23) [a] 
(0.9292724999999978, 47.09) [a] 
(0.9292741666666645, 51.35) [a] 
(0.9292762499999978, 51.36) [a] 
(0.9292768749999978, 52.52) [a] 
(0.9292779166666645, 52.53) [a] 
(0.9292783333333311, 52.56) [a] 
(0.9292797916666644, 52.58) [a] 
(0.9292808333333311, 57.12) [a] 
(0.9292839583333311, 57.14) [a] 
(0.9292847916666644, 57.2) [a] 
(0.9292860416666644, 58.42) [a] 
(0.9293262499999977, 63.21) [a] 
(0.9293272916666643, 65.33) [a] 
(0.9293543749999976, 69.17) [a] 
(0.9293599999999976, 71.35) [a] 
(0.9293631249999976, 71.63) [a] 
(0.9293635416666642, 71.64) [a] 
(0.9293641666666642, 72.69) [a] 
(0.9293645833333308, 72.79) [a] 
(0.9293656249999975, 78.4) [a] 
(0.9293906249999975, 82.68) [a] 
(0.9293918749999974, 82.69) [a] 
(0.9293933333333307, 82.7) [a] 
(0.9293937499999974, 84.51) [a] 
(0.9294012499999974, 85.24) [a] 
(0.9294020833333307, 85.25) [a] 
(0.9294110416666641, 85.43) [a] 
(0.9294166666666641, 87.01) [a] 
(0.9294181249999974, 91.52) [a] 
(0.929418541666664, 91.55) [a] 
(0.9294193749999974, 91.58) [a] 
(0.9294204166666641, 91.65) [a] 
(0.9294208333333307, 91.66) [a] 
(0.9294268749999974, 94.98) [a] 
(0.9294302083333308, 94.99) [a] 
(0.9294306249999974, 95) [a] 
(0.9294331249999974, 95.05) [a] 
(0.9294339583333308, 104.63) [a] 
(0.9294381249999975, 111.54) [a] 
(0.9294383333333308, 127.17) [a] 
(0.9294481249999975, 129.95) [a] 
(0.9294487499999975, 137.63) [a] 
(0.9294633333333309, 143.13) [a] 
(0.9295227083333308, 143.99) [a] 
(0.9295233333333308, 144.15) [a] 
(0.9295243749999975, 144.16) [a] 
(0.9295314583333308, 145.19) [a] 
(0.9295377083333308, 146.58) [a] 
(0.9295414583333308, 146.81) [a] 
(0.9295445833333308, 151.49) [a] 
(0.9295452083333308, 154.44) [a] 
(0.9295502083333308, 154.45) [a] 
(0.9295558333333308, 156.32) [a] 
(0.9295579166666641, 156.33) [a] 
(0.9295602083333308, 156.53) [a] 
(0.9295668749999975, 159.31) [a] 
(0.9295683333333308, 159.32) [a] 
(0.9295695833333307, 159.82) [a] 
(0.9295699999999973, 161.01) [a] 
(0.9295702083333307, 170.48) [a] 
(0.9295737499999974, 192.74) [a] 
(0.9295914583333308, 197.27) [a] 
(0.9295924999999975, 197.46) [a] 
(0.9295968749999974, 202.57) [a] 
(0.9296041666666641, 202.66) [a] 
(0.9297212499999975, 203.14) [a] 
(0.9297356249999975, 203.91) [a] 
(0.9297393749999975, 204.01) [a] 
(0.9297397916666641, 204.13) [a] 
(0.9297629166666641, 210.66) [a] 
(0.9297804166666641, 210.94) [a] 
(0.9297879166666642, 211.03) [a] 
(0.9297960416666642, 211.48) [a] 
(0.9298039583333308, 211.57) [a] 
(0.9298170833333308, 211.85) [a] 
(0.9298199999999974, 238.11) [a] 
(0.9298433333333308, 241.96) [a] 
(0.9298668749999974, 242.71) [a] 
(0.9298683333333307, 260.63) [a] 
(0.9298687499999974, 261.33) [a] 
(0.929872916666664, 264.08) [a] 
(0.9298977083333307, 268.2) [a] 
(0.9299004166666641, 283.51) [a] 
(0.9299114583333308, 283.78) [a] 
(0.9299197916666642, 283.79) [a] 
(0.9299479166666642, 283.81) [a] 
(0.9299487499999975, 285.19) [a] 
(0.9299510416666642, 285.5) [a] 
(0.9299514583333308, 288.07) [a] 
(0.9299629166666642, 294.51) [a] 
(0.9299656249999976, 294.58) [a] 
(0.9299662499999976, 299.39) [a] 
(0.9299983333333309, 302.82) [a] 
(0.9299987499999975, 303.82) [a] 
(0.9300039583333308, 390.04) [a] 
(0.9300199999999975, 390.05) [a] 
(0.9300227083333309, 390.12) [a] 
(0.9300462499999975, 391.94) [a] 
(0.9300560416666642, 395.26) [a] 
(0.9300637499999975, 397.09) [a] 
(0.9300649999999975, 403.19) [a] 
(0.9300664583333308, 403.35) [a] 
(0.9301835416666642, 410.51) [a] 
(0.9301837499999975, 418.82) [a] 
(0.9301839583333309, 419.45) [a] 
(0.9301870833333309, 421.53) [a] 
(0.9301893749999975, 440.68) [a] 
(0.9301935416666642, 452.12) [a] 
(0.9302529166666642, 490.66) [a] 
(0.9302539583333309, 496.93) [a] 
(0.9302545833333309, 496.94) [a] 
(0.9302554166666642, 496.95) [a] 
(0.9302591666666642, 498.64) [a] 
(0.9302597916666642, 498.69) [a] 
(0.9302606249999975, 498.8) [a] 
(0.9302610416666641, 498.91) [a] 
(0.9302612499999975, 500.97) [a] 
(0.9302614583333308, 501.69) [a] 
(0.9302624999999975, 501.81) [a] 
(0.9303218749999975, 506) [a] 
(0.9303222916666641, 534.73) [a] 
(0.9303243749999974, 534.8) [a] 
(0.930324791666664, 534.9) [a] 
(0.9303662499999974, 543.34) [a] 
(0.930366666666664, 553.63) [a] 
(0.9303706249999973, 582.49) [a] 
(0.930372291666664, 582.66) [a] 
(0.930374791666664, 582.67) [a] 
(0.9303749999999974, 592.97) [a] 
(0.9303752083333308, 630.13) [a] 
(0.9303760416666641, 630.16) [a] 
(0.9303766666666641, 630.35) [a] 
(0.9303770833333307, 630.85) [a] 
(0.930382291666664, 632.37) [a] 
(0.9303835416666639, 632.38) [a] 
(0.9303845833333306, 656.59) [a] 
(0.930409166666664, 690.24) [a] 
(0.9304102083333307, 707.86) [a] 
(0.9304133333333306, 712.65) [a] 
(0.930414166666664, 722.32) [a] 
(0.9304143749999974, 722.86) [a] 
(0.930416666666664, 729.97) [a] 
(0.9304570833333307, 776.98) [a] 
(0.9304581249999974, 802.4) [a] 
(0.9304591666666641, 802.45) [a] 
(0.9304606249999974, 805.93) [a] 
(0.930461041666664, 871.02) [a] 
(0.9304612499999974, 886.23) [a] 
(0.930461666666664, 886.8) [a] 
(0.9304627083333307, 896.33) [a] 
(0.9304637499999974, 899.38) [a] 
(0.9304658333333307, 909.41) [a] 
(0.9304720833333306, 909.52) [a] 
(0.9304956249999973, 1135.4) [a] 
(0.9304960416666639, 1151.9) [a] 
(0.9306131249999973, 1154.5) [a] 
(0.9306135416666639, 1182.4) [a] 
(0.9306137499999972, 1182.5) [a] 
(0.9306166666666639, 1312.8) [a] 
(0.9306399999999972, 1420.6) [a] 
(0.9306635416666639, 1427.2) [a] 
(0.9306870833333305, 1427.7) [a] 
(0.9306912499999972, 1474.5) [a] 
(0.9306943749999972, 1569.5) [a] 
(0.9306954166666638, 1691.4) [a] 
(0.9306958333333304, 1693.3) [a] 
(0.9306970833333303, 1695) [a] 
(0.9307270833333303, 1700.9) [a] 
(0.930731874999997, 1701) [a] 
(0.930738749999997, 1706.8) [a] 
(0.9307393749999969, 1792.6) [a] 
(0.930773749999997, 1915.2) [a] 
(0.9307789583333302, 1915.3) [a] 
(0.9307818749999969, 1915.8) [a] 
(0.9307877083333302, 1915.9) [a] 
(0.9307945833333302, 1935.6) [a] 
(0.9307949999999968, 1982.3) [a] 
(0.9307954166666634, 1982.4) [a] 
(0.9307956249999968, 1982.9) [a] 
(0.9307960416666634, 2030) [a] 
(0.9307974999999967, 2096.3) [a] 
(0.9307985416666633, 2097.2) [a] 
(0.93079958333333, 2097.4) [a] 
(0.93080083333333, 2162.3) [a] 
(0.9308037499999966, 2162.5) [a] 
(0.9308052083333299, 2162.6) [a] 
(0.9308066666666632, 2162.7) [a] 
(0.9308093749999966, 2162.9) [a] 
(0.9308137499999966, 2163) [a] 
(0.9308152083333299, 2163.1) [a] 
(0.9308193749999966, 2163.2) [a] 
(0.9308208333333299, 2163.3) [a] 
(0.9308237499999965, 2163.4) [a] 
(0.9308252083333298, 2163.6) [a] 
(0.9308279166666632, 2163.8) [a] 
(0.9308293749999965, 2164) [a] 
(0.9308308333333298, 2164.1) [a] 
(0.9308322916666631, 2164.2) [a] 
(0.9308337499999965, 2164.4) [a] 
(0.9308349999999964, 2164.5) [a] 
(0.9308364583333297, 2164.7) [a] 
(0.930837916666663, 2164.8) [a] 
(0.9308408333333297, 2165.5) [a] 
(0.930842291666663, 2166) [a] 
(0.9308435416666629, 2167.2) [a] 
(0.9308464583333296, 2167.6) [a] 
(0.9308468749999962, 2407.8) [a] 
(0.9308470833333296, 2408.1) [a] 
(0.9308474999999962, 2411.4) [a] 
(0.9308499999999962, 2842.3) [a] 
(0.9308568749999961, 2842.4) [a] 
(0.9308587499999962, 2842.5) [a] 
(0.9308589583333295, 2842.8) [a] 
(0.9308649999999963, 2844.5) [a] 
(0.9308652083333296, 2844.6) [a] 
(0.9308662499999963, 2846.5) [a] 
(0.9308677083333297, 2846.6) [a] 
(0.9308687499999964, 3010.7) [a] 
(0.9308708333333297, 3011.3) [a] 
(0.9308729166666629, 3082.5) [a] 
(0.9308808333333296, 3111) [a] 
(0.930881041666663, 3229.5) [a] 
},{(0.8307789366087409, 0) [b] 
(0.8666208176106086, 0.001) [b] 
(0.87037668281831, 0.002) [b] 
(0.8750268465058788, 0.003) [b] 
(0.8773893838638056, 0.004) [b] 
(0.8800157232196543, 0.005) [b] 
(0.8803748006493531, 0.006) [b] 
(0.8805321811166891, 0.007) [b] 
(0.8814512338000001, 0.008) [b] 
(0.8821963761239036, 0.009) [b] 
(0.8828028317825062, 0.01) [b] 
(0.8841281252244865, 0.011) [b] 
(0.8850066057867141, 0.012) [b] 
(0.8854865156381502, 0.013) [b] 
(0.885804967013119, 0.014) [b] 
(0.8869785141800728, 0.016) [b] 
(0.8874088437656108, 0.02) [b] 
(0.8876304749712846, 0.022) [b] 
(0.8877028129342475, 0.023) [b] 
(0.8877347169669172, 0.025) [b] 
(0.8877652196511534, 0.026) [b] 
(0.8878499609944992, 0.027) [b] 
(0.8878735721056104, 0.035) [b] 
(0.8879805356711783, 0.036) [b] 
(0.8882237657668488, 0.037) [b] 
(0.8882957545748263, 0.038) [b] 
(0.8883025109663727, 0.04) [b] 
(0.8886019188965674, 0.043) [b] 
(0.8889195233600052, 0.044) [b] 
(0.8889899062428881, 0.045) [b] 
(0.8889926862446673, 0.047) [b] 
(0.889168971905328, 0.048) [b] 
(0.8894665413497724, 0.05) [b] 
(0.8898207080164391, 0.051) [b] 
(0.8898286941275502, 0.052) [b] 
(0.8899145829444237, 0.057) [b] 
(0.8899450856286599, 0.059) [b] 
(0.8900102353069139, 0.06) [b] 
(0.8902582511799297, 0.069) [b] 
(0.8903286340628126, 0.07) [b] 
(0.8910393327857087, 0.072) [b] 
(0.8912253446904705, 0.073) [b] 
(0.8916732839859359, 0.075) [b] 
(0.8917882145414915, 0.078) [b] 
(0.8919639089859359, 0.079) [b] 
(0.8921139089859359, 0.081) [b] 
(0.8921395658003858, 0.083) [b] 
(0.8921534658092818, 0.086) [b] 
(0.8923675379040564, 0.088) [b] 
(0.8923737879040564, 0.091) [b] 
(0.8924357918723104, 0.095) [b] 
(0.8924592792830582, 0.096) [b] 
(0.8924658101085545, 0.101) [b] 
(0.8924723286734273, 0.102) [b] 
(0.8925209397845384, 0.103) [b] 
(0.8927146897845384, 0.104) [b] 
(0.892871981451205, 0.105) [b] 
(0.8929704966203892, 0.106) [b] 
(0.8930451493981669, 0.107) [b] 
(0.8930770938426114, 0.108) [b] 
(0.8936857743981669, 0.11) [b] 
(0.8938076493981669, 0.111) [b] 
(0.8944284827315003, 0.112) [b] 
(0.895506451676569, 0.113) [b] 
(0.8969874233188794, 0.114) [b] 
(0.8984525177321128, 0.115) [b] 
(0.898457726065446, 0.116) [b] 
(0.8986150177321127, 0.117) [b] 
(0.8995258577344324, 0.118) [b] 
(0.9007284396572273, 0.119) [b] 
(0.9007555663933384, 0.121) [b] 
(0.9009150865566871, 0.126) [b] 
(0.9009490739226901, 0.128) [b] 
(0.9014827544782457, 0.129) [b] 
(0.9018247683671345, 0.135) [b] 
(0.9023806711449123, 0.137) [b] 
(0.902420967727762, 0.141) [b] 
(0.9024528717604318, 0.147) [b] 
(0.9024659088901773, 0.148) [b] 
(0.9024964115744135, 0.152) [b] 
(0.9026049185188579, 0.157) [b] 
(0.9027967067367001, 0.158) [b] 
(0.9028668999083084, 0.159) [b] 
(0.9028974025925446, 0.162) [b] 
(0.9029254798611879, 0.165) [b] 
(0.9029385169909334, 0.167) [b] 
(0.9032510169909335, 0.169) [b] 
(0.9033343503242668, 0.17) [b] 
(0.9033764662272318, 0.171) [b] 
(0.9034181328938985, 0.172) [b] 
(0.9038743372734606, 0.176) [b] 
(0.9039879449717036, 0.18) [b] 
(0.9040009821014492, 0.189) [b] 
(0.9040244695121971, 0.192) [b] 
(0.904027594512197, 0.194) [b] 
(0.9040754872325035, 0.202) [b] 
(0.9041712726731166, 0.203) [b] 
(0.9041881636519826, 0.227) [b] 
(0.904249169020455, 0.229) [b] 
(0.9042908356871218, 0.237) [b] 
(0.904321338371358, 0.24) [b] 
(0.9043327967046914, 0.242) [b] 
(0.9043429312920109, 0.252) [b] 
(0.9043710085606542, 0.272) [b] 
(0.9043918418939876, 0.277) [b] 
(0.9044223445782238, 0.28) [b] 
(0.9051882026809456, 0.289) [b] 
(0.9052116900916934, 0.293) [b] 
(0.9052181062707305, 0.303) [b] 
(0.905568800715175, 0.311) [b] 
(0.9056918100519014, 0.323) [b] 
(0.9057152974626492, 0.357) [b] 
(0.9058364780182048, 0.358) [b] 
(0.9058378532011041, 0.372) [b] 
(0.9058434087566597, 0.415) [b] 
(0.9063760476455486, 0.424) [b] 
(0.9063781309788819, 0.448) [b] 
(0.9065261137926186, 0.452) [b] 
(0.9065323637926186, 0.462) [b] 
(0.9065386137926186, 0.504) [b] 
(0.9068631781930285, 0.513) [b] 
(0.9077350531930285, 0.545) [b] 
(0.9077585406037764, 0.555) [b] 
(0.9077679156037763, 0.589) [b] 
(0.9077706933815541, 0.595) [b] 
(0.907772082270443, 0.626) [b] 
(0.907820307579085, 0.637) [b] 
(0.9079175298013072, 0.64) [b] 
(0.9079410172120551, 0.641) [b] 
(0.908058058410557, 0.676) [b] 
(0.9081174126365779, 0.687) [b] 
(0.9081188015254668, 0.7) [b] 
(0.9081518703085355, 0.723) [b] 
(0.9082059214409066, 0.728) [b] 
(0.9082066090323563, 0.737) [b] 
(0.908216743619676, 0.75) [b] 
(0.9083337848181778, 0.778) [b] 
(0.9083504514848445, 0.781) [b] 
(0.9083737153737333, 0.788) [b] 
(0.9083747570404, 0.796) [b] 
(0.9083848916277196, 0.803) [b] 
(0.9083889574149425, 0.806) [b] 
(0.9084382087939812, 0.809) [b] 
(0.90845209768287, 0.943) [b] 
(0.9084661363171918, 0.945) [b] 
(0.9084992051002605, 0.962) [b] 
(0.9085977078583377, 0.983) [b] 
(0.908622333547857, 0.993) [b] 
(0.908658780526699, 1.029) [b] 
(0.9086636416378101, 1.038) [b] 
(0.9086782249711435, 1.07) [b] 
(0.908685169415588, 1.074) [b] 
(0.908696280526699, 1.086) [b] 
(0.9088245645989487, 1.087) [b] 
(0.9089416057974505, 1.112) [b] 
(0.9090698898697002, 1.155) [b] 
(0.9092310762010992, 1.245) [b] 
(0.909242196208216, 1.31) [b] 
(0.9092446267637716, 1.36) [b] 
(0.9093616679622735, 1.445) [b] 
(0.9093752096289401, 1.555) [b] 
(0.909445592511823, 1.651) [b] 
(0.9094480230673786, 1.708) [b] 
(0.9094535786229342, 1.764) [b] 
(0.9095048922518341, 1.776) [b] 
(0.9095295450296118, 1.822) [b] 
(0.9096631777001271, 1.836) [b] 
(0.9140756077541147, 1.854) [b] 
(0.914083941087448, 1.887) [b] 
(0.9141352547163478, 1.894) [b] 
(0.9141369908274589, 1.927) [b] 
(0.9143027685911054, 1.931) [b] 
(0.9143159630355499, 2.065) [b] 
(0.9143223792145869, 2.104) [b] 
(0.9143921708812536, 2.135) [b] 
(0.9145327958812536, 2.137) [b] 
(0.9146810597701425, 2.188) [b] 
(0.914767865325698, 2.207) [b] 
(0.9149407819923646, 2.227) [b] 
(0.9151859208812535, 2.27) [b] 
(0.9154277003783522, 2.272) [b] 
(0.91568047815613, 2.291) [b] 
(0.9158561726005744, 2.294) [b] 
(0.915947680653283, 2.378) [b] 
(0.916030961115297, 2.379) [b] 
(0.9160614637995332, 2.4) [b] 
(0.9160843804661999, 2.426) [b] 
(0.9160854221328666, 2.543) [b] 
(0.9161173261655363, 2.578) [b] 
(0.916149230198206, 2.583) [b] 
(0.9161561746426505, 2.586) [b] 
(0.9161892434257193, 2.595) [b] 
(0.916222312208788, 2.596) [b] 
(0.9162542162414578, 2.598) [b] 
(0.9162861202741275, 2.619) [b] 
(0.9164167367840543, 2.657) [b] 
(0.916619192374441, 2.664) [b] 
(0.9169711067888554, 2.668) [b] 
(0.9171822554375041, 2.669) [b] 
(0.917252638320387, 2.673) [b] 
(0.9172591691458833, 2.698) [b] 
(0.9172787616223724, 2.711) [b] 
(0.9176306760367868, 2.717) [b] 
(0.9177010589196697, 2.718) [b] 
(0.9178452900433638, 2.726) [b] 
(0.9179156729262467, 2.737) [b] 
(0.9179437501948899, 2.745) [b] 
(0.9179534724171121, 2.825) [b] 
(0.9179812501948899, 2.899) [b] 
(0.9179822918615566, 2.967) [b] 
(0.9180526747444394, 3.17) [b] 
(0.9180648275222172, 3.285) [b] 
(0.9181123969666616, 3.294) [b] 
(0.9181370226561809, 3.331) [b] 
(0.9181987701660053, 3.524) [b] 
(0.9182090840377503, 3.602) [b] 
(0.9182603976666501, 3.849) [b] 
(0.9182617728495495, 3.856) [b] 
(0.9182874296639993, 3.863) [b] 
(0.9183130864784492, 3.876) [b] 
(0.9183402132145604, 3.893) [b] 
(0.9183915268434603, 4.034) [b] 
(0.9184171836579101, 4.044) [b] 
(0.91844284047236, 4.063) [b] 
(0.9184529099168045, 4.132) [b] 
(0.9184834126010407, 4.423) [b] 
(0.9185105393371519, 5.274) [b] 
(0.9185340267478997, 6.32) [b] 
(0.9185575141586476, 6.33) [b] 
(0.9185620280475365, 6.481) [b] 
(0.9185866537370558, 6.572) [b] 
(0.9186257651262923, 6.669) [b] 
(0.9186518393857834, 6.67) [b] 
(0.9186583579506562, 6.674) [b] 
(0.9186713950804017, 6.679) [b] 
(0.9186844322101473, 6.727) [b] 
(0.9188014734086491, 6.729) [b] 
(0.9193356614428372, 6.773) [b] 
(0.9193950156688581, 6.805) [b] 
(0.9194543698948789, 7.065) [b] 
(0.9194848725791152, 7.174) [b] 
(0.9195153752633514, 7.175) [b] 
(0.9195458779475876, 7.178) [b] 
(0.9195763806318238, 7.184) [b] 
(0.91960688331606, 7.185) [b] 
(0.9196662375420809, 7.444) [b] 
(0.9197272429105533, 7.52) [b] 
(0.9197577455947895, 7.535) [b] 
(0.9197882482790257, 7.542) [b] 
(0.9199494346104248, 7.608) [b] 
(0.9199624717401703, 7.615) [b] 
(0.9200027683230201, 7.676) [b] 
(0.9200731512059029, 7.69) [b] 
(0.920099225465394, 7.883) [b] 
(0.9201297281496302, 7.885) [b] 
(0.9201300753718524, 7.886) [b] 
(0.9201605780560886, 8.265) [b] 
(0.9201736151858342, 8.348) [b] 
(0.920180133750707, 8.353) [b] 
(0.9202583565291801, 8.374) [b] 
(0.9202986531120299, 8.542) [b] 
(0.9203051839375262, 8.65) [b] 
(0.9203183783819707, 8.701) [b] 
(0.920320461715304, 8.957) [b] 
(0.9203798159413249, 9.343) [b] 
(0.9203831941370981, 10.044) [b] 
(0.9203865723328714, 10.076) [b] 
(0.9203893523346506, 10.139) [b] 
(0.9208505068485394, 10.363) [b] 
(0.9209318870568728, 10.365) [b] 
(0.920959013792984, 10.369) [b] 
(0.9210132672652062, 10.37) [b] 
(0.9210675207374284, 10.387) [b] 
(0.9211845619359302, 10.404) [b] 
(0.9212116886720414, 10.451) [b] 
(0.9212388154081526, 10.493) [b] 
(0.9213744490887081, 10.511) [b] 
(0.9214287025609303, 10.513) [b] 
(0.9214558292970415, 10.556) [b] 
(0.9214814861114914, 10.915) [b] 
(0.9215071429259413, 11.035) [b] 
(0.9215342696620524, 11.491) [b] 
(0.9215763855650174, 11.607) [b] 
(0.9215904241993391, 11.608) [b] 
(0.9216044628336608, 11.621) [b] 
(0.9216185014679825, 11.636) [b] 
(0.9216240570235381, 11.676) [b] 
(0.9216687504677662, 11.82) [b] 
(0.9218048935747994, 11.823) [b] 
(0.9218305503892493, 11.876) [b] 
(0.9218562072036992, 12.009) [b] 
(0.9219519193017084, 12.058) [b] 
(0.9220157273670478, 12.059) [b] 
(0.9220476313997176, 12.152) [b] 
(0.9220606685294631, 12.189) [b] 
(0.9220932613538269, 12.203) [b] 
(0.922120388089938, 12.263) [b] 
(0.9221475148260492, 12.37) [b] 
(0.9221794188587189, 12.383) [b] 
(0.9222113228913886, 12.384) [b] 
(0.9222432269240584, 12.388) [b] 
(0.9222751309567281, 12.397) [b] 
(0.9223022576928392, 12.418) [b] 
(0.9223029452842889, 12.423) [b] 
(0.9223036328757386, 12.425) [b] 
(0.9223043204671882, 12.449) [b] 
(0.9223070708329869, 12.508) [b] 
(0.9223327276474368, 12.556) [b] 
(0.9224031105303196, 12.831) [b] 
(0.9224051733046686, 13.094) [b] 
(0.922406548487568, 13.095) [b] 
(0.9224205871218897, 13.105) [b] 
(0.9224212747133393, 13.115) [b] 
(0.9224916575962222, 13.138) [b] 
(0.9224964707363699, 13.77) [b] 
(0.9225221275508197, 13.773) [b] 
(0.9225612389400563, 14.219) [b] 
(0.922567757504929, 14.221) [b] 
(0.9226094241715957, 17.726) [b] 
(0.9226927575049291, 17.728) [b] 
(0.9227135908382624, 17.748) [b] 
(0.9227344241715958, 17.852) [b] 
(0.9227409427364686, 17.863) [b] 
(0.9227617760698019, 17.932) [b] 
(0.9229492760698019, 17.94) [b] 
(0.9230326094031353, 17.941) [b] 
(0.9230919636291561, 19.355) [b] 
(0.923115451039904, 19.69) [b] 
(0.9232324922384059, 21.557) [b] 
(0.9232533255717392, 24.699) [b] 
(0.923487407968743, 25.227) [b] 
(0.9234877551909652, 25.239) [b] 
(0.9235439097282517, 27.037) [b] 
(0.923571986996895, 27.078) [b] 
(0.9236000642655383, 27.083) [b] 
(0.92361410289986, 27.104) [b] 
(0.9236281415341817, 27.115) [b] 
(0.9236421801685034, 27.117) [b] 
(0.9236562188028251, 27.233) [b] 
(0.9236600382472696, 30.22) [b] 
(0.923660732691714, 30.526) [b] 
(0.9236655938028251, 30.696) [b] 
(0.923691250617275, 31.663) [b] 
(0.9237425642461748, 31.678) [b] 
(0.9237682210606247, 31.697) [b] 
(0.9237938778750746, 32.298) [b] 
(0.9237966578768538, 32.33) [b] 
(0.9238223146913037, 32.355) [b] 
(0.9238230091357481, 33.06) [b] 
(0.923848665950198, 33.554) [b] 
(0.9238999795790979, 33.558) [b] 
(0.9239256363935477, 33.581) [b] 
(0.9239512932079976, 34.113) [b] 
(0.9239769500224475, 34.17) [b] 
(0.9239908389113364, 35.049) [b] 
(0.9240151444668919, 35.069) [b] 
(0.9240179222446697, 35.125) [b] 
(0.9240186166891141, 37.771) [b] 
(0.924020005578003, 37.801) [b] 
(0.924023130578003, 38.419) [b] 
(0.9240266028002252, 44.411) [b] 
(0.9240453528002252, 48.345) [b] 
(0.9240870194668919, 49.173) [b] 
(0.9241703528002252, 49.174) [b] 
(0.9241911861335586, 49.176) [b] 
(0.9242328528002253, 49.194) [b] 
(0.9242953528002252, 49.25) [b] 
(0.9243009083557808, 51.511) [b] 
(0.924304380578003, 52.093) [b] 
(0.9243184192123247, 55.868) [b] 
(0.9243440760267746, 65.031) [b] 
(0.9243697328412245, 67.36) [b] 
(0.9245638300634467, 67.567) [b] 
(0.9245704272856688, 68.058) [b] 
(0.9245745939523355, 68.292) [b] 
(0.9246050966365718, 72.028) [b] 
(0.9246061383032385, 81.295) [b] 
(0.9246064855254607, 82.268) [b] 
(0.9246079506702521, 84.663) [b] 
(0.924755704807368, 85.275) [b] 
(0.9247803304968873, 85.299) [b] 
(0.9248973716953892, 87.26) [b] 
(0.9249230285098391, 89.081) [b] 
(0.9249476541993584, 89.662) [b] 
(0.9249733110138083, 92.942) [b] 
(0.9250903522123102, 94.529) [b] 
(0.9251097966567546, 100.033) [b] 
(0.9251108383234213, 100.41) [b] 
(0.9251364951378712, 100.942) [b] 
(0.9251621519523211, 100.968) [b] 
(0.9251631936189878, 101.415) [b] 
(0.9251888504334377, 101.494) [b] 
(0.9251898921001044, 101.556) [b] 
(0.9251919754334377, 101.689) [b] 
(0.9251930171001044, 103.159) [b] 
(0.9251947532112155, 103.6) [b] 
(0.9252204100256654, 105.299) [b] 
(0.9252717236545652, 105.3) [b] 
(0.9252720708767874, 106.674) [b] 
(0.9252772792101207, 107.117) [b] 
(0.9252915153212318, 127.078) [b] 
(0.9252979315002688, 133.182) [b] 
(0.9253927231669355, 133.254) [b] 
(0.9254680703891578, 133.404) [b] 
(0.9254798759447134, 133.844) [b] 
(0.9254982787224911, 134.392) [b] 
(0.9254998903817617, 138.155) [b] 
(0.9255139290160834, 157.698) [b] 
(0.9255279676504051, 157.724) [b] 
(0.9255390787615162, 162.28) [b] 
(0.9255422037615162, 162.526) [b] 
(0.9255439398726273, 164.461) [b] 
(0.9255463704281829, 170.275) [b] 
(0.9255866670110326, 208.82) [b] 
(0.9255887503443659, 209.519) [b] 
(0.9261325380678599, 217.437) [b] 
(0.9261738575123043, 228.088) [b] 
(0.9261767045684483, 234.533) [b] 
(0.9262038313045594, 240.412) [b] 
(0.9262052548326314, 248.645) [b] 
(0.9263222960311333, 287.225) [b] 
(0.9263247265866889, 314.965) [b] 
(0.9264428794166627, 322.204) [b] 
(0.9265022336426836, 334.605) [b] 
(0.9265257210534315, 340.169) [b] 
(0.9265492084641793, 341.794) [b] 
(0.9265526806864015, 385.153) [b] 
(0.9265714306864015, 385.649) [b] 
(0.9265887917975125, 386.069) [b] 
(0.9265974723530681, 386.309) [b] 
(0.926764025137489, 392.742) [b] 
(0.9267709695819335, 398.039) [b] 
(0.9267738166380775, 400.354) [b] 
(0.9268342333047441, 406.557) [b] 
(0.9268404833047441, 406.624) [b] 
(0.9268436083047441, 407.402) [b] 
(0.9268592333047441, 412.587) [b] 
(0.9268644416380774, 413.606) [b] 
(0.9268665249714106, 433.693) [b] 
(0.9271314555269662, 434.067) [b] 
(0.9271349277491884, 434.232) [b] 
(0.9271415249714106, 434.737) [b] 
(0.9271425666380773, 434.89) [b] 
(0.9271446499714106, 435.068) [b] 
(0.9271449971936327, 435.219) [b] 
(0.9271529833047438, 439.4) [b] 
(0.9271543721936327, 440.255) [b] 
(0.9271547194158549, 442.078) [b] 
(0.9271654833047438, 465.308) [b] 
(0.9271758999714105, 466.151) [b] 
(0.9276134652796655, 472.61) [b] 
(0.9276199961051619, 476.724) [b] 
(0.927640482216273, 477.21) [b] 
(0.9276723862489428, 494.186) [b] 
(0.9276970119384621, 500.222) [b] 
(0.9276973049674204, 543.618) [b] 
(0.9276983466340871, 548.458) [b] 
(0.9277087633007538, 549.042) [b] 
(0.9277322507115017, 575.344) [b] 
(0.9277631534892794, 602.504) [b] 
(0.9277683618226127, 602.566) [b] 
(0.9277700979337238, 602.709) [b] 
(0.9277718340448349, 602.85) [b] 
(0.9278482229337237, 631.116) [b] 
(0.9278485701559459, 631.204) [b] 
(0.9279044729337237, 645.245) [b] 
(0.9279114173781682, 645.329) [b] 
(0.9279152368226127, 645.531) [b] 
(0.9281444034892793, 670.192) [b] 
(0.9281510007115015, 670.321) [b] 
(0.9281891951559459, 670.582) [b] 
(0.9281916257115015, 670.998) [b] 
(0.9281919729337237, 671.677) [b] 
(0.9281937090448348, 672.835) [b] 
(0.9281964868226126, 674.249) [b] 
(0.9281975284892793, 676.308) [b] 
(0.9282013479337238, 681.889) [b] 
(0.9282072507115016, 681.956) [b] 
(0.928214195155946, 992.509) [b] 
(0.9282152368226128, 992.954) [b] 
(0.9282319034892794, 1002.26) [b] 
(0.9282381534892794, 1014.3) [b] 
(0.9282383000037585, 1078.62) [b] 
(0.9282976542297794, 1127.53) [b] 
(0.9283570084558003, 1127.54) [b] 
(0.9283604806780225, 1149.46) [b] 
(0.9284375640113558, 1203.87) [b] 
(0.9284462445669114, 1203.98) [b] 
(0.9284483279002447, 1208) [b] 
(0.9284718153109925, 1221.59) [b] 
(0.9284745953127718, 1226.01) [b] 
(0.9285204286461051, 1287.51) [b] 
(0.9285214703127718, 1288.5) [b] 
(0.9285228592016607, 1288.99) [b] 
(0.9285232064238829, 1289.43) [b] 
(0.9285284147572161, 1289.96) [b] 
(0.9285287619794383, 1290.42) [b] 
(0.9285315419812176, 1294.47) [b] 
(0.9288603657316873, 1301.12) [b] 
(0.9289308279639308, 1301.13) [b] 
(0.9289543153746787, 1301.19) [b] 
(0.9290247776069221, 1301.88) [b] 
(0.92904826501767, 1303.18) [b] 
(0.9290717524284179, 1304.69) [b] 
(0.9290952398391658, 1304.7) [b] 
(0.9291187272499136, 1304.73) [b] 
(0.9291422146606615, 1304.96) [b] 
(0.9291547146606615, 1317.8) [b] 
(0.9291554091051059, 1317.87) [b] 
(0.9291564507717726, 1318.03) [b] 
(0.9291567979939948, 1318.27) [b] 
(0.9291592285495504, 1318.57) [b] 
(0.9292185827755712, 1339.71) [b] 
(0.9292779370015921, 1339.84) [b] 
(0.9292807147793699, 1368.48) [b] 
(0.9293042021901178, 1370.82) [b] 
(0.9293511770116134, 1370.89) [b] 
(0.9293838159005023, 1373.03) [b] 
(0.9293900659005023, 1373.12) [b] 
(0.929391107567169, 1373.31) [b] 
(0.9293952742338357, 1373.51) [b] 
(0.9293959686782801, 1373.77) [b] 
(0.9294362652611299, 1523.73) [b] 
(0.9294765618439796, 1524.48) [b] 
(0.929501187533499, 1540.09) [b] 
(0.9295039653112768, 1642.4) [b] 
(0.9296210065097786, 1778.77) [b] 
(0.929621299538737, 1907.47) [b] 
(0.9296268550942925, 1911.22) [b] 
(0.9296272023165147, 1912.17) [b] 
(0.9296348412054036, 1912.84) [b] 
(0.9296351884276258, 1913.57) [b] 
(0.929635535649848, 1919.04) [b] 
(0.9296358828720702, 1919.76) [b] 
(0.9296393550942924, 2004.8) [b] 
(0.9296397023165146, 2004.91) [b] 
(0.9296442162054035, 2005.28) [b] 
(0.9296452578720702, 2005.39) [b] 
(0.9296462995387369, 2007.34) [b] 
(0.9296909973630127, 2092.57) [b] 
(0.9296920390296795, 2148.89) [b] 
(0.9296927334741238, 2360.01) [b] 
(0.9297000251407905, 2360.1) [b] 
(0.9297024556963461, 2360.21) [b] 
(0.9297066223630128, 2360.3) [b] 
(0.9297107890296795, 2363.09) [b] 
(0.9297510856125293, 2573.76) [b] 
(0.929791382195379, 2574.93) [b] 
(0.9299122719439283, 2575.85) [b] 
(0.9299449108328173, 2642.6) [b] 
(0.9299483830550395, 2642.72) [b] 
(0.9299737302772617, 2643.01) [b] 
(0.9299761608328173, 2643.16) [b] 
(0.929980327499484, 2644.59) [b] 
(0.9299831052772618, 2644.77) [b] 
(0.9299976886105952, 2645.04) [b] 
(0.9300018552772619, 2645.21) [b] 
(0.9300164386105952, 2670.76) [b] 
(0.9301776249419943, 2844.51) [b] 
(0.9302582181076938, 2844.88) [b] 
(0.9302985146905436, 2977.39) [b] 
(0.93030504551604, 3308.24) [b] 
(0.9304220867145419, 3399.84) [b] 
},{(0.9101962708333334, 0.001) [c] 
(0.9101962708333334, 2.8048337916666664) [c] 
(0.9101962708333334, 3600) [c] 
}}}{legend pos=north west}}
% 	\subfloat[depth=7]{\cactus{Average Accuracy}{CPU time}{\budalg, \murtree, \cart}{{{(0.9082726060274735, 0) [a] 
(0.9161712920449518, 0.01) [a] 
(0.9210576809338408, 0.02) [a] 
(0.9250673337116188, 0.03) [a] 
(0.9292812226005075, 0.04) [a] 
(0.9294024726005073, 0.05) [a] 
(0.9294203892671741, 0.06) [a] 
(0.929734139267174, 0.07) [a] 
(0.9299326809338406, 0.08) [a] 
(0.930259139267174, 0.09) [a] 
(0.9303139309338406, 0.1) [a] 
(0.9303558059338407, 0.11) [a] 
(0.9304814309338407, 0.12) [a] 
(0.9304828892671739, 0.13) [a] 
(0.9304937226005073, 0.14) [a] 
(0.9305249726005074, 0.15) [a] 
(0.9306637226005074, 0.16) [a] 
(0.9307772642671742, 0.17) [a] 
(0.9307776809338408, 0.19) [a] 
(0.9307853892671741, 0.2) [a] 
(0.9308099726005075, 0.21) [a] 
(0.9309449726005077, 0.22) [a] 
(0.9309453892671743, 0.23) [a] 
(0.9309468476005076, 0.24) [a] 
(0.9336063328527567, 0.25) [a] 
(0.93363841618609, 0.26) [a] 
(0.9336388328527566, 0.27) [a] 
(0.9336400828527566, 0.28) [a] 
(0.9336407078527565, 0.29) [a] 
(0.9336421661860899, 0.3) [a] 
(0.9337259161860899, 0.31) [a] 
(0.9337467495194233, 0.32) [a] 
(0.9337675828527566, 0.33) [a] 
(0.9338309161860899, 0.34) [a] 
(0.9338313328527567, 0.35) [a] 
(0.93383216618609, 0.37) [a] 
(0.9338323745194234, 0.39) [a] 
(0.9338325828527567, 0.4) [a] 
(0.9338342495194234, 0.43) [a] 
(0.9338794578527567, 0.44) [a] 
(0.9338923745194234, 0.46) [a] 
(0.9338988328527568, 0.47) [a] 
(0.9339232078527568, 0.5) [a] 
(0.9383428165711888, 0.51) [a] 
(0.9383663582378554, 0.53) [a] 
(0.938367399904522, 0.54) [a] 
(0.9383682332378553, 0.55) [a] 
(0.9383684415711887, 0.57) [a] 
(0.9383686499045221, 0.58) [a] 
(0.9383690665711887, 0.6) [a] 
(0.938486774904522, 0.61) [a] 
(0.9384876082378553, 0.62) [a] 
(0.9386503165711887, 0.63) [a] 
(0.9387319832378554, 0.64) [a] 
(0.9387869832378554, 0.65) [a] 
(0.9387871915711887, 0.66) [a] 
(0.9388415665711888, 0.69) [a] 
(0.9388434415711887, 0.7) [a] 
(0.9388440665711887, 0.71) [a] 
(0.938844274904522, 0.73) [a] 
(0.9388446915711887, 0.75) [a] 
(0.9388548999045221, 0.76) [a] 
(0.9388884415711887, 0.77) [a] 
(0.9388888582378553, 0.79) [a] 
(0.9388894832378553, 0.8) [a] 
(0.9390521915711887, 0.81) [a] 
(0.9390526082378553, 0.82) [a] 
(0.9390559415711887, 0.83) [a] 
(0.9390601082378552, 0.84) [a] 
(0.9390676082378553, 0.85) [a] 
(0.9390680249045219, 0.86) [a] 
(0.9390821915711886, 0.87) [a] 
(0.9391188582378551, 0.95) [a] 
(0.93972625, 0.96) [a] 
(0.9397264583333333, 1.02) [a] 
(0.9397306249999999, 1.06) [a] 
(0.9397547916666666, 1.09) [a] 
(0.9397552083333333, 1.11) [a] 
(0.9397770833333332, 1.17) [a] 
(0.9397774999999998, 1.2) [a] 
(0.9398381249999999, 1.24) [a] 
(0.9398414583333332, 1.25) [a] 
(0.9398818749999999, 1.41) [a] 
(0.9399633333333333, 1.45) [a] 
(0.9399695833333332, 1.52) [a] 
(0.9399699999999999, 1.56) [a] 
(0.9399704166666665, 1.57) [a] 
(0.9399724999999998, 1.69) [a] 
(0.9399737499999997, 1.72) [a] 
(0.9399741666666663, 1.73) [a] 
(0.9399756249999996, 1.77) [a] 
(0.939977083333333, 1.79) [a] 
(0.9399777083333329, 1.8) [a] 
(0.9399783333333329, 1.81) [a] 
(0.9399787499999995, 1.86) [a] 
(0.9401583333333329, 1.93) [a] 
(0.9404149999999997, 1.94) [a] 
(0.940415208333333, 1.96) [a] 
(0.9404160416666664, 1.99) [a] 
(0.9404172916666663, 2) [a] 
(0.9404177083333329, 2.02) [a] 
(0.9405429166666663, 2.04) [a] 
(0.9409020833333331, 2.05) [a] 
(0.9409022916666665, 2.08) [a] 
(0.9409162499999998, 2.1) [a] 
(0.9409224999999999, 2.11) [a] 
(0.940964375, 2.12) [a] 
(0.9409935416666666, 2.13) [a] 
(0.9410191666666666, 2.24) [a] 
(0.9410704166666666, 2.31) [a] 
(0.9411106249999999, 2.42) [a] 
(0.9411618749999998, 2.44) [a] 
(0.9411629166666665, 2.45) [a] 
(0.9412033333333332, 2.47) [a] 
(0.9412043749999999, 2.5) [a] 
(0.9412085416666666, 2.54) [a] 
(0.9412087499999999, 2.57) [a] 
(0.9412229166666666, 2.61) [a] 
(0.9414539583333332, 2.62) [a] 
(0.9414541666666666, 2.72) [a] 
(0.9414797916666666, 2.75) [a] 
(0.9417108333333333, 2.77) [a] 
(0.9417110416666666, 2.82) [a] 
(0.9417366666666667, 2.89) [a] 
(0.9417370833333333, 2.93) [a] 
(0.9417733333333332, 3.03) [a] 
(0.941809375, 3.04) [a] 
(0.9418302083333333, 3.19) [a] 
(0.941884375, 3.22) [a] 
(0.941890625, 3.29) [a] 
(0.9418916666666667, 3.46) [a] 
(0.941905625, 4.06) [a] 
(0.9419185416666667, 4.2) [a] 
(0.9419504166666667, 4.58) [a] 
(0.9420141666666667, 4.67) [a] 
(0.9420145833333333, 4.8) [a] 
(0.9420354166666667, 4.85) [a] 
(0.9420358333333333, 4.95) [a] 
(0.9420677083333333, 5.16) [a] 
(0.9420681249999999, 6.24) [a] 
(0.9420683333333333, 6.25) [a] 
(0.9420691666666666, 6.26) [a] 
(0.942069375, 6.28) [a] 
(0.9420697916666666, 6.31) [a] 
(0.9420704166666666, 6.33) [a] 
(0.9420708333333332, 6.34) [a] 
(0.9420714583333332, 6.37) [a] 
(0.9420716666666665, 6.57) [a] 
(0.9420727083333332, 6.73) [a] 
(0.9420731249999998, 7.7) [a] 
(0.9420987499999999, 7.73) [a] 
(0.9421020833333332, 8.02) [a] 
(0.9421277083333333, 8.05) [a] 
(0.942140625, 8.49) [a] 
(0.9421470833333333, 8.55) [a] 
(0.9421679166666667, 8.77) [a] 
(0.94218875, 8.78) [a] 
(0.942189375, 10.15) [a] 
(0.942215, 10.3) [a] 
(0.9422410416666667, 10.68) [a] 
(0.9422541666666667, 10.69) [a] 
(0.9422797916666668, 10.72) [a] 
(0.9422862500000001, 10.73) [a] 
(0.9423125000000001, 11.48) [a] 
(0.9424102083333334, 11.5) [a] 
(0.9426447916666667, 11.51) [a] 
(0.9426450000000001, 11.67) [a] 
(0.9426454166666667, 11.69) [a] 
(0.9427093750000001, 12.08) [a] 
(0.9428050000000001, 12.1) [a] 
(0.9428056250000001, 12.45) [a] 
(0.9428060416666667, 12.82) [a] 
(0.9428127083333334, 12.96) [a] 
(0.9428191666666668, 12.97) [a] 
(0.9428195833333334, 13.62) [a] 
(0.9429183333333334, 13.65) [a] 
(0.9429189583333334, 13.66) [a] 
(0.9429204166666667, 13.7) [a] 
(0.9429206250000001, 13.75) [a] 
(0.9429214583333334, 13.76) [a] 
(0.9429220833333334, 13.8) [a] 
(0.9429225, 13.85) [a] 
(0.9431566666666668, 13.91) [a] 
(0.9432743750000001, 13.94) [a] 
(0.9432810416666668, 13.96) [a] 
(0.9432814583333334, 14.03) [a] 
(0.9432816666666668, 14.08) [a] 
(0.9433010416666668, 14.29) [a] 
(0.9433020833333335, 15.04) [a] 
(0.9433025000000002, 15.73) [a] 
(0.9433089583333335, 16.11) [a] 
(0.9433093750000001, 16.48) [a] 
(0.9433160416666668, 16.86) [a] 
(0.9433170833333335, 17.31) [a] 
(0.9433175000000001, 17.44) [a] 
(0.9434345833333335, 17.79) [a] 
(0.9434350000000001, 18.05) [a] 
(0.9434558333333335, 18.24) [a] 
(0.9434562500000001, 18.3) [a] 
(0.9434770833333335, 18.51) [a] 
(0.9434979166666668, 18.75) [a] 
(0.9434987500000002, 19.19) [a] 
(0.9435318750000001, 19.5) [a] 
(0.9435325000000001, 19.73) [a] 
(0.9435331250000001, 20) [a] 
(0.9435335416666667, 20.05) [a] 
(0.94354, 21.19) [a] 
(0.9436358333333335, 22.66) [a] 
(0.9436677083333335, 22.67) [a] 
(0.9436995833333335, 22.68) [a] 
(0.9437633333333335, 22.7) [a] 
(0.9437954166666669, 22.71) [a] 
(0.9438272916666669, 22.89) [a] 
(0.944139791666667, 23.41) [a] 
(0.9441814583333337, 23.43) [a] 
(0.9442231250000004, 23.44) [a] 
(0.9442775000000004, 24.55) [a] 
(0.9444402083333339, 24.57) [a] 
(0.9444672916666672, 24.94) [a] 
(0.9444991666666672, 25.01) [a] 
(0.9445208333333339, 25.15) [a] 
(0.9445537500000006, 25.16) [a] 
(0.944555208333334, 25.17) [a] 
(0.9446093750000006, 26.06) [a] 
(0.9446366666666673, 26.07) [a] 
(0.9446637500000007, 26.21) [a] 
(0.944690833333334, 26.23) [a] 
(0.9447179166666674, 26.35) [a] 
(0.9447387500000007, 26.54) [a] 
(0.9447595833333341, 26.55) [a] 
(0.9447914583333341, 26.61) [a] 
(0.9447922916666674, 26.78) [a] 
(0.9449091666666675, 27.14) [a] 
(0.9449362500000008, 27.29) [a] 
(0.9449635416666675, 27.3) [a] 
(0.9450806250000009, 27.33) [a] 
(0.9451125000000009, 27.69) [a] 
(0.9451395833333343, 27.73) [a] 
(0.9451714583333343, 28.68) [a] 
(0.9451720833333342, 30.27) [a] 
(0.9451725000000009, 30.41) [a] 
(0.9451729166666675, 30.42) [a] 
(0.9451731250000008, 30.48) [a] 
(0.9451739583333341, 30.62) [a] 
(0.9451741666666674, 30.63) [a] 
(0.945174583333334, 30.64) [a] 
(0.9451750000000007, 30.69) [a] 
(0.9451760416666674, 30.72) [a] 
(0.9451762500000007, 30.76) [a] 
(0.9451766666666673, 30.78) [a] 
(0.9452037500000007, 30.8) [a] 
(0.9452041666666673, 31.05) [a] 
(0.9452250000000006, 31.06) [a] 
(0.9452254166666673, 31.12) [a] 
(0.9453295833333341, 31.66) [a] 
(0.9453504166666674, 31.73) [a] 
(0.9453506250000008, 32.63) [a] 
(0.9453510416666674, 33.49) [a] 
(0.9453927083333341, 33.64) [a] 
(0.9453931250000007, 34.14) [a] 
(0.9454252083333341, 34.34) [a] 
(0.9454570833333341, 34.74) [a] 
(0.9454766666666674, 38.36) [a] 
(0.9455037500000008, 39.46) [a] 
(0.9455245833333341, 40.01) [a] 
(0.9455250000000007, 42.81) [a] 
(0.9455252083333341, 42.97) [a] 
(0.9455570833333341, 44.98) [a] 
(0.9455604166666675, 45.9) [a] 
(0.9455812500000008, 46.88) [a] 
(0.9456020833333342, 46.97) [a] 
(0.9456025000000008, 47.54) [a] 
(0.9456027083333342, 47.64) [a] 
(0.9456031250000008, 47.68) [a] 
(0.9456035416666674, 47.73) [a] 
(0.9456037500000007, 47.76) [a] 
(0.9456045833333341, 48.52) [a] 
(0.9456050000000007, 49.21) [a] 
(0.9456056250000007, 50.92) [a] 
(0.9456362500000006, 52.21) [a] 
(0.9456366666666672, 52.47) [a] 
(0.9456372916666672, 52.63) [a] 
(0.9456581250000006, 53.99) [a] 
(0.9456585416666672, 54.87) [a] 
(0.9456706250000005, 56.52) [a] 
(0.9456710416666672, 60.6) [a] 
(0.9456712500000005, 60.69) [a] 
(0.9456777083333339, 61.63) [a] 
(0.9456779166666672, 63.11) [a] 
(0.9457612500000007, 64.45) [a] 
(0.945782083333334, 64.47) [a] 
(0.9458029166666674, 64.49) [a] 
(0.9458041666666673, 65.06) [a] 
(0.9458104166666673, 65.07) [a] 
(0.945814583333334, 65.08) [a] 
(0.9458154166666674, 65.09) [a] 
(0.9458181250000006, 65.22) [a] 
(0.9459304166666672, 66.22) [a] 
(0.9460006250000005, 66.23) [a] 
(0.9460010416666671, 66.61) [a] 
(0.9460150000000004, 66.64) [a] 
(0.9460431250000004, 66.66) [a] 
(0.9460437500000004, 66.78) [a] 
(0.9460577083333337, 66.88) [a] 
(0.9460583333333337, 67.02) [a] 
(0.9460587500000003, 67.06) [a] 
(0.9460870833333336, 67.07) [a] 
(0.9460875000000002, 67.09) [a] 
(0.9460881250000002, 67.23) [a] 
(0.9460885416666668, 67.25) [a] 
(0.9461447916666668, 67.66) [a] 
(0.94615875, 67.7) [a] 
(0.9461729166666667, 67.72) [a] 
(0.946186875, 67.77) [a] 
(0.946215, 67.96) [a] 
(0.9462289583333333, 68.66) [a] 
(0.9462429166666666, 68.68) [a] 
(0.9462714583333333, 68.69) [a] 
(0.9462716666666666, 68.7) [a] 
(0.9462858333333333, 68.71) [a] 
(0.9463137499999998, 68.86) [a] 
(0.9463279166666665, 68.97) [a] 
(0.9463283333333331, 70.66) [a] 
(0.9464454166666665, 72.09) [a] 
(0.9464870833333332, 72.17) [a] 
(0.9464874999999998, 76.11) [a] 
(0.9464877083333332, 76.45) [a] 
(0.9464881249999998, 76.82) [a] 
(0.9464916666666665, 78.64) [a] 
(0.9465037499999999, 86.61) [a] 
(0.9465158333333332, 86.64) [a] 
(0.9465164583333332, 87.11) [a] 
(0.9465168749999998, 87.16) [a] 
(0.9465172916666664, 87.18) [a] 
(0.9465293749999998, 87.21) [a] 
(0.9465295833333331, 87.22) [a] 
(0.9465304166666665, 87.3) [a] 
(0.9465310416666665, 87.31) [a] 
(0.9465314583333331, 87.35) [a] 
(0.9465316666666664, 87.36) [a] 
(0.9465679166666665, 87.37) [a] 
(0.9465683333333331, 87.49) [a] 
(0.9465687499999997, 88.85) [a] 
(0.9465960416666664, 90.3) [a] 
(0.9465964583333331, 91.96) [a] 
(0.946597083333333, 91.97) [a] 
(0.9465974999999996, 92.02) [a] 
(0.9465979166666663, 92.05) [a] 
(0.9465981249999996, 92.06) [a] 
(0.9465985416666662, 92.09) [a] 
(0.9465989583333329, 92.41) [a] 
(0.9465991666666662, 92.44) [a] 
(0.9465995833333328, 93.44) [a] 
(0.9465999999999994, 96.71) [a] 
(0.9466318749999995, 97.18) [a] 
(0.9468002083333327, 98.36) [a] 
(0.9468143749999993, 98.43) [a] 
(0.9468147916666659, 98.67) [a] 
(0.9468708333333327, 98.7) [a] 
(0.9468849999999993, 98.72) [a] 
(0.9469131249999992, 99.97) [a] 
(0.9469410416666659, 100.12) [a] 
(0.946969166666666, 100.14) [a] 
(0.946969791666666, 107.75) [a] 
(0.9469702083333326, 107.79) [a] 
(0.9469706249999992, 107.8) [a] 
(0.9469708333333325, 107.83) [a] 
(0.9469712499999992, 107.84) [a] 
(0.9469718749999991, 107.85) [a] 
(0.9469722916666657, 107.9) [a] 
(0.9469729166666657, 107.91) [a] 
(0.9469737499999991, 107.92) [a] 
(0.9470279166666657, 111.06) [a] 
(0.9470549999999991, 111.13) [a] 
(0.9470806249999991, 114.23) [a] 
(0.9471833333333325, 114.33) [a] 
(0.9472089583333325, 115.91) [a] 
(0.9473116666666659, 116.01) [a] 
(0.9473118749999992, 118.02) [a] 
(0.9473183333333326, 118.54) [a] 
(0.9473187499999992, 119.66) [a] 
(0.9473220833333326, 119.82) [a] 
(0.9473339583333326, 120.25) [a] 
(0.9473756249999993, 125.13) [a] 
(0.9473964583333326, 125.14) [a] 
(0.947417291666666, 125.64) [a] 
(0.947468541666666, 125.71) [a] 
(0.9474824999999993, 126.47) [a] 
(0.9474858333333327, 126.63) [a] 
(0.947489166666666, 126.74) [a] 
(0.9475404166666661, 127.54) [a] 
(0.9475674999999995, 128.14) [a] 
(0.9475677083333328, 132.47) [a] 
(0.9475679166666662, 134.68) [a] 
(0.9475683333333328, 134.74) [a] 
(0.9475687499999994, 134.78) [a] 
(0.9475727083333327, 138.88) [a] 
(0.9475729166666661, 138.89) [a] 
(0.9475731249999995, 139.53) [a] 
(0.9475862499999995, 146) [a] 
(0.9475927083333329, 146.04) [a] 
(0.9475931249999995, 148.58) [a] 
(0.9475933333333328, 148.6) [a] 
(0.9476206249999996, 150.53) [a] 
(0.9476477083333329, 150.54) [a] 
(0.9476481249999995, 152.21) [a] 
(0.9476485416666661, 152.26) [a] 
(0.9476487499999995, 153) [a] 
(0.9476495833333328, 153.02) [a] 
(0.9476497916666662, 154.22) [a] 
(0.9476502083333328, 154.24) [a] 
(0.9476506249999994, 154.26) [a] 
(0.9476508333333328, 156.26) [a] 
(0.9476510416666661, 159.85) [a] 
(0.9476514583333328, 163.66) [a] 
(0.9476518749999994, 166.62) [a] 
(0.947652291666666, 167.15) [a] 
(0.9476731249999993, 167.58) [a] 
(0.9476733333333327, 168.25) [a] 
(0.947674166666666, 168.74) [a] 
(0.9476949999999994, 169.64) [a] 
(0.9476970833333327, 186.71) [a] 
(0.9476974999999993, 186.77) [a] 
(0.9476985416666659, 187.25) [a] 
(0.9476989583333325, 187.36) [a] 
(0.9476991666666659, 187.37) [a] 
(0.9476995833333325, 187.71) [a] 
(0.9476999999999991, 187.87) [a] 
(0.9477004166666657, 188.09) [a] 
(0.9477047916666657, 196.26) [a] 
(0.947706249999999, 196.27) [a] 
(0.9477077083333323, 196.99) [a] 
(0.947709374999999, 200.36) [a] 
(0.9477099999999989, 200.52) [a] 
(0.9477104166666656, 200.92) [a] 
(0.9477114583333323, 202.43) [a] 
(0.947712499999999, 219.69) [a] 
(0.9477129166666656, 219.73) [a] 
(0.9477131249999989, 219.76) [a] 
(0.9477141666666656, 220.11) [a] 
(0.9477164583333323, 221.75) [a] 
(0.9477420833333323, 223.5) [a] 
(0.9477677083333323, 226.01) [a] 
(0.9477885416666657, 241.27) [a] 
(0.9478093749999991, 241.85) [a] 
(0.947810624999999, 297.43) [a] 
(0.9478247916666657, 297.73) [a] 
(0.947838749999999, 299.6) [a] 
(0.9478808333333323, 299.71) [a] 
(0.9478949999999989, 299.73) [a] 
(0.9479420833333322, 309) [a] 
(0.9479654166666656, 309.01) [a] 
(0.9479889583333322, 309.11) [a] 
(0.9480124999999988, 315.77) [a] 
(0.9480127083333322, 328.3) [a] 
(0.9480137499999989, 365.04) [a] 
(0.9480168749999989, 365.5) [a] 
(0.9480231249999989, 365.51) [a] 
(0.9480252083333323, 365.54) [a] 
(0.9480283333333323, 365.56) [a] 
(0.948029374999999, 365.69) [a] 
(0.9480304166666657, 369.66) [a] 
(0.9480314583333324, 369.81) [a] 
(0.948031874999999, 392.96) [a] 
(0.9480322916666656, 393) [a] 
(0.9480329166666656, 413.58) [a] 
(0.9480335416666655, 413.79) [a] 
(0.9480345833333322, 413.8) [a] 
(0.9480349999999989, 414.82) [a] 
(0.9480354166666655, 421.67) [a] 
(0.9480447916666654, 427.97) [a] 
(0.9480541666666655, 427.98) [a] 
(0.9480552083333322, 428.01) [a] 
(0.9480562499999989, 428.02) [a] 
(0.9480583333333323, 428.3) [a] 
(0.948062499999999, 428.33) [a] 
(0.9480656249999991, 428.6) [a] 
(0.9480660416666657, 429.89) [a] 
(0.9480670833333324, 432.05) [a] 
(0.9480672916666658, 432.22) [a] 
(0.9480677083333324, 432.44) [a] 
(0.948068124999999, 432.45) [a] 
(0.9480685416666657, 433.63) [a] 
(0.948068749999999, 443.97) [a] 
(0.948069374999999, 461.02) [a] 
(0.9480697916666656, 461.03) [a] 
(0.9480718749999989, 461.07) [a] 
(0.9480720833333323, 461.08) [a] 
(0.948073124999999, 461.17) [a] 
(0.9480735416666656, 461.18) [a] 
(0.9480739583333322, 461.2) [a] 
(0.9480741666666656, 461.74) [a] 
(0.9480745833333322, 485.69) [a] 
(0.9480749999999988, 485.74) [a] 
(0.9480752083333321, 485.77) [a] 
(0.9480772916666654, 485.83) [a] 
(0.9480783333333321, 485.85) [a] 
(0.9480791666666655, 485.94) [a] 
(0.9480793749999988, 485.95) [a] 
(0.9480814583333321, 486.13) [a] 
(0.9480818749999987, 487.03) [a] 
(0.9481987499999988, 491.78) [a] 
(0.9484099999999986, 501.64) [a] 
(0.9484335416666653, 501.66) [a] 
(0.9484806249999985, 501.83) [a] 
(0.9485039583333319, 502.21) [a] 
(0.9485274999999985, 502.23) [a] 
(0.9485510416666652, 502.24) [a] 
(0.9485745833333318, 502.33) [a] 
(0.9485979166666652, 502.34) [a] 
(0.9486449999999985, 502.35) [a] 
(0.9486918749999985, 503) [a] 
(0.9487154166666651, 503.1) [a] 
(0.9487389583333318, 503.12) [a] 
(0.9487624999999984, 506.58) [a] 
(0.9487858333333318, 509.98) [a] 
(0.9488093749999984, 510.55) [a] 
(0.9489264583333318, 518.25) [a] 
(0.9489318749999984, 525.28) [a] 
(0.9489327083333318, 525.3) [a] 
(0.9489345833333317, 525.32) [a] 
(0.9489381249999984, 525.45) [a] 
(0.948961666666665, 525.88) [a] 
(0.9489733333333317, 533.18) [a] 
(0.948975416666665, 533.19) [a] 
(0.9489987499999983, 533.74) [a] 
(0.9490002083333317, 581.72) [a] 
(0.949001666666665, 581.73) [a] 
(0.9490064583333316, 581.74) [a] 
(0.9490535416666649, 605.87) [a] 
(0.9490770833333315, 605.88) [a] 
(0.9491004166666649, 605.96) [a] 
(0.9491239583333315, 606.08) [a] 
(0.9491474999999981, 618.46) [a] 
(0.9491479166666648, 625.68) [a] 
(0.9491483333333314, 629.38) [a] 
(0.949148749999998, 638.95) [a] 
(0.9491491666666646, 638.96) [a] 
(0.949149374999998, 639.06) [a] 
(0.9491497916666646, 641.65) [a] 
(0.9491502083333312, 648.27) [a] 
(0.9491504166666646, 648.95) [a] 
(0.9491506249999979, 659.04) [a] 
(0.9491510416666645, 659.08) [a] 
(0.9491514583333311, 659.13) [a] 
(0.9491516666666645, 659.14) [a] 
(0.9491520833333311, 659.23) [a] 
(0.9491524999999977, 661.22) [a] 
(0.9491527083333311, 661.24) [a] 
(0.9491797916666644, 674.76) [a] 
(0.9492068749999978, 683.91) [a] 
(0.9492072916666644, 685.06) [a] 
(0.949207708333331, 685.07) [a] 
(0.9492081249999976, 696.32) [a] 
(0.9492087499999976, 696.33) [a] 
(0.9492091666666642, 696.42) [a] 
(0.9492095833333308, 696.64) [a] 
(0.9492102083333308, 696.65) [a] 
(0.9492106249999974, 700.87) [a] 
(0.9492112499999974, 728.27) [a] 
(0.949211666666664, 728.35) [a] 
(0.949212291666664, 728.36) [a] 
(0.9492131249999973, 728.59) [a] 
(0.9492133333333307, 748.4) [a] 
(0.949213541666664, 775.79) [a] 
(0.9492137499999974, 782.65) [a] 
(0.9492152083333307, 802.84) [a] 
(0.949216666666664, 802.91) [a] 
(0.9492170833333307, 867.98) [a] 
(0.9492174999999973, 867.99) [a] 
(0.9492185416666639, 868) [a] 
(0.9492187499999972, 868.04) [a] 
(0.9492191666666638, 868.06) [a] 
(0.9492195833333305, 872.87) [a] 
(0.9493366666666638, 880.02) [a] 
(0.9493372916666638, 882.54) [a] 
(0.9493377083333304, 882.55) [a] 
(0.9493379166666638, 882.56) [a] 
(0.9493383333333304, 882.65) [a] 
(0.949338749999997, 886.65) [a] 
(0.949339374999997, 886.73) [a] 
(0.9493397916666636, 886.74) [a] 
(0.949339999999997, 886.75) [a] 
(0.9493404166666636, 890.84) [a] 
(0.9493408333333302, 897.44) [a] 
(0.9493414583333302, 897.46) [a] 
(0.9493418749999968, 897.48) [a] 
(0.9493420833333301, 897.5) [a] 
(0.9493424999999968, 900.92) [a] 
(0.9493429166666634, 900.93) [a] 
(0.9493435416666633, 900.94) [a] 
(0.94934395833333, 900.95) [a] 
(0.9493441666666633, 918.25) [a] 
(0.9493712499999967, 993.31) [a] 
(0.9493985416666634, 993.52) [a] 
(0.9494068749999968, 1008.9) [a] 
(0.9494143749999968, 1009) [a] 
(0.94942395833333, 1009.6) [a] 
(0.9494274999999968, 1009.9) [a] 
(0.9494281249999967, 1010.3) [a] 
(0.9494285416666634, 1013.1) [a] 
(0.94942895833333, 1013.3) [a] 
(0.9494593749999967, 1016.5) [a] 
(0.94947895833333, 1055.2) [a] 
(0.9494985416666633, 1056.5) [a] 
(0.9494989583333299, 1058.9) [a] 
(0.9494993749999965, 1061.1) [a] 
(0.9495385416666632, 1063.2) [a] 
(0.9495389583333298, 1064.3) [a] 
(0.9495454166666631, 1064.7) [a] 
(0.9495456249999965, 1099.6) [a] 
(0.9495470833333297, 1130) [a] 
(0.9495481249999963, 1130.1) [a] 
(0.9495483333333297, 1130.9) [a] 
(0.9495487499999963, 1131.4) [a] 
(0.9495493749999963, 1132) [a] 
(0.9495497916666629, 1139.1) [a] 
(0.9495502083333295, 1157.4) [a] 
(0.9495529166666627, 1177.5) [a] 
(0.9495535416666627, 1177.6) [a] 
(0.9495581249999959, 1177.7) [a] 
(0.9495587499999959, 1178) [a] 
(0.9495606249999959, 1178.4) [a] 
(0.9495608333333293, 1180.1) [a] 
(0.9495622916666626, 1225.2) [a] 
(0.9495629166666626, 1339.9) [a] 
(0.9495633333333292, 1341.1) [a] 
(0.9495841666666626, 1385.6) [a] 
(0.9495845833333292, 1398.6) [a] 
(0.9495914583333291, 1399.2) [a] 
(0.949606458333329, 1399.3) [a] 
(0.9496085416666623, 1400.7) [a] 
(0.9496162499999956, 1400.8) [a] 
(0.949619583333329, 1410.2) [a] 
(0.9496431249999956, 1422.8) [a] 
(0.949666458333329, 1434.5) [a] 
(0.9496672916666623, 1439.2) [a] 
(0.9496677083333289, 1770) [a] 
(0.9496679166666623, 1793.9) [a] 
(0.9496681249999956, 1933.1) [a] 
(0.949668333333329, 1933.5) [a] 
(0.9496687499999956, 1988.9) [a] 
(0.949668958333329, 1989.1) [a] 
(0.9496691666666623, 1989.7) [a] 
(0.9496756249999957, 2014.9) [a] 
(0.9497279166666625, 2048.9) [a] 
(0.9497345833333292, 2052.1) [a] 
(0.9497410416666625, 2053.6) [a] 
(0.9497412499999959, 2054.1) [a] 
(0.9497416666666625, 2080.6) [a] 
(0.9497483333333292, 2327.1) [a] 
(0.9497547916666625, 2327.2) [a] 
(0.9497549999999959, 2471.8) [a] 
(0.9497554166666625, 2504.6) [a] 
(0.9497556249999959, 2536.5) [a] 
(0.9497558333333292, 2673.9) [a] 
(0.9497689583333292, 3397.5) [a] 
(0.9497754166666625, 3405.6) [a] 
(0.9498012499999958, 3422.7) [a] 
(0.9498524999999958, 3422.9) [a] 
(0.9499293749999959, 3424) [a] 
(0.9499808333333292, 3424.3) [a] 
(0.9500064583333292, 3425.3) [a] 
(0.9500299999999958, 3431.4) [a] 
(0.9500558333333291, 3499.9) [a] 
(0.950107083333329, 3500.2) [a] 
(0.9501839583333291, 3501.2) [a] 
(0.9502354166666624, 3501.5) [a] 
(0.9502610416666625, 3502.6) [a] 
(0.9502616666666625, 3575) [a] 
},{(0.8320577141972472, 0) [b] 
(0.8785612655830084, 0.001) [b] 
(0.88766039577434, 0.002) [b] 
(0.8904697129059341, 0.003) [b] 
(0.8933552480410565, 0.004) [b] 
(0.8952262991649315, 0.005) [b] 
(0.8964664304149172, 0.006) [b] 
(0.8973171179255857, 0.007) [b] 
(0.897544470024074, 0.008) [b] 
(0.8979746672205384, 0.009) [b] 
(0.8985733898382936, 0.01) [b] 
(0.8989129789102164, 0.011) [b] 
(0.8994975094264307, 0.013) [b] 
(0.9005339578435448, 0.014) [b] 
(0.9014228778684537, 0.015) [b] 
(0.9015344674582934, 0.016) [b] 
(0.9019498619270643, 0.017) [b] 
(0.9020206952603976, 0.018) [b] 
(0.9023516090652087, 0.019) [b] 
(0.902663326289818, 0.02) [b] 
(0.9026719404486476, 0.021) [b] 
(0.9027321497785713, 0.023) [b] 
(0.9031447978902969, 0.024) [b] 
(0.903295930223186, 0.025) [b] 
(0.9032987102249652, 0.027) [b] 
(0.9033221976357131, 0.028) [b] 
(0.9033628226357131, 0.03) [b] 
(0.9033659476357131, 0.031) [b] 
(0.903699486223562, 0.032) [b] 
(0.9040586816258608, 0.034) [b] 
(0.9041665555142061, 0.036) [b] 
(0.9043073212799718, 0.037) [b] 
(0.9044480870457375, 0.038) [b] 
(0.9048416681268187, 0.039) [b] 
(0.9048472281303771, 0.04) [b] 
(0.905040129364945, 0.045) [b] 
(0.905064242019266, 0.048) [b] 
(0.9050906331321564, 0.049) [b] 
(0.9051041747988231, 0.05) [b] 
(0.9051392442432675, 0.051) [b] 
(0.9051457750687639, 0.052) [b] 
(0.9059395272927653, 0.053) [b] 
(0.9059402217372097, 0.054) [b] 
(0.905947166181654, 0.056) [b] 
(0.9060497934394537, 0.057) [b] 
(0.9060837808054567, 0.058) [b] 
(0.9061864080632563, 0.06) [b] 
(0.9062098954740042, 0.062) [b] 
(0.9062515621406709, 0.068) [b] 
(0.9063135661089249, 0.071) [b] 
(0.906326398466999, 0.072) [b] 
(0.906388402435253, 0.074) [b] 
(0.9069525545951498, 0.075) [b] 
(0.9069589707741869, 0.076) [b] 
(0.9070464707741869, 0.077) [b] 
(0.9072591798028338, 0.08) [b] 
(0.9074091798028338, 0.083) [b] 
(0.9075911242472782, 0.084) [b] 
(0.9076451143422622, 0.09) [b] 
(0.9076646700368806, 0.092) [b] 
(0.9076917967729917, 0.099) [b] 
(0.9078092736248435, 0.107) [b] 
(0.9078333862791645, 0.11) [b] 
(0.9080017890569423, 0.111) [b] 
(0.9082402873208312, 0.112) [b] 
(0.9083982360111706, 0.113) [b] 
(0.9084620440765101, 0.114) [b] 
(0.9084943608746443, 0.115) [b] 
(0.908495048466094, 0.116) [b] 
(0.9103127300436626, 0.119) [b] 
(0.910400298575043, 0.12) [b] 
(0.9104564531123296, 0.121) [b] 
(0.91060930319754, 0.122) [b] 
(0.9106310498920424, 0.123) [b] 
(0.9106615525762786, 0.125) [b] 
(0.9106650247985008, 0.127) [b] 
(0.9107407192429452, 0.131) [b] 
(0.91157404626678, 0.137) [b] 
(0.9117136296001133, 0.139) [b] 
(0.9118345193486626, 0.144) [b] 
(0.9118411165708848, 0.148) [b] 
(0.9118551552052065, 0.15) [b] 
(0.9119061968718732, 0.151) [b] 
(0.9119065440940954, 0.153) [b] 
(0.9119225163163176, 0.156) [b] 
(0.9119818705423385, 0.157) [b] 
(0.9121058288756718, 0.158) [b] 
(0.9122256205423385, 0.159) [b] 
(0.9122311760978941, 0.16) [b] 
(0.9124468672354602, 0.161) [b] 
(0.9126197839021268, 0.165) [b] 
(0.9126673533465712, 0.166) [b] 
(0.9128395755687935, 0.167) [b] 
(0.9128878394576824, 0.168) [b] 
(0.9130048533465712, 0.169) [b] 
(0.9132017283465712, 0.17) [b] 
(0.913993395013238, 0.171) [b] 
(0.9141385339021268, 0.172) [b] 
(0.9142065894576824, 0.173) [b] 
(0.9145982561243491, 0.174) [b] 
(0.9146645755687934, 0.175) [b] 
(0.9150034644576823, 0.176) [b] 
(0.9153340200132379, 0.177) [b] 
(0.9154531172354602, 0.178) [b] 
(0.9154670061243491, 0.179) [b] 
(0.91553436723546, 0.18) [b] 
(0.9155392283465711, 0.181) [b] 
(0.9159600616799044, 0.182) [b] 
(0.9160819366799045, 0.183) [b] 
(0.9165173533465711, 0.184) [b] 
(0.9165677005687933, 0.185) [b] 
(0.91666874223546, 0.186) [b] 
(0.9167114505687933, 0.187) [b] 
(0.9169638811243489, 0.188) [b] 
(0.917082957572592, 0.189) [b] 
(0.9171270547948143, 0.19) [b] 
(0.9171277492392587, 0.191) [b] 
(0.9173228881281476, 0.192) [b] 
(0.9176699069711457, 0.193) [b] 
(0.9176956014155901, 0.194) [b] 
(0.9177004625267012, 0.195) [b] 
(0.9177008097489234, 0.196) [b] 
(0.9180181708600345, 0.197) [b] 
(0.9181469903044789, 0.198) [b] 
(0.9182167819711455, 0.199) [b] 
(0.9183702541933677, 0.201) [b] 
(0.9183766703724048, 0.203) [b] 
(0.9189995870390715, 0.205) [b] 
(0.9190311842612937, 0.207) [b] 
(0.9192995870390714, 0.209) [b] 
(0.9194193787057381, 0.211) [b] 
(0.9201124718714376, 0.214) [b] 
(0.9206694163158821, 0.217) [b] 
(0.9207476390943552, 0.219) [b] 
(0.9209729267803516, 0.221) [b] 
(0.9210091491416366, 0.223) [b] 
(0.9211278575936783, 0.226) [b] 
(0.9211754270381227, 0.228) [b] 
(0.9211785520381227, 0.229) [b] 
(0.9212219548159005, 0.231) [b] 
(0.9212795937047894, 0.232) [b] 
(0.9212927881492339, 0.234) [b] 
(0.9213297070125072, 0.243) [b] 
(0.9213300542347294, 0.248) [b] 
(0.9213413315248775, 0.249) [b] 
(0.9214006857508984, 0.256) [b] 
(0.9214071019299355, 0.265) [b] 
(0.9214091852632688, 0.27) [b] 
(0.9217653106193942, 0.272) [b] 
(0.9225311687221159, 0.286) [b] 
(0.9225546561328638, 0.289) [b] 
(0.9225997950217527, 0.297) [b] 
(0.9226591492477736, 0.303) [b] 
(0.9226910532804433, 0.315) [b] 
(0.9227867653784525, 0.316) [b] 
(0.9227881889065245, 0.33) [b] 
(0.9227981536030283, 0.332) [b] 
(0.9228686158352718, 0.335) [b] 
(0.922899118519508, 0.336) [b] 
(0.9229601238879804, 0.337) [b] 
(0.9231017376833277, 0.339) [b] 
(0.9231627430518001, 0.34) [b] 
(0.9232908680518002, 0.347) [b] 
(0.9234545737800771, 0.352) [b] 
(0.9234574208362211, 0.357) [b] 
(0.9235184262046935, 0.361) [b] 
(0.9235343984269158, 0.366) [b] 
(0.9236468571446017, 0.368) [b] 
(0.9236705045150142, 0.375) [b] 
(0.9236738827107874, 0.39) [b] 
(0.923684017298107, 0.391) [b] 
(0.9236957063527513, 0.401) [b] 
(0.9239689633886454, 0.402) [b] 
(0.9241272967219787, 0.408) [b] 
(0.9241321098621263, 0.409) [b] 
(0.924132797453576, 0.412) [b] 
(0.924136923002274, 0.423) [b] 
(0.9241390063356073, 0.431) [b] 
(0.9241636320251266, 0.463) [b] 
(0.9241967008081954, 0.465) [b] 
(0.924245952187234, 0.468) [b] 
(0.9242705778767534, 0.478) [b] 
(0.9243132591218283, 0.483) [b] 
(0.9243378848113476, 0.488) [b] 
(0.9243625105008669, 0.493) [b] 
(0.9243833438342003, 0.503) [b] 
(0.9246184678978707, 0.528) [b] 
(0.9246215928978707, 0.544) [b] 
(0.9246247178978707, 0.548) [b] 
(0.9249492822982807, 0.552) [b] 
(0.924949976742725, 0.556) [b] 
(0.9249717234372274, 0.56) [b] 
(0.9249727651038941, 0.584) [b] 
(0.9249769317705608, 0.588) [b] 
(0.9249800567705608, 0.607) [b] 
(0.9249821401038941, 0.611) [b] 
(0.9249831817705608, 0.616) [b] 
(0.9250133901038942, 0.625) [b] 
(0.925397742683327, 0.628) [b] 
(0.9254112843499936, 0.631) [b] 
(0.9254154510166603, 0.652) [b] 
(0.9255336038466342, 0.667) [b] 
(0.9255367288466342, 0.675) [b] 
(0.9256499232910786, 0.756) [b] 
(0.9256541938752946, 0.768) [b] 
(0.9257979702105639, 0.807) [b] 
(0.9259150114090657, 0.81) [b] 
(0.9259217678006122, 0.811) [b] 
(0.9259473913059078, 0.845) [b] 
(0.925950769501681, 0.852) [b] 
(0.9260110511374835, 0.858) [b] 
(0.9260152178041502, 0.863) [b] 
(0.9260185959999234, 0.876) [b] 
(0.9260326346342451, 1.112) [b] 
(0.9260446909614056, 1.145) [b] 
(0.9260929162700476, 1.149) [b] 
(0.9262099574685495, 1.157) [b] 
(0.9262148185796606, 1.161) [b] 
(0.9262179435796606, 1.228) [b] 
(0.9262299999068211, 1.255) [b] 
(0.9262420562339816, 1.256) [b] 
(0.9262739602666513, 1.346) [b] 
(0.9262777797110958, 1.38) [b] 
(0.9262784741555402, 1.388) [b] 
(0.9263025868098612, 1.423) [b] 
(0.9263146431370217, 1.426) [b] 
(0.9263210593160588, 1.477) [b] 
(0.9263221009827255, 1.523) [b] 
(0.9266321208239953, 1.683) [b] 
(0.9267657534945105, 1.785) [b] 
(0.9267851071916197, 1.829) [b] 
(0.9268378849693975, 1.85) [b] 
(0.9268383457717095, 1.876) [b] 
(0.9268876513272651, 1.907) [b] 
(0.9269291235353561, 1.949) [b] 
(0.9313415535893437, 2.041) [b] 
(0.9313672104037936, 2.062) [b] 
(0.9313681320084178, 2.074) [b] 
(0.9313690536130421, 2.077) [b] 
(0.931447835427492, 2.104) [b] 
(0.9314513076497142, 2.165) [b] 
(0.9316061687608252, 2.236) [b] 
(0.9316700576497141, 2.258) [b] 
(0.9317634604274919, 2.261) [b] 
(0.9317738770941586, 2.318) [b] 
(0.931852696538603, 2.326) [b] 
(0.9318537382052697, 2.332) [b] 
(0.9318547638066238, 2.377) [b] 
(0.9318552033500612, 2.421) [b] 
(0.9318616195290983, 2.438) [b] 
(0.9319084945290983, 2.509) [b] 
(0.931915786195765, 2.547) [b] 
(0.9319627610172606, 2.589) [b] 
(0.9319932637014968, 2.832) [b] 
(0.9320445773303967, 2.836) [b] 
(0.9321888084540908, 2.845) [b] 
(0.9322401220829907, 2.878) [b] 
(0.932246652908487, 3.196) [b] 
(0.9322532501307091, 3.331) [b] 
(0.9322803768668203, 3.36) [b] 
(0.9323974180653222, 3.549) [b] 
(0.9324443928868178, 3.722) [b] 
(0.9324461289979289, 3.91) [b] 
(0.9324612560098214, 3.984) [b] 
(0.9324619436012711, 3.985) [b] 
(0.9324646939670698, 3.99) [b] 
(0.9324688195157678, 3.998) [b] 
(0.9324701946986671, 4.079) [b] 
(0.9324708822901168, 4.081) [b] 
(0.9324814807432955, 4.12) [b] 
(0.9325095580119388, 4.232) [b] 
(0.9325235966462605, 4.234) [b] 
(0.9325516739149038, 4.246) [b] 
(0.9325523615063535, 4.534) [b] 
(0.9326064126387247, 4.705) [b] 
(0.9326300600091371, 4.706) [b] 
(0.9326537073795494, 4.713) [b] 
(0.9326672201626423, 4.714) [b] 
(0.9326880534959756, 4.731) [b] 
(0.9326894286788749, 4.733) [b] 
(0.9327102620122083, 4.742) [b] 
(0.9327213731233194, 4.907) [b] 
(0.9327234564566527, 5.002) [b] 
(0.9327244981233194, 5.036) [b] 
(0.9327862456331437, 5.124) [b] 
(0.932795000877074, 5.148) [b] 
(0.9327959224816983, 5.164) [b] 
(0.9328019129117558, 5.174) [b] 
(0.9328086693033022, 5.264) [b] 
(0.9328120474990754, 5.266) [b] 
(0.9328154256948487, 5.268) [b] 
(0.9328221820863951, 5.277) [b] 
(0.9328225293086173, 5.31) [b] 
(0.9329837156400164, 5.703) [b] 
(0.9329870938357896, 5.952) [b] 
(0.9329877814272393, 6.35) [b] 
(0.9329941976062763, 6.67) [b] 
(0.933223364272943, 7.595) [b] 
(0.9332650309396097, 7.599) [b] 
(0.9333066976062764, 7.6) [b] 
(0.9333275309396097, 7.603) [b] 
(0.9333483642729431, 7.627) [b] 
(0.9334733642729431, 7.675) [b] 
(0.9335358642729431, 7.676) [b] 
(0.9335566976062765, 7.708) [b] 
(0.9335897663893452, 7.765) [b] 
(0.9336559039554828, 7.766) [b] 
(0.9337120584927694, 7.897) [b] 
(0.9337260971270911, 7.898) [b] 
(0.9337541743957344, 7.902) [b] 
(0.9337682130300561, 7.908) [b] 
(0.9337962902986994, 7.916) [b] 
(0.9339226380075942, 7.993) [b] 
(0.9339647539105592, 7.994) [b] 
(0.9339671844661148, 8.219) [b] 
(0.9339812231004365, 8.311) [b] 
(0.9340083498365477, 8.361) [b] 
(0.934089730044881, 8.362) [b] 
(0.9341168567809922, 8.384) [b] 
(0.9341234540032144, 8.475) [b] 
(0.9341777074754366, 8.773) [b] 
(0.9342048342115478, 8.799) [b] 
(0.9342265809060502, 9.24) [b] 
(0.9342537076421613, 9.502) [b] 
(0.9342544020866057, 9.549) [b] 
(0.9343570293444053, 9.657) [b] 
(0.9343826861588552, 9.658) [b] 
(0.9344083429733051, 9.66) [b] 
(0.934433999787755, 9.672) [b] 
(0.9344596566022049, 9.784) [b] 
(0.9344853134166548, 9.98) [b] 
(0.9345109702311046, 9.982) [b] 
(0.9346135974889043, 10.035) [b] 
(0.9346392543033542, 10.037) [b] 
(0.934664911117804, 10.039) [b] 
(0.9346905679322539, 10.051) [b] 
(0.9347162247467038, 10.171) [b] 
(0.934743351482815, 10.33) [b] 
(0.9347690082972648, 10.37) [b] 
(0.9347946651117147, 10.372) [b] 
(0.9348010812907518, 10.618) [b] 
(0.9348074974697889, 10.724) [b] 
(0.9348139136488259, 10.763) [b] 
(0.9348374010595738, 12.43) [b] 
(0.9348494573867343, 14.701) [b] 
(0.934853624053401, 16.561) [b] 
(0.934877736707722, 16.59) [b] 
(0.9348897930348825, 16.601) [b] 
(0.9348939185835805, 17.043) [b] 
(0.9348987317237282, 17.045) [b] 
(0.9348994193151778, 17.046) [b] 
(0.9349114756423383, 17.233) [b] 
(0.9349235319694988, 17.316) [b] 
(0.9349355882966593, 17.384) [b] 
(0.934938018852215, 17.406) [b] 
(0.9350620267887229, 17.82) [b] 
(0.9351860347252308, 17.823) [b] 
(0.9352480386934847, 17.861) [b] 
(0.9353100426617387, 17.899) [b] 
(0.9353431114448074, 18.236) [b] 
(0.9353750154774771, 18.426) [b] 
(0.9354069195101469, 18.427) [b] 
(0.9354689234784008, 18.519) [b] 
(0.935475520700623, 18.839) [b] 
(0.935537524668877, 19.161) [b] 
(0.9355389135577659, 20.059) [b] 
(0.9355509698849264, 21.396) [b] 
(0.9355744572956742, 21.997) [b] 
(0.9355754989623409, 22.63) [b] 
(0.9355989863730888, 23.153) [b] 
(0.9356795795387883, 25.256) [b] 
(0.9356802739832327, 25.438) [b] 
(0.9357107766674689, 25.64) [b] 
(0.9357303323620872, 26.149) [b] 
(0.93573685092696, 26.784) [b] 
(0.93574310092696, 26.822) [b] 
(0.9357549064825156, 26.829) [b] 
(0.93575560092696, 26.867) [b] 
(0.9358176048952139, 27.481) [b] 
(0.9358196882285472, 27.951) [b] 
(0.935851592261217, 29.787) [b] 
(0.9358834962938867, 29.789) [b] 
(0.9358900271193831, 29.98) [b] 
(0.9359161504213684, 29.986) [b] 
(0.9359226812468648, 29.992) [b] 
(0.9359292120723612, 30.004) [b] 
(0.935981458676332, 30.011) [b] 
(0.9359879895018284, 30.065) [b] 
(0.9360020281361501, 30.5) [b] 
(0.9360301054047934, 30.504) [b] 
(0.9360367026270155, 30.547) [b] 
(0.936043233452512, 30.782) [b] 
(0.9360572720868336, 30.846) [b] 
(0.9360853493554769, 30.847) [b] 
(0.936091765534514, 32.916) [b] 
(0.9361188922706252, 33.273) [b] 
(0.936190731351085, 34.082) [b] 
(0.936229916304063, 34.089) [b] 
(0.9363469575025649, 34.529) [b] 
(0.936364318613676, 35.52) [b] 
(0.9363726519470094, 35.574) [b] 
(0.9363809852803426, 35.695) [b] 
(0.9363834158358982, 35.984) [b] 
(0.9363844575025649, 37.754) [b] 
(0.9363882769470094, 38.77) [b] 
(0.9364201809796792, 39.224) [b] 
(0.9364503893130125, 41.227) [b] 
(0.9364507365352347, 43.128) [b] 
(0.936452111718134, 43.241) [b] 
(0.9364618339403562, 46.465) [b] 
(0.9364621811625784, 46.531) [b] 
(0.9364625283848006, 47.476) [b] 
(0.9364649589403562, 47.538) [b] 
(0.9364683371361294, 47.818) [b] 
(0.936548930301829, 47.913) [b] 
(0.9365496178932786, 54.515) [b] 
(0.9365503054847283, 54.536) [b] 
(0.936550993076178, 54.918) [b] 
(0.9365645058592709, 56.755) [b] 
(0.9365780186423638, 56.762) [b] 
(0.936581396838137, 56.765) [b] 
(0.9365847750339102, 56.775) [b] 
(0.9365881532296835, 56.863) [b] 
(0.9365949096212299, 56.898) [b] 
(0.936611800600096, 56.902) [b] 
(0.9366185569916424, 56.949) [b] 
(0.9366219351874157, 57.089) [b] 
(0.9366254074096378, 57.236) [b] 
(0.9366267962985267, 58.251) [b] 
(0.9366301744943, 58.475) [b] 
(0.9366335526900732, 58.479) [b] 
(0.9366403090816195, 58.521) [b] 
(0.9366674358177307, 58.615) [b] 
(0.9366802681758049, 61.275) [b] 
(0.9366866843548419, 61.292) [b] 
(0.9366900625506152, 62.692) [b] 
(0.936715719365065, 63.661) [b] 
(0.9367560159479148, 65.107) [b] 
(0.9368855298368037, 65.816) [b] 
(0.9368914326145815, 65.883) [b] 
(0.9368917798368037, 65.991) [b] 
(0.9369015020590259, 66.158) [b] 
(0.9369042798368037, 66.386) [b] 
(0.936925113170137, 66.403) [b] 
(0.9369486005808849, 67.64) [b] 
(0.9370010311364405, 68.214) [b] 
(0.9370090172475516, 68.307) [b] 
(0.9370346740620015, 68.325) [b] 
(0.9370380522577747, 69.453) [b] 
(0.9370383994799969, 70.66) [b] 
(0.937043260591108, 70.823) [b] 
(0.9370547189244414, 72.595) [b] 
(0.9370581911466636, 73.578) [b] 
(0.937059656291455, 73.723) [b] 
(0.9371234643567945, 74.366) [b] 
(0.9371244899581486, 77.623) [b] 
(0.9371501467725984, 78.774) [b] 
(0.9371758035870483, 85.08) [b] 
(0.9371764980314927, 86.086) [b] 
(0.9372547208099659, 91.948) [b] 
(0.9372612393748386, 91.963) [b] 
(0.9372873136343297, 93.644) [b] 
(0.9373108010450776, 96.346) [b] 
(0.9373510976279273, 97.657) [b] 
(0.9373517920723717, 99.245) [b] 
(0.937357000405705, 99.75) [b] 
(0.9373653337390384, 99.755) [b] 
(0.9373663754057051, 99.763) [b] 
(0.9373715837390384, 99.771) [b] 
(0.9373736670723717, 99.936) [b] 
(0.9373767920723717, 100.018) [b] 
(0.9373823476279273, 100.3) [b] 
(0.937383389294594, 100.604) [b] 
(0.9373844309612607, 100.738) [b] 
(0.9373875559612607, 101.182) [b] 
(0.9373885976279274, 101.247) [b] 
(0.9373896392945941, 101.329) [b] 
(0.9373906809612608, 102.865) [b] 
(0.9374163377757107, 107.911) [b] 
(0.9374933082190604, 107.912) [b] 
(0.9375189650335103, 107.957) [b] 
(0.9375446218479602, 113.409) [b] 
(0.93757027866241, 113.978) [b] 
(0.9376021826950798, 115.498) [b] 
(0.9376278395095297, 117.574) [b] 
(0.9377048099528794, 117.575) [b] 
(0.9377304667673293, 117.625) [b] 
(0.9377561235817792, 123.729) [b] 
(0.9377817803962291, 124.363) [b] 
(0.9377821276184513, 127.407) [b] 
(0.9377828152099009, 129.859) [b] 
(0.9377835028013506, 129.861) [b] 
(0.9377841903928003, 129.874) [b] 
(0.9377848779842499, 129.888) [b] 
(0.9377855655756996, 130.126) [b] 
(0.9378258621585494, 130.354) [b] 
(0.9378589309416181, 132.472) [b] 
(0.9378599726082848, 136.916) [b] 
(0.937887099344396, 136.989) [b] 
(0.9378881410110627, 137.002) [b] 
(0.9379011781408082, 137.34) [b] 
(0.9379014711697665, 138.633) [b] 
(0.9381355535667703, 138.817) [b] 
(0.9381376369001035, 141.496) [b] 
(0.9381424980112146, 142.048) [b] 
(0.938145935968463, 145.551) [b] 
(0.938149060968463, 145.721) [b] 
(0.938153186517161, 146.255) [b] 
(0.9382158217523278, 147.664) [b] 
(0.9382165161967722, 148.584) [b] 
(0.9382168634189944, 150.466) [b] 
(0.9382296751716421, 155.58) [b] 
(0.9382302612295588, 186.376) [b] 
(0.9382366774085958, 199.391) [b] 
(0.9382430935876329, 201.292) [b] 
(0.9382665809983808, 203.373) [b] 
(0.9382900684091287, 203.385) [b] 
(0.9383135558198765, 203.414) [b] 
(0.9383370432306244, 204.16) [b] 
(0.9383605306413723, 206.606) [b] 
(0.9383840180521201, 207.346) [b] 
(0.9383904342311572, 211.529) [b] 
(0.9384223382638269, 217.692) [b] 
(0.9385818584271756, 217.694) [b] 
(0.9386137624598453, 217.695) [b] 
(0.938645666492515, 217.702) [b] 
(0.9386775705251847, 217.712) [b] 
(0.9387094745578545, 217.846) [b] 
(0.9387732826231939, 217.884) [b] 
(0.9388051866558637, 217.885) [b] 
(0.9388370906885334, 217.93) [b] 
(0.9389008987538728, 223.74) [b] 
(0.9389328027865426, 223.744) [b] 
(0.9389647068192123, 223.944) [b] 
(0.9389952095034485, 228.509) [b] 
(0.9390562148719209, 228.51) [b] 
(0.9390600343163654, 230.425) [b] 
(0.9390601808308445, 235.685) [b] 
(0.9390836682415924, 245.282) [b] 
(0.9390870464373656, 253.946) [b] 
(0.9390877340288153, 263.663) [b] 
(0.939088421620265, 263.67) [b] 
(0.9390891092117146, 263.812) [b] 
(0.939090484394614, 273.724) [b] 
(0.9390918595775133, 273.725) [b] 
(0.939105898211835, 283.244) [b] 
(0.9391177037673906, 283.665) [b] 
(0.9391399259896128, 283.725) [b] 
(0.9391718300222825, 287.876) [b] 
(0.9392675421202917, 287.878) [b] 
(0.9392994461529615, 287.879) [b] 
(0.9393313501856312, 287.894) [b] 
(0.9393951582509706, 287.967) [b] 
(0.9394270622836404, 287.969) [b] 
(0.9394284511725293, 291.95) [b] 
(0.9394768511269763, 295.882) [b] 
(0.9394886566825319, 303.265) [b] 
(0.9394994205714208, 303.318) [b] 
(0.9396164617699226, 303.77) [b] 
(0.939617156214367, 304.105) [b] 
(0.9396192395477003, 304.838) [b] 
(0.9396830476130398, 308.167) [b] 
(0.9397149516457095, 308.17) [b] 
(0.9397468556783792, 308.171) [b] 
(0.9397475501228236, 311.377) [b] 
(0.9397527584561569, 318.275) [b] 
(0.9397592770210297, 323.684) [b] 
(0.9397723141507752, 323.685) [b] 
(0.9399389808174419, 325.078) [b] 
(0.9403348141507752, 325.08) [b] 
(0.9403556474841086, 325.195) [b] 
(0.9404598141507753, 325.249) [b] 
(0.940501480817442, 325.25) [b] 
(0.9405223141507754, 325.252) [b] 
(0.9405431474841087, 325.256) [b] 
(0.9406056474841087, 326.346) [b] 
(0.940626480817442, 326.36) [b] 
(0.9406681474841088, 326.37) [b] 
(0.9406889808174421, 329.141) [b] 
(0.9407098141507755, 329.149) [b] 
(0.9407306474841088, 335.777) [b] 
(0.9407514808174422, 338.007) [b] 
(0.9407833848501119, 344.452) [b] 
(0.9408250515167786, 350.543) [b] 
(0.940845884850112, 350.544) [b] 
(0.9408730115862232, 358.298) [b] 
(0.9409146782528899, 361.618) [b] 
(0.9409563449195566, 361.619) [b] 
(0.9409980115862233, 361.815) [b] 
(0.9410188449195567, 362.511) [b] 
(0.9411873085314165, 363.405) [b] 
(0.9412715403373464, 363.407) [b] 
(0.9412855789716681, 363.409) [b] 
(0.9413136562403114, 363.482) [b] 
(0.9413276948746331, 363.483) [b] 
(0.9413417335089548, 363.504) [b] 
(0.9413557721432765, 363.818) [b] 
(0.9413698107775982, 363.85) [b] 
(0.9413731889733714, 364.377) [b] 
(0.9413940223067048, 365.929) [b] 
(0.9414080609410265, 366.302) [b] 
(0.9414361382096698, 367.748) [b] 
(0.9414782541126347, 369.35) [b] 
(0.9415203700155997, 369.351) [b] 
(0.9415344086499214, 369.417) [b] 
(0.9416186404558513, 379.256) [b] 
(0.941632679090173, 379.26) [b] 
(0.9416467177244947, 379.272) [b] 
(0.9416607563588164, 380.296) [b] 
(0.9417421365671498, 386.795) [b] 
(0.9417692633032609, 386.796) [b] 
(0.9417963900393721, 386.797) [b] 
(0.9418506435115943, 386.803) [b] 
(0.9418777702477055, 386.818) [b] 
(0.9419320237199277, 387.473) [b] 
(0.9419591504560388, 388.862) [b] 
(0.94198627719215, 388.875) [b] 
(0.9420003158264717, 390.194) [b] 
(0.9420143544607934, 390.199) [b] 
(0.9420283930951151, 390.265) [b] 
(0.9420564703637584, 390.939) [b] 
(0.9420985862667234, 392.617) [b] 
(0.9423326686637271, 411.769) [b] 
(0.9423394250552735, 412.336) [b] 
(0.9423428032510467, 413.418) [b] 
(0.94236363658438, 419.296) [b] 
(0.9423844699177134, 419.301) [b] 
(0.9424053032510468, 419.491) [b] 
(0.9424261365843801, 420.312) [b] 
(0.9424678032510468, 447.472) [b] 
(0.9424962738124864, 462.864) [b] 
(0.9424976489953857, 480.68) [b] 
(0.9425101489953857, 481.828) [b] 
(0.9425181351064967, 481.929) [b] 
(0.9425216073287189, 482.843) [b] 
(0.9425257739953856, 483.018) [b] 
(0.94252646843983, 486) [b] 
(0.9425473017731634, 487.715) [b] 
(0.9425681351064967, 492.27) [b] 
(0.9425684823287189, 492.746) [b] 
(0.9425698712176078, 493.95) [b] 
(0.94257021843983, 510.367) [b] 
(0.9425715936227294, 555.776) [b] 
(0.9425770943543267, 555.777) [b] 
(0.9425777819457763, 555.779) [b] 
(0.9425798447201253, 555.796) [b] 
(0.942580532311575, 555.798) [b] 
(0.9425812199030247, 555.853) [b] 
(0.9425819074944743, 555.927) [b] 
(0.9425878102722521, 578.477) [b] 
(0.9425881574944743, 579.338) [b] 
(0.942742324161141, 586.951) [b] 
(0.9427440602722521, 587.027) [b] 
(0.9427451019389188, 587.187) [b] 
(0.9427482269389188, 589.903) [b] 
(0.942748574161141, 589.996) [b] 
(0.9427510047166966, 590.213) [b] 
(0.9427513519389188, 593.69) [b] 
(0.9427836436055854, 595.638) [b] 
(0.9427940602722521, 598.346) [b] 
(0.9427975324944743, 598.407) [b] 
(0.942798574161141, 598.97) [b] 
(0.9427989213833632, 599.319) [b] 
(0.9427992686055854, 599.741) [b] 
(0.9428123302565781, 614.704) [b] 
(0.9428188610820745, 614.709) [b] 
(0.9428384535585634, 614.715) [b] 
(0.9428449843840598, 614.721) [b] 
(0.942845331606282, 622.154) [b] 
(0.9428456788285042, 631.762) [b] 
(0.9428458253429833, 658.377) [b] 
(0.943188895608329, 681.641) [b] 
(0.9431988603048328, 740.028) [b] 
(0.9432320380713055, 828.793) [b] 
(0.9432908155007727, 836.648) [b] 
(0.9433104079772616, 836.655) [b] 
(0.943316938802758, 836.896) [b] 
(0.9433176332472024, 878.04) [b] 
(0.9433402125604964, 881.884) [b] 
(0.9433467311253692, 933.911) [b] 
(0.9433532496902419, 934.169) [b] 
(0.9433597682551147, 934.177) [b] 
(0.9434768094536166, 936.037) [b] 
(0.943486878898061, 944.215) [b] 
(0.9434955594536166, 944.431) [b] 
(0.9435018094536166, 945.03) [b] 
(0.943502503898061, 1037.36) [b] 
(0.9435049344536166, 1045.87) [b] 
(0.9435073650091722, 1078.7) [b] 
(0.9435084066758389, 1084.98) [b] 
(0.9435355334119501, 1119.46) [b] 
(0.9435626601480612, 1147.98) [b] 
(0.9435827990369501, 1148.55) [b] 
(0.9435907851480612, 1148.62) [b] 
(0.9435925212591723, 1148.7) [b] 
(0.9435932157036167, 1148.76) [b] 
(0.9435942573702834, 1148.9) [b] 
(0.9435946045925055, 1149.04) [b] 
(0.9435959934813944, 1158.21) [b] 
(0.9436178684813944, 1188.31) [b] 
(0.9436737712591722, 1188.41) [b] 
(0.9436789795925055, 1189.02) [b] 
(0.9436796740369499, 1195.27) [b] 
(0.9436860902159869, 1230.55) [b] 
(0.9438875731302359, 1262.61) [b] 
(0.9439278697130856, 1263.44) [b] 
(0.9439681662959354, 1263.99) [b] 
(0.9440084628787851, 1267.99) [b] 
(0.9440132760189328, 1288.61) [b] 
(0.9440146512018321, 1289.25) [b] 
(0.9440194643419798, 1289.27) [b] 
(0.9440213075512283, 1340.63) [b] 
(0.9440247797734505, 1438.57) [b] 
(0.9440254742178948, 1447.32) [b] 
(0.9440275575512281, 1484.41) [b] 
(0.944028946440117, 1500.93) [b] 
(0.9440546032545669, 1759.21) [b] 
(0.9440802600690168, 1759.59) [b] 
(0.9441059168834667, 1761.46) [b] 
(0.9441315736979166, 1763.15) [b] 
(0.9441572305123664, 1765.47) [b] 
(0.9441575777345886, 1802.23) [b] 
(0.9441603555123664, 1802.28) [b] 
(0.9441610499568108, 1802.95) [b] 
(0.9441867067712607, 1815.52) [b] 
(0.9442123635857106, 1815.56) [b] 
(0.9442179191412662, 1885.54) [b] 
(0.9442435759557161, 1975.6) [b] 
(0.944269232770166, 1975.66) [b] 
(0.9442948895846158, 1975.82) [b] 
(0.9443205463990657, 1979.11) [b] 
(0.9443462032135156, 1981.78) [b] 
(0.9443466640158277, 2012.47) [b] 
(0.9443723208302776, 2027.94) [b] 
(0.9443979776447274, 2029.02) [b] 
(0.9444236344591773, 2031.96) [b] 
(0.9444492912736272, 2270.08) [b] 
(0.9445306714819606, 2279.91) [b] 
(0.9445447101162823, 2290.44) [b] 
(0.944558748750604, 2290.67) [b] 
(0.944600864653569, 2291.2) [b] 
(0.9446279913896801, 2296.18) [b] 
(0.9446420300240018, 2360.33) [b] 
(0.9446427244684462, 2483.51) [b] 
(0.9446444605795573, 2488.9) [b] 
(0.9446451550240017, 2493.2) [b] 
(0.9446458494684461, 2494.57) [b] 
(0.9446555716906683, 2601.7) [b] 
(0.944659738357335, 2601.71) [b] 
(0.9446611272462239, 2601.75) [b] 
(0.9446632105795572, 2601.87) [b] 
(0.9446639050240015, 2602.01) [b] 
(0.9446659883573348, 2603.22) [b] 
(0.9446687661351126, 2603.24) [b] 
(0.9446708494684459, 2603.32) [b] 
(0.9446722383573348, 2615.3) [b] 
(0.9446729328017792, 2615.42) [b] 
(0.9446736272462236, 2615.52) [b] 
(0.9446846287094182, 2676.95) [b] 
(0.9447190082819014, 2677.24) [b] 
(0.944719695873351, 2677.26) [b] 
(0.9447210710562504, 2677.41) [b] 
(0.9447231338305994, 2677.99) [b] 
(0.944723821422049, 2678) [b] 
(0.9447251966049484, 2679.93) [b] 
(0.9447561382201832, 2689.28) [b] 
(0.9447623265432302, 2689.34) [b] 
(0.9447637017261296, 2689.35) [b] 
(0.9447671396833779, 2689.43) [b] 
(0.9447705776406262, 2689.48) [b] 
(0.9447712720850706, 2720.96) [b] 
(0.944771966529515, 2758.33) [b] 
(0.9447726609739594, 2761.94) [b] 
(0.9447733554184038, 2762.07) [b] 
(0.9447740498628482, 2762.34) [b] 
(0.9447747443072926, 2762.81) [b] 
(0.9447761194901919, 3013.04) [b] 
(0.9447774946730912, 3013.1) [b] 
(0.9447781822645409, 3013.7) [b] 
(0.9447788698559906, 3014.9) [b] 
(0.9447795574474402, 3014.92) [b] 
(0.9447797039619193, 3019.79) [b] 
(0.9448068306980305, 3286.48) [b] 
(0.9448075182894802, 3327.61) [b] 
(0.9448082058809298, 3327.64) [b] 
(0.9448088934723795, 3327.65) [b] 
(0.9448095810638292, 3327.75) [b] 
(0.9448102686552788, 3327.76) [b] 
(0.9448116438381782, 3328.29) [b] 
(0.9448123314296278, 3335.9) [b] 
(0.9448130190210775, 3336.82) [b] 
(0.9448131655355566, 3483.49) [b] 
},{(0.9320879791666666, 0.001) [c] 
(0.9320879791666666, 3.683737729166666) [c] 
(0.9320879791666666, 3600) [c] 
}}}{legend pos=north west}}
%
% 	\subfloat[depth=8]{\cactus{Average Accuracy}{CPU time}{\budalg, \murtree, \cart}{{{(0.9114044084619304, 0) [a] 
(0.9220837226005072, 0.01) [a] 
(0.9268135142671737, 0.02) [a] 
(0.9317708753782848, 0.03) [a] 
(0.9363843476005069, 0.04) [a] 
(0.9365616392671735, 0.05) [a] 
(0.9367974726005068, 0.06) [a] 
(0.9369528892671735, 0.07) [a] 
(0.93698080593384, 0.08) [a] 
(0.93698268093384, 0.09) [a] 
(0.9370133059338396, 0.1) [a] 
(0.9370739309338396, 0.11) [a] 
(0.9372039309338397, 0.12) [a] 
(0.9373478892671727, 0.13) [a] 
(0.9373487226005061, 0.14) [a] 
(0.9373495559338394, 0.15) [a] 
(0.9373914309338393, 0.16) [a] 
(0.9373955976005057, 0.17) [a] 
(0.9375370559338391, 0.18) [a] 
(0.937537680933839, 0.19) [a] 
(0.9375405976005056, 0.2) [a] 
(0.9375422642671722, 0.21) [a] 
(0.9375964309338389, 0.22) [a] 
(0.9376241392671723, 0.23) [a] 
(0.9376393476005054, 0.24) [a] 
(0.9376664309338386, 0.25) [a] 
(0.9376683059338385, 0.27) [a] 
(0.9409625828527545, 0.28) [a] 
(0.9409692495194212, 0.29) [a] 
(0.9409840411860877, 0.3) [a] 
(0.9409909161860878, 0.31) [a] 
(0.9409942495194211, 0.32) [a] 
(0.9410182078527544, 0.33) [a] 
(0.9410190411860877, 0.34) [a] 
(0.9410425828527543, 0.37) [a] 
(0.941046124519421, 0.39) [a] 
(0.9410600828527543, 0.4) [a] 
(0.9410634161860877, 0.41) [a] 
(0.9410640411860877, 0.42) [a] 
(0.9410667495194209, 0.43) [a] 
(0.9410682078527542, 0.44) [a] 
(0.941069249519421, 0.45) [a] 
(0.9454911499045197, 0.47) [a] 
(0.9454915665711864, 0.48) [a] 
(0.945491983237853, 0.49) [a] 
(0.9455398999045199, 0.51) [a] 
(0.9455401082378533, 0.52) [a] 
(0.9456771915711868, 0.53) [a] 
(0.9456807332378535, 0.55) [a] 
(0.9456811499045201, 0.56) [a] 
(0.9457080249045202, 0.6) [a] 
(0.9457117749045203, 0.61) [a] 
(0.945725941571187, 0.63) [a] 
(0.945807191571187, 0.64) [a] 
(0.9458076082378536, 0.67) [a] 
(0.9458436499045203, 0.68) [a] 
(0.9458467749045202, 0.7) [a] 
(0.9458478165711869, 0.71) [a] 
(0.9459867749045202, 0.72) [a] 
(0.9460690665711868, 0.73) [a] 
(0.9460701082378535, 0.74) [a] 
(0.9460973999045202, 0.75) [a] 
(0.9464201082378536, 0.76) [a] 
(0.9464480249045202, 0.77) [a] 
(0.9465298999045202, 0.78) [a] 
(0.9465578165711868, 0.79) [a] 
(0.9465851082378535, 0.8) [a] 
(0.9465855249045201, 0.81) [a] 
(0.9465903165711866, 0.85) [a] 
(0.94659052490452, 0.92) [a] 
(0.9472960416666649, 0.98) [a] 
(0.9472993749999983, 1.03) [a] 
(0.9473556249999983, 1.05) [a] 
(0.9473695833333315, 1.06) [a] 
(0.9473837499999982, 1.07) [a] 
(0.9473870833333315, 1.09) [a] 
(0.9475666666666648, 1.11) [a] 
(0.9477270833333316, 1.12) [a] 
(0.9478041666666649, 1.15) [a] 
(0.9478074999999982, 1.16) [a] 
(0.9478079166666649, 1.17) [a] 
(0.9478081249999982, 1.21) [a] 
(0.9478089583333315, 1.26) [a] 
(0.9478091666666648, 1.27) [a] 
(0.947988749999998, 1.33) [a] 
(0.9481431249999981, 1.34) [a] 
(0.9481441666666648, 1.35) [a] 
(0.9481697916666648, 1.37) [a] 
(0.9482216666666647, 1.38) [a] 
(0.9482249999999981, 1.41) [a] 
(0.948226249999998, 1.42) [a] 
(0.948587499999998, 1.43) [a] 
(0.9486131249999981, 1.45) [a] 
(0.9487758333333314, 1.53) [a] 
(0.9488933333333313, 1.54) [a] 
(0.949098749999998, 1.68) [a] 
(0.9492527083333312, 1.69) [a] 
(0.9493020833333312, 1.7) [a] 
(0.9493024999999978, 1.77) [a] 
(0.9493027083333312, 1.8) [a] 
(0.9493031249999978, 1.83) [a] 
(0.9493033333333312, 1.91) [a] 
(0.9493035416666645, 1.95) [a] 
(0.9493039583333311, 2.09) [a] 
(0.9493043749999978, 2.34) [a] 
(0.949318333333331, 2.59) [a] 
(0.9493502083333311, 2.79) [a] 
(0.9494141666666644, 2.8) [a] 
(0.9494460416666645, 2.81) [a] 
(0.9494474999999978, 2.87) [a] 
(0.9494504166666644, 2.95) [a] 
(0.9494568749999978, 3.08) [a] 
(0.9494577083333311, 3.63) [a] 
(0.9494583333333311, 3.66) [a] 
(0.9494779166666644, 3.77) [a] 
(0.9494785416666643, 3.93) [a] 
(0.9494787499999977, 4.1) [a] 
(0.9496454166666646, 4.16) [a] 
(0.9496662499999979, 4.17) [a] 
(0.9496870833333313, 4.21) [a] 
(0.949728749999998, 4.38) [a] 
(0.9497289583333314, 4.55) [a] 
(0.9497560416666647, 4.9) [a] 
(0.9497564583333313, 5) [a] 
(0.949756874999998, 5.19) [a] 
(0.9497708333333312, 5.2) [a] 
(0.9497849999999979, 5.23) [a] 
(0.9498437499999978, 5.24) [a] 
(0.949857708333331, 5.25) [a] 
(0.9499543749999976, 5.47) [a] 
(0.9499822916666643, 5.48) [a] 
(0.9499964583333309, 5.69) [a] 
(0.9499972916666642, 5.84) [a] 
(0.9500291666666643, 6.03) [a] 
(0.9502206249999977, 6.08) [a] 
(0.9502589583333311, 6.1) [a] 
(0.9502908333333311, 6.13) [a] 
(0.9503227083333311, 6.15) [a] 
(0.9503545833333311, 6.16) [a] 
(0.9504185416666644, 6.2) [a] 
(0.950418958333331, 6.5) [a] 
(0.9504254166666644, 6.52) [a] 
(0.950425833333331, 6.53) [a] 
(0.950426458333331, 6.72) [a] 
(0.9504268749999976, 6.8) [a] 
(0.9504304166666643, 7.21) [a] 
(0.9504308333333309, 7.34) [a] 
(0.9504627083333309, 7.59) [a] 
(0.9504835416666643, 8.13) [a] 
(0.9504868749999976, 8.15) [a] 
(0.9504935416666643, 8.16) [a] 
(0.950517708333331, 8.24) [a] 
(0.9505245833333311, 8.31) [a] 
(0.9505249999999977, 8.36) [a] 
(0.9505283333333311, 8.5) [a] 
(0.9505316666666644, 8.59) [a] 
(0.9505418749999979, 9) [a] 
(0.9505424999999978, 9.5) [a] 
(0.9505429166666645, 9.65) [a] 
(0.9505491666666644, 10.85) [a] 
(0.9505495833333311, 11.27) [a] 
(0.950550208333331, 11.41) [a] 
(0.9505568749999977, 12.06) [a] 
(0.9505637499999978, 12.07) [a] 
(0.9505956249999978, 12.84) [a] 
(0.9506274999999978, 13.21) [a] 
(0.9506279166666645, 13.67) [a] 
(0.9506283333333311, 13.68) [a] 
(0.9506285416666644, 13.69) [a] 
(0.9506287499999978, 14.49) [a] 
(0.9506297916666644, 14.53) [a] 
(0.950630208333331, 14.55) [a] 
(0.9506306249999976, 14.63) [a] 
(0.9506312499999976, 14.66) [a] 
(0.9506316666666642, 14.67) [a] 
(0.9506381249999976, 14.7) [a] 
(0.9506385416666642, 14.84) [a] 
(0.9506449999999975, 14.98) [a] 
(0.9506514583333309, 14.99) [a] 
(0.9506524999999976, 15.46) [a] 
(0.9506664583333309, 15.84) [a] 
(0.9506668749999975, 15.89) [a] 
(0.9506670833333308, 15.9) [a] 
(0.9506679166666642, 15.96) [a] 
(0.9506683333333308, 16.04) [a] 
(0.9507806249999974, 16.19) [a] 
(0.9507924999999974, 16.4) [a] 
(0.9508287499999973, 16.44) [a] 
(0.950829166666664, 16.57) [a] 
(0.9508572916666639, 16.58) [a] 
(0.9508712499999972, 16.59) [a] 
(0.9508854166666638, 16.62) [a] 
(0.9508993749999971, 16.63) [a] 
(0.9509618749999972, 16.86) [a] 
(0.9510243749999971, 16.87) [a] 
(0.9510452083333305, 16.89) [a] 
(0.9510660416666639, 16.9) [a] 
(0.9511077083333306, 16.95) [a] 
(0.9512743749999975, 17.01) [a] 
(0.9512952083333308, 17.04) [a] 
(0.9513993749999976, 17.11) [a] 
(0.9513999999999976, 17.57) [a] 
(0.9514208333333309, 17.83) [a] 
(0.9514416666666643, 18.33) [a] 
(0.9514422916666643, 20.17) [a] 
(0.9514429166666643, 20.19) [a] 
(0.9514443749999976, 20.2) [a] 
(0.9514447916666642, 20.22) [a] 
(0.9514449999999975, 20.24) [a] 
(0.9514452083333309, 20.69) [a] 
(0.9514708333333309, 20.78) [a] 
(0.9514712499999975, 20.87) [a] 
(0.9514968749999976, 20.9) [a] 
(0.951650833333331, 20.94) [a] 
(0.9516518749999977, 21.08) [a] 
(0.951703333333331, 21.28) [a] 
(0.951728958333331, 21.9) [a] 
(0.9517545833333311, 22.02) [a] 
(0.9517804166666644, 22.06) [a] 
(0.9519085416666645, 22.07) [a] 
(0.9519599999999978, 22.42) [a] 
(0.9519602083333312, 23.1) [a] 
(0.9519606249999978, 23.11) [a] 
(0.9519610416666644, 23.13) [a] 
(0.9519620833333311, 23.16) [a] 
(0.9519627083333311, 23.19) [a] 
(0.951963333333331, 23.3) [a] 
(0.9519697916666644, 23.32) [a] 
(0.9520060416666645, 23.49) [a] 
(0.9520181249999978, 23.54) [a] 
(0.9520437499999979, 23.55) [a] 
(0.9520452083333312, 23.64) [a] 
(0.9520454166666645, 23.68) [a] 
(0.9520458333333311, 23.69) [a] 
(0.9520477083333311, 23.7) [a] 
(0.9520485416666643, 23.72) [a] 
(0.9520499999999976, 23.83) [a] 
(0.952050833333331, 23.95) [a] 
(0.9520812499999977, 24.18) [a] 
(0.952081458333331, 24.25) [a] 
(0.9520818749999976, 24.36) [a] 
(0.9520822916666642, 24.37) [a] 
(0.9520824999999976, 24.41) [a] 
(0.9520833333333308, 24.44) [a] 
(0.9520835416666642, 24.69) [a] 
(0.9520839583333308, 24.7) [a] 
(0.9521095833333308, 24.82) [a] 
(0.9521099999999975, 24.93) [a] 
(0.9521102083333308, 27.35) [a] 
(0.9521106249999974, 27.83) [a] 
(0.952111041666664, 27.86) [a] 
(0.9521112499999974, 27.89) [a] 
(0.952111666666664, 27.9) [a] 
(0.9521120833333306, 28.1) [a] 
(0.952112291666664, 28.11) [a] 
(0.9521127083333306, 28.18) [a] 
(0.9521131249999972, 28.77) [a] 
(0.9521133333333306, 28.78) [a] 
(0.9521137499999972, 29.36) [a] 
(0.9521139583333306, 29.38) [a] 
(0.9521147916666638, 29.39) [a] 
(0.9521962499999972, 29.65) [a] 
(0.9522504166666639, 29.76) [a] 
(0.9522506249999972, 29.95) [a] 
(0.9522508333333306, 30.03) [a] 
(0.9522512499999972, 30.06) [a] 
(0.9522516666666638, 30.94) [a] 
(0.9522520833333304, 32.38) [a] 
(0.9522524999999971, 32.75) [a] 
(0.9522797916666638, 33.18) [a] 
(0.9523068749999971, 33.22) [a] 
(0.9523079166666638, 34.08) [a] 
(0.9523087499999972, 34.09) [a] 
(0.9523089583333305, 34.36) [a] 
(0.9523093749999971, 34.5) [a] 
(0.9523097916666637, 34.51) [a] 
(0.9523368749999971, 36.78) [a] 
(0.9523910416666638, 36.79) [a] 
(0.9523916666666637, 37.16) [a] 
(0.9523924999999971, 37.19) [a] 
(0.9523927083333305, 37.43) [a] 
(0.9524199999999972, 37.57) [a] 
(0.9524204166666638, 37.75) [a] 
(0.9524210416666637, 37.86) [a] 
(0.9524418749999971, 37.91) [a] 
(0.9524424999999971, 38.21) [a] 
(0.9524695833333304, 40.62) [a] 
(0.9524704166666638, 40.83) [a] 
(0.9524710416666637, 40.84) [a] 
(0.9524714583333304, 40.86) [a] 
(0.9524716666666637, 41.05) [a] 
(0.9524720833333303, 41.06) [a] 
(0.952472499999997, 41.08) [a] 
(0.9524727083333303, 41.09) [a] 
(0.9524731249999969, 41.37) [a] 
(0.9524733333333303, 42.21) [a] 
(0.9524737499999969, 42.22) [a] 
(0.9524741666666635, 43.14) [a] 
(0.9524745833333301, 43.15) [a] 
(0.9525914583333301, 43.31) [a] 
(0.9525916666666635, 43.35) [a] 
(0.9525924999999967, 43.36) [a] 
(0.9525935416666634, 43.4) [a] 
(0.9525941666666634, 43.75) [a] 
(0.95259458333333, 44.57) [a] 
(0.9525949999999966, 44.89) [a] 
(0.95259520833333, 45.15) [a] 
(0.95259583333333, 45.2) [a] 
(0.9525962499999966, 45.21) [a] 
(0.9525977083333299, 45.8) [a] 
(0.9526116666666632, 46.64) [a] 
(0.9526118749999966, 46.93) [a] 
(0.9526122916666632, 46.98) [a] 
(0.9526404166666631, 47) [a] 
(0.9526410416666631, 47.03) [a] 
(0.9526552083333297, 47.06) [a] 
(0.9526695833333296, 47.18) [a] 
(0.9526699999999962, 47.26) [a] 
(0.9526839583333295, 47.71) [a] 
(0.9526989583333294, 47.73) [a] 
(0.9527410416666626, 47.74) [a] 
(0.9527416666666626, 47.92) [a] 
(0.9527570833333292, 48.28) [a] 
(0.9527712499999959, 48.39) [a] 
(0.9527716666666625, 52.89) [a] 
(0.9527731249999958, 56.23) [a] 
(0.9527741666666624, 56.24) [a] 
(0.9527747916666623, 56.25) [a] 
(0.9527956249999957, 57.82) [a] 
(0.952809583333329, 58.35) [a] 
(0.9528099999999956, 62.13) [a] 
(0.952837083333329, 63.6) [a] 
(0.9528374999999956, 65.81) [a] 
(0.9528377083333289, 65.84) [a] 
(0.9528381249999955, 65.85) [a] 
(0.9528383333333289, 66.37) [a] 
(0.9528387499999955, 66.41) [a] 
(0.9528391666666621, 66.64) [a] 
(0.9528395833333287, 66.87) [a] 
(0.9528397916666621, 66.88) [a] 
(0.9528408333333288, 67.13) [a] 
(0.9528679166666622, 70.13) [a] 
(0.9528681249999955, 70.24) [a] 
(0.9528685416666621, 70.31) [a] 
(0.9528691666666621, 70.32) [a] 
(0.9528724999999955, 71.78) [a] 
(0.9528731249999954, 73.38) [a] 
(0.9528870833333287, 74.82) [a] 
(0.9529152083333288, 75.34) [a] 
(0.9529293749999954, 75.44) [a] 
(0.9529433333333287, 75.46) [a] 
(0.952957291666662, 76.01) [a] 
(0.952963541666662, 76.97) [a] 
(0.9529777083333286, 77.48) [a] 
(0.9529781249999952, 80.8) [a] 
(0.9529785416666618, 81.37) [a] 
(0.9529789583333285, 82.65) [a] 
(0.9529799999999952, 83.9) [a] 
(0.9529804166666618, 83.95) [a] 
(0.9529806249999951, 83.96) [a] 
(0.9529808333333285, 88.47) [a] 
(0.9530016666666619, 88.9) [a] 
(0.9530224999999952, 89.02) [a] 
(0.9530641666666619, 89.06) [a] 
(0.9530849999999953, 91.52) [a] 
(0.9530852083333287, 94.02) [a] 
(0.9530862499999954, 96.93) [a] 
(0.9532033333333287, 97.33) [a] 
(0.9532035416666621, 104.8) [a] 
(0.9532049999999954, 119.52) [a] 
(0.9532070833333287, 119.53) [a] 
(0.9532104166666621, 119.54) [a] 
(0.9532108333333287, 145.69) [a] 
(0.953211041666662, 148.93) [a] 
(0.9532114583333287, 155.7) [a] 
(0.9532118749999953, 155.72) [a] 
(0.9532120833333286, 156.29) [a] 
(0.953212916666662, 156.3) [a] 
(0.9532139583333287, 156.34) [a] 
(0.953214166666662, 156.36) [a] 
(0.9532145833333286, 156.44) [a] 
(0.9532149999999953, 156.45) [a] 
(0.9532420833333286, 156.54) [a] 
(0.953242291666662, 163.36) [a] 
(0.9532427083333286, 163.37) [a] 
(0.9532433333333286, 163.4) [a] 
(0.9532437499999952, 163.48) [a] 
(0.9532441666666618, 163.5) [a] 
(0.9532443749999951, 167.27) [a] 
(0.9532447916666618, 168.76) [a] 
(0.9532449999999951, 178.11) [a] 
(0.9532454166666617, 178.52) [a] 
(0.953285624999995, 182.45) [a] 
(0.9532860416666616, 182.81) [a] 
(0.9532864583333283, 182.83) [a] 
(0.9532866666666616, 183.27) [a] 
(0.9532870833333282, 183.72) [a] 
(0.9532874999999948, 187.72) [a] 
(0.9532877083333282, 187.81) [a] 
(0.9532908333333282, 187.82) [a] 
(0.9532912499999948, 193.61) [a] 
(0.9532916666666614, 201.47) [a] 
(0.9533172916666615, 220.22) [a] 
(0.9533429166666615, 220.43) [a] 
(0.9533972916666615, 226.72) [a] 
(0.9534243749999949, 226.88) [a] 
(0.9534247916666615, 226.91) [a] 
(0.9534252083333281, 226.92) [a] 
(0.9534254166666615, 226.93) [a] 
(0.9534262499999947, 226.95) [a] 
(0.9534264583333281, 226.98) [a] 
(0.9534268749999947, 227.08) [a] 
(0.9534272916666613, 227.12) [a] 
(0.9534279166666613, 227.17) [a] 
(0.9534283333333279, 227.2) [a] 
(0.9534285416666612, 227.23) [a] 
(0.9534289583333279, 227.26) [a] 
(0.9534299999999946, 227.29) [a] 
(0.9534306249999945, 227.33) [a] 
(0.9534310416666611, 227.42) [a] 
(0.9534320833333279, 227.49) [a] 
(0.9534324999999945, 227.58) [a] 
(0.9534331249999944, 227.59) [a] 
(0.9534337499999944, 227.6) [a] 
(0.9534358333333277, 227.63) [a] 
(0.9534362499999943, 227.76) [a] 
(0.953437291666661, 227.88) [a] 
(0.953437916666661, 227.89) [a] 
(0.9534383333333276, 227.92) [a] 
(0.9534970833333276, 229.57) [a] 
(0.9535099999999943, 230.24) [a] 
(0.9535362499999943, 230.25) [a] 
(0.9535427083333277, 231.58) [a] 
(0.9535545833333277, 238) [a] 
(0.9535549999999943, 241.29) [a] 
(0.9535554166666609, 245.32) [a] 
(0.9535562499999942, 248.68) [a] 
(0.9535566666666608, 248.94) [a] 
(0.9535568749999942, 249.71) [a] 
(0.9535572916666608, 251.08) [a] 
(0.9535574999999942, 284.46) [a] 
(0.9535579166666608, 286.52) [a] 
(0.9535581249999941, 286.56) [a] 
(0.9535589583333274, 286.6) [a] 
(0.9535595833333274, 286.68) [a] 
(0.953559999999994, 286.71) [a] 
(0.9535606249999939, 286.72) [a] 
(0.9535610416666606, 286.73) [a] 
(0.9535612499999939, 286.74) [a] 
(0.9535620833333273, 286.75) [a] 
(0.9535627083333272, 287.18) [a] 
(0.9535691666666606, 287.7) [a] 
(0.9535756249999939, 288.42) [a] 
(0.9535760416666605, 293.65) [a] 
(0.9535762499999939, 319.58) [a] 
(0.9535766666666605, 329.1) [a] 
(0.9535777083333272, 336.55) [a] 
(0.9535781249999938, 338.89) [a] 
(0.9535783333333272, 338.99) [a] 
(0.9535787499999938, 339.01) [a] 
(0.9535791666666604, 339.06) [a] 
(0.9535793749999938, 341.71) [a] 
(0.9535797916666604, 360.37) [a] 
(0.953580208333327, 360.38) [a] 
(0.9535804166666604, 360.39) [a] 
(0.9535924999999937, 361.17) [a] 
(0.9536133333333271, 362.2) [a] 
(0.9536254166666605, 365.52) [a] 
(0.9536374999999938, 365.95) [a] 
(0.9536389583333271, 379.67) [a] 
(0.9536393749999937, 379.78) [a] 
(0.9536397916666604, 384.13) [a] 
(0.9537568749999937, 391.44) [a] 
(0.9537579166666603, 394.47) [a] 
(0.9537583333333269, 394.5) [a] 
(0.9537585416666603, 394.51) [a] 
(0.9537793749999937, 408.99) [a] 
(0.953800208333327, 409.38) [a] 
(0.9538006249999936, 415.66) [a] 
(0.9538012499999936, 416.79) [a] 
(0.953918333333327, 420.18) [a] 
(0.9539191666666603, 428.34) [a] 
(0.9539204166666603, 428.35) [a] 
(0.9539412499999936, 472.55) [a] 
(0.9539829166666604, 472.64) [a] 
(0.9540245833333271, 473.05) [a] 
(0.9540454166666604, 473.06) [a] 
(0.9540870833333271, 473.07) [a] 
(0.9541106249999938, 558.4) [a] 
(0.9541341666666604, 558.49) [a] 
(0.9541356249999937, 566.97) [a] 
(0.9541564583333271, 577.24) [a] 
(0.9542735416666605, 612.8) [a] 
(0.9542739583333271, 623.99) [a] 
(0.954274583333327, 649.82) [a] 
(0.954275208333327, 660.78) [a] 
(0.9542754166666604, 674.07) [a] 
(0.954275833333327, 734.12) [a] 
(0.9542760416666604, 734.13) [a] 
(0.9542768749999937, 734.14) [a] 
(0.9542770833333271, 734.24) [a] 
(0.9542774999999937, 734.27) [a] 
(0.9542781249999936, 734.28) [a] 
(0.9542785416666603, 734.33) [a] 
(0.9542799999999936, 734.5) [a] 
(0.9542802083333269, 734.57) [a] 
(0.9542806249999936, 734.58) [a] 
(0.9542810416666602, 734.61) [a] 
(0.9542812499999935, 735.08) [a] 
(0.9542816666666601, 735.19) [a] 
(0.9542820833333268, 740.86) [a] 
(0.9542824999999934, 745.69) [a] 
(0.9542956249999933, 750.49) [a] 
(0.95429666666666, 752.64) [a] 
(0.9542968749999934, 758.78) [a] 
(0.9542970833333267, 769.22) [a] 
(0.9543252083333268, 771.71) [a] 
(0.9543391666666601, 771.72) [a] 
(0.9543531249999934, 771.81) [a] 
(0.95436729166666, 771.86) [a] 
(0.9544233333333266, 772.02) [a] 
(0.95446541666666, 773.01) [a] 
(0.9544795833333266, 773.02) [a] 
(0.9544935416666599, 775.58) [a] 
(0.9545077083333265, 776.41) [a] 
(0.9545085416666598, 780.25) [a] 
(0.9545087499999931, 780.29) [a] 
(0.9545227083333264, 784.42) [a] 
(0.9545366666666597, 784.74) [a] 
(0.9545508333333264, 784.76) [a] 
(0.954578749999993, 825.73) [a] 
(0.9546068749999931, 825.76) [a] 
(0.9546210416666597, 825.91) [a] 
(0.9546489583333263, 832.71) [a] 
(0.954663124999993, 832.95) [a] 
(0.9546770833333262, 832.97) [a] 
(0.9546772916666596, 926.8) [a] 
(0.954677499999993, 926.81) [a] 
(0.954705624999993, 931.77) [a] 
(0.9547195833333263, 938.13) [a] 
(0.9547337499999929, 938.16) [a] 
(0.9547339583333263, 942.69) [a] 
(0.9547341666666597, 942.7) [a] 
(0.954734374999993, 942.82) [a] 
(0.9547377083333264, 959.9) [a] 
(0.9547379166666597, 1038.6) [a] 
(0.9547412499999931, 1113.2) [a] 
(0.9547445833333265, 1116) [a] 
(0.9547449999999931, 1151.9) [a] 
(0.9547454166666597, 1160.2) [a] 
(0.954753124999993, 1171.5) [a] 
(0.954753749999993, 1171.6) [a] 
(0.9547545833333263, 1171.8) [a] 
(0.9547558333333263, 1171.9) [a] 
(0.9547566666666596, 1172.1) [a] 
(0.9547649999999928, 1172.5) [a] 
(0.9547656249999927, 1172.8) [a] 
(0.954767083333326, 1173) [a] 
(0.9547704166666593, 1173.6) [a] 
(0.9547712499999926, 1176.6) [a] 
(0.9547718749999926, 1183) [a] 
(0.9547724999999926, 1190.6) [a] 
(0.9547739583333259, 1192.8) [a] 
(0.9547747916666592, 1222.7) [a] 
(0.9547760416666591, 1225.2) [a] 
(0.9547849999999923, 1230.7) [a] 
(0.9547856249999923, 1267.3) [a] 
(0.9547862499999923, 1267.4) [a] 
(0.9547866666666589, 1267.5) [a] 
(0.9547872916666589, 1267.6) [a] 
(0.9547877083333255, 1270.9) [a] 
(0.9547881249999921, 1271) [a] 
(0.9547883333333255, 1300.9) [a] 
(0.9547891666666588, 1301) [a] 
(0.9547893749999922, 1302.2) [a] 
(0.9547897916666588, 1302.7) [a] 
(0.9547902083333254, 1305.6) [a] 
(0.9547908333333254, 1360.8) [a] 
(0.954791874999992, 1360.9) [a] 
(0.9547935416666585, 1361) [a] 
(0.9547939583333251, 1361.2) [a] 
(0.9547943749999918, 1363.1) [a] 
(0.9547945833333251, 1376.3) [a] 
(0.9547954166666585, 1392.9) [a] 
(0.9547956249999918, 1404) [a] 
(0.9547960416666584, 1404.1) [a] 
(0.954796458333325, 1408.6) [a] 
(0.9547968749999917, 1450.1) [a] 
(0.954797083333325, 1482) [a] 
(0.9547974999999916, 1485) [a] 
(0.9547987499999916, 1508.8) [a] 
(0.9547991666666582, 1511.6) [a] 
(0.9547995833333248, 1523.1) [a] 
(0.9548127083333247, 1603.1) [a] 
(0.9548202083333245, 1603.5) [a] 
(0.9548216666666578, 1603.6) [a] 
(0.954830624999991, 1604.1) [a] 
(0.9548470833333244, 1604.2) [a] 
(0.9548506249999911, 1604.4) [a] 
(0.954851249999991, 1604.8) [a] 
(0.9548691666666577, 1611) [a] 
(0.9548927083333243, 1654) [a] 
(0.9549135416666577, 1675.8) [a] 
(0.9549552083333244, 1697.7) [a] 
(0.9549760416666577, 1713.4) [a] 
(0.9549764583333243, 1717.5) [a] 
(0.9549766666666577, 1726.3) [a] 
(0.9549770833333243, 1757.4) [a] 
(0.9549772916666577, 1816.4) [a] 
(0.954979374999991, 1981.3) [a] 
(0.9549854166666577, 1982.5) [a] 
(0.9549902083333243, 1991.5) [a] 
(0.9550158333333243, 2166) [a] 
(0.9550414583333243, 2166.1) [a] 
(0.9550927083333244, 2213.1) [a] 
(0.9551066666666577, 2259.7) [a] 
(0.9551322916666577, 2352.5) [a] 
(0.955158124999991, 2397.2) [a] 
(0.955183749999991, 2400.2) [a] 
(0.9552095833333243, 2445.1) [a] 
(0.9552127083333243, 2545.9) [a] 
(0.9552141666666575, 2546) [a] 
(0.9552152083333241, 2569.4) [a] 
(0.9552154166666574, 2575.8) [a] 
(0.9552156249999908, 2575.9) [a] 
(0.9552164583333241, 2576.6) [a] 
(0.9552166666666575, 2576.7) [a] 
(0.9552168749999909, 2660.3) [a] 
(0.9552233333333242, 2662.9) [a] 
(0.9553066666666576, 2724.4) [a] 
(0.955327499999991, 2724.7) [a] 
(0.9553691666666577, 2727.9) [a] 
(0.9553693749999911, 2836.4) [a] 
(0.9553708333333244, 2876.2) [a] 
(0.9553710416666578, 2919.3) [a] 
(0.9553712499999911, 3055.3) [a] 
(0.9553714583333245, 3251.2) [a] 
(0.9553720833333245, 3475.2) [a] 
},{(0.8382336937226955, 0) [b] 
(0.878129069674984, 0.001) [b] 
(0.8890266899770825, 0.002) [b] 
(0.8944582788610956, 0.003) [b] 
(0.898171366543596, 0.004) [b] 
(0.9008209088814668, 0.005) [b] 
(0.9025653564900693, 0.006) [b] 
(0.9031911654707528, 0.007) [b] 
(0.903514854814555, 0.008) [b] 
(0.9041084196741267, 0.009) [b] 
(0.9043417768766402, 0.01) [b] 
(0.9048647031821019, 0.011) [b] 
(0.9050166151295679, 0.012) [b] 
(0.9068590580565562, 0.013) [b] 
(0.9069349946841997, 0.014) [b] 
(0.9069544546966541, 0.015) [b] 
(0.9077637726609434, 0.016) [b] 
(0.9078419498740146, 0.017) [b] 
(0.9081963951749187, 0.018) [b] 
(0.9087018638232044, 0.019) [b] 
(0.909215367033434, 0.02) [b] 
(0.9093198809337828, 0.021) [b] 
(0.9097051241771922, 0.022) [b] 
(0.9097825565713902, 0.023) [b] 
(0.9097893129629366, 0.024) [b] 
(0.9102601973136455, 0.025) [b] 
(0.9102629773154247, 0.026) [b] 
(0.9103099521369203, 0.027) [b] 
(0.9103512076239001, 0.028) [b] 
(0.9103581131743773, 0.029) [b] 
(0.91037117482537, 0.03) [b] 
(0.9103862551263584, 0.031) [b] 
(0.9104608804921571, 0.032) [b] 
(0.9104702554921571, 0.033) [b] 
(0.9105039130537647, 0.036) [b] 
(0.9106626109937028, 0.038) [b] 
(0.9107082532084357, 0.042) [b] 
(0.910727808903054, 0.044) [b] 
(0.9108337116808318, 0.045) [b] 
(0.9108806865023275, 0.046) [b] 
(0.9108950887924756, 0.047) [b] 
(0.910901619617972, 0.048) [b] 
(0.911276619617972, 0.049) [b] 
(0.9115978001735277, 0.05) [b] 
(0.9116200223957499, 0.052) [b] 
(0.9116432862846388, 0.053) [b] 
(0.9116516196179721, 0.054) [b] 
(0.9116800918401943, 0.055) [b] 
(0.9116870362846387, 0.056) [b] 
(0.9116905085068608, 0.057) [b] 
(0.9118154669589026, 0.059) [b] 
(0.9118459696431388, 0.06) [b] 
(0.9118792501051527, 0.061) [b] 
(0.911900083438486, 0.062) [b] 
(0.9119042501051526, 0.063) [b] 
(0.9119375430267113, 0.064) [b] 
(0.911950604677704, 0.065) [b] 
(0.911989789630682, 0.072) [b] 
(0.9121640744122584, 0.074) [b] 
(0.912366895592814, 0.076) [b] 
(0.9123915212823334, 0.077) [b] 
(0.9125415212823333, 0.078) [b] 
(0.9125665212823334, 0.079) [b] 
(0.9125911469718527, 0.081) [b] 
(0.9130788726327167, 0.083) [b] 
(0.9132145063132723, 0.085) [b] 
(0.9132673004055695, 0.087) [b] 
(0.91327935673273, 0.089) [b] 
(0.9133980651847717, 0.09) [b] 
(0.9134258158742911, 0.091) [b] 
(0.91347095476318, 0.096) [b] 
(0.9137109900703094, 0.099) [b] 
(0.9137381168064206, 0.101) [b] 
(0.9138380763946459, 0.102) [b] 
(0.9140064791724236, 0.104) [b] 
(0.9140658333984445, 0.105) [b] 
(0.9148851217879576, 0.107) [b] 
(0.9148892884546242, 0.109) [b] 
(0.9153877072724169, 0.113) [b] 
(0.9161037062924213, 0.115) [b] 
(0.9161068312924213, 0.116) [b] 
(0.9161335297735379, 0.12) [b] 
(0.9161591865879878, 0.122) [b] 
(0.9162359226990989, 0.123) [b] 
(0.9162515476990989, 0.126) [b] 
(0.9163217408707072, 0.127) [b] 
(0.9164200113109587, 0.128) [b] 
(0.9165502196442921, 0.129) [b] 
(0.9165550807554032, 0.13) [b] 
(0.9165571640887364, 0.131) [b] 
(0.9165779974220698, 0.132) [b] 
(0.9165800807554031, 0.133) [b] 
(0.9166338249398586, 0.136) [b] 
(0.9166560471620807, 0.137) [b] 
(0.9167067416065251, 0.14) [b] 
(0.9167342155648585, 0.141) [b] 
(0.9168871524148859, 0.142) [b] 
(0.9169128092293358, 0.143) [b] 
(0.9169135036737802, 0.144) [b] 
(0.9169395077104523, 0.145) [b] 
(0.9169651645249022, 0.146) [b] 
(0.9169675950804578, 0.147) [b] 
(0.9169905117471244, 0.148) [b] 
(0.9170185890157677, 0.149) [b] 
(0.9170758806824344, 0.151) [b] 
(0.9171092140157677, 0.152) [b] 
(0.9173602932925784, 0.153) [b] 
(0.9173675849592451, 0.154) [b] 
(0.9173700155148007, 0.155) [b] 
(0.9173703627370229, 0.156) [b] 
(0.9180120294036896, 0.159) [b] 
(0.9180137655148007, 0.162) [b] 
(0.9180179321814674, 0.163) [b] 
(0.9180182794036896, 0.166) [b] 
(0.9180394599592452, 0.167) [b] 
(0.9180498766259119, 0.168) [b] 
(0.9180769599592452, 0.169) [b] 
(0.9181036960703562, 0.17) [b] 
(0.9181693210703562, 0.172) [b] 
(0.9182731405148007, 0.174) [b] 
(0.9183786960703563, 0.175) [b] 
(0.9184352932925784, 0.176) [b] 
(0.9186335571814673, 0.177) [b] 
(0.9186977932925785, 0.178) [b] 
(0.918914112737023, 0.179) [b] 
(0.918925223848134, 0.18) [b] 
(0.9192338515323701, 0.181) [b] 
(0.9194598403277175, 0.182) [b] 
(0.9196851875499397, 0.183) [b] 
(0.9200296319943841, 0.184) [b] 
(0.9202560208832731, 0.185) [b] 
(0.9205261597721618, 0.186) [b] 
(0.9205535903277173, 0.187) [b] 
(0.9213087986610506, 0.188) [b] 
(0.9213351875499395, 0.189) [b] 
(0.9219962986610507, 0.19) [b] 
(0.9225685208832728, 0.191) [b] 
(0.9228803264388283, 0.192) [b] 
(0.9231025486610505, 0.193) [b] 
(0.9232678264388282, 0.194) [b] 
(0.9234317153277171, 0.195) [b] 
(0.9235157431054948, 0.196) [b] 
(0.9238848403277169, 0.197) [b] 
(0.924000812549939, 0.198) [b] 
(0.9245848403277169, 0.199) [b] 
(0.9246428264388279, 0.2) [b] 
(0.9254247708832724, 0.201) [b] 
(0.9255216458832721, 0.202) [b] 
(0.9256799792166056, 0.203) [b] 
(0.9257143542166055, 0.204) [b] 
(0.9257424792166055, 0.205) [b] 
(0.9259436035552298, 0.206) [b] 
(0.9261147841107853, 0.207) [b] 
(0.9261696452218964, 0.208) [b] 
(0.9261738118885631, 0.209) [b] 
(0.9263123535552296, 0.21) [b] 
(0.9263227702218961, 0.211) [b] 
(0.9264252007774516, 0.212) [b] 
(0.9264297146663404, 0.213) [b] 
(0.9264300618885626, 0.214) [b] 
(0.9265109646663403, 0.217) [b] 
(0.9265123535552292, 0.218) [b] 
(0.926569992444118, 0.219) [b] 
(0.9265752007774514, 0.22) [b] 
(0.9266099229996736, 0.221) [b] 
(0.9266255479996736, 0.222) [b] 
(0.9266366591107846, 0.223) [b] 
(0.9266745063330069, 0.224) [b] 
(0.9266908257774513, 0.225) [b] 
(0.9266939507774513, 0.226) [b] 
(0.9267213813330069, 0.227) [b] 
(0.9272772841107846, 0.233) [b] 
(0.9272776313330068, 0.235) [b] 
(0.9272939507774512, 0.237) [b] 
(0.9272942979996734, 0.239) [b] 
(0.92729533966634, 0.244) [b] 
(0.9273317979996734, 0.245) [b] 
(0.9273852702218955, 0.246) [b] 
(0.9273866591107844, 0.248) [b] 
(0.9274300618885623, 0.249) [b] 
(0.9274877007774511, 0.25) [b] 
(0.9274995063330067, 0.251) [b] 
(0.9275290202218957, 0.252) [b] 
(0.9275422146663401, 0.253) [b] 
(0.9275477702218957, 0.255) [b] 
(0.9275894368885624, 0.262) [b] 
(0.9276519368885624, 0.263) [b] 
(0.9276540202218957, 0.264) [b] 
(0.9276567979996735, 0.276) [b] 
(0.9284226561023953, 0.283) [b] 
(0.9284461435131431, 0.293) [b] 
(0.9284666296242543, 0.303) [b] 
(0.9286159351798099, 0.312) [b] 
(0.9286394225905578, 0.321) [b] 
(0.9287112975905578, 0.322) [b] 
(0.9287127211186298, 0.33) [b] 
(0.9287226858151336, 0.332) [b] 
(0.9289567682121374, 0.338) [b] 
(0.9289667329086412, 0.343) [b] 
(0.9291115245753079, 0.347) [b] 
(0.929117940754345, 0.35) [b] 
(0.9291207185321227, 0.353) [b] 
(0.9292844242603997, 0.363) [b] 
(0.9293482323257392, 0.365) [b] 
(0.9293510793818831, 0.367) [b] 
(0.9293653146626029, 0.37) [b] 
(0.9293710087748908, 0.373) [b] 
(0.9294569639399317, 0.376) [b] 
(0.9294637203314782, 0.378) [b] 
(0.929480734220367, 0.379) [b] 
(0.929593192938053, 0.384) [b] 
(0.9296679108587059, 0.386) [b] 
(0.9296778755552098, 0.388) [b] 
(0.9297348166780887, 0.396) [b] 
(0.9297428027891997, 0.402) [b] 
(0.9297466222336442, 0.411) [b] 
(0.9298916453817924, 0.436) [b] 
(0.9299398706904344, 0.437) [b] 
(0.9300569118889362, 0.442) [b] 
(0.9300689682160967, 0.443) [b] 
(0.9300717459938745, 0.445) [b] 
(0.9300751241896478, 0.448) [b] 
(0.930078502385421, 0.468) [b] 
(0.9300905587125815, 0.475) [b] 
(0.9300936837125815, 0.493) [b] 
(0.9302690309348037, 0.511) [b] 
(0.9302693781570259, 0.543) [b] 
(0.9302763226014704, 0.55) [b] 
(0.9306008870018804, 0.58) [b] 
(0.9307179282003822, 0.624) [b] 
(0.931102280779815, 0.655) [b] 
(0.9311029752242594, 0.657) [b] 
(0.9311043641131483, 0.658) [b] 
(0.9311061002242594, 0.661) [b] 
(0.9313879587825102, 0.674) [b] 
(0.9314136155969601, 0.687) [b] 
(0.9315937723951879, 0.713) [b] 
(0.9316051606197637, 0.752) [b] 
(0.9316089800642082, 0.759) [b] 
(0.931646132841986, 0.76) [b] 
(0.9316952770671838, 0.771) [b] 
(0.9317271810998535, 0.797) [b] 
(0.9317292644331868, 0.815) [b] 
(0.9317611684658565, 0.835) [b] 
(0.9317654390500725, 0.851) [b] 
(0.9319092153853418, 0.89) [b] 
(0.9319119931631196, 0.894) [b] 
(0.9319376166684151, 0.928) [b] 
(0.932054657866917, 0.934) [b] 
(0.9320908268483985, 0.952) [b] 
(0.9321149395027195, 0.957) [b] 
(0.9321177865588635, 0.97) [b] 
(0.9322973842600129, 0.984) [b] 
(0.9323094405871734, 0.988) [b] 
(0.9323607542160732, 0.999) [b] 
(0.9323864110305231, 1.01) [b] 
(0.9325660087316725, 1.056) [b] 
(0.9326173223605724, 1.071) [b] 
(0.9326429791750223, 1.081) [b] 
(0.9326942928039221, 1.09) [b] 
(0.932719949618372, 1.095) [b] 
(0.9327712632472719, 1.157) [b] 
(0.9327969200617218, 1.161) [b] 
(0.932802128395055, 1.223) [b] 
(0.9328277852095049, 1.341) [b] 
(0.9328790988384048, 1.373) [b] 
(0.9329047556528547, 1.397) [b] 
(0.9329560692817546, 1.428) [b] 
(0.9329769026150879, 1.491) [b] 
(0.9330238774365835, 1.539) [b] 
(0.9330379160709052, 1.783) [b] 
(0.9331715487414205, 1.852) [b] 
(0.9331909024385296, 1.897) [b] 
(0.9331918240431539, 1.905) [b] 
(0.9331987360778358, 1.927) [b] 
(0.9332001184847721, 1.93) [b] 
(0.9332386601514387, 1.953) [b] 
(0.9332391209537508, 1.965) [b] 
(0.9332418857676236, 1.979) [b] 
(0.9332547181256977, 1.999) [b] 
(0.9332654820145866, 2.001) [b] 
(0.9376779120685742, 2.015) [b] 
(0.9377227037352409, 2.036) [b] 
(0.9377641759433318, 2.062) [b] 
(0.937811398165554, 2.067) [b] 
(0.9378147763613273, 2.101) [b] 
(0.9378526235835495, 2.12) [b] 
(0.9379217208057717, 2.124) [b] 
(0.9379290124724384, 2.13) [b] 
(0.9380158180279939, 2.131) [b] 
(0.9380469155561889, 2.185) [b] 
(0.9380519843816223, 2.209) [b] 
(0.9380558038260668, 2.25) [b] 
(0.9380766371594002, 2.253) [b] 
(0.9380775587640244, 2.258) [b] 
(0.9380784803686487, 2.265) [b] 
(0.9380789199120861, 2.363) [b] 
(0.9380792129410445, 2.383) [b] 
(0.9380802385423985, 2.385) [b] 
(0.938080678085836, 2.387) [b] 
(0.938080824600315, 2.388) [b] 
(0.9381138933833838, 2.391) [b] 
(0.9381143329268212, 2.43) [b] 
(0.9381146259557795, 2.432) [b] 
(0.9381286645901012, 2.436) [b] 
(0.9381474145901012, 2.453) [b] 
(0.9381478541335386, 2.463) [b] 
(0.9381674098281569, 2.466) [b] 
(0.9381688910382611, 2.491) [b] 
(0.9381720160382611, 2.775) [b] 
(0.9381753942340343, 2.866) [b] 
(0.938176435900701, 2.936) [b] 
(0.9381816442340343, 2.962) [b] 
(0.938192060900701, 2.969) [b] 
(0.9381941442340342, 3.036) [b] 
(0.9383383753577283, 3.061) [b] 
(0.9383463614688394, 3.087) [b] 
(0.9384634026673413, 3.118) [b] 
(0.93849647145041, 3.142) [b] 
(0.9385308510228931, 3.38) [b] 
(0.9385322262057925, 3.381) [b] 
(0.9385384145288395, 3.384) [b] 
(0.9385404978621728, 3.385) [b] 
(0.9385411854536224, 3.399) [b] 
(0.9385418730450721, 3.454) [b] 
(0.9385425606365217, 3.522) [b] 
(0.9385480613681191, 3.611) [b] 
(0.9385494365510184, 3.614) [b] 
(0.9385501241424681, 3.626) [b] 
(0.9386284940484242, 3.824) [b] 
(0.9386480865249132, 3.831) [b] 
(0.9386532948582464, 4.065) [b] 
(0.9386663565092391, 4.113) [b] 
(0.938883370398128, 4.222) [b] 
(0.9389104971342391, 4.224) [b] 
(0.9389376238703503, 4.225) [b] 
(0.9389918773425725, 4.23) [b] 
(0.9390190040786837, 4.236) [b] 
(0.9390272985203019, 4.239) [b] 
(0.9390293818536352, 4.29) [b] 
(0.9390434204879569, 4.367) [b] 
(0.9390855363909218, 4.394) [b] 
(0.939112663127033, 4.433) [b] 
(0.9391397898631442, 4.499) [b] 
(0.9391418526374932, 4.503) [b] 
(0.9391524510906719, 4.511) [b] 
(0.9391531386821216, 4.527) [b] 
(0.939158668309867, 4.675) [b] 
(0.9391905723425368, 4.783) [b] 
(0.9392224763752065, 4.784) [b] 
(0.9392231708196509, 5.088) [b] 
(0.9392299272111974, 5.097) [b] 
(0.9392333054069706, 5.101) [b] 
(0.9392501963858366, 5.103) [b] 
(0.9392535745816099, 5.104) [b] 
(0.9392840772658461, 5.44) [b] 
(0.9393081899201671, 6.007) [b] 
(0.9393146060992041, 6.065) [b] 
(0.939338093509952, 6.144) [b] 
(0.9393422601766187, 6.16) [b] 
(0.9393486763556558, 6.167) [b] 
(0.9393497180223225, 6.17) [b] 
(0.9393511004292588, 6.185) [b] 
(0.9393530637005498, 6.19) [b] 
(0.9394148112103742, 6.216) [b] 
(0.9394235664543045, 6.241) [b] 
(0.9394267920704893, 6.245) [b] 
(0.9394277136751136, 6.249) [b] 
(0.9394286352797379, 6.265) [b] 
(0.9394521226904857, 6.273) [b] 
(0.9394544267020464, 6.284) [b] 
(0.9394604171321039, 6.298) [b] 
(0.9394613387367282, 6.301) [b] 
(0.939464564352913, 6.306) [b] 
(0.9394666476862463, 6.307) [b] 
(0.9394671084885584, 6.322) [b] 
(0.9395062198777949, 6.453) [b] 
(0.9395303325321159, 6.463) [b] 
(0.9395306797543381, 6.798) [b] 
(0.9395931797543381, 6.986) [b] 
(0.9396348464210048, 6.987) [b] 
(0.9396469027481653, 7.055) [b] 
(0.9396496805259431, 7.28) [b] 
(0.9397479509661947, 7.412) [b] 
(0.9397619896005164, 7.421) [b] 
(0.9398828793490657, 7.441) [b] 
(0.9399249952520307, 7.463) [b] 
(0.9399390338863524, 7.464) [b] 
(0.9399418116641302, 7.556) [b] 
(0.9400627014126794, 7.659) [b] 
(0.9400767400470011, 7.987) [b] 
(0.9400887963741617, 7.996) [b] 
(0.9401129090284827, 7.997) [b] 
(0.9401249653556432, 8.064) [b] 
(0.9401530426242865, 8.317) [b] 
(0.940165098951447, 8.876) [b] 
(0.9401692656181136, 9.902) [b] 
(0.9401827784012065, 10.656) [b] 
(0.9401861565969797, 10.66) [b] 
(0.9401895347927529, 10.89) [b] 
(0.9401929129885261, 10.92) [b] 
(0.9401949963218594, 11.196) [b] 
(0.9401977466876581, 11.688) [b] 
(0.9402011846449064, 11.689) [b] 
(0.9402032474192554, 11.697) [b] 
(0.9402039350107051, 11.933) [b] 
(0.9402053101936044, 12.323) [b] 
(0.9402066853765038, 12.367) [b] 
(0.9402073729679534, 12.851) [b] 
(0.9402080605594031, 12.858) [b] 
(0.9402385632436393, 13.118) [b] 
(0.9402690659278755, 13.119) [b] 
(0.9403009699605452, 13.306) [b] 
(0.9403023588494341, 15.293) [b] 
(0.9403065255161008, 15.332) [b] 
(0.9403685294843548, 15.772) [b] 
(0.9404305334526087, 15.773) [b] 
(0.9404312210440584, 15.779) [b] 
(0.9404319086355081, 15.785) [b] 
(0.9404683669688414, 16.187) [b] 
(0.9404777419688414, 16.397) [b] 
(0.9404841581478784, 20.356) [b] 
(0.9404906767127512, 21.433) [b] 
(0.9405047153470729, 21.913) [b] 
(0.9405187539813946, 21.937) [b] 
(0.9405211845369502, 22.014) [b] 
(0.9405263928702835, 22.246) [b] 
(0.9405404315046052, 23.396) [b] 
(0.9405544701389269, 23.431) [b] 
(0.9405572479167047, 24.194) [b] 
(0.9405617618055936, 24.241) [b] 
(0.9405631506944825, 24.35) [b] 
(0.9405839840278158, 25.706) [b] 
(0.9407010252263177, 26.033) [b] 
(0.9407044034220909, 26.432) [b] 
(0.9407077816178642, 26.45) [b] 
(0.9407111598136374, 26.962) [b] 
(0.9407514563964872, 28.922) [b] 
(0.9407535397298205, 36.821) [b] 
(0.940765596056981, 40.648) [b] 
(0.9407701099458698, 41.239) [b] 
(0.940770457168092, 41.766) [b] 
(0.9407939445788399, 43.986) [b] 
(0.9408210713149511, 46.346) [b] 
(0.9408332240927288, 47.514) [b] 
(0.9408544046482844, 47.568) [b] 
(0.9408564879816177, 47.794) [b] 
(0.9408571824260621, 47.926) [b] 
(0.940858571314951, 48.195) [b] 
(0.9409160006430954, 48.802) [b] 
(0.9409193788388687, 48.804) [b] 
(0.9409227570346419, 48.806) [b] 
(0.9409328916219615, 48.809) [b] 
(0.9409362698177347, 48.873) [b] 
(0.9409531607966007, 48.917) [b] 
(0.9409768081670131, 48.918) [b] 
(0.9409801863627864, 48.934) [b] 
(0.9409835645585596, 48.948) [b] 
(0.9409869427543328, 48.949) [b] 
(0.9410188467870025, 50.366) [b] 
(0.9410507508196723, 50.368) [b] 
(0.9411783669503512, 50.369) [b] 
(0.9412102709830209, 50.37) [b] 
(0.9412421750156906, 50.377) [b] 
(0.9412740790483604, 50.397) [b] 
(0.9413059830810301, 50.399) [b] 
(0.9413378871136998, 50.401) [b] 
(0.9413697911463695, 50.402) [b] 
(0.9413906244797029, 51.005) [b] 
(0.9414309210625527, 52.099) [b] 
(0.9414375182847748, 52.554) [b] 
(0.941437865506997, 52.714) [b] 
(0.9414396016181081, 52.805) [b] 
(0.9414399488403303, 52.981) [b] 
(0.9414634362510782, 53.598) [b] 
(0.9414735708383979, 55.166) [b] 
(0.9414769490341711, 55.219) [b] 
(0.9415088530668408, 56.494) [b] 
(0.9415726611321803, 56.495) [b] 
(0.94160456516485, 56.497) [b] 
(0.9417002772628592, 56.498) [b] 
(0.9417321812955289, 56.509) [b] 
(0.9418278933935381, 56.51) [b] 
(0.9418597974262078, 56.575) [b] 
(0.9418917014588776, 56.61) [b] 
(0.9419236054915473, 56.613) [b] 
(0.9419356618187078, 57.491) [b] 
(0.9419477181458683, 57.738) [b] 
(0.9419597744730288, 58.519) [b] 
(0.9419604689174732, 61.15) [b] 
(0.9419611633619176, 61.411) [b] 
(0.941961857806362, 62.987) [b] 
(0.9419694966952509, 63.039) [b] 
(0.9419760939174731, 63.089) [b] 
(0.9419781772508063, 63.209) [b] 
(0.941992215885128, 64.983) [b] 
(0.9419936810299195, 70.772) [b] 
(0.9419940282521417, 71.652) [b] 
(0.9419957864258915, 80.157) [b] 
(0.9420166197592249, 82.924) [b] 
(0.9420374530925583, 83.039) [b] 
(0.9420582864258916, 83.045) [b] 
(0.9420723250602133, 83.425) [b] 
(0.9420931583935467, 83.569) [b] 
(0.94211399172688, 83.903) [b] 
(0.9421458957595498, 84.804) [b] 
(0.9421789645426185, 88.786) [b] 
(0.9421879032314642, 88.975) [b] 
(0.9421968419203098, 88.977) [b] 
(0.9421982171032092, 88.988) [b] 
(0.9422009674690078, 88.991) [b] 
(0.9422016550604575, 89.056) [b] 
(0.9422023426519072, 89.061) [b] 
(0.9422037178348065, 89.158) [b] 
(0.9422057806091555, 89.161) [b] 
(0.9422085309749542, 89.509) [b] 
(0.9427096128562888, 90.01) [b] 
(0.9427103004477385, 91.512) [b] 
(0.9427109880391882, 91.513) [b] 
(0.9427123632220875, 91.515) [b] 
(0.9428735495534866, 91.706) [b] 
(0.9428970369642344, 95.225) [b] 
(0.9428977245556841, 96.084) [b] 
(0.9428984121471338, 96.097) [b] 
(0.9428990997385834, 101.53) [b] 
(0.9428997873300331, 101.531) [b] 
(0.9429004749214828, 101.534) [b] 
(0.942900822143705, 105.772) [b] 
(0.9429015165881494, 106.104) [b] 
(0.9429018638103716, 114.574) [b] 
(0.9429029054770383, 123.195) [b] 
(0.942907072143705, 123.2) [b] 
(0.9429081138103717, 123.276) [b] 
(0.9429091554770384, 123.515) [b] 
(0.9429112388103716, 128.027) [b] 
(0.9429122804770383, 129.095) [b] 
(0.942912968068488, 132.58) [b] 
(0.942974972036742, 136.263) [b] 
(0.9430369760049959, 136.264) [b] 
(0.9430989799732499, 136.266) [b] 
(0.9431609839415038, 136.28) [b] 
(0.9432229879097578, 136.332) [b] 
(0.9432849918780117, 136.348) [b] 
(0.9433469958462657, 136.36) [b] 
(0.9434089998145196, 137.26) [b] 
(0.9434492963973694, 140.129) [b] 
(0.9435113003656234, 144.783) [b] 
(0.9435226885901992, 147.508) [b] 
(0.943530327479088, 153.108) [b] 
(0.9435320635901991, 153.433) [b] 
(0.9435355358124213, 154.015) [b] 
(0.9435362302568657, 157.853) [b] 
(0.943538313590199, 157.956) [b] 
(0.9435390080346434, 159.138) [b] 
(0.9435393552568656, 161.588) [b] 
(0.9435533938911873, 170.19) [b] 
(0.9435662262492615, 170.274) [b] 
(0.9435943035179047, 171.291) [b] 
(0.9436177909286526, 188.257) [b] 
(0.9436181381508748, 192.789) [b] 
(0.943618485373097, 208.993) [b] 
(0.9436419727838449, 209.635) [b] 
(0.9437233529921782, 223.455) [b] 
(0.9437504797282894, 223.457) [b] 
(0.9438318599366228, 223.562) [b] 
(0.9438589866727339, 223.563) [b] 
(0.9438861134088451, 223.582) [b] 
(0.9439132401449563, 225.012) [b] 
(0.9439340734782896, 230.065) [b] 
(0.9439612002144008, 231.931) [b] 
(0.9440862002144008, 233.376) [b] 
(0.9441695335477341, 233.377) [b] 
(0.9441903668810675, 233.641) [b] 
(0.9442737002144008, 233.642) [b] 
(0.9443153668810675, 233.643) [b] 
(0.9443570335477343, 233.644) [b] 
(0.9443778668810676, 233.645) [b] 
(0.9443842830601047, 235.806) [b] 
(0.9443846302823269, 248.108) [b] 
(0.9443849775045491, 248.352) [b] 
(0.9443856719489935, 248.476) [b] 
(0.9443881025045491, 248.662) [b] 
(0.9443884497267713, 249.208) [b] 
(0.9443898386156602, 251.335) [b] 
(0.9443962547946972, 255.931) [b] 
(0.9444026709737343, 258.807) [b] 
(0.9444090871527714, 259.197) [b] 
(0.944507357593023, 268.311) [b] 
(0.9445354348616662, 268.312) [b] 
(0.944549473495988, 268.321) [b] 
(0.9445775507646312, 268.38) [b] 
(0.9445915893989529, 269.014) [b] 
(0.9446056280332746, 269.149) [b] 
(0.9446196666675963, 269.361) [b] 
(0.944633705301918, 269.381) [b] 
(0.9447179371078479, 271.294) [b] 
(0.9447460143764912, 271.295) [b] 
(0.9447600530108129, 271.309) [b] 
(0.9447881302794562, 271.335) [b] 
(0.9448021689137779, 271.337) [b] 
(0.9448162075480996, 271.4) [b] 
(0.9448675211769995, 276.56) [b] 
(0.9448931779914493, 276.566) [b] 
(0.9449188348058992, 276.594) [b] 
(0.9449444916203491, 276.68) [b] 
(0.944970148434799, 276.684) [b] 
(0.9449841870691207, 279.307) [b] 
(0.9450098438835706, 280.151) [b] 
(0.9450355006980204, 280.96) [b] 
(0.9450611575124703, 287.078) [b] 
(0.9450868143269202, 287.197) [b] 
(0.9451124711413701, 287.354) [b] 
(0.94516378477027, 297.278) [b] 
(0.9451894415847198, 297.289) [b] 
(0.9452150983991697, 297.316) [b] 
(0.9452407552136196, 297.41) [b] 
(0.9452664120280695, 297.414) [b] 
(0.9452920688425194, 301.136) [b] 
(0.9453177256569693, 302.017) [b] 
(0.9453412130677171, 303.621) [b] 
(0.9453415602899393, 308.188) [b] 
(0.9453672171043892, 308.712) [b] 
(0.9453928739188391, 308.8) [b] 
(0.945418530733289, 309.011) [b] 
(0.9454186772477681, 325.187) [b] 
(0.9454421646585159, 329.489) [b] 
(0.9454462902072139, 345.915) [b] 
(0.9454469777986636, 345.916) [b] 
(0.9454483529815629, 345.95) [b] 
(0.9454490405730126, 345.968) [b] 
(0.9454497281644623, 345.982) [b] 
(0.9454504157559119, 346.034) [b] 
(0.9454524785302609, 346.073) [b] 
(0.9454531661217106, 366.331) [b] 
(0.9454538537131603, 366.333) [b] 
(0.9454545481576047, 391.344) [b] 
(0.9455715893561065, 406.218) [b] 
(0.9455977126580919, 439.448) [b] 
(0.9456173051345809, 439.454) [b] 
(0.94563689761107, 439.462) [b] 
(0.945656490087559, 439.464) [b] 
(0.9456630209130554, 439.484) [b] 
(0.9456695517385518, 439.586) [b] 
(0.9456729299343251, 467.696) [b] 
(0.9456796863258714, 473.208) [b] 
(0.9456830645216446, 473.209) [b] 
(0.9456864427174179, 473.214) [b] 
(0.9456898209131911, 473.216) [b] 
(0.9456931991089643, 473.217) [b] 
(0.9456999555005108, 473.32) [b] 
(0.9457067118920572, 473.321) [b] 
(0.9457168464793768, 473.326) [b] 
(0.94572022467515, 473.328) [b] 
(0.9457236028709233, 473.369) [b] 
(0.9457269810666965, 474.895) [b] 
(0.9457371156540161, 474.929) [b] 
(0.9457454489873494, 481) [b] 
(0.9457464906540161, 481.052) [b] 
(0.945747879542905, 481.366) [b] 
(0.9457506573206828, 481.481) [b] 
(0.945751004542905, 481.964) [b] 
(0.9457543827386782, 488.185) [b] 
(0.9457577609344514, 488.232) [b] 
(0.9457611391302246, 488.566) [b] 
(0.9457653057968913, 534.83) [b] 
(0.9457698196857802, 534.894) [b] 
(0.9457712085746691, 535.189) [b] 
(0.9457715557968913, 536.919) [b] 
(0.9457722502413357, 537.774) [b] 
(0.9457725974635579, 540.352) [b] 
(0.9457791160284307, 541.95) [b] 
(0.9457798104728751, 544.813) [b] 
(0.9457863290377478, 551.632) [b] 
(0.94578667625997, 557.962) [b] 
(0.9458148012599701, 563.134) [b] 
(0.9458172318155257, 563.184) [b] 
(0.9458936207044145, 569.712) [b] 
(0.9459498707044145, 569.81) [b] 
(0.9459547318155256, 570.463) [b] 
(0.9459568151488589, 571.983) [b] 
(0.9459578568155256, 576.529) [b] 
(0.9459595929266367, 583.827) [b] 
(0.9459602873710811, 583.872) [b] 
(0.9459606345933033, 584.057) [b] 
(0.9459675790377478, 589.06) [b] 
(0.9459947057738589, 590.988) [b] 
(0.9459981779960811, 591.814) [b] 
(0.9460092891071922, 595.648) [b] 
(0.9460157052862292, 612.401) [b] 
(0.9460484222503899, 747.695) [b] 
(0.9460491098418395, 836.01) [b] 
(0.9460504850247389, 836.011) [b] 
(0.9460518602076382, 836.015) [b] 
(0.9460725963116837, 843.896) [b] 
(0.9460730571139958, 876.358) [b] 
(0.9460870957483175, 909.535) [b] 
(0.9461151730169608, 909.566) [b] 
(0.9461292116512825, 909.667) [b] 
(0.9461432502856042, 909.676) [b] 
(0.9461572889199259, 909.793) [b] 
(0.9461713275542476, 909.986) [b] 
(0.9461853661885693, 911.024) [b] 
(0.9461855127030484, 1002.53) [b] 
(0.9463025539015503, 1003.46) [b] 
(0.9463260413122981, 1013.72) [b] 
(0.946349528723046, 1015.55) [b] 
(0.9463635673573677, 1038.76) [b] 
(0.9463640281596798, 1040.87) [b] 
(0.946364375381902, 1077.42) [b] 
(0.9463914587152353, 1113.67) [b] 
(0.9463963198263464, 1113.73) [b] 
(0.9463966670485686, 1113.91) [b] 
(0.9463977087152353, 1116.42) [b] 
(0.946401875381902, 1116.49) [b] 
(0.9464067364930131, 1116.57) [b] 
(0.9464074309374575, 1118.01) [b] 
(0.9464077781596797, 1119.71) [b] 
(0.9464141943387168, 1132.93) [b] 
(0.9464206105177538, 1132.95) [b] 
(0.9464207570322329, 1451.91) [b] 
(0.9465446097648154, 1588.32) [b] 
(0.9465511283296881, 1588.42) [b] 
(0.9465576468945609, 1588.92) [b] 
(0.946583721154052, 1589.15) [b] 
(0.9465854572651631, 1594.1) [b] 
(0.9465858044873853, 1602.54) [b] 
(0.946592323052258, 1615.85) [b] 
(0.9466314344414946, 1617.32) [b] 
(0.9466379530063673, 1617.34) [b] 
(0.9466444715712401, 1617.73) [b] 
(0.9466493326823512, 1645.24) [b] 
(0.9466701660156845, 1655.17) [b] 
(0.9466897584921736, 1670.16) [b] 
(0.94669628931767, 1670.18) [b] 
(0.9467028201431664, 1670.34) [b] 
(0.9467035145876108, 1685.84) [b] 
(0.9467243479209442, 1695.04) [b] 
(0.9467451812542775, 1695.12) [b] 
(0.9470576812542776, 1695.13) [b] 
(0.9470993479209443, 1695.14) [b] 
(0.9471201812542777, 1695.18) [b] 
(0.9473701812542776, 1695.28) [b] 
(0.947391014587611, 1695.29) [b] 
(0.9474118479209443, 1696.82) [b] 
(0.9474326812542777, 1697.44) [b] 
(0.9474330284764999, 1697.68) [b] 
(0.947434764587611, 1700.02) [b] 
(0.9474351118098332, 1700.85) [b] 
(0.9476017784765, 1705.87) [b] 
(0.9476226118098333, 1705.88) [b] 
(0.9476434451431667, 1705.89) [b] 
(0.9476642784765, 1705.95) [b] 
(0.9476851118098334, 1705.97) [b] 
(0.9477059451431668, 1705.98) [b] 
(0.9477267784765001, 1706.24) [b] 
(0.9477476118098335, 1706.25) [b] 
(0.9477710992205813, 1718.48) [b] 
(0.9477945866313292, 1721.47) [b] 
(0.9478362532979959, 1762.33) [b] 
(0.9478570866313293, 1762.34) [b] 
(0.9478779199646626, 1762.35) [b] 
(0.947898753297996, 1762.78) [b] 
(0.9479195866313294, 1762.8) [b] 
(0.9479404199646627, 1762.84) [b] 
(0.9479612532979961, 1769.73) [b] 
(0.9481240137146628, 1816.17) [b] 
(0.948151140450774, 1816.18) [b] 
(0.9481782671868851, 1816.19) [b] 
(0.9482053939229963, 1816.2) [b] 
(0.9482325206591075, 1816.39) [b] 
(0.9482596473952186, 1816.45) [b] 
(0.9483139008674409, 1816.66) [b] 
(0.948802182117441, 1816.67) [b] 
(0.9488293088535522, 1816.68) [b] 
(0.94901919600633, 1816.77) [b] 
(0.9490734494785523, 1817.12) [b] 
(0.9491005762146635, 1817.14) [b] 
(0.9491548296868857, 1818.08) [b] 
(0.9492362098952191, 1822.77) [b] 
(0.9492570432285524, 1838.27) [b] 
(0.9492827000430023, 1851.1) [b] 
(0.9493369535152246, 1854.49) [b] 
(0.949418333723558, 1880.84) [b] 
(0.9494454604596692, 1880.85) [b] 
(0.9494465021263359, 1907.94) [b] 
(0.949473628862447, 1912.07) [b] 
(0.9494826566402248, 1914.6) [b] 
(0.949504878862447, 1914.69) [b] 
(0.949508003862447, 1916.73) [b] 
(0.9495086983068914, 1917.11) [b] 
(0.9495093927513358, 1917.93) [b] 
(0.9495107816402247, 1925.22) [b] 
(0.9495121705291136, 1931.32) [b] 
(0.9495125177513358, 1961.89) [b] 
(0.949512864973558, 1961.95) [b] 
(0.9495135594180024, 1963.43) [b] 
(0.9495146010846691, 1964.13) [b] 
(0.949540257899119, 1969.68) [b] 
(0.9495673846352302, 2021.73) [b] 
(0.9496216381074525, 2029.06) [b] 
(0.9496487648435636, 2029.07) [b] 
(0.9497572717880081, 2029.13) [b] 
(0.9497843985241192, 2029.38) [b] 
(0.9497984371584409, 2073.72) [b] 
(0.9498124757927626, 2083.79) [b] 
(0.9498265144270843, 2085.1) [b] 
(0.9498278896099837, 2198.96) [b] 
(0.9498285772014333, 2198.99) [b] 
(0.949829264792883, 2199.16) [b] 
(0.9498433034272047, 2259.49) [b] 
(0.9498689602416546, 2315.09) [b] 
(0.9498946170561045, 2315.11) [b] 
(0.9499972443139041, 2315.14) [b] 
(0.950022901128354, 2316.16) [b] 
(0.9500485579428039, 2316.17) [b] 
(0.9500742147572537, 2327.52) [b] 
(0.9500745619794759, 2397.74) [b] 
(0.950076298090587, 2397.79) [b] 
(0.9500766453128092, 2397.89) [b] 
(0.9500769925350314, 2398.44) [b] 
(0.9501026493494813, 2454.41) [b] 
(0.9501283061639312, 2454.42) [b] 
(0.9502309334217308, 2454.45) [b] 
(0.9502565902361807, 2455.51) [b] 
(0.9502822470506306, 2455.52) [b] 
(0.9502884970506306, 2467.43) [b] 
(0.9502919692728528, 2467.44) [b] 
(0.950294052606186, 2467.47) [b] 
(0.9503197094206359, 2467.51) [b] 
(0.9503210983095248, 2467.79) [b] 
(0.9503349871984137, 2469.39) [b] 
(0.9503391538650803, 2469.4) [b] 
(0.950343320531747, 2469.42) [b] 
(0.9503440149761914, 2469.64) [b] 
(0.95034676534199, 2481.38) [b] 
(0.9503481405248894, 2483.24) [b] 
(0.9503495157077887, 2483.27) [b] 
(0.9503502032992384, 2483.28) [b] 
(0.9503736907099862, 2487.18) [b] 
(0.9503743851544306, 2489.3) [b] 
(0.9503978725651785, 2490.95) [b] 
(0.9503985601566282, 2496.79) [b] 
(0.9504006229309772, 2498.6) [b] 
(0.9504013105224268, 2498.72) [b] 
(0.9504019981138765, 2499.17) [b] 
(0.9504026857053262, 2506.06) [b] 
(0.9504030329275484, 2518.82) [b] 
(0.9504037273719927, 2521.67) [b] 
(0.9504870607053261, 2609.93) [b] 
(0.9504904389010993, 2803.97) [b] 
(0.950491126492549, 2840.57) [b] 
(0.9504938768583476, 2840.58) [b] 
(0.9504959396326966, 2840.63) [b] 
(0.9505026960242431, 2847.34) [b] 
(0.9505298227603542, 2894.72) [b] 
(0.9505569494964654, 2894.74) [b] 
(0.9508282168575766, 2895.23) [b] 
(0.9508553435936877, 2895.25) [b] 
(0.9509367238020211, 2895.35) [b] 
(0.9510452307464656, 2895.36) [b] 
(0.9510473140797989, 2908.65) [b] 
(0.95105217519091, 2908.7) [b] 
(0.9511355085242434, 2979.34) [b] 
(0.9511980085242435, 2979.35) [b] 
},{(0.9403091041666664, 0.001) [c] 
(0.9403091041666664, 3.510360354166668) [c] 
(0.9403091041666664, 3600) [c] 
}}}{legend pos=north west}}
% 	\subfloat[depth=9]{\cactus{Average Accuracy}{CPU time}{\budalg, \murtree, \cart}{{{(0.9200019225068743, 0) [a] 
(0.9266353892671741, 0.01) [a] 
(0.9317151809338406, 0.02) [a] 
(0.934541014267174, 0.03) [a] 
(0.9364792781560626, 0.04) [a] 
(0.9438495559338403, 0.05) [a] 
(0.9438605976005068, 0.06) [a] 
(0.9442853892671736, 0.07) [a] 
(0.9445437226005069, 0.08) [a] 
(0.9445676809338402, 0.09) [a] 
(0.9447670559338399, 0.1) [a] 
(0.9450195559338391, 0.11) [a] 
(0.9452937226005056, 0.12) [a] 
(0.9453672642671722, 0.13) [a] 
(0.9454118476005053, 0.14) [a] 
(0.9455551809338384, 0.15) [a] 
(0.9455891392671716, 0.16) [a] 
(0.9456141392671715, 0.17) [a] 
(0.945647055933838, 0.18) [a] 
(0.9456660142671711, 0.19) [a] 
(0.945669139267171, 0.2) [a] 
(0.9456780976005041, 0.21) [a] 
(0.9456808059338374, 0.22) [a] 
(0.945683097600504, 0.23) [a] 
(0.9456874726005038, 0.24) [a] 
(0.9457366392671704, 0.25) [a] 
(0.9457391392671705, 0.26) [a] 
(0.9458228892671705, 0.28) [a] 
(0.9494979995194198, 0.29) [a] 
(0.9494988328527532, 0.3) [a] 
(0.9495042495194198, 0.31) [a] 
(0.9496502911860865, 0.32) [a] 
(0.9497679995194198, 0.33) [a] 
(0.9499367495194199, 0.34) [a] 
(0.9499394578527532, 0.35) [a] 
(0.9499442495194199, 0.36) [a] 
(0.9499800828527533, 0.37) [a] 
(0.95008424951942, 0.4) [a] 
(0.9502015411860868, 0.45) [a] 
(0.9502019578527534, 0.46) [a] 
(0.9502290411860868, 0.47) [a] 
(0.9503748745194203, 0.48) [a] 
(0.9503752911860869, 0.49) [a] 
(0.9503769578527536, 0.5) [a] 
(0.9548036499045187, 0.51) [a] 
(0.9548040665711853, 0.52) [a] 
(0.9548042749045187, 0.53) [a] 
(0.9548317749045188, 0.55) [a] 
(0.9548526082378521, 0.58) [a] 
(0.9548840665711855, 0.59) [a] 
(0.9549378165711856, 0.6) [a] 
(0.9549919832378523, 0.61) [a] 
(0.9553119832378523, 0.62) [a] 
(0.9553940665711856, 0.63) [a] 
(0.9554176082378523, 0.65) [a] 
(0.9554240665711856, 0.66) [a] 
(0.9554513582378523, 0.67) [a] 
(0.9554932332378524, 0.68) [a] 
(0.9555140665711858, 0.71) [a] 
(0.9555411499045191, 0.72) [a] 
(0.9555417749045191, 0.76) [a] 
(0.9556055249045191, 0.79) [a] 
(0.9556290665711857, 0.9) [a] 
(0.955643024904519, 0.94) [a] 
(0.9556494832378524, 1.02) [a] 
(0.955663649904519, 1.03) [a] 
(0.9556651082378523, 1.06) [a] 
(0.9556665665711856, 1.07) [a] 
(0.9564808333333306, 1.09) [a] 
(0.9565041666666639, 1.18) [a] 
(0.9565277083333306, 1.23) [a] 
(0.9565512499999972, 1.24) [a] 
(0.9565793749999971, 1.26) [a] 
(0.9566027083333305, 1.27) [a] 
(0.9566047916666638, 1.35) [a] 
(0.9566095833333305, 1.37) [a] 
(0.9566122916666638, 1.39) [a] 
(0.9566402083333305, 1.46) [a] 
(0.9566916666666638, 1.49) [a] 
(0.9566922916666638, 1.5) [a] 
(0.9567437499999971, 1.51) [a] 
(0.9567472916666637, 1.59) [a] 
(0.9567508333333303, 1.6) [a] 
(0.956754999999997, 1.61) [a] 
(0.9567577083333303, 1.62) [a] 
(0.9567612499999969, 1.63) [a] 
(0.9567645833333301, 1.64) [a] 
(0.9567666666666634, 1.65) [a] 
(0.9567674999999968, 1.66) [a] 
(0.9567737499999968, 1.67) [a] 
(0.9568214583333301, 1.69) [a] 
(0.9568235416666634, 1.75) [a] 
(0.95685208333333, 1.76) [a] 
(0.9568541666666633, 1.77) [a] 
(0.9568881249999966, 1.78) [a] 
(0.9568914583333299, 1.79) [a] 
(0.9568970833333299, 1.8) [a] 
(0.9568979166666632, 1.82) [a] 
(0.9569120833333299, 1.85) [a] 
(0.9569391666666632, 1.98) [a] 
(0.9569395833333298, 1.99) [a] 
(0.9569399999999965, 2.02) [a] 
(0.9569487499999965, 2.05) [a] 
(0.9569552083333298, 2.07) [a] 
(0.9569566666666631, 2.2) [a] 
(0.9569572916666631, 2.24) [a] 
(0.9569579166666631, 2.25) [a] 
(0.9569593749999964, 2.32) [a] 
(0.9569658333333297, 2.51) [a] 
(0.956967916666663, 2.55) [a] 
(0.9569766666666629, 2.57) [a] 
(0.9569906249999962, 2.62) [a] 
(0.9570735416666629, 2.63) [a] 
(0.9571033333333295, 2.64) [a] 
(0.9571062499999962, 2.65) [a] 
(0.9571083333333295, 2.66) [a] 
(0.9571116666666628, 2.67) [a] 
(0.9571131249999961, 2.68) [a] 
(0.9571154166666628, 2.69) [a] 
(0.9571158333333294, 2.7) [a] 
(0.9571164583333294, 2.73) [a] 
(0.957116874999996, 2.74) [a] 
(0.9571189583333293, 2.75) [a] 
(0.957120624999996, 2.76) [a] 
(0.9571227083333292, 2.77) [a] 
(0.9571239583333292, 2.78) [a] 
(0.9571245833333292, 2.79) [a] 
(0.9571256249999959, 2.8) [a] 
(0.9571262499999958, 2.82) [a] 
(0.9571295833333292, 2.84) [a] 
(0.9571299999999958, 2.85) [a] 
(0.9571304166666624, 2.92) [a] 
(0.9571316666666624, 2.93) [a] 
(0.9571337499999957, 2.94) [a] 
(0.957135208333329, 2.95) [a] 
(0.9571366666666623, 2.98) [a] 
(0.9571372916666623, 3.01) [a] 
(0.9571381249999956, 3.18) [a] 
(0.957138333333329, 3.33) [a] 
(0.9571456249999957, 3.34) [a] 
(0.9571691666666623, 3.54) [a] 
(0.9571947916666623, 3.73) [a] 
(0.957195833333329, 3.77) [a] 
(0.9572279166666624, 3.78) [a] 
(0.9572379166666625, 3.8) [a] 
(0.9572520833333291, 4.03) [a] 
(0.9572527083333291, 4.14) [a] 
(0.9572943749999958, 4.19) [a] 
(0.9573568749999959, 4.25) [a] 
(0.9573777083333292, 4.26) [a] 
(0.9573781249999959, 4.48) [a] 
(0.9573989583333292, 4.53) [a] 
(0.9573991666666626, 4.55) [a] 
(0.957399374999996, 4.6) [a] 
(0.9574133333333292, 4.77) [a] 
(0.9574174999999958, 4.79) [a] 
(0.9574189583333291, 4.8) [a] 
(0.9574191666666625, 4.98) [a] 
(0.9574312499999958, 5.03) [a] 
(0.9574322916666625, 5.29) [a] 
(0.9574531249999959, 5.34) [a] 
(0.9574770833333293, 5.43) [a] 
(0.9574897916666626, 5.48) [a] 
(0.957489999999996, 5.5) [a] 
(0.957515624999996, 5.53) [a] 
(0.957541249999996, 5.64) [a] 
(0.957541874999996, 6) [a] 
(0.9575433333333293, 6.01) [a] 
(0.9575572916666626, 6.08) [a] 
(0.9575714583333292, 6.09) [a] 
(0.9575718749999959, 6.26) [a] 
(0.9575722916666625, 6.32) [a] 
(0.9575724999999958, 6.33) [a] 
(0.9575729166666624, 6.34) [a] 
(0.9575735416666624, 6.54) [a] 
(0.957573958333329, 6.55) [a] 
(0.9575868749999957, 6.77) [a] 
(0.9575872916666623, 6.78) [a] 
(0.9576012499999956, 6.8) [a] 
(0.9576016666666622, 6.81) [a] 
(0.9576020833333289, 6.91) [a] 
(0.9576033333333288, 6.95) [a] 
(0.9576241666666622, 7.19) [a] 
(0.9576247916666621, 7.2) [a] 
(0.9576256249999955, 7.21) [a] 
(0.9576262499999955, 7.22) [a] 
(0.9576402083333287, 7.26) [a] 
(0.9576437499999954, 7.28) [a] 
(0.9579399999999952, 7.31) [a] 
(0.9579412499999952, 7.33) [a] 
(0.9579427083333285, 7.44) [a] 
(0.9579831249999952, 7.73) [a] 
(0.9579845833333285, 7.82) [a] 
(0.9579852083333285, 7.83) [a] 
(0.9579858333333284, 7.84) [a] 
(0.9579866666666618, 7.85) [a] 
(0.9579883333333284, 7.99) [a] 
(0.9579887499999951, 8) [a] 
(0.9579891666666617, 8.03) [a] 
(0.957989374999995, 8.13) [a] 
(0.957989999999995, 8.55) [a] 
(0.9579914583333283, 9.06) [a] 
(0.957991874999995, 9.41) [a] 
(0.9579922916666617, 9.46) [a] 
(0.9579927083333283, 9.86) [a] 
(0.958023124999995, 10.09) [a] 
(0.9580241666666617, 10.11) [a] 
(0.9580245833333283, 10.3) [a] 
(0.9580260416666616, 10.74) [a] 
(0.9580274999999949, 10.84) [a] 
(0.9580289583333282, 11.49) [a] 
(0.9580316666666616, 11.5) [a] 
(0.958052499999995, 11.7) [a] 
(0.9580733333333283, 11.71) [a] 
(0.9580735416666617, 11.76) [a] 
(0.958079999999995, 12.05) [a] 
(0.9580802083333284, 12.12) [a] 
(0.9580804166666618, 12.27) [a] 
(0.9580808333333284, 12.3) [a] 
(0.958081249999995, 12.33) [a] 
(0.9580816666666616, 12.37) [a] 
(0.9580820833333282, 12.38) [a] 
(0.9580822916666616, 12.4) [a] 
(0.9580827083333282, 12.43) [a] 
(0.9580831249999948, 12.44) [a] 
(0.9580833333333282, 12.46) [a] 
(0.9580837499999948, 12.48) [a] 
(0.9580841666666614, 12.49) [a] 
(0.9580872916666614, 12.5) [a] 
(0.958087708333328, 12.51) [a] 
(0.9580881249999946, 12.53) [a] 
(0.958088333333328, 12.54) [a] 
(0.9580887499999946, 12.55) [a] 
(0.9580891666666612, 12.6) [a] 
(0.9580895833333278, 14.34) [a] 
(0.9580897916666612, 14.39) [a] 
(0.9580899999999946, 14.41) [a] 
(0.9580908333333279, 14.44) [a] 
(0.9580914583333279, 14.79) [a] 
(0.9580918749999945, 14.8) [a] 
(0.9580924999999945, 14.82) [a] 
(0.9580929166666611, 14.86) [a] 
(0.9580939583333277, 14.88) [a] 
(0.9580943749999943, 14.89) [a] 
(0.9580945833333276, 14.9) [a] 
(0.9580956249999943, 14.91) [a] 
(0.9580964583333277, 15.08) [a] 
(0.958097916666661, 15.2) [a] 
(0.9580981249999944, 15.53) [a] 
(0.9580987499999943, 16.15) [a] 
(0.958099166666661, 16.71) [a] 
(0.9581006249999943, 17.27) [a] 
(0.9581020833333276, 17.35) [a] 
(0.9581022916666609, 17.79) [a] 
(0.9581093749999943, 17.99) [a] 
(0.9581095833333276, 18.29) [a] 
(0.958109791666661, 18.45) [a] 
(0.9581099999999944, 18.66) [a] 
(0.9581108333333277, 18.89) [a] 
(0.9581110416666611, 19.14) [a] 
(0.9581114583333277, 19.15) [a] 
(0.9581118749999943, 19.16) [a] 
(0.9581120833333276, 19.17) [a] 
(0.9581131249999943, 19.19) [a] 
(0.9581139583333277, 19.22) [a] 
(0.9581145833333277, 19.23) [a] 
(0.9581149999999943, 19.29) [a] 
(0.9581152083333276, 19.3) [a] 
(0.958116041666661, 19.33) [a] 
(0.9581170833333277, 19.37) [a] 
(0.958117291666661, 19.4) [a] 
(0.9581183333333277, 19.41) [a] 
(0.9581520833333279, 19.42) [a] 
(0.9581591666666612, 19.43) [a] 
(0.9581597916666612, 19.5) [a] 
(0.9581604166666612, 19.53) [a] 
(0.9581624999999946, 19.72) [a] 
(0.9581635416666613, 19.73) [a] 
(0.9581649999999946, 19.9) [a] 
(0.9581660416666613, 20.02) [a] 
(0.9581664583333279, 20.47) [a] 
(0.9581666666666613, 20.52) [a] 
(0.9581670833333279, 20.53) [a] 
(0.9581681249999946, 20.56) [a] 
(0.9581693749999945, 20.67) [a] 
(0.9581697916666612, 20.73) [a] 
(0.9581718749999945, 21.18) [a] 
(0.9581724999999944, 21.24) [a] 
(0.958172916666661, 21.25) [a] 
(0.9581733333333277, 21.27) [a] 
(0.9581739583333276, 21.37) [a] 
(0.9581756249999943, 22.61) [a] 
(0.9581760416666609, 22.62) [a] 
(0.9581781249999942, 22.65) [a] 
(0.9581795833333275, 23.15) [a] 
(0.9581802083333275, 23.16) [a] 
(0.9581810416666608, 23.18) [a] 
(0.9581843749999941, 23.19) [a] 
(0.9581849999999941, 23.23) [a] 
(0.9581864583333274, 23.9) [a] 
(0.9581879166666607, 24.19) [a] 
(0.9581885416666607, 24.2) [a] 
(0.9581889583333273, 24.23) [a] 
(0.9581893749999939, 24.25) [a] 
(0.9581908333333272, 24.3) [a] 
(0.9581910416666606, 24.32) [a] 
(0.9581914583333272, 24.33) [a] 
(0.9581920833333272, 24.34) [a] 
(0.9581929166666605, 24.35) [a] 
(0.9581931249999939, 24.39) [a] 
(0.9581941666666606, 24.44) [a] 
(0.9581945833333272, 24.47) [a] 
(0.9581949999999938, 24.49) [a] 
(0.9581952083333272, 24.5) [a] 
(0.9581956249999938, 24.51) [a] 
(0.9581960416666604, 24.58) [a] 
(0.9581962499999938, 24.77) [a] 
(0.9581966666666604, 24.92) [a] 
(0.9582174999999937, 24.94) [a] 
(0.958231458333327, 25.03) [a] 
(0.958232083333327, 25.55) [a] 
(0.9582322916666604, 25.69) [a] 
(0.958232708333327, 25.71) [a] 
(0.9582356249999936, 25.8) [a] 
(0.9582360416666602, 25.82) [a] 
(0.9582372916666602, 25.86) [a] 
(0.9582374999999935, 25.99) [a] 
(0.9582379166666601, 26.04) [a] 
(0.9582383333333268, 26.92) [a] 
(0.9582412499999934, 27.02) [a] 
(0.9582427083333267, 27.1) [a] 
(0.9582456249999933, 27.49) [a] 
(0.9582460416666599, 27.6) [a] 
(0.9582470833333265, 27.61) [a] 
(0.9582479166666598, 27.63) [a] 
(0.9582485416666597, 27.7) [a] 
(0.9582502083333263, 27.71) [a] 
(0.9582504166666597, 27.72) [a] 
(0.9582508333333263, 27.78) [a] 
(0.9582512499999929, 27.97) [a] 
(0.9583683333333263, 28.22) [a] 
(0.9583691666666596, 28.28) [a] 
(0.9583706249999929, 28.42) [a] 
(0.9583733333333262, 28.43) [a] 
(0.9583747916666595, 29.91) [a] 
(0.9583752083333261, 29.98) [a] 
(0.9583781249999928, 30.07) [a] 
(0.9583793749999927, 31.63) [a] 
(0.9584074999999926, 31.75) [a] 
(0.9584214583333259, 31.83) [a] 
(0.9584220833333259, 33.69) [a] 
(0.9584362499999926, 34.49) [a] 
(0.9584368749999925, 34.59) [a] 
(0.9584383333333258, 34.6) [a] 
(0.9584416666666592, 34.61) [a] 
(0.9584458333333258, 34.63) [a] 
(0.9584597916666591, 34.77) [a] 
(0.9584608333333258, 36.2) [a] 
(0.9584754166666591, 36.21) [a] 
(0.9584756249999925, 36.29) [a] 
(0.9584764583333258, 36.3) [a] 
(0.9584766666666592, 36.33) [a] 
(0.9584770833333258, 36.34) [a] 
(0.9584781249999925, 36.38) [a] 
(0.9584839583333259, 36.4) [a] 
(0.9584906249999926, 36.41) [a] 
(0.958490833333326, 37.16) [a] 
(0.9584910416666593, 37.17) [a] 
(0.9584918749999927, 37.18) [a] 
(0.958492083333326, 37.19) [a] 
(0.9585087499999927, 37.22) [a] 
(0.9585091666666593, 38.66) [a] 
(0.9585093749999927, 38.67) [a] 
(0.9585118749999927, 38.68) [a] 
(0.9585122916666593, 38.71) [a] 
(0.9585124999999927, 38.76) [a] 
(0.9585129166666593, 38.77) [a] 
(0.9585133333333259, 38.79) [a] 
(0.9585137499999925, 38.8) [a] 
(0.9585149999999925, 38.82) [a] 
(0.9585158333333258, 38.83) [a] 
(0.9585160416666592, 38.87) [a] 
(0.9585164583333258, 38.89) [a] 
(0.9585189583333258, 38.95) [a] 
(0.9585195833333258, 39.16) [a] 
(0.9585199999999924, 39.88) [a] 
(0.9585216666666591, 40.9) [a] 
(0.9585231249999924, 40.91) [a] 
(0.958523541666659, 41.08) [a] 
(0.9585252083333257, 41.09) [a] 
(0.9585262499999924, 41.1) [a] 
(0.9585270833333257, 41.13) [a] 
(0.9585281249999924, 42.9) [a] 
(0.9585291666666591, 42.93) [a] 
(0.9585324999999925, 42.94) [a] 
(0.9585335416666592, 42.96) [a] 
(0.9585339583333258, 43.06) [a] 
(0.9585374999999925, 43.16) [a] 
(0.9585408333333258, 43.17) [a] 
(0.9585429166666593, 43.62) [a] 
(0.958543958333326, 43.99) [a] 
(0.9585449999999927, 44) [a] 
(0.958546458333326, 44.03) [a] 
(0.9585466666666593, 44.26) [a] 
(0.9585470833333259, 45.78) [a] 
(0.9585477083333259, 46.2) [a] 
(0.9585487499999926, 46.3) [a] 
(0.9585629166666593, 46.44) [a] 
(0.9585768749999926, 46.57) [a] 
(0.9586049999999925, 47.04) [a] 
(0.9586470833333257, 47.06) [a] 
(0.9586504166666591, 47.28) [a] 
(0.9586508333333257, 47.6) [a] 
(0.9586743749999923, 48.63) [a] 
(0.9586979166666589, 48.71) [a] 
(0.9586983333333255, 48.76) [a] 
(0.9587122916666588, 48.77) [a] 
(0.9587591666666588, 48.8) [a] 
(0.9587827083333255, 48.83) [a] 
(0.9587833333333254, 51.42) [a] 
(0.9587974999999921, 52.28) [a] 
(0.9588114583333254, 52.29) [a] 
(0.958811874999992, 52.79) [a] 
(0.9588122916666586, 52.81) [a] 
(0.958812499999992, 52.82) [a] 
(0.9588129166666586, 52.83) [a] 
(0.9588143749999919, 52.97) [a] 
(0.9588154166666586, 53.03) [a] 
(0.9588158333333252, 53.07) [a] 
(0.9588172916666585, 53.19) [a] 
(0.9588210416666585, 53.2) [a] 
(0.9588216666666585, 53.24) [a] 
(0.9588220833333251, 53.25) [a] 
(0.9588231249999917, 53.26) [a] 
(0.9588235416666583, 53.27) [a] 
(0.9588239583333249, 53.29) [a] 
(0.9588241666666583, 53.3) [a] 
(0.9588245833333249, 53.33) [a] 
(0.9588249999999915, 53.46) [a] 
(0.9588252083333249, 53.47) [a] 
(0.9588254166666582, 53.53) [a] 
(0.9588258333333248, 53.56) [a] 
(0.9588262499999914, 53.85) [a] 
(0.9588266666666581, 54.36) [a] 
(0.9588268749999914, 54.68) [a] 
(0.9588277083333248, 54.69) [a] 
(0.9588279166666581, 54.75) [a] 
(0.9588281249999915, 55.04) [a] 
(0.9588285416666581, 55.16) [a] 
(0.9588289583333247, 55.23) [a] 
(0.958829791666658, 55.31) [a] 
(0.9588302083333247, 55.67) [a] 
(0.9588537499999913, 56.05) [a] 
(0.9588818749999912, 56.27) [a] 
(0.9589099999999912, 56.41) [a] 
(0.9589239583333244, 56.42) [a] 
(0.9589379166666577, 56.5) [a] 
(0.9589660416666577, 57.75) [a] 
(0.9589941666666576, 57.76) [a] 
(0.9589945833333242, 57.8) [a] 
(0.9589952083333242, 58.11) [a] 
(0.9589956249999908, 58.15) [a] 
(0.9589997916666575, 58.27) [a] 
(0.9590008333333242, 58.41) [a] 
(0.9590018749999909, 58.45) [a] 
(0.9590020833333243, 60.06) [a] 
(0.959003124999991, 60.15) [a] 
(0.9590033333333243, 61.55) [a] 
(0.9590037499999909, 66.77) [a] 
(0.9590058333333242, 67.49) [a] 
(0.9590079166666576, 67.5) [a] 
(0.9590089583333243, 67.51) [a] 
(0.959009999999991, 67.65) [a] 
(0.9590110416666577, 68.5) [a] 
(0.9590116666666577, 74.52) [a] 
(0.959012499999991, 74.56) [a] 
(0.9590139583333244, 74.82) [a] 
(0.9590149999999911, 74.83) [a] 
(0.9590152083333244, 74.88) [a] 
(0.959015624999991, 75.57) [a] 
(0.959021874999991, 76.31) [a] 
(0.9590339583333244, 76.63) [a] 
(0.9590460416666577, 76.66) [a] 
(0.9591631249999911, 79.52) [a] 
(0.9591679166666578, 80.47) [a] 
(0.9591683333333244, 81.54) [a] 
(0.959168749999991, 83.15) [a] 
(0.9591689583333244, 83.17) [a] 
(0.9591708333333243, 83.25) [a] 
(0.9591712499999909, 83.26) [a] 
(0.9591716666666575, 83.27) [a] 
(0.9591718749999909, 83.28) [a] 
(0.9591743749999909, 83.31) [a] 
(0.9591749999999909, 83.35) [a] 
(0.9591758333333241, 83.41) [a] 
(0.9591760416666575, 83.46) [a] 
(0.9591762499999908, 83.49) [a] 
(0.9591766666666575, 83.53) [a] 
(0.9591770833333241, 83.56) [a] 
(0.9591772916666574, 83.75) [a] 
(0.959177708333324, 83.91) [a] 
(0.9591818749999907, 84) [a] 
(0.9591822916666574, 84.09) [a] 
(0.959182708333324, 84.1) [a] 
(0.9591829166666573, 84.13) [a] 
(0.9591847916666574, 84.14) [a] 
(0.9591849999999907, 84.28) [a] 
(0.9591852083333241, 84.82) [a] 
(0.9591866666666574, 88.74) [a] 
(0.959187083333324, 89.4) [a] 
(0.959188333333324, 90.07) [a] 
(0.9591891666666573, 91) [a] 
(0.9591897916666573, 91.3) [a] 
(0.959196458333324, 91.95) [a] 
(0.959197083333324, 92.41) [a] 
(0.959203333333324, 97.27) [a] 
(0.959203958333324, 97.49) [a] 
(0.9592081249999905, 98.13) [a] 
(0.9592083333333239, 99.32) [a] 
(0.9592087499999905, 99.33) [a] 
(0.9592093749999905, 99.34) [a] 
(0.9592097916666571, 99.37) [a] 
(0.9592102083333237, 99.38) [a] 
(0.9592112499999904, 99.44) [a] 
(0.9592114583333238, 99.49) [a] 
(0.9592139583333237, 99.78) [a] 
(0.9592145833333237, 99.79) [a] 
(0.9592162499999903, 99.81) [a] 
(0.9592170833333237, 100.42) [a] 
(0.9592174999999903, 100.88) [a] 
(0.9592179166666569, 100.91) [a] 
(0.9592181249999903, 101.77) [a] 
(0.9592206249999903, 104.51) [a] 
(0.9592210416666569, 104.56) [a] 
(0.9592212499999903, 104.64) [a] 
(0.9592220833333236, 104.65) [a] 
(0.9592239583333236, 104.66) [a] 
(0.959224166666657, 104.71) [a] 
(0.9592249999999903, 104.74) [a] 
(0.9592252083333237, 104.75) [a] 
(0.9592256249999903, 104.76) [a] 
(0.9592260416666569, 104.97) [a] 
(0.9592262499999903, 105.83) [a] 
(0.9592664583333236, 106.22) [a] 
(0.9592806249999902, 106.23) [a] 
(0.9592864583333236, 106.24) [a] 
(0.9592868749999902, 106.47) [a] 
(0.9593808333333235, 106.68) [a] 
(0.9594041666666568, 106.69) [a] 
(0.9594277083333235, 106.77) [a] 
(0.9594512499999901, 106.89) [a] 
(0.9594747916666567, 107.07) [a] 
(0.9595216666666567, 107.08) [a] 
(0.9595452083333234, 107.5) [a] 
(0.9595499999999899, 107.89) [a] 
(0.9595514583333232, 107.9) [a] 
(0.9595520833333232, 108.12) [a] 
(0.9595547916666565, 108.13) [a] 
(0.9595568749999898, 108.15) [a] 
(0.9595583333333231, 108.25) [a] 
(0.9595597916666564, 108.55) [a] 
(0.9595610416666563, 108.56) [a] 
(0.9595618749999897, 108.57) [a] 
(0.9595631249999896, 108.63) [a] 
(0.959563958333323, 108.64) [a] 
(0.9595656249999895, 108.65) [a] 
(0.9595662499999895, 108.66) [a] 
(0.9595666666666561, 108.67) [a] 
(0.9595670833333227, 108.73) [a] 
(0.9595672916666561, 108.74) [a] 
(0.9595677083333227, 108.78) [a] 
(0.9595683333333227, 108.84) [a] 
(0.959569166666656, 108.91) [a] 
(0.959569791666656, 108.97) [a] 
(0.9595702083333226, 109.12) [a] 
(0.959571041666656, 109.32) [a] 
(0.9595716666666559, 109.46) [a] 
(0.9595722916666559, 109.47) [a] 
(0.9595737499999892, 109.72) [a] 
(0.9595758333333225, 109.73) [a] 
(0.9595772916666558, 109.85) [a] 
(0.9595785416666558, 109.96) [a] 
(0.9595793749999891, 110.37) [a] 
(0.9595814583333224, 110.38) [a] 
(0.9595820833333224, 110.41) [a] 
(0.9595827083333224, 110.47) [a] 
(0.9595835416666557, 110.75) [a] 
(0.959585624999989, 110.93) [a] 
(0.9595883333333223, 110.96) [a] 
(0.9595889583333223, 111.49) [a] 
(0.9595897916666556, 111.68) [a] 
(0.959589999999989, 111.85) [a] 
(0.9595906249999889, 112.19) [a] 
(0.9595912499999889, 112.26) [a] 
(0.9595920833333222, 112.33) [a] 
(0.9595927083333222, 112.43) [a] 
(0.9595941666666555, 112.49) [a] 
(0.9595995833333222, 112.61) [a] 
(0.9596004166666555, 112.66) [a] 
(0.9596010416666555, 112.84) [a] 
(0.9596016666666555, 112.85) [a] 
(0.9596024999999888, 113.16) [a] 
(0.9596031249999888, 113.27) [a] 
(0.9596058333333221, 113.37) [a] 
(0.9596066666666554, 113.9) [a] 
(0.9596079166666553, 114.04) [a] 
(0.9596087499999887, 114.3) [a] 
(0.9596093749999887, 114.52) [a] 
(0.9596097916666553, 116.31) [a] 
(0.9596104166666553, 121.01) [a] 
(0.9596133333333219, 121.07) [a] 
(0.9596139583333219, 122.57) [a] 
(0.9596145833333218, 122.6) [a] 
(0.9596154166666552, 122.64) [a] 
(0.9596160416666552, 122.72) [a] 
(0.9596166666666551, 123.46) [a] 
(0.9596174999999885, 123.51) [a] 
(0.9596181249999884, 123.52) [a] 
(0.9596187499999884, 124.62) [a] 
(0.9596195833333218, 125.19) [a] 
(0.9596208333333217, 125.53) [a] 
(0.959621666666655, 125.54) [a] 
(0.9596237499999883, 126.37) [a] 
(0.9596249999999883, 126.38) [a] 
(0.9596258333333216, 126.41) [a] 
(0.9596264583333216, 126.86) [a] 
(0.9596270833333216, 126.98) [a] 
(0.9596279166666549, 127.03) [a] 
(0.9596299999999882, 127.09) [a] 
(0.9596364583333216, 128.66) [a] 
(0.9596370833333215, 129.58) [a] 
(0.9596377083333215, 130.7) [a] 
(0.9596385416666549, 135) [a] 
(0.9596389583333215, 145.56) [a] 
(0.9596393749999881, 145.67) [a] 
(0.9596395833333214, 145.7) [a] 
(0.9596404166666548, 145.71) [a] 
(0.9596435416666548, 145.79) [a] 
(0.9596437499999881, 145.84) [a] 
(0.9596441666666548, 145.85) [a] 
(0.9596445833333214, 145.87) [a] 
(0.9596447916666547, 145.9) [a] 
(0.9596452083333213, 145.91) [a] 
(0.959645624999988, 146.01) [a] 
(0.9596458333333213, 146.02) [a] 
(0.9596462499999879, 146.03) [a] 
(0.9596472916666546, 146.08) [a] 
(0.9596708333333213, 147.03) [a] 
(0.9596710416666546, 147.67) [a] 
(0.9596714583333212, 148.04) [a] 
(0.9596720833333212, 149.07) [a] 
(0.9596729166666546, 149.43) [a] 
(0.9596733333333212, 150.67) [a] 
(0.9596737499999878, 150.71) [a] 
(0.9596739583333211, 150.72) [a] 
(0.9596743749999878, 150.74) [a] 
(0.9596747916666544, 150.75) [a] 
(0.9596754166666543, 150.83) [a] 
(0.9596760416666543, 150.88) [a] 
(0.9596764583333209, 150.92) [a] 
(0.9596774999999876, 151) [a] 
(0.9596779166666543, 151.05) [a] 
(0.9596785416666542, 151.08) [a] 
(0.9596789583333208, 151.17) [a] 
(0.9596791666666542, 151.24) [a] 
(0.9596795833333208, 151.26) [a] 
(0.9596799999999874, 156.4) [a] 
(0.9596806249999874, 161.95) [a] 
(0.9596808333333208, 164.12) [a] 
(0.9596810416666541, 165.8) [a] 
(0.9596818749999875, 165.92) [a] 
(0.9596820833333208, 165.93) [a] 
(0.9596829166666542, 165.94) [a] 
(0.9596831249999875, 165.98) [a] 
(0.9596839583333208, 166) [a] 
(0.9596841666666541, 166.02) [a] 
(0.9596849999999875, 166.03) [a] 
(0.9596856249999874, 166.98) [a] 
(0.9596864583333208, 169.93) [a] 
(0.9596866666666541, 174.09) [a] 
(0.9596902083333209, 174.1) [a] 
(0.9596937499999876, 174.12) [a] 
(0.9596989583333209, 174.13) [a] 
(0.9596991666666542, 174.16) [a] 
(0.9596995833333208, 174.42) [a] 
(0.9596997916666542, 175.73) [a] 
(0.9597008333333209, 175.74) [a] 
(0.9597012499999875, 175.76) [a] 
(0.9597016666666541, 175.77) [a] 
(0.9597018749999875, 176.41) [a] 
(0.9597022916666541, 176.42) [a] 
(0.9597027083333207, 184.26) [a] 
(0.9597033333333207, 184.29) [a] 
(0.9597037499999873, 184.32) [a] 
(0.9597039583333207, 184.33) [a] 
(0.9597043749999873, 184.35) [a] 
(0.9597047916666539, 184.4) [a] 
(0.9597054166666539, 184.53) [a] 
(0.9597058333333205, 184.59) [a] 
(0.9597060416666539, 184.68) [a] 
(0.9597064583333205, 184.72) [a] 
(0.9597068749999871, 184.74) [a] 
(0.9597070833333204, 185.02) [a] 
(0.9597074999999871, 185.69) [a] 
(0.9597077083333204, 186.54) [a] 
(0.959708124999987, 186.95) [a] 
(0.9597089583333204, 188.33) [a] 
(0.959709374999987, 188.37) [a] 
(0.959709999999987, 188.38) [a] 
(0.9597108333333203, 188.39) [a] 
(0.9597110416666537, 188.42) [a] 
(0.9597114583333203, 188.57) [a] 
(0.9597127083333202, 188.6) [a] 
(0.9597131249999868, 188.61) [a] 
(0.9597137499999868, 188.62) [a] 
(0.9597141666666534, 188.63) [a] 
(0.9597143749999868, 188.88) [a] 
(0.9597147916666534, 192.71) [a] 
(0.95971520833332, 192.72) [a] 
(0.9597154166666534, 192.75) [a] 
(0.95971583333332, 192.76) [a] 
(0.9597162499999866, 194.22) [a] 
(0.9597168749999866, 200.74) [a] 
(0.95972020833332, 205.38) [a] 
(0.9597268749999867, 212.34) [a] 
(0.9597272916666533, 218.57) [a] 
(0.9597274999999866, 218.59) [a] 
(0.9597279166666532, 218.6) [a] 
(0.9597283333333199, 218.66) [a] 
(0.9597299999999865, 218.69) [a] 
(0.9597341666666532, 218.7) [a] 
(0.9597347916666532, 218.71) [a] 
(0.9597352083333198, 218.77) [a] 
(0.9597358333333198, 218.79) [a] 
(0.959736666666653, 218.8) [a] 
(0.9597368749999864, 218.86) [a] 
(0.959737291666653, 219.3) [a] 
(0.9597377083333196, 219.34) [a] 
(0.959737916666653, 220.55) [a] 
(0.9597383333333196, 220.88) [a] 
(0.9597387499999862, 221.59) [a] 
(0.9597389583333196, 221.63) [a] 
(0.9597397916666528, 222.26) [a] 
(0.9597404166666528, 222.3) [a] 
(0.9597408333333194, 222.31) [a] 
(0.9597420833333193, 222.32) [a] 
(0.9597439583333194, 222.33) [a] 
(0.9597441666666527, 222.52) [a] 
(0.9597449999999861, 222.56) [a] 
(0.9597452083333194, 222.71) [a] 
(0.9597456249999861, 225.9) [a] 
(0.9597460416666527, 239.16) [a] 
(0.9597477083333193, 239.17) [a] 
(0.959748124999986, 239.26) [a] 
(0.9597487499999859, 239.27) [a] 
(0.9597491666666526, 239.3) [a] 
(0.9597493749999859, 239.31) [a] 
(0.9597497916666525, 239.32) [a] 
(0.9597502083333191, 239.33) [a] 
(0.9597504166666525, 239.35) [a] 
(0.9597512499999858, 239.36) [a] 
(0.9597514583333192, 239.37) [a] 
(0.9597518749999858, 239.38) [a] 
(0.9597529166666525, 239.42) [a] 
(0.9597533333333191, 239.43) [a] 
(0.9597535416666525, 239.46) [a] 
(0.9597545833333192, 239.48) [a] 
(0.9597549999999858, 239.54) [a] 
(0.9597554166666524, 239.55) [a] 
(0.9597556249999858, 239.56) [a] 
(0.9597560416666524, 239.59) [a] 
(0.959756458333319, 239.86) [a] 
(0.9597566666666524, 240.36) [a] 
(0.959757083333319, 241.08) [a] 
(0.9597574999999856, 244.8) [a] 
(0.9597579166666522, 245.94) [a] 
(0.9597583333333188, 245.95) [a] 
(0.9597585416666522, 246.05) [a] 
(0.9597589583333188, 246.09) [a] 
(0.9597593749999854, 246.25) [a] 
(0.959759791666652, 246.71) [a] 
(0.9597608333333186, 246.74) [a] 
(0.9597618749999852, 247.98) [a] 
(0.9597620833333186, 248.12) [a] 
(0.9597629166666519, 249.97) [a] 
(0.9597662499999853, 258.57) [a] 
(0.9597666666666519, 258.66) [a] 
(0.9597672916666519, 259.83) [a] 
(0.9597679166666518, 259.89) [a] 
(0.9597685416666518, 260.03) [a] 
(0.9597693749999852, 260.55) [a] 
(0.9597697916666518, 260.78) [a] 
(0.9597729166666518, 260.79) [a] 
(0.9597733333333184, 260.8) [a] 
(0.9597739583333184, 260.81) [a] 
(0.959774374999985, 260.86) [a] 
(0.9597745833333183, 260.87) [a] 
(0.9597754166666516, 260.9) [a] 
(0.9597756249999849, 260.93) [a] 
(0.9597764583333183, 260.94) [a] 
(0.9597766666666516, 260.96) [a] 
(0.9597770833333182, 260.97) [a] 
(0.9597774999999849, 261.03) [a] 
(0.9597777083333182, 261.1) [a] 
(0.9597781249999848, 261.11) [a] 
(0.9597785416666514, 262.09) [a] 
(0.959778958333318, 262.73) [a] 
(0.9597791666666514, 262.75) [a] 
(0.9597797916666514, 264.34) [a] 
(0.9597806249999847, 264.42) [a] 
(0.9597812499999847, 264.53) [a] 
(0.9597816666666513, 264.54) [a] 
(0.9597881249999847, 264.55) [a] 
(0.959788333333318, 264.69) [a] 
(0.9597891666666514, 274.23) [a] 
(0.9597893749999847, 274.24) [a] 
(0.9597897916666513, 274.28) [a] 
(0.9597904166666513, 274.31) [a] 
(0.9597912499999847, 274.33) [a] 
(0.9597918749999846, 274.34) [a] 
(0.9597922916666513, 274.35) [a] 
(0.959793333333318, 274.37) [a] 
(0.9597935416666513, 276.72) [a] 
(0.9597939583333179, 278.08) [a] 
(0.9597943749999845, 283.18) [a] 
(0.9597945833333179, 283.19) [a] 
(0.9597949999999845, 283.21) [a] 
(0.9597954166666511, 283.27) [a] 
(0.9597956249999845, 283.73) [a] 
(0.9597960416666511, 283.96) [a] 
(0.9597964583333177, 288.65) [a] 
(0.9597968749999843, 288.66) [a] 
(0.9597974999999843, 288.68) [a] 
(0.9597979166666509, 290.51) [a] 
(0.9597989583333176, 290.88) [a] 
(0.9597999999999843, 291.12) [a] 
(0.9598256249999844, 297.88) [a] 
(0.9598512499999844, 298.08) [a] 
(0.9598768749999844, 298.92) [a] 
(0.9599024999999844, 299.12) [a] 
(0.9599283333333177, 299.2) [a] 
(0.959954166666651, 300.19) [a] 
(0.959954791666651, 300.51) [a] 
(0.959956666666651, 300.52) [a] 
(0.9599568749999844, 300.53) [a] 
(0.959957291666651, 300.54) [a] 
(0.9599577083333176, 300.57) [a] 
(0.959957916666651, 300.59) [a] 
(0.9599583333333176, 300.6) [a] 
(0.9599589583333176, 300.63) [a] 
(0.9599593749999842, 300.64) [a] 
(0.9599597916666508, 300.65) [a] 
(0.9599604166666508, 300.67) [a] 
(0.9599608333333174, 300.68) [a] 
(0.9599620833333173, 300.71) [a] 
(0.959962499999984, 300.72) [a] 
(0.9599641666666506, 300.86) [a] 
(0.959964999999984, 300.88) [a] 
(0.9599660416666507, 300.9) [a] 
(0.9599666666666506, 300.97) [a] 
(0.9599670833333173, 301.01) [a] 
(0.9599672916666506, 301.54) [a] 
(0.9599677083333172, 301.56) [a] 
(0.9599702083333171, 301.61) [a] 
(0.9599704166666505, 302.14) [a] 
(0.9599708333333171, 302.17) [a] 
(0.9599712499999837, 302.53) [a] 
(0.9599714583333171, 302.8) [a] 
(0.9600889583333171, 303.26) [a] 
(0.9600893749999837, 303.31) [a] 
(0.9600895833333171, 303.35) [a] 
(0.9600899999999837, 303.38) [a] 
(0.9600904166666503, 303.39) [a] 
(0.9600906249999837, 303.43) [a] 
(0.9600910416666503, 303.46) [a] 
(0.9600914583333169, 303.48) [a] 
(0.9600916666666502, 304.13) [a] 
(0.9600918749999836, 304.53) [a] 
(0.9600922916666502, 307.9) [a] 
(0.9600927083333168, 311.52) [a] 
(0.9600929166666502, 315.03) [a] 
(0.9602099999999836, 316.84) [a] 
(0.9602106249999836, 325.36) [a] 
(0.9602110416666502, 330.09) [a] 
(0.9602116666666501, 330.1) [a] 
(0.9602122916666501, 330.13) [a] 
(0.9602127083333167, 330.15) [a] 
(0.9602143749999834, 332.06) [a] 
(0.9602164583333167, 336.91) [a] 
(0.9602168749999833, 336.92) [a] 
(0.9602174999999833, 336.94) [a] 
(0.9602179166666499, 337.06) [a] 
(0.9602185416666499, 337.13) [a] 
(0.9602189583333165, 337.18) [a] 
(0.9602191666666499, 337.62) [a] 
(0.9602195833333165, 337.63) [a] 
(0.9602452083333165, 339.86) [a] 
(0.9602456249999831, 340.02) [a] 
(0.9602712499999831, 341.33) [a] 
(0.9602714583333165, 343.1) [a] 
(0.9602743749999831, 345.67) [a] 
(0.9602752083333165, 346.57) [a] 
(0.9602754166666498, 346.6) [a] 
(0.9602758333333165, 346.93) [a] 
(0.9602762499999831, 346.99) [a] 
(0.9602764583333164, 347) [a] 
(0.960276874999983, 352.94) [a] 
(0.960277499999983, 352.96) [a] 
(0.9602779166666496, 352.97) [a] 
(0.9602783333333162, 353.01) [a] 
(0.9602785416666496, 353.02) [a] 
(0.9602789583333162, 353.04) [a] 
(0.9602793749999828, 353.05) [a] 
(0.9602795833333162, 353.07) [a] 
(0.9602799999999828, 353.11) [a] 
(0.9602827083333162, 353.16) [a] 
(0.9602835416666494, 353.19) [a] 
(0.9602837499999828, 353.25) [a] 
(0.9602841666666494, 353.68) [a] 
(0.960284583333316, 353.72) [a] 
(0.9602856249999827, 353.86) [a] 
(0.9602858333333161, 354.28) [a] 
(0.9602862499999827, 356.31) [a] 
(0.9602868749999827, 356.32) [a] 
(0.9603149999999826, 356.46) [a] 
(0.9603164583333159, 358.41) [a] 
(0.9603174999999826, 366.01) [a] 
(0.9603181249999826, 374.92) [a] 
(0.9603185416666492, 375.03) [a] 
(0.9603187499999826, 375.09) [a] 
(0.9603195833333159, 375.1) [a] 
(0.9603197916666493, 375.18) [a] 
(0.9603206249999826, 375.21) [a] 
(0.960320833333316, 375.27) [a] 
(0.9603212499999826, 375.28) [a] 
(0.9603218749999826, 375.41) [a] 
(0.9603227083333159, 375.55) [a] 
(0.9603233333333159, 375.86) [a] 
(0.9603235416666492, 376.17) [a] 
(0.9603239583333159, 376.19) [a] 
(0.9603304166666492, 376.43) [a] 
(0.9603370833333159, 376.49) [a] 
(0.9603379166666492, 377.27) [a] 
(0.9603381249999826, 377.28) [a] 
(0.9603391666666493, 377.37) [a] 
(0.9603395833333159, 377.38) [a] 
(0.9603402083333159, 377.45) [a] 
(0.9603406249999825, 377.95) [a] 
(0.9603418749999825, 377.96) [a] 
(0.9603422916666491, 378.02) [a] 
(0.9603433333333157, 378.03) [a] 
(0.960344791666649, 380.91) [a] 
(0.9603468749999823, 380.96) [a] 
(0.9603474999999823, 380.97) [a] 
(0.9603479166666489, 381.19) [a] 
(0.9603483333333155, 381.38) [a] 
(0.9603487499999821, 382.19) [a] 
(0.9603493749999821, 382.25) [a] 
(0.9603508333333154, 387.75) [a] 
(0.9603529166666487, 389.64) [a] 
(0.9603541666666486, 395.17) [a] 
(0.960355624999982, 395.21) [a] 
(0.9603558333333153, 396.23) [a] 
(0.9603562499999819, 396.24) [a] 
(0.9603566666666485, 401.41) [a] 
(0.9603570833333152, 401.97) [a] 
(0.9603572916666485, 402) [a] 
(0.9603577083333151, 402.1) [a] 
(0.9603581249999817, 402.11) [a] 
(0.9603583333333151, 402.46) [a] 
(0.9603591666666484, 402.57) [a] 
(0.9603595833333151, 403.51) [a] 
(0.9603599999999817, 412.24) [a] 
(0.9603604166666483, 414.72) [a] 
(0.9603606249999816, 415.69) [a] 
(0.9603610416666483, 423.16) [a] 
(0.9603614583333149, 424.11) [a] 
(0.9603616666666482, 424.2) [a] 
(0.9603624999999816, 424.22) [a] 
(0.9603629166666482, 426.57) [a] 
(0.9603639583333149, 426.58) [a] 
(0.9603641666666483, 426.6) [a] 
(0.9603645833333149, 426.62) [a] 
(0.9603649999999815, 426.73) [a] 
(0.9603656249999815, 426.74) [a] 
(0.9603683333333147, 435.25) [a] 
(0.9603689583333147, 435.26) [a] 
(0.960369791666648, 435.27) [a] 
(0.960370416666648, 435.35) [a] 
(0.960371041666648, 435.8) [a] 
(0.9603724999999813, 435.81) [a] 
(0.9603729166666479, 439.85) [a] 
(0.9603756249999813, 439.86) [a] 
(0.9603762499999813, 439.91) [a] 
(0.9603764583333146, 441.88) [a] 
(0.9603768749999813, 444.55) [a] 
(0.960377916666648, 444.56) [a] 
(0.9603785416666479, 448.96) [a] 
(0.9603799999999812, 449.28) [a] 
(0.9603808333333146, 453.72) [a] 
(0.9603814583333146, 459.45) [a] 
(0.9603824999999813, 459.46) [a] 
(0.9603829166666479, 459.52) [a] 
(0.9603835416666479, 459.53) [a] 
(0.9603839583333145, 459.54) [a] 
(0.9603841666666478, 459.57) [a] 
(0.9603845833333144, 459.6) [a] 
(0.9603852083333144, 459.69) [a] 
(0.9603858333333144, 460.85) [a] 
(0.9603860416666478, 460.89) [a] 
(0.9603866666666477, 460.95) [a] 
(0.9603870833333144, 462.04) [a] 
(0.9603877083333143, 470.55) [a] 
(0.9603881249999809, 473.41) [a] 
(0.9603885416666476, 473.42) [a] 
(0.9603887499999809, 473.81) [a] 
(0.9603895833333143, 474.25) [a] 
(0.9603902083333142, 488.29) [a] 
(0.9603908333333142, 493.13) [a] 
(0.9603935416666475, 493.15) [a] 
(0.9603964583333141, 493.29) [a] 
(0.9603966666666475, 494.1) [a] 
(0.9603970833333141, 494.63) [a] 
(0.9603977083333141, 495.26) [a] 
(0.9603997916666474, 495.37) [a] 
(0.9604008333333139, 497.37) [a] 
(0.9604010416666473, 497.56) [a] 
(0.9604014583333139, 500.27) [a] 
(0.9604018749999805, 503.38) [a] 
(0.9604020833333139, 503.39) [a] 
(0.9604024999999805, 503.96) [a] 
(0.9605193749999805, 505.05) [a] 
(0.9605197916666471, 517.68) [a] 
(0.9605204166666471, 521.92) [a] 
(0.9605210416666471, 521.93) [a] 
(0.9605218749999804, 521.94) [a] 
(0.9605220833333138, 522.06) [a] 
(0.9605239583333138, 522.21) [a] 
(0.9605247916666472, 523.84) [a] 
(0.9605249999999805, 523.85) [a] 
(0.9605254166666471, 523.86) [a] 
(0.9605258333333138, 523.88) [a] 
(0.9605260416666471, 523.94) [a] 
(0.9605264583333137, 524.97) [a] 
(0.9605268749999804, 526.82) [a] 
(0.9605277083333137, 532.41) [a] 
(0.960527916666647, 532.42) [a] 
(0.9605281249999804, 543.02) [a] 
(0.9605287499999804, 548.69) [a] 
(0.960529166666647, 548.71) [a] 
(0.9605293749999804, 549.74) [a] 
(0.9605308333333137, 551.84) [a] 
(0.9605312499999803, 573.96) [a] 
(0.9605314583333137, 574.04) [a] 
(0.9605318749999803, 574.05) [a] 
(0.9605322916666469, 574.07) [a] 
(0.9605324999999802, 574.83) [a] 
(0.9605329166666469, 574.85) [a] 
(0.9605333333333135, 582.12) [a] 
(0.9605343749999802, 582.17) [a] 
(0.9605347916666468, 582.73) [a] 
(0.9605352083333134, 593.49) [a] 
(0.9605354166666468, 593.72) [a] 
(0.9605360416666467, 594.61) [a] 
(0.9605366666666467, 594.62) [a] 
(0.9605374999999801, 594.7) [a] 
(0.96053812499998, 594.71) [a] 
(0.9605389583333134, 598.64) [a] 
(0.9605391666666467, 598.68) [a] 
(0.9605393749999801, 599.22) [a] 
(0.9605397916666467, 611.68) [a] 
(0.9605402083333133, 614.34) [a] 
(0.9605412499999799, 614.35) [a] 
(0.9605433333333132, 614.36) [a] 
(0.9605441666666465, 614.42) [a] 
(0.9605443749999799, 614.45) [a] 
(0.9605447916666465, 614.59) [a] 
(0.9605458333333132, 615.22) [a] 
(0.9605464583333132, 615.23) [a] 
(0.9605472916666465, 615.28) [a] 
(0.9605474999999799, 615.35) [a] 
(0.9605479166666465, 615.36) [a] 
(0.9605483333333131, 615.58) [a] 
(0.9605487499999797, 615.59) [a] 
(0.9605489583333131, 615.62) [a] 
(0.9605491666666465, 617.38) [a] 
(0.9605495833333131, 621.31) [a] 
(0.9605499999999797, 623.47) [a] 
(0.9605504166666463, 635.02) [a] 
(0.9605506249999797, 635.03) [a] 
(0.9605510416666463, 635.07) [a] 
(0.9605514583333129, 635.08) [a] 
(0.9605516666666463, 635.09) [a] 
(0.9605520833333129, 635.34) [a] 
(0.9605524999999795, 645.32) [a] 
(0.9605533333333128, 654.53) [a] 
(0.9605539583333128, 654.88) [a] 
(0.9605554166666461, 654.9) [a] 
(0.9605566666666461, 655.08) [a] 
(0.9605577083333128, 655.32) [a] 
(0.9605579166666461, 655.35) [a] 
(0.9605583333333128, 656.26) [a] 
(0.9605585416666461, 658.98) [a] 
(0.9605589583333127, 658.99) [a] 
(0.9605593749999793, 659.36) [a] 
(0.9605595833333127, 662.18) [a] 
(0.9605599999999793, 662.19) [a] 
(0.9605608333333127, 668.86) [a] 
(0.9605612499999793, 669.67) [a] 
(0.9605614583333126, 669.96) [a] 
(0.9605624999999793, 673.34) [a] 
(0.960562916666646, 674.98) [a] 
(0.9605631249999793, 675.01) [a] 
(0.9605635416666459, 676.14) [a] 
(0.9605639583333125, 682.14) [a] 
(0.9605643749999792, 688.63) [a] 
(0.9605647916666458, 689.04) [a] 
(0.9605654166666457, 689.09) [a] 
(0.9605656249999791, 689.19) [a] 
(0.9605664583333123, 689.29) [a] 
(0.960566874999979, 689.94) [a] 
(0.9605670833333123, 690.07) [a] 
(0.9606295833333124, 690.26) [a] 
(0.9606504166666457, 690.3) [a] 
(0.9606508333333124, 694.23) [a] 
(0.9607133333333124, 698.04) [a] 
(0.9607139583333124, 710.52) [a] 
(0.9607141666666458, 719.41) [a] 
(0.9607145833333124, 719.44) [a] 
(0.960714999999979, 720.51) [a] 
(0.9607152083333124, 720.52) [a] 
(0.9607166666666457, 734.29) [a] 
(0.9607170833333123, 740.87) [a] 
(0.9607172916666457, 740.88) [a] 
(0.9607191666666457, 769.13) [a] 
(0.9607193749999791, 769.18) [a] 
(0.9607197916666457, 770.21) [a] 
(0.9607202083333123, 775.57) [a] 
(0.9607410416666456, 776.51) [a] 
(0.9607416666666456, 782.85) [a] 
(0.9607420833333122, 803.99) [a] 
(0.9607422916666456, 804.04) [a] 
(0.9607427083333122, 804.05) [a] 
(0.9607431249999788, 804.19) [a] 
(0.9607433333333122, 806.63) [a] 
(0.9607437499999788, 809.71) [a] 
(0.9607441666666454, 858.08) [a] 
(0.960744583333312, 858.36) [a] 
(0.9607447916666454, 858.39) [a] 
(0.960745208333312, 858.62) [a] 
(0.9607456249999786, 858.63) [a] 
(0.9607460416666452, 869.05) [a] 
(0.9607464583333118, 869.25) [a] 
(0.9607466666666452, 869.27) [a] 
(0.9607470833333118, 869.33) [a] 
(0.9607474999999784, 882.77) [a] 
(0.9607477083333118, 882.78) [a] 
(0.9607481249999784, 882.85) [a] 
(0.960748541666645, 882.86) [a] 
(0.9607487499999784, 883.15) [a] 
(0.9607493749999784, 884.88) [a] 
(0.9607499999999783, 885.06) [a] 
(0.9607508333333117, 885.11) [a] 
(0.9607514583333117, 885.41) [a] 
(0.9607518749999783, 889.43) [a] 
(0.9607522916666449, 889.45) [a] 
(0.9607524999999782, 889.47) [a] 
(0.9607529166666449, 889.51) [a] 
(0.9607533333333115, 889.7) [a] 
(0.9607672916666448, 898.77) [a] 
(0.9607679166666447, 899.68) [a] 
(0.960781874999978, 899.69) [a] 
(0.9607827083333114, 899.7) [a] 
(0.9607833333333113, 899.79) [a] 
(0.960797499999978, 899.84) [a] 
(0.9608114583333113, 905.78) [a] 
(0.9608256249999779, 905.79) [a] 
(0.9608260416666445, 905.94) [a] 
(0.9608399999999778, 906.3) [a] 
(0.9608539583333111, 909.78) [a] 
(0.9608541666666445, 915.33) [a] 
(0.9608545833333111, 915.35) [a] 
(0.9608549999999777, 915.59) [a] 
(0.9608552083333111, 926.97) [a] 
(0.9608556249999777, 926.98) [a] 
(0.9608560416666443, 927.86) [a] 
(0.9608564583333109, 944.31) [a] 
(0.9608566666666443, 945.42) [a] 
(0.9608570833333109, 945.52) [a] 
(0.9608577083333109, 954.81) [a] 
(0.9608581249999775, 954.82) [a] 
(0.9608583333333108, 954.83) [a] 
(0.9608585416666442, 962.69) [a] 
(0.9608589583333108, 968.44) [a] 
(0.9608595833333108, 968.49) [a] 
(0.9608599999999774, 968.5) [a] 
(0.9608608333333106, 972.49) [a] 
(0.9608618749999773, 972.5) [a] 
(0.9608622916666439, 972.51) [a] 
(0.9608629166666439, 972.54) [a] 
(0.9608643749999772, 972.55) [a] 
(0.9608647916666438, 972.56) [a] 
(0.9608654166666438, 972.62) [a] 
(0.9608660416666438, 972.63) [a] 
(0.9608668749999771, 972.64) [a] 
(0.9608693749999772, 972.65) [a] 
(0.9608697916666438, 972.74) [a] 
(0.9608702083333104, 972.79) [a] 
(0.960870624999977, 972.84) [a] 
(0.9608708333333104, 973.28) [a] 
(0.960871249999977, 973.34) [a] 
(0.9608716666666436, 973.35) [a] 
(0.9608722916666436, 973.37) [a] 
(0.9608727083333102, 973.47) [a] 
(0.9608731249999768, 985.63) [a] 
(0.9608733333333102, 985.92) [a] 
(0.9608737499999768, 986.52) [a] 
(0.9608741666666434, 989.8) [a] 
(0.9608747916666434, 994.21) [a] 
(0.9608749999999767, 994.77) [a] 
(0.9608754166666433, 994.79) [a] 
(0.96087583333331, 994.89) [a] 
(0.9608760416666433, 995.19) [a] 
(0.9608764583333099, 995.78) [a] 
(0.9608768749999765, 996.1) [a] 
(0.9608770833333099, 996.34) [a] 
(0.9608774999999765, 997.27) [a] 
(0.9608779166666431, 998.89) [a] 
(0.9608781249999765, 1005.1) [a] 
(0.9608785416666431, 1005.2) [a] 
(0.9608787499999765, 1007.5) [a] 
(0.9608789583333098, 1016.5) [a] 
(0.9608791666666432, 1018.8) [a] 
(0.9608820833333097, 1019) [a] 
(0.9608822916666431, 1019.2) [a] 
(0.9608827083333097, 1019.4) [a] 
(0.960883541666643, 1019.8) [a] 
(0.960884166666643, 1019.9) [a] 
(0.9608845833333096, 1022.6) [a] 
(0.9608849999999762, 1029.6) [a] 
(0.9608885416666428, 1032.5) [a] 
(0.9608912499999761, 1032.9) [a] 
(0.9608927083333094, 1036.8) [a] 
(0.9608929166666428, 1062.9) [a] 
(0.9608937499999761, 1067.7) [a] 
(0.9608945833333093, 1073.1) [a] 
(0.9608966666666426, 1073.2) [a] 
(0.9608972916666426, 1088.6) [a] 
(0.9609114583333093, 1089.7) [a] 
(0.9609135416666426, 1090.7) [a] 
(0.9609156249999758, 1091.2) [a] 
(0.9609170833333092, 1110) [a] 
(0.9609174999999758, 1117.5) [a] 
(0.9609179166666424, 1136.3) [a] 
(0.960918333333309, 1201.9) [a] 
(0.960918958333309, 1202) [a] 
(0.9609191666666423, 1204.4) [a] 
(0.960919583333309, 1230.8) [a] 
(0.9609210416666423, 1277.9) [a] 
(0.9609212499999756, 1302.9) [a] 
(0.9609216666666422, 1303.7) [a] 
(0.9609233333333088, 1350.7) [a] 
(0.9609237499999754, 1350.8) [a] 
(0.9609239583333088, 1350.9) [a] 
(0.9609252083333087, 1368.5) [a] 
(0.9609260416666421, 1373.6) [a] 
(0.9609266666666421, 1373.8) [a] 
(0.960927291666642, 1373.9) [a] 
(0.9609287499999754, 1374) [a] 
(0.960929166666642, 1380.6) [a] 
(0.9609431249999753, 1383.7) [a] 
(0.9609570833333085, 1385.4) [a] 
(0.9609712499999752, 1386.8) [a] 
(0.9609852083333085, 1389.3) [a] 
(0.9609993749999751, 1389.4) [a] 
(0.9609997916666417, 1390.3) [a] 
(0.9609999999999751, 1390.5) [a] 
(0.9610004166666417, 1398.5) [a] 
(0.9610008333333083, 1419.1) [a] 
(0.9610043749999749, 1429.3) [a] 
(0.9610085416666415, 1430) [a] 
(0.9610091666666415, 1430.2) [a] 
(0.9610097916666415, 1431.7) [a] 
(0.9610112499999748, 1431.8) [a] 
(0.9610114583333081, 1432.5) [a] 
(0.9610118749999748, 1447.4) [a] 
(0.9610122916666414, 1448.1) [a] 
(0.961012708333308, 1449.8) [a] 
(0.9610137499999746, 1466.6) [a] 
(0.9610139583333079, 1466.8) [a] 
(0.9610147916666413, 1491.2) [a] 
(0.9610154166666413, 1494.8) [a] 
(0.9610160416666412, 1506.6) [a] 
(0.9610164583333078, 1506.7) [a] 
(0.9610170833333078, 1511.1) [a] 
(0.9610172916666412, 1529.1) [a] 
(0.9611343749999746, 1540.8) [a] 
(0.9611483333333078, 1553.4) [a] 
(0.9611622916666411, 1557) [a] 
(0.9611624999999745, 1565.4) [a] 
(0.9611629166666411, 1584.2) [a] 
(0.9611631249999745, 1606) [a] 
(0.9611645833333078, 1615.2) [a] 
(0.9611652083333078, 1615.5) [a] 
(0.9611666666666411, 1615.7) [a] 
(0.9611681249999744, 1628.5) [a] 
(0.9611687499999744, 1688.7) [a] 
(0.9611693749999743, 1688.9) [a] 
(0.9611708333333077, 1691) [a] 
(0.961172291666641, 1693.8) [a] 
(0.961172916666641, 1712) [a] 
(0.9611735416666409, 1737.8) [a] 
(0.9611743749999743, 1737.9) [a] 
(0.9611745833333076, 1792.2) [a] 
(0.961174791666641, 1793) [a] 
(0.961175416666641, 1800.1) [a] 
(0.9611758333333076, 1809.1) [a] 
(0.9611764583333076, 1820.2) [a] 
(0.9611906249999742, 1987.7) [a] 
(0.9612045833333075, 1988) [a] 
(0.9612187499999741, 1989) [a] 
(0.9612466666666407, 1995.3) [a] 
(0.9612747916666406, 1995.4) [a] 
(0.961298124999974, 2117) [a] 
(0.9612985416666406, 2193.2) [a] 
(0.9612989583333073, 2196.5) [a] 
(0.9613002083333072, 2220.3) [a] 
(0.9613006249999738, 2220.4) [a] 
(0.9613010416666404, 2221) [a] 
(0.9613016666666404, 2230) [a] 
(0.961302083333307, 2230.3) [a] 
(0.9613022916666404, 2243.3) [a] 
(0.961302708333307, 2243.6) [a] 
(0.9613031249999736, 2243.9) [a] 
(0.9613041666666403, 2296.2) [a] 
(0.9613047916666403, 2312.1) [a] 
(0.961305833333307, 2345.6) [a] 
(0.961306458333307, 2345.7) [a] 
(0.9613068749999736, 2346.2) [a] 
(0.9613072916666402, 2358.3) [a] 
(0.9613074999999736, 2359.2) [a] 
(0.9613079166666402, 2365.4) [a] 
(0.9613083333333068, 2522.5) [a] 
(0.9613085416666401, 2667.2) [a] 
(0.9613089583333068, 2684.8) [a] 
(0.9613095833333067, 2892.6) [a] 
(0.9613104166666401, 2922.7) [a] 
(0.9613108333333067, 3042.7) [a] 
(0.96132479166664, 3064.7) [a] 
(0.9613389583333066, 3064.8) [a] 
(0.9613529166666399, 3142) [a] 
(0.96138104166664, 3216.4) [a] 
(0.9613816666666399, 3343.4) [a] 
(0.9613822916666399, 3343.6) [a] 
(0.9613831249999732, 3343.8) [a] 
(0.9613837499999732, 3384.4) [a] 
(0.9613977083333065, 3433.3) [a] 
(0.9614118749999732, 3460.9) [a] 
(0.9614327083333065, 3477.7) [a] 
(0.9614535416666399, 3477.8) [a] 
(0.9614952083333066, 3478.1) [a] 
(0.9615368749999733, 3478.3) [a] 
(0.9615374999999733, 3484.1) [a] 
(0.9615583333333066, 3491.5) [a] 
(0.96157916666664, 3491.7) [a] 
(0.9615999999999734, 3492.9) [a] 
(0.9616208333333067, 3498.2) [a] 
(0.9616416666666401, 3512.7) [a] 
(0.9616624999999734, 3512.9) [a] 
},{(0.8341387402290723, 0) [b] 
(0.876006881023629, 0.001) [b] 
(0.8902511773095898, 0.002) [b] 
(0.8952786742173512, 0.003) [b] 
(0.8995719348148522, 0.004) [b] 
(0.9020430570121037, 0.005) [b] 
(0.9036811280715706, 0.006) [b] 
(0.90418048404727, 0.007) [b] 
(0.9055121494023545, 0.008) [b] 
(0.906696294887623, 0.009) [b] 
(0.9081013992845841, 0.01) [b] 
(0.9100857030079612, 0.011) [b] 
(0.9104242746827336, 0.012) [b] 
(0.9117623307824767, 0.013) [b] 
(0.9119107713982533, 0.014) [b] 
(0.9119849237598686, 0.015) [b] 
(0.9120322185006933, 0.016) [b] 
(0.9127039355855388, 0.017) [b] 
(0.9129548597108733, 0.018) [b] 
(0.9134753639760872, 0.019) [b] 
(0.9139233591117885, 0.02) [b] 
(0.9144839802157528, 0.021) [b] 
(0.9148521108701348, 0.022) [b] 
(0.9148589867846315, 0.023) [b] 
(0.9148920555677003, 0.024) [b] 
(0.9149788303645552, 0.025) [b] 
(0.9150413303645551, 0.026) [b] 
(0.9150669950807034, 0.028) [b] 
(0.9151652835310989, 0.029) [b] 
(0.9152086086454224, 0.03) [b] 
(0.915224861693141, 0.031) [b] 
(0.9152927362055114, 0.032) [b] 
(0.9153156528721782, 0.033) [b] 
(0.9153252791524735, 0.034) [b] 
(0.9153381263746957, 0.035) [b] 
(0.9154589181600707, 0.036) [b] 
(0.915465127042102, 0.037) [b] 
(0.9154685649993504, 0.039) [b] 
(0.9154765511104614, 0.047) [b] 
(0.9162341899993504, 0.049) [b] 
(0.9162733749523284, 0.051) [b] 
(0.9162754582856617, 0.059) [b] 
(0.9162768471745506, 0.06) [b] 
(0.9167405956970092, 0.061) [b] 
(0.916768373474787, 0.062) [b] 
(0.9167892068081204, 0.064) [b] 
(0.9168273072261159, 0.065) [b] 
(0.9168328627816715, 0.066) [b] 
(0.9168467516705605, 0.067) [b] 
(0.9169338836743591, 0.069) [b] 
(0.9169401336743591, 0.07) [b] 
(0.9171186058965812, 0.071) [b] 
(0.9171220781188034, 0.072) [b] 
(0.9171248558965811, 0.073) [b] 
(0.91724186978547, 0.074) [b] 
(0.9172752465215811, 0.076) [b] 
(0.9173023732576923, 0.077) [b] 
(0.9174523732576922, 0.078) [b] 
(0.9175337534660256, 0.079) [b] 
(0.91759147916047, 0.08) [b] 
(0.9175928680493588, 0.081) [b] 
(0.9176522222753797, 0.082) [b] 
(0.917654305608713, 0.084) [b] 
(0.9176619444976019, 0.085) [b] 
(0.917665416719824, 0.086) [b] 
(0.9176681944976018, 0.087) [b] 
(0.9176723611642685, 0.088) [b] 
(0.9176734028309352, 0.092) [b] 
(0.9179973534930148, 0.096) [b] 
(0.9180394693959798, 0.097) [b] 
(0.9180713734286495, 0.099) [b] 
(0.9181196373175384, 0.102) [b] 
(0.9183366512064273, 0.104) [b] 
(0.9183669029425384, 0.107) [b] 
(0.9183682781254378, 0.11) [b] 
(0.9184103940284027, 0.111) [b] 
(0.9184176856950694, 0.112) [b] 
(0.9190035792790284, 0.113) [b] 
(0.9194717440730359, 0.115) [b] 
(0.9195187188945315, 0.121) [b] 
(0.9196003537268977, 0.124) [b] 
(0.9196760481713422, 0.125) [b] 
(0.9198059092824533, 0.131) [b] 
(0.91980695094912, 0.133) [b] 
(0.9198090342824533, 0.137) [b] 
(0.9198632877546755, 0.14) [b] 
(0.9200260481713421, 0.141) [b] 
(0.9200865082407866, 0.142) [b] 
(0.9200875499074533, 0.143) [b] 
(0.92008859157412, 0.144) [b] 
(0.9209078799636331, 0.146) [b] 
(0.920936048366411, 0.15) [b] 
(0.9209377844775221, 0.152) [b] 
(0.9209920379497443, 0.153) [b] 
(0.9209979407275221, 0.156) [b] 
(0.9211809268386332, 0.157) [b] 
(0.921194815727522, 0.158) [b] 
(0.9212017601719665, 0.159) [b] 
(0.9212437740608554, 0.16) [b] 
(0.9212458573941887, 0.161) [b] 
(0.9212493296164109, 0.163) [b] 
(0.9212569685052998, 0.164) [b] 
(0.9212826629497441, 0.165) [b] 
(0.9212833573941885, 0.166) [b] 
(0.9212889129497441, 0.167) [b] 
(0.9213118296164108, 0.168) [b] 
(0.9213194685052996, 0.169) [b] 
(0.9213243296164106, 0.17) [b] 
(0.9213295379497439, 0.171) [b] 
(0.9213298851719661, 0.172) [b] 
(0.9213302323941883, 0.175) [b] 
(0.9213534962830772, 0.177) [b] 
(0.9213655526102377, 0.178) [b] 
(0.9213700664991267, 0.179) [b] 
(0.9213811776102377, 0.18) [b] 
(0.9214033998324599, 0.181) [b] 
(0.921408260943571, 0.182) [b] 
(0.9215037470546821, 0.183) [b] 
(0.9215645109435708, 0.184) [b] 
(0.9216395109435708, 0.185) [b] 
(0.9216516637213485, 0.186) [b] 
(0.9216902053880152, 0.187) [b] 
(0.921847777880178, 0.188) [b] 
(0.9219220834357335, 0.19) [b] 
(0.9220515973246223, 0.191) [b] 
(0.9220873612135111, 0.192) [b] 
(0.9222349306579554, 0.193) [b] 
(0.9222370139912885, 0.194) [b] 
(0.922239444546844, 0.195) [b] 
(0.9222439584357329, 0.196) [b] 
(0.9222679167690663, 0.197) [b] 
(0.9222884028801774, 0.198) [b] 
(0.9223224306579552, 0.199) [b] 
(0.9228363195468441, 0.201) [b] 
(0.9230248612135108, 0.202) [b] 
(0.9230811112135108, 0.203) [b] 
(0.9234279862135109, 0.204) [b] 
(0.9234540278801776, 0.205) [b] 
(0.9235196528801776, 0.206) [b] 
(0.9236588889912887, 0.207) [b] 
(0.9243463889912886, 0.208) [b] 
(0.9243515973246219, 0.209) [b] 
(0.9244102778801774, 0.21) [b] 
(0.9244793751023994, 0.211) [b] 
(0.9245088889912882, 0.212) [b] 
(0.9245512501023992, 0.213) [b] 
(0.9246068056579548, 0.214) [b] 
(0.9246689584357325, 0.215) [b] 
(0.924782152880177, 0.216) [b] 
(0.9249585417690659, 0.217) [b] 
(0.9251706945468435, 0.218) [b] 
(0.925273125102399, 0.219) [b] 
(0.9262786806579546, 0.22) [b] 
(0.9264345834357323, 0.221) [b] 
(0.9264894445468433, 0.222) [b] 
(0.9265147917690656, 0.223) [b] 
(0.9265415278801766, 0.224) [b] 
(0.9265540278801766, 0.225) [b] 
(0.9267901936105136, 0.226) [b] 
(0.9269981523096457, 0.227) [b] 
(0.9271002356429789, 0.228) [b] 
(0.9271269717540901, 0.229) [b] 
(0.9271339161985346, 0.23) [b] 
(0.9273287078652012, 0.231) [b] 
(0.9273439856429788, 0.232) [b] 
(0.9274356523096454, 0.233) [b] 
(0.9277925967540899, 0.234) [b] 
(0.9278568328652009, 0.235) [b] 
(0.9278669023096454, 0.236) [b] 
(0.9281530134207564, 0.237) [b] 
(0.9282731523096452, 0.238) [b] 
(0.9287728050874231, 0.239) [b] 
(0.9288311384207564, 0.24) [b] 
(0.9288349578652006, 0.241) [b] 
(0.9288738467540896, 0.242) [b] 
(0.9292544023096451, 0.243) [b] 
(0.9292835689763117, 0.244) [b] 
(0.9293362230537261, 0.245) [b] 
(0.9294563619426146, 0.246) [b] 
(0.9295039313870589, 0.247) [b] 
(0.9297462924981699, 0.248) [b] 
(0.9297553202759474, 0.249) [b] 
(0.9298008063870585, 0.25) [b] 
(0.9298070563870585, 0.251) [b] 
(0.9298119174981697, 0.252) [b] 
(0.9305113822230495, 0.253) [b] 
(0.9305568683341606, 0.254) [b] 
(0.9305783961119384, 0.255) [b] 
(0.930816937778605, 0.256) [b] 
(0.9308325627786049, 0.257) [b] 
(0.9309554794452715, 0.258) [b] 
(0.9309661493077113, 0.259) [b] 
(0.9310335104188223, 0.26) [b] 
(0.9313161493077111, 0.261) [b] 
(0.9313533020854888, 0.262) [b] 
(0.9313553854188221, 0.263) [b] 
(0.9313598993077109, 0.264) [b] 
(0.931369621529933, 0.265) [b] 
(0.9316848993077107, 0.266) [b] 
(0.9317235423765537, 0.267) [b] 
(0.9317272677945492, 0.268) [b] 
(0.9317321289056603, 0.269) [b] 
(0.9317425455723269, 0.27) [b] 
(0.931780045572327, 0.271) [b] 
(0.9318019205723269, 0.272) [b] 
(0.9318116427945491, 0.273) [b] 
(0.931813031683438, 0.274) [b] 
(0.9318147677945491, 0.275) [b] 
(0.9319689344612156, 0.276) [b] 
(0.9320706705723267, 0.278) [b] 
(0.9321476410156765, 0.279) [b] 
(0.932181321571232, 0.28) [b] 
(0.9321820160156764, 0.281) [b] 
(0.9321927799045653, 0.282) [b] 
(0.9329624574517316, 0.283) [b] 
(0.9330137710806314, 0.284) [b] 
(0.9330141183028536, 0.29) [b] 
(0.9330175905250758, 0.292) [b] 
(0.9330269655250758, 0.293) [b] 
(0.9330300905250758, 0.294) [b] 
(0.9330328683028535, 0.295) [b] 
(0.9330855708206423, 0.299) [b] 
(0.9331675015590971, 0.3) [b] 
(0.9331768765590971, 0.304) [b] 
(0.9332692308172134, 0.305) [b] 
(0.9332706197061023, 0.306) [b] 
(0.9332831197061022, 0.307) [b] 
(0.9333147169283245, 0.309) [b] 
(0.9333161058172134, 0.312) [b] 
(0.9333338141505467, 0.313) [b] 
(0.9333365919283245, 0.314) [b] 
(0.9333744391505467, 0.315) [b] 
(0.9333876267419964, 0.316) [b] 
(0.9333923869736637, 0.317) [b] 
(0.9333965536403304, 0.318) [b] 
(0.9334031508625525, 0.319) [b] 
(0.9334049216128467, 0.326) [b] 
(0.9334080466128467, 0.327) [b] 
(0.9339749557537949, 0.328) [b] 
(0.9339763792818668, 0.331) [b] 
(0.9339849204502987, 0.334) [b] 
(0.9340012398947432, 0.335) [b] 
(0.9340043648947431, 0.336) [b] 
(0.9340439482280765, 0.337) [b] 
(0.9340540176725209, 0.338) [b] 
(0.9340814482280765, 0.339) [b] 
(0.9340981148947431, 0.34) [b] 
(0.934108079591247, 0.343) [b] 
(0.9341109266473909, 0.344) [b] 
(0.9341227322029465, 0.345) [b] 
(0.9341298498433064, 0.346) [b] 
(0.9341347109544175, 0.351) [b] 
(0.9341388776210842, 0.352) [b] 
(0.9342069765794175, 0.353) [b] 
(0.9342643550516397, 0.354) [b] 
(0.9343522890794175, 0.355) [b] 
(0.9343634001905285, 0.356) [b] 
(0.9343644418571952, 0.357) [b] 
(0.9343647890794174, 0.358) [b] 
(0.9343732609694831, 0.359) [b] 
(0.934416663747261, 0.362) [b] 
(0.9344437904833721, 0.363) [b] 
(0.9345674884000388, 0.364) [b] 
(0.9345953095805943, 0.365) [b] 
(0.9347605005528166, 0.366) [b] 
(0.9348786429139277, 0.367) [b] 
(0.9348928790250388, 0.368) [b] 
(0.9349060734694833, 0.37) [b] 
(0.9349125920343561, 0.372) [b] 
(0.935076297762633, 0.373) [b] 
(0.9351034244987442, 0.374) [b] 
(0.9351107161654109, 0.376) [b] 
(0.9351142576659992, 0.377) [b] 
(0.935128492946719, 0.38) [b] 
(0.9351327635309349, 0.381) [b] 
(0.9351370341151508, 0.383) [b] 
(0.9351427282274387, 0.387) [b] 
(0.9351455752835827, 0.388) [b] 
(0.9352025197280271, 0.392) [b] 
(0.9352090382928999, 0.394) [b] 
(0.9352305660706777, 0.4) [b] 
(0.9353430247883636, 0.403) [b] 
(0.9353529894848674, 0.407) [b] 
(0.9353672247655872, 0.409) [b] 
(0.9354241658884661, 0.417) [b] 
(0.9354300686662439, 0.425) [b] 
(0.9355026685979145, 0.427) [b] 
(0.9355332241534701, 0.434) [b] 
(0.93555888096792, 0.437) [b] 
(0.935560617079031, 0.439) [b] 
(0.9356375875223808, 0.44) [b] 
(0.9356889011512807, 0.451) [b] 
(0.9356916789290585, 0.468) [b] 
(0.9357429925579583, 0.471) [b] 
(0.9358068006232978, 0.474) [b] 
(0.9358387046559675, 0.475) [b] 
(0.9358706086886373, 0.476) [b] 
(0.935912275355304, 0.5) [b] 
(0.9359584559108596, 0.508) [b] 
(0.9359681781330818, 0.549) [b] 
(0.936022431605304, 0.565) [b] 
(0.9360495583414151, 0.566) [b] 
(0.9364293326469707, 0.568) [b] 
(0.9365616243136373, 0.606) [b] 
(0.9368861887140473, 0.609) [b] 
(0.9368868831584917, 0.611) [b] 
(0.9368875776029361, 0.615) [b] 
(0.9369147043390472, 0.624) [b] 
(0.9369674417050503, 0.64) [b] 
(0.9369677889272725, 0.644) [b] 
(0.9369949156633837, 0.659) [b] 
(0.9370220423994948, 0.662) [b] 
(0.9370476992139447, 0.664) [b] 
(0.9370685325472781, 0.68) [b] 
(0.9374528851267109, 0.683) [b] 
(0.9374571557109268, 0.696) [b] 
(0.9377390142691776, 0.715) [b] 
(0.9377489789656814, 0.72) [b] 
(0.9378671317956553, 0.765) [b] 
(0.9378796317956553, 0.821) [b] 
(0.9379338852678775, 0.823) [b] 
(0.9379881387400997, 0.827) [b] 
(0.9379995269646755, 0.844) [b] 
(0.9380009504927475, 0.845) [b] 
(0.9380530990117295, 0.877) [b] 
(0.9381508774848208, 0.898) [b] 
(0.938154349707043, 0.931) [b] 
(0.938158620291259, 0.947) [b] 
(0.9383023966265283, 0.982) [b] 
(0.9383280201318238, 1.019) [b] 
(0.9383287145762682, 1.051) [b] 
(0.9383315616324122, 1.098) [b] 
(0.9383379778114492, 1.212) [b] 
(0.9383443939904863, 1.214) [b] 
(0.9383508101695234, 1.23) [b] 
(0.9383572263485604, 1.246) [b] 
(0.9383610457930049, 1.257) [b] 
(0.938362781904116, 1.338) [b] 
(0.9383631291263382, 1.484) [b] 
(0.9383645180152271, 1.511) [b] 
(0.9383709341942642, 1.625) [b] 
(0.9383743123900374, 1.722) [b] 
(0.9383776905858107, 1.754) [b] 
(0.9385113232563259, 1.792) [b] 
(0.9385306769534351, 1.836) [b] 
(0.9385315985580593, 1.845) [b] 
(0.9385320593603714, 1.847) [b] 
(0.9385329809649957, 1.857) [b] 
(0.9385398929996775, 1.878) [b] 
(0.9385403538019896, 1.881) [b] 
(0.9429527838559773, 1.882) [b] 
(0.9429739644115328, 1.886) [b] 
(0.9429808764462146, 1.888) [b] 
(0.9429933764462146, 1.895) [b] 
(0.9429944181128813, 1.922) [b] 
(0.9429948789151934, 1.924) [b] 
(0.9430122400263045, 1.939) [b] 
(0.9430136224332408, 1.944) [b] 
(0.9430163872471136, 1.947) [b] 
(0.9430173088517378, 1.95) [b] 
(0.9430593227406268, 1.958) [b] 
(0.9430864494767379, 1.967) [b] 
(0.9430947439183561, 1.984) [b] 
(0.9430970479299168, 1.992) [b] 
(0.9430975087322289, 1.997) [b] 
(0.9430988911391652, 1.999) [b] 
(0.9431082661391652, 2.003) [b] 
(0.9431497383472561, 2.059) [b] 
(0.9431552939028117, 2.067) [b] 
(0.9431771689028117, 2.108) [b] 
(0.9431912075371334, 2.176) [b] 
(0.9432398186482445, 2.177) [b] 
(0.9433023186482444, 2.179) [b] 
(0.9433101522875506, 2.18) [b] 
(0.9433686471668492, 2.203) [b] 
(0.9433690867102866, 2.208) [b] 
(0.9433695262537241, 2.209) [b] 
(0.9433930136644719, 2.211) [b] 
(0.9433980824899053, 2.226) [b] 
(0.9433987769343497, 2.227) [b] 
(0.943399069963308, 2.229) [b] 
(0.9434039310744191, 2.249) [b] 
(0.9434042241033774, 2.25) [b] 
(0.9434052497047315, 2.252) [b] 
(0.9434056892481689, 2.254) [b] 
(0.943405835762648, 2.257) [b] 
(0.9434391975746751, 2.258) [b] 
(0.9434396371181125, 2.297) [b] 
(0.9434413125540072, 2.304) [b] 
(0.9434416055829655, 2.31) [b] 
(0.9434418986119238, 2.311) [b] 
(0.9434450236119238, 2.314) [b] 
(0.9434459452165481, 2.325) [b] 
(0.9434468668211724, 2.334) [b] 
(0.9434477884257967, 2.338) [b] 
(0.943449170832733, 2.34) [b] 
(0.9434496103761705, 2.341) [b] 
(0.9434503642074409, 2.343) [b] 
(0.9434738516181888, 2.36) [b] 
(0.9434742911616262, 2.369) [b] 
(0.9435201244949595, 2.405) [b] 
(0.9435924624579225, 2.423) [b] 
(0.9436165751122435, 2.425) [b] 
(0.9436217834455768, 2.475) [b] 
(0.9438558658425805, 2.477) [b] 
(0.9438572547314694, 2.484) [b] 
(0.9439005729927982, 2.56) [b] 
(0.9439060737243955, 2.561) [b] 
(0.9439081364987445, 2.624) [b] 
(0.9439136372303418, 2.629) [b] 
(0.9439153733414529, 2.647) [b] 
(0.9439160609329026, 2.668) [b] 
(0.9439167485243523, 2.699) [b] 
(0.9439170957465745, 2.72) [b] 
(0.9439174429687966, 2.931) [b] 
(0.9439236929687966, 2.985) [b] 
(0.9439759395727674, 3.085) [b] 
(0.9439824703982638, 3.091) [b] 
(0.9441267015219579, 3.115) [b] 
(0.944153828258069, 3.287) [b] 
(0.9441603590835654, 3.527) [b] 
(0.9441718174168988, 3.707) [b] 
(0.9441858560512205, 3.78) [b] 
(0.9443028972497224, 3.796) [b] 
(0.9443285540641723, 3.807) [b] 
(0.944342592698494, 3.832) [b] 
(0.9443443288096051, 4.04) [b] 
(0.9443450164010547, 4.076) [b] 
(0.9443706732155046, 4.14) [b] 
(0.9443963300299545, 4.255) [b] 
(0.9444064676808211, 4.274) [b] 
(0.9444129189131909, 4.281) [b] 
(0.9444262821802424, 4.401) [b] 
(0.9444519389946923, 4.414) [b] 
(0.9445476510927014, 4.443) [b] 
(0.9445795551253712, 4.444) [b] 
(0.9446114591580409, 4.461) [b] 
(0.9447323489065902, 4.465) [b] 
(0.9447447905690175, 4.496) [b] 
(0.9448067945372715, 4.602) [b] 
(0.9448309071915925, 4.633) [b] 
(0.944842963518753, 4.65) [b] 
(0.9448656182990922, 4.653) [b] 
(0.9448674615083407, 4.662) [b] 
(0.9448688503972296, 4.741) [b] 
(0.944874380024975, 4.82) [b] 
(0.9449256936538749, 4.896) [b] 
(0.9449513504683248, 4.898) [b] 
(0.9449531936775732, 4.981) [b] 
(0.9449788504920231, 5.035) [b] 
(0.9449923632751159, 5.216) [b] 
(0.9450193888413015, 5.217) [b] 
(0.9450329016243944, 5.226) [b] 
(0.9450362798201676, 5.23) [b] 
(0.9450430362117139, 5.391) [b] 
(0.9450464144074872, 5.392) [b] 
(0.9451277946158205, 5.606) [b] 
(0.9451549213519317, 5.607) [b] 
(0.9451820480880428, 5.611) [b] 
(0.9451869091991539, 5.723) [b] 
(0.9451902873949272, 5.85) [b] 
(0.9451936655907004, 5.853) [b] 
(0.9453951485049492, 5.989) [b] 
(0.9453968846160603, 6.145) [b] 
(0.9454273873002965, 6.2) [b] 
(0.9454280817447409, 6.335) [b] 
(0.9454561590133842, 6.509) [b] 
(0.9455263521849925, 6.51) [b] 
(0.9455403908193142, 6.517) [b] 
(0.9455544294536359, 6.521) [b] 
(0.9455684680879576, 6.55) [b] 
(0.9456246226252442, 6.58) [b] 
(0.9456667385282091, 6.581) [b] 
(0.9456807771625309, 6.844) [b] 
(0.945685385185652, 6.946) [b] 
(0.9456867675925884, 6.963) [b] 
(0.9456913756157096, 6.967) [b] 
(0.9456922972203339, 6.972) [b] 
(0.9457540447301582, 7.002) [b] 
(0.9457627999740885, 7.026) [b] 
(0.9457660255902733, 7.031) [b] 
(0.9457677617013844, 7.034) [b] 
(0.9457686833060087, 7.035) [b] 
(0.9457700657129451, 7.038) [b] 
(0.9457709873175694, 7.041) [b] 
(0.9457719089221936, 7.044) [b] 
(0.9457728305268179, 7.061) [b] 
(0.9457774385499391, 7.095) [b] 
(0.9457797425614998, 7.098) [b] 
(0.9457815857707482, 7.1) [b] 
(0.9457875762008058, 7.114) [b] 
(0.9457884978054301, 7.117) [b] 
(0.9457917234216149, 7.124) [b] 
(0.945792184223927, 7.132) [b] 
(0.945792645026239, 7.149) [b] 
(0.9457931058285511, 7.153) [b] 
(0.9457935666308632, 7.157) [b] 
(0.9457940274331753, 7.162) [b] 
(0.9458148607665087, 7.695) [b] 
(0.9458520135442865, 7.896) [b] 
(0.9458660521786082, 8.797) [b] 
(0.9458979562112779, 8.922) [b] 
(0.9459298602439477, 8.924) [b] 
(0.9460574763746266, 8.925) [b] 
(0.9460893804072963, 8.926) [b] 
(0.946121284439966, 8.932) [b] 
(0.9461531884726357, 8.933) [b] 
(0.9461850925053055, 8.935) [b] 
(0.9462169965379752, 9.135) [b] 
(0.9462489005706449, 9.136) [b] 
(0.9462808046033147, 9.137) [b] 
(0.9463127086359844, 9.138) [b] 
(0.9463765167013238, 9.14) [b] 
(0.9464084207339936, 9.144) [b] 
(0.946472228799333, 9.145) [b] 
(0.9465041328320027, 9.148) [b] 
(0.9465360368646725, 9.149) [b] 
(0.9465679408973422, 9.15) [b] 
(0.9465734416289395, 9.487) [b] 
(0.9465741292203892, 9.488) [b] 
(0.9465748168118389, 9.518) [b] 
(0.9465761919947382, 9.585) [b] 
(0.9465768795861879, 9.596) [b] 
(0.9465782547690872, 9.607) [b] 
(0.9465846709481243, 10.76) [b] 
(0.946591189512997, 13.195) [b] 
(0.9466183162491082, 14.234) [b] 
(0.9466454429852194, 14.235) [b] 
(0.9466662763185527, 14.86) [b] 
(0.9467287763185527, 14.861) [b] 
(0.9467322485407749, 15.336) [b] 
(0.9467343318741082, 15.345) [b] 
(0.9467357207629971, 15.373) [b] 
(0.9467364152074415, 15.421) [b] 
(0.9467391929852192, 17.169) [b] 
(0.9467395402074414, 17.518) [b] 
(0.9467726089905102, 17.752) [b] 
(0.9468056777735789, 18.033) [b] 
(0.9468101916624678, 19.308) [b] 
(0.9468108861069122, 19.361) [b] 
(0.9468112333291344, 19.452) [b] 
(0.9468122749958011, 19.687) [b] 
(0.9468539416624678, 20.476) [b] 
(0.9469822257347175, 20.804) [b] 
(0.9471105098069671, 20.805) [b] 
(0.947136166621417, 20.837) [b] 
(0.9471618234358669, 20.858) [b] 
(0.9472387938792166, 20.868) [b] 
(0.9472644506936665, 20.877) [b] 
(0.9472852840269999, 21.145) [b] 
(0.9474135680992495, 22.194) [b] 
(0.9475418521714991, 22.196) [b] 
(0.947567508985949, 22.232) [b] 
(0.9475931658003989, 22.255) [b] 
(0.9476701362437486, 22.266) [b] 
(0.9476957930581985, 22.276) [b] 
(0.9478128342567004, 22.914) [b] 
(0.9478384910711503, 25.79) [b] 
(0.9478641478856001, 26.037) [b] 
(0.9478675260813734, 26.549) [b] 
(0.9478709042771466, 26.612) [b] 
(0.9478742824729198, 26.644) [b] 
(0.9478863388000803, 26.654) [b] 
(0.9478983951272408, 26.657) [b] 
(0.9479240519416907, 27.495) [b] 
(0.9479497087561406, 27.746) [b] 
(0.9479538754228073, 28.944) [b] 
(0.9479810021589185, 30.241) [b] 
(0.9480081288950296, 30.305) [b] 
(0.9480316163057775, 32.04) [b] 
(0.948544086411688, 33.579) [b] 
(0.9485513780783547, 35.618) [b] 
(0.9485635308561324, 35.666) [b] 
(0.9485891876705823, 36.875) [b] 
(0.9486148444850322, 37.402) [b] 
(0.9486288831193539, 37.502) [b] 
(0.9486429217536756, 37.95) [b] 
(0.9486685785681255, 39.22) [b] 
(0.9486869813459032, 39.353) [b] 
(0.9486883702347921, 39.409) [b] 
(0.9486890646792365, 39.516) [b] 
(0.9486894119014587, 39.728) [b] 
(0.9487150687159086, 39.775) [b] 
(0.9487154159381308, 39.835) [b] 
(0.9487168048270197, 41.058) [b] 
(0.9487439315631309, 42.671) [b] 
(0.948771058299242, 42.672) [b] 
(0.9487981850353532, 42.673) [b] 
(0.9488253117714643, 42.674) [b] 
(0.9488524385075755, 42.692) [b] 
(0.9488795652436867, 42.693) [b] 
(0.94896094545202, 42.694) [b] 
(0.9489880721881312, 42.695) [b] 
(0.9490694523964646, 42.696) [b] 
(0.9490965791325757, 42.701) [b] 
(0.9491237058686869, 42.709) [b] 
(0.949150832604798, 42.71) [b] 
(0.9491779593409092, 42.711) [b] 
(0.9492050860770204, 42.712) [b] 
(0.9492322128131315, 42.713) [b] 
(0.9492329072575759, 43.34) [b] 
(0.9492353378131315, 45.33) [b] 
(0.9492367267020204, 45.377) [b] 
(0.9492388100353537, 45.705) [b] 
(0.9492516423934279, 48.05) [b] 
(0.9492617118378723, 48.142) [b] 
(0.949262753504539, 52.596) [b] 
(0.9492644896156501, 54.018) [b] 
(0.9492651840600945, 54.062) [b] 
(0.9492662257267612, 54.11) [b] 
(0.9492669201712056, 54.166) [b] 
(0.9492672673934278, 54.394) [b] 
(0.9492683090600945, 54.405) [b] 
(0.9492686562823167, 55.129) [b] 
(0.9492720344780899, 55.26) [b] 
(0.9492754126738632, 55.261) [b] 
(0.9492956818485024, 55.355) [b] 
(0.9492960290707246, 55.387) [b] 
(0.9493363256535744, 56.601) [b] 
(0.9493503642878961, 57.375) [b] 
(0.9493518294326876, 67.852) [b] 
(0.949391709736768, 71.64) [b] 
(0.9493944601025667, 71.642) [b] 
(0.9494013360170633, 71.644) [b] 
(0.949402023608513, 71.706) [b] 
(0.9494027111999627, 71.712) [b] 
(0.94940821193156, 71.803) [b] 
(0.9494088995230097, 71.817) [b] 
(0.9494095871144593, 71.88) [b] 
(0.9494116383171674, 72.552) [b] 
(0.9494130135000667, 72.739) [b] 
(0.94944109076871, 73.574) [b] 
(0.9494667475831599, 76.751) [b] 
(0.9494715607233075, 77.155) [b] 
(0.9494736234976565, 77.157) [b] 
(0.9494743110891062, 77.182) [b] 
(0.9494756862720055, 77.214) [b] 
(0.9494763738634552, 77.216) [b] 
(0.9494770614549048, 77.393) [b] 
(0.949477408677127, 78.594) [b] 
(0.9494780962685767, 78.856) [b] 
(0.9495037530830266, 79.963) [b] 
(0.9495047947496933, 80.101) [b] 
(0.9495304515641432, 81.151) [b] 
(0.9495561083785931, 84.736) [b] 
(0.9495626392040895, 86.265) [b] 
(0.9495726039005933, 93.173) [b] 
(0.9495739790834926, 94.094) [b] 
(0.9495746666749423, 94.1) [b] 
(0.949575354266392, 94.113) [b] 
(0.9495757014886141, 95.367) [b] 
(0.9495822323141105, 95.547) [b] 
(0.9496442362823645, 96.14) [b] 
(0.949654305726809, 99.657) [b] 
(0.9496550001712534, 99.692) [b] 
(0.9496622918379201, 99.704) [b] 
(0.9496626390601423, 100.001) [b] 
(0.9496671529490311, 100.112) [b] 
(0.9496678473934755, 100.163) [b] 
(0.9496699307268088, 100.298) [b] 
(0.9496709723934755, 100.314) [b] 
(0.9496723612823644, 110.485) [b] 
(0.9496757394781377, 113.2) [b] 
(0.9496791176739109, 113.251) [b] 
(0.949680853785022, 115.114) [b] 
(0.9496818954516887, 115.242) [b] 
(0.9496822426739109, 115.808) [b] 
(0.9496829371183553, 122.838) [b] 
(0.9496832843405775, 122.979) [b] 
(0.9496898029054502, 124.096) [b] 
(0.9496899494199293, 125.74) [b] 
(0.9498069906184312, 126.066) [b] 
(0.9498304780291791, 131.472) [b] 
(0.9498368942082162, 132.04) [b] 
(0.9498433103872532, 132.106) [b] 
(0.9498667977980011, 136.655) [b] 
(0.9498876311313345, 141.528) [b] 
(0.9498940473103715, 147.191) [b] 
(0.9498943945325937, 179.003) [b] 
(0.9498947417548159, 179.167) [b] 
(0.9498950889770381, 181.025) [b] 
(0.9499367556437048, 217.941) [b] 
(0.9499784223103716, 217.942) [b] 
(0.9499794639770383, 225.396) [b] 
(0.9499815473103715, 225.403) [b] 
(0.9499825889770382, 228.777) [b] 
(0.9499857139770382, 228.785) [b] 
(0.9499867556437049, 229.485) [b] 
(0.9499888389770382, 229.681) [b] 
(0.9499909223103715, 233.986) [b] 
(0.9500232139770382, 233.993) [b] 
(0.9500273806437048, 233.999) [b] 
(0.9500305056437048, 234.019) [b] 
(0.9500315473103715, 234.15) [b] 
(0.9500430056437049, 234.535) [b] 
(0.9500443808266043, 240.092) [b] 
(0.9500450684180539, 240.498) [b] 
(0.9500464436009532, 244.132) [b] 
(0.9500478187838526, 252.231) [b] 
(0.9500485063753022, 252.247) [b] 
(0.9501655475738041, 252.771) [b] 
(0.9501662351652538, 253.377) [b] 
(0.9501669296096982, 254.109) [b] 
(0.9501676240541426, 255.08) [b] 
(0.9501816626884643, 258.602) [b] 
(0.9501820099106865, 258.696) [b] 
(0.9501823571329087, 258.798) [b] 
(0.9501837460217976, 259.379) [b] 
(0.9501847876884643, 259.973) [b] 
(0.9501851349106865, 260.607) [b] 
(0.9501885131064597, 272.259) [b] 
(0.9502446676437463, 277.635) [b] 
(0.9502867835467113, 277.636) [b] 
(0.950300822181033, 277.761) [b] 
(0.9503148608153547, 278.051) [b] 
(0.9503429380839981, 278.936) [b] 
(0.9503569767183198, 278.937) [b] 
(0.9503710153526415, 278.938) [b] 
(0.9503751820193082, 291.534) [b] 
(0.9503821264637526, 291.604) [b] 
(0.950382820908197, 298.104) [b] 
(0.9504350675121678, 300.136) [b] 
(0.9504415983376642, 300.261) [b] 
(0.9504546599886569, 300.268) [b] 
(0.9504574377664347, 301.459) [b] 
(0.9504575842809138, 312.185) [b] 
(0.9504582718723634, 312.549) [b] 
(0.9504589594638131, 312.552) [b] 
(0.9504596470552628, 312.565) [b] 
(0.9504603346467124, 312.603) [b] 
(0.9504610222381621, 312.607) [b] 
(0.9504623974210614, 312.616) [b] 
(0.9504630850125111, 312.666) [b] 
(0.9504637726039608, 312.866) [b] 
(0.9504651614928497, 323.054) [b] 
(0.9504658490842993, 329.238) [b] 
(0.9505232784124438, 436.596) [b] 
(0.950526656608217, 436.599) [b] 
(0.9505300348039902, 436.6) [b] 
(0.9505334129997635, 436.609) [b] 
(0.9505367911955367, 436.639) [b] 
(0.950543547587083, 436.71) [b] 
(0.9505536821744026, 437.861) [b] 
(0.950560438565949, 437.862) [b] 
(0.9505671949574954, 437.865) [b] 
(0.9505807077405882, 437.87) [b] 
(0.9505840859363615, 437.871) [b] 
(0.9505874641321347, 437.893) [b] 
(0.9505908423279079, 438.047) [b] 
(0.9505942205236811, 438.428) [b] 
(0.9505975987194544, 438.434) [b] 
(0.9505996820527877, 439.151) [b] 
(0.9506007237194544, 440.49) [b] 
(0.9506017653861211, 442.144) [b] 
(0.9506028070527878, 447.696) [b] 
(0.9506038487194545, 448.585) [b] 
(0.950607668163899, 465.365) [b] 
(0.9506087098305657, 465.41) [b] 
(0.9506090570527879, 465.687) [b] 
(0.9506094042750101, 465.781) [b] 
(0.9506097514972323, 465.92) [b] 
(0.950610793163899, 468.379) [b] 
(0.9506142653861211, 485.763) [b] 
(0.9506173903861211, 492.234) [b] 
(0.9506194737194544, 509.321) [b] 
(0.9506201681638988, 509.438) [b] 
(0.9506208626083432, 509.604) [b] 
(0.9506443500190911, 514.772) [b] 
(0.950667837429839, 514.78) [b] 
(0.9506913248405868, 515.561) [b] 
(0.9507246581739202, 537.981) [b] 
(0.9507430609516979, 538.051) [b] 
(0.9507472276183646, 538.278) [b] 
(0.9507482692850313, 538.528) [b] 
(0.950749310951698, 538.645) [b] 
(0.9507503526183647, 541.781) [b] 
(0.9507517761464367, 543.206) [b] 
(0.9507625400353256, 557.117) [b] 
(0.95076323447977, 557.165) [b] 
(0.9507639289242144, 557.263) [b] 
(0.9507642761464365, 560.887) [b] 
(0.9507746928131032, 569.157) [b] 
(0.9507750400353254, 569.201) [b] 
(0.9507753872575476, 569.246) [b] 
(0.9507757344797698, 569.342) [b] 
(0.9507764289242142, 569.574) [b] 
(0.9507771233686586, 570.176) [b] 
(0.9507835395476957, 574.22) [b] 
(0.9507897895476957, 649.22) [b] 
(0.9507901367699179, 649.539) [b] 
(0.9507911784365846, 659.433) [b] 
(0.9509771903413464, 659.442) [b] 
(0.9510391943096004, 659.458) [b] 
(0.9511011982778543, 659.479) [b] 
(0.9511632022461083, 659.694) [b] 
(0.9512252062143622, 660.378) [b] 
(0.9512253527288413, 666.188) [b] 
(0.9512873566970953, 668.407) [b] 
(0.9513145440335105, 707.471) [b] 
(0.9513353773668438, 708.818) [b] 
(0.9513978773668438, 708.819) [b] 
(0.9514187107001771, 708.821) [b] 
(0.9514603773668439, 708.822) [b] 
(0.9514812107001772, 708.823) [b] 
(0.9515228773668439, 708.824) [b] 
(0.9515437107001773, 708.825) [b] 
(0.9515645440335107, 708.826) [b] 
(0.9516687107001773, 708.828) [b] 
(0.9519187107001773, 708.829) [b] 
(0.951960377366844, 708.843) [b] 
(0.952022877366844, 708.849) [b] 
(0.9520437107001773, 708.975) [b] 
(0.9520853773668441, 709.105) [b] 
(0.9521270440335108, 709.107) [b] 
(0.9521478773668441, 709.114) [b] 
(0.9521687107001775, 709.122) [b] 
(0.9521895440335109, 709.449) [b] 
(0.9522103773668442, 710.659) [b] 
(0.9522312107001776, 712.973) [b] 
(0.9524395440335109, 716.816) [b] 
(0.9524603773668443, 716.818) [b] 
(0.9524812107001777, 716.819) [b] 
(0.9525437107001776, 716.824) [b] 
(0.952564544033511, 716.835) [b] 
(0.9526062107001777, 716.932) [b] 
(0.9526270440335111, 717.233) [b] 
(0.9526478773668444, 717.236) [b] 
(0.9526687107001778, 717.252) [b] 
(0.9527248652374644, 727.118) [b] 
(0.9527389038717861, 727.121) [b] 
(0.9527529425061078, 727.125) [b] 
(0.9527669811404295, 727.146) [b] 
(0.9527810197747512, 727.341) [b] 
(0.9527950584090729, 733.571) [b] 
(0.9528090970433946, 733.595) [b] 
(0.9528371743120378, 733.629) [b] 
(0.9528512129463595, 733.723) [b] 
(0.952879290215003, 733.725) [b] 
(0.9528933288493247, 733.739) [b] 
(0.9529073674836464, 734.052) [b] 
(0.952921406117968, 735.458) [b] 
(0.9529354447522898, 735.461) [b] 
(0.9529455824031564, 740.537) [b] 
(0.9529664157364898, 742.664) [b] 
(0.9529671101809342, 772.5) [b] 
(0.9529811488152559, 801.723) [b] 
(0.9530981900137577, 860.191) [b] 
(0.9531312587968265, 929.138) [b] 
(0.9531316060190487, 1050.32) [b] 
(0.9533399393523819, 1050.75) [b] 
(0.9533607726857153, 1050.76) [b] 
(0.9533816060190486, 1050.81) [b] 
(0.9534649393523821, 1050.91) [b] 
(0.9534714579172548, 1069.3) [b] 
(0.9534721523616992, 1257.37) [b] 
(0.9534735412505881, 1257.39) [b] 
(0.9534742356950325, 1257.51) [b] 
(0.9534749301394769, 1258.19) [b] 
(0.9534790968061436, 1260.48) [b] 
(0.953479791250588, 1260.51) [b] 
(0.9534811801394769, 1261.34) [b] 
(0.953482916250588, 1262.24) [b] 
(0.9534870829172547, 1265.25) [b] 
(0.9534905551394769, 1268.4) [b] 
(0.9534909023616991, 1269.02) [b] 
(0.9534936801394769, 1272.6) [b] 
(0.9534964579172547, 1273.96) [b] 
(0.9534992356950325, 1273.97) [b] 
(0.9534999301394769, 1274.23) [b] 
(0.9535020134728102, 1422.47) [b] 
(0.9535037495839213, 1422.52) [b] 
(0.9535040968061435, 1422.8) [b] 
(0.9535044440283656, 1524.53) [b] 
(0.95350513847281, 1527.69) [b] 
(0.9535054856950322, 1530.48) [b] 
(0.9535157995667772, 1572.35) [b] 
(0.9535178623411262, 1572.36) [b] 
(0.9535192375240256, 1572.38) [b] 
(0.9535247382556229, 1572.43) [b] 
(0.9535254258470726, 1572.89) [b] 
(0.9535268010299719, 1572.91) [b] 
(0.9535281762128712, 1573.03) [b] 
(0.9535295513957706, 1574.68) [b] 
(0.9535302389872202, 1574.82) [b] 
(0.9535316141701196, 1576.88) [b] 
(0.9535336769444686, 1576.96) [b] 
(0.9535350521273679, 1577.02) [b] 
(0.9535357397188176, 1577.11) [b] 
(0.9535364273102672, 1577.12) [b] 
(0.9535371149017169, 1585.43) [b] 
(0.9535378024931666, 1608.18) [b] 
(0.9535391776760659, 1618.95) [b] 
(0.9535398652675156, 1618.96) [b] 
(0.9535488930452933, 1624.29) [b] 
(0.9535558374897378, 1624.35) [b] 
(0.9535589624897378, 1624.41) [b] 
(0.9535596569341822, 1624.68) [b] 
(0.9535600041564044, 1626.46) [b] 
(0.9535613793393037, 1687.77) [b] 
(0.9535627545222031, 1692.22) [b] 
(0.9535661924794514, 1695.45) [b] 
(0.9535682552538004, 1705.94) [b] 
(0.9535689496982448, 1880.42) [b] 
(0.9535696372896945, 2210.81) [b] 
(0.9535760534687315, 2316.67) [b] 
(0.9535805673576204, 2350.59) [b] 
(0.9535826506909537, 2350.65) [b] 
(0.9535840395798426, 2351.04) [b] 
(0.9535843868020648, 2351.41) [b] 
(0.9535850812465092, 2352.64) [b] 
(0.9535857756909536, 2362.13) [b] 
(0.9536274423576203, 2415.05) [b] 
(0.9536482756909537, 2415.06) [b] 
(0.9537524423576204, 2415.14) [b] 
(0.9537732756909537, 2415.5) [b] 
(0.9538149423576204, 2467.39) [b] 
(0.9538357756909538, 2469.27) [b] 
(0.9538389006909538, 2491.86) [b] 
(0.9538441090242871, 2491.87) [b] 
(0.9538451506909538, 2491.89) [b] 
(0.9538576506909539, 2491.9) [b] 
(0.9538597340242873, 2492.54) [b] 
(0.953860775690954, 2499.04) [b] 
(0.9538618173576207, 2502.31) [b] 
(0.9538628590242874, 2834.41) [b] 
(0.9538662372200606, 2975.22) [b] 
(0.9538897246308085, 3146.19) [b] 
(0.9540067658293103, 3236.51) [b] 
(0.9541238070278122, 3237.01) [b] 
(0.9541269320278122, 3240.51) [b] 
(0.9541283209167011, 3243.64) [b] 
(0.9541307514722567, 3275.1) [b] 
(0.9541317931389234, 3316.39) [b] 
(0.9541328348055901, 3368.6) [b] 
(0.9541362130013633, 3437.02) [b] 
(0.9541463475886829, 3437.03) [b] 
(0.9541531039802293, 3437.14) [b] 
(0.9541666167633223, 3438.11) [b] 
(0.9541699949590955, 3438.12) [b] 
(0.9541801295464151, 3438.21) [b] 
(0.9541835077421883, 3452.19) [b] 
(0.9541902641337348, 3452.37) [b] 
(0.954193642329508, 3529.61) [b] 
(0.9541970205252812, 3531.58) [b] 
},{(0.9475928541666668, 0.001) [c] 
(0.9475928541666668, 3.9281546249999995) [c] 
(0.9475928541666668, 3600) [c] 
}}}{legend pos=north west}}
% 	\subfloat[depth=10]{\cactus{Average Accuracy}{CPU time}{\budalg, \murtree, \cart}{{{(0.9292995559338405, 0) [a] 
(0.9348699726005072, 0.01) [a] 
(0.9371491392671739, 0.02) [a] 
(0.9398039309338403, 0.03) [a] 
(0.9457276809338403, 0.04) [a] 
(0.947646361489396, 0.05) [a] 
(0.9495583059338403, 0.06) [a] 
(0.9495697642671735, 0.07) [a] 
(0.9495922642671736, 0.08) [a] 
(0.9497424726005064, 0.09) [a] 
(0.9497997642671728, 0.1) [a] 
(0.9498235142671729, 0.11) [a] 
(0.9498253892671729, 0.12) [a] 
(0.9498305976005061, 0.13) [a] 
(0.9498778892671728, 0.14) [a] 
(0.949881222600506, 0.15) [a] 
(0.9498855976005061, 0.16) [a] 
(0.9498891392671727, 0.17) [a] 
(0.9498905976005059, 0.18) [a] 
(0.9498918476005058, 0.19) [a] 
(0.9498922642671724, 0.2) [a] 
(0.9498943476005057, 0.21) [a] 
(0.9499464309338391, 0.22) [a] 
(0.9499978892671725, 0.23) [a] 
(0.9500114309338391, 0.24) [a] 
(0.9500187226005058, 0.25) [a] 
(0.9500439309338391, 0.26) [a] 
(0.9501235142671723, 0.27) [a] 
(0.9502016392671724, 0.28) [a] 
(0.9502266392671725, 0.3) [a] 
(0.9539715411860884, 0.31) [a] 
(0.9539765411860884, 0.32) [a] 
(0.9540000828527552, 0.33) [a] 
(0.9540223745194217, 0.35) [a] 
(0.9540234161860883, 0.36) [a] 
(0.9540244578527549, 0.37) [a] 
(0.9540532078527548, 0.38) [a] 
(0.9541394578527546, 0.39) [a] 
(0.9541821661860879, 0.4) [a] 
(0.954254874519421, 0.41) [a] 
(0.9544021661860874, 0.42) [a] 
(0.9544848745194205, 0.43) [a] 
(0.9545588328527538, 0.44) [a] 
(0.9546357078527538, 0.45) [a] 
(0.954649666186087, 0.47) [a] 
(0.9546513328527537, 0.48) [a] 
(0.9590826082378523, 0.49) [a] 
(0.9590838582378524, 0.5) [a] 
(0.9591582332378522, 0.51) [a] 
(0.9592705249045188, 0.52) [a] 
(0.9593134415711854, 0.55) [a] 
(0.9593144832378521, 0.57) [a] 
(0.9593688582378521, 0.58) [a] 
(0.9593988582378521, 0.59) [a] 
(0.9594534415711855, 0.6) [a] 
(0.9594546915711855, 0.61) [a] 
(0.9594567749045187, 0.62) [a] 
(0.9594573999045187, 0.63) [a] 
(0.9594582332378521, 0.64) [a] 
(0.9594594832378519, 0.65) [a] 
(0.9595767749045186, 0.66) [a] 
(0.959604483237852, 0.67) [a] 
(0.9596065665711853, 0.71) [a] 
(0.9596071915711852, 0.72) [a] 
(0.9597278165711853, 0.73) [a] 
(0.959728233237852, 0.81) [a] 
(0.9597288582378519, 0.82) [a] 
(0.9597296915711852, 0.84) [a] 
(0.9597436499045184, 0.85) [a] 
(0.9597438582378518, 0.88) [a] 
(0.9597440665711852, 0.9) [a] 
(0.9597863582378519, 0.91) [a] 
(0.9597867749045185, 0.93) [a] 
(0.9597878165711852, 0.95) [a] 
(0.9598080249045184, 0.96) [a] 
(0.9598084415711851, 0.98) [a] 
(0.9598101082378517, 0.99) [a] 
(0.959811566571185, 1) [a] 
(0.9598196915711851, 1.01) [a] 
(0.9607527083333299, 1.13) [a] 
(0.9610354166666633, 1.14) [a] 
(0.9610358333333299, 1.16) [a] 
(0.9610660416666631, 1.17) [a] 
(0.9610799999999964, 1.24) [a] 
(0.9611362499999963, 1.28) [a] 
(0.9611502083333295, 1.29) [a] 
(0.9611645833333295, 1.39) [a] 
(0.9611924999999961, 1.4) [a] 
(0.9612066666666628, 1.41) [a] 
(0.961220624999996, 1.42) [a] 
(0.9612270833333294, 1.46) [a] 
(0.9612347916666627, 1.47) [a] 
(0.961248749999996, 1.48) [a] 
(0.9612552083333293, 1.52) [a] 
(0.961256249999996, 1.72) [a] 
(0.9612570833333294, 1.77) [a] 
(0.9612577083333294, 1.78) [a] 
(0.9612608333333293, 1.81) [a] 
(0.9612672916666627, 1.82) [a] 
(0.961269374999996, 1.93) [a] 
(0.9612741666666625, 1.94) [a] 
(0.9612747916666625, 1.95) [a] 
(0.9612752083333291, 1.98) [a] 
(0.9613008333333292, 2) [a] 
(0.9613020833333291, 2.23) [a] 
(0.9613277083333291, 2.28) [a] 
(0.9613533333333292, 2.51) [a] 
(0.9613789583333292, 2.69) [a] 
(0.9613793749999958, 2.82) [a] 
(0.9613797916666624, 2.89) [a] 
(0.9614214583333291, 2.93) [a] 
(0.9614631249999959, 2.94) [a] 
(0.9614839583333292, 2.96) [a] 
(0.9614841666666626, 2.98) [a] 
(0.9615260416666627, 2.99) [a] 
(0.961546874999996, 3) [a] 
(0.9615677083333294, 3.02) [a] 
(0.961568124999996, 3.11) [a] 
(0.9615916666666626, 3.4) [a] 
(0.961592499999996, 3.57) [a] 
(0.9615935416666627, 3.59) [a] 
(0.9615947916666626, 3.6) [a] 
(0.961615624999996, 3.75) [a] 
(0.9616170833333293, 3.8) [a] 
(0.9616174999999959, 4.01) [a] 
(0.9616177083333293, 4.02) [a] 
(0.9616179166666626, 4.24) [a] 
(0.961618749999996, 4.53) [a] 
(0.961619374999996, 4.67) [a] 
(0.9616195833333293, 4.85) [a] 
(0.961626249999996, 5.16) [a] 
(0.9616520833333293, 6.25) [a] 
(0.9616585416666626, 6.35) [a] 
(0.9616843749999959, 6.8) [a] 
(0.961709999999996, 7.19) [a] 
(0.9617114583333293, 7.42) [a] 
(0.9617137499999959, 7.44) [a] 
(0.9617202083333293, 7.7) [a] 
(0.9617204166666626, 7.76) [a] 
(0.9617210416666626, 7.8) [a] 
(0.9617220833333293, 7.85) [a] 
(0.9617477083333293, 7.86) [a] 
(0.9617483333333293, 7.87) [a] 
(0.9617487499999959, 7.89) [a] 
(0.9617491666666625, 7.91) [a] 
(0.9617493749999959, 7.92) [a] 
(0.9617495833333293, 8.15) [a] 
(0.9617497916666626, 8.18) [a] 
(0.961749999999996, 8.25) [a] 
(0.9617504166666626, 8.43) [a] 
(0.9617508333333292, 8.45) [a] 
(0.9618133333333293, 9.04) [a] 
(0.961854999999996, 9.25) [a] 
(0.9618758333333294, 9.28) [a] 
(0.961876249999996, 9.65) [a] 
(0.961876874999996, 9.97) [a] 
(0.9618770833333293, 9.98) [a] 
(0.9618772916666627, 10.05) [a] 
(0.9618914583333293, 10.51) [a] 
(0.9618920833333293, 10.56) [a] 
(0.9618924999999959, 10.59) [a] 
(0.9618929166666625, 11.34) [a] 
(0.9618931249999959, 11.63) [a] 
(0.9618935416666625, 11.64) [a] 
(0.9618997916666625, 12.1) [a] 
(0.9619137499999958, 12.25) [a] 
(0.9619158333333291, 12.5) [a] 
(0.9619191666666624, 12.55) [a] 
(0.961919583333329, 12.56) [a] 
(0.961920208333329, 12.58) [a] 
(0.9619206249999956, 12.59) [a] 
(0.9619210416666623, 12.67) [a] 
(0.9619214583333289, 12.74) [a] 
(0.9619216666666622, 12.84) [a] 
(0.9619231249999955, 13.64) [a] 
(0.9619241666666623, 13.68) [a] 
(0.9619243749999956, 13.69) [a] 
(0.9619247916666622, 13.7) [a] 
(0.9619252083333288, 13.71) [a] 
(0.9619254166666622, 13.73) [a] 
(0.9619264583333289, 13.75) [a] 
(0.9619274999999956, 13.76) [a] 
(0.9619283333333289, 13.79) [a] 
(0.9619293749999956, 13.84) [a] 
(0.961930208333329, 14.01) [a] 
(0.961937708333329, 14.02) [a] 
(0.9620547916666624, 14.57) [a] 
(0.962055833333329, 14.96) [a] 
(0.9620793749999956, 14.98) [a] 
(0.9620808333333289, 14.99) [a] 
(0.9620810416666623, 15.02) [a] 
(0.9620814583333289, 15.03) [a] 
(0.9620818749999955, 15.05) [a] 
(0.9620820833333289, 15.07) [a] 
(0.9620824999999955, 15.27) [a] 
(0.9620827083333289, 15.6) [a] 
(0.9620835416666621, 15.61) [a] 
(0.9620837499999955, 15.63) [a] 
(0.9620841666666621, 15.64) [a] 
(0.9620845833333287, 15.66) [a] 
(0.962084791666662, 15.69) [a] 
(0.9620852083333287, 15.78) [a] 
(0.9620856249999953, 15.83) [a] 
(0.9620860416666619, 15.84) [a] 
(0.9620866666666619, 15.9) [a] 
(0.9620870833333285, 15.93) [a] 
(0.9620872916666618, 16.32) [a] 
(0.9620874999999952, 16.42) [a] 
(0.9620879166666618, 17.08) [a] 
(0.9621020833333285, 17.27) [a] 
(0.9621029166666618, 17.49) [a] 
(0.9621237499999952, 17.54) [a] 
(0.9621445833333285, 17.95) [a] 
(0.9621654166666619, 18.01) [a] 
(0.9621862499999952, 18.38) [a] 
(0.9622070833333286, 18.49) [a] 
(0.9622074999999952, 18.92) [a] 
(0.9622081249999952, 18.93) [a] 
(0.9622316666666618, 18.94) [a] 
(0.9622456249999951, 19.11) [a] 
(0.9622458333333285, 19.12) [a] 
(0.9622597916666618, 19.14) [a] 
(0.9622739583333284, 19.98) [a] 
(0.9622741666666618, 20.12) [a] 
(0.962288124999995, 20.57) [a] 
(0.9623114583333284, 21.17) [a] 
(0.9623322916666618, 21.72) [a] 
(0.9623531249999951, 21.74) [a] 
(0.9623739583333285, 21.75) [a] 
(0.9623745833333285, 21.87) [a] 
(0.9623752083333285, 21.96) [a] 
(0.9623768749999951, 22.08) [a] 
(0.9623772916666617, 22.09) [a] 
(0.9623787499999951, 22.14) [a] 
(0.9623789583333284, 22.22) [a] 
(0.962379374999995, 22.26) [a] 
(0.9624049999999951, 22.62) [a] 
(0.9624054166666617, 23.51) [a] 
(0.9624060416666617, 23.52) [a] 
(0.9624316666666617, 25.06) [a] 
(0.962432499999995, 26.38) [a] 
(0.9624389583333284, 26.52) [a] 
(0.9624395833333284, 26.68) [a] 
(0.9624402083333283, 26.7) [a] 
(0.9624410416666617, 26.83) [a] 
(0.962461874999995, 26.94) [a] 
(0.962463124999995, 27.04) [a] 
(0.9624839583333283, 27.29) [a] 
(0.9624841666666617, 27.34) [a] 
(0.9625049999999951, 28.85) [a] 
(0.9625114583333284, 29.25) [a] 
(0.9625181249999951, 30.15) [a] 
(0.9625185416666617, 30.6) [a] 
(0.9625189583333283, 31.29) [a] 
(0.9625191666666617, 31.3) [a] 
(0.962520624999995, 31.31) [a] 
(0.9625222916666617, 31.32) [a] 
(0.9625227083333283, 31.33) [a] 
(0.9625231249999949, 31.81) [a] 
(0.9625241666666616, 32.47) [a] 
(0.9625245833333282, 33.02) [a] 
(0.9625256249999949, 33.04) [a] 
(0.9625262499999949, 33.05) [a] 
(0.9625270833333283, 33.11) [a] 
(0.9625272916666616, 34.13) [a] 
(0.9625277083333282, 34.14) [a] 
(0.9625287499999948, 34.16) [a] 
(0.9625291666666614, 34.17) [a] 
(0.9625299999999948, 34.68) [a] 
(0.9625306249999948, 34.69) [a] 
(0.9625466666666613, 34.82) [a] 
(0.9625479166666613, 34.85) [a] 
(0.9625508333333279, 34.86) [a] 
(0.9625520833333279, 35.15) [a] 
(0.9625556249999945, 35.2) [a] 
(0.9625631249999944, 35.41) [a] 
(0.9625645833333277, 35.43) [a] 
(0.9625652083333277, 35.44) [a] 
(0.9625662499999943, 36.31) [a] 
(0.9625668749999943, 36.32) [a] 
(0.9626839583333276, 36.51) [a] 
(0.9626843749999943, 36.85) [a] 
(0.9626858333333276, 36.92) [a] 
(0.9627093749999942, 38.44) [a] 
(0.9627329166666608, 38.47) [a] 
(0.9627537499999942, 38.72) [a] 
(0.9627745833333275, 38.77) [a] 
(0.9628058333333276, 39.36) [a] 
(0.9628097916666609, 39.37) [a] 
(0.9628139583333276, 40.91) [a] 
(0.962814166666661, 41.26) [a] 
(0.9628145833333276, 41.36) [a] 
(0.9628149999999942, 41.37) [a] 
(0.9628152083333276, 41.4) [a] 
(0.9628156249999942, 41.42) [a] 
(0.9628158333333275, 41.8) [a] 
(0.9628162499999942, 41.81) [a] 
(0.9628166666666608, 41.82) [a] 
(0.9628168749999941, 41.84) [a] 
(0.9628172916666607, 41.86) [a] 
(0.9628408333333274, 42.01) [a] 
(0.962841249999994, 42.07) [a] 
(0.9628414583333273, 42.13) [a] 
(0.962841874999994, 42.24) [a] 
(0.9628422916666606, 42.29) [a] 
(0.9628427083333272, 43.36) [a] 
(0.9628447916666605, 46.76) [a] 
(0.9628454166666605, 46.77) [a] 
(0.9628481249999937, 46.78) [a] 
(0.962849583333327, 46.79) [a] 
(0.9628510416666604, 46.8) [a] 
(0.9628516666666603, 46.82) [a] 
(0.9628522916666603, 46.83) [a] 
(0.9628531249999936, 46.88) [a] 
(0.9628543749999936, 46.9) [a] 
(0.9628579166666602, 46.95) [a] 
(0.9628585416666602, 46.96) [a] 
(0.9628606249999935, 46.98) [a] 
(0.9628608333333268, 47.03) [a] 
(0.9628616666666602, 47.04) [a] 
(0.9628622916666602, 47.07) [a] 
(0.9628637499999935, 47.1) [a] 
(0.9628714583333268, 47.11) [a] 
(0.9628762499999933, 47.12) [a] 
(0.9628789583333266, 47.13) [a] 
(0.9628797916666599, 47.14) [a] 
(0.9628810416666599, 47.15) [a] 
(0.9628839583333265, 47.16) [a] 
(0.9628872916666598, 47.17) [a] 
(0.9628902083333264, 47.18) [a] 
(0.9628922916666597, 47.19) [a] 
(0.9628935416666596, 47.23) [a] 
(0.9628956249999929, 47.24) [a] 
(0.9628997916666595, 47.25) [a] 
(0.9629047916666594, 47.26) [a] 
(0.9629054166666594, 47.29) [a] 
(0.9629060416666594, 47.3) [a] 
(0.9629068749999927, 47.44) [a] 
(0.9629085416666593, 47.46) [a] 
(0.9629093749999926, 47.48) [a] 
(0.9629114583333259, 47.49) [a] 
(0.9629135416666592, 47.5) [a] 
(0.9629162499999925, 47.51) [a] 
(0.9629189583333257, 47.52) [a] 
(0.9629218749999924, 47.53) [a] 
(0.9629224999999924, 47.54) [a] 
(0.9629231249999923, 47.55) [a] 
(0.9629239583333257, 47.58) [a] 
(0.9629245833333256, 47.7) [a] 
(0.9629289583333256, 47.71) [a] 
(0.9629299999999923, 47.72) [a] 
(0.9629308333333256, 47.78) [a] 
(0.9629314583333256, 47.79) [a] 
(0.962931666666659, 47.95) [a] 
(0.9629337499999923, 48.14) [a] 
(0.9629358333333256, 48.15) [a] 
(0.9629379166666588, 48.16) [a] 
(0.9629408333333255, 48.17) [a] 
(0.9629414583333255, 48.18) [a] 
(0.9629429166666588, 48.2) [a] 
(0.9629435416666587, 48.21) [a] 
(0.9629470833333253, 48.22) [a] 
(0.9629491666666586, 48.23) [a] 
(0.9629497916666586, 48.24) [a] 
(0.9629504166666586, 48.25) [a] 
(0.9629518749999919, 48.26) [a] 
(0.9629524999999919, 48.27) [a] 
(0.9629533333333252, 48.5) [a] 
(0.9629554166666585, 49.16) [a] 
(0.9629560416666585, 49.17) [a] 
(0.9629566666666585, 49.18) [a] 
(0.9629581249999918, 49.19) [a] 
(0.9629587499999918, 49.38) [a] 
(0.962960833333325, 49.79) [a] 
(0.9629629166666583, 49.8) [a] 
(0.962965833333325, 49.81) [a] 
(0.9629685416666582, 49.82) [a] 
(0.9629712499999915, 49.83) [a] 
(0.9629720833333248, 49.85) [a] 
(0.9629727083333248, 49.91) [a] 
(0.9629733333333248, 49.94) [a] 
(0.9629747916666581, 50.2) [a] 
(0.9629754166666581, 50.21) [a] 
(0.9629774999999914, 50.37) [a] 
(0.9629783333333247, 50.55) [a] 
(0.9629795833333247, 50.56) [a] 
(0.962981666666658, 50.6) [a] 
(0.9629824999999913, 50.61) [a] 
(0.9629831249999913, 50.75) [a] 
(0.9629845833333246, 50.92) [a] 
(0.9629899999999912, 50.93) [a] 
(0.9629914583333246, 50.94) [a] 
(0.9629949999999912, 50.95) [a] 
(0.9629970833333245, 50.96) [a] 
(0.9629977083333244, 50.97) [a] 
(0.9629997916666577, 50.98) [a] 
(0.963001874999991, 50.99) [a] 
(0.963002499999991, 51) [a] 
(0.9630033333333243, 51.11) [a] 
(0.9630045833333243, 51.22) [a] 
(0.9630054166666576, 52.17) [a] 
(0.9630060416666576, 52.18) [a] 
(0.9630066666666576, 52.23) [a] 
(0.9630072916666575, 52.45) [a] 
(0.9630077083333242, 52.46) [a] 
(0.9630079166666575, 52.47) [a] 
(0.9630104166666575, 52.5) [a] 
(0.9630110416666575, 52.56) [a] 
(0.9630114583333241, 52.57) [a] 
(0.9630118749999907, 52.58) [a] 
(0.9630120833333241, 52.61) [a] 
(0.9630129166666573, 52.62) [a] 
(0.9630131249999907, 52.63) [a] 
(0.9630135416666573, 52.64) [a] 
(0.9630139583333239, 52.7) [a] 
(0.9630145833333239, 52.73) [a] 
(0.9630160416666572, 52.86) [a] 
(0.9630195833333238, 52.88) [a] 
(0.9630202083333238, 52.89) [a] 
(0.9630222916666571, 52.9) [a] 
(0.9630224999999905, 53.03) [a] 
(0.9630231249999904, 53.41) [a] 
(0.9630237499999904, 53.62) [a] 
(0.963024166666657, 53.65) [a] 
(0.9630245833333236, 53.72) [a] 
(0.963025416666657, 53.86) [a] 
(0.9630258333333236, 53.87) [a] 
(0.963026041666657, 54.22) [a] 
(0.9630264583333236, 54.23) [a] 
(0.9630270833333235, 54.24) [a] 
(0.9630274999999902, 54.61) [a] 
(0.9630281249999901, 56.38) [a] 
(0.9630285416666567, 56.41) [a] 
(0.9630295833333233, 56.48) [a] 
(0.96302999999999, 56.5) [a] 
(0.9630302083333233, 56.52) [a] 
(0.9630306249999899, 56.55) [a] 
(0.9630310416666565, 56.59) [a] 
(0.9630312499999899, 56.6) [a] 
(0.9630318749999899, 56.61) [a] 
(0.9630335416666564, 56.63) [a] 
(0.9630337499999898, 56.66) [a] 
(0.9630352083333231, 56.72) [a] 
(0.9630356249999897, 56.75) [a] 
(0.9630358333333231, 56.76) [a] 
(0.9630362499999897, 56.79) [a] 
(0.9630366666666563, 56.81) [a] 
(0.9630368749999897, 56.87) [a] 
(0.9630372916666563, 56.88) [a] 
(0.9630377083333229, 56.9) [a] 
(0.9630379166666563, 56.92) [a] 
(0.9630383333333229, 56.94) [a] 
(0.9630387499999895, 56.96) [a] 
(0.9630389583333229, 57.06) [a] 
(0.9630393749999895, 57.07) [a] 
(0.9630397916666561, 57.11) [a] 
(0.9630404166666561, 57.47) [a] 
(0.9630412499999894, 57.59) [a] 
(0.9630424999999894, 57.69) [a] 
(0.9630431249999893, 57.83) [a] 
(0.9630433333333227, 58.96) [a] 
(0.9630574999999894, 59.49) [a] 
(0.9630808333333227, 59.52) [a] 
(0.963094791666656, 60.17) [a] 
(0.9632118749999894, 60.25) [a] 
(0.9632131249999893, 61.12) [a] 
(0.9632139583333227, 61.17) [a] 
(0.9632152083333226, 61.26) [a] 
(0.9632170833333227, 61.38) [a] 
(0.963217916666656, 64.28) [a] 
(0.9632193749999893, 64.4) [a] 
(0.9632199999999893, 64.65) [a] 
(0.9632204166666559, 64.87) [a] 
(0.9632210416666559, 65.25) [a] 
(0.9632216666666559, 65.47) [a] 
(0.9632262499999892, 65.71) [a] 
(0.9632264583333225, 65.77) [a] 
(0.9632268749999892, 67.87) [a] 
(0.9632274999999891, 67.92) [a] 
(0.9632281249999891, 67.95) [a] 
(0.9632285416666557, 67.98) [a] 
(0.9632289583333223, 68.06) [a] 
(0.9632291666666557, 68.07) [a] 
(0.9632295833333223, 68.09) [a] 
(0.9632302083333223, 68.1) [a] 
(0.9632510416666556, 69.19) [a] 
(0.963251249999989, 70.34) [a] 
(0.9632516666666556, 70.35) [a] 
(0.9632520833333222, 70.49) [a] 
(0.9632522916666556, 70.51) [a] 
(0.9632527083333222, 70.61) [a] 
(0.9632531249999888, 70.62) [a] 
(0.9632533333333222, 70.69) [a] 
(0.9632537499999888, 70.73) [a] 
(0.9632541666666554, 70.8) [a] 
(0.9632543749999888, 70.95) [a] 
(0.9632547916666554, 70.96) [a] 
(0.963259583333322, 72.14) [a] 
(0.9632610416666553, 72.15) [a] 
(0.9632616666666552, 72.16) [a] 
(0.9632620833333219, 72.28) [a] 
(0.9632627083333218, 72.29) [a] 
(0.9632631249999885, 72.37) [a] 
(0.9632637499999884, 72.41) [a] 
(0.963267291666655, 72.84) [a] 
(0.9632706249999884, 72.86) [a] 
(0.963276666666655, 73.37) [a] 
(0.9632793749999884, 73.5) [a] 
(0.9632810416666551, 73.6) [a] 
(0.9632820833333218, 73.61) [a] 
(0.9632824999999884, 73.62) [a] 
(0.963282916666655, 73.9) [a] 
(0.9632833333333216, 73.91) [a] 
(0.9632972916666549, 75.71) [a] 
(0.9633114583333215, 75.74) [a] 
(0.9633124999999882, 79.22) [a] 
(0.9633127083333216, 84.2) [a] 
(0.9633131249999882, 84.41) [a] 
(0.9633135416666548, 84.53) [a] 
(0.9633137499999882, 84.6) [a] 
(0.9633141666666548, 84.65) [a] 
(0.9633147916666548, 84.69) [a] 
(0.9633383333333214, 84.99) [a] 
(0.963338749999988, 85.18) [a] 
(0.963385624999988, 85.29) [a] 
(0.9633995833333213, 85.81) [a] 
(0.9633997916666547, 90.67) [a] 
(0.9634002083333213, 92.4) [a] 
(0.9634004166666547, 93.33) [a] 
(0.9634008333333213, 93.34) [a] 
(0.9634014583333212, 93.35) [a] 
(0.9634018749999879, 93.38) [a] 
(0.9634022916666545, 93.43) [a] 
(0.9634024999999878, 93.44) [a] 
(0.9634029166666545, 93.46) [a] 
(0.9634033333333211, 93.5) [a] 
(0.963403958333321, 93.53) [a] 
(0.9634049999999877, 93.55) [a] 
(0.9634054166666544, 93.59) [a] 
(0.9634056249999877, 93.74) [a] 
(0.9634064583333211, 95.65) [a] 
(0.963407083333321, 95.68) [a] 
(0.9634074999999876, 95.82) [a] 
(0.963407708333321, 95.93) [a] 
(0.9634081249999876, 95.96) [a] 
(0.9634085416666542, 95.97) [a] 
(0.9634097916666543, 96.08) [a] 
(0.9634102083333209, 96.1) [a] 
(0.9634112499999875, 96.22) [a] 
(0.9634116666666541, 96.23) [a] 
(0.9634133333333208, 96.36) [a] 
(0.9634137499999874, 96.37) [a] 
(0.9634139583333208, 96.45) [a] 
(0.9634141666666541, 102.04) [a] 
(0.9634145833333208, 102.71) [a] 
(0.9634149999999874, 103.28) [a] 
(0.963415416666654, 103.29) [a] 
(0.963416041666654, 103.45) [a] 
(0.9634164583333206, 103.46) [a] 
(0.9634166666666539, 103.48) [a] 
(0.9634172916666539, 103.92) [a] 
(0.9634177083333205, 106.42) [a] 
(0.9634181249999871, 107.7) [a] 
(0.9634185416666537, 107.71) [a] 
(0.963420624999987, 111.04) [a] 
(0.9634245833333204, 111.05) [a] 
(0.963424999999987, 111.12) [a] 
(0.9634254166666536, 111.13) [a] 
(0.9634260416666536, 111.15) [a] 
(0.9634266666666536, 111.3) [a] 
(0.9634272916666535, 118.7) [a] 
(0.9634277083333201, 118.74) [a] 
(0.9634416666666534, 122.51) [a] 
(0.9634558333333201, 124.02) [a] 
(0.9634697916666534, 124.03) [a] 
(0.96347020833332, 124.34) [a] 
(0.9634712499999867, 124.38) [a] 
(0.96347583333332, 124.41) [a] 
(0.9634760416666533, 124.76) [a] 
(0.9634774999999867, 124.81) [a] 
(0.96347895833332, 125.57) [a] 
(0.96347958333332, 127.32) [a] 
(0.9634799999999866, 128.91) [a] 
(0.9635035416666532, 134.14) [a] 
(0.9635056249999866, 144.24) [a] 
(0.9635060416666532, 145.38) [a] 
(0.9635066666666532, 145.4) [a] 
(0.9635070833333198, 145.41) [a] 
(0.9635072916666532, 146.63) [a] 
(0.9635077083333198, 146.64) [a] 
(0.9635083333333198, 146.65) [a] 
(0.9635318749999864, 147.1) [a] 
(0.9635335416666531, 147.5) [a] 
(0.963534166666653, 147.52) [a] 
(0.9635345833333196, 147.85) [a] 
(0.9635349999999863, 147.91) [a] 
(0.9635356249999862, 147.94) [a] 
(0.9635360416666529, 147.95) [a] 
(0.9635362499999862, 147.96) [a] 
(0.9635370833333196, 148.11) [a] 
(0.9635404166666529, 148.15) [a] 
(0.9635408333333195, 148.24) [a] 
(0.9635412499999861, 153.85) [a] 
(0.9635414583333195, 162.61) [a] 
(0.9635418749999861, 162.62) [a] 
(0.9635422916666527, 162.68) [a] 
(0.9635424999999861, 162.69) [a] 
(0.9635429166666527, 162.96) [a] 
(0.9635433333333193, 163.66) [a] 
(0.9635437499999859, 163.68) [a] 
(0.9635441666666525, 163.9) [a] 
(0.9635443749999859, 164.71) [a] 
(0.9635447916666525, 165.34) [a] 
(0.9635454166666525, 165.37) [a] 
(0.9635458333333191, 165.4) [a] 
(0.9635462499999857, 166.08) [a] 
(0.9635464583333191, 168.94) [a] 
(0.9635491666666525, 182.65) [a] 
(0.9635529166666524, 182.66) [a] 
(0.9635579166666524, 182.67) [a] 
(0.963559583333319, 182.68) [a] 
(0.9635599999999857, 182.69) [a] 
(0.9635616666666523, 182.7) [a] 
(0.9635622916666523, 182.72) [a] 
(0.9635631249999856, 183.69) [a] 
(0.963563333333319, 186.85) [a] 
(0.9635637499999856, 187.14) [a] 
(0.9635647916666523, 187.15) [a] 
(0.9635652083333189, 187.16) [a] 
(0.9635668749999856, 187.17) [a] 
(0.9635672916666522, 187.51) [a] 
(0.9635674999999856, 187.87) [a] 
(0.963588333333319, 187.99) [a] 
(0.9636012499999856, 188) [a] 
(0.9636079166666524, 188.01) [a] 
(0.9636106249999857, 188.15) [a] 
(0.9636112499999857, 188.17) [a] 
(0.9636118749999857, 190.45) [a] 
(0.9636122916666523, 190.49) [a] 
(0.9636124999999857, 190.56) [a] 
(0.9636149999999857, 190.8) [a] 
(0.9636154166666523, 190.88) [a] 
(0.9636156249999857, 192.99) [a] 
(0.9636160416666523, 193.02) [a] 
(0.9636164583333189, 193.06) [a] 
(0.9636166666666522, 193.2) [a] 
(0.9636170833333189, 193.34) [a] 
(0.9636174999999855, 193.35) [a] 
(0.9636177083333188, 193.47) [a] 
(0.9636181249999854, 193.57) [a] 
(0.9636247916666522, 198.1) [a] 
(0.9636252083333188, 203.63) [a] 
(0.9636254166666521, 204.45) [a] 
(0.9636258333333187, 215.76) [a] 
(0.9636262499999854, 216.91) [a] 
(0.963626666666652, 218.38) [a] 
(0.9636277083333187, 220.84) [a] 
(0.9636281249999853, 238.25) [a] 
(0.963629166666652, 238.31) [a] 
(0.9636295833333186, 238.33) [a] 
(0.963629791666652, 238.51) [a] 
(0.9636302083333186, 238.52) [a] 
(0.9637470833333186, 245.69) [a] 
(0.9637477083333186, 246.26) [a] 
(0.9637479166666519, 253.36) [a] 
(0.9637487499999853, 253.37) [a] 
(0.9637489583333186, 253.41) [a] 
(0.963750416666652, 253.43) [a] 
(0.9637545833333185, 253.44) [a] 
(0.9637558333333185, 253.45) [a] 
(0.9637572916666518, 253.48) [a] 
(0.9637577083333184, 253.52) [a] 
(0.9637583333333184, 254.1) [a] 
(0.963761249999985, 254.15) [a] 
(0.963761874999985, 254.21) [a] 
(0.9637622916666516, 256.28) [a] 
(0.9637633333333182, 256.43) [a] 
(0.9637635416666516, 256.44) [a] 
(0.9637639583333182, 257.02) [a] 
(0.9637645833333182, 257.66) [a] 
(0.9637654166666515, 257.75) [a] 
(0.9638056249999848, 258.45) [a] 
(0.9638060416666514, 268.48) [a] 
(0.9638066666666514, 268.94) [a] 
(0.963807083333318, 268.97) [a] 
(0.9639241666666514, 271.69) [a] 
(0.9639247916666513, 287.34) [a] 
(0.9639249999999847, 287.81) [a] 
(0.963925833333318, 287.89) [a] 
(0.9639260416666514, 287.9) [a] 
(0.963926458333318, 287.95) [a] 
(0.9639266666666514, 289.27) [a] 
(0.963927083333318, 289.39) [a] 
(0.9639274999999846, 289.43) [a] 
(0.9639279166666512, 289.44) [a] 
(0.9639281249999846, 289.45) [a] 
(0.9639285416666512, 289.47) [a] 
(0.9639295833333179, 289.48) [a] 
(0.9639299999999845, 289.51) [a] 
(0.9639302083333179, 289.52) [a] 
(0.9639306249999845, 289.57) [a] 
(0.9639310416666511, 289.61) [a] 
(0.9639316666666511, 289.62) [a] 
(0.9639320833333177, 289.64) [a] 
(0.9639322916666511, 289.89) [a] 
(0.9639327083333177, 289.93) [a] 
(0.9639331249999843, 291.17) [a] 
(0.9639337499999843, 291.18) [a] 
(0.9639343749999842, 292.53) [a] 
(0.9639349999999842, 298.6) [a] 
(0.9639364583333175, 298.67) [a] 
(0.9639370833333175, 299.05) [a] 
(0.9639379166666509, 299.21) [a] 
(0.9639420833333174, 299.22) [a] 
(0.9639427083333174, 299.23) [a] 
(0.963946249999984, 299.24) [a] 
(0.963946874999984, 299.26) [a] 
(0.9639489583333173, 299.35) [a] 
(0.9639516666666507, 299.84) [a] 
(0.963953124999984, 299.85) [a] 
(0.963953749999984, 299.86) [a] 
(0.9639545833333173, 299.87) [a] 
(0.9639552083333173, 299.93) [a] 
(0.9639572916666506, 299.99) [a] 
(0.9639579166666505, 300.02) [a] 
(0.9639587499999839, 300.22) [a] 
(0.9639608333333172, 300.23) [a] 
(0.9639614583333171, 300.24) [a] 
(0.9639635416666504, 300.71) [a] 
(0.9639649999999838, 300.82) [a] 
(0.9639656249999837, 300.96) [a] 
(0.9639662499999837, 301.05) [a] 
(0.963967083333317, 301.06) [a] 
(0.963967708333317, 301.18) [a] 
(0.9639691666666503, 302.08) [a] 
(0.9639697916666503, 302.1) [a] 
(0.9639704166666503, 302.11) [a] 
(0.9639712499999836, 302.41) [a] 
(0.9639718749999836, 302.42) [a] 
(0.9639724999999836, 302.6) [a] 
(0.9639733333333169, 302.67) [a] 
(0.9639739583333169, 302.7) [a] 
(0.9639745833333169, 302.71) [a] 
(0.9639754166666502, 302.99) [a] 
(0.9639760416666502, 303.03) [a] 
(0.9639766666666502, 303.04) [a] 
(0.9639774999999835, 304.08) [a] 
(0.9639779166666501, 306.49) [a] 
(0.9639791666666501, 308.79) [a] 
(0.9639799999999834, 308.81) [a] 
(0.9639806249999834, 308.96) [a] 
(0.9639812499999834, 310.51) [a] 
(0.9639820833333167, 311.33) [a] 
(0.9639827083333167, 311.36) [a] 
(0.9639833333333166, 311.41) [a] 
(0.9639854166666499, 311.87) [a] 
(0.9639868749999833, 313.07) [a] 
(0.9639874999999832, 313.37) [a] 
(0.9639889583333165, 313.4) [a] 
(0.9639910416666498, 313.41) [a] 
(0.9639916666666498, 313.47) [a] 
(0.9639924999999832, 313.63) [a] 
(0.9639931249999831, 313.81) [a] 
(0.9639937499999831, 313.86) [a] 
(0.9639945833333164, 314.04) [a] 
(0.9639952083333164, 314.62) [a] 
(0.9639958333333164, 314.63) [a] 
(0.9639966666666497, 314.8) [a] 
(0.9639979166666497, 315.47) [a] 
(0.963999374999983, 315.65) [a] 
(0.9640008333333163, 316.04) [a] 
(0.9640014583333163, 316.05) [a] 
(0.9640020833333163, 316.72) [a] 
(0.9640029166666496, 316.73) [a] 
(0.9640041666666496, 316.93) [a] 
(0.9640049999999829, 317.17) [a] 
(0.9640056249999829, 319.44) [a] 
(0.9640077083333162, 319.45) [a] 
(0.9640083333333161, 319.54) [a] 
(0.9640091666666495, 319.61) [a] 
(0.9640104166666494, 320.42) [a] 
(0.964010833333316, 322.01) [a] 
(0.9640131249999827, 322.02) [a] 
(0.9640135416666493, 322.04) [a] 
(0.9640139583333159, 322.06) [a] 
(0.9640141666666493, 322.08) [a] 
(0.9640145833333159, 322.27) [a] 
(0.9640156249999826, 322.28) [a] 
(0.9640160416666492, 322.29) [a] 
(0.9640162499999826, 322.32) [a] 
(0.9640166666666492, 322.33) [a] 
(0.9640170833333158, 322.34) [a] 
(0.9640181249999825, 322.36) [a] 
(0.9640183333333159, 322.37) [a] 
(0.9640187499999825, 322.39) [a] 
(0.9640191666666491, 322.4) [a] 
(0.9640193749999825, 322.42) [a] 
(0.9640202083333157, 322.43) [a] 
(0.964022291666649, 322.49) [a] 
(0.9640224999999824, 322.79) [a] 
(0.964022916666649, 324.12) [a] 
(0.9640235416666489, 324.14) [a] 
(0.9640239583333156, 324.38) [a] 
(0.9640243749999822, 324.4) [a] 
(0.9640252083333155, 325.33) [a] 
(0.9640258333333155, 325.46) [a] 
(0.9640272916666488, 327.24) [a] 
(0.9640279166666488, 328.73) [a] 
(0.9640285416666488, 329.12) [a] 
(0.9640293749999821, 329.8) [a] 
(0.9640302083333154, 333.57) [a] 
(0.9640314583333154, 334.12) [a] 
(0.9640329166666487, 334.13) [a] 
(0.9640341666666487, 334.15) [a] 
(0.964034999999982, 334.29) [a] 
(0.9640362499999819, 334.82) [a] 
(0.9640370833333153, 334.88) [a] 
(0.964038124999982, 335.18) [a] 
(0.964038749999982, 335.19) [a] 
(0.9640391666666486, 335.2) [a] 
(0.9640397916666485, 337.25) [a] 
(0.9640399999999819, 337.64) [a] 
(0.9640404166666485, 353.92) [a] 
(0.9640408333333151, 353.98) [a] 
(0.9640420833333152, 354.01) [a] 
(0.9640424999999818, 354.04) [a] 
(0.9640429166666484, 354.08) [a] 
(0.9640431249999818, 354.09) [a] 
(0.9640441666666485, 354.11) [a] 
(0.9640445833333151, 354.12) [a] 
(0.9640452083333151, 354.13) [a] 
(0.9640456249999817, 354.33) [a] 
(0.9640458333333151, 359.8) [a] 
(0.9640462499999817, 363.09) [a] 
(0.9640466666666483, 363.29) [a] 
(0.9640468749999817, 363.82) [a] 
(0.9640472916666483, 363.92) [a] 
(0.9640477083333149, 364.03) [a] 
(0.9640479166666482, 364.04) [a] 
(0.9640483333333149, 364.06) [a] 
(0.9640487499999815, 365.3) [a] 
(0.9640489583333148, 369.2) [a] 
(0.9640493749999814, 369.22) [a] 
(0.9640497916666481, 370.28) [a] 
(0.9640502083333147, 376.1) [a] 
(0.964051666666648, 377.48) [a] 
(0.964052291666648, 377.49) [a] 
(0.9640529166666479, 377.5) [a] 
(0.9640533333333146, 377.53) [a] 
(0.9640539583333145, 380.97) [a] 
(0.9640545833333145, 381.97) [a] 
(0.9640552083333145, 397.8) [a] 
(0.9640556249999811, 397.89) [a] 
(0.9641727083333145, 398.81) [a] 
(0.9641729166666478, 412.81) [a] 
(0.9641737499999812, 413.3) [a] 
(0.9641745833333145, 416.44) [a] 
(0.9641752083333145, 420.96) [a] 
(0.9641758333333145, 436.26) [a] 
(0.9641762499999811, 441.77) [a] 
(0.9641777083333144, 455.72) [a] 
(0.9641791666666477, 455.73) [a] 
(0.964180624999981, 456.85) [a] 
(0.964181874999981, 458.28) [a] 
(0.9641833333333143, 458.32) [a] 
(0.9641837499999809, 461.81) [a] 
(0.9641839583333143, 461.85) [a] 
(0.9641843749999809, 461.89) [a] 
(0.9641847916666475, 461.9) [a] 
(0.9641849999999809, 461.95) [a] 
(0.9641854166666475, 461.96) [a] 
(0.9641858333333141, 461.98) [a] 
(0.9641860416666475, 461.99) [a] 
(0.9641864583333141, 462.32) [a] 
(0.9641868749999807, 462.66) [a] 
(0.964188333333314, 463.58) [a] 
(0.9641897916666473, 464.13) [a] 
(0.964193958333314, 464.54) [a] 
(0.9641954166666473, 464.71) [a] 
(0.9641968749999806, 465.73) [a] 
(0.964198333333314, 466.54) [a] 
(0.9641997916666473, 466.55) [a] 
(0.9642012499999806, 466.59) [a] 
(0.964203958333314, 467.8) [a] 
(0.9642054166666473, 468.64) [a] 
(0.9642068749999806, 470.11) [a] 
(0.9642083333333139, 470.12) [a] 
(0.9642097916666472, 470.13) [a] 
(0.9642110416666472, 470.14) [a] 
(0.9642124999999805, 470.15) [a] 
(0.9642139583333138, 470.17) [a] 
(0.9642154166666471, 470.18) [a] 
(0.9642168749999804, 470.2) [a] 
(0.9642195833333138, 470.22) [a] 
(0.9642210416666471, 470.23) [a] 
(0.9642224999999804, 470.24) [a] 
(0.9642239583333138, 470.25) [a] 
(0.9642281249999805, 470.28) [a] 
(0.9642295833333138, 470.29) [a] 
(0.9642324999999804, 470.31) [a] 
(0.9642339583333137, 470.32) [a] 
(0.964235416666647, 470.33) [a] 
(0.9642381249999803, 470.34) [a] 
(0.9642395833333136, 470.36) [a] 
(0.9642424999999802, 470.4) [a] 
(0.9642439583333136, 470.46) [a] 
(0.9642452083333135, 470.51) [a] 
(0.9642481249999801, 470.66) [a] 
(0.9642510416666468, 470.8) [a] 
(0.9642524999999801, 470.87) [a] 
(0.96425374999998, 470.98) [a] 
(0.9642566666666467, 471.05) [a] 
(0.96425812499998, 471.09) [a] 
(0.9642595833333133, 471.14) [a] 
(0.9642610416666466, 471.2) [a] 
(0.9642622916666466, 471.27) [a] 
(0.9642637499999799, 471.34) [a] 
(0.9642652083333132, 471.36) [a] 
(0.9642666666666465, 471.41) [a] 
(0.9642681249999798, 471.47) [a] 
(0.9642695833333131, 471.54) [a] 
(0.9642708333333131, 471.65) [a] 
(0.9642722916666464, 472.31) [a] 
(0.9642737499999797, 472.88) [a] 
(0.9642739583333131, 473.36) [a] 
(0.9642768749999797, 474.71) [a] 
(0.9642772916666463, 477.21) [a] 
(0.9642777083333129, 477.24) [a] 
(0.9642779166666463, 477.25) [a] 
(0.9642783333333129, 477.27) [a] 
(0.9642787499999795, 477.36) [a] 
(0.9642789583333129, 477.49) [a] 
(0.9642793749999795, 477.51) [a] 
(0.9642797916666461, 477.54) [a] 
(0.9642802083333127, 478.1) [a] 
(0.9642804166666461, 478.16) [a] 
(0.9642808333333127, 480.51) [a] 
(0.964282291666646, 484.91) [a] 
(0.964282916666646, 488.04) [a] 
(0.964283541666646, 488.05) [a] 
(0.9642837499999793, 490.31) [a] 
(0.964284166666646, 490.32) [a] 
(0.9642847916666459, 490.33) [a] 
(0.9642862499999793, 490.34) [a] 
(0.9642866666666459, 490.41) [a] 
(0.9642868749999792, 490.43) [a] 
(0.9642872916666458, 490.51) [a] 
(0.9642877083333125, 490.52) [a] 
(0.9642881249999791, 492.2) [a] 
(0.9642883333333124, 492.28) [a] 
(0.964288749999979, 492.3) [a] 
(0.9642895833333124, 498.63) [a] 
(0.964289999999979, 499.39) [a] 
(0.9642902083333124, 499.45) [a] 
(0.964290624999979, 501.95) [a] 
(0.964291249999979, 503.61) [a] 
(0.9642916666666456, 514.3) [a] 
(0.9642918749999789, 526.3) [a] 
(0.9644087499999789, 544.89) [a] 
(0.9644091666666456, 545.89) [a] 
(0.9644097916666455, 554.4) [a] 
(0.9644106249999789, 589.75) [a] 
(0.9644118749999788, 589.82) [a] 
(0.9644154166666454, 590.21) [a] 
(0.9644160416666454, 590.22) [a] 
(0.9644168749999787, 590.23) [a] 
(0.9644181249999787, 590.24) [a] 
(0.964418958333312, 604.06) [a] 
(0.964419583333312, 612.53) [a] 
(0.9644216666666453, 612.54) [a] 
(0.9644237499999786, 612.59) [a] 
(0.9644252083333119, 612.6) [a] 
(0.9644264583333119, 625.04) [a] 
(0.9644266666666452, 664.06) [a] 
(0.9644270833333118, 664.08) [a] 
(0.9644272916666452, 666.34) [a] 
(0.9644279166666452, 673.72) [a] 
(0.9644283333333118, 698) [a] 
(0.9644285416666452, 712.43) [a] 
(0.964589791666645, 736.92) [a] 
(0.9646299999999783, 737.11) [a] 
(0.964670416666645, 737.13) [a] 
(0.9647912499999783, 743.38) [a] 
(0.9647918749999783, 752.89) [a] 
(0.9647927083333117, 753.54) [a] 
(0.964792916666645, 763.64) [a] 
(0.9647933333333116, 763.65) [a] 
(0.9647945833333116, 772.94) [a] 
(0.964794791666645, 777.33) [a] 
(0.9647952083333116, 777.35) [a] 
(0.9647956249999782, 778.04) [a] 
(0.9647958333333115, 779.08) [a] 
(0.9647962499999782, 779.11) [a] 
(0.9647966666666448, 779.86) [a] 
(0.9648108333333114, 783.42) [a] 
(0.964838749999978, 783.43) [a] 
(0.9648529166666446, 789.3) [a] 
(0.9648668749999779, 789.31) [a] 
(0.9648672916666445, 789.82) [a] 
(0.9648674999999779, 789.83) [a] 
(0.9648816666666445, 798.12) [a] 
(0.9648956249999778, 798.17) [a] 
(0.9649237499999778, 798.22) [a] 
(0.9649516666666443, 798.35) [a] 
(0.964965833333311, 798.78) [a] 
(0.9649662499999776, 800.34) [a] 
(0.9649802083333109, 800.69) [a] 
(0.9649943749999775, 800.7) [a] 
(0.9649947916666441, 813.88) [a] 
(0.9649952083333108, 813.92) [a] 
(0.9649954166666441, 813.95) [a] 
(0.9649958333333107, 814.26) [a] 
(0.965009791666644, 819.01) [a] 
(0.9650237499999773, 819.45) [a] 
(0.965037916666644, 819.47) [a] 
(0.9650393749999773, 820.74) [a] 
(0.9650397916666439, 890.61) [a] 
(0.9650402083333105, 898.53) [a] 
(0.9650406249999771, 898.54) [a] 
(0.9650408333333105, 898.58) [a] 
(0.9650416666666438, 898.59) [a] 
(0.9650418749999772, 905.31) [a] 
(0.9650420833333105, 915.99) [a] 
(0.9650435416666439, 916.06) [a] 
(0.9650441666666438, 916.07) [a] 
(0.9650456249999771, 918.56) [a] 
(0.9650477083333104, 918.61) [a] 
(0.9650483333333104, 919.04) [a] 
(0.9650489583333104, 919.08) [a] 
(0.9650510416666437, 919.4) [a] 
(0.965064999999977, 922.3) [a] 
(0.9650791666666436, 922.32) [a] 
(0.9650931249999769, 923.26) [a] 
(0.9651070833333102, 923.28) [a] 
(0.9651212499999768, 923.33) [a] 
(0.9651352083333101, 923.4) [a] 
(0.9651493749999768, 923.5) [a] 
(0.9651502083333101, 925.87) [a] 
(0.9651520833333102, 926.24) [a] 
(0.9651568749999768, 926.29) [a] 
(0.9651970833333101, 938.66) [a] 
(0.9652374999999768, 938.67) [a] 
(0.9652389583333101, 939.27) [a] 
(0.9652420833333101, 939.28) [a] 
(0.96524270833331, 939.8) [a] 
(0.96524333333331, 939.9) [a] 
(0.9652835416666433, 941.79) [a] 
(0.9652849999999766, 944.32) [a] 
(0.9652989583333099, 948.9) [a] 
(0.9653129166666432, 949.34) [a] 
(0.9653270833333099, 950.38) [a] 
(0.9653410416666431, 952.38) [a] 
(0.9653549999999764, 953.4) [a] 
(0.9653556249999764, 988.15) [a] 
(0.9653564583333097, 988.2) [a] 
(0.9653570833333097, 988.29) [a] 
(0.9653574999999763, 989.8) [a] 
(0.9653581249999763, 1001.1) [a] 
(0.965372291666643, 1007.7) [a] 
(0.9653862499999762, 1008.5) [a] 
(0.9653864583333096, 1022.4) [a] 
(0.965386666666643, 1120) [a] 
(0.9653870833333096, 1142.2) [a] 
(0.9653877083333096, 1142.3) [a] 
(0.9654006249999763, 1159.5) [a] 
(0.9654010416666429, 1276.8) [a] 
(0.9654014583333095, 1742) [a] 
(0.9654022916666428, 1880.8) [a] 
(0.9654087499999762, 1896.7) [a] 
(0.9654093749999761, 1932.8) [a] 
(0.9654095833333095, 2124.1) [a] 
(0.9654099999999761, 2141.2) [a] 
(0.9654104166666427, 2519.5) [a] 
(0.9654106249999761, 2519.6) [a] 
(0.9654108333333095, 2590.7) [a] 
(0.9654114583333094, 2601.5) [a] 
(0.965416874999976, 2601.7) [a] 
(0.9654237499999759, 2602.1) [a] 
(0.9654472916666426, 2607.5) [a] 
(0.9654487499999759, 2610) [a] 
(0.9654956249999759, 2616.4) [a] 
(0.9655191666666425, 2616.5) [a] 
(0.9655427083333091, 2618.6) [a] 
(0.9655660416666425, 2625.8) [a] 
(0.9655687499999758, 2631.3) [a] 
(0.9655708333333091, 2631.4) [a] 
(0.9655729166666424, 2632.1) [a] 
(0.965581458333309, 2746.4) [a] 
(0.965582708333309, 2748.9) [a] 
(0.9655885416666422, 2789.5) [a] 
(0.9655899999999755, 2789.8) [a] 
(0.9655912499999755, 2790.1) [a] 
(0.9655927083333088, 2790.2) [a] 
(0.9655985416666422, 2791.4) [a] 
(0.9655997916666421, 2793.7) [a] 
(0.9656012499999754, 2865) [a] 
(0.9656247916666421, 3059) [a] 
},{(0.8409019251496185, 0) [b] 
(0.8794239765593256, 0.001) [b] 
(0.8941673956904447, 0.002) [b] 
(0.8980740368648389, 0.003) [b] 
(0.9007888562503038, 0.004) [b] 
(0.9032523422438659, 0.005) [b] 
(0.9043891045563026, 0.006) [b] 
(0.90555831365529, 0.007) [b] 
(0.9060748465270825, 0.008) [b] 
(0.9072333621897868, 0.009) [b] 
(0.9077799506233065, 0.01) [b] 
(0.9082630420074872, 0.011) [b] 
(0.908907881506822, 0.012) [b] 
(0.9099842106403497, 0.013) [b] 
(0.9105338251738081, 0.014) [b] 
(0.9114343656786252, 0.015) [b] 
(0.9127861665149333, 0.016) [b] 
(0.9136942873309926, 0.017) [b] 
(0.9140815761482136, 0.018) [b] 
(0.9143084870713347, 0.019) [b] 
(0.9144112377608541, 0.02) [b] 
(0.91483915437218, 0.021) [b] 
(0.9154262017553301, 0.022) [b] 
(0.9157128072076507, 0.023) [b] 
(0.915716245164899, 0.024) [b] 
(0.9158078676522398, 0.025) [b] 
(0.9158477479563203, 0.026) [b] 
(0.9158944671689204, 0.027) [b] 
(0.916006101401039, 0.028) [b] 
(0.9160067889924887, 0.029) [b] 
(0.9160562955768644, 0.03) [b] 
(0.9161662639497443, 0.031) [b] 
(0.9162617249336855, 0.032) [b] 
(0.9162898499336856, 0.033) [b] 
(0.9162926208584686, 0.034) [b] 
(0.9163214334499182, 0.035) [b] 
(0.9163297599302568, 0.036) [b] 
(0.9164407649820239, 0.037) [b] 
(0.9164456192401401, 0.038) [b] 
(0.9164786880232089, 0.041) [b] 
(0.9164814658009867, 0.042) [b] 
(0.9164856324676534, 0.043) [b] 
(0.9165048850282439, 0.045) [b] 
(0.916543390149425, 0.046) [b] 
(0.9169017234827583, 0.047) [b] 
(0.916904473848557, 0.049) [b] 
(0.9173079460707793, 0.05) [b] 
(0.9173159321818903, 0.052) [b] 
(0.9173180155152237, 0.053) [b] 
(0.917321393710997, 0.054) [b] 
(0.9173247719067702, 0.055) [b] 
(0.9173424802401035, 0.056) [b] 
(0.9174048458018798, 0.058) [b] 
(0.917408223997653, 0.059) [b] 
(0.9174691326914961, 0.06) [b] 
(0.9175217811882533, 0.061) [b] 
(0.9176727577894618, 0.062) [b] 
(0.9177385807933781, 0.063) [b] 
(0.9177396224600448, 0.064) [b] 
(0.917766749196156, 0.07) [b] 
(0.9179307379065405, 0.073) [b] 
(0.9179349045732071, 0.075) [b] 
(0.9179376823509849, 0.076) [b] 
(0.9179411545732071, 0.077) [b] 
(0.9187630266349147, 0.078) [b] 
(0.918813764482137, 0.079) [b] 
(0.9188408912182482, 0.08) [b] 
(0.9190833391349148, 0.081) [b] 
(0.9190840335793592, 0.082) [b] 
(0.9190854224682481, 0.083) [b] 
(0.9190909780238037, 0.085) [b] 
(0.9190944502460259, 0.086) [b] 
(0.9191041724682482, 0.087) [b] 
(0.9191125058015815, 0.088) [b] 
(0.9191263946904704, 0.089) [b] 
(0.9191329919126927, 0.09) [b] 
(0.9193008454093747, 0.091) [b] 
(0.9193057065204858, 0.092) [b] 
(0.9193101263829258, 0.093) [b] 
(0.9193428086745925, 0.094) [b] 
(0.9193713242995926, 0.095) [b] 
(0.9193803520773702, 0.096) [b] 
(0.9194944145773701, 0.097) [b] 
(0.9195841281190368, 0.098) [b] 
(0.9196119492995923, 0.099) [b] 
(0.9198198486051479, 0.1) [b] 
(0.91984207082737, 0.101) [b] 
(0.9199582166607033, 0.102) [b] 
(0.9199596055495922, 0.103) [b] 
(0.9201325222162589, 0.104) [b] 
(0.9201339111051476, 0.105) [b] 
(0.9201852999940364, 0.106) [b] 
(0.9201873833273697, 0.107) [b] 
(0.9201958828397401, 0.108) [b] 
(0.9202674106175177, 0.109) [b] 
(0.9202687995064066, 0.11) [b] 
(0.9202715772841844, 0.111) [b] 
(0.9202743550619621, 0.112) [b] 
(0.9202760911730732, 0.113) [b] 
(0.9207442559670806, 0.115) [b] 
(0.9214527531580918, 0.116) [b] 
(0.9214537948247585, 0.117) [b] 
(0.9214642114914252, 0.118) [b] 
(0.9220494174839345, 0.119) [b] 
(0.9220521952617123, 0.12) [b] 
(0.9220535841506012, 0.122) [b] 
(0.9220556674839345, 0.123) [b] 
(0.9220570563728233, 0.124) [b] 
(0.9220584452617121, 0.125) [b] 
(0.9220601813728232, 0.126) [b] 
(0.9220608758172676, 0.127) [b] 
(0.9220619174839343, 0.128) [b] 
(0.9221938619283787, 0.131) [b] 
(0.922195945261712, 0.132) [b] 
(0.9222045471599181, 0.133) [b] 
(0.9222094082710292, 0.134) [b] 
(0.922210797159918, 0.135) [b] 
(0.9222128804932513, 0.136) [b] 
(0.9222142693821402, 0.137) [b] 
(0.9222177416043624, 0.138) [b] 
(0.9222934360488069, 0.139) [b] 
(0.922320562784918, 0.145) [b] 
(0.9223413961182514, 0.146) [b] 
(0.9223622294515847, 0.149) [b] 
(0.9232602179573319, 0.151) [b] 
(0.9237476974318803, 0.152) [b] 
(0.9237990110607802, 0.153) [b] 
(0.9250048813399264, 0.154) [b] 
(0.9255436744433747, 0.155) [b] 
(0.9258575342545405, 0.156) [b] 
(0.9259088478834404, 0.157) [b] 
(0.9260704652890234, 0.164) [b] 
(0.926198749361273, 0.167) [b] 
(0.9262227076946062, 0.17) [b] 
(0.9262465423275763, 0.171) [b] 
(0.926247583994243, 0.175) [b] 
(0.9262479312164652, 0.176) [b] 
(0.9262486256609096, 0.178) [b] 
(0.9263255961042594, 0.18) [b] 
(0.9263776041776036, 0.181) [b] 
(0.9264095486220479, 0.182) [b] 
(0.9264865190653977, 0.184) [b] 
(0.9265440826942976, 0.185) [b] 
(0.9265597076942976, 0.186) [b] 
(0.9265614438054087, 0.187) [b] 
(0.9265617910276309, 0.188) [b] 
(0.9265819299165198, 0.191) [b] 
(0.9265843604720754, 0.192) [b] 
(0.9265850549165198, 0.193) [b] 
(0.9265982493609642, 0.194) [b] 
(0.9266013743609642, 0.195) [b] 
(0.9266399594304088, 0.196) [b] 
(0.9266778066526309, 0.197) [b] 
(0.9267041955415196, 0.198) [b] 
(0.9267069733192973, 0.199) [b] 
(0.9267118344304083, 0.2) [b] 
(0.9267153066526305, 0.201) [b] 
(0.9267156538748527, 0.202) [b] 
(0.9267413106893025, 0.203) [b] 
(0.9267423523559691, 0.204) [b] 
(0.9267440884670802, 0.205) [b] 
(0.9267447829115246, 0.206) [b] 
(0.9267475606893023, 0.207) [b] 
(0.9267871440226356, 0.208) [b] 
(0.9267885329115245, 0.209) [b] 
(0.9267902690226355, 0.211) [b] 
(0.9268200925037521, 0.212) [b] 
(0.9268270369481965, 0.213) [b] 
(0.9268273841704187, 0.215) [b] 
(0.9268291202815298, 0.216) [b] 
(0.9268308563926408, 0.217) [b] 
(0.926831203614863, 0.218) [b] 
(0.9268318980593074, 0.219) [b] 
(0.9268391897259741, 0.22) [b] 
(0.9270728702815298, 0.221) [b] 
(0.9271190508370853, 0.222) [b] 
(0.9271273841704187, 0.223) [b] 
(0.9271770369481965, 0.224) [b] 
(0.9272023841704187, 0.225) [b] 
(0.9273075925037519, 0.226) [b] 
(0.9273079397259741, 0.227) [b] 
(0.9273305091704186, 0.228) [b] 
(0.9274259952815297, 0.229) [b] 
(0.927513842503752, 0.23) [b] 
(0.9275197452815297, 0.231) [b] 
(0.9275242591704184, 0.232) [b] 
(0.9275537730593073, 0.233) [b] 
(0.9275749536148629, 0.234) [b] 
(0.9276847301198043, 0.235) [b] 
(0.9277354245642487, 0.236) [b] 
(0.9277538273420264, 0.237) [b] 
(0.9277562578975818, 0.238) [b] 
(0.9278946152522095, 0.239) [b] 
(0.9280019069188762, 0.24) [b] 
(0.9280095183957859, 0.241) [b] 
(0.928228615618008, 0.242) [b] 
(0.9283931989513413, 0.243) [b] 
(0.9284761850624523, 0.244) [b] 
(0.9285470183957855, 0.245) [b] 
(0.92873833784023, 0.246) [b] 
(0.9288678517291188, 0.247) [b] 
(0.928935560062452, 0.248) [b] 
(0.9289466711735631, 0.249) [b] 
(0.9290372961735632, 0.25) [b] 
(0.9290661156180076, 0.251) [b] 
(0.9292591711735632, 0.252) [b] 
(0.9293310461735631, 0.253) [b] 
(0.9293761850624519, 0.254) [b] 
(0.9294171161667053, 0.255) [b] 
(0.9294730189444831, 0.256) [b] 
(0.9295087828333719, 0.257) [b] 
(0.9295178106111496, 0.258) [b] 
(0.9295292689444827, 0.259) [b] 
(0.9299212828333717, 0.26) [b] 
(0.9300053106111493, 0.261) [b] 
(0.930842463388927, 0.262) [b] 
(0.9310275328333714, 0.263) [b] 
(0.931049060611149, 0.264) [b] 
(0.9310560050555933, 0.265) [b] 
(0.9310882967222598, 0.266) [b] 
(0.9310952411667043, 0.267) [b] 
(0.9310962828333709, 0.268) [b] 
(0.9314636439444818, 0.269) [b] 
(0.9316011439444816, 0.27) [b] 
(0.9317174633889259, 0.271) [b] 
(0.9318476717222592, 0.272) [b] 
(0.9318730189444814, 0.273) [b] 
(0.931908435611148, 0.274) [b] 
(0.9319403800555924, 0.275) [b] 
(0.9319525328333701, 0.276) [b] 
(0.9325733661667035, 0.277) [b] 
(0.9325837828333701, 0.278) [b] 
(0.9325966300555921, 0.279) [b] 
(0.9327771856111476, 0.28) [b] 
(0.9327907272778141, 0.281) [b] 
(0.9328112133889251, 0.282) [b] 
(0.9328271856111472, 0.283) [b] 
(0.9335982520472023, 0.284) [b] 
(0.9336322798249799, 0.285) [b] 
(0.9336610992694243, 0.286) [b] 
(0.9336683909360908, 0.287) [b] 
(0.9338277659360908, 0.288) [b] 
(0.9343985992694241, 0.289) [b] 
(0.9344246409360909, 0.29) [b] 
(0.9345795020472021, 0.291) [b] 
(0.9345836687138688, 0.292) [b] 
(0.9346454742694242, 0.293) [b] 
(0.934676377047202, 0.294) [b] 
(0.9346965159360908, 0.295) [b] 
(0.9352673492694242, 0.296) [b] 
(0.9352759634282539, 0.297) [b] 
(0.9353717967615872, 0.298) [b] 
(0.9354124217615872, 0.299) [b] 
(0.9358158939838095, 0.3) [b] 
(0.9358419356504761, 0.301) [b] 
(0.9359224912060317, 0.302) [b] 
(0.9360013106504762, 0.303) [b] 
(0.9363464495393649, 0.304) [b] 
(0.9363641578726981, 0.305) [b] 
(0.9364308245393648, 0.306) [b] 
(0.9364478384282535, 0.307) [b] 
(0.9366204078726978, 0.308) [b] 
(0.9366818662060311, 0.309) [b] 
(0.9367408939838088, 0.31) [b] 
(0.9367839495393642, 0.311) [b] 
(0.9368238800949197, 0.312) [b] 
(0.9368537412060307, 0.313) [b] 
(0.9368568662060307, 0.314) [b] 
(0.9368933245393639, 0.315) [b] 
(0.9369426300949194, 0.316) [b] 
(0.9369565189838083, 0.317) [b] 
(0.9371106856504748, 0.318) [b] 
(0.9374252689838082, 0.319) [b] 
(0.9375804773171414, 0.32) [b] 
(0.9377865298403519, 0.321) [b] 
(0.9378174326181298, 0.322) [b] 
(0.937877502062574, 0.323) [b] 
(0.9379163909514628, 0.324) [b] 
(0.9379306270625738, 0.325) [b] 
(0.9379514603959069, 0.326) [b] 
(0.9379747242847957, 0.327) [b] 
(0.9379934742847956, 0.328) [b] 
(0.9379993770625732, 0.329) [b] 
(0.9380174326181286, 0.33) [b] 
(0.9380282311462005, 0.331) [b] 
(0.9380355228128671, 0.332) [b] 
(0.9380454875093709, 0.333) [b] 
(0.9380465291760375, 0.334) [b] 
(0.9380479180649264, 0.335) [b] 
(0.9380493415929984, 0.336) [b] 
(0.9380906610374428, 0.337) [b] 
(0.938634758259665, 0.338) [b] 
(0.9387563133470558, 0.339) [b] 
(0.9387697156265987, 0.34) [b] 
(0.9390027017377097, 0.341) [b] 
(0.9390129443780696, 0.342) [b] 
(0.9390188471558473, 0.343) [b] 
(0.9391428054891805, 0.344) [b] 
(0.9391556527114027, 0.345) [b] 
(0.9392299582669582, 0.346) [b] 
(0.9392348193780694, 0.347) [b] 
(0.9392771804891803, 0.348) [b] 
(0.9392785693780691, 0.349) [b] 
(0.9393077360447357, 0.35) [b] 
(0.939309819378069, 0.351) [b] 
(0.9393222146301283, 0.352) [b] 
(0.9393465894640501, 0.353) [b] 
(0.939351103352939, 0.354) [b] 
(0.9393645749108481, 0.355) [b] 
(0.9393705816061751, 0.356) [b] 
(0.9393709288283973, 0.357) [b] 
(0.9393764843839529, 0.36) [b] 
(0.9393837760506196, 0.361) [b] 
(0.9393851995786916, 0.366) [b] 
(0.9393855468009138, 0.367) [b] 
(0.9393862412453582, 0.368) [b] 
(0.9393928384675804, 0.369) [b] 
(0.9394022134675803, 0.37) [b] 
(0.9394447565382017, 0.371) [b] 
(0.9394822999409795, 0.372) [b] 
(0.9395013616936273, 0.373) [b] 
(0.9395027505825162, 0.374) [b] 
(0.9395055283602939, 0.375) [b] 
(0.9395069172491828, 0.377) [b] 
(0.9395090005825161, 0.378) [b] 
(0.9395096950269605, 0.379) [b] 
(0.9395100422491827, 0.381) [b] 
(0.9395103894714049, 0.382) [b] 
(0.9395114311380715, 0.383) [b] 
(0.9395152505825158, 0.384) [b] 
(0.9395159450269602, 0.385) [b] 
(0.9395166394714046, 0.386) [b] 
(0.9395169866936268, 0.387) [b] 
(0.9395183755825157, 0.39) [b] 
(0.9396820813107927, 0.391) [b] 
(0.9396824285330149, 0.393) [b] 
(0.9396852755891588, 0.396) [b] 
(0.9397026358698786, 0.398) [b] 
(0.939716975898539, 0.4) [b] 
(0.9397242675652057, 0.401) [b] 
(0.9397342322617095, 0.402) [b] 
(0.9397402389570365, 0.404) [b] 
(0.9397939882927234, 0.406) [b] 
(0.9397991966260567, 0.407) [b] 
(0.9398009327371678, 0.408) [b] 
(0.9398140224336716, 0.409) [b] 
(0.9398258279892272, 0.414) [b] 
(0.9398338141003383, 0.415) [b] 
(0.939850480767005, 0.416) [b] 
(0.9398511752114493, 0.418) [b] 
(0.9398567307670049, 0.42) [b] 
(0.939858466878116, 0.421) [b] 
(0.9398605502114493, 0.422) [b] 
(0.9398668002114493, 0.425) [b] 
(0.9398674946558937, 0.426) [b] 
(0.9399827311513574, 0.428) [b] 
(0.940063286706913, 0.432) [b] 
(0.9400732514034168, 0.433) [b] 
(0.9400874866841366, 0.435) [b] 
(0.9400960278525684, 0.437) [b] 
(0.9401054028525684, 0.438) [b] 
(0.9401110969648563, 0.439) [b] 
(0.9401114441870785, 0.442) [b] 
(0.9401472012229726, 0.443) [b] 
(0.9401482428896393, 0.444) [b] 
(0.9402048401118615, 0.446) [b] 
(0.9402631701236293, 0.447) [b] 
(0.940264211790296, 0.448) [b] 
(0.940290215826968, 0.449) [b] 
(0.9403161610430174, 0.45) [b] 
(0.9403168554874618, 0.451) [b] 
(0.9403527232939, 0.454) [b] 
(0.9403569938781159, 0.458) [b] 
(0.9403583827670048, 0.46) [b] 
(0.940361760962778, 0.461) [b] 
(0.9403624554072224, 0.462) [b] 
(0.9404423470055597, 0.463) [b] 
(0.9404433886722263, 0.464) [b] 
(0.9404458192277819, 0.465) [b] 
(0.9404619300067365, 0.466) [b] 
(0.9404640133400698, 0.467) [b] 
(0.9405146701545197, 0.468) [b] 
(0.9405157118211864, 0.469) [b] 
(0.9406161942785755, 0.471) [b] 
(0.9406283470563532, 0.472) [b] 
(0.9406290415007976, 0.473) [b] 
(0.9406304303896865, 0.474) [b] 
(0.9406321665007976, 0.476) [b] 
(0.9406422359452421, 0.477) [b] 
(0.9406584589390693, 0.478) [b] 
(0.9406588061612915, 0.48) [b] 
(0.9406622783835137, 0.481) [b] 
(0.9406688756057359, 0.482) [b] 
(0.9406695700501803, 0.483) [b] 
(0.9406716533835136, 0.484) [b] 
(0.9412650561612913, 0.495) [b] 
(0.941271932075788, 0.501) [b] 
(0.9412882515202324, 0.503) [b] 
(0.9412892931868991, 0.509) [b] 
(0.9412899807783488, 0.51) [b] 
(0.9412906752227932, 0.513) [b] 
(0.9413888426610648, 0.514) [b] 
(0.9414852932783487, 0.515) [b] 
(0.9414901543894598, 0.525) [b] 
(0.941490501611682, 0.532) [b] 
(0.9414946682783487, 0.533) [b] 
(0.9414977932783487, 0.534) [b] 
(0.9415325589033487, 0.535) [b] 
(0.9415690172366821, 0.536) [b] 
(0.941573531125571, 0.537) [b] 
(0.9415738783477932, 0.538) [b] 
(0.9415745727922376, 0.539) [b] 
(0.9415776977922375, 0.54) [b] 
(0.9415780450144597, 0.541) [b] 
(0.9415797811255708, 0.542) [b] 
(0.9415842950144597, 0.543) [b] 
(0.9416179755700153, 0.544) [b] 
(0.9416214477922374, 0.545) [b] 
(0.941660990713796, 0.546) [b] 
(0.9417265348902466, 0.547) [b] 
(0.9417282710013577, 0.548) [b] 
(0.9418569287987033, 0.549) [b] 
(0.9418929994980396, 0.55) [b] 
(0.9419568075633791, 0.551) [b] 
(0.9419613214522679, 0.552) [b] 
(0.941988448188379, 0.556) [b] 
(0.9420318509661568, 0.557) [b] 
(0.9420894898550457, 0.559) [b] 
(0.9421012954106013, 0.56) [b] 
(0.9421019898550457, 0.561) [b] 
(0.9421033787439346, 0.562) [b] 
(0.942222850497992, 0.564) [b] 
(0.9422235449424364, 0.565) [b] 
(0.9422249338313253, 0.566) [b] 
(0.9422544477202142, 0.568) [b] 
(0.9422547949424364, 0.571) [b] 
(0.9422690310535474, 0.572) [b] 
(0.9422718088313252, 0.573) [b] 
(0.9422735449424363, 0.574) [b] 
(0.9423280184195505, 0.576) [b] 
(0.9423731168966647, 0.577) [b] 
(0.9424050209293344, 0.579) [b] 
(0.9424369249620042, 0.58) [b] 
(0.9424688289946739, 0.583) [b] 
(0.9425326370600133, 0.584) [b] 
(0.9425645410926831, 0.586) [b] 
(0.9426081842841825, 0.587) [b] 
(0.9426400883168522, 0.588) [b] 
(0.9426501577612967, 0.59) [b] 
(0.9426518938724078, 0.592) [b] 
(0.9426574494279634, 0.593) [b] 
(0.9426591855390745, 0.594) [b] 
(0.9426595327612967, 0.596) [b] 
(0.942660227205741, 0.597) [b] 
(0.9426605744279632, 0.599) [b] 
(0.94266161609463, 0.607) [b] 
(0.9426640466501855, 0.611) [b] 
(0.9426661299835188, 0.612) [b] 
(0.9426671716501855, 0.623) [b] 
(0.94267099109463, 0.648) [b] 
(0.94299555549504, 0.653) [b] 
(0.9429962499394844, 0.682) [b] 
(0.9430102885738061, 0.697) [b] 
(0.9430243272081278, 0.699) [b] 
(0.9430253688747945, 0.704) [b] 
(0.9430305772081278, 0.709) [b] 
(0.9430316188747945, 0.713) [b] 
(0.9430326605414612, 0.715) [b] 
(0.9430378688747945, 0.717) [b] 
(0.9430519075091162, 0.72) [b] 
(0.9430529491757829, 0.722) [b] 
(0.9430539908424496, 0.727) [b] 
(0.9434383434218824, 0.728) [b] 
(0.9434498017552158, 0.732) [b] 
(0.943451885088549, 0.737) [b] 
(0.9434525795329934, 0.74) [b] 
(0.9434568501172094, 0.741) [b] 
(0.9434776834505427, 0.77) [b] 
(0.9435193501172094, 0.771) [b] 
(0.9438012086754602, 0.774) [b] 
(0.9438029447865713, 0.776) [b] 
(0.943803986453238, 0.777) [b] 
(0.9438139511497419, 0.779) [b] 
(0.9438224923181737, 0.781) [b] 
(0.9438266589848404, 0.782) [b] 
(0.9438270062070626, 0.783) [b] 
(0.9438405478737293, 0.787) [b] 
(0.943841589540396, 0.791) [b] 
(0.9438629424614756, 0.801) [b] 
(0.94386363690592, 0.811) [b] 
(0.9438723174614756, 0.821) [b] 
(0.9438726646836978, 0.832) [b] 
(0.9439908175136716, 0.836) [b] 
(0.9439918591803383, 0.87) [b] 
(0.9439946369581161, 0.874) [b] 
(0.9439953314025605, 0.877) [b] 
(0.944002275847005, 0.878) [b] 
(0.9440033175136717, 0.884) [b] 
(0.9440043591803384, 0.889) [b] 
(0.9440054008470051, 0.907) [b] 
(0.9440060952914495, 0.925) [b] 
(0.9445014830604963, 0.929) [b] 
(0.9445128712850721, 0.96) [b] 
(0.9445142948131441, 0.962) [b] 
(0.9445146420353663, 0.986) [b] 
(0.9445189126195822, 1.062) [b] 
(0.9446626889548515, 1.101) [b] 
(0.9446630361770737, 1.11) [b] 
(0.9446886596823693, 1.139) [b] 
(0.9447079082194805, 1.159) [b] 
(0.9447313956302283, 1.161) [b] 
(0.9447378118092654, 1.162) [b] 
(0.9447612992200133, 1.194) [b] 
(0.9447627227480853, 1.262) [b] 
(0.9447655698042292, 1.266) [b] 
(0.9447707781375625, 1.314) [b] 
(0.9447771943165996, 1.39) [b] 
(0.9447921248721551, 1.522) [b] 
(0.944802888761044, 1.567) [b] 
(0.9448035832054884, 1.74) [b] 
(0.9449372158760037, 1.796) [b] 
(0.9449565695731128, 1.84) [b] 
(0.9449574911777371, 1.849) [b] 
(0.9449579519800492, 1.851) [b] 
(0.9449584127823613, 1.856) [b] 
(0.9449593343869855, 1.864) [b] 
(0.9449597951892976, 1.866) [b] 
(0.9449644032124188, 1.876) [b] 
(0.9449648640147309, 1.878) [b] 
(0.9451989464117346, 1.887) [b] 
(0.9452058584464165, 1.891) [b] 
(0.9452063192487286, 1.893) [b] 
(0.9452067800510406, 1.896) [b] 
(0.9452123096787861, 1.903) [b] 
(0.9452162297394153, 1.909) [b] 
(0.9452176121463517, 1.912) [b] 
(0.9452194553556001, 1.917) [b] 
(0.9496318854095878, 1.933) [b] 
(0.9496323462118998, 1.951) [b] 
(0.9496611656563443, 1.972) [b] 
(0.949663930470217, 1.973) [b] 
(0.9496648520748413, 1.981) [b] 
(0.9496662344817777, 1.985) [b] 
(0.949667156086402, 1.99) [b] 
(0.9496699209002747, 1.993) [b] 
(0.949670842504899, 1.996) [b] 
(0.9496713033072111, 2.002) [b] 
(0.9496969601216609, 2.01) [b] 
(0.9497052545632791, 2.032) [b] 
(0.9497103233887125, 2.039) [b] 
(0.9497126274002732, 2.046) [b] 
(0.9497144706095216, 2.053) [b] 
(0.9497149314118337, 2.056) [b] 
(0.949715739436368, 2.061) [b] 
(0.9497162002386801, 2.064) [b] 
(0.9497175826456165, 2.066) [b] 
(0.9497432394600663, 2.098) [b] 
(0.9497847116681573, 2.129) [b] 
(0.9498010311126017, 2.206) [b] 
(0.949829503334824, 2.207) [b] 
(0.9498791561126018, 2.208) [b] 
(0.9498869897519079, 2.246) [b] 
(0.9498960175296857, 2.257) [b] 
(0.9499006042939322, 2.266) [b] 
(0.9499010438373696, 2.268) [b] 
(0.949901483380807, 2.27) [b] 
(0.9499037873923677, 2.274) [b] 
(0.9499153074501707, 2.285) [b] 
(0.949915600479129, 2.289) [b] 
(0.9499437254791291, 2.29) [b] 
(0.9499464902930018, 2.298) [b] 
(0.9499515591184352, 2.321) [b] 
(0.9499517056329143, 2.325) [b] 
(0.9499519986618726, 2.33) [b] 
(0.949952291690831, 2.331) [b] 
(0.9499524382053101, 2.333) [b] 
(0.9499534638066641, 2.335) [b] 
(0.9499539033501015, 2.337) [b] 
(0.9499540498645807, 2.34) [b] 
(0.9499541963790598, 2.343) [b] 
(0.9499544894080181, 2.345) [b] 
(0.9499546359224972, 2.346) [b] 
(0.9499560183294335, 2.365) [b] 
(0.9499574007363699, 2.367) [b] 
(0.9499578402798073, 2.386) [b] 
(0.9499581333087657, 2.392) [b] 
(0.949966466642099, 2.395) [b] 
(0.9499667596710574, 2.399) [b] 
(0.9499688959092641, 2.405) [b] 
(0.9499691889382225, 2.407) [b] 
(0.9499719537520952, 2.419) [b] 
(0.9499733361590316, 2.434) [b] 
(0.9499742365047811, 2.437) [b] 
(0.9499745295337394, 2.439) [b] 
(0.9499746760482185, 2.441) [b] 
(0.9499748225626976, 2.443) [b] 
(0.9499757441673219, 2.463) [b] 
(0.9499761837107593, 2.47) [b] 
(0.9499771053153836, 2.472) [b] 
(0.9499775661176957, 2.482) [b] 
(0.94997848772232, 2.485) [b] 
(0.9499794093269442, 2.488) [b] 
(0.9499807917338806, 2.495) [b] 
(0.949982174140817, 2.497) [b] 
(0.9499826349431291, 2.5) [b] 
(0.9499830957454412, 2.506) [b] 
(0.9499838283178369, 2.522) [b] 
(0.9500188977622813, 2.61) [b] 
(0.9500222759580546, 2.646) [b] 
(0.9500256541538278, 2.649) [b] 
(0.9500517774558132, 2.662) [b] 
(0.9500585338473596, 2.665) [b] 
(0.9500619120431328, 2.667) [b] 
(0.9500684428686292, 2.669) [b] 
(0.9500726095352959, 2.768) [b] 
(0.950074345646407, 2.775) [b] 
(0.9500757208293064, 2.782) [b] 
(0.950076408420756, 2.783) [b] 
(0.9500771028652004, 2.863) [b] 
(0.9500777904566501, 2.868) [b] 
(0.9500784780480998, 2.892) [b] 
(0.9500919908311926, 2.956) [b] 
(0.9500953690269658, 2.957) [b] 
(0.950098747222739, 2.958) [b] 
(0.9501055036142854, 2.967) [b] 
(0.9502225448127872, 3.035) [b] 
(0.9503667759364813, 3.283) [b] 
(0.9504876656850306, 3.304) [b] 
(0.9504890545739195, 3.416) [b] 
(0.9504894017961417, 3.647) [b] 
(0.9504939156850306, 3.662) [b] 
(0.9504972938808038, 3.76) [b] 
(0.950524420616915, 3.837) [b] 
(0.9507662001140137, 4.18) [b] 
(0.9507744945556319, 4.427) [b] 
(0.9507763377648804, 4.43) [b] 
(0.9507767985671924, 4.433) [b] 
(0.9507781809741288, 4.436) [b] 
(0.9507846322064986, 4.444) [b] 
(0.9507897010319319, 4.452) [b] 
(0.950833451031932, 4.535) [b] 
(0.9508468142989834, 4.575) [b] 
(0.9508481967059198, 4.578) [b] 
(0.9508606383683471, 4.688) [b] 
(0.9508772272515835, 4.781) [b] 
(0.9508809136700805, 4.787) [b] 
(0.9509080404061917, 4.804) [b] 
(0.9509351671423029, 4.808) [b] 
(0.9509457655954816, 4.94) [b] 
(0.9509462263977937, 4.949) [b] 
(0.9509480696070421, 4.955) [b] 
(0.9509485304093542, 4.957) [b] 
(0.9509489912116663, 4.96) [b] 
(0.9509507273227774, 5.004) [b] 
(0.9509562569505229, 5.115) [b] 
(0.9509702955848446, 5.378) [b] 
(0.950972138794093, 5.436) [b] 
(0.9509730603987173, 5.439) [b] 
(0.9510853694732906, 5.74) [b] 
(0.9511274853762556, 5.741) [b] 
(0.9511555626448989, 5.745) [b] 
(0.9512257558165071, 5.847) [b] 
(0.9512397944508288, 5.848) [b] 
(0.9512538330851505, 5.857) [b] 
(0.9512819103537938, 6.125) [b] 
(0.9512959489881155, 6.14) [b] 
(0.9512966434325599, 6.179) [b] 
(0.9512976850992266, 6.195) [b] 
(0.9514226850992267, 6.452) [b] 
(0.9514231459015388, 6.916) [b] 
(0.9514248820126499, 7.006) [b] 
(0.9514505388270997, 7.053) [b] 
(0.9514761956415496, 7.213) [b] 
(0.9515018524559995, 7.367) [b] 
(0.9515275092704494, 7.564) [b] 
(0.9515531660848993, 7.601) [b] 
(0.9515577741080204, 7.879) [b] 
(0.9515633037357659, 7.883) [b] 
(0.951563764538078, 7.888) [b] 
(0.9515642253403901, 7.895) [b] 
(0.95158988215484, 7.9) [b] 
(0.9515912645617763, 7.908) [b] 
(0.9515958725848975, 7.913) [b] 
(0.9515981765964582, 7.917) [b] 
(0.9515990982010825, 7.921) [b] 
(0.9516000198057067, 7.929) [b] 
(0.9516617673155311, 7.958) [b] 
(0.9516705225594614, 7.983) [b] 
(0.9516737481756462, 7.988) [b] 
(0.9516746697802705, 7.993) [b] 
(0.9516751305825826, 7.995) [b] 
(0.9516765129895189, 8.002) [b] 
(0.9516774345941432, 8.009) [b] 
(0.9516783561987675, 8.012) [b] 
(0.9516788170010796, 8.016) [b] 
(0.9516792778033917, 8.018) [b] 
(0.951680199408016, 8.031) [b] 
(0.9516811210126402, 8.069) [b] 
(0.9516857290357614, 8.083) [b] 
(0.9516884938496342, 8.086) [b] 
(0.9516894154542584, 8.09) [b] 
(0.9516917194658191, 8.093) [b] 
(0.9516935626750676, 8.096) [b] 
(0.9516940234773796, 8.099) [b] 
(0.9517009355120615, 8.114) [b] 
(0.9517018571166858, 8.119) [b] 
(0.9517050827328706, 8.123) [b] 
(0.9517073867444312, 8.131) [b] 
(0.9517078475467433, 8.137) [b] 
(0.9517083083490554, 8.14) [b] 
(0.9517087691513675, 8.142) [b] 
(0.9517092299536796, 8.161) [b] 
(0.9517101515583039, 8.165) [b] 
(0.9517119947675523, 8.168) [b] 
(0.9517124555698644, 8.172) [b] 
(0.9517152203837371, 8.176) [b] 
(0.9517156811860492, 8.179) [b] 
(0.9517161419883613, 8.189) [b] 
(0.9517166027906734, 8.192) [b] 
(0.951717290382123, 8.953) [b] 
(0.951719353156472, 8.973) [b] 
(0.9517245614898053, 9.161) [b] 
(0.9517259503786942, 9.19) [b] 
(0.951728728156472, 9.197) [b] 
(0.9517294226009164, 9.456) [b] 
(0.9517301101923661, 9.579) [b] 
(0.9517572369284772, 10.491) [b] 
(0.9517843636645884, 10.497) [b] 
(0.9518114904006996, 10.631) [b] 
(0.951919997345144, 10.632) [b] 
(0.9519471240812551, 10.633) [b] 
(0.9519742508173663, 10.634) [b] 
(0.9520013775534775, 10.636) [b] 
(0.9520285042895886, 10.652) [b] 
(0.952109884497922, 10.653) [b] 
(0.9521370112340332, 10.654) [b] 
(0.9522455181784776, 10.656) [b] 
(0.9522726449145887, 10.661) [b] 
(0.9522997716506999, 10.662) [b] 
(0.9523268983868111, 10.663) [b] 
(0.9523279400534778, 12.023) [b] 
(0.9523514274642256, 12.126) [b] 
(0.9523749148749735, 12.162) [b] 
(0.9523756024664232, 12.192) [b] 
(0.9523762900578728, 12.201) [b] 
(0.9523769776493225, 13.036) [b] 
(0.9523776720937669, 13.336) [b] 
(0.9523839220937669, 13.598) [b] 
(0.9523904406586396, 13.836) [b] 
(0.9523911282500893, 14.889) [b] 
(0.9523914754723115, 15.362) [b] 
(0.9523998088056449, 15.727) [b] 
(0.9524005032500893, 15.774) [b] 
(0.952419009115025, 17.814) [b] 
(0.9524197035594694, 17.858) [b] 
(0.952484889208197, 18.253) [b] 
(0.9524914077730697, 18.254) [b] 
(0.952493491106403, 20.385) [b] 
(0.9525000219318993, 20.803) [b] 
(0.9525620259001533, 20.852) [b] 
(0.9525703592334867, 24.194) [b] 
(0.9525776509001533, 24.241) [b] 
(0.9525783453445977, 24.507) [b] 
(0.9526018327553456, 28.243) [b] 
(0.9526021799775678, 28.836) [b] 
(0.9526230133109012, 29.008) [b] 
(0.9526438466442345, 29.12) [b] 
(0.9526502628232716, 30.077) [b] 
(0.9526506100454938, 31.437) [b] 
(0.9526634424035679, 31.731) [b] 
(0.952669858582605, 31.735) [b] 
(0.9526762747616421, 31.828) [b] 
(0.9526826909406791, 32.453) [b] 
(0.9526891071197162, 32.638) [b] 
(0.9526932737863829, 37.045) [b] 
(0.9526961208425269, 37.062) [b] 
(0.9526964680647491, 37.093) [b] 
(0.9527037597314157, 38.366) [b] 
(0.9527054958425268, 38.415) [b] 
(0.9527061902869712, 38.525) [b] 
(0.9527082736203045, 38.989) [b] 
(0.9527148044458009, 40.429) [b] 
(0.952721220624838, 42.278) [b] 
(0.9527225958077373, 42.615) [b] 
(0.9527239709906367, 42.813) [b] 
(0.9527260337649857, 42.828) [b] 
(0.95272672820943, 43.119) [b] 
(0.9527274226538744, 43.404) [b] 
(0.95273297820943, 48.601) [b] 
(0.9527364504316522, 48.65) [b] 
(0.9527371448760966, 48.747) [b] 
(0.9527472794634162, 56.591) [b] 
(0.9527607922465091, 56.598) [b] 
(0.9527675486380556, 56.751) [b] 
(0.9527709268338288, 56.757) [b] 
(0.9528108071379092, 57.94) [b] 
(0.9528114947293589, 57.941) [b] 
(0.9528142450951576, 57.942) [b] 
(0.9528156202780569, 57.943) [b] 
(0.9528197458267549, 57.951) [b] 
(0.9528259341498019, 57.985) [b] 
(0.9528266217412515, 57.989) [b] 
(0.9528273093327012, 58.008) [b] 
(0.9528279969241509, 58.014) [b] 
(0.9528313751199241, 58.705) [b] 
(0.9528716717027739, 59.524) [b] 
(0.9528757972514719, 59.873) [b] 
(0.9528764848429215, 60.291) [b] 
(0.9528778600258209, 60.314) [b] 
(0.9528785476172705, 60.834) [b] 
(0.9528792352087202, 60.84) [b] 
(0.9528799228001699, 61) [b] 
(0.9528806103916195, 61.537) [b] 
(0.9528812979830692, 61.544) [b] 
(0.9528826166133815, 64.291) [b] 
(0.9528829638356037, 68.253) [b] 
(0.9528833110578259, 69.982) [b] 
(0.952885047168937, 70.048) [b] 
(0.9528874777244926, 70.093) [b] 
(0.9528878249467148, 70.56) [b] 
(0.9529113123574626, 70.848) [b] 
(0.952913656589129, 71.078) [b] 
(0.9529140038113512, 71.14) [b] 
(0.9529591427002401, 81.399) [b] 
(0.9529598371446845, 81.458) [b] 
(0.9529601843669067, 81.774) [b] 
(0.9529605315891289, 81.829) [b] 
(0.9529643510335734, 82.129) [b] 
(0.9529900078480232, 84.584) [b] 
(0.9529920911813565, 86.107) [b] 
(0.9530177479958064, 87.3) [b] 
(0.9530180952180286, 87.619) [b] 
(0.953038928551362, 87.795) [b] 
(0.9530597618846953, 87.796) [b] 
(0.9530805952180287, 87.797) [b] 
(0.953101428551362, 87.798) [b] 
(0.9531222618846954, 87.799) [b] 
(0.9531430952180288, 87.8) [b] 
(0.9531639285513621, 87.801) [b] 
(0.9531847618846955, 87.802) [b] 
(0.9533305952180288, 87.804) [b] 
(0.9533514285513621, 87.805) [b] 
(0.9533722618846955, 87.808) [b] 
(0.9533930952180288, 87.81) [b] 
(0.9534139285513622, 87.827) [b] 
(0.9534347618846956, 87.83) [b] 
(0.9534555952180289, 87.865) [b] 
(0.9537472618846955, 87.938) [b] 
(0.9537680952180289, 88.006) [b] 
(0.9537889285513622, 89.247) [b] 
(0.9538097618846956, 101.249) [b] 
(0.9538305952180289, 101.252) [b] 
(0.9538514285513623, 101.255) [b] 
(0.9538722618846956, 101.26) [b] 
(0.953893095218029, 101.457) [b] 
(0.9539347618846957, 101.474) [b] 
(0.9539361854127677, 104.536) [b] 
(0.9539570187461011, 109.807) [b] 
(0.9539778520794344, 109.812) [b] 
(0.9539986854127678, 109.823) [b] 
(0.9541861854127678, 109.825) [b] 
(0.9542486854127677, 109.827) [b] 
(0.9542695187461011, 109.842) [b] 
(0.9542903520794345, 109.856) [b] 
(0.9543111854127678, 109.949) [b] 
(0.9543528520794345, 110.269) [b] 
(0.9543736854127679, 110.274) [b] 
(0.9544067541958366, 110.435) [b] 
(0.9544398229789054, 110.436) [b] 
(0.9544606563122388, 114.117) [b] 
(0.954461003534461, 119.639) [b] 
(0.9544616979789053, 119.684) [b] 
(0.9544620452011275, 119.728) [b] 
(0.9544623924233497, 119.774) [b] 
(0.9544682952011275, 119.896) [b] 
(0.9544759340900164, 119.947) [b] 
(0.9544776702011275, 120.2) [b] 
(0.9544985035344609, 122.464) [b] 
(0.9546443368677942, 122.465) [b] 
(0.9546651702011275, 122.469) [b] 
(0.9547068368677942, 122.471) [b] 
(0.9547276702011276, 122.472) [b] 
(0.9547901702011276, 122.473) [b] 
(0.9548318368677943, 122.474) [b] 
(0.954873503534461, 122.475) [b] 
(0.9549151702011277, 122.476) [b] 
(0.9549360035344611, 122.488) [b] 
(0.9549568368677944, 122.489) [b] 
(0.9549776702011278, 122.49) [b] 
(0.9549790590900167, 124.958) [b] 
(0.9549794063122389, 125.326) [b] 
(0.9549801007566833, 125.598) [b] 
(0.9549804479789055, 125.892) [b] 
(0.9550974891774073, 132.035) [b] 
(0.9552145303759092, 146.82) [b] 
(0.9552179085716824, 150.037) [b] 
(0.9552413959824303, 165.179) [b] 
(0.9552462570935414, 165.304) [b] 
(0.9552476459824303, 165.35) [b] 
(0.9552711333931782, 165.885) [b] 
(0.9552714806154003, 166.131) [b] 
(0.9552728695042892, 169.724) [b] 
(0.9552963569150371, 173.523) [b] 
(0.955319844325785, 184.653) [b] 
(0.9553433317365329, 189.014) [b] 
(0.9553668191472807, 191.868) [b] 
(0.9553675067387304, 201.003) [b] 
(0.95536819433018, 201.023) [b] 
(0.9553817359968467, 201.492) [b] 
(0.9553824235882964, 201.513) [b] 
(0.9553862430327409, 201.567) [b] 
(0.9554018680327409, 201.67) [b] 
(0.9554025556241905, 201.729) [b] 
(0.9554140139575239, 202.038) [b] 
(0.9554153891404232, 203.764) [b] 
(0.9554160767318729, 203.945) [b] 
(0.9554167643233226, 203.947) [b] 
(0.9554174519147722, 203.952) [b] 
(0.9554314905490939, 208.854) [b] 
(0.9554328657319933, 209.868) [b] 
(0.9554342409148926, 209.93) [b] 
(0.9554359770260037, 210.351) [b] 
(0.9554373659148926, 210.443) [b] 
(0.9554514045492143, 232.341) [b] 
(0.9554935204521793, 257.666) [b] 
(0.955507559086501, 257.667) [b] 
(0.9555215977208227, 257.671) [b] 
(0.9555356363551444, 257.678) [b] 
(0.9555612931695943, 257.73) [b] 
(0.9556174477068808, 257.732) [b] 
(0.9556314863412025, 257.733) [b] 
(0.9556455249755242, 258.105) [b] 
(0.9556595636098459, 258.109) [b] 
(0.9556736022441676, 259.645) [b] 
(0.9557157181471326, 260.206) [b] 
(0.9557297567814543, 260.21) [b] 
(0.955743795415776, 260.225) [b] 
(0.9557578340500977, 260.25) [b] 
(0.9557718726844194, 260.597) [b] 
(0.9557859113187411, 266.178) [b] 
(0.955811568133191, 266.495) [b] 
(0.955814693133191, 272.281) [b] 
(0.9558160820220799, 272.338) [b] 
(0.9558171236887466, 272.646) [b] 
(0.9558174709109688, 273.2) [b] 
(0.955817818133191, 273.536) [b] 
(0.9558188597998577, 273.736) [b] 
(0.9558195542443021, 274.093) [b] 
(0.9558522083717839, 285.109) [b] 
(0.9558587391972801, 285.115) [b] 
(0.9558590864195023, 307.832) [b] 
(0.955873125053824, 316.906) [b] 
(0.9558871636881457, 317.503) [b] 
(0.9561693886470949, 325.265) [b] 
(0.9561950454615448, 325.317) [b] 
(0.9562207022759946, 325.323) [b] 
(0.9562210494982168, 325.697) [b] 
(0.956221396720439, 327.433) [b] 
(0.9562470535348889, 327.678) [b] 
(0.9562727103493388, 327.838) [b] 
(0.9562737520160055, 328.278) [b] 
(0.9562994088304554, 328.297) [b] 
(0.9563001032748998, 328.396) [b] 
(0.9565823282338489, 335.648) [b] 
(0.9566079850482988, 335.703) [b] 
(0.9566336418627487, 335.709) [b] 
(0.9566592986771986, 338.136) [b] 
(0.9566849554916484, 338.303) [b] 
(0.9567106123060983, 338.786) [b] 
(0.9567362691205482, 349.649) [b] 
(0.9567619259349981, 357.715) [b] 
(0.956787582749448, 361.01) [b] 
(0.9568132395638979, 369.815) [b] 
(0.95681358678612, 458.756) [b] 
(0.9568142812305644, 459.368) [b] 
(0.9568187951194533, 466.082) [b] 
(0.95681983678612, 466.127) [b] 
(0.9568201840083422, 466.22) [b] 
(0.9568205312305644, 467.169) [b] 
(0.9568239094263377, 482.056) [b] 
(0.9568272876221109, 482.061) [b] 
(0.9568306658178841, 482.123) [b] 
(0.9568340440136573, 482.811) [b] 
(0.9568405748391537, 497.581) [b] 
(0.9568471056646501, 497.589) [b] 
(0.9568481473313168, 509.878) [b] 
(0.9568491889979835, 509.882) [b] 
(0.9568502306646502, 509.887) [b] 
(0.9568533556646502, 509.893) [b] 
(0.9568575223313169, 509.897) [b] 
(0.9568585639979836, 509.907) [b] 
(0.9568596056646503, 509.912) [b] 
(0.956860647331317, 509.926) [b] 
(0.9568616889979837, 510.079) [b] 
(0.9568627306646504, 510.094) [b] 
(0.9568637723313171, 511.454) [b] 
(0.9568648139979838, 511.46) [b] 
(0.9568658556646505, 514.767) [b] 
(0.9568762723313172, 571.715) [b] 
(0.9568825223313172, 571.775) [b] 
(0.9568832167757616, 571.888) [b] 
(0.956883911220206, 573.924) [b] 
(0.9568846056646504, 574.038) [b] 
(0.9568849528868726, 576.43) [b] 
(0.9568884251090948, 577.386) [b] 
(0.9568905084424281, 577.434) [b] 
(0.9568908556646503, 579.484) [b] 
(0.9568995362202058, 580.096) [b] 
(0.956899682734685, 588.943) [b] 
(0.9569010716235739, 695.815) [b] 
(0.9569017660680182, 701.632) [b] 
(0.956916050939694, 704.171) [b] 
(0.9569188157535667, 720.42) [b] 
(0.9569444725680166, 804.239) [b] 
(0.9569701293824665, 831.701) [b] 
(0.9569704766046887, 876.35) [b] 
(0.9569715182713554, 887.597) [b] 
(0.9569756849380221, 913.388) [b] 
(0.9569791571602443, 926.333) [b] 
(0.9569795043824665, 926.379) [b] 
(0.9569798516046887, 926.474) [b] 
(0.9569801988269109, 927.123) [b] 
(0.9569836710491331, 953.981) [b] 
(0.9569902018746295, 992.555) [b] 
(0.9569966180536665, 1053.16) [b] 
(0.9569973124981109, 1055.72) [b] 
(0.9570005381142958, 1060.41) [b] 
(0.9570040103365179, 1066.82) [b] 
(0.9570054338645899, 1073.23) [b] 
(0.9570231421979233, 1108.86) [b] 
(0.9570300866423678, 1108.92) [b] 
(0.9570314755312567, 1108.98) [b] 
(0.9570321699757011, 1109.04) [b] 
(0.9570328644201455, 1109.23) [b] 
(0.9570339060868122, 1109.29) [b] 
(0.9570342533090344, 1118.92) [b] 
(0.9570356768371063, 1134.34) [b] 
(0.957039843503773, 1321.81) [b] 
(0.9570412323926619, 1321.82) [b] 
(0.9570440101704397, 1321.83) [b] 
(0.9570447046148841, 1322) [b] 
(0.9570453990593285, 1322.34) [b] 
(0.9570488712815507, 1322.77) [b] 
(0.957050954614884, 1322.78) [b] 
(0.9570516490593284, 1323.05) [b] 
(0.9570530379482173, 1327.78) [b] 
(0.957067076582539, 1331.69) [b] 
(0.9570691599158723, 1334.79) [b] 
(0.9570705488047612, 1334.85) [b] 
(0.9570932393226, 1359.28) [b] 
(0.9570946145054994, 1359.3) [b] 
(0.9571028656028954, 1359.33) [b] 
(0.957103553194345, 1359.44) [b] 
(0.9571049283772444, 1360.27) [b] 
(0.957105615968694, 1361.25) [b] 
(0.9571063035601437, 1361.26) [b] 
(0.957107678743043, 1364.03) [b] 
(0.9571145546575397, 1364.06) [b] 
(0.9571152422489894, 1364.07) [b] 
(0.957115929840439, 1364.32) [b] 
(0.9571299684747607, 1365.56) [b] 
(0.9571306560662104, 1366.01) [b] 
(0.9571313436576601, 1366.36) [b] 
(0.9571327188405594, 1385.81) [b] 
(0.9571334064320091, 1441.69) [b] 
(0.9571347816149084, 1442.51) [b] 
(0.9571361567978077, 1442.68) [b] 
(0.9571368443892574, 1442.72) [b] 
(0.9571375319807071, 1462.47) [b] 
(0.9571382195721567, 1467.59) [b] 
(0.9571389071636064, 1467.73) [b] 
(0.957141657529405, 1473.43) [b] 
(0.9571423451208547, 1473.98) [b] 
(0.9571430327123044, 1474.51) [b] 
(0.9571450954866534, 1499.51) [b] 
(0.9571464706695527, 1503.83) [b] 
(0.9571471582610024, 1507.17) [b] 
(0.9571485334439017, 1634.51) [b] 
(0.9571523528883462, 1688.57) [b] 
(0.9571527001105684, 1688.66) [b] 
(0.957161380666124, 1690.34) [b] 
(0.9571652001105685, 1690.41) [b] 
(0.9571693667772352, 1690.47) [b] 
(0.9571704084439019, 1690.59) [b] 
(0.9571735334439019, 1691.22) [b] 
(0.9571756167772352, 1692.36) [b] 
(0.957178394555013, 1692.61) [b] 
(0.9571794362216797, 1696.16) [b] 
(0.9571795827361588, 1773.74) [b] 
(0.9571823605139366, 1838.63) [b] 
(0.9571834021806033, 1838.69) [b] 
(0.9571840966250477, 1838.92) [b] 
(0.9571854855139366, 1839.04) [b] 
(0.9571868744028255, 1840.3) [b] 
(0.9571872216250477, 1847.43) [b] 
(0.9572107090357955, 1856.78) [b] 
(0.9572247476701172, 1864.48) [b] 
(0.9572387863044389, 1911.35) [b] 
(0.9572453048693117, 2085.82) [b] 
(0.9572454513837908, 2103.26) [b] 
(0.957245798606013, 2308.91) [b] 
(0.9572461458282352, 2324.64) [b] 
(0.9572601844625569, 2442.39) [b] 
(0.9573841923990648, 2604.76) [b] 
(0.9575082003355727, 2604.77) [b] 
(0.9575702043038267, 2605.27) [b] 
(0.9576322082720806, 2612.14) [b] 
(0.9577492494705825, 2618.8) [b] 
(0.9578662906690844, 2619.22) [b] 
(0.957867332335751, 2855.77) [b] 
(0.9579213834681223, 2933.36) [b] 
(0.9579247616638955, 2933.39) [b] 
(0.957948409034308, 2942.29) [b] 
(0.9579551654258545, 2942.32) [b] 
(0.9579585436216277, 2942.45) [b] 
(0.9579619218174009, 2942.47) [b] 
(0.9579653000131741, 2942.77) [b] 
(0.9579793386474958, 3112.9) [b] 
(0.9580074159161391, 3113.11) [b] 
(0.9580105409161391, 3212.87) [b] 
(0.9580112353605835, 3224.26) [b] 
(0.9580129714716946, 3235.74) [b] 
(0.9580133186939168, 3246.64) [b] 
(0.9580147011008532, 3246.77) [b] 
(0.9580153955452976, 3247.03) [b] 
(0.958016089989742, 3247.7) [b] 
(0.9580167844341864, 3366.83) [b] 
},{(0.9534035625000002, 0.001) [c] 
(0.9534035625000002, 4.812143979166668) [c] 
(0.9534035625000002, 3600) [c] 
}}}{legend pos=north west}}
% 	\caption{\label{fig:cactus}Accuracy over time}
% \end{figure}

\subsection{Factor analysis}

\begin{table}[htbp]
\begin{center}
\begin{footnotesize}
\tabcolsep=1.5pt
\begin{tabular}{lrrrrrrrrrrrr}
\toprule
\multirow{2}{*}{$\mdepth$}&  \multicolumn{3}{c}{\budalg} & \multicolumn{3}{c}{\noheuristic} & \multicolumn{3}{c}{\nopreprocessing} & \multicolumn{3}{c}{\nolb}\\
\cmidrule(rr){2-4}\cmidrule(rr){5-7}\cmidrule(rr){8-10}\cmidrule(rr){11-13}
& \multicolumn{1}{c}{error} & \multicolumn{1}{c}{opt.} & \multicolumn{1}{c}{cpu} & \multicolumn{1}{c}{error} & \multicolumn{1}{c}{opt.} & \multicolumn{1}{c}{cpu$^*$} & \multicolumn{1}{c}{error} & \multicolumn{1}{c}{opt.} & \multicolumn{1}{c}{cpu$^*$} & \multicolumn{1}{c}{error} & \multicolumn{1}{c}{opt.} & \multicolumn{1}{c}{cpu$^*$} \\
\midrule

\texttt{3} & 1328 & 0.93 & 465 & \textbf{1326} & 0.93 & $\mathsmaller{\times}$0.84 & 1328 & 0.93 & $\mathsmaller{\times}$2.92 & 1328 & 0.93 & $\mathsmaller{\times}$1.01\\
\texttt{4} & 1144 & 0.61 & 594 & 1237 & 0.61 & $\mathsmaller{\times}$0.82 & 1140 & 0.54 & $\mathsmaller{\times}$5.15 & \textbf{1140} & 0.61 & $\mathsmaller{\times}$1.08\\
\texttt{5} & \textbf{1010} & 0.52 & 826 & 1154 & \textbf{0.54} & $\mathsmaller{\times}$1.10 & 1011 & 0.41 & $\mathsmaller{\times}$8.02 & 1010 & 0.52 & $\mathsmaller{\times}$1.23\\
\texttt{6} & \textbf{889} & 0.41 & 1139 & 1116 & 0.41 & $\mathsmaller{\times}$1789 & 891 & 0.35 & $\mathsmaller{\times}$7.52 & 889 & 0.41 & $\mathsmaller{\times}$1.28\\
\texttt{7} & \textbf{789} & 0.39 & 1215 & 1094 & 0.39 & $\mathsmaller{\times}$147 & 790 & 0.35 & $\mathsmaller{\times}$13 & 790 & 0.39 & $\mathsmaller{\times}$1.40\\
\texttt{8} & 704 & 0.43 & 792 & 1069 & 0.41 & $\mathsmaller{\times}$810 & \textbf{704} & 0.33 & $\mathsmaller{\times}$2.08 & 704 & 0.43 & $\mathsmaller{\times}$1.42\\
\texttt{9} & \textbf{637} & 0.43 & 788 & 1058 & 0.41 & $\mathsmaller{\times}$370 & 641 & 0.35 & $\mathsmaller{\times}$2.68 & 637 & 0.43 & $\mathsmaller{\times}$1.36\\
\texttt{10} & \textbf{575} & \textbf{0.52} & 678 & 1005 & 0.46 & $\mathsmaller{\times}$509 & 576 & 0.39 & $\mathsmaller{\times}$3.24 & 575 & 0.50 & $\mathsmaller{\times}$1.14\\
\bottomrule
\end{tabular}

\end{footnotesize}
\end{center}
\caption{\label{tab:factor} Factor analysis}
\end{table}


\bibliographystyle{plain}
\bibliography{src/references}

\clearpage

% \newgeometry{bottom=2cm,top=2cm,margin=1cm}

\section*{Appendix}

\begin{table}[htbp]%
\begin{center}%
\begin{footnotesize}%
\tabcolsep=2pt%
\begin{tabular}{lccrrrrrrrrrrrr}
\toprule
\multirow{2}{*}{}& && \multicolumn{2}{c}{\budalg} & \multicolumn{2}{c}{\murtree} & \multicolumn{2}{c}{\dleight} & \multicolumn{2}{c}{\cp} & \multicolumn{2}{c}{binoct} & \multicolumn{2}{c}{\cart}\\
\cmidrule(rr){4-5}\cmidrule(rr){6-7}\cmidrule(rr){8-9}\cmidrule(rr){10-11}\cmidrule(rr){12-13}\cmidrule(rr){14-15}
&\multirow{1}{*}{$\#ex.$} & \multirow{1}{*}{\#feat.} &  \multicolumn{1}{c}{error} & \multicolumn{1}{c}{cpu} & \multicolumn{1}{c}{error} & \multicolumn{1}{c}{cpu} & \multicolumn{1}{c}{error} & \multicolumn{1}{c}{cpu} & \multicolumn{1}{c}{error} & \multicolumn{1}{c}{cpu} & \multicolumn{1}{c}{error} & \multicolumn{1}{c}{cpu} & \multicolumn{1}{c}{error} & \multicolumn{1}{c}{cpu} \\
\midrule

\texttt{adult\_discretized} & \multicolumn{1}{r}{30299} & \multicolumn{1}{r}{59}  & 5020 & 0.43$^*$ & 5020 & 0.84$^*$ & 5020 & 10$^*$ & 5020 & 6.4$^*$ & 5600 & $\mathsmaller{\geq}1$h & 5758 & 0.05\\
\texttt{anneal} & \multicolumn{1}{r}{812} & \multicolumn{1}{r}{93}  & 112 & 0.03$^*$ & 112 & 0.14$^*$ & 112 & 2.4$^*$ & 112 & 6.0$^*$ & 123 & $\mathsmaller{\geq}1$h & 149 & 0.00\\
\texttt{audiology} & \multicolumn{1}{r}{216} & \multicolumn{1}{r}{148}  & 5 & 0.06$^*$ & 5 & 0.13$^*$ & 5 & 4.5$^*$ & 5 & 9.1$^*$ & 6 & $\mathsmaller{\geq}1$h & 6 & 0.00\\
\texttt{australian-credit} & \multicolumn{1}{r}{653} & \multicolumn{1}{r}{125}  & 73 & 0.14$^*$ & 73 & 0.35$^*$ & 73 & 9.6$^*$ & 73 & 14$^*$ & 87 & $\mathsmaller{\geq}1$h & 87 & 0.00\\
\texttt{bank} & \multicolumn{1}{r}{45211} & \multicolumn{1}{r}{9531}  & 4453 & 259 & 5289 & 0.84 & 4805 & $\mathsmaller{\geq}1$h & 4453 & $\mathsmaller{\geq}1$h & - & - & 4462 & 33\\
\texttt{breast-cancer} & \multicolumn{1}{r}{683} & \multicolumn{1}{r}{89}  & 24 & 0.16$^*$ & 24 & 0.07$^*$ & 24 & 0.98$^*$ & 24 & 5.7$^*$ & 25 & $\mathsmaller{\geq}1$h & 28 & 0.00\\
\texttt{breast-wisconsin} & \multicolumn{1}{r}{683} & \multicolumn{1}{r}{120}  & 15 & 0.05$^*$ & 15 & 0.20$^*$ & 15 & 6.4$^*$ & 15 & 11$^*$ & 18 & $\mathsmaller{\geq}1$h & 26 & 0.00\\
\texttt{car} & \multicolumn{1}{r}{1728} & \multicolumn{1}{r}{21}  & 192 & 0.01$^*$ & 192 & 0.01$^*$ & 192 & 0.04$^*$ & 192 & 1.7$^*$ & 192 & $\mathsmaller{\geq}1$h & 202 & 0.00\\
\texttt{compas\_discretized} & \multicolumn{1}{r}{6167} & \multicolumn{1}{r}{25}  & 2004 & 0.00$^*$ & 2004 & 0.06$^*$ & 2004 & 0.23$^*$ & 2004 & 1.8$^*$ & 2032 & $\mathsmaller{\geq}1$h & 2072 & 0.01\\
\texttt{diabetes} & \multicolumn{1}{r}{768} & \multicolumn{1}{r}{112}  & 162 & 0.09$^*$ & 162 & 0.37$^*$ & 162 & 11$^*$ & 162 & 12$^*$ & 165 & $\mathsmaller{\geq}1$h & 177 & 0.00\\
\texttt{forest-fires} & \multicolumn{1}{r}{517} & \multicolumn{1}{r}{989}  & 193 & 20$^*$ & 193 & 9.6$^*$ & - & - & 193 & 2836$^*$ & 198 & $\mathsmaller{\geq}1$h & 198 & 0.01\\
\texttt{german-credit} & \multicolumn{1}{r}{1000} & \multicolumn{1}{r}{112}  & 236 & 0.26$^*$ & 236 & 0.38$^*$ & 236 & 7.7$^*$ & 236 & 13$^*$ & 244 & $\mathsmaller{\geq}1$h & 251 & 0.00\\
\texttt{heart-cleveland} & \multicolumn{1}{r}{296} & \multicolumn{1}{r}{95}  & 41 & 0.05$^*$ & 41 & 0.12$^*$ & 41 & 3.5$^*$ & 41 & 6.8$^*$ & 42 & $\mathsmaller{\geq}1$h & 43 & 0.00\\
\texttt{hepatitis} & \multicolumn{1}{r}{137} & \multicolumn{1}{r}{68}  & 10 & 0.00$^*$ & 10 & 0.03$^*$ & 10 & 1.2$^*$ & 10 & 3.9$^*$ & 10 & $\mathsmaller{\geq}1$h & 16 & 0.00\\
\texttt{hypothyroid} & \multicolumn{1}{r}{3247} & \multicolumn{1}{r}{88}  & 61 & 0.07$^*$ & 61 & 0.41$^*$ & 61 & 4.4$^*$ & 61 & 6.6$^*$ & 62 & $\mathsmaller{\geq}1$h & 62 & 0.01\\
\texttt{ionosphere} & \multicolumn{1}{r}{351} & \multicolumn{1}{r}{445}  & 22 & 3.8$^*$ & 22 & 12$^*$ & 22 & 410$^*$ & 22 & 460$^*$ & 27 & $\mathsmaller{\geq}1$h & 29 & 0.01\\
\texttt{kr-vs-kp} & \multicolumn{1}{r}{3196} & \multicolumn{1}{r}{73}  & 198 & 0.09$^*$ & 198 & 0.22$^*$ & 198 & 2.4$^*$ & 198 & 4.8$^*$ & 375 & $\mathsmaller{\geq}1$h & 306 & 0.01\\
\texttt{letter} & \multicolumn{1}{r}{20000} & \multicolumn{1}{r}{224}  & 369 & 10$^*$ & 369 & 34$^*$ & 369 & 443$^*$ & 369 & 158$^*$ & 813 & 1251 & 677 & 0.17\\
\texttt{lymph} & \multicolumn{1}{r}{148} & \multicolumn{1}{r}{68}  & 12 & 0.01$^*$ & 12 & 0.03$^*$ & 12 & 0.76$^*$ & 12 & 3.7$^*$ & 14 & $\mathsmaller{\geq}1$h & 17 & 0.00\\
\texttt{mnist\_0} & \multicolumn{1}{r}{60000} & \multicolumn{1}{r}{784}  & 2557 & 1994$^*$ & 2557 & 568$^*$ & 3319 & $\mathsmaller{\geq}1$h & 2557 & $\mathsmaller{\geq}1$h & - & - & 3329 & 2.5\\
\texttt{mnist\_1} & \multicolumn{1}{r}{60000} & \multicolumn{1}{r}{784}  & 3462 & 1896$^*$ & 3462 & 538$^*$ & 4552 & $\mathsmaller{\geq}1$h & 3462 & $\mathsmaller{\geq}1$h & - & - & 3534 & 2.5\\
\texttt{mnist\_2} & \multicolumn{1}{r}{60000} & \multicolumn{1}{r}{784}  & 3938 & 1946$^*$ & 3938 & 672$^*$ & 4289 & $\mathsmaller{\geq}1$h & 3938 & $\mathsmaller{\geq}1$h & - & - & 4530 & 2.6\\
\texttt{mnist\_3} & \multicolumn{1}{r}{60000} & \multicolumn{1}{r}{784}  & 4354 & 2054$^*$ & 4354 & 644$^*$ & 4974 & $\mathsmaller{\geq}1$h & 4354 & $\mathsmaller{\geq}1$h & - & - & 6131 & 2.5\\
\texttt{mnist\_4} & \multicolumn{1}{r}{60000} & \multicolumn{1}{r}{784}  & 4729 & 2070$^*$ & 4729 & 700$^*$ & 5580 & $\mathsmaller{\geq}1$h & 4729 & $\mathsmaller{\geq}1$h & - & - & 5037 & 2.6\\
\texttt{mnist\_5} & \multicolumn{1}{r}{60000} & \multicolumn{1}{r}{784}  & 3539 & 2095$^*$ & 3539 & 715$^*$ & 4379 & $\mathsmaller{\geq}1$h & 3539 & $\mathsmaller{\geq}1$h & - & - & 4032 & 2.6\\
\texttt{mnist\_6} & \multicolumn{1}{r}{60000} & \multicolumn{1}{r}{784}  & 2756 & 1916$^*$ & 2756 & 664$^*$ & 2756 & $\mathsmaller{\geq}1$h & 2756 & $\mathsmaller{\geq}1$h & - & - & 2893 & 2.6\\
\texttt{mnist\_7} & \multicolumn{1}{r}{60000} & \multicolumn{1}{r}{784}  & 3483 & 1928$^*$ & 3483 & 570$^*$ & 4546 & $\mathsmaller{\geq}1$h & 3483 & $\mathsmaller{\geq}1$h & - & - & 3788 & 2.5\\
\texttt{mnist\_8} & \multicolumn{1}{r}{60000} & \multicolumn{1}{r}{784}  & 3583 & 2061$^*$ & 3583 & 593$^*$ & 4609 & $\mathsmaller{\geq}1$h & 3583 & $\mathsmaller{\geq}1$h & - & - & 4250 & 2.6\\
\texttt{mnist\_9} & \multicolumn{1}{r}{60000} & \multicolumn{1}{r}{784}  & 4590 & 2039$^*$ & 4590 & 746$^*$ & 5253 & $\mathsmaller{\geq}1$h & 4590 & $\mathsmaller{\geq}1$h & - & - & 5355 & 2.6\\
\texttt{mushroom} & \multicolumn{1}{r}{8124} & \multicolumn{1}{r}{119}  & 8 & 0.79$^*$ & 8 & 0.53$^*$ & 8 & 6.3$^*$ & 8 & 8.4$^*$ & 180 & $\mathsmaller{\geq}1$h & 280 & 0.02\\
\texttt{pendigits} & \multicolumn{1}{r}{7494} & \multicolumn{1}{r}{216}  & 47 & 3.3$^*$ & 47 & 11$^*$ & 47 & 134$^*$ & 47 & 70$^*$ & 477 & $\mathsmaller{\geq}1$h & 51 & 0.05\\
\texttt{primary-tumor} & \multicolumn{1}{r}{336} & \multicolumn{1}{r}{31}  & 46 & 0.00$^*$ & 46 & 0.01$^*$ & 46 & 0.14$^*$ & 46 & 2.0$^*$ & 46 & $\mathsmaller{\geq}1$h & 53 & 0.00\\
\texttt{segment} & \multicolumn{1}{r}{2310} & \multicolumn{1}{r}{235}  & 0 & 0.03$^*$ & 0 & 0.13$^*$ & 0 & 2.3$^*$ & 0 & 4.1$^*$ & 4 & $\mathsmaller{\geq}1$h & 5 & 0.01\\
\texttt{soybean} & \multicolumn{1}{r}{630} & \multicolumn{1}{r}{50}  & 29 & 0.01$^*$ & 29 & 0.02$^*$ & 29 & 0.29$^*$ & 29 & 2.3$^*$ & 31 & $\mathsmaller{\geq}1$h & 47 & 0.00\\
\texttt{splice-1} & \multicolumn{1}{r}{3190} & \multicolumn{1}{r}{287}  & 224 & 9.8$^*$ & 224 & 5.3$^*$ & 224 & 114$^*$ & 224 & 173$^*$ & 453 & $\mathsmaller{\geq}1$h & 279 & 0.03\\
\texttt{surgical-deepnet} & \multicolumn{1}{r}{14635} & \multicolumn{1}{r}{6047}  & 2512 & 953 & 2512 & 3523 & - & - & 2512 & $\mathsmaller{\geq}1$h & - & - & 2924 & 5.7\\
\texttt{taiwan\_binarised} & \multicolumn{1}{r}{30000} & \multicolumn{1}{r}{205}  & 5326 & 48$^*$ & 5326 & 45$^*$ & 5326 & 526$^*$ & 5326 & 190$^*$ & 6636 & 1639 & 5346 & 0.26\\
\texttt{tic-tac-toe} & \multicolumn{1}{r}{958} & \multicolumn{1}{r}{27}  & 216 & 0.01$^*$ & 216 & 0.02$^*$ & 216 & 0.13$^*$ & 216 & 1.8$^*$ & 232 & $\mathsmaller{\geq}1$h & 236 & 0.00\\
\texttt{titanic} & \multicolumn{1}{r}{887} & \multicolumn{1}{r}{333}  & 143 & 6.7$^*$ & 143 & 11$^*$ & 143 & 167$^*$ & 143 & 173$^*$ & 150 & $\mathsmaller{\geq}1$h & 148 & 0.01\\
\texttt{vehicle} & \multicolumn{1}{r}{846} & \multicolumn{1}{r}{252}  & 26 & 0.93$^*$ & 26 & 2.2$^*$ & 26 & 64$^*$ & 26 & 66$^*$ & 42 & $\mathsmaller{\geq}1$h & 66 & 0.01\\
\texttt{vote} & \multicolumn{1}{r}{435} & \multicolumn{1}{r}{48}  & 12 & 0.02$^*$ & 12 & 0.02$^*$ & 12 & 0.34$^*$ & 12 & 2.6$^*$ & 13 & $\mathsmaller{\geq}1$h & 14 & 0.00\\
\texttt{weather-aus} & \multicolumn{1}{r}{142193} & \multicolumn{1}{r}{4759}  & 1756 & 14 & 1756 & 611 & - & - & 1756 & $\mathsmaller{\geq}1$h & - & - & 1761 & 20\\
\texttt{wine1} & \multicolumn{1}{r}{178} & \multicolumn{1}{r}{1276}  & 43 & 16$^*$ & 43 & 9.0$^*$ & - & - & 43 & $\mathsmaller{\geq}1$h & 44 & $\mathsmaller{\geq}1$h & 45 & 0.00\\
\texttt{wine2} & \multicolumn{1}{r}{178} & \multicolumn{1}{r}{1276}  & 49 & 17$^*$ & 49 & 5.8$^*$ & - & - & 49 & $\mathsmaller{\geq}1$h & 57 & $\mathsmaller{\geq}1$h & 52 & 0.00\\
\texttt{wine3} & \multicolumn{1}{r}{178} & \multicolumn{1}{r}{1276}  & 33 & 16$^*$ & 33 & 8.4$^*$ & - & - & 33 & $\mathsmaller{\geq}1$h & 35 & $\mathsmaller{\geq}1$h & 35 & 0.00\\
\texttt{yeast} & \multicolumn{1}{r}{1484} & \multicolumn{1}{r}{89}  & 403 & 0.07$^*$ & 403 & 0.34$^*$ & 403 & 6.1$^*$ & 403 & 7.7$^*$ & 434 & $\mathsmaller{\geq}1$h & 418 & 0.00\\
\bottomrule
\end{tabular}
%
\end{footnotesize}%
\end{center}%
\caption{\label{tab:all3} Comparison with state of the art: depth 3}%
\end{table}%

\begin{table}[htbp]
\begin{center}
\begin{footnotesize}
\tabcolsep=2pt
\begin{tabular}{lrrrrrrrrrrrr}
\toprule
\multirow{2}{*}{}&  \multicolumn{2}{c}{\budalg} & \multicolumn{2}{c}{\murtree} & \multicolumn{2}{c}{\dleight} & \multicolumn{2}{c}{\cp} & \multicolumn{2}{c}{binoct} & \multicolumn{2}{c}{\cart}\\
\cmidrule(rr){2-3}\cmidrule(rr){4-5}\cmidrule(rr){6-7}\cmidrule(rr){8-9}\cmidrule(rr){10-11}\cmidrule(rr){12-13}
& \multicolumn{1}{c}{error} & \multicolumn{1}{c}{cpu} & \multicolumn{1}{c}{error} & \multicolumn{1}{c}{cpu} & \multicolumn{1}{c}{error} & \multicolumn{1}{c}{cpu} & \multicolumn{1}{c}{error} & \multicolumn{1}{c}{cpu} & \multicolumn{1}{c}{error} & \multicolumn{1}{c}{cpu} & \multicolumn{1}{c}{error} & \multicolumn{1}{c}{cpu} \\
\midrule

\texttt{hepatitis} & 3 & 0.32$^*$ & 3 & 0.73$^*$ & 3 & 28$^*$ & 3 & 70$^*$ & 11 & 510 & 12 & 0.00\\
\texttt{lymph} & 3 & 0.74$^*$ & 3 & 0.63$^*$ & 3 & 14$^*$ & 3 & 64$^*$ & 7 & 2987 & 10 & 0.00\\
\texttt{wine1} & 37 & 1674 & 37 & 1831$^*$ & - & - & 39 & $\mathsmaller{\geq}1$h & 45 & 3506 & 42 & 0.01\\
\texttt{wine2} & 43 & 17 & 43 & 1833$^*$ & - & - & 46 & $\mathsmaller{\geq}1$h & 57 & 3232 & 47 & 0.01\\
\texttt{wine3} & 28 & 33 & 28 & 2537$^*$ & - & - & 30 & $\mathsmaller{\geq}1$h & 32 & 3388 & 32 & 0.01\\
\texttt{audiology} & 1 & 4.0$^*$ & 1 & 6.4$^*$ & 1 & 128$^*$ & 1 & 773$^*$ & 2 & 2687 & 3 & 0.00\\
\texttt{heart-cleveland} & 25 & 3.1$^*$ & 25 & 4.8$^*$ & 25 & 154$^*$ & 25 & 391$^*$ & 37 & 2750 & 38 & 0.00\\
\texttt{primary-tumor} & 34 & 0.03$^*$ & 34 & 0.11$^*$ & 34 & 2.0$^*$ & 34 & 5.6$^*$ & 38 & 3132 & 44 & 0.00\\
\texttt{ionosphere} & 7 & 730$^*$ & 7 & 1683$^*$ & - & - & 8 & $\mathsmaller{\geq}1$h & 24 & 751 & 27 & 0.01\\
\texttt{vote} & 5 & 1.2$^*$ & 5 & 0.50$^*$ & 5 & 7.6$^*$ & 5 & 21$^*$ & 12 & 3311 & 8 & 0.00\\
\texttt{forest-fires} & 173 & 15 & \textbf{171} & 2907$^*$ & - & - & 179 & $\mathsmaller{\geq}1$h & 196 & 3356 & 186 & 0.01\\
\texttt{soybean} & 14 & 0.62$^*$ & 14 & 0.46$^*$ & 14 & 5.1$^*$ & 14 & 22$^*$ & 22 & 2906 & 32 & 0.00\\
\texttt{australian-credit} & 56 & 10$^*$ & 56 & 24$^*$ & 56 & 470$^*$ & 56 & 1170$^*$ & 83 & 3258 & 74 & 0.00\\
\texttt{breast-cancer} & 16 & 9.6$^*$ & 16 & 2.9$^*$ & 16 & 28$^*$ & 16 & 219$^*$ & 22 & 2746 & 21 & 0.00\\
\texttt{breast-wisconsin} & 7 & 3.1$^*$ & 7 & 9.3$^*$ & 7 & 245$^*$ & 7 & 662$^*$ & 15 & 3460 & 16 & 0.00\\
\texttt{diabetes} & 137 & 5.7$^*$ & 137 & 22$^*$ & 137 & 550$^*$ & 137 & 1001$^*$ & 180 & 2663 & 166 & 0.00\\
\texttt{anneal} & 91 & 1.5$^*$ & 91 & 5.0$^*$ & 91 & 102$^*$ & 91 & 193$^*$ & 108 & 2954 & 135 & 0.00\\
\texttt{vehicle} & 12 & 71$^*$ & 12 & 172$^*$ & - & - & 12 & $\mathsmaller{\geq}1$h & 30 & 3410 & 28 & 0.01\\
\texttt{titanic} & 119 & 1604$^*$ & 119 & 2104$^*$ & - & - & 119 & $\mathsmaller{\geq}1$h & 135 & 3501 & 134 & 0.01\\
\texttt{tic-tac-toe} & 137 & 0.38$^*$ & 137 & 0.26$^*$ & 137 & 1.8$^*$ & 137 & 7.2$^*$ & 162 & 2511 & 150 & 0.00\\
\texttt{german-credit} & 204 & 28$^*$ & 204 & 27$^*$ & 204 & 423$^*$ & 204 & 1008$^*$ & 236 & 3306 & 231 & 0.00\\
\texttt{yeast} & 366 & 3.4$^*$ & 366 & 18$^*$ & 366 & 257$^*$ & 366 & 386$^*$ & 438 & 888 & 394 & 0.01\\
\texttt{car} & 136 & 0.19$^*$ & 136 & 0.16$^*$ & 136 & 0.36$^*$ & 136 & 2.8$^*$ & 178 & 871 & 178 & 0.00\\
\texttt{segment} & 0 & 0.00$^*$ & 0 & 0.02$^*$ & 0 & 1.6$^*$ & 0 & 2.5$^*$ & 1 & 3501 & 1 & 0.01\\
\texttt{splice-1} & 141 & 3241$^*$ & 141 & 644$^*$ & - & - & 141 & $\mathsmaller{\geq}1$h & 568 & 3416 & 141 & 0.03\\
\texttt{kr-vs-kp} & 144 & 2.8$^*$ & 144 & 6.9$^*$ & 144 & 88$^*$ & 144 & 141$^*$ & 189 & 2850 & 189 & 0.01\\
\texttt{hypothyroid} & 53 & 2.9$^*$ & 53 & 16$^*$ & 53 & 181$^*$ & 53 & 254$^*$ & 55 & 3071 & 53 & 0.01\\
\texttt{compas\_discretized} & 1954 & 0.07$^*$ & 1954 & 1.0$^*$ & 1954 & 3.5$^*$ & 1954 & 6.3$^*$ & 1991 & 3390 & 1997 & 0.01\\
\texttt{pendigits} & 13 & 230$^*$ & 13 & 833$^*$ & - & - & 14 & $\mathsmaller{\geq}1$h & 780 & 0.00 & 25 & 0.07\\
\texttt{mushroom} & 0 & 0.00$^*$ & 0 & 0.03$^*$ & 0 & 41$^*$ & 0 & 0.07$^*$ & 192 & 3354 & 4 & 0.02\\
\texttt{surgical-deepnet} & \textbf{2269} & 49 & 2506 & 489 & - & - & 3690 & $\mathsmaller{\geq}1$h & - & - & 2704 & 6.2\\
\texttt{letter} & 261 & 1185$^*$ & 261 & 2956$^*$ & 335 & $\mathsmaller{\geq}1$h & 261 & $\mathsmaller{\geq}1$h & 813 & 0.00 & 462 & 0.20\\
\texttt{taiwan\_binarised} & 5273 & 6.2 & 5273 & 37 & 5307 & $\mathsmaller{\geq}1$h & 5273 & $\mathsmaller{\geq}1$h & 6521 & 75 & 5306 & 0.27\\
\texttt{adult\_discretized} & 4609 & 14$^*$ & 4609 & 30$^*$ & 4609 & 271$^*$ & 4609 & 246$^*$ & 5659 & 3392 & 5022 & 0.06\\
\texttt{bank} & \textbf{4314} & 290 & 4686 & 2.5 & 4808 & $\mathsmaller{\geq}1$h & 5289 & $\mathsmaller{\geq}1$h & - & - & 4420 & 32\\
\texttt{mnist\_8} & 3165 & 1206 & 3165 & 564 & 4609 & $\mathsmaller{\geq}1$h & 5851 & $\mathsmaller{\geq}1$h & - & - & 3987 & 4.5\\
\texttt{mnist\_9} & 3977 & 2061 & 3977 & 1211 & 5252 & $\mathsmaller{\geq}1$h & 5949 & $\mathsmaller{\geq}1$h & - & - & 4231 & 3.1\\
\texttt{mnist\_0} & 2173 & 2158 & \textbf{1951} & 3542 & 3319 & $\mathsmaller{\geq}1$h & 5923 & $\mathsmaller{\geq}1$h & - & - & 2311 & 3.8\\
\texttt{mnist\_6} & 1940 & 2752 & 1940 & 1300 & 2755 & $\mathsmaller{\geq}1$h & 5918 & $\mathsmaller{\geq}1$h & - & - & 2251 & 4.1\\
\texttt{mnist\_5} & 3312 & 219 & \textbf{3085} & 1942 & 4373 & $\mathsmaller{\geq}1$h & 5421 & $\mathsmaller{\geq}1$h & - & - & 3648 & 3.8\\
\texttt{mnist\_3} & 3485 & 2225 & 3485 & 1262 & 4900 & $\mathsmaller{\geq}1$h & 6131 & $\mathsmaller{\geq}1$h & - & - & 4367 & 4.9\\
\texttt{mnist\_2} & 3358 & 169 & \textbf{3116} & 2748 & 4289 & $\mathsmaller{\geq}1$h & 5958 & $\mathsmaller{\geq}1$h & - & - & 4326 & 3.1\\
\texttt{mnist\_4} & 3670 & 2476 & \textbf{3615} & 1605 & 5580 & $\mathsmaller{\geq}1$h & 5842 & $\mathsmaller{\geq}1$h & - & - & 4129 & 3.2\\
\texttt{mnist\_7} & 2793 & 52 & 2793 & 640 & 4546 & $\mathsmaller{\geq}1$h & 6265 & $\mathsmaller{\geq}1$h & - & - & 3218 & 3.9\\
\texttt{mnist\_1} & 2332 & 2248 & 2332 & 671 & 4551 & $\mathsmaller{\geq}1$h & 6742 & $\mathsmaller{\geq}1$h & - & - & 2501 & 3.6\\
\texttt{weather-aus} & \textbf{1749} & 2525 & 1750 & 1243 & - & - & 1752 & $\mathsmaller{\geq}1$h & - & - & 1761 & 20\\
\bottomrule
\end{tabular}

\end{footnotesize}
\end{center}
\caption{\label{tab:all4} Comparison with state of the art: depth 4}
\end{table}

\begin{table}[htbp]
\begin{center}
\begin{footnotesize}
\tabcolsep=2pt
\begin{tabular}{lccrrrrrrrrrrrr}
\toprule
\multirow{2}{*}{}& && \multicolumn{2}{c}{\budalg} & \multicolumn{2}{c}{\murtree} & \multicolumn{2}{c}{\dleight} & \multicolumn{2}{c}{\cp} & \multicolumn{2}{c}{binoct} & \multicolumn{2}{c}{\cart}\\
\cmidrule(rr){4-5}\cmidrule(rr){6-7}\cmidrule(rr){8-9}\cmidrule(rr){10-11}\cmidrule(rr){12-13}\cmidrule(rr){14-15}
&\multirow{1}{*}{$\#ex.$} & \multirow{1}{*}{\#feat.} &  \multicolumn{1}{c}{error} & \multicolumn{1}{c}{cpu} & \multicolumn{1}{c}{error} & \multicolumn{1}{c}{cpu} & \multicolumn{1}{c}{error} & \multicolumn{1}{c}{cpu} & \multicolumn{1}{c}{error} & \multicolumn{1}{c}{cpu} & \multicolumn{1}{c}{error} & \multicolumn{1}{c}{cpu} & \multicolumn{1}{c}{error} & \multicolumn{1}{c}{cpu} \\
\midrule

\texttt{adult\_discretized} & \multicolumn{1}{r}{30299} & \multicolumn{1}{r}{59}  & 4423 & 725$^*$ & 4423 & 794$^*$ & 4442 & $\mathsmaller{\geq}1$h & 4423 & $\mathsmaller{\geq}1$h & 7511 & 452 & 4728 & 0.08\\
\texttt{anneal} & \multicolumn{1}{r}{812} & \multicolumn{1}{r}{93}  & 70 & 44$^*$ & 70 & 148$^*$ & - & - & 75 & $\mathsmaller{\geq}1$h & 101 & $\mathsmaller{\geq}1$h & 123 & 0.00\\
\texttt{audiology} & \multicolumn{1}{r}{216} & \multicolumn{1}{r}{148}  & 0 & 0.00$^*$ & 0 & 0.02$^*$ & 0 & 0.05$^*$ & 0 & 7.0$^*$ & 1 & $\mathsmaller{\geq}1$h & 2 & 0.00\\
\texttt{australian-credit} & \multicolumn{1}{r}{653} & \multicolumn{1}{r}{125}  & 39 & 658$^*$ & 39 & 872$^*$ & - & - & 40 & $\mathsmaller{\geq}1$h & 93 & 3387 & 64 & 0.00\\
\texttt{bank} & \multicolumn{1}{r}{45211} & \multicolumn{1}{r}{9531}  & \textbf{4187} & 1152 & 4365 & 2093 & 4809 & $\mathsmaller{\geq}1$h & 5289 & $\mathsmaller{\geq}1$h & - & - & 4358 & 47\\
\texttt{breast-cancer} & \multicolumn{1}{r}{683} & \multicolumn{1}{r}{89}  & 6 & 725$^*$ & 6 & 72$^*$ & 6 & 438$^*$ & 6 & $\mathsmaller{\geq}1$h & 14 & $\mathsmaller{\geq}1$h & 16 & 0.00\\
\texttt{breast-wisconsin} & \multicolumn{1}{r}{683} & \multicolumn{1}{r}{120}  & 0 & 20$^*$ & 0 & 72$^*$ & - & - & 1 & $\mathsmaller{\geq}1$h & 16 & $\mathsmaller{\geq}1$h & 13 & 0.00\\
\texttt{car} & \multicolumn{1}{r}{1728} & \multicolumn{1}{r}{21}  & 86 & 2.4$^*$ & 86 & 1.2$^*$ & 86 & 2.7$^*$ & 86 & 21$^*$ & 138 & $\mathsmaller{\geq}1$h & 106 & 0.01\\
\texttt{compas\_discretized} & \multicolumn{1}{r}{6167} & \multicolumn{1}{r}{25}  & 1919 & 1.1$^*$ & 1919 & 11$^*$ & 1919 & 26$^*$ & 1919 & 77$^*$ & 1952 & $\mathsmaller{\geq}1$h & 1968 & 0.01\\
\texttt{diabetes} & \multicolumn{1}{r}{768} & \multicolumn{1}{r}{112}  & 106 & 312$^*$ & 106 & 920$^*$ & - & - & 107 & $\mathsmaller{\geq}1$h & 189 & 3174 & 141 & 0.00\\
\texttt{forest-fires} & \multicolumn{1}{r}{517} & \multicolumn{1}{r}{989}  & 156 & 777 & \textbf{149} & 2977 & - & - & 172 & $\mathsmaller{\geq}1$h & 270 & 107 & 177 & 0.01\\
\texttt{german-credit} & \multicolumn{1}{r}{1000} & \multicolumn{1}{r}{112}  & 161 & 2741$^*$ & 161 & 973$^*$ & - & - & 161 & $\mathsmaller{\geq}1$h & 294 & 515 & 209 & 0.01\\
\texttt{heart-cleveland} & \multicolumn{1}{r}{296} & \multicolumn{1}{r}{95}  & 7 & 93$^*$ & 7 & 101$^*$ & - & - & 7 & $\mathsmaller{\geq}1$h & 26 & $\mathsmaller{\geq}1$h & 26 & 0.00\\
\texttt{hepatitis} & \multicolumn{1}{r}{137} & \multicolumn{1}{r}{68}  & 0 & 0.05$^*$ & 0 & 0.18$^*$ & 0 & 71$^*$ & 0 & 12$^*$ & 6 & $\mathsmaller{\geq}1$h & 8 & 0.00\\
\texttt{hypothyroid} & \multicolumn{1}{r}{3247} & \multicolumn{1}{r}{88}  & 44 & 87$^*$ & 44 & 343$^*$ & - & - & 45 & $\mathsmaller{\geq}1$h & 134 & 420 & 50 & 0.01\\
\texttt{ionosphere} & \multicolumn{1}{r}{351} & \multicolumn{1}{r}{445}  & 0 & 506$^*$ & 0 & 1340$^*$ & - & - & 4 & $\mathsmaller{\geq}1$h & 66 & 505 & 17 & 0.01\\
\texttt{kr-vs-kp} & \multicolumn{1}{r}{3196} & \multicolumn{1}{r}{73}  & 81 & 65$^*$ & 81 & 150$^*$ & - & - & 81 & $\mathsmaller{\geq}1$h & 1527 & 564 & 189 & 0.01\\
\texttt{letter} & \multicolumn{1}{r}{20000} & \multicolumn{1}{r}{224}  & \textbf{168} & 3082 & 190 & 549 & 352 & $\mathsmaller{\geq}1$h & 813 & $\mathsmaller{\geq}1$h & - & - & 335 & 0.32\\
\texttt{lymph} & \multicolumn{1}{r}{148} & \multicolumn{1}{r}{68}  & 0 & 0.00$^*$ & 0 & 0.00$^*$ & 0 & 14$^*$ & 0 & 2.7$^*$ & 7 & $\mathsmaller{\geq}1$h & 4 & 0.00\\
\texttt{mnist\_0} & \multicolumn{1}{r}{60000} & \multicolumn{1}{r}{784}  & \textbf{1714} & 284 & 2066 & 2149 & 3319 & $\mathsmaller{\geq}1$h & 5923 & $\mathsmaller{\geq}1$h & - & - & 2021 & 4.5\\
\texttt{mnist\_1} & \multicolumn{1}{r}{60000} & \multicolumn{1}{r}{784}  & \textbf{1585} & 3111 & 1790 & 993 & 4029 & $\mathsmaller{\geq}1$h & 6742 & $\mathsmaller{\geq}1$h & - & - & 1965 & 3.6\\
\texttt{mnist\_2} & \multicolumn{1}{r}{60000} & \multicolumn{1}{r}{784}  & 3118 & 3230 & \textbf{2963} & 2671 & 4026 & $\mathsmaller{\geq}1$h & 5958 & $\mathsmaller{\geq}1$h & - & - & 3676 & 3.9\\
\texttt{mnist\_3} & \multicolumn{1}{r}{60000} & \multicolumn{1}{r}{784}  & \textbf{2893} & 1936 & 3184 & 398 & 4900 & $\mathsmaller{\geq}1$h & 6131 & $\mathsmaller{\geq}1$h & - & - & 3768 & 6.0\\
\texttt{mnist\_4} & \multicolumn{1}{r}{60000} & \multicolumn{1}{r}{784}  & \textbf{2864} & 708 & 3164 & 107 & 5580 & $\mathsmaller{\geq}1$h & 5842 & $\mathsmaller{\geq}1$h & - & - & 3619 & 4.5\\
\texttt{mnist\_5} & \multicolumn{1}{r}{60000} & \multicolumn{1}{r}{784}  & \textbf{3138} & 2411 & 3163 & 2007 & 4376 & $\mathsmaller{\geq}1$h & 5421 & $\mathsmaller{\geq}1$h & - & - & 3479 & 5.8\\
\texttt{mnist\_6} & \multicolumn{1}{r}{60000} & \multicolumn{1}{r}{784}  & \textbf{1485} & 2097 & 1653 & 646 & 2753 & $\mathsmaller{\geq}1$h & 5918 & $\mathsmaller{\geq}1$h & - & - & 1900 & 4.4\\
\texttt{mnist\_7} & \multicolumn{1}{r}{60000} & \multicolumn{1}{r}{784}  & 2532 & 1793 & \textbf{2464} & 2363 & 4542 & $\mathsmaller{\geq}1$h & 6265 & $\mathsmaller{\geq}1$h & - & - & 2848 & 6.7\\
\texttt{mnist\_8} & \multicolumn{1}{r}{60000} & \multicolumn{1}{r}{784}  & \textbf{2547} & 2847 & 2818 & 1149 & 4609 & $\mathsmaller{\geq}1$h & 5851 & $\mathsmaller{\geq}1$h & - & - & 3172 & 6.3\\
\texttt{mnist\_9} & \multicolumn{1}{r}{60000} & \multicolumn{1}{r}{784}  & \textbf{3352} & 1695 & 3521 & 1368 & 5252 & $\mathsmaller{\geq}1$h & 5949 & $\mathsmaller{\geq}1$h & - & - & 3830 & 6.8\\
\texttt{mushroom} & \multicolumn{1}{r}{8124} & \multicolumn{1}{r}{119}  & 0 & 0.00$^*$ & 0 & 0.03$^*$ & 0 & 36$^*$ & 0 & 0.10$^*$ & 4208 & 437 & 3 & 0.03\\
\texttt{pendigits} & \multicolumn{1}{r}{7494} & \multicolumn{1}{r}{216}  & 0 & 284$^*$ & 0 & 1295$^*$ & - & - & 780 & $\mathsmaller{\geq}1$h & 780 & 618 & 11 & 0.07\\
\texttt{primary-tumor} & \multicolumn{1}{r}{336} & \multicolumn{1}{r}{31}  & 26 & 0.38$^*$ & 26 & 1.5$^*$ & 26 & 24$^*$ & 26 & 103$^*$ & 34 & $\mathsmaller{\geq}1$h & 35 & 0.00\\
\texttt{segment} & \multicolumn{1}{r}{2310} & \multicolumn{1}{r}{235}  & 0 & 0.00$^*$ & 0 & 0.02$^*$ & 0 & 1.0$^*$ & 0 & 2.0$^*$ & 330 & 1053 & 1 & 0.01\\
\texttt{soybean} & \multicolumn{1}{r}{630} & \multicolumn{1}{r}{50}  & 8 & 20$^*$ & 8 & 7.6$^*$ & 8 & 63$^*$ & 8 & 752$^*$ & 14 & $\mathsmaller{\geq}1$h & 23 & 0.00\\
\texttt{splice-1} & \multicolumn{1}{r}{3190} & \multicolumn{1}{r}{287}  & 101 & 24 & \textbf{100} & 3308 & - & - & 1535 & $\mathsmaller{\geq}1$h & 1655 & 542 & 117 & 0.04\\
\texttt{surgical-deepnet} & \multicolumn{1}{r}{14635} & \multicolumn{1}{r}{6047}  & \textbf{2131} & 2168 & 2337 & 400 & - & - & 3690 & $\mathsmaller{\geq}1$h & - & - & 2245 & 8.4\\
\texttt{taiwan\_binarised} & \multicolumn{1}{r}{30000} & \multicolumn{1}{r}{205}  & \textbf{5200} & 105 & 5261 & 38 & 5412 & $\mathsmaller{\geq}1$h & 6636 & $\mathsmaller{\geq}1$h & - & - & 5280 & 0.37\\
\texttt{tic-tac-toe} & \multicolumn{1}{r}{958} & \multicolumn{1}{r}{27}  & 63 & 10$^*$ & 63 & 2.3$^*$ & 63 & 14$^*$ & 63 & 89$^*$ & 125 & $\mathsmaller{\geq}1$h & 78 & 0.00\\
\texttt{titanic} & \multicolumn{1}{r}{887} & \multicolumn{1}{r}{333}  & 95 & 1428 & 95 & 1371 & - & - & 342 & $\mathsmaller{\geq}1$h & 342 & 83 & 130 & 0.01\\
\texttt{vehicle} & \multicolumn{1}{r}{846} & \multicolumn{1}{r}{252}  & 1 & 690 & 1 & 1540 & - & - & 218 & $\mathsmaller{\geq}1$h & 218 & 1780 & 23 & 0.01\\
\texttt{vote} & \multicolumn{1}{r}{435} & \multicolumn{1}{r}{48}  & 1 & 24$^*$ & 1 & 6.1$^*$ & 1 & 45$^*$ & 1 & 522$^*$ & 8 & $\mathsmaller{\geq}1$h & 6 & 0.00\\
\texttt{weather-aus} & \multicolumn{1}{r}{142193} & \multicolumn{1}{r}{4759}  & 1735 & 419 & 1735 & 1907 & - & - & 1761 & $\mathsmaller{\geq}1$h & - & - & 1751 & 26\\
\texttt{wine1} & \multicolumn{1}{r}{178} & \multicolumn{1}{r}{1276}  & 33 & 1154 & 33 & 287 & - & - & 38 & $\mathsmaller{\geq}1$h & 58 & 1261 & 39 & 0.01\\
\texttt{wine2} & \multicolumn{1}{r}{178} & \multicolumn{1}{r}{1276}  & 39 & 411 & \textbf{37} & 3400 & - & - & 42 & $\mathsmaller{\geq}1$h & 71 & 638 & 44 & 0.01\\
\texttt{wine3} & \multicolumn{1}{r}{178} & \multicolumn{1}{r}{1276}  & 25 & 17 & 25 & 25 & - & - & 28 & $\mathsmaller{\geq}1$h & 48 & 1054 & 30 & 0.01\\
\texttt{yeast} & \multicolumn{1}{r}{1484} & \multicolumn{1}{r}{89}  & 313 & 139$^*$ & 313 & 558$^*$ & - & - & 315 & $\mathsmaller{\geq}1$h & 463 & 1438 & 367 & 0.01\\
\bottomrule
\end{tabular}

\end{footnotesize}
\end{center}
\caption{\label{tab:all5} Comparison with state of the art: depth 5}
\end{table}


\begin{table}[htbp]
\begin{center}
\begin{footnotesize}
\tabcolsep=2pt
\begin{tabular}{lccrrrrrrrrrrr}
\toprule
& && \multicolumn{3}{c}{\budalg} & \multicolumn{3}{c}{\murtree} & \multicolumn{3}{c}{\dleight} & \multicolumn{2}{c}{\cart}\\
\cmidrule(rr){4-6}\cmidrule(rr){7-9}\cmidrule(rr){10-12}\cmidrule(rr){13-14}
&\multirow{1}{*}{$\#ex.$} & \multirow{1}{*}{\#feat.} &  \multicolumn{1}{c}{error} & \multicolumn{1}{c}{cpu} & \multicolumn{1}{c}{opt.} & \multicolumn{1}{c}{error} & \multicolumn{1}{c}{cpu} & \multicolumn{1}{c}{opt.} & \multicolumn{1}{c}{error} & \multicolumn{1}{c}{cpu} & \multicolumn{1}{c}{opt.} & \multicolumn{1}{c}{error} & \multicolumn{1}{c}{cpu} \\
\midrule

\texttt{adult\_discretized} & \multicolumn{1}{r}{30299} & \multicolumn{1}{r}{59}  & 4281 & 1326.0 & 0 & 4281 & 196.2 & 0 & - & - & 0 & 4532 & \textbf{0.1}\\
\texttt{anneal} & \multicolumn{1}{r}{812} & \multicolumn{1}{r}{93}  & \textbf{51} & 1330.0 & 1 & 53 & 2492.3 & 0 & - & - & 0 & 106 & \textbf{0.0}\\
\texttt{audiology} & \multicolumn{1}{r}{216} & \multicolumn{1}{r}{148}  & 0 & \textbf{0.0} & 1 & 0 & 0.0 & 1 & 0 & 0.0 & 1 & 1 & 0.0\\
\texttt{australian-credit} & \multicolumn{1}{r}{653} & \multicolumn{1}{r}{125}  & 15 & 341.9 & 0 & 15 & 748.7 & 0 & - & - & 0 & 56 & \textbf{0.0}\\
\texttt{bank} & \multicolumn{1}{r}{45211} & \multicolumn{1}{r}{9531}  & \textbf{4046} & 338.6 & 0 & 4270 & 2970.7 & 0 & 4810 & 3604.4 & 0 & 4245 & \textbf{42.9}\\
\texttt{breast-cancer} & \multicolumn{1}{r}{683} & \multicolumn{1}{r}{89}  & 1 & 3328.4 & 0 & 1 & 2338.5 & 1 & - & - & 0 & 13 & \textbf{0.0}\\
\texttt{breast-wisconsin} & \multicolumn{1}{r}{683} & \multicolumn{1}{r}{120}  & 0 & 5.9 & 1 & 0 & 59.7 & 1 & - & - & 0 & 7 & \textbf{0.0}\\
\texttt{car} & \multicolumn{1}{r}{1728} & \multicolumn{1}{r}{21}  & 36 & 27.3 & 1 & 36 & 4.2 & 1 & 36 & 7.9 & 1 & 90 & \textbf{0.0}\\
\texttt{compas\_discretized} & \multicolumn{1}{r}{6167} & \multicolumn{1}{r}{25}  & 1887 & 17.2 & 1 & 1887 & 67.7 & 1 & 1887 & 160.8 & 1 & 1955 & \textbf{0.0}\\
\texttt{diabetes} & \multicolumn{1}{r}{768} & \multicolumn{1}{r}{112}  & \textbf{60} & 2705.8 & 0 & 62 & 694.8 & 0 & - & - & 0 & 130 & \textbf{0.0}\\
\texttt{forest-fires} & \multicolumn{1}{r}{517} & \multicolumn{1}{r}{989}  & \textbf{132} & 1934.5 & 0 & 150 & 1464.2 & 0 & - & - & 0 & 171 & \textbf{0.0}\\
\texttt{german-credit} & \multicolumn{1}{r}{1000} & \multicolumn{1}{r}{112}  & 101 & 2883.3 & 0 & 101 & 2034.4 & 0 & - & - & 0 & 171 & \textbf{0.0}\\
\texttt{heart-cleveland} & \multicolumn{1}{r}{296} & \multicolumn{1}{r}{95}  & 0 & 0.0 & 1 & 0 & 0.2 & 1 & - & - & 0 & 15 & \textbf{0.0}\\
\texttt{hepatitis} & \multicolumn{1}{r}{137} & \multicolumn{1}{r}{68}  & 0 & \textbf{0.0} & 1 & 0 & 0.0 & 1 & 0 & 29.6 & 1 & 3 & 0.0\\
\texttt{hypothyroid} & \multicolumn{1}{r}{3247} & \multicolumn{1}{r}{88}  & 32 & 2390.9 & 1 & 32 & 3488.1 & 0 & - & - & 0 & 47 & \textbf{0.0}\\
\texttt{ionosphere} & \multicolumn{1}{r}{351} & \multicolumn{1}{r}{445}  & 0 & 4.4 & 1 & 0 & 24.4 & 1 & - & - & 0 & 11 & \textbf{0.0}\\
\texttt{kr-vs-kp} & \multicolumn{1}{r}{3196} & \multicolumn{1}{r}{73}  & 45 & 1694.0 & 1 & 45 & 2385.2 & 0 & - & - & 0 & 184 & \textbf{0.0}\\
\texttt{letter} & \multicolumn{1}{r}{20000} & \multicolumn{1}{r}{224}  & \textbf{118} & 2186.1 & 0 & 275 & 171.8 & 0 & 387 & 3600.0 & 0 & 217 & \textbf{0.3}\\
\texttt{lymph} & \multicolumn{1}{r}{148} & \multicolumn{1}{r}{68}  & 0 & \textbf{0.0} & 1 & 0 & 0.0 & 1 & 0 & 0.6 & 1 & 1 & 0.0\\
\texttt{mnist\_0} & \multicolumn{1}{r}{60000} & \multicolumn{1}{r}{784}  & \textbf{1468} & 2513.2 & 0 & 1885 & 3591.1 & 0 & 3319 & 3600.3 & 0 & 1781 & \textbf{5.4}\\
\texttt{mnist\_1} & \multicolumn{1}{r}{60000} & \multicolumn{1}{r}{784}  & \textbf{1167} & 1874.6 & 0 & 1778 & 3586.7 & 0 & 4551 & 3600.3 & 0 & 1542 & \textbf{5.1}\\
\texttt{mnist\_2} & \multicolumn{1}{r}{60000} & \multicolumn{1}{r}{784}  & \textbf{2519} & 229.7 & 0 & 2687 & 1160.7 & 0 & 4232 & 3600.3 & 0 & 2818 & \textbf{5.6}\\
\texttt{mnist\_3} & \multicolumn{1}{r}{60000} & \multicolumn{1}{r}{784}  & \textbf{2486} & 2792.6 & 0 & 2923 & 1923.2 & 0 & 4900 & 3600.3 & 0 & 2902 & \textbf{7.8}\\
\texttt{mnist\_4} & \multicolumn{1}{r}{60000} & \multicolumn{1}{r}{784}  & \textbf{2180} & 3375.4 & 0 & 2973 & 2185.0 & 0 & 5580 & 3600.3 & 0 & 2543 & \textbf{4.4}\\
\texttt{mnist\_5} & \multicolumn{1}{r}{60000} & \multicolumn{1}{r}{784}  & \textbf{2930} & 1759.2 & 0 & 3060 & 1215.9 & 0 & 4376 & 3600.3 & 0 & 3402 & \textbf{7.2}\\
\texttt{mnist\_6} & \multicolumn{1}{r}{60000} & \multicolumn{1}{r}{784}  & \textbf{1278} & 2110.7 & 0 & 1474 & 2711.4 & 0 & 2750 & 3600.3 & 0 & 1686 & \textbf{5.5}\\
\texttt{mnist\_7} & \multicolumn{1}{r}{60000} & \multicolumn{1}{r}{784}  & \textbf{2074} & 2011.6 & 0 & 2304 & 545.0 & 0 & 4543 & 3600.2 & 0 & 2163 & \textbf{5.2}\\
\texttt{mnist\_8} & \multicolumn{1}{r}{60000} & \multicolumn{1}{r}{784}  & \textbf{2060} & 806.1 & 0 & 3228 & 95.9 & 0 & 4656 & 3600.3 & 0 & 2633 & \textbf{6.1}\\
\texttt{mnist\_9} & \multicolumn{1}{r}{60000} & \multicolumn{1}{r}{784}  & \textbf{2879} & 2229.3 & 0 & 3327 & 1787.6 & 0 & 5252 & 3600.3 & 0 & 3366 & \textbf{6.6}\\
\texttt{mushroom} & \multicolumn{1}{r}{8124} & \multicolumn{1}{r}{119}  & 0 & \textbf{0.0} & 1 & 0 & 0.0 & 1 & 0 & 31.5 & 1 & 3 & 0.0\\
\texttt{pendigits} & \multicolumn{1}{r}{7494} & \multicolumn{1}{r}{216}  & 0 & \textbf{0.0} & 1 & 0 & 0.4 & 1 & - & - & 0 & 5 & 0.1\\
\texttt{primary-tumor} & \multicolumn{1}{r}{336} & \multicolumn{1}{r}{31}  & 18 & 3.1 & 1 & 18 & 23.1 & 1 & 18 & 138.3 & 1 & 28 & \textbf{0.0}\\
\texttt{segment} & \multicolumn{1}{r}{2310} & \multicolumn{1}{r}{235}  & 0 & \textbf{0.0} & 1 & 0 & 0.0 & 1 & 0 & 0.4 & 1 & 0 & 0.0\\
\texttt{soybean} & \multicolumn{1}{r}{630} & \multicolumn{1}{r}{50}  & 3 & 353.7 & 1 & 3 & 122.5 & 1 & 3 & 512.9 & 1 & 15 & \textbf{0.0}\\
\texttt{splice-1} & \multicolumn{1}{r}{3190} & \multicolumn{1}{r}{287}  & \textbf{68} & 3608.7 & 0 & 80 & 1723.1 & 0 & - & - & 0 & 87 & \textbf{0.0}\\
\texttt{surgical-deepnet} & \multicolumn{1}{r}{14635} & \multicolumn{1}{r}{6047}  & \textbf{1767} & 2342.7 & 0 & 2110 & 230.9 & 0 & - & - & 0 & 1969 & \textbf{7.4}\\
\texttt{taiwan\_binarised} & \multicolumn{1}{r}{30000} & \multicolumn{1}{r}{205}  & \textbf{5073} & 1472.9 & 0 & 5169 & 3396.2 & 0 & - & - & 0 & 5250 & \textbf{0.5}\\
\texttt{tic-tac-toe} & \multicolumn{1}{r}{958} & \multicolumn{1}{r}{27}  & 12 & 126.2 & 1 & 12 & 15.7 & 1 & 12 & 46.8 & 1 & 49 & \textbf{0.0}\\
\texttt{titanic} & \multicolumn{1}{r}{887} & \multicolumn{1}{r}{333}  & \textbf{78} & 1233.9 & 0 & 108 & 1509.4 & 0 & - & - & 0 & 119 & \textbf{0.0}\\
\texttt{vehicle} & \multicolumn{1}{r}{846} & \multicolumn{1}{r}{252}  & 0 & 0.1 & 1 & 0 & 0.6 & 1 & - & - & 0 & 9 & \textbf{0.0}\\
\texttt{vote} & \multicolumn{1}{r}{435} & \multicolumn{1}{r}{48}  & 0 & \textbf{0.0} & 1 & 0 & 0.0 & 1 & 0 & 0.6 & 1 & 2 & 0.0\\
\texttt{weather-aus} & \multicolumn{1}{r}{142193} & \multicolumn{1}{r}{4759}  & \textbf{1713} & 417.9 & 0 & 1736 & 813.2 & 0 & - & - & 0 & 1734 & \textbf{21.7}\\
\texttt{wine1} & \multicolumn{1}{r}{178} & \multicolumn{1}{r}{1276}  & \textbf{31} & 2113.1 & 0 & 32 & 1498.0 & 0 & - & - & 0 & 36 & \textbf{0.0}\\
\texttt{wine2} & \multicolumn{1}{r}{178} & \multicolumn{1}{r}{1276}  & 34 & 43.7 & 0 & 34 & 504.1 & 0 & - & - & 0 & 41 & \textbf{0.0}\\
\texttt{wine3} & \multicolumn{1}{r}{178} & \multicolumn{1}{r}{1276}  & 22 & 93.5 & 0 & 22 & 925.5 & 0 & - & - & 0 & 27 & \textbf{0.0}\\
\texttt{yeast} & \multicolumn{1}{r}{1484} & \multicolumn{1}{r}{89}  & \textbf{245} & 388.2 & 0 & 258 & 1736.6 & 0 & - & - & 0 & 346 & \textbf{0.0}\\
\bottomrule
\end{tabular}

\end{footnotesize}
\end{center}
\caption{\label{tab:all6} Comparison with state of the art: depth 6}
\end{table}

\begin{table}[htbp]
\begin{center}
\begin{footnotesize}
\tabcolsep=2pt
\begin{tabular}{lccrrrrrrrrrrrrrrr}
\toprule
& && \multicolumn{4}{c}{\budalg} & \multicolumn{4}{c}{\murtree} & \multicolumn{4}{c}{\dleight} & \multicolumn{3}{c}{\cart}\\
\cmidrule(rr){4-7}\cmidrule(rr){8-11}\cmidrule(rr){12-15}\cmidrule(rr){16-18}
&\multirow{1}{*}{$\#ex.$} & \multirow{1}{*}{\#feat.} &  \multicolumn{1}{c}{error} & \multicolumn{1}{c}{acc.} & \multicolumn{1}{c}{cpu} & \multicolumn{1}{c}{opt.} & \multicolumn{1}{c}{error} & \multicolumn{1}{c}{acc.} & \multicolumn{1}{c}{cpu} & \multicolumn{1}{c}{opt.} & \multicolumn{1}{c}{error} & \multicolumn{1}{c}{acc.} & \multicolumn{1}{c}{cpu} & \multicolumn{1}{c}{opt.} & \multicolumn{1}{c}{error} & \multicolumn{1}{c}{acc.} & \multicolumn{1}{c}{cpu} \\
\midrule

\texttt{adult\_discretized} & \multicolumn{1}{r}{30299} & \multicolumn{1}{r}{59}  & 4190 & 0.8617 & 1410.2 & 0.00 & \textbf{4137} & 0.8635 & 3336.8 & 0.00 & 4998 & 0.8350 & 3600.0 & 0.00 & 4481 & 0.8521 & \textbf{0.1}\\
\texttt{anneal} & \multicolumn{1}{r}{812} & \multicolumn{1}{r}{93}  & \textbf{41} & 0.9495 & 3425.3 & 0.00 & 50 & 0.9384 & 2270.1 & 0.00 & - & - & - & 0.00 & 96 & 0.8818 & \textbf{0.0}\\
\texttt{audiology} & \multicolumn{1}{r}{216} & \multicolumn{1}{r}{148}  & 0 & 1.0000 & \textbf{0.0} & 1.00 & 0 & 1.0000 & 0.0 & 1.00 & 0 & 1.0000 & 0.0 & 1.00 & 0 & 1.0000 & 0.0\\
\texttt{australian-credit} & \multicolumn{1}{r}{653} & \multicolumn{1}{r}{125}  & 0 & 1.0000 & 97.2 & 1.00 & 0 & 1.0000 & 344.5 & 1.00 & - & - & - & 0.00 & 43 & 0.9342 & \textbf{0.0}\\
\texttt{bank} & \multicolumn{1}{r}{45211} & \multicolumn{1}{r}{9531}  & \textbf{3964} & 0.9123 & 393.0 & 0.00 & 4232 & 0.9064 & 2012.5 & 0.00 & 4807 & 0.8937 & 3604.2 & 0.00 & 4038 & 0.9107 & \textbf{76.9}\\
\texttt{breast-cancer} & \multicolumn{1}{r}{683} & \multicolumn{1}{r}{89}  & 0 & 1.0000 & 1016.5 & 1.00 & 0 & 1.0000 & 228.5 & 1.00 & 0 & 1.0000 & 449.8 & 1.00 & 8 & 0.9883 & \textbf{0.0}\\
\texttt{breast-wisconsin} & \multicolumn{1}{r}{683} & \multicolumn{1}{r}{120}  & 0 & 1.0000 & 0.0 & 1.00 & 0 & 1.0000 & 0.4 & 1.00 & - & - & - & 0.00 & 4 & 0.9941 & \textbf{0.0}\\
\texttt{car} & \multicolumn{1}{r}{1728} & \multicolumn{1}{r}{21}  & 11 & 0.9936 & 218.3 & 1.00 & 11 & 0.9936 & 24.1 & 1.00 & 11 & 0.9936 & 16.3 & 1.00 & 50 & 0.9711 & \textbf{0.0}\\
\texttt{compas\_discretized} & \multicolumn{1}{r}{6167} & \multicolumn{1}{r}{25}  & 1852 & 0.6997 & 225.4 & 1.00 & 1852 & 0.6997 & 476.5 & 1.00 & 1852 & 0.6997 & 574.7 & 1.00 & 1941 & 0.6853 & \textbf{0.0}\\
\texttt{diabetes} & \multicolumn{1}{r}{768} & \multicolumn{1}{r}{112}  & \textbf{21} & 0.9727 & 993.5 & 0.00 & 83 & 0.8919 & 3286.5 & 0.00 & - & - & - & 0.00 & 100 & 0.8698 & \textbf{0.0}\\
\texttt{forest-fires} & \multicolumn{1}{r}{517} & \multicolumn{1}{r}{989}  & 159 & 0.6925 & 2.5 & 0.00 & \textbf{145} & 0.7195 & 1268.0 & 0.00 & - & - & - & 0.00 & 161 & 0.6886 & \textbf{0.0}\\
\texttt{german-credit} & \multicolumn{1}{r}{1000} & \multicolumn{1}{r}{112}  & \textbf{56} & 0.9440 & 1385.6 & 0.00 & 87 & 0.9130 & 492.3 & 0.00 & - & - & - & 0.00 & 150 & 0.8500 & \textbf{0.0}\\
\texttt{heart-cleveland} & \multicolumn{1}{r}{296} & \multicolumn{1}{r}{95}  & 0 & 1.0000 & \textbf{0.0} & 1.00 & 0 & 1.0000 & 0.0 & 1.00 & - & - & - & 0.00 & 6 & 0.9797 & 0.0\\
\texttt{hepatitis} & \multicolumn{1}{r}{137} & \multicolumn{1}{r}{68}  & 0 & 1.0000 & \textbf{0.0} & 1.00 & 0 & 1.0000 & 0.0 & 1.00 & 0 & 1.0000 & 8.9 & 1.00 & 1 & 0.9927 & 0.0\\
\texttt{hypothyroid} & \multicolumn{1}{r}{3247} & \multicolumn{1}{r}{88}  & 23 & 0.9929 & 118.5 & 0.00 & 23 & 0.9929 & 1230.5 & 0.00 & - & - & - & 0.00 & 42 & 0.9871 & \textbf{0.0}\\
\texttt{ionosphere} & \multicolumn{1}{r}{351} & \multicolumn{1}{r}{445}  & 0 & 1.0000 & 0.1 & 1.00 & 0 & 1.0000 & 0.4 & 1.00 & - & - & - & 0.00 & 7 & 0.9801 & \textbf{0.0}\\
\texttt{kr-vs-kp} & \multicolumn{1}{r}{3196} & \multicolumn{1}{r}{73}  & \textbf{18} & 0.9944 & 2327.2 & 0.00 & 34 & 0.9894 & 934.2 & 0.00 & - & - & - & 0.00 & 103 & 0.9678 & \textbf{0.0}\\
\texttt{letter} & \multicolumn{1}{r}{20000} & \multicolumn{1}{r}{224}  & \textbf{70} & 0.9965 & 432.1 & 0.00 & 131 & 0.9935 & 137.0 & 0.00 & 488 & 0.9756 & 3600.0 & 0.00 & 153 & 0.9923 & \textbf{0.3}\\
\texttt{lymph} & \multicolumn{1}{r}{148} & \multicolumn{1}{r}{68}  & 0 & 1.0000 & \textbf{0.0} & 1.00 & 0 & 1.0000 & 0.0 & 1.00 & 0 & 1.0000 & 0.0 & 1.00 & 0 & 1.0000 & 0.0\\
\texttt{mnist\_0} & \multicolumn{1}{r}{60000} & \multicolumn{1}{r}{784}  & \textbf{1223} & 0.9796 & 487.0 & 0.00 & 1601 & 0.9733 & 878.0 & 0.00 & - & - & - & 0.00 & 1323 & 0.9779 & \textbf{8.5}\\
\texttt{mnist\_1} & \multicolumn{1}{r}{60000} & \multicolumn{1}{r}{784}  & \textbf{1066} & 0.9822 & 748.4 & 0.00 & 1488 & 0.9752 & 47.5 & 0.00 & - & - & - & 0.00 & 1129 & 0.9812 & \textbf{6.0}\\
\texttt{mnist\_2} & \multicolumn{1}{r}{60000} & \multicolumn{1}{r}{784}  & \textbf{2394} & 0.9601 & 2080.6 & 0.00 & 3121 & 0.9480 & 1438.6 & 0.00 & - & - & - & 0.00 & 2502 & 0.9583 & \textbf{5.2}\\
\texttt{mnist\_3} & \multicolumn{1}{r}{60000} & \multicolumn{1}{r}{784}  & \textbf{2114} & 0.9648 & 1064.3 & 0.00 & 2492 & 0.9585 & 1500.9 & 0.00 & 5172 & 0.9138 & 3600.3 & 0.00 & 2274 & 0.9621 & \textbf{4.9}\\
\texttt{mnist\_4} & \multicolumn{1}{r}{60000} & \multicolumn{1}{r}{784}  & \textbf{1968} & 0.9672 & 2504.6 & 0.00 & 2807 & 0.9532 & 1803.0 & 0.00 & - & - & - & 0.00 & 2072 & 0.9655 & \textbf{7.1}\\
\texttt{mnist\_5} & \multicolumn{1}{r}{60000} & \multicolumn{1}{r}{784}  & 3007 & 0.9499 & 1099.6 & 0.00 & \textbf{2922} & 0.9513 & 2494.6 & 0.00 & - & - & - & 0.00 & 3117 & 0.9480 & \textbf{6.0}\\
\texttt{mnist\_6} & \multicolumn{1}{r}{60000} & \multicolumn{1}{r}{784}  & 1408 & 0.9765 & 3575.0 & 0.00 & \textbf{1235} & 0.9794 & 1158.2 & 0.00 & - & - & - & 0.00 & 1483 & 0.9753 & \textbf{7.8}\\
\texttt{mnist\_7} & \multicolumn{1}{r}{60000} & \multicolumn{1}{r}{784}  & \textbf{1724} & 0.9713 & 1013.3 & 0.00 & 1957 & 0.9674 & 590.2 & 0.00 & - & - & - & 0.00 & 1864 & 0.9689 & \textbf{5.2}\\
\texttt{mnist\_8} & \multicolumn{1}{r}{60000} & \multicolumn{1}{r}{784}  & \textbf{1937} & 0.9677 & 1439.2 & 0.00 & 2819 & 0.9530 & 1885.5 & 0.00 & - & - & - & 0.00 & 2101 & 0.9650 & \textbf{5.8}\\
\texttt{mnist\_9} & \multicolumn{1}{r}{60000} & \multicolumn{1}{r}{784}  & \textbf{2731} & 0.9545 & 1793.9 & 0.00 & 2946 & 0.9509 & 1195.3 & 0.00 & - & - & - & 0.00 & 2811 & 0.9532 & \textbf{5.4}\\
\texttt{mushroom} & \multicolumn{1}{r}{8124} & \multicolumn{1}{r}{119}  & 0 & 1.0000 & \textbf{0.0} & 1.00 & 0 & 1.0000 & 0.0 & 1.00 & 0 & 1.0000 & 10.1 & 1.00 & 0 & 1.0000 & 0.0\\
\texttt{pendigits} & \multicolumn{1}{r}{7494} & \multicolumn{1}{r}{216}  & 0 & 1.0000 & \textbf{0.0} & 1.00 & 0 & 1.0000 & 0.1 & 1.00 & - & - & - & 0.00 & 1 & 0.9999 & 0.1\\
\texttt{primary-tumor} & \multicolumn{1}{r}{336} & \multicolumn{1}{r}{31}  & 16 & 0.9524 & 16.6 & 1.00 & 16 & 0.9524 & 250.8 & 1.00 & 16 & 0.9524 & 457.9 & 1.00 & 26 & 0.9226 & \textbf{0.0}\\
\texttt{segment} & \multicolumn{1}{r}{2310} & \multicolumn{1}{r}{235}  & 0 & 1.0000 & \textbf{0.0} & 1.00 & 0 & 1.0000 & 0.0 & 1.00 & 0 & 1.0000 & 0.2 & 1.00 & 0 & 1.0000 & 0.0\\
\texttt{soybean} & \multicolumn{1}{r}{630} & \multicolumn{1}{r}{50}  & 2 & 0.9968 & 19.5 & 1.00 & 2 & 0.9968 & 1710.3 & 1.00 & - & - & - & 0.00 & 11 & 0.9825 & \textbf{0.0}\\
\texttt{splice-1} & \multicolumn{1}{r}{3190} & \multicolumn{1}{r}{287}  & \textbf{29} & 0.9909 & 3405.6 & 0.00 & 47 & 0.9853 & 836.9 & 0.00 & - & - & - & 0.00 & 58 & 0.9818 & \textbf{0.0}\\
\texttt{surgical-deepnet} & \multicolumn{1}{r}{14635} & \multicolumn{1}{r}{6047}  & \textbf{1814} & 0.8760 & 802.9 & 0.00 & 1890 & 0.8709 & 740.0 & 0.00 & - & - & - & 0.00 & 1871 & 0.8722 & \textbf{9.9}\\
\texttt{taiwan\_binarised} & \multicolumn{1}{r}{30000} & \multicolumn{1}{r}{205}  & \textbf{5061} & 0.8313 & 65.2 & 0.00 & 5189 & 0.8270 & 2762.8 & 0.00 & 5412 & 0.8196 & 3600.0 & 0.00 & 5161 & 0.8280 & \textbf{0.6}\\
\texttt{tic-tac-toe} & \multicolumn{1}{r}{958} & \multicolumn{1}{r}{27}  & 0 & 1.0000 & 25.1 & 1.00 & 0 & 1.0000 & 9.2 & 1.00 & 0 & 1.0000 & 28.8 & 1.00 & 22 & 0.9770 & \textbf{0.0}\\
\texttt{titanic} & \multicolumn{1}{r}{887} & \multicolumn{1}{r}{333}  & \textbf{62} & 0.9301 & 3431.4 & 0.00 & 97 & 0.8906 & 245.3 & 0.00 & - & - & - & 0.00 & 111 & 0.8749 & \textbf{0.0}\\
\texttt{vehicle} & \multicolumn{1}{r}{846} & \multicolumn{1}{r}{252}  & 0 & 1.0000 & 0.1 & 1.00 & 0 & 1.0000 & 0.6 & 1.00 & - & - & - & 0.00 & 4 & 0.9953 & \textbf{0.0}\\
\texttt{vote} & \multicolumn{1}{r}{435} & \multicolumn{1}{r}{48}  & 0 & 1.0000 & \textbf{0.0} & 1.00 & 0 & 1.0000 & 0.0 & 1.00 & 0 & 1.0000 & 0.2 & 1.00 & 2 & 0.9954 & 0.0\\
\texttt{weather-aus} & \multicolumn{1}{r}{142193} & \multicolumn{1}{r}{4759}  & \textbf{1687} & 0.9881 & 2673.9 & 0.00 & 1724 & 0.9879 & 3483.5 & 0.00 & - & - & - & 0.00 & 1721 & 0.9879 & \textbf{26.7}\\
\texttt{wine1} & \multicolumn{1}{r}{178} & \multicolumn{1}{r}{1276}  & \textbf{28} & 0.8427 & 880.0 & 0.00 & 29 & 0.8371 & 936.0 & 0.00 & - & - & - & 0.00 & 33 & 0.8146 & \textbf{0.0}\\
\texttt{wine2} & \multicolumn{1}{r}{178} & \multicolumn{1}{r}{1276}  & 31 & 0.8258 & 27.3 & 0.00 & 31 & 0.8258 & 411.8 & 0.00 & - & - & - & 0.00 & 38 & 0.7865 & \textbf{0.0}\\
\texttt{wine3} & \multicolumn{1}{r}{178} & \multicolumn{1}{r}{1276}  & \textbf{21} & 0.8820 & 518.2 & 0.00 & 22 & 0.8764 & 303.8 & 0.00 & - & - & - & 0.00 & 24 & 0.8652 & \textbf{0.0}\\
\texttt{yeast} & \multicolumn{1}{r}{1484} & \multicolumn{1}{r}{89}  & \textbf{203} & 0.8632 & 299.7 & 0.00 & 222 & 0.8504 & 2360.3 & 0.00 & - & - & - & 0.00 & 306 & 0.7938 & \textbf{0.0}\\
\bottomrule
\end{tabular}

\end{footnotesize}
\end{center}
\caption{\label{tab:all7} Comparison with state of the art: depth 7}
\end{table}

\begin{table}[htbp]
\begin{center}
\begin{footnotesize}
\tabcolsep=2pt
\begin{tabular}{lccrrrrrrrrrrrrrrr}
\toprule
& && \multicolumn{4}{c}{\budalg} & \multicolumn{4}{c}{\murtree} & \multicolumn{4}{c}{\dleight} & \multicolumn{3}{c}{\cart}\\
\cmidrule(rr){4-7}\cmidrule(rr){8-11}\cmidrule(rr){12-15}\cmidrule(rr){16-18}
&\multirow{1}{*}{$\#ex.$} & \multirow{1}{*}{\#feat.} &  \multicolumn{1}{c}{error} & \multicolumn{1}{c}{acc.} & \multicolumn{1}{c}{cpu} & \multicolumn{1}{c}{opt.} & \multicolumn{1}{c}{error} & \multicolumn{1}{c}{acc.} & \multicolumn{1}{c}{cpu} & \multicolumn{1}{c}{opt.} & \multicolumn{1}{c}{error} & \multicolumn{1}{c}{acc.} & \multicolumn{1}{c}{cpu} & \multicolumn{1}{c}{opt.} & \multicolumn{1}{c}{error} & \multicolumn{1}{c}{acc.} & \multicolumn{1}{c}{cpu} \\
\midrule

\texttt{adult\_discretized} & \multicolumn{1}{r}{30299} & \multicolumn{1}{r}{59}  & \textbf{4146} & 0.8632 & 1991.5 & 0.00 & 4190 & 0.8617 & 2840.6 & 0.00 & 4957 & 0.8364 & 3600.0 & 0.00 & 4399 & 0.8548 & \textbf{0.1}\\
\texttt{anneal} & \multicolumn{1}{r}{812} & \multicolumn{1}{r}{93}  & \textbf{36} & 0.9557 & 2397.2 & 0.00 & 40 & 0.9507 & 2467.5 & 0.00 & - & - & - & 0.00 & 88 & 0.8916 & \textbf{0.0}\\
\texttt{audiology} & \multicolumn{1}{r}{216} & \multicolumn{1}{r}{148}  & 0 & 1.0000 & \textbf{0.0} & 1.00 & 0 & 1.0000 & 0.0 & 1.00 & 0 & 1.0000 & 0.0 & 1.00 & 0 & 1.0000 & 0.0\\
\texttt{australian-credit} & \multicolumn{1}{r}{653} & \multicolumn{1}{r}{125}  & 0 & 1.0000 & 13.2 & 1.00 & 0 & 1.0000 & 84.8 & 1.00 & - & - & - & 0.00 & 33 & 0.9495 & \textbf{0.0}\\
\texttt{bank} & \multicolumn{1}{r}{45211} & \multicolumn{1}{r}{9531}  & \textbf{3747} & 0.9171 & 1225.2 & 0.00 & 4127 & 0.9087 & 1040.9 & 0.00 & 4810 & 0.8936 & 3604.9 & 0.00 & 3814 & 0.9156 & \textbf{72.9}\\
\texttt{breast-cancer} & \multicolumn{1}{r}{683} & \multicolumn{1}{r}{89}  & 0 & 1.0000 & 24.2 & 1.00 & 0 & 1.0000 & 13.1 & 1.00 & 0 & 1.0000 & 13.5 & 1.00 & 4 & 0.9941 & \textbf{0.0}\\
\texttt{breast-wisconsin} & \multicolumn{1}{r}{683} & \multicolumn{1}{r}{120}  & 0 & 1.0000 & \textbf{0.0} & 1.00 & 0 & 1.0000 & 0.0 & 1.00 & 0 & 1.0000 & 402.0 & 1.00 & 0 & 1.0000 & 0.0\\
\texttt{car} & \multicolumn{1}{r}{1728} & \multicolumn{1}{r}{21}  & 0 & 1.0000 & 365.9 & 1.00 & 0 & 1.0000 & 58.5 & 1.00 & 0 & 1.0000 & 13.3 & 1.00 & 36 & 0.9792 & \textbf{0.0}\\
\texttt{compas\_discretized} & \multicolumn{1}{r}{6167} & \multicolumn{1}{r}{25}  & 1832 & 0.7029 & 1449.5 & 1.00 & 1832 & 0.7029 & 2878.1 & 1.00 & 1832 & 0.7029 & 1637.8 & 1.00 & 1904 & 0.6913 & \textbf{0.0}\\
\texttt{diabetes} & \multicolumn{1}{r}{768} & \multicolumn{1}{r}{112}  & 0 & 1.0000 & 226.9 & 1.00 & 0 & 1.0000 & 2895.4 & 1.00 & - & - & - & 0.00 & 79 & 0.8971 & \textbf{0.0}\\
\texttt{forest-fires} & \multicolumn{1}{r}{517} & \multicolumn{1}{r}{989}  & 151 & 0.7079 & 182.4 & 0.00 & \textbf{145} & 0.7195 & 140.1 & 0.00 & - & - & - & 0.00 & 157 & 0.6963 & \textbf{0.0}\\
\texttt{german-credit} & \multicolumn{1}{r}{1000} & \multicolumn{1}{r}{112}  & \textbf{23} & 0.9770 & 2727.9 & 0.00 & 40 & 0.9600 & 2979.3 & 0.00 & - & - & - & 0.00 & 117 & 0.8830 & \textbf{0.0}\\
\texttt{heart-cleveland} & \multicolumn{1}{r}{296} & \multicolumn{1}{r}{95}  & 0 & 1.0000 & \textbf{0.0} & 1.00 & 0 & 1.0000 & 0.0 & 1.00 & - & - & - & 0.00 & 2 & 0.9932 & 0.0\\
\texttt{hepatitis} & \multicolumn{1}{r}{137} & \multicolumn{1}{r}{68}  & 0 & 1.0000 & \textbf{0.0} & 1.00 & 0 & 1.0000 & 0.0 & 1.00 & 0 & 1.0000 & 0.6 & 1.00 & 0 & 1.0000 & 0.0\\
\texttt{hypothyroid} & \multicolumn{1}{r}{3247} & \multicolumn{1}{r}{88}  & \textbf{17} & 0.9948 & 77.0 & \textbf{1.00} & 18 & 0.9945 & 1133.0 & 0.00 & - & - & - & 0.00 & 38 & 0.9883 & \textbf{0.0}\\
\texttt{ionosphere} & \multicolumn{1}{r}{351} & \multicolumn{1}{r}{445}  & 0 & 1.0000 & \textbf{0.0} & 1.00 & 0 & 1.0000 & 0.2 & 1.00 & - & - & - & 0.00 & 3 & 0.9915 & 0.0\\
\texttt{kr-vs-kp} & \multicolumn{1}{r}{3196} & \multicolumn{1}{r}{73}  & \textbf{13} & 0.9959 & 2662.9 & 0.00 & 16 & 0.9950 & 1617.7 & 0.00 & - & - & - & 0.00 & 48 & 0.9850 & \textbf{0.0}\\
\texttt{letter} & \multicolumn{1}{r}{20000} & \multicolumn{1}{r}{224}  & \textbf{66} & 0.9967 & 752.6 & 0.00 & 111 & 0.9944 & 1907.9 & 0.00 & 601 & 0.9699 & 3600.0 & 0.00 & 94 & 0.9953 & \textbf{0.4}\\
\texttt{lymph} & \multicolumn{1}{r}{148} & \multicolumn{1}{r}{68}  & 0 & 1.0000 & \textbf{0.0} & 1.00 & 0 & 1.0000 & 0.0 & 1.00 & 0 & 1.0000 & 0.0 & 1.00 & 0 & 1.0000 & 0.0\\
\texttt{mnist\_0} & \multicolumn{1}{r}{60000} & \multicolumn{1}{r}{784}  & \textbf{913} & 0.9848 & 1816.4 & 0.00 & 1242 & 0.9793 & 572.0 & 0.00 & - & - & - & 0.00 & 991 & 0.9835 & \textbf{7.0}\\
\texttt{mnist\_1} & \multicolumn{1}{r}{60000} & \multicolumn{1}{r}{784}  & \textbf{756} & 0.9874 & 1757.4 & 0.00 & 1190 & 0.9802 & 1964.1 & 0.00 & 4548 & 0.9242 & 3600.3 & 0.00 & 781 & 0.9870 & \textbf{6.5}\\
\texttt{mnist\_2} & \multicolumn{1}{r}{60000} & \multicolumn{1}{r}{784}  & \textbf{2164} & 0.9639 & 2660.3 & 0.00 & 2496 & 0.9584 & 2398.4 & 0.00 & - & - & - & 0.00 & 2234 & 0.9628 & \textbf{6.8}\\
\texttt{mnist\_3} & \multicolumn{1}{r}{60000} & \multicolumn{1}{r}{784}  & \textbf{1595} & 0.9734 & 1717.5 & 0.00 & 2341 & 0.9610 & 48.2 & 0.00 & - & - & - & 0.00 & 1692 & 0.9718 & \textbf{5.5}\\
\texttt{mnist\_4} & \multicolumn{1}{r}{60000} & \multicolumn{1}{r}{784}  & \textbf{1581} & 0.9737 & 251.1 & 0.00 & 2228 & 0.9629 & 2908.7 & 0.00 & - & - & - & 0.00 & 1662 & 0.9723 & \textbf{6.2}\\
\texttt{mnist\_5} & \multicolumn{1}{r}{60000} & \multicolumn{1}{r}{784}  & 2685 & 0.9553 & 187.7 & 0.00 & \textbf{2610} & 0.9565 & 1963.4 & 0.00 & - & - & - & 0.00 & 2726 & 0.9546 & \textbf{7.2}\\
\texttt{mnist\_6} & \multicolumn{1}{r}{60000} & \multicolumn{1}{r}{784}  & 1280 & 0.9787 & 415.7 & 0.00 & \textbf{1141} & 0.9810 & 1113.9 & 0.00 & - & - & - & 0.00 & 1356 & 0.9774 & \textbf{7.2}\\
\texttt{mnist\_7} & \multicolumn{1}{r}{60000} & \multicolumn{1}{r}{784}  & \textbf{1440} & 0.9760 & 2569.4 & 0.00 & 1753 & 0.9708 & 1700.8 & 0.00 & 4544 & 0.9243 & 3600.3 & 0.00 & 1538 & 0.9744 & \textbf{6.7}\\
\texttt{mnist\_8} & \multicolumn{1}{r}{60000} & \multicolumn{1}{r}{784}  & \textbf{1533} & 0.9745 & 1450.1 & 0.00 & 2535 & 0.9577 & 2521.7 & 0.00 & - & - & - & 0.00 & 1705 & 0.9716 & \textbf{5.3}\\
\texttt{mnist\_9} & \multicolumn{1}{r}{60000} & \multicolumn{1}{r}{784}  & \textbf{2259} & 0.9624 & 3475.2 & 0.00 & 2639 & 0.9560 & 576.5 & 0.00 & 5254 & 0.9124 & 3600.3 & 0.00 & 2379 & 0.9604 & \textbf{5.9}\\
\texttt{mushroom} & \multicolumn{1}{r}{8124} & \multicolumn{1}{r}{119}  & 0 & 1.0000 & \textbf{0.0} & 1.00 & 0 & 1.0000 & 0.0 & 1.00 & 0 & 1.0000 & 5.4 & 1.00 & 0 & 1.0000 & 0.0\\
\texttt{pendigits} & \multicolumn{1}{r}{7494} & \multicolumn{1}{r}{216}  & 0 & 1.0000 & \textbf{0.0} & 1.00 & 0 & 1.0000 & 0.1 & 1.00 & - & - & - & 0.00 & 0 & 1.0000 & 0.1\\
\texttt{primary-tumor} & \multicolumn{1}{r}{336} & \multicolumn{1}{r}{31}  & 15 & 0.9554 & 0.0 & 1.00 & 15 & 0.9554 & 1814.7 & 1.00 & - & - & - & 0.00 & 22 & 0.9345 & \textbf{0.0}\\
\texttt{segment} & \multicolumn{1}{r}{2310} & \multicolumn{1}{r}{235}  & 0 & 1.0000 & \textbf{0.0} & 1.00 & 0 & 1.0000 & 0.0 & 1.00 & 0 & 1.0000 & 0.3 & 1.00 & 0 & 1.0000 & 0.0\\
\texttt{soybean} & \multicolumn{1}{r}{630} & \multicolumn{1}{r}{50}  & 2 & 0.9968 & 0.2 & \textbf{1.00} & 2 & 0.9968 & 88.8 & 0.00 & - & - & - & 0.00 & 8 & 0.9873 & \textbf{0.0}\\
\texttt{splice-1} & \multicolumn{1}{r}{3190} & \multicolumn{1}{r}{287}  & \textbf{24} & 0.9925 & 0.4 & 0.00 & 31 & 0.9903 & 1670.3 & 0.00 & - & - & - & 0.00 & 34 & 0.9893 & \textbf{0.0}\\
\texttt{surgical-deepnet} & \multicolumn{1}{r}{14635} & \multicolumn{1}{r}{6047}  & \textbf{1368} & 0.9065 & 2876.2 & 0.00 & 1609 & 0.8901 & 147.5 & 0.00 & - & - & - & 0.00 & 1400 & 0.9043 & \textbf{8.5}\\
\texttt{taiwan\_binarised} & \multicolumn{1}{r}{30000} & \multicolumn{1}{r}{205}  & \textbf{4899} & 0.8367 & 1230.7 & 0.00 & 5172 & 0.8276 & 2489.3 & 0.00 & 5412 & 0.8196 & 3600.0 & 0.00 & 5043 & 0.8319 & \textbf{0.7}\\
\texttt{tic-tac-toe} & \multicolumn{1}{r}{958} & \multicolumn{1}{r}{27}  & 0 & 1.0000 & \textbf{0.0} & 1.00 & 0 & 1.0000 & 0.0 & 1.00 & 0 & 1.0000 & 1.5 & 1.00 & 13 & 0.9864 & 0.0\\
\texttt{titanic} & \multicolumn{1}{r}{887} & \multicolumn{1}{r}{333}  & \textbf{83} & 0.9064 & 1654.0 & 0.00 & 88 & 0.9008 & 2490.9 & 0.00 & - & - & - & 0.00 & 105 & 0.8816 & \textbf{0.0}\\
\texttt{vehicle} & \multicolumn{1}{r}{846} & \multicolumn{1}{r}{252}  & 0 & 1.0000 & \textbf{0.0} & 1.00 & 0 & 1.0000 & 0.1 & 1.00 & 0 & 1.0000 & 23.9 & 1.00 & 3 & 0.9965 & 0.0\\
\texttt{vote} & \multicolumn{1}{r}{435} & \multicolumn{1}{r}{48}  & 0 & 1.0000 & \textbf{0.0} & 1.00 & 0 & 1.0000 & 0.0 & 1.00 & 0 & 1.0000 & 0.0 & 1.00 & 1 & 0.9977 & 0.0\\
\texttt{weather-aus} & \multicolumn{1}{r}{142193} & \multicolumn{1}{r}{4759}  & \textbf{1664} & 0.9883 & 3251.2 & 0.00 & 1709 & 0.9880 & 1451.9 & 0.00 & - & - & - & 0.00 & 1703 & 0.9880 & \textbf{21.3}\\
\texttt{wine1} & \multicolumn{1}{r}{178} & \multicolumn{1}{r}{1276}  & 27 & 0.8483 & 43.3 & 0.00 & 27 & 0.8483 & 406.2 & 0.00 & - & - & - & 0.00 & 30 & 0.8315 & \textbf{0.0}\\
\texttt{wine2} & \multicolumn{1}{r}{178} & \multicolumn{1}{r}{1276}  & \textbf{30} & 0.8315 & 612.8 & 0.00 & 31 & 0.8258 & 1003.5 & 0.00 & - & - & - & 0.00 & 35 & 0.8034 & \textbf{0.0}\\
\texttt{wine3} & \multicolumn{1}{r}{178} & \multicolumn{1}{r}{1276}  & \textbf{20} & 0.8876 & 420.2 & 0.00 & 21 & 0.8820 & 26.0 & 0.00 & - & - & - & 0.00 & 24 & 0.8652 & \textbf{0.0}\\
\texttt{yeast} & \multicolumn{1}{r}{1484} & \multicolumn{1}{r}{89}  & \textbf{132} & 0.9111 & 2259.7 & 0.00 & 222 & 0.8504 & 2259.5 & 0.00 & - & - & - & 0.00 & 261 & 0.8241 & \textbf{0.0}\\
\bottomrule
\end{tabular}

\end{footnotesize}
\end{center}
\caption{\label{tab:all8} Comparison with state of the art: depth 8}
\end{table}

\begin{table}[htbp]
\begin{center}
\begin{footnotesize}
\tabcolsep=2pt
\begin{tabular}{lccrrrrrrrrrrrrrrr}
\toprule
& && \multicolumn{4}{c}{\budalg} & \multicolumn{4}{c}{\murtree} & \multicolumn{4}{c}{\dleight} & \multicolumn{3}{c}{\cart}\\
\cmidrule(rr){4-7}\cmidrule(rr){8-11}\cmidrule(rr){12-15}\cmidrule(rr){16-18}
&\multirow{1}{*}{$\#ex.$} & \multirow{1}{*}{\#feat.} &  \multicolumn{1}{c}{error} & \multicolumn{1}{c}{acc.} & \multicolumn{1}{c}{cpu} & \multicolumn{1}{c}{opt.} & \multicolumn{1}{c}{error} & \multicolumn{1}{c}{acc.} & \multicolumn{1}{c}{cpu} & \multicolumn{1}{c}{opt.} & \multicolumn{1}{c}{error} & \multicolumn{1}{c}{acc.} & \multicolumn{1}{c}{cpu} & \multicolumn{1}{c}{opt.} & \multicolumn{1}{c}{error} & \multicolumn{1}{c}{acc.} & \multicolumn{1}{c}{cpu} \\
\midrule

\texttt{adult\_discretized} & \multicolumn{1}{r}{30299} & \multicolumn{1}{r}{59}  & \textbf{4034} & 0.8669 & 1107.0 & 0.00 & 4094 & 0.8649 & 2210.8 & 0.00 & 6200 & 0.7954 & 3600.0 & 0.00 & 4252 & 0.8597 & \textbf{0.1}\\
\texttt{anneal} & \multicolumn{1}{r}{812} & \multicolumn{1}{r}{93}  & \textbf{35} & 0.9569 & 363.9 & 0.00 & 46 & 0.9433 & 84.7 & 0.00 & - & - & - & 0.00 & 74 & 0.9089 & \textbf{0.0}\\
\texttt{audiology} & \multicolumn{1}{r}{216} & \multicolumn{1}{r}{148}  & 0 & 1.0000 & \textbf{0.0} & 1.00 & 0 & 1.0000 & 0.0 & 1.00 & 0 & 1.0000 & 0.0 & 1.00 & 0 & 1.0000 & 0.0\\
\texttt{australian-credit} & \multicolumn{1}{r}{653} & \multicolumn{1}{r}{125}  & 0 & 1.0000 & 2.2 & 1.00 & 0 & 1.0000 & 9.2 & 1.00 & - & - & - & 0.00 & 19 & 0.9709 & \textbf{0.0}\\
\texttt{bank} & \multicolumn{1}{r}{45211} & \multicolumn{1}{r}{9531}  & \textbf{3482} & 0.9230 & 1343.4 & 0.00 & 3955 & 0.9125 & 740.5 & 0.00 & 4817 & 0.8935 & 3604.6 & 0.00 & 3575 & 0.9209 & \textbf{75.7}\\
\texttt{breast-cancer} & \multicolumn{1}{r}{683} & \multicolumn{1}{r}{89}  & 0 & 1.0000 & 9.4 & 1.00 & 0 & 1.0000 & 6.2 & 1.00 & 0 & 1.0000 & \textbf{0.0} & 1.00 & 1 & 0.9985 & 0.0\\
\texttt{breast-wisconsin} & \multicolumn{1}{r}{683} & \multicolumn{1}{r}{120}  & 0 & 1.0000 & \textbf{0.0} & 1.00 & 0 & 1.0000 & 0.0 & 1.00 & 0 & 1.0000 & 42.3 & 1.00 & 0 & 1.0000 & 0.0\\
\texttt{car} & \multicolumn{1}{r}{1728} & \multicolumn{1}{r}{21}  & 0 & 1.0000 & 63.5 & 1.00 & 0 & 1.0000 & 26.7 & 1.00 & 0 & 1.0000 & 1.3 & 1.00 & 15 & 0.9913 & \textbf{0.0}\\
\texttt{compas\_discretized} & \multicolumn{1}{r}{6167} & \multicolumn{1}{r}{25}  & 1828 & 0.7036 & 166.3 & \textbf{1.00} & 1828 & 0.7036 & 3531.6 & 0.00 & - & - & - & 0.00 & 1891 & 0.6934 & \textbf{0.0}\\
\texttt{diabetes} & \multicolumn{1}{r}{768} & \multicolumn{1}{r}{112}  & 0 & 1.0000 & 3.4 & 1.00 & 0 & 1.0000 & 42.8 & 1.00 & - & - & - & 0.00 & 55 & 0.9284 & \textbf{0.0}\\
\texttt{forest-fires} & \multicolumn{1}{r}{517} & \multicolumn{1}{r}{989}  & 150 & 0.7099 & \textbf{0.0} & 0.00 & \textbf{140} & 0.7292 & 56.6 & 0.00 & - & - & - & 0.00 & 152 & 0.7060 & 0.0\\
\texttt{german-credit} & \multicolumn{1}{r}{1000} & \multicolumn{1}{r}{112}  & \textbf{15} & 0.9850 & 674.5 & 0.00 & 19 & 0.9810 & 2469.3 & 0.00 & - & - & - & 0.00 & 97 & 0.9030 & \textbf{0.0}\\
\texttt{heart-cleveland} & \multicolumn{1}{r}{296} & \multicolumn{1}{r}{95}  & 0 & 1.0000 & \textbf{0.0} & 1.00 & 0 & 1.0000 & 0.0 & 1.00 & 0 & 1.0000 & 130.3 & 1.00 & 0 & 1.0000 & 0.0\\
\texttt{hepatitis} & \multicolumn{1}{r}{137} & \multicolumn{1}{r}{68}  & 0 & 1.0000 & \textbf{0.0} & 1.00 & 0 & 1.0000 & 0.0 & 1.00 & 0 & 1.0000 & 0.0 & 1.00 & 0 & 1.0000 & 0.0\\
\texttt{hypothyroid} & \multicolumn{1}{r}{3247} & \multicolumn{1}{r}{88}  & \textbf{17} & 0.9948 & 59.3 & \textbf{1.00} & 18 & 0.9945 & 2316.7 & 0.00 & - & - & - & 0.00 & 36 & 0.9889 & \textbf{0.0}\\
\texttt{ionosphere} & \multicolumn{1}{r}{351} & \multicolumn{1}{r}{445}  & 0 & 1.0000 & \textbf{0.0} & 1.00 & 0 & 1.0000 & 0.2 & 1.00 & 0 & 1.0000 & 317.0 & 1.00 & 0 & 1.0000 & 0.0\\
\texttt{kr-vs-kp} & \multicolumn{1}{r}{3196} & \multicolumn{1}{r}{73}  & \textbf{5} & 0.9984 & 3580.9 & 0.00 & 32 & 0.9900 & 1069.3 & 0.00 & - & - & - & 0.00 & 23 & 0.9928 & \textbf{0.0}\\
\texttt{letter} & \multicolumn{1}{r}{20000} & \multicolumn{1}{r}{224}  & 24 & 0.9988 & 84.4 & 0.00 & \textbf{4} & 0.9998 & 3368.6 & 0.00 & 697 & 0.9651 & 3600.0 & 0.00 & 48 & 0.9976 & \textbf{0.4}\\
\texttt{lymph} & \multicolumn{1}{r}{148} & \multicolumn{1}{r}{68}  & 0 & 1.0000 & \textbf{0.0} & 1.00 & 0 & 1.0000 & 0.0 & 1.00 & 0 & 1.0000 & 0.0 & 1.00 & 0 & 1.0000 & 0.0\\
\texttt{mnist\_0} & \multicolumn{1}{r}{60000} & \multicolumn{1}{r}{784}  & \textbf{575} & 0.9904 & 1691.2 & 0.00 & 1084 & 0.9819 & 43.3 & 0.00 & - & - & - & 0.00 & 710 & 0.9882 & \textbf{8.6}\\
\texttt{mnist\_1} & \multicolumn{1}{r}{60000} & \multicolumn{1}{r}{784}  & \textbf{554} & 0.9908 & 448.8 & 0.00 & 846 & 0.9859 & 1530.5 & 0.00 & 4547 & 0.9242 & 3600.3 & 0.00 & 573 & 0.9905 & \textbf{6.5}\\
\texttt{mnist\_2} & \multicolumn{1}{r}{60000} & \multicolumn{1}{r}{784}  & \textbf{1973} & 0.9671 & 1315.8 & 0.00 & 2158 & 0.9640 & 3275.1 & 0.00 & - & - & - & 0.00 & 2058 & 0.9657 & \textbf{7.2}\\
\texttt{mnist\_3} & \multicolumn{1}{r}{60000} & \multicolumn{1}{r}{784}  & \textbf{1321} & 0.9780 & 372.5 & 0.00 & 2022 & 0.9663 & 659.4 & 0.00 & - & - & - & 0.00 & 1442 & 0.9760 & \textbf{6.9}\\
\texttt{mnist\_4} & \multicolumn{1}{r}{60000} & \multicolumn{1}{r}{784}  & \textbf{1207} & 0.9799 & 139.0 & 0.00 & 1870 & 0.9688 & 1050.3 & 0.00 & 5580 & 0.9070 & 3600.4 & 0.00 & 1306 & 0.9782 & \textbf{5.4}\\
\texttt{mnist\_5} & \multicolumn{1}{r}{60000} & \multicolumn{1}{r}{784}  & 2454 & 0.9591 & 183.0 & 0.00 & \textbf{2327} & 0.9612 & 1272.6 & 0.00 & 4379 & 0.9270 & 3600.4 & 0.00 & 2553 & 0.9575 & \textbf{9.1}\\
\texttt{mnist\_6} & \multicolumn{1}{r}{60000} & \multicolumn{1}{r}{784}  & 1188 & 0.9802 & 861.6 & 0.00 & \textbf{1070} & 0.9822 & 1626.5 & 0.00 & 2718 & 0.9547 & 3600.3 & 0.00 & 1245 & 0.9792 & \textbf{6.2}\\
\texttt{mnist\_7} & \multicolumn{1}{r}{60000} & \multicolumn{1}{r}{784}  & \textbf{1294} & 0.9784 & 2894.8 & 0.00 & 1576 & 0.9737 & 560.9 & 0.00 & 4546 & 0.9242 & 3600.4 & 0.00 & 1371 & 0.9771 & \textbf{7.2}\\
\texttt{mnist\_8} & \multicolumn{1}{r}{60000} & \multicolumn{1}{r}{784}  & \textbf{1164} & 0.9806 & 3270.8 & 0.00 & 2051 & 0.9658 & 99.7 & 0.00 & - & - & - & 0.00 & 1267 & 0.9789 & \textbf{6.9}\\
\texttt{mnist\_9} & \multicolumn{1}{r}{60000} & \multicolumn{1}{r}{784}  & \textbf{1969} & 0.9672 & 211.3 & 0.00 & 2479 & 0.9587 & 541.8 & 0.00 & - & - & - & 0.00 & 2110 & 0.9648 & \textbf{9.3}\\
\texttt{mushroom} & \multicolumn{1}{r}{8124} & \multicolumn{1}{r}{119}  & 0 & 1.0000 & \textbf{0.0} & 1.00 & 0 & 1.0000 & 0.0 & 1.00 & 0 & 1.0000 & 1.6 & 1.00 & 0 & 1.0000 & 0.0\\
\texttt{pendigits} & \multicolumn{1}{r}{7494} & \multicolumn{1}{r}{216}  & 0 & 1.0000 & \textbf{0.0} & 1.00 & 0 & 1.0000 & 0.1 & 1.00 & - & - & - & 0.00 & 0 & 1.0000 & 0.1\\
\texttt{primary-tumor} & \multicolumn{1}{r}{336} & \multicolumn{1}{r}{31}  & 15 & 0.9554 & \textbf{0.0} & \textbf{1.00} & 15 & 0.9554 & 668.4 & 0.00 & - & - & - & 0.00 & 21 & 0.9375 & 0.0\\
\texttt{segment} & \multicolumn{1}{r}{2310} & \multicolumn{1}{r}{235}  & 0 & 1.0000 & \textbf{0.0} & 1.00 & 0 & 1.0000 & 0.0 & 1.00 & 0 & 1.0000 & 0.2 & 1.00 & 0 & 1.0000 & 0.0\\
\texttt{soybean} & \multicolumn{1}{r}{630} & \multicolumn{1}{r}{50}  & 2 & 0.9968 & 0.0 & \textbf{1.00} & 2 & 0.9968 & 929.1 & 0.00 & - & - & - & 0.00 & 5 & 0.9921 & \textbf{0.0}\\
\texttt{splice-1} & \multicolumn{1}{r}{3190} & \multicolumn{1}{r}{287}  & \textbf{12} & 0.9962 & 220.2 & 0.00 & 24 & 0.9925 & 300.3 & 0.00 & - & - & - & 0.00 & 18 & 0.9944 & \textbf{0.1}\\
\texttt{surgical-deepnet} & \multicolumn{1}{r}{14635} & \multicolumn{1}{r}{6047}  & \textbf{1149} & 0.9215 & 693.4 & 0.00 & 1503 & 0.8973 & 543.2 & 0.00 & - & - & - & 0.00 & 1193 & 0.9185 & \textbf{10.9}\\
\texttt{taiwan\_binarised} & \multicolumn{1}{r}{30000} & \multicolumn{1}{r}{205}  & \textbf{4669} & 0.8444 & 3384.6 & 0.00 & 5123 & 0.8292 & 1880.4 & 0.00 & - & - & - & 0.00 & 4911 & 0.8363 & \textbf{0.6}\\
\texttt{tic-tac-toe} & \multicolumn{1}{r}{958} & \multicolumn{1}{r}{27}  & 0 & 1.0000 & \textbf{0.0} & 1.00 & 0 & 1.0000 & 0.0 & 1.00 & 0 & 1.0000 & 0.2 & 1.00 & 10 & 0.9896 & 0.0\\
\texttt{titanic} & \multicolumn{1}{r}{887} & \multicolumn{1}{r}{333}  & \textbf{67} & 0.9245 & 2480.3 & 0.00 & 85 & 0.9042 & 3146.2 & 0.00 & - & - & - & 0.00 & 93 & 0.8952 & \textbf{0.0}\\
\texttt{vehicle} & \multicolumn{1}{r}{846} & \multicolumn{1}{r}{252}  & 0 & 1.0000 & \textbf{0.0} & 1.00 & 0 & 1.0000 & 0.1 & 1.00 & 0 & 1.0000 & 6.9 & 1.00 & 1 & 0.9988 & 0.0\\
\texttt{vote} & \multicolumn{1}{r}{435} & \multicolumn{1}{r}{48}  & 0 & 1.0000 & \textbf{0.0} & 1.00 & 0 & 1.0000 & 0.0 & 1.00 & 0 & 1.0000 & 0.0 & 1.00 & 1 & 0.9977 & 0.0\\
\texttt{weather-aus} & \multicolumn{1}{r}{142193} & \multicolumn{1}{r}{4759}  & \textbf{1664} & 0.9883 & 971.0 & 0.00 & 1694 & 0.9881 & 666.2 & 0.00 & - & - & - & 0.00 & 1677 & 0.9882 & \textbf{27.1}\\
\texttt{wine1} & \multicolumn{1}{r}{178} & \multicolumn{1}{r}{1276}  & \textbf{24} & 0.8652 & 1543.7 & 0.00 & 25 & 0.8596 & 252.8 & 0.00 & - & - & - & 0.00 & 27 & 0.8483 & \textbf{0.0}\\
\texttt{wine2} & \multicolumn{1}{r}{178} & \multicolumn{1}{r}{1276}  & \textbf{27} & 0.8483 & 505.5 & 0.00 & 28 & 0.8427 & 860.2 & 0.00 & - & - & - & 0.00 & 32 & 0.8202 & \textbf{0.0}\\
\texttt{wine3} & \multicolumn{1}{r}{178} & \multicolumn{1}{r}{1276}  & 18 & 0.8989 & 319.3 & 0.00 & \textbf{17} & 0.9045 & 3237.0 & 0.00 & - & - & - & 0.00 & 18 & 0.8989 & \textbf{0.0}\\
\texttt{yeast} & \multicolumn{1}{r}{1484} & \multicolumn{1}{r}{89}  & \textbf{61} & 0.9589 & 2386.8 & 0.00 & 207 & 0.8605 & 801.7 & 0.00 & - & - & - & 0.00 & 232 & 0.8437 & \textbf{0.0}\\
\bottomrule
\end{tabular}

\end{footnotesize}
\end{center}
\caption{\label{tab:all9} Comparison with state of the art: depth 9}
\end{table}

\begin{table}[htbp]
\begin{center}
\begin{footnotesize}
\tabcolsep=2pt
\begin{tabular}{lccrrrrrrrrrrrr}
\toprule
\multirow{2}{*}{}& && \multicolumn{2}{c}{\budalg} & \multicolumn{2}{c}{\murtree} & \multicolumn{2}{c}{\dleight} & \multicolumn{2}{c}{\cp} & \multicolumn{2}{c}{binoct} & \multicolumn{2}{c}{\cart}\\
\cmidrule(rr){4-5}\cmidrule(rr){6-7}\cmidrule(rr){8-9}\cmidrule(rr){10-11}\cmidrule(rr){12-13}\cmidrule(rr){14-15}
&\multirow{1}{*}{$\#ex.$} & \multirow{1}{*}{\#feat.} &  \multicolumn{1}{c}{error} & \multicolumn{1}{c}{cpu} & \multicolumn{1}{c}{error} & \multicolumn{1}{c}{cpu} & \multicolumn{1}{c}{error} & \multicolumn{1}{c}{cpu} & \multicolumn{1}{c}{error} & \multicolumn{1}{c}{cpu} & \multicolumn{1}{c}{error} & \multicolumn{1}{c}{cpu} & \multicolumn{1}{c}{error} & \multicolumn{1}{c}{cpu} \\
\midrule

\texttt{adult\_discretized} & \multicolumn{1}{r}{30299} & \multicolumn{1}{r}{59}  & \textbf{3841} & 2632 & 4052 & 1695 & - & - & 7511 & $\mathsmaller{\geq}1$h & - & - & 4148 & 0.12\\
\texttt{anneal} & \multicolumn{1}{r}{812} & \multicolumn{1}{r}{93}  & \textbf{34} & 23$^*$ & 39 & 225 & - & - & 187 & $\mathsmaller{\geq}1$h & - & - & 59 & 0.00\\
\texttt{audiology} & \multicolumn{1}{r}{216} & \multicolumn{1}{r}{148}  & 0 & 0.00$^*$ & 0 & 0.01$^*$ & 0 & 0.00$^*$ & 0 & 1.4$^*$ & 57 & 115 & 0 & 0.00\\
\texttt{australian-credit} & \multicolumn{1}{r}{653} & \multicolumn{1}{r}{125}  & 0 & 0.04$^*$ & 0 & 0.27$^*$ & - & - & 0 & 464$^*$ & - & - & 12 & 0.01\\
\texttt{bank} & \multicolumn{1}{r}{45211} & \multicolumn{1}{r}{9531}  & \textbf{3242} & 800 & 3767 & 3270 & 4826 & $\mathsmaller{\geq}1$h & 5289 & $\mathsmaller{\geq}1$h & - & - & 3327 & 102\\
\texttt{breast-cancer} & \multicolumn{1}{r}{683} & \multicolumn{1}{r}{89}  & 0 & 0.00$^*$ & 0 & 0.01$^*$ & 0 & 0.00$^*$ & 0 & 2.4$^*$ & - & - & 0 & 0.00\\
\texttt{breast-wisconsin} & \multicolumn{1}{r}{683} & \multicolumn{1}{r}{120}  & 0 & 0.00$^*$ & 0 & 0.01$^*$ & 0 & 3.4$^*$ & 0 & 7.8$^*$ & - & - & 0 & 0.00\\
\texttt{car} & \multicolumn{1}{r}{1728} & \multicolumn{1}{r}{21}  & 0 & 0.26$^*$ & 0 & 0.48$^*$ & 0 & 0.03$^*$ & 0 & 3.3$^*$ & 518 & 0.00 & 11 & 0.00\\
\texttt{compas\_discretized} & \multicolumn{1}{r}{6167} & \multicolumn{1}{r}{25}  & \textbf{1828} & 0.73$^*$ & 1842 & 2465 & - & - & 2809 & $\mathsmaller{\geq}1$h & - & - & 1871 & 0.01\\
\texttt{diabetes} & \multicolumn{1}{r}{768} & \multicolumn{1}{r}{112}  & 0 & 0.67$^*$ & 0 & 3.7$^*$ & - & - & 0 & 463$^*$ & - & - & 35 & 0.01\\
\texttt{forest-fires} & \multicolumn{1}{r}{517} & \multicolumn{1}{r}{989}  & \textbf{113} & 942 & 119 & 458 & - & - & 247 & $\mathsmaller{\geq}1$h & - & - & 146 & 0.02\\
\texttt{german-credit} & \multicolumn{1}{r}{1000} & \multicolumn{1}{r}{112}  & 0 & 69$^*$ & 0 & 74$^*$ & - & - & 0 & 28$^*$ & - & - & 66 & 0.01\\
\texttt{heart-cleveland} & \multicolumn{1}{r}{296} & \multicolumn{1}{r}{95}  & 0 & 0.00$^*$ & 0 & 0.02$^*$ & 0 & 0.08$^*$ & 0 & 1.2$^*$ & 160 & 88 & 0 & 0.00\\
\texttt{hepatitis} & \multicolumn{1}{r}{137} & \multicolumn{1}{r}{68}  & 0 & 0.00$^*$ & 0 & 0.00$^*$ & 0 & 0.00$^*$ & 0 & 1.3$^*$ & 111 & 733 & 0 & 0.00\\
\texttt{hypothyroid} & \multicolumn{1}{r}{3247} & \multicolumn{1}{r}{88}  & 17 & 0.96$^*$ & 17 & 517 & - & - & 277 & $\mathsmaller{\geq}1$h & - & - & 31 & 0.01\\
\texttt{ionosphere} & \multicolumn{1}{r}{351} & \multicolumn{1}{r}{445}  & 0 & 0.00$^*$ & 0 & 0.14$^*$ & 0 & 110$^*$ & 0 & 8.1$^*$ & - & - & 0 & 0.01\\
\texttt{kr-vs-kp} & \multicolumn{1}{r}{3196} & \multicolumn{1}{r}{73}  & \textbf{0} & 1897$^*$ & 24 & 711 & - & - & 784 & $\mathsmaller{\geq}1$h & - & - & 12 & 0.01\\
\texttt{letter} & \multicolumn{1}{r}{20000} & \multicolumn{1}{r}{224}  & 0 & 79$^*$ & 0 & 278$^*$ & 725 & $\mathsmaller{\geq}1$h & 813 & $\mathsmaller{\geq}1$h & - & - & 21 & 0.31\\
\texttt{lymph} & \multicolumn{1}{r}{148} & \multicolumn{1}{r}{68}  & 0 & 0.00$^*$ & 0 & 0.00$^*$ & 0 & 0.00$^*$ & 0 & 1.2$^*$ & 81 & 82 & 0 & 0.00\\
\texttt{mnist\_0} & \multicolumn{1}{r}{60000} & \multicolumn{1}{r}{784}  & \textbf{383} & 413 & 880 & 101 & 3314 & $\mathsmaller{\geq}1$h & 5923 & $\mathsmaller{\geq}1$h & - & - & 477 & 8.5\\
\texttt{mnist\_1} & \multicolumn{1}{r}{60000} & \multicolumn{1}{r}{784}  & \textbf{331} & 360 & 610 & 2286 & 4544 & $\mathsmaller{\geq}1$h & 6742 & $\mathsmaller{\geq}1$h & - & - & 439 & 7.8\\
\texttt{mnist\_2} & \multicolumn{1}{r}{60000} & \multicolumn{1}{r}{784}  & \textbf{1522} & 2520 & 1899 & 3408 & - & - & 5958 & $\mathsmaller{\geq}1$h & - & - & 1959 & 8.7\\
\texttt{mnist\_3} & \multicolumn{1}{r}{60000} & \multicolumn{1}{r}{784}  & \textbf{1079} & 376 & 1568 & 290 & 5171 & $\mathsmaller{\geq}1$h & 6131 & $\mathsmaller{\geq}1$h & - & - & 1169 & 6.7\\
\texttt{mnist\_4} & \multicolumn{1}{r}{60000} & \multicolumn{1}{r}{784}  & \textbf{801} & 398 & 1517 & 404 & 5580 & $\mathsmaller{\geq}1$h & 5842 & $\mathsmaller{\geq}1$h & - & - & 1010 & 10\\
\texttt{mnist\_5} & \multicolumn{1}{r}{60000} & \multicolumn{1}{r}{784}  & \textbf{1973} & 491 & 2088 & 988 & 4379 & $\mathsmaller{\geq}1$h & 5421 & $\mathsmaller{\geq}1$h & - & - & 2266 & 6.9\\
\texttt{mnist\_6} & \multicolumn{1}{r}{60000} & \multicolumn{1}{r}{784}  & 965 & 1742 & \textbf{952} & 1178 & 2699 & $\mathsmaller{\geq}1$h & 5918 & $\mathsmaller{\geq}1$h & - & - & 1211 & 7.4\\
\texttt{mnist\_7} & \multicolumn{1}{r}{60000} & \multicolumn{1}{r}{784}  & \textbf{1082} & 1277 & 1264 & 1761 & - & - & 6265 & $\mathsmaller{\geq}1$h & - & - & 1263 & 11\\
\texttt{mnist\_8} & \multicolumn{1}{r}{60000} & \multicolumn{1}{r}{784}  & \textbf{696} & 2141 & 1556 & 3399 & - & - & 5851 & $\mathsmaller{\geq}1$h & - & - & 916 & 7.9\\
\texttt{mnist\_9} & \multicolumn{1}{r}{60000} & \multicolumn{1}{r}{784}  & \textbf{1594} & 2124 & 2221 & 3379 & - & - & 5949 & $\mathsmaller{\geq}1$h & - & - & 1722 & 7.1\\
\texttt{mushroom} & \multicolumn{1}{r}{8124} & \multicolumn{1}{r}{119}  & 0 & 0.00$^*$ & 0 & 0.02$^*$ & 0 & 1.1$^*$ & 0 & 1.2$^*$ & - & - & 0 & 0.04\\
\texttt{pendigits} & \multicolumn{1}{r}{7494} & \multicolumn{1}{r}{216}  & 0 & 0.00$^*$ & 0 & 0.11$^*$ & 0 & 1247$^*$ & 0 & 5.3$^*$ & - & - & 0 & 0.07\\
\texttt{primary-tumor} & \multicolumn{1}{r}{336} & \multicolumn{1}{r}{31}  & 15 & 0.00$^*$ & 15 & 1820 & - & - & 82 & $\mathsmaller{\geq}1$h & 82 & 80 & 20 & 0.00\\
\texttt{segment} & \multicolumn{1}{r}{2310} & \multicolumn{1}{r}{235}  & 0 & 0.00$^*$ & 0 & 0.01$^*$ & 0 & 0.08$^*$ & 0 & 1.9$^*$ & - & - & 0 & 0.01\\
\texttt{soybean} & \multicolumn{1}{r}{630} & \multicolumn{1}{r}{50}  & 2 & 0.00$^*$ & 2 & 18 & - & - & 92 & $\mathsmaller{\geq}1$h & 92 & 112 & 2 & 0.00\\
\texttt{splice-1} & \multicolumn{1}{r}{3190} & \multicolumn{1}{r}{287}  & \textbf{5} & 1160 & 13 & 993 & - & - & 1535 & $\mathsmaller{\geq}1$h & - & - & 12 & 0.05\\
\texttt{surgical-deepnet} & \multicolumn{1}{r}{14635} & \multicolumn{1}{r}{6047}  & \textbf{965} & 2865 & 1382 & 885 & - & - & 3690 & $\mathsmaller{\geq}1$h & - & - & 1089 & 14\\
\texttt{taiwan\_binarised} & \multicolumn{1}{r}{30000} & \multicolumn{1}{r}{205}  & \textbf{4217} & 1001 & 4993 & 1437 & - & - & 6636 & $\mathsmaller{\geq}1$h & - & - & 4710 & 0.54\\
\texttt{tic-tac-toe} & \multicolumn{1}{r}{958} & \multicolumn{1}{r}{27}  & 0 & 0.00$^*$ & 0 & 0.01$^*$ & 0 & 0.03$^*$ & 0 & 0.81$^*$ & 626 & 89 & 6 & 0.00\\
\texttt{titanic} & \multicolumn{1}{r}{887} & \multicolumn{1}{r}{333}  & \textbf{35} & 3059 & 77 & 1852 & - & - & 342 & $\mathsmaller{\geq}1$h & - & - & 78 & 0.01\\
\texttt{vehicle} & \multicolumn{1}{r}{846} & \multicolumn{1}{r}{252}  & 0 & 0.00$^*$ & 0 & 0.08$^*$ & 0 & 0.37$^*$ & 0 & 4.2$^*$ & - & - & 0 & 0.01\\
\texttt{vote} & \multicolumn{1}{r}{435} & \multicolumn{1}{r}{48}  & 0 & 0.00$^*$ & 0 & 0.00$^*$ & 0 & 0.00$^*$ & 0 & 2.3$^*$ & 267 & 68 & 0 & 0.00\\
\texttt{weather-aus} & \multicolumn{1}{r}{142193} & \multicolumn{1}{r}{4759}  & \textbf{1601} & 2591 & 1675 & 2076 & - & - & 1761 & $\mathsmaller{\geq}1$h & - & - & 1642 & 32\\
\texttt{wine1} & \multicolumn{1}{r}{178} & \multicolumn{1}{r}{1276}  & 22 & 545 & 22 & 319 & - & - & 27 & $\mathsmaller{\geq}1$h & - & - & 25 & 0.01\\
\texttt{wine2} & \multicolumn{1}{r}{178} & \multicolumn{1}{r}{1276}  & 24 & 399 & 24 & 2440 & - & - & 29 & $\mathsmaller{\geq}1$h & - & - & 29 & 0.02\\
\texttt{wine3} & \multicolumn{1}{r}{178} & \multicolumn{1}{r}{1276}  & 16 & 272 & \textbf{15} & 151 & - & - & 19 & $\mathsmaller{\geq}1$h & - & - & 19 & 0.01\\
\texttt{yeast} & \multicolumn{1}{r}{1484} & \multicolumn{1}{r}{89}  & \textbf{28} & 1008 & 170 & 962 & - & - & 463 & $\mathsmaller{\geq}1$h & - & - & 185 & 0.01\\
\bottomrule
\end{tabular}

\end{footnotesize}
\end{center}
\caption{\label{tab:all10} Comparison with state of the art: depth 10}
\end{table}




\begin{table}[htbp]%
\begin{center}%
\begin{footnotesize}%
\tabcolsep=2pt%
\begin{tabular}{lccrrrrrrrr}
\toprule
\multirow{2}{*}{}& && \multicolumn{2}{c}{\budalg} & \multicolumn{2}{c}{\noheuristic} & \multicolumn{2}{c}{\nopreprocessing} & \multicolumn{2}{c}{\nolb}\\
\cmidrule(rr){4-5}\cmidrule(rr){6-7}\cmidrule(rr){8-9}\cmidrule(rr){10-11}
&\multirow{1}{*}{('$|\\allex|$', '$|\\features|$')} & \multirow{1}{*}{(<function <lambda> at 0x7f593885b578>, <function <lambda> at 0x7f593885b320>)} &  \multicolumn{1}{c}{error} & \multicolumn{1}{c}{cpu} & \multicolumn{1}{c}{error} & \multicolumn{1}{c}{cpu} & \multicolumn{1}{c}{error} & \multicolumn{1}{c}{cpu} & \multicolumn{1}{c}{error} & \multicolumn{1}{c}{cpu} \\
\midrule
%
\end{footnotesize}%
\end{center}%
\caption{\label{tab:fact3} Factor analysis: depth 3}%
\end{table}%

\begin{table}[htbp]
\begin{center}
\begin{footnotesize}
\tabcolsep=2pt
\begin{tabular}{lccrrrrrrrr}
\toprule
\multirow{2}{*}{}& && \multicolumn{2}{c}{\budalg} & \multicolumn{2}{c}{\noheuristic} & \multicolumn{2}{c}{\nopreprocessing} & \multicolumn{2}{c}{\nolb}\\
\cmidrule(rr){4-5}\cmidrule(rr){6-7}\cmidrule(rr){8-9}\cmidrule(rr){10-11}
&\multirow{1}{*}{$|\allex|$} & \multirow{1}{*}{$|\features|$} &  \multicolumn{1}{c}{error} & \multicolumn{1}{c}{cpu} & \multicolumn{1}{c}{error} & \multicolumn{1}{c}{cpu} & \multicolumn{1}{c}{error} & \multicolumn{1}{c}{cpu} & \multicolumn{1}{c}{error} & \multicolumn{1}{c}{cpu} \\
\midrule

\texttt{hepatitis} & 3 & 0.32$^*$ & 3 & 0.20$^*$ & 3 & 3.0$^*$ & 3 & 0.31$^*$\\
\texttt{lymph} & 3 & 0.74$^*$ & 3 & 0.57$^*$ & 3 & 2.4$^*$ & 3 & 0.91$^*$\\
\texttt{wine1} & 37 & 1674 & 37 & 1808 & 38 & 2248 & 37 & 1617\\
\texttt{wine2} & 43 & 17 & 43 & 0.02 & 43 & 110 & 43 & 16\\
\texttt{wine3} & 28 & 33 & 28 & 190 & 28 & 222 & 28 & 33\\
\texttt{audiology} & 1 & 4.0$^*$ & 1 & 3.2$^*$ & 1 & 29$^*$ & 1 & 4.5$^*$\\
\texttt{heart-cleveland} & 25 & 3.1$^*$ & 25 & 2.3$^*$ & 25 & 19$^*$ & 25 & 3.3$^*$\\
\texttt{primary-tumor} & 34 & 0.03$^*$ & 34 & 0.02$^*$ & 34 & 0.22$^*$ & 34 & 0.03$^*$\\
\texttt{ionosphere} & 7 & 730$^*$ & 7 & 548$^*$ & 8 & 55 & 7 & 1026$^*$\\
\texttt{vote} & 5 & 1.2$^*$ & 5 & 0.91$^*$ & 5 & 1.2$^*$ & 5 & 1.4$^*$\\
\texttt{forest-fires} & 173 & 15 & 173 & 11 & 173 & 48 & 173 & 15\\
\texttt{soybean} & 14 & 0.62$^*$ & 14 & 0.50$^*$ & 14 & 1.1$^*$ & 14 & 0.71$^*$\\
\texttt{australian-credit} & 56 & 10$^*$ & 56 & 8.5$^*$ & 56 & 68$^*$ & 56 & 11$^*$\\
\texttt{breast-cancer} & 16 & 9.6$^*$ & 16 & 7.6$^*$ & 16 & 9.1$^*$ & 16 & 8.9$^*$\\
\texttt{breast-wisconsin} & 7 & 3.1$^*$ & 7 & 2.1$^*$ & 7 & 33$^*$ & 7 & 3.4$^*$\\
\texttt{diabetes} & 137 & 5.7$^*$ & 137 & 4.8$^*$ & 137 & 59$^*$ & 137 & 6.0$^*$\\
\texttt{anneal} & 91 & 1.5$^*$ & 91 & 1.0$^*$ & 91 & 11$^*$ & 91 & 1.3$^*$\\
\texttt{vehicle} & 12 & 71$^*$ & 12 & 60$^*$ & 12 & 706$^*$ & 12 & 91$^*$\\
\texttt{titanic} & 119 & 1604$^*$ & 119 & 1318$^*$ & 119 & 1620$^*$ & 119 & 1722$^*$\\
\texttt{tic-tac-toe} & 137 & 0.38$^*$ & 137 & 0.34$^*$ & 137 & 0.38$^*$ & 137 & 0.38$^*$\\
\texttt{german-credit} & 204 & 28$^*$ & 204 & 22$^*$ & 204 & 66$^*$ & 204 & 29$^*$\\
\texttt{yeast} & 366 & 3.4$^*$ & 366 & 3.0$^*$ & 366 & 29$^*$ & 366 & 3.4$^*$\\
\texttt{car} & 136 & 0.19$^*$ & 136 & 0.16$^*$ & 136 & 0.14$^*$ & 136 & 0.16$^*$\\
\texttt{segment} & 0 & 0.00$^*$ & 0 & 0.00$^*$ & 0 & 0.00$^*$ & 0 & 0.00$^*$\\
\texttt{splice-1} & 141 & 3241$^*$ & 141 & 2519$^*$ & 141 & 0.00 & 141 & 3563$^*$\\
\texttt{kr-vs-kp} & 144 & 2.8$^*$ & 144 & 2.4$^*$ & 144 & 14$^*$ & 144 & 2.5$^*$\\
\texttt{hypothyroid} & 53 & 2.9$^*$ & 53 & 2.5$^*$ & 53 & 23$^*$ & 53 & 3.1$^*$\\
\texttt{compas\_discretized} & 1954 & 0.07$^*$ & 1954 & 0.05$^*$ & 1954 & 0.69$^*$ & 1954 & 0.07$^*$\\
\texttt{pendigits} & 13 & 230$^*$ & 13 & 237$^*$ & 13 & 1871$^*$ & 13 & 341$^*$\\
\texttt{mushroom} & 0 & 0.00$^*$ & 0 & 0.00$^*$ & 0 & 0.00$^*$ & 0 & 0.00$^*$\\
\texttt{surgical-deepnet} & 2269 & 49 & 2414 & 1479 & 2269 & 46 & 2269 & 51\\
\texttt{letter} & 261 & 1185$^*$ & 261 & 813$^*$ & 261 & 292 & 261 & 1407$^*$\\
\texttt{taiwan\_binarised} & 5273 & 6.2 & 5273 & 39 & 5273 & 6.2 & 5273 & 7.1\\
\texttt{adult\_discretized} & 4609 & 14$^*$ & 4609 & 14$^*$ & 4609 & 43$^*$ & 4609 & 14$^*$\\
\texttt{bank} & 4314 & 290 & 4326 & 1102 & 4314 & 258 & 4314 & 308\\
\texttt{mnist\_8} & 3165 & 1206 & 3572 & 2358 & 3165 & 1104 & 3165 & 1413\\
\texttt{mnist\_9} & 3977 & 2061 & 4182 & 3123 & 3977 & 1706 & 3977 & 1954\\
\texttt{mnist\_0} & 2173 & 2158 & 2229 & 3292 & 2173 & 1844 & 2173 & 2444\\
\texttt{mnist\_6} & 1940 & 2752 & 2009 & 2570 & 1940 & 2518 & 1940 & 2175\\
\texttt{mnist\_5} & 3312 & 219 & 3312 & 1971 & 3312 & 198 & 3312 & 230\\
\texttt{mnist\_3} & 3485 & 2225 & 4230 & 1476 & 3485 & 2025 & 3485 & 2435\\
\texttt{mnist\_2} & 3358 & 169 & 3647 & 2672 & 3143 & 3477 & 3143 & 3472\\
\texttt{mnist\_4} & 3670 & 2476 & 4466 & 500 & 3670 & 2258 & 3670 & 3062\\
\texttt{mnist\_7} & 2793 & 52 & 3212 & 787 & 2793 & 50 & 2793 & 57\\
\texttt{mnist\_1} & 2332 & 2248 & 3462 & 563 & 2332 & 2211 & 2332 & 2698\\
\texttt{weather-aus} & 1749 & 2525 & 1750 & 2646 & 1749 & 2142 & 1749 & 2638\\
\bottomrule
\end{tabular}

\end{footnotesize}
\end{center}
\caption{\label{tab:fact4} Factor analysis: depth 4}
\end{table}

\begin{table}[htbp]
\begin{center}
\begin{footnotesize}
\tabcolsep=2pt
\begin{tabular}{lccrrrrrrrrrrrr}
\toprule
\multirow{2}{*}{}& && \multicolumn{3}{c}{\budalg} & \multicolumn{3}{c}{\noheuristic} & \multicolumn{3}{c}{\nopreprocessing} & \multicolumn{3}{c}{\nolb}\\
\cmidrule(rr){4-6}\cmidrule(rr){7-9}\cmidrule(rr){10-12}\cmidrule(rr){13-15}
&\multirow{1}{*}{$\#ex.$} & \multirow{1}{*}{\#feat.} &  \multicolumn{1}{c}{error} & \multicolumn{1}{c}{cpu} & \multicolumn{1}{c}{opt.} & \multicolumn{1}{c}{error} & \multicolumn{1}{c}{cpu} & \multicolumn{1}{c}{opt.} & \multicolumn{1}{c}{error} & \multicolumn{1}{c}{cpu} & \multicolumn{1}{c}{opt.} & \multicolumn{1}{c}{error} & \multicolumn{1}{c}{cpu} & \multicolumn{1}{c}{opt.} \\
\midrule

\texttt{adult\_discretized} & \multicolumn{1}{r}{30299} & \multicolumn{1}{r}{59}  & 4423 & 725.1 & 1 & 4423 & \textbf{693.3} & 1 & 4423 & 2388.0 & 1 & 4423 & 754.7 & 1\\
\texttt{anneal} & \multicolumn{1}{r}{812} & \multicolumn{1}{r}{93}  & 70 & 43.9 & 1 & 70 & \textbf{37.7} & 1 & 70 & 736.5 & 1 & 70 & 50.4 & 1\\
\texttt{audiology} & \multicolumn{1}{r}{216} & \multicolumn{1}{r}{148}  & 0 & 0.0 & 1 & 0 & 0.0 & 1 & 0 & 0.0 & 1 & 0 & 0.0 & 1\\
\texttt{australian-credit} & \multicolumn{1}{r}{653} & \multicolumn{1}{r}{125}  & 39 & 657.5 & 1 & 39 & 513.0 & 1 & 40 & \textbf{39.9} & 0 & 39 & 839.1 & 1\\
\texttt{bank} & \multicolumn{1}{r}{45211} & \multicolumn{1}{r}{9531}  & 4187 & 1151.9 & 0 & 4309 & 1112.8 & 0 & 4187 & \textbf{1073.2} & 0 & 4187 & 1205.4 & 0\\
\texttt{breast-cancer} & \multicolumn{1}{r}{683} & \multicolumn{1}{r}{89}  & 6 & 725.0 & 1 & 6 & \textbf{603.8} & 1 & 6 & 763.8 & 1 & 6 & 764.0 & 1\\
\texttt{breast-wisconsin} & \multicolumn{1}{r}{683} & \multicolumn{1}{r}{120}  & 0 & 19.9 & 1 & 0 & \textbf{15.8} & 1 & 0 & 477.7 & 1 & 0 & 31.2 & 1\\
\texttt{car} & \multicolumn{1}{r}{1728} & \multicolumn{1}{r}{21}  & 86 & \textbf{2.4} & 1 & 86 & 2.5 & 1 & 86 & 2.5 & 1 & 86 & 2.9 & 1\\
\texttt{compas\_discretized} & \multicolumn{1}{r}{6167} & \multicolumn{1}{r}{25}  & 1919 & \textbf{1.1} & 1 & 1919 & 1.1 & 1 & 1919 & 14.0 & 1 & 1919 & 1.3 & 1\\
\texttt{diabetes} & \multicolumn{1}{r}{768} & \multicolumn{1}{r}{112}  & 106 & 312.4 & 1 & 106 & \textbf{244.8} & 1 & 106 & 1425.1 & 0 & 106 & 357.1 & 1\\
\texttt{forest-fires} & \multicolumn{1}{r}{517} & \multicolumn{1}{r}{989}  & 156 & 777.0 & 0 & 157 & \textbf{60.7} & 0 & 156 & 2891.0 & 0 & 156 & 759.8 & 0\\
\texttt{german-credit} & \multicolumn{1}{r}{1000} & \multicolumn{1}{r}{112}  & 161 & 2741.0 & 1 & 161 & 2037.3 & 1 & 161 & \textbf{82.3} & 0 & 161 & 2885.0 & 1\\
\texttt{heart-cleveland} & \multicolumn{1}{r}{296} & \multicolumn{1}{r}{95}  & 7 & 93.5 & 1 & 7 & \textbf{77.8} & 1 & 7 & 1223.6 & 1 & 7 & 155.5 & 1\\
\texttt{hepatitis} & \multicolumn{1}{r}{137} & \multicolumn{1}{r}{68}  & 0 & 0.1 & 1 & 0 & 0.1 & 1 & 0 & 0.4 & 1 & 0 & 0.1 & 1\\
\texttt{hypothyroid} & \multicolumn{1}{r}{3247} & \multicolumn{1}{r}{88}  & 44 & 87.4 & 1 & 44 & \textbf{85.4} & 1 & 44 & 1538.8 & 1 & 44 & 103.2 & 1\\
\texttt{ionosphere} & \multicolumn{1}{r}{351} & \multicolumn{1}{r}{445}  & 0 & 506.0 & 1 & 0 & \textbf{444.4} & 1 & 2 & 1746.3 & 0 & 0 & 805.8 & 1\\
\texttt{kr-vs-kp} & \multicolumn{1}{r}{3196} & \multicolumn{1}{r}{73}  & 81 & \textbf{64.6} & 1 & 81 & 64.7 & 1 & 81 & 823.1 & 1 & 81 & 80.7 & 1\\
\texttt{letter} & \multicolumn{1}{r}{20000} & \multicolumn{1}{r}{224}  & \textbf{168} & 3082.5 & 0 & 172 & 2110.2 & 0 & 192 & \textbf{207.6} & 0 & 173 & 2313.2 & 0\\
\texttt{lymph} & \multicolumn{1}{r}{148} & \multicolumn{1}{r}{68}  & 0 & 0.0 & 1 & 0 & 0.0 & 1 & 0 & 0.0 & 1 & 0 & 0.0 & 1\\
\texttt{mnist\_0} & \multicolumn{1}{r}{60000} & \multicolumn{1}{r}{784}  & 1714 & 283.8 & 0 & 2075 & 1861.5 & 0 & 1714 & \textbf{240.5} & 0 & 1714 & 300.2 & 0\\
\texttt{mnist\_1} & \multicolumn{1}{r}{60000} & \multicolumn{1}{r}{784}  & 1585 & 3111.0 & 0 & 2852 & 3369.2 & 0 & 1585 & \textbf{2451.8} & 0 & 1585 & 2471.8 & 0\\
\texttt{mnist\_2} & \multicolumn{1}{r}{60000} & \multicolumn{1}{r}{784}  & 3118 & 3229.5 & 0 & 3440 & \textbf{1361.5} & 0 & 3118 & 3214.3 & 0 & 3118 & 3569.6 & 0\\
\texttt{mnist\_3} & \multicolumn{1}{r}{60000} & \multicolumn{1}{r}{784}  & 2893 & 1935.6 & 0 & 4187 & 2564.0 & 0 & 2893 & \textbf{1606.8} & 0 & 2893 & 2304.8 & 0\\
\texttt{mnist\_4} & \multicolumn{1}{r}{60000} & \multicolumn{1}{r}{784}  & 2864 & 707.9 & 0 & 3766 & 3356.9 & 0 & 2864 & \textbf{500.7} & 0 & 2864 & 1033.8 & 0\\
\texttt{mnist\_5} & \multicolumn{1}{r}{60000} & \multicolumn{1}{r}{784}  & 3138 & 2411.4 & 0 & 3241 & 2433.8 & 0 & 3138 & \textbf{2093.8} & 0 & 3138 & 2674.9 & 0\\
\texttt{mnist\_6} & \multicolumn{1}{r}{60000} & \multicolumn{1}{r}{784}  & 1485 & 2097.4 & 0 & 1989 & 2723.0 & 0 & 1485 & \textbf{1527.7} & 0 & 1485 & 2139.5 & 0\\
\texttt{mnist\_7} & \multicolumn{1}{r}{60000} & \multicolumn{1}{r}{784}  & 2532 & 1792.6 & 0 & 2919 & 2090.4 & 0 & 2532 & \textbf{1627.5} & 0 & 2532 & 1828.8 & 0\\
\texttt{mnist\_8} & \multicolumn{1}{r}{60000} & \multicolumn{1}{r}{784}  & 2547 & 2846.6 & 0 & 3022 & 2732.9 & 0 & 2547 & \textbf{2051.8} & 0 & 2547 & 3241.9 & 0\\
\texttt{mnist\_9} & \multicolumn{1}{r}{60000} & \multicolumn{1}{r}{784}  & 3352 & 1695.0 & 0 & 4066 & 2178.0 & 0 & 3352 & \textbf{1491.4} & 0 & 3352 & 1791.7 & 0\\
\texttt{mushroom} & \multicolumn{1}{r}{8124} & \multicolumn{1}{r}{119}  & 0 & 0.0 & 1 & 0 & 0.0 & 1 & 0 & 0.0 & 1 & 0 & 0.0 & 1\\
\texttt{pendigits} & \multicolumn{1}{r}{7494} & \multicolumn{1}{r}{216}  & 0 & 283.5 & 1 & 0 & 725.4 & 1 & 2 & \textbf{55.3} & 0 & 0 & 446.9 & 1\\
\texttt{primary-tumor} & \multicolumn{1}{r}{336} & \multicolumn{1}{r}{31}  & 26 & \textbf{0.4} & 1 & 26 & 0.4 & 1 & 26 & 6.7 & 1 & 26 & 0.5 & 1\\
\texttt{segment} & \multicolumn{1}{r}{2310} & \multicolumn{1}{r}{235}  & 0 & 0.0 & 1 & 0 & 0.0 & 1 & 0 & 0.0 & 1 & 0 & 0.0 & 1\\
\texttt{soybean} & \multicolumn{1}{r}{630} & \multicolumn{1}{r}{50}  & 8 & 19.6 & 1 & 8 & \textbf{15.7} & 1 & 8 & 39.8 & 1 & 8 & 25.8 & 1\\
\texttt{splice-1} & \multicolumn{1}{r}{3190} & \multicolumn{1}{r}{287}  & 101 & \textbf{23.8} & 0 & 101 & 1861.3 & 0 & 101 & 25.8 & 0 & 101 & 25.5 & 0\\
\texttt{surgical-deepnet} & \multicolumn{1}{r}{14635} & \multicolumn{1}{r}{6047}  & 2131 & 2167.6 & 0 & 2310 & 2836.1 & 0 & 2131 & \textbf{1932.0} & 0 & 2131 & 2286.5 & 0\\
\texttt{taiwan\_binarised} & \multicolumn{1}{r}{30000} & \multicolumn{1}{r}{205}  & 5200 & 104.6 & 0 & 5201 & 3306.3 & 0 & 5200 & \textbf{82.6} & 0 & 5200 & 115.3 & 0\\
\texttt{tic-tac-toe} & \multicolumn{1}{r}{958} & \multicolumn{1}{r}{27}  & 63 & 10.2 & 1 & 63 & \textbf{8.7} & 1 & 63 & 9.3 & 1 & 63 & 11.4 & 1\\
\texttt{titanic} & \multicolumn{1}{r}{887} & \multicolumn{1}{r}{333}  & 95 & 1427.7 & 0 & 95 & \textbf{1057.1} & 0 & 95 & 1464.4 & 0 & 95 & 1465.0 & 0\\
\texttt{vehicle} & \multicolumn{1}{r}{846} & \multicolumn{1}{r}{252}  & 1 & 690.2 & 0 & 1 & 3525.3 & 1 & 3 & \textbf{42.2} & 0 & 1 & 1142.4 & 0\\
\texttt{vote} & \multicolumn{1}{r}{435} & \multicolumn{1}{r}{48}  & 1 & 23.9 & 1 & 1 & \textbf{21.1} & 1 & 1 & 25.6 & 1 & 1 & 44.6 & 1\\
\texttt{weather-aus} & \multicolumn{1}{r}{142193} & \multicolumn{1}{r}{4759}  & 1735 & 419.4 & 0 & 1749 & 1835.3 & 0 & 1735 & \textbf{350.4} & 0 & 1735 & 400.5 & 0\\
\texttt{wine1} & \multicolumn{1}{r}{178} & \multicolumn{1}{r}{1276}  & 33 & 1154.5 & 0 & 33 & \textbf{950.1} & 0 & 34 & 1318.8 & 0 & 33 & 1158.5 & 0\\
\texttt{wine2} & \multicolumn{1}{r}{178} & \multicolumn{1}{r}{1276}  & 39 & 410.5 & 0 & \textbf{37} & \textbf{12.6} & 0 & 39 & 2755.6 & 0 & 39 & 409.1 & 0\\
\texttt{wine3} & \multicolumn{1}{r}{178} & \multicolumn{1}{r}{1276}  & 25 & 16.7 & 0 & 25 & 89.9 & 0 & 25 & 99.7 & 0 & 25 & \textbf{16.4} & 0\\
\texttt{yeast} & \multicolumn{1}{r}{1484} & \multicolumn{1}{r}{89}  & 313 & 139.2 & 1 & 313 & \textbf{122.6} & 1 & 313 & 2348.5 & 1 & 313 & 150.5 & 1\\
\bottomrule
\end{tabular}

\end{footnotesize}
\end{center}
\caption{\label{tab:fact5} Factor analysis: depth 5}
\end{table}

\begin{table}[htbp]
\begin{center}
\begin{footnotesize}
\tabcolsep=2pt
\begin{tabular}{lccrrrrrrrr}
\toprule
\multirow{2}{*}{}& && \multicolumn{2}{c}{\budalg} & \multicolumn{2}{c}{\noheuristic} & \multicolumn{2}{c}{\nopreprocessing} & \multicolumn{2}{c}{\nolb}\\
\cmidrule(rr){4-5}\cmidrule(rr){6-7}\cmidrule(rr){8-9}\cmidrule(rr){10-11}
&\multirow{1}{*}{$\#ex.$} & \multirow{1}{*}{\#feat.} &  \multicolumn{1}{c}{error} & \multicolumn{1}{c}{cpu} & \multicolumn{1}{c}{error} & \multicolumn{1}{c}{cpu} & \multicolumn{1}{c}{error} & \multicolumn{1}{c}{cpu} & \multicolumn{1}{c}{error} & \multicolumn{1}{c}{cpu} \\
\midrule

\texttt{adult\_discretized} & \multicolumn{1}{r}{30299} & \multicolumn{1}{r}{59}  & 4281 & 1326 & 4286 & 3325 & 4281 & 458 & 4281 & 1433\\
\texttt{anneal} & \multicolumn{1}{r}{812} & \multicolumn{1}{r}{93}  & 51 & 1330$^*$ & 51 & 1114$^*$ & 68 & 396 & 51 & 1799$^*$\\
\texttt{audiology} & \multicolumn{1}{r}{216} & \multicolumn{1}{r}{148}  & 0 & 0.00$^*$ & 0 & 0.00$^*$ & 0 & 0.00$^*$ & 0 & 0.00$^*$\\
\texttt{australian-credit} & \multicolumn{1}{r}{653} & \multicolumn{1}{r}{125}  & 15 & 342 & 15 & 267 & 15 & 1207 & 15 & 511\\
\texttt{bank} & \multicolumn{1}{r}{45211} & \multicolumn{1}{r}{9531}  & 4046 & 339 & 4311 & 519 & 4046 & 308 & 4046 & 353\\
\texttt{breast-cancer} & \multicolumn{1}{r}{683} & \multicolumn{1}{r}{89}  & 1 & 3328 & 1 & 2193 & 1 & 3329 & 1 & 3557\\
\texttt{breast-wisconsin} & \multicolumn{1}{r}{683} & \multicolumn{1}{r}{120}  & 0 & 5.9$^*$ & 0 & 0.27$^*$ & 0 & 133$^*$ & 0 & 8.4$^*$\\
\texttt{car} & \multicolumn{1}{r}{1728} & \multicolumn{1}{r}{21}  & 36 & 27$^*$ & 36 & 28$^*$ & 36 & 28$^*$ & 36 & 47$^*$\\
\texttt{compas\_discretized} & \multicolumn{1}{r}{6167} & \multicolumn{1}{r}{25}  & 1887 & 17$^*$ & 1887 & 16$^*$ & 1887 & 263$^*$ & 1887 & 21$^*$\\
\texttt{diabetes} & \multicolumn{1}{r}{768} & \multicolumn{1}{r}{112}  & \textbf{60} & 2706 & 62 & 591 & 62 & 3190 & 62 & 306\\
\texttt{forest-fires} & \multicolumn{1}{r}{517} & \multicolumn{1}{r}{989}  & 132 & 1934 & 137 & 234 & 137 & 1775 & 132 & 1955\\
\texttt{german-credit} & \multicolumn{1}{r}{1000} & \multicolumn{1}{r}{112}  & 101 & 2883 & 143 & 2121 & 113 & 114 & 101 & 3356\\
\texttt{heart-cleveland} & \multicolumn{1}{r}{296} & \multicolumn{1}{r}{95}  & 0 & 0.03$^*$ & 0 & 7.2$^*$ & 0 & 0.22$^*$ & 0 & 0.03$^*$\\
\texttt{hepatitis} & \multicolumn{1}{r}{137} & \multicolumn{1}{r}{68}  & 0 & 0.00$^*$ & 0 & 0.00$^*$ & 0 & 0.00$^*$ & 0 & 0.00$^*$\\
\texttt{hypothyroid} & \multicolumn{1}{r}{3247} & \multicolumn{1}{r}{88}  & 32 & 2391$^*$ & 32 & 2216$^*$ & 33 & 616 & 32 & 3353$^*$\\
\texttt{ionosphere} & \multicolumn{1}{r}{351} & \multicolumn{1}{r}{445}  & 0 & 4.4$^*$ & 0 & 4.3$^*$ & 0 & 51$^*$ & 0 & 4.8$^*$\\
\texttt{kr-vs-kp} & \multicolumn{1}{r}{3196} & \multicolumn{1}{r}{73}  & 45 & 1694$^*$ & 45 & 1599$^*$ & 47 & 3002 & 45 & 2469$^*$\\
\texttt{letter} & \multicolumn{1}{r}{20000} & \multicolumn{1}{r}{224}  & 118 & 2186 & 173 & 2285 & 139 & 25 & 118 & 2601\\
\texttt{lymph} & \multicolumn{1}{r}{148} & \multicolumn{1}{r}{68}  & 0 & 0.00$^*$ & 0 & 0.00$^*$ & 0 & 0.00$^*$ & 0 & 0.00$^*$\\
\texttt{mnist\_0} & \multicolumn{1}{r}{60000} & \multicolumn{1}{r}{784}  & 1468 & 2513 & 1604 & 3454 & 1468 & 2094 & 1468 & 2858\\
\texttt{mnist\_1} & \multicolumn{1}{r}{60000} & \multicolumn{1}{r}{784}  & 1167 & 1875 & 2582 & 3574 & 1167 & 2132 & 1167 & 2224\\
\texttt{mnist\_2} & \multicolumn{1}{r}{60000} & \multicolumn{1}{r}{784}  & 2519 & 230 & 3337 & 2858 & 2519 & 229 & 2519 & 259\\
\texttt{mnist\_3} & \multicolumn{1}{r}{60000} & \multicolumn{1}{r}{784}  & 2486 & 2793 & 4145 & 1515 & 2486 & 2753 & 2486 & 2486\\
\texttt{mnist\_4} & \multicolumn{1}{r}{60000} & \multicolumn{1}{r}{784}  & 2180 & 3375 & 3600 & 2538 & 2180 & 3122 & 2180 & 3358\\
\texttt{mnist\_5} & \multicolumn{1}{r}{60000} & \multicolumn{1}{r}{784}  & 2930 & 1759 & 3178 & 1909 & 2930 & 1637 & 2930 & 1610\\
\texttt{mnist\_6} & \multicolumn{1}{r}{60000} & \multicolumn{1}{r}{784}  & 1278 & 2111 & 1854 & 2072 & 1278 & 2159 & 1278 & 1839\\
\texttt{mnist\_7} & \multicolumn{1}{r}{60000} & \multicolumn{1}{r}{784}  & 2074 & 2012 & 2806 & 2962 & 2074 & 1964 & 2074 & 1920\\
\texttt{mnist\_8} & \multicolumn{1}{r}{60000} & \multicolumn{1}{r}{784}  & 2060 & 806 & 3392 & 2590 & 2060 & 912 & 2060 & 797\\
\texttt{mnist\_9} & \multicolumn{1}{r}{60000} & \multicolumn{1}{r}{784}  & 2879 & 2229 & 3959 & 3209 & 2879 & 1878 & 2879 & 2447\\
\texttt{mushroom} & \multicolumn{1}{r}{8124} & \multicolumn{1}{r}{119}  & 0 & 0.00$^*$ & 0 & 0.00$^*$ & 0 & 0.00$^*$ & 0 & 0.00$^*$\\
\texttt{pendigits} & \multicolumn{1}{r}{7494} & \multicolumn{1}{r}{216}  & 0 & 0.01$^*$ & 0 & 57$^*$ & 0 & 0.06$^*$ & 0 & 0.01$^*$\\
\texttt{primary-tumor} & \multicolumn{1}{r}{336} & \multicolumn{1}{r}{31}  & 18 & 3.1$^*$ & 18 & 3.0$^*$ & 18 & 140$^*$ & 18 & 4.8$^*$\\
\texttt{segment} & \multicolumn{1}{r}{2310} & \multicolumn{1}{r}{235}  & 0 & 0.00$^*$ & 0 & 0.00$^*$ & 0 & 0.00$^*$ & 0 & 0.00$^*$\\
\texttt{soybean} & \multicolumn{1}{r}{630} & \multicolumn{1}{r}{50}  & 3 & 354$^*$ & 3 & 307$^*$ & 3 & 1186$^*$ & 3 & 753$^*$\\
\texttt{splice-1} & \multicolumn{1}{r}{3190} & \multicolumn{1}{r}{287}  & 68 & $\mathsmaller{\geq}1$h & 77 & 2384 & 68 & 3406 & 68 & 3584\\
\texttt{surgical-deepnet} & \multicolumn{1}{r}{14635} & \multicolumn{1}{r}{6047}  & 1767 & 2343 & 2314 & 2370 & 1767 & 2257 & 1767 & 2442\\
\texttt{taiwan\_binarised} & \multicolumn{1}{r}{30000} & \multicolumn{1}{r}{205}  & 5073 & 1473 & 5095 & 3416 & 5073 & 2210 & 5073 & 1459\\
\texttt{tic-tac-toe} & \multicolumn{1}{r}{958} & \multicolumn{1}{r}{27}  & 12 & 126$^*$ & 12 & 112$^*$ & 12 & 127$^*$ & 12 & 257$^*$\\
\texttt{titanic} & \multicolumn{1}{r}{887} & \multicolumn{1}{r}{333}  & 78 & 1234 & 88 & 3045 & 78 & 1299 & 78 & 1327\\
\texttt{vehicle} & \multicolumn{1}{r}{846} & \multicolumn{1}{r}{252}  & 0 & 0.08$^*$ & 0 & 2783$^*$ & 0 & 0.44$^*$ & 0 & 0.08$^*$\\
\texttt{vote} & \multicolumn{1}{r}{435} & \multicolumn{1}{r}{48}  & 0 & 0.00$^*$ & 0 & 0.05$^*$ & 0 & 0.00$^*$ & 0 & 0.00$^*$\\
\texttt{weather-aus} & \multicolumn{1}{r}{142193} & \multicolumn{1}{r}{4759}  & 1713 & 418 & 1748 & 1696 & 1713 & 384 & 1713 & 412\\
\texttt{wine1} & \multicolumn{1}{r}{178} & \multicolumn{1}{r}{1276}  & 31 & 2113 & \textbf{30} & 1137 & 32 & 1017 & 31 & 2178\\
\texttt{wine2} & \multicolumn{1}{r}{178} & \multicolumn{1}{r}{1276}  & 34 & 44 & 34 & 12 & 34 & 282 & 34 & 44\\
\texttt{wine3} & \multicolumn{1}{r}{178} & \multicolumn{1}{r}{1276}  & 22 & 93 & 22 & 714 & 22 & 604 & 22 & 91\\
\texttt{yeast} & \multicolumn{1}{r}{1484} & \multicolumn{1}{r}{89}  & 245 & 388 & 245 & 2099 & 272 & 407 & 245 & 455\\
\bottomrule
\end{tabular}

\end{footnotesize}
\end{center}
\caption{\label{tab:fact6} Factor analysis: depth 6}
\end{table}

\begin{table}[htbp]
\begin{center}
\begin{footnotesize}
\tabcolsep=2pt
\begin{tabular}{lccrrrrrrrrrrrr}
\toprule
\multirow{2}{*}{}& && \multicolumn{3}{c}{\budalg} & \multicolumn{3}{c}{\noheuristic} & \multicolumn{3}{c}{\nopreprocessing} & \multicolumn{3}{c}{\nolb}\\
\cmidrule(rr){4-6}\cmidrule(rr){7-9}\cmidrule(rr){10-12}\cmidrule(rr){13-15}
&\multirow{1}{*}{$\#ex.$} & \multirow{1}{*}{\#feat.} &  \multicolumn{1}{c}{error} & \multicolumn{1}{c}{cpu} & \multicolumn{1}{c}{opt.} & \multicolumn{1}{c}{error} & \multicolumn{1}{c}{cpu} & \multicolumn{1}{c}{opt.} & \multicolumn{1}{c}{error} & \multicolumn{1}{c}{cpu} & \multicolumn{1}{c}{opt.} & \multicolumn{1}{c}{error} & \multicolumn{1}{c}{cpu} & \multicolumn{1}{c}{opt.} \\
\midrule

\texttt{adult\_discretized} & \multicolumn{1}{r}{30299} & \multicolumn{1}{r}{59}  & 4191 & \textbf{534.2} & 0 & 4203 & 685.6 & 0 & \textbf{4162} & 2417.7 & 0 & 4191 & 553.1 & 0\\
\texttt{anneal} & \multicolumn{1}{r}{812} & \multicolumn{1}{r}{93}  & \textbf{41} & 3035.9 & 0 & 49 & 2818.3 & 0 & 58 & 272.1 & 0 & 50 & \textbf{232.0} & 0\\
\texttt{audiology} & \multicolumn{1}{r}{216} & \multicolumn{1}{r}{148}  & 0 & 0.0 & 1 & 0 & 0.0 & 1 & 0 & 0.0 & 1 & 0 & 0.0 & 1\\
\texttt{australian-credit} & \multicolumn{1}{r}{653} & \multicolumn{1}{r}{125}  & 0 & \textbf{101.3} & 1 & 0 & 477.3 & 1 & 0 & 1002.5 & 1 & 0 & 153.0 & 1\\
\texttt{bank} & \multicolumn{1}{r}{45211} & \multicolumn{1}{r}{9531}  & 3844 & 2368.7 & 0 & 4303 & \textbf{251.9} & 0 & 3844 & 2350.9 & 0 & 3844 & 2460.4 & 0\\
\texttt{breast-cancer} & \multicolumn{1}{r}{683} & \multicolumn{1}{r}{89}  & 0 & 1006.6 & 1 & 0 & \textbf{823.6} & 1 & 0 & 1023.5 & 1 & 0 & 1194.2 & 1\\
\texttt{breast-wisconsin} & \multicolumn{1}{r}{683} & \multicolumn{1}{r}{120}  & 0 & \textbf{0.0} & 1 & 0 & 0.2 & 1 & 0 & 0.3 & 1 & 0 & 0.0 & 1\\
\texttt{car} & \multicolumn{1}{r}{1728} & \multicolumn{1}{r}{21}  & 11 & \textbf{231.4} & 1 & 11 & 255.9 & 1 & 11 & 232.9 & 1 & 11 & 627.1 & 1\\
\texttt{compas\_discretized} & \multicolumn{1}{r}{6167} & \multicolumn{1}{r}{25}  & 1852 & 198.4 & 1 & 1852 & \textbf{183.8} & 1 & 1852 & 2029.8 & 0 & 1852 & 299.5 & 1\\
\texttt{diabetes} & \multicolumn{1}{r}{768} & \multicolumn{1}{r}{112}  & 21 & 827.2 & 0 & 27 & \textbf{237.7} & 0 & 26 & 3164.2 & 0 & 21 & 1324.2 & 0\\
\texttt{forest-fires} & \multicolumn{1}{r}{517} & \multicolumn{1}{r}{989}  & 146 & 125.1 & 0 & 142 & 140.3 & 0 & \textbf{132} & 1345.8 & 0 & 146 & \textbf{124.2} & 0\\
\texttt{german-credit} & \multicolumn{1}{r}{1000} & \multicolumn{1}{r}{112}  & 56 & \textbf{1191.7} & 0 & 117 & 2789.2 & 0 & 56 & 2472.1 & 0 & 56 & 1445.7 & 0\\
\texttt{heart-cleveland} & \multicolumn{1}{r}{296} & \multicolumn{1}{r}{95}  & 0 & 0.0 & 1 & 0 & 3.0 & 1 & 0 & 0.0 & 1 & 0 & 0.0 & 1\\
\texttt{hepatitis} & \multicolumn{1}{r}{137} & \multicolumn{1}{r}{68}  & 0 & 0.0 & 1 & 0 & 0.0 & 1 & 0 & 0.0 & 1 & 0 & 0.0 & 1\\
\texttt{hypothyroid} & \multicolumn{1}{r}{3247} & \multicolumn{1}{r}{88}  & \textbf{22} & 3478.5 & 0 & 23 & 146.7 & 0 & 27 & \textbf{113.0} & 0 & 23 & 171.1 & 0\\
\texttt{ionosphere} & \multicolumn{1}{r}{351} & \multicolumn{1}{r}{445}  & 0 & 0.1 & 1 & 0 & 0.1 & 1 & 0 & 0.5 & 1 & 0 & 0.1 & 1\\
\texttt{kr-vs-kp} & \multicolumn{1}{r}{3196} & \multicolumn{1}{r}{73}  & 18 & 2549.9 & 0 & 18 & \textbf{1422.7} & 0 & 34 & 3089.9 & 0 & 21 & 1756.1 & 0\\
\texttt{letter} & \multicolumn{1}{r}{20000} & \multicolumn{1}{r}{224}  & 68 & \textbf{177.0} & 0 & 168 & 2143.2 & 0 & 70 & 3525.2 & 0 & 68 & 193.5 & 0\\
\texttt{lymph} & \multicolumn{1}{r}{148} & \multicolumn{1}{r}{68}  & 0 & 0.0 & 1 & 0 & 0.0 & 1 & 0 & 0.0 & 1 & 0 & 0.0 & 1\\
\texttt{mnist\_0} & \multicolumn{1}{r}{60000} & \multicolumn{1}{r}{784}  & 1107 & 2894.7 & 0 & 1556 & \textbf{1538.7} & 0 & 1107 & 2983.1 & 0 & 1107 & 2735.1 & 0\\
\texttt{mnist\_1} & \multicolumn{1}{r}{60000} & \multicolumn{1}{r}{784}  & 810 & 510.3 & 0 & 2657 & 1574.1 & 0 & 810 & \textbf{496.2} & 0 & 810 & 542.5 & 0\\
\texttt{mnist\_2} & \multicolumn{1}{r}{60000} & \multicolumn{1}{r}{784}  & 2133 & 2575.3 & 0 & 3288 & 3470.6 & 0 & 2133 & 2383.1 & 0 & 2133 & \textbf{2285.2} & 0\\
\texttt{mnist\_3} & \multicolumn{1}{r}{60000} & \multicolumn{1}{r}{784}  & 1843 & 3188.4 & 0 & 4120 & \textbf{1269.1} & 0 & 1843 & 2881.8 & 0 & 1843 & 3188.7 & 0\\
\texttt{mnist\_4} & \multicolumn{1}{r}{60000} & \multicolumn{1}{r}{784}  & 1727 & 3233.7 & 0 & 3423 & \textbf{2207.5} & 0 & 1727 & 3075.7 & 0 & 1727 & 3401.3 & 0\\
\texttt{mnist\_5} & \multicolumn{1}{r}{60000} & \multicolumn{1}{r}{784}  & 2830 & 2556.2 & 0 & 3112 & 2791.0 & 0 & 2830 & 2472.3 & 0 & 2830 & \textbf{2299.2} & 0\\
\texttt{mnist\_6} & \multicolumn{1}{r}{60000} & \multicolumn{1}{r}{784}  & 1208 & 2901.6 & 0 & 1787 & 3465.3 & 0 & 1208 & 2665.1 & 0 & 1208 & \textbf{2244.0} & 0\\
\texttt{mnist\_7} & \multicolumn{1}{r}{60000} & \multicolumn{1}{r}{784}  & 1659 & 2853.0 & 0 & 2666 & 3428.0 & 0 & 1659 & \textbf{2756.6} & 0 & 1659 & 2897.4 & 0\\
\texttt{mnist\_8} & \multicolumn{1}{r}{60000} & \multicolumn{1}{r}{784}  & 1566 & 3229.8 & 0 & 2928 & \textbf{2701.3} & 0 & 1566 & 2799.9 & 0 & 1566 & 3089.5 & 0\\
\texttt{mnist\_9} & \multicolumn{1}{r}{60000} & \multicolumn{1}{r}{784}  & 2550 & 1862.6 & 0 & 4420 & \textbf{1577.3} & 0 & 2550 & 1904.8 & 0 & 2550 & 2015.6 & 0\\
\texttt{mushroom} & \multicolumn{1}{r}{8124} & \multicolumn{1}{r}{119}  & 0 & 0.0 & 1 & 0 & 0.0 & 1 & 0 & 0.0 & 1 & 0 & 0.0 & 1\\
\texttt{pendigits} & \multicolumn{1}{r}{7494} & \multicolumn{1}{r}{216}  & 0 & 0.0 & 1 & 0 & 3.5 & 1 & 0 & 0.0 & 1 & 0 & 0.0 & 1\\
\texttt{primary-tumor} & \multicolumn{1}{r}{336} & \multicolumn{1}{r}{31}  & 16 & 18.2 & 1 & 16 & \textbf{17.1} & 1 & 16 & 2865.7 & 1 & 16 & 39.0 & 1\\
\texttt{segment} & \multicolumn{1}{r}{2310} & \multicolumn{1}{r}{235}  & 0 & 0.0 & 1 & 0 & 0.0 & 1 & 0 & 0.0 & 1 & 0 & 0.0 & 1\\
\texttt{soybean} & \multicolumn{1}{r}{630} & \multicolumn{1}{r}{50}  & 2 & 19.3 & 1 & 2 & \textbf{6.1} & 1 & 2 & 729.5 & 0 & 2 & 32.4 & 1\\
\texttt{splice-1} & \multicolumn{1}{r}{3190} & \multicolumn{1}{r}{287}  & 29 & 3483.8 & 0 & 46 & \textbf{3380.3} & 0 & 29 & 3574.6 & 0 & 29 & 3407.6 & 0\\
\texttt{surgical-deepnet} & \multicolumn{1}{r}{14635} & \multicolumn{1}{r}{6047}  & 1647 & 1247.9 & 0 & 2246 & 3101.8 & 0 & 1647 & \textbf{1085.7} & 0 & 1647 & 1288.2 & 0\\
\texttt{taiwan\_binarised} & \multicolumn{1}{r}{30000} & \multicolumn{1}{r}{205}  & 4896 & 1957.6 & 0 & 5016 & 2961.0 & 0 & 4909 & \textbf{1426.5} & 0 & 4896 & 2054.9 & 0\\
\texttt{tic-tac-toe} & \multicolumn{1}{r}{958} & \multicolumn{1}{r}{27}  & 0 & 32.1 & 1 & 0 & 83.4 & 1 & 0 & \textbf{30.8} & 1 & 0 & 100.0 & 1\\
\texttt{titanic} & \multicolumn{1}{r}{887} & \multicolumn{1}{r}{333}  & 72 & \textbf{442.3} & 0 & 78 & 2696.1 & 0 & 72 & 471.3 & 0 & 72 & 500.2 & 0\\
\texttt{vehicle} & \multicolumn{1}{r}{846} & \multicolumn{1}{r}{252}  & 0 & \textbf{0.1} & 1 & 0 & 195.9 & 1 & 0 & 0.7 & 1 & 0 & 0.1 & 1\\
\texttt{vote} & \multicolumn{1}{r}{435} & \multicolumn{1}{r}{48}  & 0 & 0.0 & 1 & 0 & 0.0 & 1 & 0 & 0.0 & 1 & 0 & 0.0 & 1\\
\texttt{weather-aus} & \multicolumn{1}{r}{142193} & \multicolumn{1}{r}{4759}  & 1685 & 2047.6 & 0 & 1747 & \textbf{1685.2} & 0 & 1685 & 1948.5 & 0 & 1685 & 2082.9 & 0\\
\texttt{wine1} & \multicolumn{1}{r}{178} & \multicolumn{1}{r}{1276}  & 28 & 891.7 & 0 & 28 & 2666.5 & 0 & 29 & \textbf{487.1} & 0 & 28 & 892.0 & 0\\
\texttt{wine2} & \multicolumn{1}{r}{178} & \multicolumn{1}{r}{1276}  & 31 & 28.2 & 0 & 31 & \textbf{23.1} & 0 & 31 & 167.5 & 0 & 31 & 28.1 & 0\\
\texttt{wine3} & \multicolumn{1}{r}{178} & \multicolumn{1}{r}{1276}  & 21 & 524.2 & 0 & 21 & 1062.3 & 0 & \textbf{20} & \textbf{296.4} & 0 & 21 & 531.0 & 0\\
\texttt{yeast} & \multicolumn{1}{r}{1484} & \multicolumn{1}{r}{89}  & \textbf{182} & 3557.8 & 0 & 234 & 1611.4 & 0 & 210 & 1190.9 & 0 & 203 & \textbf{410.5} & 0\\
\bottomrule
\end{tabular}

\end{footnotesize}
\end{center}
\caption{\label{tab:fact7} Factor analysis: depth 7}
\end{table}

\begin{table}[htbp]
\begin{center}
\begin{footnotesize}
\tabcolsep=2pt
\begin{tabular}{lccrrrrrrrr}
\toprule
\multirow{2}{*}{}& && \multicolumn{2}{c}{\budalg} & \multicolumn{2}{c}{\noheuristic} & \multicolumn{2}{c}{\nopreprocessing} & \multicolumn{2}{c}{\nolb}\\
\cmidrule(rr){4-5}\cmidrule(rr){6-7}\cmidrule(rr){8-9}\cmidrule(rr){10-11}
&\multirow{1}{*}{$|\allex|$} & \multirow{1}{*}{$|\features|$} &  \multicolumn{1}{c}{error} & \multicolumn{1}{c}{cpu} & \multicolumn{1}{c}{error} & \multicolumn{1}{c}{cpu} & \multicolumn{1}{c}{error} & \multicolumn{1}{c}{cpu} & \multicolumn{1}{c}{error} & \multicolumn{1}{c}{cpu} \\
\midrule

\texttt{hepatitis} & 0 & 0.00$^*$ & 0 & 0.00$^*$ & 0 & 0.00$^*$ & 0 & 0.00$^*$\\
\texttt{lymph} & 0 & 0.00$^*$ & 0 & 0.00$^*$ & 0 & 0.00$^*$ & 0 & 0.00$^*$\\
\texttt{wine1} & 27 & 43 & \textbf{25} & 1520 & 27 & 189 & 27 & 44\\
\texttt{wine2} & 30 & 605 & \textbf{28} & 0.04 & 31 & 592 & 30 & 603\\
\texttt{wine3} & 20 & 406 & 19 & 951 & 19 & 2406 & 20 & 408\\
\texttt{audiology} & 0 & 0.00$^*$ & 0 & 0.00$^*$ & 0 & 0.00$^*$ & 0 & 0.00$^*$\\
\texttt{heart-cleveland} & 0 & 0.00$^*$ & 0 & 1.7$^*$ & 0 & 0.00$^*$ & 0 & 0.00$^*$\\
\texttt{primary-tumor} & 15 & 0.00$^*$ & 15 & 0.55$^*$ & 15 & 25 & 15 & 0.01$^*$\\
\texttt{ionosphere} & 0 & 0.00$^*$ & 0 & 0.08$^*$ & 0 & 0.02$^*$ & 0 & 0.00$^*$\\
\texttt{vote} & 0 & 0.00$^*$ & 0 & 0.03$^*$ & 0 & 0.00$^*$ & 0 & 0.00$^*$\\
\texttt{forest-fires} & 137 & 190 & 133 & 108 & 133 & 326 & 137 & 195\\
\texttt{soybean} & 2 & 0.13$^*$ & 2 & 0.20$^*$ & 2 & 1.8 & 2 & 0.18$^*$\\
\texttt{australian-credit} & 0 & 13$^*$ & 0 & 47$^*$ & 0 & 144$^*$ & 0 & 17$^*$\\
\texttt{breast-cancer} & 0 & 25$^*$ & 0 & 22$^*$ & 0 & 27$^*$ & 0 & 27$^*$\\
\texttt{breast-wisconsin} & 0 & 0.00$^*$ & 0 & 0.00$^*$ & 0 & 0.00$^*$ & 0 & 0.00$^*$\\
\texttt{diabetes} & 0 & 220$^*$ & 16 & 2998 & 4 & 2216 & 0 & 443$^*$\\
\texttt{anneal} & \textbf{36} & 2221 & 47 & 1669 & 41 & 560 & 40 & 296\\
\texttt{vehicle} & 0 & 0.00$^*$ & 0 & 151$^*$ & 0 & 0.04$^*$ & 0 & 0.00$^*$\\
\texttt{titanic} & 64 & 1500 & 77 & 1409 & 75 & 18 & 64 & 1756\\
\texttt{tic-tac-toe} & 0 & 0.00$^*$ & 0 & 0.00$^*$ & 0 & 0.00$^*$ & 0 & 0.00$^*$\\
\texttt{german-credit} & 23 & 2235 & 107 & 3373 & 34 & 1315 & 23 & 2836\\
\texttt{yeast} & 132 & 1682 & 159 & 3450 & 166 & 969 & 132 & 2194\\
\texttt{car} & 0 & 404$^*$ & 0 & 536$^*$ & 0 & 407$^*$ & 0 & 1806$^*$\\
\texttt{segment} & 0 & 0.00$^*$ & 0 & 0.00$^*$ & 0 & 0.00$^*$ & 0 & 0.00$^*$\\
\texttt{splice-1} & 24 & 0.47 & 38 & 1642 & 24 & 0.47 & 24 & 0.50\\
\texttt{kr-vs-kp} & 13 & 2736 & \textbf{8} & 456 & 16 & 1725 & 14 & 1384\\
\texttt{hypothyroid} & 17 & 99$^*$ & 17 & 300$^*$ & 23 & $\mathsmaller{\geq}1$h & 17 & 181$^*$\\
\texttt{compas\_discretized} & 1832 & 1462$^*$ & 1832 & 1255$^*$ & 1835 & 2508 & 1832 & 3290$^*$\\
\texttt{pendigits} & 0 & 0.00$^*$ & 0 & 0.59$^*$ & 0 & 0.00$^*$ & 0 & 0.00$^*$\\
\texttt{mushroom} & 0 & 0.00$^*$ & 0 & 0.00$^*$ & 0 & 0.00$^*$ & 0 & 0.00$^*$\\
\texttt{surgical-deepnet} & 1297 & 66 & 1979 & 2061 & 1297 & 62 & 1297 & 68\\
\texttt{letter} & 24 & 297 & 158 & 3358 & 25 & 962 & 24 & 362\\
\texttt{taiwan\_binarised} & 4727 & 3246 & 4990 & 2579 & 4727 & 3521 & 4727 & 3349\\
\texttt{adult\_discretized} & 4148 & 450 & 4219 & 475 & \textbf{4058} & 2826 & 4148 & 477\\
\texttt{bank} & 3709 & 1815 & 4246 & 3454 & 3709 & 1704 & 3709 & 1814\\
\texttt{mnist\_8} & 1192 & 3087 & 2764 & 886 & \textbf{1183} & 3445 & 1192 & 3142\\
\texttt{mnist\_9} & 2186 & 2934 & 4681 & 1054 & 2187 & 2909 & 2186 & 3034\\
\texttt{mnist\_0} & 788 & 1983 & 1913 & 1368 & 788 & 1795 & 788 & 2158\\
\texttt{mnist\_6} & 1203 & 109 & 1563 & 1368 & 1203 & 104 & 1203 & 109\\
\texttt{mnist\_5} & 2519 & 1518 & 3082 & 2519 & 2519 & 1007 & 2519 & 1455\\
\texttt{mnist\_3} & 1436 & 1680 & 4112 & 829 & 1436 & 1675 & 1436 & 1822\\
\texttt{mnist\_2} & 1857 & 291 & 3131 & 1828 & 1856 & 3536 & 1856 & 3582\\
\texttt{mnist\_4} & 1279 & 1522 & 3251 & 1375 & 1279 & 1222 & 1279 & 1324\\
\texttt{mnist\_7} & 1429 & 347 & 2520 & 2431 & 1429 & 343 & 1429 & 365\\
\texttt{mnist\_1} & 565 & 1177 & 2282 & 1805 & 565 & 1189 & 565 & 1327\\
\texttt{weather-aus} & 1657 & 2060 & 1746 & 1698 & \textbf{1656} & 2025 & 1657 & 2162\\
\bottomrule
\end{tabular}

\end{footnotesize}
\end{center}
\caption{\label{tab:fact8} Factor analysis: depth 8}
\end{table}

\begin{table}[htbp]
\begin{center}
\begin{footnotesize}
\tabcolsep=2pt
\begin{tabular}{lccrrrrrrrr}
\toprule
\multirow{2}{*}{}& && \multicolumn{2}{c}{\budalg} & \multicolumn{2}{c}{\noheuristic} & \multicolumn{2}{c}{\nopreprocessing} & \multicolumn{2}{c}{\nolb}\\
\cmidrule(rr){4-5}\cmidrule(rr){6-7}\cmidrule(rr){8-9}\cmidrule(rr){10-11}
&\multirow{1}{*}{$|\allex|$} & \multirow{1}{*}{$|\features|$} &  \multicolumn{1}{c}{error} & \multicolumn{1}{c}{cpu} & \multicolumn{1}{c}{error} & \multicolumn{1}{c}{cpu} & \multicolumn{1}{c}{error} & \multicolumn{1}{c}{cpu} & \multicolumn{1}{c}{error} & \multicolumn{1}{c}{cpu} \\
\midrule

\texttt{hepatitis} & 0 & 0.00$^*$ & 0 & 0.00$^*$ & 0 & 0.00$^*$ & 0 & 0.00$^*$\\
\texttt{lymph} & 0 & 0.00$^*$ & 0 & 0.00$^*$ & 0 & 0.00$^*$ & 0 & 0.00$^*$\\
\texttt{wine3} & 18 & 317 & \textbf{17} & 419 & 18 & 2032 & 18 & 320\\
\texttt{wine2} & 27 & 505 & \textbf{24} & 28 & 27 & 3367 & 27 & 514\\
\texttt{wine1} & 24 & 1541 & \textbf{22} & 1440 & 25 & 85 & 24 & 1536\\
\texttt{audiology} & 0 & 0.00$^*$ & 0 & 0.00$^*$ & 0 & 0.00$^*$ & 0 & 0.00$^*$\\
\texttt{heart-cleveland} & 0 & 0.00$^*$ & 0 & 0.51$^*$ & 0 & 0.00$^*$ & 0 & 0.00$^*$\\
\texttt{primary-tumor} & 15 & 0.00$^*$ & 15 & 0.00$^*$ & 15 & 0.28 & 15 & 0.00$^*$\\
\texttt{ionosphere} & 0 & 0.00$^*$ & 0 & 0.02$^*$ & 0 & 0.00$^*$ & 0 & 0.00$^*$\\
\texttt{vote} & 0 & 0.00$^*$ & 0 & 0.00$^*$ & 0 & 0.00$^*$ & 0 & 0.00$^*$\\
\texttt{forest-fires} & 133 & 7.7 & 137 & 87 & \textbf{132} & 18 & 133 & 7.9\\
\texttt{soybean} & 2 & 0.04$^*$ & 2 & 0.36$^*$ & 2 & 0.16 & 2 & 0.06$^*$\\
\texttt{australian-credit} & 0 & 1.8$^*$ & 0 & 8.5$^*$ & 0 & 11$^*$ & 0 & 2.3$^*$\\
\texttt{breast-wisconsin} & 0 & 0.00$^*$ & 0 & 0.00$^*$ & 0 & 0.00$^*$ & 0 & 0.00$^*$\\
\texttt{breast-cancer} & 0 & 10$^*$ & 0 & 11$^*$ & 0 & 9.9$^*$ & 0 & 9.9$^*$\\
\texttt{diabetes} & 0 & 2.6$^*$ & 9 & 2205 & 0 & 61$^*$ & 0 & 4.0$^*$\\
\texttt{anneal} & 35 & 340 & 36 & 1433 & 39 & 141 & 35 & 533\\
\texttt{vehicle} & 0 & 0.00$^*$ & 0 & 70$^*$ & 0 & 0.00$^*$ & 0 & 0.00$^*$\\
\texttt{titanic} & 47 & 2117 & 66 & 2898 & 50 & 1176 & 47 & 2568\\
\texttt{tic-tac-toe} & 0 & 0.00$^*$ & 0 & 0.00$^*$ & 0 & 0.00$^*$ & 0 & 0.00$^*$\\
\texttt{german-credit} & \textbf{4} & 3513 & 74 & 3281 & 15 & 3438 & 16 & 980\\
\texttt{yeast} & \textbf{68} & 3461 & 72 & 2722 & 119 & 810 & 75 & 2826\\
\texttt{car} & 0 & 77$^*$ & 0 & 88$^*$ & 0 & 78$^*$ & 0 & 305$^*$\\
\texttt{segment} & 0 & 0.00$^*$ & 0 & 0.00$^*$ & 0 & 0.00$^*$ & 0 & 0.00$^*$\\
\texttt{splice-1} & 12 & 212 & 20 & 1696 & 12 & 213 & 12 & 215\\
\texttt{kr-vs-kp} & 5 & 376 & \textbf{1} & 1510 & 8 & 307 & 5 & 711\\
\texttt{hypothyroid} & 17 & 76$^*$ & 17 & 103$^*$ & 18 & 2583 & 17 & 139$^*$\\
\texttt{compas\_discretized} & 1828 & 205$^*$ & 1828 & 310$^*$ & 1829 & 703 & 1828 & 664$^*$\\
\texttt{pendigits} & 0 & 0.00$^*$ & 0 & 0.00$^*$ & 0 & 0.00$^*$ & 0 & 0.00$^*$\\
\texttt{mushroom} & 0 & 0.00$^*$ & 0 & 0.00$^*$ & 0 & 0.00$^*$ & 0 & 0.00$^*$\\
\texttt{surgical-deepnet} & 1127 & 53 & 1956 & 2314 & 1127 & 52 & 1127 & 53\\
\texttt{letter} & 2 & 366 & 139 & 2398 & 10 & 810 & 2 & 495\\
\texttt{taiwan\_binarised} & 4427 & 3484 & 4941 & 1856 & \textbf{4423} & 3557 & 4432 & 3185\\
\texttt{adult\_discretized} & 3999 & 669 & 4108 & 1911 & \textbf{3969} & 109 & 3999 & 704\\
\texttt{bank} & 3493 & 944 & 4194 & 3396 & 3493 & 923 & 3493 & 949\\
\texttt{mnist\_0} & 595 & 1511 & 1798 & 786 & 595 & 1305 & 595 & 1744\\
\texttt{mnist\_3} & 1213 & 543 & 4105 & 2454 & 1374 & 3471 & 1213 & 543\\
\texttt{mnist\_1} & 465 & 500 & 1735 & 1441 & 466 & 429 & 465 & 473\\
\texttt{mnist\_7} & 1261 & 1792 & 2331 & 2675 & 1261 & 1761 & 1261 & 1880\\
\texttt{mnist\_6} & 1138 & 1606 & 1444 & 3145 & 1138 & 1483 & 1138 & 1791\\
\texttt{mnist\_8} & 849 & 1304 & 2602 & 941 & 849 & 1268 & 849 & 1347\\
\texttt{mnist\_9} & 1829 & 2522 & 4291 & 1572 & 1829 & 2810 & 1829 & 2979\\
\texttt{mnist\_4} & 908 & 3043 & 4478 & 1696 & 908 & 3029 & 908 & 3482\\
\texttt{mnist\_2} & 1682 & 2193 & 3028 & 1636 & 1682 & 1728 & 1682 & 2076\\
\texttt{mnist\_5} & 2424 & 612 & 3419 & 697 & 2424 & 645 & 2424 & 899\\
\texttt{weather-aus} & 1638 & 2359 & 1739 & 1599 & \textbf{1637} & 2730 & 1638 & 2264\\
\bottomrule
\end{tabular}

\end{footnotesize}
\end{center}
\caption{\label{tab:fact9} Factor analysis: depth 9}
\end{table}

\begin{table}[htbp]
\begin{center}
\begin{footnotesize}
\tabcolsep=2pt
\begin{tabular}{lccrrrrrrrr}
\toprule
\multirow{2}{*}{}& && \multicolumn{2}{c}{\budalg} & \multicolumn{2}{c}{\noheuristic} & \multicolumn{2}{c}{\nopreprocessing} & \multicolumn{2}{c}{\nolb}\\
\cmidrule(rr){4-5}\cmidrule(rr){6-7}\cmidrule(rr){8-9}\cmidrule(rr){10-11}
&\multirow{1}{*}{$|\allex|$} & \multirow{1}{*}{$|\features|$} &  \multicolumn{1}{c}{error} & \multicolumn{1}{c}{cpu} & \multicolumn{1}{c}{error} & \multicolumn{1}{c}{cpu} & \multicolumn{1}{c}{error} & \multicolumn{1}{c}{cpu} & \multicolumn{1}{c}{error} & \multicolumn{1}{c}{cpu} \\
\midrule

\texttt{hepatitis} & 0 & 0.00$^*$ & 0 & 0.00$^*$ & 0 & 0.00$^*$ & 0 & 0.00$^*$\\
\texttt{lymph} & 0 & 0.00$^*$ & 0 & 0.00$^*$ & 0 & 0.00$^*$ & 0 & 0.00$^*$\\
\texttt{wine1} & 22 & 545 & \textbf{20} & 1469 & 22 & 3227 & 22 & 539\\
\texttt{wine2} & 24 & 399 & \textbf{21} & 20 & 24 & 2832 & 24 & 415\\
\texttt{wine3} & 16 & 272 & 17 & 690 & 18 & 1802 & 16 & 270\\
\texttt{audiology} & 0 & 0.00$^*$ & 0 & 0.00$^*$ & 0 & 0.00$^*$ & 0 & 0.00$^*$\\
\texttt{heart-cleveland} & 0 & 0.00$^*$ & 0 & 0.00$^*$ & 0 & 0.00$^*$ & 0 & 0.00$^*$\\
\texttt{primary-tumor} & 15 & 0.00$^*$ & 15 & 0.00$^*$ & 15 & 0.28 & 15 & 0.00$^*$\\
\texttt{ionosphere} & 0 & 0.00$^*$ & 0 & 0.02$^*$ & 0 & 0.00$^*$ & 0 & 0.00$^*$\\
\texttt{vote} & 0 & 0.00$^*$ & 0 & 0.00$^*$ & 0 & 0.00$^*$ & 0 & 0.00$^*$\\
\texttt{forest-fires} & 113 & 942 & 114 & 3068 & 118 & 3167 & 113 & 1003\\
\texttt{soybean} & 2 & 0.00$^*$ & 2 & 0.43$^*$ & 2 & 0.00 & 2 & 0.00$^*$\\
\texttt{australian-credit} & 0 & 0.04$^*$ & 0 & 0.15$^*$ & 0 & 0.26$^*$ & 0 & 0.04$^*$\\
\texttt{breast-cancer} & 0 & 0.00$^*$ & 0 & 0.31$^*$ & 0 & 0.00$^*$ & 0 & 0.00$^*$\\
\texttt{breast-wisconsin} & 0 & 0.00$^*$ & 0 & 0.00$^*$ & 0 & 0.00$^*$ & 0 & 0.00$^*$\\
\texttt{diabetes} & 0 & 0.67$^*$ & 0 & 3026$^*$ & 0 & 11$^*$ & 0 & 0.60$^*$\\
\texttt{anneal} & 34 & 23$^*$ & 36 & 1986 & 36 & 661 & 34 & 32$^*$\\
\texttt{vehicle} & 0 & 0.00$^*$ & 0 & 60$^*$ & 0 & 0.00$^*$ & 0 & 0.00$^*$\\
\texttt{titanic} & \textbf{35} & 3059 & 52 & 943 & 45 & 1077 & 42 & 180\\
\texttt{tic-tac-toe} & 0 & 0.00$^*$ & 0 & 0.00$^*$ & 0 & 0.00$^*$ & 0 & 0.00$^*$\\
\texttt{german-credit} & 0 & 69$^*$ & 62 & 2594 & 0 & 173$^*$ & 0 & 96$^*$\\
\texttt{yeast} & 28 & 1008 & 68 & 2610 & 67 & 466 & 28 & 1633\\
\texttt{car} & 0 & 0.26$^*$ & 0 & 21$^*$ & 0 & 0.32$^*$ & 0 & 0.44$^*$\\
\texttt{segment} & 0 & 0.00$^*$ & 0 & 0.00$^*$ & 0 & 0.00$^*$ & 0 & 0.00$^*$\\
\texttt{splice-1} & 5 & 1160 & 12 & 1676 & \textbf{4} & 3506 & 5 & 1205\\
\texttt{kr-vs-kp} & 0 & 1897$^*$ & 0 & 752$^*$ & 5 & 86 & 1 & 400\\
\texttt{hypothyroid} & 17 & 0.96$^*$ & 17 & 40$^*$ & 17 & 72 & 17 & 1.5$^*$\\
\texttt{compas\_discretized} & 1828 & 0.73$^*$ & 1828 & 9.1$^*$ & 1828 & 323 & 1828 & 1.4$^*$\\
\texttt{pendigits} & 0 & 0.00$^*$ & 0 & 0.00$^*$ & 0 & 0.00$^*$ & 0 & 0.00$^*$\\
\texttt{mushroom} & 0 & 0.00$^*$ & 0 & 0.00$^*$ & 0 & 0.00$^*$ & 0 & 0.00$^*$\\
\texttt{surgical-deepnet} & 965 & 2865 & 1849 & 3204 & 965 & 3133 & 965 & 3192\\
\texttt{letter} & 0 & 79$^*$ & 88 & 1825 & 0 & 1535$^*$ & 0 & 104$^*$\\
\texttt{taiwan\_binarised} & 4217 & 1001 & 4896 & 2890 & \textbf{4189} & 1046 & 4217 & 1041\\
\texttt{adult\_discretized} & 3841 & 2632 & 4119 & 3075 & \textbf{3775} & 2994 & 3841 & 2988\\
\texttt{bank} & 3242 & 800 & 4200 & 20 & 3245 & 851 & 3242 & 845\\
\texttt{mnist\_6} & 965 & 1742 & 1370 & 2484 & 965 & 2228 & 965 & 2418\\
\texttt{mnist\_2} & 1522 & 2520 & 2968 & 1804 & 1522 & 2466 & 1522 & 3022\\
\texttt{mnist\_3} & 1079 & 376 & 4098 & 2365 & \textbf{1065} & 3406 & 1079 & 406\\
\texttt{mnist\_8} & 696 & 2141 & 2450 & 1144 & 801 & 1936 & 696 & 2466\\
\texttt{mnist\_5} & 1973 & 491 & 2866 & 3420 & 1974 & 509 & 1973 & 950\\
\texttt{mnist\_7} & 1082 & 1277 & 2228 & 2846 & 1082 & 1544 & 1082 & 1798\\
\texttt{mnist\_4} & 801 & 398 & 4289 & 1461 & 801 & 424 & 801 & 447\\
\texttt{mnist\_9} & 1594 & 2124 & 3598 & 2205 & 1596 & 2234 & 1594 & 2299\\
\texttt{mnist\_0} & 383 & 413 & 1721 & 3235 & 383 & 404 & 383 & 450\\
\texttt{mnist\_1} & 331 & 360 & 1493 & 1783 & \textbf{330} & 355 & 331 & 382\\
\texttt{weather-aus} & 1601 & 2591 & 1734 & 2391 & 1603 & 1988 & 1601 & 2758\\
\bottomrule
\end{tabular}

\end{footnotesize}
\end{center}
\caption{\label{tab:fact10} Factor analysis: depth 10}
\end{table}


\end{document}

\begin{table}[htbp]
\begin{center}
\begin{normalsize}
\tabcolsep=4pt
\begin{tabular}{lrrrrrrrr}
\toprule
&  \multicolumn{2}{c}{\budalg} & \multicolumn{2}{c}{\murtree} & \multicolumn{4}{c}{\dleight}\\
\cmidrule(rr){2-3}\cmidrule(rr){4-5}\cmidrule(rr){6-9}
& \multicolumn{1}{c}{acc.} & \multicolumn{1}{c}{opt.} & \multicolumn{1}{c}{acc.} & \multicolumn{1}{c}{opt.} & \multicolumn{1}{c}{acc.} & \multicolumn{1}{c}{acc. (r)} & \multicolumn{1}{c}{cpu (r)} & \multicolumn{1}{c}{opt.} \\
\midrule

\texttt{adult\_discretized} & \textbf{0.8669} & 0.00 & 0.8289 & \textbf{1.00} & 0.7954 & -0.0825 & - & 0.00\\
\texttt{anneal} & \textbf{0.9569} & 0.00 & 0.8855 & \textbf{1.00} & - & - & - & 0.00\\
\texttt{audiology} & 1.0000 & 1.00 & 1.0000 & 1.00 & 1.0000 & 0.0000 & +1.01 & 1.00\\
\texttt{australian-credit} & 1.0000 & 1.00 & 1.0000 & 1.00 & - & - & - & 0.00\\
\texttt{bank} & \textbf{0.9230} & 0.00 & 0.8833 & 0.00 & 0.8935 & -0.0320 & - & 0.00\\
\texttt{breast-cancer} & 1.0000 & 1.00 & 1.0000 & 1.00 & 1.0000 & 0.0000 & -0.98 & 1.00\\
\texttt{breast-wisconsin} & 1.0000 & 1.00 & 1.0000 & 1.00 & 1.0000 & 0.0000 & +424.02 & 1.00\\
\texttt{car} & 1.0000 & 1.00 & 1.0000 & 1.00 & 1.0000 & 0.0000 & -0.98 & 1.00\\
\texttt{compas\_discretized} & 0.7036 & 1.00 & \textbf{0.7036} & 1.00 & - & - & - & 0.00\\
\texttt{diabetes} & \textbf{1.0000} & 1.00 & 0.8737 & 1.00 & - & - & - & 0.00\\
\texttt{forest-fires} & 0.7099 & 0.00 & \textbf{0.7563} & 0.00 & - & - & - & 0.00\\
\texttt{german-credit} & 0.9850 & 0.00 & \textbf{1.0000} & \textbf{1.00} & - & - & - & 0.00\\
\texttt{heart-cleveland} & 1.0000 & 1.00 & 1.0000 & 1.00 & 1.0000 & 0.0000 & +1304.02 & 1.00\\
\texttt{hepatitis} & 1.0000 & 1.00 & 1.0000 & 1.00 & 1.0000 & 0.0000 & +1.03 & 1.00\\
\texttt{hypothyroid} & \textbf{0.9948} & 1.00 & 0.9769 & 1.00 & - & - & - & 0.00\\
\texttt{ionosphere} & 1.0000 & 1.00 & 1.0000 & 1.00 & 1.0000 & 0.0000 & +3171.02 & 1.00\\
\texttt{kr-vs-kp} & \textbf{0.9984} & 0.00 & 0.9671 & \textbf{1.00} & - & - & - & 0.00\\
\texttt{letter} & \textbf{0.9988} & 0.00 & 0.9730 & 0.00 & 0.9651 & -0.0337 & - & 0.00\\
\texttt{lymph} & 1.0000 & 1.00 & 1.0000 & 1.00 & 1.0000 & 0.0000 & +1.01 & 1.00\\
\texttt{mnist\_0} & \textbf{0.9904} & 0.00 & 0.9574 & 0.00 & - & - & - & 0.00\\
\texttt{mnist\_1} & \textbf{0.9908} & 0.00 & 0.9425 & \textbf{1.00} & 0.9242 & -0.0672 & - & 0.00\\
\texttt{mnist\_2} & \textbf{0.9671} & 0.00 & 0.9346 & 0.00 & - & - & - & 0.00\\
\texttt{mnist\_3} & \textbf{0.9780} & 0.00 & 0.9275 & 0.00 & - & - & - & 0.00\\
\texttt{mnist\_4} & \textbf{0.9799} & 0.00 & 0.9214 & \textbf{1.00} & 0.9070 & -0.0744 & - & 0.00\\
\texttt{mnist\_5} & \textbf{0.9591} & 0.00 & 0.9410 & 0.00 & 0.9270 & -0.0335 & - & 0.00\\
\texttt{mnist\_6} & \textbf{0.9802} & 0.00 & 0.9550 & 0.00 & 0.9547 & -0.0260 & - & 0.00\\
\texttt{mnist\_7} & \textbf{0.9784} & 0.00 & 0.9419 & 0.00 & 0.9242 & -0.0554 & - & 0.00\\
\texttt{mnist\_8} & \textbf{0.9806} & 0.00 & 0.9403 & \textbf{1.00} & - & - & - & 0.00\\
\texttt{mnist\_9} & \textbf{0.9672} & 0.00 & 0.9235 & 0.00 & - & - & - & 0.00\\
\texttt{mushroom} & 1.0000 & 1.00 & 1.0000 & 1.00 & 1.0000 & 0.0000 & +16.77 & 1.00\\
\texttt{pendigits} & 1.0000 & 1.00 & 1.0000 & 1.00 & - & - & - & 0.00\\
\texttt{primary-tumor} & \textbf{0.9554} & 1.00 & 0.9554 & 1.00 & - & - & - & 0.00\\
\texttt{segment} & 1.0000 & 1.00 & 1.0000 & 1.00 & 1.0000 & 0.0000 & +2.58 & 1.00\\
\texttt{soybean} & \textbf{0.9968} & 1.00 & 0.9968 & 1.00 & - & - & - & 0.00\\
\texttt{splice-1} & \textbf{0.9962} & 0.00 & 0.6821 & 0.00 & - & - & - & 0.00\\
\texttt{surgical-deepnet} & \textbf{0.9215} & 0.00 & 0.8194 & 0.00 & - & - & - & 0.00\\
\texttt{taiwan\_binarised} & \textbf{0.8444} & 0.00 & 0.8032 & 0.00 & - & - & - & 0.00\\
\texttt{tic-tac-toe} & 1.0000 & 1.00 & 1.0000 & 1.00 & 1.0000 & 0.0000 & +3.34 & 1.00\\
\texttt{titanic} & \textbf{0.9245} & 0.00 & 0.9019 & \textbf{1.00} & - & - & - & 0.00\\
\texttt{vehicle} & 1.0000 & 1.00 & 1.0000 & 1.00 & 1.0000 & 0.0000 & +69.76 & 1.00\\
\texttt{vote} & 1.0000 & 1.00 & 1.0000 & 1.00 & 1.0000 & 0.0000 & +1.02 & 1.00\\
\texttt{weather-aus} & \textbf{0.9883} & 0.00 & 0.9625 & 0.00 & - & - & - & 0.00\\
\texttt{wine1} & \textbf{0.8652} & 0.00 & 0.8146 & \textbf{1.00} & - & - & - & 0.00\\
\texttt{wine2} & \textbf{0.8483} & 0.00 & 0.7584 & \textbf{1.00} & - & - & - & 0.00\\
\texttt{wine3} & \textbf{0.8989} & 0.00 & 0.8483 & \textbf{1.00} & - & - & - & 0.00\\
\texttt{yeast} & \textbf{0.9589} & 0.00 & 0.8059 & \textbf{1.00} & - & - & - & 0.00\\
\bottomrule
\end{tabular}

\end{normalsize}
\end{center}
\caption{\label{tab:all} Comparison with state of the art: depth 9}
\end{table}


\begin{table}[htbp]
\begin{center}
\begin{normalsize}
\tabcolsep=4pt
\begin{tabular}{lrrrrrrrrrrrr}
\toprule
&  \multicolumn{5}{c}{\budalg} & \multicolumn{5}{c}{\murtree} & \multicolumn{2}{c}{\cart}\\
\cmidrule(rr){2-6}\cmidrule(rr){7-11}\cmidrule(rr){12-13}
& \multicolumn{1}{c}{cpu} & \multicolumn{1}{c}{first} & \multicolumn{1}{c}{$\leq$3s} & \multicolumn{1}{c}{$\leq$10s} & \multicolumn{1}{c}{$\leq$1m} & \multicolumn{1}{c}{cpu} & \multicolumn{1}{c}{first} & \multicolumn{1}{c}{$\leq$3s} & \multicolumn{1}{c}{$\leq$10s} & \multicolumn{1}{c}{$\leq$1m} & \multicolumn{1}{c}{cpu} & \multicolumn{1}{c}{first} \\
\midrule

\texttt{adult\_discretized} & 0.00 & \textbf{5554} & 5020 & 5020 & 5020 & 0.00 & 7511 & 5020 & 5020 & 5020 & 0.05 & 5758\\
\texttt{anneal} & 0.00 & \textbf{137} & 112 & 112 & 112 & 0.00 & 187 & 112 & 112 & 112 & 0.00 & 149\\
\texttt{audiology} & 0.00 & 6 & 5 & 5 & 5 & 0.00 & 57 & 5 & 5 & 5 & 0.00 & 6\\
\texttt{australian-credit} & 0.00 & \textbf{82} & 73 & 73 & 73 & 0.00 & 296 & 73 & 73 & 73 & 0.00 & 87\\
\texttt{bank} & 0.82 & \textbf{4456} & \textbf{4453} & \textbf{4453} & \textbf{4453} & \textbf{0.00} & 5289 & 5287 & 5287 & 5287 & 32.54 & 4462\\
\texttt{breast-cancer} & 0.00 & 28 & 24 & 24 & 24 & 0.00 & 239 & 24 & 24 & 24 & 0.00 & 28\\
\texttt{breast-wisconsin} & 0.00 & 26 & 15 & 15 & 15 & 0.00 & 239 & 15 & 15 & 15 & 0.00 & 26\\
\texttt{car} & 0.00 & 202 & 192 & 192 & 192 & 0.00 & 518 & 192 & 192 & 192 & 0.00 & 202\\
\texttt{compas\_discretized} & 0.00 & \textbf{2067} & 2004 & 2004 & 2004 & 0.00 & 2809 & 2004 & 2004 & 2004 & 0.01 & 2072\\
\texttt{diabetes} & 0.00 & \textbf{169} & 162 & 162 & 162 & 0.00 & 268 & 162 & 162 & 162 & 0.00 & 177\\
\texttt{forest-fires} & 0.00 & 198 & 194 & 194 & 193 & 0.00 & 247 & \textbf{193} & \textbf{193} & 193 & 0.01 & 198\\
\texttt{german-credit} & 0.00 & \textbf{249} & 236 & 236 & 236 & 0.00 & 300 & 236 & 236 & 236 & 0.00 & 251\\
\texttt{heart-cleveland} & 0.00 & 43 & 41 & 41 & 41 & 0.00 & 136 & 41 & 41 & 41 & 0.00 & 43\\
\texttt{hepatitis} & 0.00 & \textbf{14} & 10 & 10 & 10 & 0.00 & 26 & 10 & 10 & 10 & 0.00 & 16\\
\texttt{hypothyroid} & 0.00 & 62 & 61 & 61 & 61 & 0.00 & 277 & 61 & 61 & 61 & 0.01 & 62\\
\texttt{ionosphere} & 0.00 & 29 & 22 & 22 & 22 & 0.00 & 126 & 22 & 22 & 22 & 0.01 & 29\\
\texttt{kr-vs-kp} & 0.00 & 306 & 198 & 198 & 198 & 0.00 & 1527 & 198 & 198 & 198 & 0.01 & 306\\
\texttt{letter} & 0.00 & \textbf{657} & \textbf{531} & \textbf{369} & 369 & 0.00 & 813 & 587 & 532 & 369 & 0.17 & 677\\
\texttt{lymph} & 0.00 & \textbf{16} & 12 & 12 & 12 & 0.00 & 67 & 12 & 12 & 12 & 0.00 & 17\\
\texttt{mnist\_0} & 0.01 & \textbf{3110} & \textbf{2862} & \textbf{2822} & \textbf{2568} & \textbf{0.00} & 5923 & 5923 & 3366 & 2717 & 2.48 & 3329\\
\texttt{mnist\_1} & 0.02 & \textbf{3533} & \textbf{3465} & \textbf{3464} & \textbf{3462} & \textbf{0.00} & 6742 & 6742 & 4725 & 3590 & 2.45 & 3534\\
\texttt{mnist\_2} & 0.01 & \textbf{4338} & \textbf{3994} & \textbf{3994} & \textbf{3938} & \textbf{0.00} & 5958 & 5958 & 4289 & 4026 & 2.57 & 4530\\
\texttt{mnist\_3} & 0.01 & \textbf{5024} & \textbf{4563} & \textbf{4354} & \textbf{4354} & \textbf{0.00} & 6131 & 6131 & 5172 & 4364 & 2.48 & 6131\\
\texttt{mnist\_4} & 0.02 & \textbf{4900} & \textbf{4891} & \textbf{4754} & \textbf{4742} & \textbf{0.00} & 5842 & 5842 & 5580 & 4751 & 2.59 & 5037\\
\texttt{mnist\_5} & 0.01 & 4032 & \textbf{3982} & \textbf{3701} & \textbf{3607} & \textbf{0.00} & 5421 & 5421 & 4400 & 3636 & 2.64 & 4032\\
\texttt{mnist\_6} & 0.01 & 2893 & \textbf{2885} & \textbf{2774} & 2774 & \textbf{0.00} & 5918 & 5918 & 2999 & \textbf{2756} & 2.64 & 2893\\
\texttt{mnist\_7} & 0.01 & \textbf{3660} & \textbf{3617} & \textbf{3483} & \textbf{3483} & \textbf{0.00} & 6265 & 6265 & 4546 & 3978 & 2.54 & 3788\\
\texttt{mnist\_8} & 0.02 & \textbf{4247} & \textbf{4003} & \textbf{4003} & \textbf{3583} & \textbf{0.00} & 5851 & 5851 & 4755 & 4437 & 2.58 & 4250\\
\texttt{mnist\_9} & 0.01 & \textbf{4874} & \textbf{4845} & \textbf{4704} & \textbf{4624} & \textbf{0.00} & 5949 & 5949 & 5254 & 4708 & 2.62 & 5355\\
\texttt{mushroom} & 0.00 & 280 & 8 & 8 & 8 & 0.00 & 3916 & 8 & 8 & 8 & 0.02 & 280\\
\texttt{pendigits} & 0.00 & 51 & \textbf{47} & 47 & 47 & 0.00 & 780 & 72 & 47 & 47 & 0.05 & 51\\
\texttt{primary-tumor} & 0.00 & \textbf{51} & 46 & 46 & 46 & 0.00 & 82 & 46 & 46 & 46 & 0.00 & 53\\
\texttt{segment} & 0.00 & 5 & 0 & 0 & 0 & 0.00 & 330 & 0 & 0 & 0 & 0.01 & 5\\
\texttt{soybean} & 0.00 & 48 & 29 & 29 & 29 & 0.00 & 92 & 29 & 29 & 29 & 0.00 & \textbf{47}\\
\texttt{splice-1} & 0.00 & 279 & \textbf{224} & 224 & 224 & 0.00 & 1535 & 444 & 224 & 224 & 0.03 & 279\\
\texttt{surgical-deepnet} & 0.25 & \textbf{2794} & \textbf{2758} & \textbf{2709} & \textbf{2530} & \textbf{0.00} & 3690 & 3690 & 3690 & 3690 & 5.68 & 2924\\
\texttt{taiwan\_binarised} & 0.00 & \textbf{5333} & \textbf{5326} & 5326 & 5326 & 0.00 & 6636 & 5369 & 5326 & 5326 & 0.26 & 5346\\
\texttt{tic-tac-toe} & 0.00 & 236 & 216 & 216 & 216 & 0.00 & 332 & 216 & 216 & 216 & 0.00 & 236\\
\texttt{titanic} & 0.00 & \textbf{146} & 145 & 143 & 143 & 0.00 & 342 & \textbf{143} & 143 & 143 & 0.01 & 148\\
\texttt{vehicle} & 0.00 & \textbf{55} & 26 & 26 & 26 & 0.00 & 218 & 26 & 26 & 26 & 0.01 & 66\\
\texttt{vote} & 0.00 & 14 & 12 & 12 & 12 & 0.00 & 168 & 12 & 12 & 12 & 0.00 & 14\\
\texttt{weather-aus} & 0.44 & \textbf{1758} & \textbf{1757} & \textbf{1757} & \textbf{1756} & \textbf{0.00} & 31877 & 31877 & 31877 & 31877 & 19.57 & 1761\\
\texttt{wine1} & 0.00 & 45 & \textbf{43} & 43 & 43 & 0.00 & 59 & 45 & 43 & 43 & 0.00 & 45\\
\texttt{wine2} & 0.00 & 52 & \textbf{49} & 49 & 49 & 0.00 & 71 & 57 & 49 & 49 & 0.00 & 52\\
\texttt{wine3} & 0.00 & 35 & \textbf{33} & 33 & 33 & 0.00 & 48 & 37 & 33 & 33 & 0.00 & 35\\
\texttt{yeast} & 0.00 & \textbf{417} & 403 & 403 & 403 & 0.00 & 463 & 403 & 403 & 403 & 0.00 & 418\\
\bottomrule
\end{tabular}

\end{normalsize}
\end{center}
\caption{\label{tab:all} Comparison with state of the art: depth 3}
\end{table}

\medskip

\begin{table}[htbp]
\begin{center}
\begin{normalsize}
\tabcolsep=4pt
\begin{tabular}{lrrrrrrrrrrrr}
\toprule
&  \multicolumn{5}{c}{\budalg} & \multicolumn{5}{c}{\murtree} & \multicolumn{2}{c}{\cart}\\
\cmidrule(rr){2-6}\cmidrule(rr){7-11}\cmidrule(rr){12-13}
& \multicolumn{1}{c}{cpu} & \multicolumn{1}{c}{first} & \multicolumn{1}{c}{$\leq$3s} & \multicolumn{1}{c}{$\leq$10s} & \multicolumn{1}{c}{$\leq$1m} & \multicolumn{1}{c}{cpu} & \multicolumn{1}{c}{first} & \multicolumn{1}{c}{$\leq$3s} & \multicolumn{1}{c}{$\leq$10s} & \multicolumn{1}{c}{$\leq$1m} & \multicolumn{1}{c}{cpu} & \multicolumn{1}{c}{first} \\
\midrule

\texttt{adult\_discretized} & 0.00 & 5149 & \textbf{4609} & \textbf{4609} & 4609 & 0.00 & 7511 & 4985 & 4613 & 4609 & 0.06 & \textbf{5022}\\
\texttt{anneal} & 0.00 & 135 & 91 & 91 & 91 & 0.00 & 187 & 91 & 91 & 91 & 0.00 & 135\\
\texttt{audiology} & 0.00 & 3 & 1 & 1 & 1 & 0.00 & 57 & 1 & 1 & 1 & 0.00 & 3\\
\texttt{australian-credit} & 0.00 & \textbf{73} & \textbf{60} & 57 & 56 & 0.00 & 296 & 66 & 57 & 56 & 0.00 & 74\\
\texttt{bank} & 0.81 & \textbf{4351} & \textbf{4343} & \textbf{4343} & \textbf{4326} & \textbf{0.00} & 5289 & 5287 & 5287 & 5287 & 32.03 & 4420\\
\texttt{breast-cancer} & 0.00 & 21 & 16 & 16 & 16 & 0.00 & 239 & 16 & 16 & 16 & 0.00 & 21\\
\texttt{breast-wisconsin} & 0.00 & 16 & \textbf{7} & 7 & 7 & 0.00 & 239 & 8 & 7 & 7 & 0.00 & 16\\
\texttt{car} & 0.00 & 178 & 136 & 136 & 136 & 0.00 & 518 & 136 & 136 & 136 & 0.00 & 178\\
\texttt{compas\_discretized} & 0.00 & 2023 & 1954 & 1954 & 1954 & 0.00 & 2809 & 1954 & 1954 & 1954 & 0.01 & \textbf{1997}\\
\texttt{diabetes} & 0.00 & \textbf{159} & \textbf{137} & 137 & 137 & 0.00 & 268 & 142 & 137 & 137 & 0.00 & 166\\
\texttt{forest-fires} & 0.00 & 191 & \textbf{179} & 179 & 173 & 0.00 & 247 & 188 & \textbf{175} & \textbf{172} & 0.01 & \textbf{186}\\
\texttt{german-credit} & 0.00 & \textbf{224} & 204 & 204 & 204 & 0.00 & 300 & 204 & 204 & 204 & 0.00 & 231\\
\texttt{heart-cleveland} & 0.00 & \textbf{36} & 25 & 25 & 25 & 0.00 & 136 & 25 & 25 & 25 & 0.00 & 38\\
\texttt{hepatitis} & 0.00 & 12 & 3 & 3 & 3 & 0.00 & 26 & 3 & 3 & 3 & 0.00 & 12\\
\texttt{hypothyroid} & 0.00 & 53 & \textbf{53} & 53 & 53 & 0.00 & 277 & 57 & 53 & 53 & 0.01 & 53\\
\texttt{ionosphere} & 0.00 & \textbf{25} & \textbf{13} & \textbf{8} & \textbf{8} & 0.00 & 126 & 20 & 16 & 9 & 0.01 & 27\\
\texttt{kr-vs-kp} & 0.00 & \textbf{188} & \textbf{144} & 144 & 144 & 0.00 & 1527 & 172 & 144 & 144 & 0.01 & 189\\
\texttt{letter} & 0.00 & \textbf{443} & \textbf{431} & \textbf{276} & \textbf{261} & 0.00 & 813 & 596 & 550 & 338 & 0.20 & 462\\
\texttt{lymph} & 0.00 & \textbf{9} & 3 & 3 & 3 & 0.00 & 67 & 3 & 3 & 3 & 0.00 & 10\\
\texttt{mnist\_0} & 0.01 & \textbf{2310} & \textbf{2285} & \textbf{2265} & \textbf{2245} & \textbf{0.00} & 5923 & 5923 & 3319 & 2717 & 3.80 & 2311\\
\texttt{mnist\_1} & 0.02 & \textbf{2478} & \textbf{2440} & \textbf{2433} & \textbf{2433} & \textbf{0.00} & 6742 & 6742 & 4583 & 3589 & 3.57 & 2501\\
\texttt{mnist\_2} & 0.01 & \textbf{3989} & \textbf{3982} & \textbf{3963} & \textbf{3963} & \textbf{0.00} & 5958 & 5957 & 4304 & 4025 & 3.09 & 4326\\
\texttt{mnist\_3} & 0.02 & \textbf{4263} & \textbf{4181} & \textbf{4180} & \textbf{4180} & \textbf{0.00} & 6131 & 6131 & 4900 & 4364 & 4.85 & 4367\\
\texttt{mnist\_4} & 0.03 & \textbf{4101} & \textbf{4037} & \textbf{4037} & \textbf{4037} & \textbf{0.00} & 5842 & 5842 & 5580 & 4751 & 3.22 & 4129\\
\texttt{mnist\_5} & 0.01 & \textbf{3605} & \textbf{3595} & \textbf{3595} & \textbf{3371} & \textbf{0.00} & 5421 & 5421 & 4401 & 3636 & 3.84 & 3648\\
\texttt{mnist\_6} & 0.01 & \textbf{2248} & \textbf{2196} & \textbf{2191} & \textbf{2124} & \textbf{0.00} & 5918 & 5915 & 2798 & 2754 & 4.12 & 2251\\
\texttt{mnist\_7} & 0.01 & \textbf{3072} & \textbf{3028} & \textbf{3028} & \textbf{2793} & \textbf{0.00} & 6265 & 6265 & 4547 & 3978 & 3.94 & 3218\\
\texttt{mnist\_8} & 0.03 & \textbf{3905} & \textbf{3785} & \textbf{3760} & \textbf{3749} & \textbf{0.00} & 5851 & 5851 & 4753 & 4422 & 4.53 & 3987\\
\texttt{mnist\_9} & 0.01 & \textbf{4226} & \textbf{4199} & \textbf{4199} & \textbf{4117} & \textbf{0.00} & 5949 & 5949 & 5254 & 4708 & 3.15 & 4231\\
\texttt{mushroom} & 0.00 & 4 & \textbf{0} & 0 & 0 & 0.00 & 3916 & 6 & 0 & 0 & 0.02 & 4\\
\texttt{pendigits} & 0.00 & \textbf{22} & \textbf{17} & \textbf{14} & \textbf{13} & 0.00 & 780 & 70 & 32 & 24 & 0.07 & 25\\
\texttt{primary-tumor} & 0.00 & \textbf{43} & 34 & 34 & 34 & 0.00 & 82 & 34 & 34 & 34 & 0.00 & 44\\
\texttt{segment} & 0.00 & 1 & 0 & 0 & 0 & 0.00 & 330 & 0 & 0 & 0 & 0.01 & 1\\
\texttt{soybean} & 0.00 & 33 & 14 & 14 & 14 & 0.00 & 92 & 14 & 14 & 14 & 0.00 & \textbf{32}\\
\texttt{splice-1} & 0.00 & 141 & \textbf{141} & \textbf{141} & \textbf{141} & 0.00 & 1535 & 701 & 224 & 213 & 0.03 & 141\\
\texttt{surgical-deepnet} & 0.21 & \textbf{2605} & \textbf{2604} & \textbf{2590} & \textbf{2485} & \textbf{0.00} & 3690 & 3690 & 3690 & 3690 & 6.16 & 2704\\
\texttt{taiwan\_binarised} & 0.00 & \textbf{5293} & \textbf{5284} & \textbf{5273} & \textbf{5273} & 0.00 & 6636 & 5522 & 5489 & 5309 & 0.27 & 5306\\
\texttt{tic-tac-toe} & 0.00 & 150 & 137 & 137 & 137 & 0.00 & 332 & 137 & 137 & 137 & 0.00 & 150\\
\texttt{titanic} & 0.00 & 134 & \textbf{122} & \textbf{119} & 119 & 0.00 & 342 & 134 & 131 & 119 & 0.01 & 134\\
\texttt{vehicle} & 0.00 & 28 & \textbf{13} & \textbf{12} & 12 & 0.00 & 218 & 22 & 16 & 12 & 0.01 & 28\\
\texttt{vote} & 0.00 & 8 & 5 & 5 & 5 & 0.00 & 168 & 5 & 5 & 5 & 0.00 & 8\\
\texttt{weather-aus} & 0.46 & \textbf{1757} & \textbf{1753} & \textbf{1752} & \textbf{1752} & \textbf{0.00} & 31877 & 31877 & 31877 & 31877 & 19.96 & 1761\\
\texttt{wine1} & 0.00 & 42 & \textbf{39} & \textbf{39} & \textbf{39} & 0.00 & 59 & 45 & 43 & 42 & 0.01 & 42\\
\texttt{wine2} & 0.01 & 47 & \textbf{46} & \textbf{46} & \textbf{43} & \textbf{0.00} & 71 & 57 & 49 & 49 & 0.01 & 47\\
\texttt{wine3} & 0.01 & 32 & \textbf{30} & \textbf{30} & \textbf{28} & \textbf{0.00} & 48 & 37 & 33 & 32 & 0.01 & 32\\
\texttt{yeast} & 0.00 & \textbf{391} & 367 & 366 & 366 & 0.00 & 463 & \textbf{366} & 366 & 366 & 0.01 & 394\\
\bottomrule
\end{tabular}

\end{normalsize}
\end{center}
\caption{\label{tab:all} Comparison with state of the art: depth 4}
\end{table}

\medskip

\begin{table}[htbp]
\begin{center}
\begin{normalsize}
\tabcolsep=4pt
\begin{tabular}{lrrrrrrrrrrrr}
\toprule
&  \multicolumn{5}{c}{\budalg} & \multicolumn{5}{c}{\murtree} & \multicolumn{2}{c}{\cart}\\
\cmidrule(rr){2-6}\cmidrule(rr){7-11}\cmidrule(rr){12-13}
& \multicolumn{1}{c}{cpu} & \multicolumn{1}{c}{first} & \multicolumn{1}{c}{$\leq$3s} & \multicolumn{1}{c}{$\leq$10s} & \multicolumn{1}{c}{$\leq$1m} & \multicolumn{1}{c}{cpu} & \multicolumn{1}{c}{first} & \multicolumn{1}{c}{$\leq$3s} & \multicolumn{1}{c}{$\leq$10s} & \multicolumn{1}{c}{$\leq$1m} & \multicolumn{1}{c}{cpu} & \multicolumn{1}{c}{first} \\
\midrule

\texttt{adult\_discretized} & 0.00 & \textbf{4183} & \textbf{4174} & \textbf{4174} & \textbf{4173} & 0.00 & 7511 & 5773 & 5773 & 5374 & 0.12 & 4252\\
\texttt{anneal} & 0.00 & 77 & \textbf{53} & \textbf{39} & \textbf{39} & 0.00 & 187 & 117 & 101 & 101 & 0.00 & \textbf{74}\\
\texttt{audiology} & 0.00 & 0 & 0 & 0 & 0 & 0.00 & 57 & 0 & 0 & 0 & 0.00 & 0\\
\texttt{australian-credit} & 0.00 & 20 & \textbf{1} & \textbf{0} & \textbf{0} & 0.00 & 296 & 152 & 122 & 120 & 0.01 & \textbf{19}\\
\texttt{bank} & 1.05 & \textbf{3522} & \textbf{3520} & \textbf{3513} & \textbf{3508} & \textbf{0.00} & 5289 & 5278 & 5278 & 5278 & 75.74 & 3575\\
\texttt{breast-cancer} & 0.00 & 1 & 1 & 0 & 0 & 0.00 & 239 & \textbf{0} & 0 & 0 & 0.00 & 1\\
\texttt{breast-wisconsin} & 0.00 & 0 & 0 & 0 & 0 & 0.00 & 239 & 0 & 0 & 0 & 0.00 & 0\\
\texttt{car} & 0.00 & 15 & \textbf{6} & 2 & 2 & 0.00 & 518 & 181 & \textbf{0} & \textbf{0} & 0.00 & 15\\
\texttt{compas\_discretized} & 0.00 & \textbf{1880} & \textbf{1836} & \textbf{1832} & \textbf{1829} & 0.00 & 2809 & 2052 & 1965 & 1938 & 0.01 & 1891\\
\texttt{diabetes} & 0.00 & \textbf{54} & \textbf{3} & \textbf{0} & \textbf{0} & 0.00 & 268 & 172 & 150 & 142 & 0.01 & 55\\
\texttt{forest-fires} & 0.00 & \textbf{151} & \textbf{150} & \textbf{150} & \textbf{150} & 0.00 & 247 & 172 & 153 & 151 & 0.02 & 152\\
\texttt{german-credit} & 0.00 & \textbf{88} & \textbf{42} & \textbf{39} & \textbf{18} & 0.00 & 300 & 196 & 181 & 171 & 0.01 & 97\\
\texttt{heart-cleveland} & 0.00 & 0 & 0 & 0 & 0 & 0.00 & 136 & 0 & 0 & 0 & 0.00 & 0\\
\texttt{hepatitis} & 0.00 & 0 & 0 & 0 & 0 & 0.00 & 26 & 0 & 0 & 0 & 0.00 & 0\\
\texttt{hypothyroid} & 0.00 & \textbf{35} & \textbf{22} & \textbf{19} & \textbf{17} & 0.00 & 277 & 185 & 151 & 151 & 0.01 & 36\\
\texttt{ionosphere} & 0.00 & 0 & \textbf{0} & 0 & 0 & 0.00 & 126 & 23 & 0 & 0 & 0.01 & 0\\
\texttt{kr-vs-kp} & 0.00 & 23 & \textbf{13} & \textbf{13} & \textbf{13} & 0.00 & 1527 & 575 & 230 & 143 & 0.01 & 23\\
\texttt{letter} & 0.00 & \textbf{47} & \textbf{28} & \textbf{27} & \textbf{26} & 0.00 & 813 & 747 & 697 & 697 & 0.37 & 48\\
\texttt{lymph} & 0.00 & 0 & 0 & 0 & 0 & 0.00 & 67 & 0 & 0 & 0 & 0.00 & 0\\
\texttt{mnist\_0} & 0.04 & \textbf{694} & \textbf{692} & \textbf{692} & \textbf{683} & \textbf{0.00} & 5923 & 5923 & 3358 & 2717 & 8.62 & 710\\
\texttt{mnist\_1} & 0.03 & \textbf{570} & \textbf{563} & \textbf{563} & \textbf{561} & \textbf{0.00} & 6742 & 6742 & 4702 & 3584 & 6.47 & 573\\
\texttt{mnist\_2} & 0.02 & \textbf{2035} & \textbf{2013} & \textbf{2012} & \textbf{2011} & \textbf{0.00} & 5958 & 5955 & 4282 & 4024 & 7.24 & 2058\\
\texttt{mnist\_3} & 0.04 & \textbf{1399} & \textbf{1381} & \textbf{1381} & \textbf{1376} & \textbf{0.00} & 6131 & 6131 & 5188 & 4363 & 6.89 & 1442\\
\texttt{mnist\_4} & 0.04 & \textbf{1247} & \textbf{1224} & \textbf{1224} & \textbf{1219} & \textbf{0.00} & 5842 & 5842 & 5580 & 4750 & 5.36 & 1306\\
\texttt{mnist\_5} & 0.03 & \textbf{2476} & \textbf{2466} & \textbf{2465} & \textbf{2458} & \textbf{0.00} & 5421 & 5421 & 4396 & 3636 & 9.13 & 2553\\
\texttt{mnist\_6} & 0.01 & \textbf{1240} & \textbf{1227} & \textbf{1226} & \textbf{1193} & \textbf{0.00} & 5918 & 5861 & 2734 & 2715 & 6.17 & 1245\\
\texttt{mnist\_7} & 0.02 & \textbf{1326} & \textbf{1321} & \textbf{1320} & \textbf{1306} & \textbf{0.00} & 6265 & 6265 & 4546 & 3978 & 7.24 & 1371\\
\texttt{mnist\_8} & 0.04 & \textbf{1229} & \textbf{1207} & \textbf{1207} & \textbf{1197} & \textbf{0.00} & 5851 & 5851 & 4754 & 4436 & 6.87 & 1267\\
\texttt{mnist\_9} & 0.02 & \textbf{2038} & \textbf{2023} & \textbf{1997} & \textbf{1970} & \textbf{0.00} & 5949 & 5949 & 5253 & 4708 & 9.32 & 2110\\
\texttt{mushroom} & 0.00 & 0 & 0 & 0 & 0 & 0.00 & 3916 & 0 & 0 & 0 & 0.03 & 0\\
\texttt{pendigits} & 0.00 & 0 & \textbf{0} & \textbf{0} & \textbf{0} & 0.00 & 780 & 465 & 465 & 460 & 0.07 & 0\\
\texttt{primary-tumor} & 0.00 & \textbf{19} & \textbf{15} & \textbf{15} & \textbf{15} & 0.00 & 82 & 37 & 26 & 26 & 0.01 & 21\\
\texttt{segment} & 0.00 & 0 & 0 & 0 & 0 & 0.00 & 330 & 0 & 0 & 0 & 0.01 & 0\\
\texttt{soybean} & 0.00 & 9 & \textbf{2} & \textbf{2} & \textbf{2} & 0.00 & 92 & 29 & 29 & 17 & 0.00 & \textbf{5}\\
\texttt{splice-1} & 0.00 & 18 & \textbf{14} & \textbf{14} & \textbf{14} & 0.00 & 1535 & 1219 & 1153 & 1145 & 0.05 & 18\\
\texttt{surgical-deepnet} & 0.29 & \textbf{1169} & \textbf{1166} & \textbf{1159} & \textbf{1159} & \textbf{0.00} & 3690 & 3687 & 3687 & 3687 & 10.93 & 1193\\
\texttt{taiwan\_binarised} & 0.01 & \textbf{4858} & \textbf{4819} & \textbf{4805} & \textbf{4805} & \textbf{0.00} & 6636 & 6316 & 5963 & 5962 & 0.63 & 4911\\
\texttt{tic-tac-toe} & 0.00 & 10 & 0 & 0 & 0 & 0.00 & 332 & 0 & 0 & 0 & 0.00 & 10\\
\texttt{titanic} & 0.00 & \textbf{87} & \textbf{72} & \textbf{71} & \textbf{68} & 0.00 & 342 & 124 & 120 & 117 & 0.01 & 93\\
\texttt{vehicle} & 0.00 & 1 & 0 & 0 & 0 & 0.00 & 218 & 0 & 0 & 0 & 0.01 & 1\\
\texttt{vote} & 0.00 & 1 & 0 & 0 & 0 & 0.00 & 168 & 0 & 0 & 0 & 0.00 & 1\\
\texttt{weather-aus} & 0.46 & \textbf{1672} & \textbf{1669} & \textbf{1667} & \textbf{1667} & \textbf{0.00} & 31877 & 31874 & 31874 & 31874 & 27.11 & 1677\\
\texttt{wine1} & 0.01 & 27 & \textbf{26} & \textbf{26} & \textbf{25} & \textbf{0.00} & 59 & 39 & 37 & 36 & 0.01 & 27\\
\texttt{wine2} & 0.00 & 32 & \textbf{29} & \textbf{29} & \textbf{29} & 0.00 & 71 & 57 & 49 & 49 & 0.01 & 32\\
\texttt{wine3} & 0.01 & 22 & \textbf{20} & \textbf{20} & \textbf{20} & \textbf{0.00} & 48 & 36 & 32 & 32 & 0.01 & \textbf{18}\\
\texttt{yeast} & 0.00 & \textbf{224} & \textbf{167} & \textbf{142} & \textbf{121} & 0.00 & 463 & 397 & 361 & 360 & 0.01 & 232\\
\bottomrule
\end{tabular}

\end{normalsize}
\end{center}
\caption{\label{tab:all} Comparison with state of the art: depth 9}
\end{table}

\medskip




\begin{table}[htbp]
\begin{center}
\begin{normalsize}
\tabcolsep=4pt
\begin{tabular}{lccrrrrrrrrrrrr}
\toprule
\multirow{2}{*}{}& && \multicolumn{2}{c}{\budalg} & \multicolumn{2}{c}{\murtree} & \multicolumn{2}{c}{\dleight} & \multicolumn{2}{c}{\cp} & \multicolumn{2}{c}{binoct} & \multicolumn{2}{c}{\cart}\\
\cmidrule(rr){4-5}\cmidrule(rr){6-7}\cmidrule(rr){8-9}\cmidrule(rr){10-11}\cmidrule(rr){12-13}\cmidrule(rr){14-15}
&\multirow{1}{*}{$\#ex.$} & \multirow{1}{*}{\#feat.} &  \multicolumn{1}{c}{error} & \multicolumn{1}{c}{cpu} & \multicolumn{1}{c}{error} & \multicolumn{1}{c}{cpu} & \multicolumn{1}{c}{error} & \multicolumn{1}{c}{cpu} & \multicolumn{1}{c}{error} & \multicolumn{1}{c}{cpu} & \multicolumn{1}{c}{error} & \multicolumn{1}{c}{cpu} & \multicolumn{1}{c}{error} & \multicolumn{1}{c}{cpu} \\
\midrule

\texttt{adult\_discretized} & \multicolumn{1}{r}{30299} & \multicolumn{1}{r}{59}  & 5020 & 0.43$^*$ & 5020 & 0.84$^*$ & 5020 & 10$^*$ & 5020 & 6.4$^*$ & 5600 & $\mathsmaller{\geq}1$h & 5758 & 0.05\\
\texttt{anneal} & \multicolumn{1}{r}{812} & \multicolumn{1}{r}{93}  & 112 & 0.03$^*$ & 112 & 0.14$^*$ & 112 & 2.4$^*$ & 112 & 6.0$^*$ & 123 & $\mathsmaller{\geq}1$h & 149 & 0.00\\
\texttt{audiology} & \multicolumn{1}{r}{216} & \multicolumn{1}{r}{148}  & 5 & 0.06$^*$ & 5 & 0.13$^*$ & 5 & 4.5$^*$ & 5 & 9.1$^*$ & 6 & $\mathsmaller{\geq}1$h & 6 & 0.00\\
\texttt{australian-credit} & \multicolumn{1}{r}{653} & \multicolumn{1}{r}{125}  & 73 & 0.14$^*$ & 73 & 0.35$^*$ & 73 & 9.6$^*$ & 73 & 14$^*$ & 87 & $\mathsmaller{\geq}1$h & 87 & 0.00\\
\texttt{bank} & \multicolumn{1}{r}{45211} & \multicolumn{1}{r}{9531}  & 4453 & 259 & 5289 & 0.84 & 4805 & $\mathsmaller{\geq}1$h & 4453 & $\mathsmaller{\geq}1$h & - & - & 4462 & 33\\
\texttt{breast-cancer} & \multicolumn{1}{r}{683} & \multicolumn{1}{r}{89}  & 24 & 0.16$^*$ & 24 & 0.07$^*$ & 24 & 0.98$^*$ & 24 & 5.7$^*$ & 25 & $\mathsmaller{\geq}1$h & 28 & 0.00\\
\texttt{breast-wisconsin} & \multicolumn{1}{r}{683} & \multicolumn{1}{r}{120}  & 15 & 0.05$^*$ & 15 & 0.20$^*$ & 15 & 6.4$^*$ & 15 & 11$^*$ & 18 & $\mathsmaller{\geq}1$h & 26 & 0.00\\
\texttt{car} & \multicolumn{1}{r}{1728} & \multicolumn{1}{r}{21}  & 192 & 0.01$^*$ & 192 & 0.01$^*$ & 192 & 0.04$^*$ & 192 & 1.7$^*$ & 192 & $\mathsmaller{\geq}1$h & 202 & 0.00\\
\texttt{compas\_discretized} & \multicolumn{1}{r}{6167} & \multicolumn{1}{r}{25}  & 2004 & 0.00$^*$ & 2004 & 0.06$^*$ & 2004 & 0.23$^*$ & 2004 & 1.8$^*$ & 2032 & $\mathsmaller{\geq}1$h & 2072 & 0.01\\
\texttt{diabetes} & \multicolumn{1}{r}{768} & \multicolumn{1}{r}{112}  & 162 & 0.09$^*$ & 162 & 0.37$^*$ & 162 & 11$^*$ & 162 & 12$^*$ & 165 & $\mathsmaller{\geq}1$h & 177 & 0.00\\
\texttt{forest-fires} & \multicolumn{1}{r}{517} & \multicolumn{1}{r}{989}  & 193 & 20$^*$ & 193 & 9.6$^*$ & - & - & 193 & 2836$^*$ & 198 & $\mathsmaller{\geq}1$h & 198 & 0.01\\
\texttt{german-credit} & \multicolumn{1}{r}{1000} & \multicolumn{1}{r}{112}  & 236 & 0.26$^*$ & 236 & 0.38$^*$ & 236 & 7.7$^*$ & 236 & 13$^*$ & 244 & $\mathsmaller{\geq}1$h & 251 & 0.00\\
\texttt{heart-cleveland} & \multicolumn{1}{r}{296} & \multicolumn{1}{r}{95}  & 41 & 0.05$^*$ & 41 & 0.12$^*$ & 41 & 3.5$^*$ & 41 & 6.8$^*$ & 42 & $\mathsmaller{\geq}1$h & 43 & 0.00\\
\texttt{hepatitis} & \multicolumn{1}{r}{137} & \multicolumn{1}{r}{68}  & 10 & 0.00$^*$ & 10 & 0.03$^*$ & 10 & 1.2$^*$ & 10 & 3.9$^*$ & 10 & $\mathsmaller{\geq}1$h & 16 & 0.00\\
\texttt{hypothyroid} & \multicolumn{1}{r}{3247} & \multicolumn{1}{r}{88}  & 61 & 0.07$^*$ & 61 & 0.41$^*$ & 61 & 4.4$^*$ & 61 & 6.6$^*$ & 62 & $\mathsmaller{\geq}1$h & 62 & 0.01\\
\texttt{ionosphere} & \multicolumn{1}{r}{351} & \multicolumn{1}{r}{445}  & 22 & 3.8$^*$ & 22 & 12$^*$ & 22 & 410$^*$ & 22 & 460$^*$ & 27 & $\mathsmaller{\geq}1$h & 29 & 0.01\\
\texttt{kr-vs-kp} & \multicolumn{1}{r}{3196} & \multicolumn{1}{r}{73}  & 198 & 0.09$^*$ & 198 & 0.22$^*$ & 198 & 2.4$^*$ & 198 & 4.8$^*$ & 375 & $\mathsmaller{\geq}1$h & 306 & 0.01\\
\texttt{letter} & \multicolumn{1}{r}{20000} & \multicolumn{1}{r}{224}  & 369 & 10$^*$ & 369 & 34$^*$ & 369 & 443$^*$ & 369 & 158$^*$ & 813 & 1251 & 677 & 0.17\\
\texttt{lymph} & \multicolumn{1}{r}{148} & \multicolumn{1}{r}{68}  & 12 & 0.01$^*$ & 12 & 0.03$^*$ & 12 & 0.76$^*$ & 12 & 3.7$^*$ & 14 & $\mathsmaller{\geq}1$h & 17 & 0.00\\
\texttt{mnist\_0} & \multicolumn{1}{r}{60000} & \multicolumn{1}{r}{784}  & 2557 & 1994$^*$ & 2557 & 568$^*$ & 3319 & $\mathsmaller{\geq}1$h & 2557 & $\mathsmaller{\geq}1$h & - & - & 3329 & 2.5\\
\texttt{mnist\_1} & \multicolumn{1}{r}{60000} & \multicolumn{1}{r}{784}  & 3462 & 1896$^*$ & 3462 & 538$^*$ & 4552 & $\mathsmaller{\geq}1$h & 3462 & $\mathsmaller{\geq}1$h & - & - & 3534 & 2.5\\
\texttt{mnist\_2} & \multicolumn{1}{r}{60000} & \multicolumn{1}{r}{784}  & 3938 & 1946$^*$ & 3938 & 672$^*$ & 4289 & $\mathsmaller{\geq}1$h & 3938 & $\mathsmaller{\geq}1$h & - & - & 4530 & 2.6\\
\texttt{mnist\_3} & \multicolumn{1}{r}{60000} & \multicolumn{1}{r}{784}  & 4354 & 2054$^*$ & 4354 & 644$^*$ & 4974 & $\mathsmaller{\geq}1$h & 4354 & $\mathsmaller{\geq}1$h & - & - & 6131 & 2.5\\
\texttt{mnist\_4} & \multicolumn{1}{r}{60000} & \multicolumn{1}{r}{784}  & 4729 & 2070$^*$ & 4729 & 700$^*$ & 5580 & $\mathsmaller{\geq}1$h & 4729 & $\mathsmaller{\geq}1$h & - & - & 5037 & 2.6\\
\texttt{mnist\_5} & \multicolumn{1}{r}{60000} & \multicolumn{1}{r}{784}  & 3539 & 2095$^*$ & 3539 & 715$^*$ & 4379 & $\mathsmaller{\geq}1$h & 3539 & $\mathsmaller{\geq}1$h & - & - & 4032 & 2.6\\
\texttt{mnist\_6} & \multicolumn{1}{r}{60000} & \multicolumn{1}{r}{784}  & 2756 & 1916$^*$ & 2756 & 664$^*$ & 2756 & $\mathsmaller{\geq}1$h & 2756 & $\mathsmaller{\geq}1$h & - & - & 2893 & 2.6\\
\texttt{mnist\_7} & \multicolumn{1}{r}{60000} & \multicolumn{1}{r}{784}  & 3483 & 1928$^*$ & 3483 & 570$^*$ & 4546 & $\mathsmaller{\geq}1$h & 3483 & $\mathsmaller{\geq}1$h & - & - & 3788 & 2.5\\
\texttt{mnist\_8} & \multicolumn{1}{r}{60000} & \multicolumn{1}{r}{784}  & 3583 & 2061$^*$ & 3583 & 593$^*$ & 4609 & $\mathsmaller{\geq}1$h & 3583 & $\mathsmaller{\geq}1$h & - & - & 4250 & 2.6\\
\texttt{mnist\_9} & \multicolumn{1}{r}{60000} & \multicolumn{1}{r}{784}  & 4590 & 2039$^*$ & 4590 & 746$^*$ & 5253 & $\mathsmaller{\geq}1$h & 4590 & $\mathsmaller{\geq}1$h & - & - & 5355 & 2.6\\
\texttt{mushroom} & \multicolumn{1}{r}{8124} & \multicolumn{1}{r}{119}  & 8 & 0.79$^*$ & 8 & 0.53$^*$ & 8 & 6.3$^*$ & 8 & 8.4$^*$ & 180 & $\mathsmaller{\geq}1$h & 280 & 0.02\\
\texttt{pendigits} & \multicolumn{1}{r}{7494} & \multicolumn{1}{r}{216}  & 47 & 3.3$^*$ & 47 & 11$^*$ & 47 & 134$^*$ & 47 & 70$^*$ & 477 & $\mathsmaller{\geq}1$h & 51 & 0.05\\
\texttt{primary-tumor} & \multicolumn{1}{r}{336} & \multicolumn{1}{r}{31}  & 46 & 0.00$^*$ & 46 & 0.01$^*$ & 46 & 0.14$^*$ & 46 & 2.0$^*$ & 46 & $\mathsmaller{\geq}1$h & 53 & 0.00\\
\texttt{segment} & \multicolumn{1}{r}{2310} & \multicolumn{1}{r}{235}  & 0 & 0.03$^*$ & 0 & 0.13$^*$ & 0 & 2.3$^*$ & 0 & 4.1$^*$ & 4 & $\mathsmaller{\geq}1$h & 5 & 0.01\\
\texttt{soybean} & \multicolumn{1}{r}{630} & \multicolumn{1}{r}{50}  & 29 & 0.01$^*$ & 29 & 0.02$^*$ & 29 & 0.29$^*$ & 29 & 2.3$^*$ & 31 & $\mathsmaller{\geq}1$h & 47 & 0.00\\
\texttt{splice-1} & \multicolumn{1}{r}{3190} & \multicolumn{1}{r}{287}  & 224 & 9.8$^*$ & 224 & 5.3$^*$ & 224 & 114$^*$ & 224 & 173$^*$ & 453 & $\mathsmaller{\geq}1$h & 279 & 0.03\\
\texttt{surgical-deepnet} & \multicolumn{1}{r}{14635} & \multicolumn{1}{r}{6047}  & 2512 & 953 & 2512 & 3523 & - & - & 2512 & $\mathsmaller{\geq}1$h & - & - & 2924 & 5.7\\
\texttt{taiwan\_binarised} & \multicolumn{1}{r}{30000} & \multicolumn{1}{r}{205}  & 5326 & 48$^*$ & 5326 & 45$^*$ & 5326 & 526$^*$ & 5326 & 190$^*$ & 6636 & 1639 & 5346 & 0.26\\
\texttt{tic-tac-toe} & \multicolumn{1}{r}{958} & \multicolumn{1}{r}{27}  & 216 & 0.01$^*$ & 216 & 0.02$^*$ & 216 & 0.13$^*$ & 216 & 1.8$^*$ & 232 & $\mathsmaller{\geq}1$h & 236 & 0.00\\
\texttt{titanic} & \multicolumn{1}{r}{887} & \multicolumn{1}{r}{333}  & 143 & 6.7$^*$ & 143 & 11$^*$ & 143 & 167$^*$ & 143 & 173$^*$ & 150 & $\mathsmaller{\geq}1$h & 148 & 0.01\\
\texttt{vehicle} & \multicolumn{1}{r}{846} & \multicolumn{1}{r}{252}  & 26 & 0.93$^*$ & 26 & 2.2$^*$ & 26 & 64$^*$ & 26 & 66$^*$ & 42 & $\mathsmaller{\geq}1$h & 66 & 0.01\\
\texttt{vote} & \multicolumn{1}{r}{435} & \multicolumn{1}{r}{48}  & 12 & 0.02$^*$ & 12 & 0.02$^*$ & 12 & 0.34$^*$ & 12 & 2.6$^*$ & 13 & $\mathsmaller{\geq}1$h & 14 & 0.00\\
\texttt{weather-aus} & \multicolumn{1}{r}{142193} & \multicolumn{1}{r}{4759}  & 1756 & 14 & 1756 & 611 & - & - & 1756 & $\mathsmaller{\geq}1$h & - & - & 1761 & 20\\
\texttt{wine1} & \multicolumn{1}{r}{178} & \multicolumn{1}{r}{1276}  & 43 & 16$^*$ & 43 & 9.0$^*$ & - & - & 43 & $\mathsmaller{\geq}1$h & 44 & $\mathsmaller{\geq}1$h & 45 & 0.00\\
\texttt{wine2} & \multicolumn{1}{r}{178} & \multicolumn{1}{r}{1276}  & 49 & 17$^*$ & 49 & 5.8$^*$ & - & - & 49 & $\mathsmaller{\geq}1$h & 57 & $\mathsmaller{\geq}1$h & 52 & 0.00\\
\texttt{wine3} & \multicolumn{1}{r}{178} & \multicolumn{1}{r}{1276}  & 33 & 16$^*$ & 33 & 8.4$^*$ & - & - & 33 & $\mathsmaller{\geq}1$h & 35 & $\mathsmaller{\geq}1$h & 35 & 0.00\\
\texttt{yeast} & \multicolumn{1}{r}{1484} & \multicolumn{1}{r}{89}  & 403 & 0.07$^*$ & 403 & 0.34$^*$ & 403 & 6.1$^*$ & 403 & 7.7$^*$ & 434 & $\mathsmaller{\geq}1$h & 418 & 0.00\\
\bottomrule
\end{tabular}

\end{normalsize}
\end{center}
\caption{\label{tab:all} Comparison with state of the art: depth 3}
\end{table}

\medskip

\begin{table}[htbp]
\begin{center}
\begin{normalsize}
\tabcolsep=4pt
\begin{tabular}{lrrrrrrrrrrrr}
\toprule
\multirow{2}{*}{}&  \multicolumn{2}{c}{\budalg} & \multicolumn{2}{c}{\murtree} & \multicolumn{2}{c}{\dleight} & \multicolumn{2}{c}{\cp} & \multicolumn{2}{c}{binoct} & \multicolumn{2}{c}{\cart}\\
\cmidrule(rr){2-3}\cmidrule(rr){4-5}\cmidrule(rr){6-7}\cmidrule(rr){8-9}\cmidrule(rr){10-11}\cmidrule(rr){12-13}
& \multicolumn{1}{c}{error} & \multicolumn{1}{c}{cpu} & \multicolumn{1}{c}{error} & \multicolumn{1}{c}{cpu} & \multicolumn{1}{c}{error} & \multicolumn{1}{c}{cpu} & \multicolumn{1}{c}{error} & \multicolumn{1}{c}{cpu} & \multicolumn{1}{c}{error} & \multicolumn{1}{c}{cpu} & \multicolumn{1}{c}{error} & \multicolumn{1}{c}{cpu} \\
\midrule

\texttt{hepatitis} & 3 & 0.32$^*$ & 3 & 0.73$^*$ & 3 & 28$^*$ & 3 & 70$^*$ & 11 & 510 & 12 & 0.00\\
\texttt{lymph} & 3 & 0.74$^*$ & 3 & 0.63$^*$ & 3 & 14$^*$ & 3 & 64$^*$ & 7 & 2987 & 10 & 0.00\\
\texttt{wine1} & 37 & 1674 & 37 & 1831$^*$ & - & - & 39 & $\mathsmaller{\geq}1$h & 45 & 3506 & 42 & 0.01\\
\texttt{wine2} & 43 & 17 & 43 & 1833$^*$ & - & - & 46 & $\mathsmaller{\geq}1$h & 57 & 3232 & 47 & 0.01\\
\texttt{wine3} & 28 & 33 & 28 & 2537$^*$ & - & - & 30 & $\mathsmaller{\geq}1$h & 32 & 3388 & 32 & 0.01\\
\texttt{audiology} & 1 & 4.0$^*$ & 1 & 6.4$^*$ & 1 & 128$^*$ & 1 & 773$^*$ & 2 & 2687 & 3 & 0.00\\
\texttt{heart-cleveland} & 25 & 3.1$^*$ & 25 & 4.8$^*$ & 25 & 154$^*$ & 25 & 391$^*$ & 37 & 2750 & 38 & 0.00\\
\texttt{primary-tumor} & 34 & 0.03$^*$ & 34 & 0.11$^*$ & 34 & 2.0$^*$ & 34 & 5.6$^*$ & 38 & 3132 & 44 & 0.00\\
\texttt{ionosphere} & 7 & 730$^*$ & 7 & 1683$^*$ & - & - & 8 & $\mathsmaller{\geq}1$h & 24 & 751 & 27 & 0.01\\
\texttt{vote} & 5 & 1.2$^*$ & 5 & 0.50$^*$ & 5 & 7.6$^*$ & 5 & 21$^*$ & 12 & 3311 & 8 & 0.00\\
\texttt{forest-fires} & 173 & 15 & \textbf{171} & 2907$^*$ & - & - & 179 & $\mathsmaller{\geq}1$h & 196 & 3356 & 186 & 0.01\\
\texttt{soybean} & 14 & 0.62$^*$ & 14 & 0.46$^*$ & 14 & 5.1$^*$ & 14 & 22$^*$ & 22 & 2906 & 32 & 0.00\\
\texttt{australian-credit} & 56 & 10$^*$ & 56 & 24$^*$ & 56 & 470$^*$ & 56 & 1170$^*$ & 83 & 3258 & 74 & 0.00\\
\texttt{breast-cancer} & 16 & 9.6$^*$ & 16 & 2.9$^*$ & 16 & 28$^*$ & 16 & 219$^*$ & 22 & 2746 & 21 & 0.00\\
\texttt{breast-wisconsin} & 7 & 3.1$^*$ & 7 & 9.3$^*$ & 7 & 245$^*$ & 7 & 662$^*$ & 15 & 3460 & 16 & 0.00\\
\texttt{diabetes} & 137 & 5.7$^*$ & 137 & 22$^*$ & 137 & 550$^*$ & 137 & 1001$^*$ & 180 & 2663 & 166 & 0.00\\
\texttt{anneal} & 91 & 1.5$^*$ & 91 & 5.0$^*$ & 91 & 102$^*$ & 91 & 193$^*$ & 108 & 2954 & 135 & 0.00\\
\texttt{vehicle} & 12 & 71$^*$ & 12 & 172$^*$ & - & - & 12 & $\mathsmaller{\geq}1$h & 30 & 3410 & 28 & 0.01\\
\texttt{titanic} & 119 & 1604$^*$ & 119 & 2104$^*$ & - & - & 119 & $\mathsmaller{\geq}1$h & 135 & 3501 & 134 & 0.01\\
\texttt{tic-tac-toe} & 137 & 0.38$^*$ & 137 & 0.26$^*$ & 137 & 1.8$^*$ & 137 & 7.2$^*$ & 162 & 2511 & 150 & 0.00\\
\texttt{german-credit} & 204 & 28$^*$ & 204 & 27$^*$ & 204 & 423$^*$ & 204 & 1008$^*$ & 236 & 3306 & 231 & 0.00\\
\texttt{yeast} & 366 & 3.4$^*$ & 366 & 18$^*$ & 366 & 257$^*$ & 366 & 386$^*$ & 438 & 888 & 394 & 0.01\\
\texttt{car} & 136 & 0.19$^*$ & 136 & 0.16$^*$ & 136 & 0.36$^*$ & 136 & 2.8$^*$ & 178 & 871 & 178 & 0.00\\
\texttt{segment} & 0 & 0.00$^*$ & 0 & 0.02$^*$ & 0 & 1.6$^*$ & 0 & 2.5$^*$ & 1 & 3501 & 1 & 0.01\\
\texttt{splice-1} & 141 & 3241$^*$ & 141 & 644$^*$ & - & - & 141 & $\mathsmaller{\geq}1$h & 568 & 3416 & 141 & 0.03\\
\texttt{kr-vs-kp} & 144 & 2.8$^*$ & 144 & 6.9$^*$ & 144 & 88$^*$ & 144 & 141$^*$ & 189 & 2850 & 189 & 0.01\\
\texttt{hypothyroid} & 53 & 2.9$^*$ & 53 & 16$^*$ & 53 & 181$^*$ & 53 & 254$^*$ & 55 & 3071 & 53 & 0.01\\
\texttt{compas\_discretized} & 1954 & 0.07$^*$ & 1954 & 1.0$^*$ & 1954 & 3.5$^*$ & 1954 & 6.3$^*$ & 1991 & 3390 & 1997 & 0.01\\
\texttt{pendigits} & 13 & 230$^*$ & 13 & 833$^*$ & - & - & 14 & $\mathsmaller{\geq}1$h & 780 & 0.00 & 25 & 0.07\\
\texttt{mushroom} & 0 & 0.00$^*$ & 0 & 0.03$^*$ & 0 & 41$^*$ & 0 & 0.07$^*$ & 192 & 3354 & 4 & 0.02\\
\texttt{surgical-deepnet} & \textbf{2269} & 49 & 2506 & 489 & - & - & 3690 & $\mathsmaller{\geq}1$h & - & - & 2704 & 6.2\\
\texttt{letter} & 261 & 1185$^*$ & 261 & 2956$^*$ & 335 & $\mathsmaller{\geq}1$h & 261 & $\mathsmaller{\geq}1$h & 813 & 0.00 & 462 & 0.20\\
\texttt{taiwan\_binarised} & 5273 & 6.2 & 5273 & 37 & 5307 & $\mathsmaller{\geq}1$h & 5273 & $\mathsmaller{\geq}1$h & 6521 & 75 & 5306 & 0.27\\
\texttt{adult\_discretized} & 4609 & 14$^*$ & 4609 & 30$^*$ & 4609 & 271$^*$ & 4609 & 246$^*$ & 5659 & 3392 & 5022 & 0.06\\
\texttt{bank} & \textbf{4314} & 290 & 4686 & 2.5 & 4808 & $\mathsmaller{\geq}1$h & 5289 & $\mathsmaller{\geq}1$h & - & - & 4420 & 32\\
\texttt{mnist\_8} & 3165 & 1206 & 3165 & 564 & 4609 & $\mathsmaller{\geq}1$h & 5851 & $\mathsmaller{\geq}1$h & - & - & 3987 & 4.5\\
\texttt{mnist\_9} & 3977 & 2061 & 3977 & 1211 & 5252 & $\mathsmaller{\geq}1$h & 5949 & $\mathsmaller{\geq}1$h & - & - & 4231 & 3.1\\
\texttt{mnist\_0} & 2173 & 2158 & \textbf{1951} & 3542 & 3319 & $\mathsmaller{\geq}1$h & 5923 & $\mathsmaller{\geq}1$h & - & - & 2311 & 3.8\\
\texttt{mnist\_6} & 1940 & 2752 & 1940 & 1300 & 2755 & $\mathsmaller{\geq}1$h & 5918 & $\mathsmaller{\geq}1$h & - & - & 2251 & 4.1\\
\texttt{mnist\_5} & 3312 & 219 & \textbf{3085} & 1942 & 4373 & $\mathsmaller{\geq}1$h & 5421 & $\mathsmaller{\geq}1$h & - & - & 3648 & 3.8\\
\texttt{mnist\_3} & 3485 & 2225 & 3485 & 1262 & 4900 & $\mathsmaller{\geq}1$h & 6131 & $\mathsmaller{\geq}1$h & - & - & 4367 & 4.9\\
\texttt{mnist\_2} & 3358 & 169 & \textbf{3116} & 2748 & 4289 & $\mathsmaller{\geq}1$h & 5958 & $\mathsmaller{\geq}1$h & - & - & 4326 & 3.1\\
\texttt{mnist\_4} & 3670 & 2476 & \textbf{3615} & 1605 & 5580 & $\mathsmaller{\geq}1$h & 5842 & $\mathsmaller{\geq}1$h & - & - & 4129 & 3.2\\
\texttt{mnist\_7} & 2793 & 52 & 2793 & 640 & 4546 & $\mathsmaller{\geq}1$h & 6265 & $\mathsmaller{\geq}1$h & - & - & 3218 & 3.9\\
\texttt{mnist\_1} & 2332 & 2248 & 2332 & 671 & 4551 & $\mathsmaller{\geq}1$h & 6742 & $\mathsmaller{\geq}1$h & - & - & 2501 & 3.6\\
\texttt{weather-aus} & \textbf{1749} & 2525 & 1750 & 1243 & - & - & 1752 & $\mathsmaller{\geq}1$h & - & - & 1761 & 20\\
\bottomrule
\end{tabular}

\end{normalsize}
\end{center}
\caption{\label{tab:all} Comparison with state of the art: depth 4}
\end{table}

\medskip

\begin{table}[htbp]
\begin{center}
\begin{normalsize}
\tabcolsep=4pt
\begin{tabular}{lccrrrrrrrrrrrr}
\toprule
\multirow{2}{*}{}& && \multicolumn{2}{c}{\budalg} & \multicolumn{2}{c}{\murtree} & \multicolumn{2}{c}{\dleight} & \multicolumn{2}{c}{\cp} & \multicolumn{2}{c}{binoct} & \multicolumn{2}{c}{\cart}\\
\cmidrule(rr){4-5}\cmidrule(rr){6-7}\cmidrule(rr){8-9}\cmidrule(rr){10-11}\cmidrule(rr){12-13}\cmidrule(rr){14-15}
&\multirow{1}{*}{$\#ex.$} & \multirow{1}{*}{\#feat.} &  \multicolumn{1}{c}{error} & \multicolumn{1}{c}{cpu} & \multicolumn{1}{c}{error} & \multicolumn{1}{c}{cpu} & \multicolumn{1}{c}{error} & \multicolumn{1}{c}{cpu} & \multicolumn{1}{c}{error} & \multicolumn{1}{c}{cpu} & \multicolumn{1}{c}{error} & \multicolumn{1}{c}{cpu} & \multicolumn{1}{c}{error} & \multicolumn{1}{c}{cpu} \\
\midrule

\texttt{adult\_discretized} & \multicolumn{1}{r}{30299} & \multicolumn{1}{r}{59}  & 4423 & 725$^*$ & 4423 & 794$^*$ & 4442 & $\mathsmaller{\geq}1$h & 4423 & $\mathsmaller{\geq}1$h & 7511 & 452 & 4728 & 0.08\\
\texttt{anneal} & \multicolumn{1}{r}{812} & \multicolumn{1}{r}{93}  & 70 & 44$^*$ & 70 & 148$^*$ & - & - & 75 & $\mathsmaller{\geq}1$h & 101 & $\mathsmaller{\geq}1$h & 123 & 0.00\\
\texttt{audiology} & \multicolumn{1}{r}{216} & \multicolumn{1}{r}{148}  & 0 & 0.00$^*$ & 0 & 0.02$^*$ & 0 & 0.05$^*$ & 0 & 7.0$^*$ & 1 & $\mathsmaller{\geq}1$h & 2 & 0.00\\
\texttt{australian-credit} & \multicolumn{1}{r}{653} & \multicolumn{1}{r}{125}  & 39 & 658$^*$ & 39 & 872$^*$ & - & - & 40 & $\mathsmaller{\geq}1$h & 93 & 3387 & 64 & 0.00\\
\texttt{bank} & \multicolumn{1}{r}{45211} & \multicolumn{1}{r}{9531}  & \textbf{4187} & 1152 & 4365 & 2093 & 4809 & $\mathsmaller{\geq}1$h & 5289 & $\mathsmaller{\geq}1$h & - & - & 4358 & 47\\
\texttt{breast-cancer} & \multicolumn{1}{r}{683} & \multicolumn{1}{r}{89}  & 6 & 725$^*$ & 6 & 72$^*$ & 6 & 438$^*$ & 6 & $\mathsmaller{\geq}1$h & 14 & $\mathsmaller{\geq}1$h & 16 & 0.00\\
\texttt{breast-wisconsin} & \multicolumn{1}{r}{683} & \multicolumn{1}{r}{120}  & 0 & 20$^*$ & 0 & 72$^*$ & - & - & 1 & $\mathsmaller{\geq}1$h & 16 & $\mathsmaller{\geq}1$h & 13 & 0.00\\
\texttt{car} & \multicolumn{1}{r}{1728} & \multicolumn{1}{r}{21}  & 86 & 2.4$^*$ & 86 & 1.2$^*$ & 86 & 2.7$^*$ & 86 & 21$^*$ & 138 & $\mathsmaller{\geq}1$h & 106 & 0.01\\
\texttt{compas\_discretized} & \multicolumn{1}{r}{6167} & \multicolumn{1}{r}{25}  & 1919 & 1.1$^*$ & 1919 & 11$^*$ & 1919 & 26$^*$ & 1919 & 77$^*$ & 1952 & $\mathsmaller{\geq}1$h & 1968 & 0.01\\
\texttt{diabetes} & \multicolumn{1}{r}{768} & \multicolumn{1}{r}{112}  & 106 & 312$^*$ & 106 & 920$^*$ & - & - & 107 & $\mathsmaller{\geq}1$h & 189 & 3174 & 141 & 0.00\\
\texttt{forest-fires} & \multicolumn{1}{r}{517} & \multicolumn{1}{r}{989}  & 156 & 777 & \textbf{149} & 2977 & - & - & 172 & $\mathsmaller{\geq}1$h & 270 & 107 & 177 & 0.01\\
\texttt{german-credit} & \multicolumn{1}{r}{1000} & \multicolumn{1}{r}{112}  & 161 & 2741$^*$ & 161 & 973$^*$ & - & - & 161 & $\mathsmaller{\geq}1$h & 294 & 515 & 209 & 0.01\\
\texttt{heart-cleveland} & \multicolumn{1}{r}{296} & \multicolumn{1}{r}{95}  & 7 & 93$^*$ & 7 & 101$^*$ & - & - & 7 & $\mathsmaller{\geq}1$h & 26 & $\mathsmaller{\geq}1$h & 26 & 0.00\\
\texttt{hepatitis} & \multicolumn{1}{r}{137} & \multicolumn{1}{r}{68}  & 0 & 0.05$^*$ & 0 & 0.18$^*$ & 0 & 71$^*$ & 0 & 12$^*$ & 6 & $\mathsmaller{\geq}1$h & 8 & 0.00\\
\texttt{hypothyroid} & \multicolumn{1}{r}{3247} & \multicolumn{1}{r}{88}  & 44 & 87$^*$ & 44 & 343$^*$ & - & - & 45 & $\mathsmaller{\geq}1$h & 134 & 420 & 50 & 0.01\\
\texttt{ionosphere} & \multicolumn{1}{r}{351} & \multicolumn{1}{r}{445}  & 0 & 506$^*$ & 0 & 1340$^*$ & - & - & 4 & $\mathsmaller{\geq}1$h & 66 & 505 & 17 & 0.01\\
\texttt{kr-vs-kp} & \multicolumn{1}{r}{3196} & \multicolumn{1}{r}{73}  & 81 & 65$^*$ & 81 & 150$^*$ & - & - & 81 & $\mathsmaller{\geq}1$h & 1527 & 564 & 189 & 0.01\\
\texttt{letter} & \multicolumn{1}{r}{20000} & \multicolumn{1}{r}{224}  & \textbf{168} & 3082 & 190 & 549 & 352 & $\mathsmaller{\geq}1$h & 813 & $\mathsmaller{\geq}1$h & - & - & 335 & 0.32\\
\texttt{lymph} & \multicolumn{1}{r}{148} & \multicolumn{1}{r}{68}  & 0 & 0.00$^*$ & 0 & 0.00$^*$ & 0 & 14$^*$ & 0 & 2.7$^*$ & 7 & $\mathsmaller{\geq}1$h & 4 & 0.00\\
\texttt{mnist\_0} & \multicolumn{1}{r}{60000} & \multicolumn{1}{r}{784}  & \textbf{1714} & 284 & 2066 & 2149 & 3319 & $\mathsmaller{\geq}1$h & 5923 & $\mathsmaller{\geq}1$h & - & - & 2021 & 4.5\\
\texttt{mnist\_1} & \multicolumn{1}{r}{60000} & \multicolumn{1}{r}{784}  & \textbf{1585} & 3111 & 1790 & 993 & 4029 & $\mathsmaller{\geq}1$h & 6742 & $\mathsmaller{\geq}1$h & - & - & 1965 & 3.6\\
\texttt{mnist\_2} & \multicolumn{1}{r}{60000} & \multicolumn{1}{r}{784}  & 3118 & 3230 & \textbf{2963} & 2671 & 4026 & $\mathsmaller{\geq}1$h & 5958 & $\mathsmaller{\geq}1$h & - & - & 3676 & 3.9\\
\texttt{mnist\_3} & \multicolumn{1}{r}{60000} & \multicolumn{1}{r}{784}  & \textbf{2893} & 1936 & 3184 & 398 & 4900 & $\mathsmaller{\geq}1$h & 6131 & $\mathsmaller{\geq}1$h & - & - & 3768 & 6.0\\
\texttt{mnist\_4} & \multicolumn{1}{r}{60000} & \multicolumn{1}{r}{784}  & \textbf{2864} & 708 & 3164 & 107 & 5580 & $\mathsmaller{\geq}1$h & 5842 & $\mathsmaller{\geq}1$h & - & - & 3619 & 4.5\\
\texttt{mnist\_5} & \multicolumn{1}{r}{60000} & \multicolumn{1}{r}{784}  & \textbf{3138} & 2411 & 3163 & 2007 & 4376 & $\mathsmaller{\geq}1$h & 5421 & $\mathsmaller{\geq}1$h & - & - & 3479 & 5.8\\
\texttt{mnist\_6} & \multicolumn{1}{r}{60000} & \multicolumn{1}{r}{784}  & \textbf{1485} & 2097 & 1653 & 646 & 2753 & $\mathsmaller{\geq}1$h & 5918 & $\mathsmaller{\geq}1$h & - & - & 1900 & 4.4\\
\texttt{mnist\_7} & \multicolumn{1}{r}{60000} & \multicolumn{1}{r}{784}  & 2532 & 1793 & \textbf{2464} & 2363 & 4542 & $\mathsmaller{\geq}1$h & 6265 & $\mathsmaller{\geq}1$h & - & - & 2848 & 6.7\\
\texttt{mnist\_8} & \multicolumn{1}{r}{60000} & \multicolumn{1}{r}{784}  & \textbf{2547} & 2847 & 2818 & 1149 & 4609 & $\mathsmaller{\geq}1$h & 5851 & $\mathsmaller{\geq}1$h & - & - & 3172 & 6.3\\
\texttt{mnist\_9} & \multicolumn{1}{r}{60000} & \multicolumn{1}{r}{784}  & \textbf{3352} & 1695 & 3521 & 1368 & 5252 & $\mathsmaller{\geq}1$h & 5949 & $\mathsmaller{\geq}1$h & - & - & 3830 & 6.8\\
\texttt{mushroom} & \multicolumn{1}{r}{8124} & \multicolumn{1}{r}{119}  & 0 & 0.00$^*$ & 0 & 0.03$^*$ & 0 & 36$^*$ & 0 & 0.10$^*$ & 4208 & 437 & 3 & 0.03\\
\texttt{pendigits} & \multicolumn{1}{r}{7494} & \multicolumn{1}{r}{216}  & 0 & 284$^*$ & 0 & 1295$^*$ & - & - & 780 & $\mathsmaller{\geq}1$h & 780 & 618 & 11 & 0.07\\
\texttt{primary-tumor} & \multicolumn{1}{r}{336} & \multicolumn{1}{r}{31}  & 26 & 0.38$^*$ & 26 & 1.5$^*$ & 26 & 24$^*$ & 26 & 103$^*$ & 34 & $\mathsmaller{\geq}1$h & 35 & 0.00\\
\texttt{segment} & \multicolumn{1}{r}{2310} & \multicolumn{1}{r}{235}  & 0 & 0.00$^*$ & 0 & 0.02$^*$ & 0 & 1.0$^*$ & 0 & 2.0$^*$ & 330 & 1053 & 1 & 0.01\\
\texttt{soybean} & \multicolumn{1}{r}{630} & \multicolumn{1}{r}{50}  & 8 & 20$^*$ & 8 & 7.6$^*$ & 8 & 63$^*$ & 8 & 752$^*$ & 14 & $\mathsmaller{\geq}1$h & 23 & 0.00\\
\texttt{splice-1} & \multicolumn{1}{r}{3190} & \multicolumn{1}{r}{287}  & 101 & 24 & \textbf{100} & 3308 & - & - & 1535 & $\mathsmaller{\geq}1$h & 1655 & 542 & 117 & 0.04\\
\texttt{surgical-deepnet} & \multicolumn{1}{r}{14635} & \multicolumn{1}{r}{6047}  & \textbf{2131} & 2168 & 2337 & 400 & - & - & 3690 & $\mathsmaller{\geq}1$h & - & - & 2245 & 8.4\\
\texttt{taiwan\_binarised} & \multicolumn{1}{r}{30000} & \multicolumn{1}{r}{205}  & \textbf{5200} & 105 & 5261 & 38 & 5412 & $\mathsmaller{\geq}1$h & 6636 & $\mathsmaller{\geq}1$h & - & - & 5280 & 0.37\\
\texttt{tic-tac-toe} & \multicolumn{1}{r}{958} & \multicolumn{1}{r}{27}  & 63 & 10$^*$ & 63 & 2.3$^*$ & 63 & 14$^*$ & 63 & 89$^*$ & 125 & $\mathsmaller{\geq}1$h & 78 & 0.00\\
\texttt{titanic} & \multicolumn{1}{r}{887} & \multicolumn{1}{r}{333}  & 95 & 1428 & 95 & 1371 & - & - & 342 & $\mathsmaller{\geq}1$h & 342 & 83 & 130 & 0.01\\
\texttt{vehicle} & \multicolumn{1}{r}{846} & \multicolumn{1}{r}{252}  & 1 & 690 & 1 & 1540 & - & - & 218 & $\mathsmaller{\geq}1$h & 218 & 1780 & 23 & 0.01\\
\texttt{vote} & \multicolumn{1}{r}{435} & \multicolumn{1}{r}{48}  & 1 & 24$^*$ & 1 & 6.1$^*$ & 1 & 45$^*$ & 1 & 522$^*$ & 8 & $\mathsmaller{\geq}1$h & 6 & 0.00\\
\texttt{weather-aus} & \multicolumn{1}{r}{142193} & \multicolumn{1}{r}{4759}  & 1735 & 419 & 1735 & 1907 & - & - & 1761 & $\mathsmaller{\geq}1$h & - & - & 1751 & 26\\
\texttt{wine1} & \multicolumn{1}{r}{178} & \multicolumn{1}{r}{1276}  & 33 & 1154 & 33 & 287 & - & - & 38 & $\mathsmaller{\geq}1$h & 58 & 1261 & 39 & 0.01\\
\texttt{wine2} & \multicolumn{1}{r}{178} & \multicolumn{1}{r}{1276}  & 39 & 411 & \textbf{37} & 3400 & - & - & 42 & $\mathsmaller{\geq}1$h & 71 & 638 & 44 & 0.01\\
\texttt{wine3} & \multicolumn{1}{r}{178} & \multicolumn{1}{r}{1276}  & 25 & 17 & 25 & 25 & - & - & 28 & $\mathsmaller{\geq}1$h & 48 & 1054 & 30 & 0.01\\
\texttt{yeast} & \multicolumn{1}{r}{1484} & \multicolumn{1}{r}{89}  & 313 & 139$^*$ & 313 & 558$^*$ & - & - & 315 & $\mathsmaller{\geq}1$h & 463 & 1438 & 367 & 0.01\\
\bottomrule
\end{tabular}

\end{normalsize}
\end{center}
\caption{\label{tab:all} Comparison with state of the art: depth 5}
\end{table}

\medskip

\begin{table}[htbp]
\begin{center}
\begin{normalsize}
\tabcolsep=4pt
\begin{tabular}{lccrrrrrrrrrrr}
\toprule
& && \multicolumn{3}{c}{\budalg} & \multicolumn{3}{c}{\murtree} & \multicolumn{3}{c}{\dleight} & \multicolumn{2}{c}{\cart}\\
\cmidrule(rr){4-6}\cmidrule(rr){7-9}\cmidrule(rr){10-12}\cmidrule(rr){13-14}
&\multirow{1}{*}{$\#ex.$} & \multirow{1}{*}{\#feat.} &  \multicolumn{1}{c}{error} & \multicolumn{1}{c}{cpu} & \multicolumn{1}{c}{opt.} & \multicolumn{1}{c}{error} & \multicolumn{1}{c}{cpu} & \multicolumn{1}{c}{opt.} & \multicolumn{1}{c}{error} & \multicolumn{1}{c}{cpu} & \multicolumn{1}{c}{opt.} & \multicolumn{1}{c}{error} & \multicolumn{1}{c}{cpu} \\
\midrule

\texttt{adult\_discretized} & \multicolumn{1}{r}{30299} & \multicolumn{1}{r}{59}  & 4281 & 1326.0 & 0 & 4281 & 196.2 & 0 & - & - & 0 & 4532 & \textbf{0.1}\\
\texttt{anneal} & \multicolumn{1}{r}{812} & \multicolumn{1}{r}{93}  & \textbf{51} & 1330.0 & 1 & 53 & 2492.3 & 0 & - & - & 0 & 106 & \textbf{0.0}\\
\texttt{audiology} & \multicolumn{1}{r}{216} & \multicolumn{1}{r}{148}  & 0 & \textbf{0.0} & 1 & 0 & 0.0 & 1 & 0 & 0.0 & 1 & 1 & 0.0\\
\texttt{australian-credit} & \multicolumn{1}{r}{653} & \multicolumn{1}{r}{125}  & 15 & 341.9 & 0 & 15 & 748.7 & 0 & - & - & 0 & 56 & \textbf{0.0}\\
\texttt{bank} & \multicolumn{1}{r}{45211} & \multicolumn{1}{r}{9531}  & \textbf{4046} & 338.6 & 0 & 4270 & 2970.7 & 0 & 4810 & 3604.4 & 0 & 4245 & \textbf{42.9}\\
\texttt{breast-cancer} & \multicolumn{1}{r}{683} & \multicolumn{1}{r}{89}  & 1 & 3328.4 & 0 & 1 & 2338.5 & 1 & - & - & 0 & 13 & \textbf{0.0}\\
\texttt{breast-wisconsin} & \multicolumn{1}{r}{683} & \multicolumn{1}{r}{120}  & 0 & 5.9 & 1 & 0 & 59.7 & 1 & - & - & 0 & 7 & \textbf{0.0}\\
\texttt{car} & \multicolumn{1}{r}{1728} & \multicolumn{1}{r}{21}  & 36 & 27.3 & 1 & 36 & 4.2 & 1 & 36 & 7.9 & 1 & 90 & \textbf{0.0}\\
\texttt{compas\_discretized} & \multicolumn{1}{r}{6167} & \multicolumn{1}{r}{25}  & 1887 & 17.2 & 1 & 1887 & 67.7 & 1 & 1887 & 160.8 & 1 & 1955 & \textbf{0.0}\\
\texttt{diabetes} & \multicolumn{1}{r}{768} & \multicolumn{1}{r}{112}  & \textbf{60} & 2705.8 & 0 & 62 & 694.8 & 0 & - & - & 0 & 130 & \textbf{0.0}\\
\texttt{forest-fires} & \multicolumn{1}{r}{517} & \multicolumn{1}{r}{989}  & \textbf{132} & 1934.5 & 0 & 150 & 1464.2 & 0 & - & - & 0 & 171 & \textbf{0.0}\\
\texttt{german-credit} & \multicolumn{1}{r}{1000} & \multicolumn{1}{r}{112}  & 101 & 2883.3 & 0 & 101 & 2034.4 & 0 & - & - & 0 & 171 & \textbf{0.0}\\
\texttt{heart-cleveland} & \multicolumn{1}{r}{296} & \multicolumn{1}{r}{95}  & 0 & 0.0 & 1 & 0 & 0.2 & 1 & - & - & 0 & 15 & \textbf{0.0}\\
\texttt{hepatitis} & \multicolumn{1}{r}{137} & \multicolumn{1}{r}{68}  & 0 & \textbf{0.0} & 1 & 0 & 0.0 & 1 & 0 & 29.6 & 1 & 3 & 0.0\\
\texttt{hypothyroid} & \multicolumn{1}{r}{3247} & \multicolumn{1}{r}{88}  & 32 & 2390.9 & 1 & 32 & 3488.1 & 0 & - & - & 0 & 47 & \textbf{0.0}\\
\texttt{ionosphere} & \multicolumn{1}{r}{351} & \multicolumn{1}{r}{445}  & 0 & 4.4 & 1 & 0 & 24.4 & 1 & - & - & 0 & 11 & \textbf{0.0}\\
\texttt{kr-vs-kp} & \multicolumn{1}{r}{3196} & \multicolumn{1}{r}{73}  & 45 & 1694.0 & 1 & 45 & 2385.2 & 0 & - & - & 0 & 184 & \textbf{0.0}\\
\texttt{letter} & \multicolumn{1}{r}{20000} & \multicolumn{1}{r}{224}  & \textbf{118} & 2186.1 & 0 & 275 & 171.8 & 0 & 387 & 3600.0 & 0 & 217 & \textbf{0.3}\\
\texttt{lymph} & \multicolumn{1}{r}{148} & \multicolumn{1}{r}{68}  & 0 & \textbf{0.0} & 1 & 0 & 0.0 & 1 & 0 & 0.6 & 1 & 1 & 0.0\\
\texttt{mnist\_0} & \multicolumn{1}{r}{60000} & \multicolumn{1}{r}{784}  & \textbf{1468} & 2513.2 & 0 & 1885 & 3591.1 & 0 & 3319 & 3600.3 & 0 & 1781 & \textbf{5.4}\\
\texttt{mnist\_1} & \multicolumn{1}{r}{60000} & \multicolumn{1}{r}{784}  & \textbf{1167} & 1874.6 & 0 & 1778 & 3586.7 & 0 & 4551 & 3600.3 & 0 & 1542 & \textbf{5.1}\\
\texttt{mnist\_2} & \multicolumn{1}{r}{60000} & \multicolumn{1}{r}{784}  & \textbf{2519} & 229.7 & 0 & 2687 & 1160.7 & 0 & 4232 & 3600.3 & 0 & 2818 & \textbf{5.6}\\
\texttt{mnist\_3} & \multicolumn{1}{r}{60000} & \multicolumn{1}{r}{784}  & \textbf{2486} & 2792.6 & 0 & 2923 & 1923.2 & 0 & 4900 & 3600.3 & 0 & 2902 & \textbf{7.8}\\
\texttt{mnist\_4} & \multicolumn{1}{r}{60000} & \multicolumn{1}{r}{784}  & \textbf{2180} & 3375.4 & 0 & 2973 & 2185.0 & 0 & 5580 & 3600.3 & 0 & 2543 & \textbf{4.4}\\
\texttt{mnist\_5} & \multicolumn{1}{r}{60000} & \multicolumn{1}{r}{784}  & \textbf{2930} & 1759.2 & 0 & 3060 & 1215.9 & 0 & 4376 & 3600.3 & 0 & 3402 & \textbf{7.2}\\
\texttt{mnist\_6} & \multicolumn{1}{r}{60000} & \multicolumn{1}{r}{784}  & \textbf{1278} & 2110.7 & 0 & 1474 & 2711.4 & 0 & 2750 & 3600.3 & 0 & 1686 & \textbf{5.5}\\
\texttt{mnist\_7} & \multicolumn{1}{r}{60000} & \multicolumn{1}{r}{784}  & \textbf{2074} & 2011.6 & 0 & 2304 & 545.0 & 0 & 4543 & 3600.2 & 0 & 2163 & \textbf{5.2}\\
\texttt{mnist\_8} & \multicolumn{1}{r}{60000} & \multicolumn{1}{r}{784}  & \textbf{2060} & 806.1 & 0 & 3228 & 95.9 & 0 & 4656 & 3600.3 & 0 & 2633 & \textbf{6.1}\\
\texttt{mnist\_9} & \multicolumn{1}{r}{60000} & \multicolumn{1}{r}{784}  & \textbf{2879} & 2229.3 & 0 & 3327 & 1787.6 & 0 & 5252 & 3600.3 & 0 & 3366 & \textbf{6.6}\\
\texttt{mushroom} & \multicolumn{1}{r}{8124} & \multicolumn{1}{r}{119}  & 0 & \textbf{0.0} & 1 & 0 & 0.0 & 1 & 0 & 31.5 & 1 & 3 & 0.0\\
\texttt{pendigits} & \multicolumn{1}{r}{7494} & \multicolumn{1}{r}{216}  & 0 & \textbf{0.0} & 1 & 0 & 0.4 & 1 & - & - & 0 & 5 & 0.1\\
\texttt{primary-tumor} & \multicolumn{1}{r}{336} & \multicolumn{1}{r}{31}  & 18 & 3.1 & 1 & 18 & 23.1 & 1 & 18 & 138.3 & 1 & 28 & \textbf{0.0}\\
\texttt{segment} & \multicolumn{1}{r}{2310} & \multicolumn{1}{r}{235}  & 0 & \textbf{0.0} & 1 & 0 & 0.0 & 1 & 0 & 0.4 & 1 & 0 & 0.0\\
\texttt{soybean} & \multicolumn{1}{r}{630} & \multicolumn{1}{r}{50}  & 3 & 353.7 & 1 & 3 & 122.5 & 1 & 3 & 512.9 & 1 & 15 & \textbf{0.0}\\
\texttt{splice-1} & \multicolumn{1}{r}{3190} & \multicolumn{1}{r}{287}  & \textbf{68} & 3608.7 & 0 & 80 & 1723.1 & 0 & - & - & 0 & 87 & \textbf{0.0}\\
\texttt{surgical-deepnet} & \multicolumn{1}{r}{14635} & \multicolumn{1}{r}{6047}  & \textbf{1767} & 2342.7 & 0 & 2110 & 230.9 & 0 & - & - & 0 & 1969 & \textbf{7.4}\\
\texttt{taiwan\_binarised} & \multicolumn{1}{r}{30000} & \multicolumn{1}{r}{205}  & \textbf{5073} & 1472.9 & 0 & 5169 & 3396.2 & 0 & - & - & 0 & 5250 & \textbf{0.5}\\
\texttt{tic-tac-toe} & \multicolumn{1}{r}{958} & \multicolumn{1}{r}{27}  & 12 & 126.2 & 1 & 12 & 15.7 & 1 & 12 & 46.8 & 1 & 49 & \textbf{0.0}\\
\texttt{titanic} & \multicolumn{1}{r}{887} & \multicolumn{1}{r}{333}  & \textbf{78} & 1233.9 & 0 & 108 & 1509.4 & 0 & - & - & 0 & 119 & \textbf{0.0}\\
\texttt{vehicle} & \multicolumn{1}{r}{846} & \multicolumn{1}{r}{252}  & 0 & 0.1 & 1 & 0 & 0.6 & 1 & - & - & 0 & 9 & \textbf{0.0}\\
\texttt{vote} & \multicolumn{1}{r}{435} & \multicolumn{1}{r}{48}  & 0 & \textbf{0.0} & 1 & 0 & 0.0 & 1 & 0 & 0.6 & 1 & 2 & 0.0\\
\texttt{weather-aus} & \multicolumn{1}{r}{142193} & \multicolumn{1}{r}{4759}  & \textbf{1713} & 417.9 & 0 & 1736 & 813.2 & 0 & - & - & 0 & 1734 & \textbf{21.7}\\
\texttt{wine1} & \multicolumn{1}{r}{178} & \multicolumn{1}{r}{1276}  & \textbf{31} & 2113.1 & 0 & 32 & 1498.0 & 0 & - & - & 0 & 36 & \textbf{0.0}\\
\texttt{wine2} & \multicolumn{1}{r}{178} & \multicolumn{1}{r}{1276}  & 34 & 43.7 & 0 & 34 & 504.1 & 0 & - & - & 0 & 41 & \textbf{0.0}\\
\texttt{wine3} & \multicolumn{1}{r}{178} & \multicolumn{1}{r}{1276}  & 22 & 93.5 & 0 & 22 & 925.5 & 0 & - & - & 0 & 27 & \textbf{0.0}\\
\texttt{yeast} & \multicolumn{1}{r}{1484} & \multicolumn{1}{r}{89}  & \textbf{245} & 388.2 & 0 & 258 & 1736.6 & 0 & - & - & 0 & 346 & \textbf{0.0}\\
\bottomrule
\end{tabular}

\end{normalsize}
\end{center}
\caption{\label{tab:all} Comparison with state of the art: depth 6}
\end{table}

\medskip


When the maximum depth and number of feature is not too large, both algorithms are comparable, although \budalg is systematically faster. However, when the depth or the number of features grows, the best solution found by \dleight is often of much lower quality. In fact, in most cases, it reaches the time or memory limit without outputing a solution (the missing entries corresponds to \dleight reaching the 50GB memory limit). Notice that \budalg uses a tiny memory space (much lower than the size of the data set).



\begin{table}[htbp]
\begin{center}
\begin{normalsize}
\tabcolsep=5pt
\input{src/tables/cpusmall3.tex}
\end{normalsize}
\end{center}
\caption{\label{tab:d3} Comparison with state of the art on shallow trees (max depth=3)}
\end{table}

\begin{table}[htbp]
\begin{center}
\begin{normalsize}
\tabcolsep=5pt
\input{src/tables/cpusmall4.tex}
\end{normalsize}
\end{center}
\caption{\label{tab:d4} Comparison with state of the art on shallow trees (max depth=4)}
\end{table}

\begin{table}[htbp]
\begin{center}
\begin{normalsize}
\tabcolsep=5pt
\begin{tabular}{lccrrrrrrrrr}
\toprule
& && \multicolumn{3}{c}{\budalg} & \multicolumn{3}{c}{\murtree} & \multicolumn{3}{c}{\dleight}\\
\cmidrule(rr){4-6}\cmidrule(rr){7-9}\cmidrule(rr){10-12}
&\multirow{1}{*}{$\#ex.$} & \multirow{1}{*}{\#feat.} &  \multicolumn{1}{c}{error} & \multicolumn{1}{c}{time} & \multicolumn{1}{c}{opt.} & \multicolumn{1}{c}{error} & \multicolumn{1}{c}{time} & \multicolumn{1}{c}{opt.} & \multicolumn{1}{c}{error} & \multicolumn{1}{c}{time} & \multicolumn{1}{c}{opt.} \\
\midrule

\texttt{adult\_discretized} & \multicolumn{1}{r}{30299} & \multicolumn{1}{r}{59}  & \cellcolor{TealBlue!30}{4423} & 805.3 & \cellcolor{TealBlue!30}{1.00} & \cellcolor{TealBlue!30}{4423} & \cellcolor{TealBlue!30}{\textbf{490.1}} & \cellcolor{TealBlue!30}{1.00} & 4442 & 3600.0 & 0.00\\
\texttt{anneal} & \multicolumn{1}{r}{812} & \multicolumn{1}{r}{93}  & \cellcolor{TealBlue!30}{70} & \cellcolor{TealBlue!30}{\textbf{49.1}} & \cellcolor{TealBlue!30}{1.00} & \cellcolor{TealBlue!30}{70} & 253.7 & \cellcolor{TealBlue!30}{1.00} & - & - & -\\
\texttt{audiology} & \multicolumn{1}{r}{216} & \multicolumn{1}{r}{148}  & \cellcolor{TealBlue!30}{0} & \cellcolor{TealBlue!30}{\textbf{0.0}} & \cellcolor{TealBlue!30}{1.00} & \cellcolor{TealBlue!30}{0} & 0.0 & \cellcolor{TealBlue!30}{1.00} & \cellcolor{TealBlue!30}{0} & 0.0 & \cellcolor{TealBlue!30}{1.00}\\
\texttt{australian-credit} & \multicolumn{1}{r}{653} & \multicolumn{1}{r}{125}  & \cellcolor{TealBlue!30}{39} & \cellcolor{TealBlue!30}{\textbf{794.2}} & \cellcolor{TealBlue!30}{1.00} & \cellcolor{TealBlue!30}{39} & 1181.3 & \cellcolor{TealBlue!30}{1.00} & - & - & -\\
\texttt{bank-un} & \multicolumn{1}{r}{45211} & \multicolumn{1}{r}{9531}  & \cellcolor{TealBlue!30}{\textbf{4241}} & \cellcolor{TealBlue!30}{\textbf{906.9}} & \cellcolor{TealBlue!30}{0.00} & 4365 & 2094.4 & \cellcolor{TealBlue!30}{0.00} & 4809 & 3603.0 & \cellcolor{TealBlue!30}{0.00}\\
\texttt{breast-cancer-un} & \multicolumn{1}{r}{683} & \multicolumn{1}{r}{89}  & \cellcolor{TealBlue!30}{6} & 743.3 & \cellcolor{TealBlue!30}{1.00} & \cellcolor{TealBlue!30}{6} & \cellcolor{TealBlue!30}{\textbf{97.6}} & \cellcolor{TealBlue!30}{1.00} & \cellcolor{TealBlue!30}{6} & 438.0 & \cellcolor{TealBlue!30}{1.00}\\
\texttt{breast-wisconsin} & \multicolumn{1}{r}{683} & \multicolumn{1}{r}{120}  & \cellcolor{TealBlue!30}{0} & \cellcolor{TealBlue!30}{\textbf{20.6}} & \cellcolor{TealBlue!30}{1.00} & \cellcolor{TealBlue!30}{0} & 183.8 & \cellcolor{TealBlue!30}{1.00} & - & - & -\\
\texttt{car-un} & \multicolumn{1}{r}{1728} & \multicolumn{1}{r}{21}  & \cellcolor{TealBlue!30}{86} & 2.5 & \cellcolor{TealBlue!30}{1.00} & \cellcolor{TealBlue!30}{86} & \cellcolor{TealBlue!30}{\textbf{0.9}} & \cellcolor{TealBlue!30}{1.00} & \cellcolor{TealBlue!30}{86} & 2.7 & \cellcolor{TealBlue!30}{1.00}\\
\texttt{compas\_discretized} & \multicolumn{1}{r}{6167} & \multicolumn{1}{r}{25}  & \cellcolor{TealBlue!30}{1919} & \cellcolor{TealBlue!30}{\textbf{1.2}} & \cellcolor{TealBlue!30}{1.00} & \cellcolor{TealBlue!30}{1919} & 7.6 & \cellcolor{TealBlue!30}{1.00} & \cellcolor{TealBlue!30}{1919} & 26.4 & \cellcolor{TealBlue!30}{1.00}\\
\texttt{diabetes} & \multicolumn{1}{r}{768} & \multicolumn{1}{r}{112}  & \cellcolor{TealBlue!30}{106} & \cellcolor{TealBlue!30}{\textbf{349.6}} & \cellcolor{TealBlue!30}{1.00} & \cellcolor{TealBlue!30}{106} & 1231.1 & \cellcolor{TealBlue!30}{1.00} & - & - & -\\
\texttt{forest-fires-un} & \multicolumn{1}{r}{517} & \multicolumn{1}{r}{989}  & \cellcolor{TealBlue!30}{\textbf{156}} & \cellcolor{TealBlue!30}{\textbf{788.6}} & \cellcolor{TealBlue!30}{0.00} & 163 & 2622.0 & \cellcolor{TealBlue!30}{0.00} & - & - & -\\
\texttt{german-credit} & \multicolumn{1}{r}{1000} & \multicolumn{1}{r}{112}  & \cellcolor{TealBlue!30}{161} & 2822.9 & \cellcolor{TealBlue!30}{1.00} & \cellcolor{TealBlue!30}{161} & \cellcolor{TealBlue!30}{\textbf{1139.2}} & \cellcolor{TealBlue!30}{1.00} & - & - & -\\
\texttt{heart-cleveland} & \multicolumn{1}{r}{296} & \multicolumn{1}{r}{95}  & \cellcolor{TealBlue!30}{7} & \cellcolor{TealBlue!30}{\textbf{112.4}} & \cellcolor{TealBlue!30}{1.00} & \cellcolor{TealBlue!30}{7} & 165.3 & \cellcolor{TealBlue!30}{1.00} & - & - & -\\
\texttt{hepatitis} & \multicolumn{1}{r}{137} & \multicolumn{1}{r}{68}  & \cellcolor{TealBlue!30}{0} & \cellcolor{TealBlue!30}{\textbf{0.0}} & \cellcolor{TealBlue!30}{1.00} & \cellcolor{TealBlue!30}{0} & 0.3 & \cellcolor{TealBlue!30}{1.00} & \cellcolor{TealBlue!30}{0} & 71.4 & \cellcolor{TealBlue!30}{1.00}\\
\texttt{hypothyroid} & \multicolumn{1}{r}{3247} & \multicolumn{1}{r}{88}  & \cellcolor{TealBlue!30}{44} & \cellcolor{TealBlue!30}{\textbf{101.0}} & \cellcolor{TealBlue!30}{1.00} & \cellcolor{TealBlue!30}{44} & 505.2 & \cellcolor{TealBlue!30}{1.00} & - & - & -\\
\texttt{ionosphere} & \multicolumn{1}{r}{351} & \multicolumn{1}{r}{445}  & \cellcolor{TealBlue!30}{0} & \cellcolor{TealBlue!30}{\textbf{570.7}} & \cellcolor{TealBlue!30}{1.00} & \cellcolor{TealBlue!30}{0} & 3399.2 & \cellcolor{TealBlue!30}{1.00} & - & - & -\\
\texttt{kr-vs-kp} & \multicolumn{1}{r}{3196} & \multicolumn{1}{r}{73}  & \cellcolor{TealBlue!30}{81} & \cellcolor{TealBlue!30}{\textbf{67.6}} & \cellcolor{TealBlue!30}{1.00} & \cellcolor{TealBlue!30}{81} & 197.3 & \cellcolor{TealBlue!30}{1.00} & - & - & -\\
\texttt{letter} & \multicolumn{1}{r}{20000} & \multicolumn{1}{r}{224}  & \cellcolor{TealBlue!30}{\textbf{168}} & 3141.4 & \cellcolor{TealBlue!30}{0.00} & 190 & \cellcolor{TealBlue!30}{\textbf{795.4}} & \cellcolor{TealBlue!30}{0.00} & 352 & 3600.0 & \cellcolor{TealBlue!30}{0.00}\\
\texttt{lymph} & \multicolumn{1}{r}{148} & \multicolumn{1}{r}{68}  & \cellcolor{TealBlue!30}{0} & \cellcolor{TealBlue!30}{\textbf{0.0}} & \cellcolor{TealBlue!30}{1.00} & \cellcolor{TealBlue!30}{0} & 0.0 & \cellcolor{TealBlue!30}{1.00} & \cellcolor{TealBlue!30}{0} & 14.0 & \cellcolor{TealBlue!30}{1.00}\\
\texttt{mnist\_0} & \multicolumn{1}{r}{60000} & \multicolumn{1}{r}{784}  & \cellcolor{TealBlue!30}{\textbf{1714}} & \cellcolor{TealBlue!30}{\textbf{262.3}} & \cellcolor{TealBlue!30}{0.00} & 2066 & 2054.0 & \cellcolor{TealBlue!30}{0.00} & 3319 & 3600.2 & \cellcolor{TealBlue!30}{0.00}\\
\texttt{mnist\_1} & \multicolumn{1}{r}{60000} & \multicolumn{1}{r}{784}  & \cellcolor{TealBlue!30}{\textbf{1737}} & \cellcolor{TealBlue!30}{\textbf{823.2}} & \cellcolor{TealBlue!30}{0.00} & 1790 & 1110.4 & \cellcolor{TealBlue!30}{0.00} & 4029 & 3600.2 & \cellcolor{TealBlue!30}{0.00}\\
\texttt{mnist\_2} & \multicolumn{1}{r}{60000} & \multicolumn{1}{r}{784}  & 3365 & \cellcolor{TealBlue!30}{\textbf{2237.4}} & \cellcolor{TealBlue!30}{0.00} & \cellcolor{TealBlue!30}{\textbf{2963}} & 2544.7 & \cellcolor{TealBlue!30}{0.00} & 4026 & 3600.2 & \cellcolor{TealBlue!30}{0.00}\\
\texttt{mnist\_3} & \multicolumn{1}{r}{60000} & \multicolumn{1}{r}{784}  & 3316 & 1175.4 & \cellcolor{TealBlue!30}{0.00} & \cellcolor{TealBlue!30}{\textbf{3184}} & \cellcolor{TealBlue!30}{\textbf{369.2}} & \cellcolor{TealBlue!30}{0.00} & 4900 & 3600.3 & \cellcolor{TealBlue!30}{0.00}\\
\texttt{mnist\_4} & \multicolumn{1}{r}{60000} & \multicolumn{1}{r}{784}  & 3212 & 1789.8 & \cellcolor{TealBlue!30}{0.00} & \cellcolor{TealBlue!30}{\textbf{3164}} & \cellcolor{TealBlue!30}{\textbf{109.5}} & \cellcolor{TealBlue!30}{0.00} & 5580 & 3600.2 & \cellcolor{TealBlue!30}{0.00}\\
\texttt{mnist\_5} & \multicolumn{1}{r}{60000} & \multicolumn{1}{r}{784}  & 3399 & \cellcolor{TealBlue!30}{\textbf{478.8}} & \cellcolor{TealBlue!30}{0.00} & \cellcolor{TealBlue!30}{\textbf{3163}} & 2038.8 & \cellcolor{TealBlue!30}{0.00} & 4376 & 3600.2 & \cellcolor{TealBlue!30}{0.00}\\
\texttt{mnist\_6} & \multicolumn{1}{r}{60000} & \multicolumn{1}{r}{784}  & 1828 & 3581.8 & \cellcolor{TealBlue!30}{0.00} & \cellcolor{TealBlue!30}{\textbf{1653}} & \cellcolor{TealBlue!30}{\textbf{624.8}} & \cellcolor{TealBlue!30}{0.00} & 2753 & 3600.2 & \cellcolor{TealBlue!30}{0.00}\\
\texttt{mnist\_7} & \multicolumn{1}{r}{60000} & \multicolumn{1}{r}{784}  & 2699 & \cellcolor{TealBlue!30}{\textbf{254.8}} & \cellcolor{TealBlue!30}{0.00} & \cellcolor{TealBlue!30}{\textbf{2464}} & 2300.9 & \cellcolor{TealBlue!30}{0.00} & 4542 & 3600.2 & \cellcolor{TealBlue!30}{0.00}\\
\texttt{mnist\_8} & \multicolumn{1}{r}{60000} & \multicolumn{1}{r}{784}  & 2843 & 3174.8 & \cellcolor{TealBlue!30}{0.00} & \cellcolor{TealBlue!30}{\textbf{2818}} & \cellcolor{TealBlue!30}{\textbf{1150.3}} & \cellcolor{TealBlue!30}{0.00} & 4609 & 3600.2 & \cellcolor{TealBlue!30}{0.00}\\
\texttt{mnist\_9} & \multicolumn{1}{r}{60000} & \multicolumn{1}{r}{784}  & 3682 & \cellcolor{TealBlue!30}{\textbf{56.4}} & \cellcolor{TealBlue!30}{0.00} & \cellcolor{TealBlue!30}{\textbf{3521}} & 1159.3 & \cellcolor{TealBlue!30}{0.00} & 5252 & 3600.2 & \cellcolor{TealBlue!30}{0.00}\\
\texttt{mushroom} & \multicolumn{1}{r}{8124} & \multicolumn{1}{r}{119}  & \cellcolor{TealBlue!30}{0} & \cellcolor{TealBlue!30}{\textbf{0.0}} & \cellcolor{TealBlue!30}{1.00} & \cellcolor{TealBlue!30}{0} & 0.0 & \cellcolor{TealBlue!30}{1.00} & \cellcolor{TealBlue!30}{0} & 35.6 & \cellcolor{TealBlue!30}{1.00}\\
\texttt{pendigits} & \multicolumn{1}{r}{7494} & \multicolumn{1}{r}{216}  & \cellcolor{TealBlue!30}{0} & \cellcolor{TealBlue!30}{\textbf{313.6}} & \cellcolor{TealBlue!30}{1.00} & \cellcolor{TealBlue!30}{0} & 2860.1 & \cellcolor{TealBlue!30}{1.00} & - & - & -\\
\texttt{primary-tumor} & \multicolumn{1}{r}{336} & \multicolumn{1}{r}{31}  & \cellcolor{TealBlue!30}{26} & \cellcolor{TealBlue!30}{\textbf{0.4}} & \cellcolor{TealBlue!30}{1.00} & \cellcolor{TealBlue!30}{26} & 2.2 & \cellcolor{TealBlue!30}{1.00} & \cellcolor{TealBlue!30}{26} & 24.0 & \cellcolor{TealBlue!30}{1.00}\\
\texttt{segment} & \multicolumn{1}{r}{2310} & \multicolumn{1}{r}{235}  & \cellcolor{TealBlue!30}{0} & \cellcolor{TealBlue!30}{\textbf{0.0}} & \cellcolor{TealBlue!30}{1.00} & \cellcolor{TealBlue!30}{0} & 0.0 & \cellcolor{TealBlue!30}{1.00} & \cellcolor{TealBlue!30}{0} & 1.0 & \cellcolor{TealBlue!30}{1.00}\\
\texttt{soybean} & \multicolumn{1}{r}{630} & \multicolumn{1}{r}{50}  & \cellcolor{TealBlue!30}{8} & 21.1 & \cellcolor{TealBlue!30}{1.00} & \cellcolor{TealBlue!30}{8} & \cellcolor{TealBlue!30}{\textbf{9.4}} & \cellcolor{TealBlue!30}{1.00} & \cellcolor{TealBlue!30}{8} & 63.1 & \cellcolor{TealBlue!30}{1.00}\\
\texttt{splice-1} & \multicolumn{1}{r}{3190} & \multicolumn{1}{r}{287}  & 101 & \cellcolor{TealBlue!30}{\textbf{1728.2}} & 0.00 & \cellcolor{TealBlue!30}{\textbf{100}} & 3121.5 & \cellcolor{TealBlue!30}{\textbf{1.00}} & - & - & -\\
\texttt{surgical-deepnet-un} & \multicolumn{1}{r}{14635} & \multicolumn{1}{r}{6047}  & \cellcolor{TealBlue!30}{\textbf{2188}} & 2139.2 & \cellcolor{TealBlue!30}{0.00} & 2337 & \cellcolor{TealBlue!30}{\textbf{456.7}} & \cellcolor{TealBlue!30}{0.00} & - & - & -\\
\texttt{taiwan\_binarised} & \multicolumn{1}{r}{30000} & \multicolumn{1}{r}{205}  & \cellcolor{TealBlue!30}{\textbf{5200}} & 886.0 & \cellcolor{TealBlue!30}{0.00} & 5261 & \cellcolor{TealBlue!30}{\textbf{33.8}} & \cellcolor{TealBlue!30}{0.00} & 5412 & 3600.0 & \cellcolor{TealBlue!30}{0.00}\\
\texttt{tic-tac-toe} & \multicolumn{1}{r}{958} & \multicolumn{1}{r}{27}  & \cellcolor{TealBlue!30}{63} & 9.7 & \cellcolor{TealBlue!30}{1.00} & \cellcolor{TealBlue!30}{63} & \cellcolor{TealBlue!30}{\textbf{2.0}} & \cellcolor{TealBlue!30}{1.00} & \cellcolor{TealBlue!30}{63} & 14.0 & \cellcolor{TealBlue!30}{1.00}\\
\texttt{titanic-un} & \multicolumn{1}{r}{887} & \multicolumn{1}{r}{333}  & \cellcolor{TealBlue!30}{95} & 2149.8 & \cellcolor{TealBlue!30}{0.00} & \cellcolor{TealBlue!30}{95} & \cellcolor{TealBlue!30}{\textbf{2119.2}} & \cellcolor{TealBlue!30}{0.00} & - & - & -\\
\texttt{vehicle} & \multicolumn{1}{r}{846} & \multicolumn{1}{r}{252}  & \cellcolor{TealBlue!30}{\textbf{1}} & \cellcolor{TealBlue!30}{\textbf{751.5}} & \cellcolor{TealBlue!30}{0.00} & 2 & 2251.8 & \cellcolor{TealBlue!30}{0.00} & - & - & -\\
\texttt{vote} & \multicolumn{1}{r}{435} & \multicolumn{1}{r}{48}  & \cellcolor{TealBlue!30}{1} & 26.0 & \cellcolor{TealBlue!30}{1.00} & \cellcolor{TealBlue!30}{1} & \cellcolor{TealBlue!30}{\textbf{7.3}} & \cellcolor{TealBlue!30}{1.00} & \cellcolor{TealBlue!30}{1} & 45.0 & \cellcolor{TealBlue!30}{1.00}\\
\texttt{weather-aus-un} & \multicolumn{1}{r}{142193} & \multicolumn{1}{r}{4759}  & \cellcolor{TealBlue!30}{1735} & \cellcolor{TealBlue!30}{\textbf{390.9}} & \cellcolor{TealBlue!30}{0.00} & \cellcolor{TealBlue!30}{1735} & 1923.7 & \cellcolor{TealBlue!30}{0.00} & - & - & -\\
\texttt{wine1-un} & \multicolumn{1}{r}{178} & \multicolumn{1}{r}{1276}  & \cellcolor{TealBlue!30}{\textbf{33}} & \cellcolor{TealBlue!30}{\textbf{1133.2}} & \cellcolor{TealBlue!30}{0.00} & 34 & 1977.9 & \cellcolor{TealBlue!30}{0.00} & - & - & -\\
\texttt{wine2-un} & \multicolumn{1}{r}{178} & \multicolumn{1}{r}{1276}  & \cellcolor{TealBlue!30}{\textbf{39}} & 416.4 & \cellcolor{TealBlue!30}{0.00} & 40 & \cellcolor{TealBlue!30}{\textbf{133.3}} & \cellcolor{TealBlue!30}{0.00} & - & - & -\\
\texttt{wine3-un} & \multicolumn{1}{r}{178} & \multicolumn{1}{r}{1276}  & \cellcolor{TealBlue!30}{25} & \cellcolor{TealBlue!30}{\textbf{15.7}} & 0.00 & \cellcolor{TealBlue!30}{25} & 289.6 & \cellcolor{TealBlue!30}{\textbf{1.00}} & - & - & -\\
\texttt{yeast} & \multicolumn{1}{r}{1484} & \multicolumn{1}{r}{89}  & \cellcolor{TealBlue!30}{313} & \cellcolor{TealBlue!30}{\textbf{147.4}} & \cellcolor{TealBlue!30}{1.00} & \cellcolor{TealBlue!30}{313} & 749.4 & \cellcolor{TealBlue!30}{1.00} & - & - & -\\
\texttt{zoo-1} & \multicolumn{1}{r}{101} & \multicolumn{1}{r}{36}  & \cellcolor{TealBlue!30}{0} & \cellcolor{TealBlue!30}{\textbf{0.0}} & \cellcolor{TealBlue!30}{1.00} & \cellcolor{TealBlue!30}{0} & 0.0 & \cellcolor{TealBlue!30}{1.00} & \cellcolor{TealBlue!30}{0} & 0.0 & \cellcolor{TealBlue!30}{1.00}\\
\bottomrule
\end{tabular}

\end{normalsize}
\end{center}
\caption{\label{tab:d5} Comparison with state of the art on shallow trees (max depth=5)}
\end{table}







\begin{table}[htbp]
\begin{center}
\begin{normalsize}
\tabcolsep=5pt
\begin{tabular}{lccrrrrrrrr}
\toprule
& && \multicolumn{2}{c}{\cart} & \multicolumn{2}{c}{first sol.} & \multicolumn{2}{c}{$\leq 3$s} & \multicolumn{2}{c}{$\leq 1$h}\\
\cmidrule(rr){4-5}\cmidrule(rr){6-7}\cmidrule(rr){8-9}\cmidrule(rr){10-11}
&\multirow{1}{*}{$\#ex.$} & \multirow{1}{*}{\#feat.} &  \multicolumn{1}{c}{error} & \multicolumn{1}{c}{time} & \multicolumn{1}{c}{error} & \multicolumn{1}{c}{time} & \multicolumn{1}{c}{error} & \multicolumn{1}{c}{time} & \multicolumn{1}{c}{error} & \multicolumn{1}{c}{time} \\
\midrule

\texttt{anneal} & \multicolumn{1}{r}{812} & \multicolumn{1}{r}{47}  & 135 & 0.00 & 135 & \cellcolor{TealBlue!30}{\textbf{0.00}} & \cellcolor{TealBlue!30}{91} & 1.50 & \cellcolor{TealBlue!30}{91} & 1.50\\
\texttt{audiology} & \multicolumn{1}{r}{216} & \multicolumn{1}{r}{79}  & 3 & 0.00 & 3 & \cellcolor{TealBlue!30}{\textbf{0.00}} & \cellcolor{TealBlue!30}{1} & 0.15 & \cellcolor{TealBlue!30}{1} & 3.84\\
\texttt{australian-credit} & \multicolumn{1}{r}{653} & \multicolumn{1}{r}{73}  & 74 & 0.00 & 73 & \cellcolor{TealBlue!30}{\textbf{0.00}} & 60 & 0.10 & \cellcolor{TealBlue!30}{\textbf{56}} & 10.60\\
\texttt{breast-cancer-un} & \multicolumn{1}{r}{683} & \multicolumn{1}{r}{89}  & 21 & 0.00 & 21 & \cellcolor{TealBlue!30}{\textbf{0.00}} & \cellcolor{TealBlue!30}{16} & 0.33 & \cellcolor{TealBlue!30}{16} & 9.10\\
\texttt{breast-wisconsin} & \multicolumn{1}{r}{683} & \multicolumn{1}{r}{120}  & 16 & 0.00 & 16 & \cellcolor{TealBlue!30}{\textbf{0.00}} & \cellcolor{TealBlue!30}{7} & 1.26 & \cellcolor{TealBlue!30}{7} & 3.02\\
\texttt{car-un} & \multicolumn{1}{r}{1728} & \multicolumn{1}{r}{21}  & 202 & 0.00 & 178 & \cellcolor{TealBlue!30}{\textbf{0.00}} & \cellcolor{TealBlue!30}{136} & 0.15 & \cellcolor{TealBlue!30}{136} & 0.15\\
\texttt{diabetes} & \multicolumn{1}{r}{768} & \multicolumn{1}{r}{112}  & 166 & 0.00 & 159 & \cellcolor{TealBlue!30}{\textbf{0.00}} & \cellcolor{TealBlue!30}{137} & 0.06 & \cellcolor{TealBlue!30}{137} & 5.37\\
\texttt{forest-fires-un} & \multicolumn{1}{r}{517} & \multicolumn{1}{r}{989}  & 186 & 0.01 & 186 & \cellcolor{TealBlue!30}{\textbf{0.00}} & 179 & 1.69 & \cellcolor{TealBlue!30}{\textbf{173}} & 14.40\\
\texttt{german-credit} & \multicolumn{1}{r}{1000} & \multicolumn{1}{r}{110}  & 231 & 0.00 & 224 & \cellcolor{TealBlue!30}{\textbf{0.00}} & \cellcolor{TealBlue!30}{204} & 0.35 & \cellcolor{TealBlue!30}{204} & 27.10\\
\texttt{heart-cleveland} & \multicolumn{1}{r}{296} & \multicolumn{1}{r}{50}  & 38 & 0.00 & 36 & \cellcolor{TealBlue!30}{\textbf{0.00}} & \cellcolor{TealBlue!30}{25} & 3.00 & \cellcolor{TealBlue!30}{25} & 3.00\\
\texttt{hepatitis} & \multicolumn{1}{r}{137} & \multicolumn{1}{r}{68}  & 11 & 0.00 & 12 & \cellcolor{TealBlue!30}{\textbf{0.00}} & \cellcolor{TealBlue!30}{3} & 0.27 & \cellcolor{TealBlue!30}{3} & 0.27\\
\texttt{hypothyroid} & \multicolumn{1}{r}{3247} & \multicolumn{1}{r}{43}  & \cellcolor{TealBlue!30}{53} & 0.01 & \cellcolor{TealBlue!30}{53} & \cellcolor{TealBlue!30}{\textbf{0.00}} & \cellcolor{TealBlue!30}{53} & 0.06 & \cellcolor{TealBlue!30}{53} & 3.68\\
\texttt{ionosphere} & \multicolumn{1}{r}{351} & \multicolumn{1}{r}{444}  & 27 & 0.01 & 25 & \cellcolor{TealBlue!30}{\textbf{0.00}} & 13 & 0.64 & \cellcolor{TealBlue!30}{\textbf{7}} & 818.00\\
\texttt{kr-vs-kp} & \multicolumn{1}{r}{3196} & \multicolumn{1}{r}{37}  & 189 & 0.01 & 188 & \cellcolor{TealBlue!30}{\textbf{0.00}} & \cellcolor{TealBlue!30}{144} & 2.16 & \cellcolor{TealBlue!30}{144} & 2.16\\
\texttt{letter} & \multicolumn{1}{r}{20000} & \multicolumn{1}{r}{224}  & 462 & 0.28 & 443 & \cellcolor{TealBlue!30}{\textbf{0.00}} & 420 & 1.62 & \cellcolor{TealBlue!30}{\textbf{261}} & 944.00\\
\texttt{lymph} & \multicolumn{1}{r}{148} & \multicolumn{1}{r}{41}  & 10 & 0.00 & 9 & \cellcolor{TealBlue!30}{\textbf{0.00}} & \cellcolor{TealBlue!30}{3} & 0.64 & \cellcolor{TealBlue!30}{3} & 0.64\\
\texttt{mushroom} & \multicolumn{1}{r}{8124} & \multicolumn{1}{r}{91}  & 4 & 0.03 & 4 & \cellcolor{TealBlue!30}{\textbf{0.00}} & \cellcolor{TealBlue!30}{0} & 0.01 & \cellcolor{TealBlue!30}{0} & 0.01\\
\texttt{pendigits} & \multicolumn{1}{r}{7494} & \multicolumn{1}{r}{216}  & 25 & 0.09 & 22 & \cellcolor{TealBlue!30}{\textbf{0.00}} & 14 & 2.64 & \cellcolor{TealBlue!30}{\textbf{13}} & 237.00\\
\texttt{primary-tumor} & \multicolumn{1}{r}{336} & \multicolumn{1}{r}{16}  & 44 & 0.00 & 43 & \cellcolor{TealBlue!30}{\textbf{0.00}} & \cellcolor{TealBlue!30}{34} & 0.04 & \cellcolor{TealBlue!30}{34} & 0.04\\
\texttt{segment} & \multicolumn{1}{r}{2310} & \multicolumn{1}{r}{234}  & 1 & 0.01 & 1 & \cellcolor{TealBlue!30}{\textbf{0.00}} & \cellcolor{TealBlue!30}{0} & 0.01 & \cellcolor{TealBlue!30}{0} & 0.01\\
\texttt{soybean} & \multicolumn{1}{r}{630} & \multicolumn{1}{r}{34}  & 32 & 0.00 & 32 & \cellcolor{TealBlue!30}{\textbf{0.00}} & \cellcolor{TealBlue!30}{14} & 0.75 & \cellcolor{TealBlue!30}{14} & 0.75\\
\texttt{splice-1} & \multicolumn{1}{r}{3190} & \multicolumn{1}{r}{227}  & \cellcolor{TealBlue!30}{141} & 0.04 & \cellcolor{TealBlue!30}{141} & \cellcolor{TealBlue!30}{\textbf{0.00}} & \cellcolor{TealBlue!30}{141} & 0.03 & \cellcolor{TealBlue!30}{141} & 3180.00\\
\texttt{taiwan\_binarised} & \multicolumn{1}{r}{30000} & \multicolumn{1}{r}{198}  & 5306 & 0.43 & 5293 & \cellcolor{TealBlue!30}{\textbf{0.00}} & 5284 & 0.03 & \cellcolor{TealBlue!30}{\textbf{5273}} & 5.14\\
\texttt{tic-tac-toe} & \multicolumn{1}{r}{958} & \multicolumn{1}{r}{18}  & 150 & 0.00 & 150 & \cellcolor{TealBlue!30}{\textbf{0.00}} & \cellcolor{TealBlue!30}{137} & 0.36 & \cellcolor{TealBlue!30}{137} & 0.36\\
\texttt{vehicle} & \multicolumn{1}{r}{846} & \multicolumn{1}{r}{252}  & 28 & 0.01 & 28 & \cellcolor{TealBlue!30}{\textbf{0.00}} & 13 & 0.61 & \cellcolor{TealBlue!30}{\textbf{12}} & 74.50\\
\texttt{vote} & \multicolumn{1}{r}{435} & \multicolumn{1}{r}{32}  & 8 & 0.00 & 8 & \cellcolor{TealBlue!30}{\textbf{0.00}} & \cellcolor{TealBlue!30}{5} & 1.27 & \cellcolor{TealBlue!30}{5} & 1.27\\
\texttt{wine1-un} & \multicolumn{1}{r}{178} & \multicolumn{1}{r}{1276}  & 42 & 0.01 & 42 & \cellcolor{TealBlue!30}{\textbf{0.01}} & 39 & 0.31 & \cellcolor{TealBlue!30}{\textbf{37}} & 1570.00\\
\texttt{wine2-un} & \multicolumn{1}{r}{178} & \multicolumn{1}{r}{1276}  & 47 & 0.01 & 47 & \cellcolor{TealBlue!30}{\textbf{0.01}} & 46 & 0.01 & \cellcolor{TealBlue!30}{\textbf{43}} & 15.50\\
\texttt{wine3-un} & \multicolumn{1}{r}{178} & \multicolumn{1}{r}{1276}  & 32 & 0.01 & 32 & \cellcolor{TealBlue!30}{\textbf{0.01}} & 30 & 0.03 & \cellcolor{TealBlue!30}{\textbf{28}} & 31.20\\
\texttt{yeast} & \multicolumn{1}{r}{1484} & \multicolumn{1}{r}{89}  & 394 & 0.01 & 392 & \cellcolor{TealBlue!30}{\textbf{0.00}} & \cellcolor{TealBlue!30}{366} & 2.46 & \cellcolor{TealBlue!30}{366} & 3.26\\
\texttt{zoo-1} & \multicolumn{1}{r}{101} & \multicolumn{1}{r}{20}  & \cellcolor{TealBlue!30}{0} & 0.00 & \cellcolor{TealBlue!30}{0} & \cellcolor{TealBlue!30}{\textbf{0.00}} & \cellcolor{TealBlue!30}{0} & 0.00 & \cellcolor{TealBlue!30}{0} & 0.00\\
\bottomrule
\end{tabular}

\end{normalsize}
\end{center}
\caption{\label{tab:f4} Comparison with state of the art heuristics (max depth=4)}
\end{table}

\begin{table}[htbp]
\begin{center}
\begin{normalsize}
\tabcolsep=5pt
\begin{tabular}{lccrrrrrrrrr}
\toprule
& && \multicolumn{3}{c}{\cart} & \multicolumn{3}{c}{\greedy} & \multicolumn{3}{c}{\budalg}\\
\cmidrule(rr){4-6}\cmidrule(rr){7-9}\cmidrule(rr){10-12}
&\multirow{1}{*}{$\#ex.$} & \multirow{1}{*}{\#feat.} &  \multicolumn{1}{c}{acc.} & \multicolumn{1}{c}{error} & \multicolumn{1}{c}{time} & \multicolumn{1}{c}{acc.} & \multicolumn{1}{c}{error} & \multicolumn{1}{c}{time} & \multicolumn{1}{c}{acc.} & \multicolumn{1}{c}{error} & \multicolumn{1}{c}{time} \\
\midrule

\texttt{anneal} & \multicolumn{1}{r}{812} & \multicolumn{1}{r}{47}  & 0.882 & 96 & 0.0 & 0.861 & 113 & \cellcolor{TealBlue!30}{\textbf{0.0}} & \cellcolor{TealBlue!30}{\textbf{0.892}} & \cellcolor{TealBlue!30}{\textbf{88}} & 0.3\\
\texttt{audiology} & \multicolumn{1}{r}{216} & \multicolumn{1}{r}{79}  & \cellcolor{TealBlue!30}{1.000} & \cellcolor{TealBlue!30}{0} & 0.0 & 0.995 & 1 & \cellcolor{TealBlue!30}{\textbf{0.0}} & \cellcolor{TealBlue!30}{1.000} & \cellcolor{TealBlue!30}{0} & 0.0\\
\texttt{australian-credit} & \multicolumn{1}{r}{653} & \multicolumn{1}{r}{73}  & \cellcolor{TealBlue!30}{\textbf{0.934}} & \cellcolor{TealBlue!30}{\textbf{43}} & 0.0 & 0.891 & 71 & \cellcolor{TealBlue!30}{\textbf{0.0}} & 0.904 & 63 & 3.1\\
\texttt{breast-cancer-un} & \multicolumn{1}{r}{683} & \multicolumn{1}{r}{89}  & 0.988 & 8 & 0.0 & 0.985 & 10 & \cellcolor{TealBlue!30}{\textbf{0.0}} & \cellcolor{TealBlue!30}{\textbf{0.997}} & \cellcolor{TealBlue!30}{\textbf{2}} & 3.0\\
\texttt{breast-wisconsin} & \multicolumn{1}{r}{683} & \multicolumn{1}{r}{120}  & 0.994 & 4 & 0.0 & 0.982 & 12 & \cellcolor{TealBlue!30}{\textbf{0.0}} & \cellcolor{TealBlue!30}{\textbf{1.000}} & \cellcolor{TealBlue!30}{\textbf{0}} & 0.1\\
\texttt{car-un} & \multicolumn{1}{r}{1728} & \multicolumn{1}{r}{21}  & 0.971 & 50 & 0.0 & 0.950 & 87 & \cellcolor{TealBlue!30}{\textbf{0.0}} & \cellcolor{TealBlue!30}{\textbf{0.983}} & \cellcolor{TealBlue!30}{\textbf{29}} & 3.0\\
\texttt{diabetes} & \multicolumn{1}{r}{768} & \multicolumn{1}{r}{112}  & 0.870 & 100 & 0.0 & 0.850 & 115 & \cellcolor{TealBlue!30}{\textbf{0.0}} & \cellcolor{TealBlue!30}{\textbf{0.880}} & \cellcolor{TealBlue!30}{\textbf{92}} & 3.0\\
\texttt{forest-fires-un} & \multicolumn{1}{r}{517} & \multicolumn{1}{r}{989}  & 0.687 & 162 & 0.0 & 0.692 & 159 & \cellcolor{TealBlue!30}{\textbf{0.0}} & \cellcolor{TealBlue!30}{\textbf{0.702}} & \cellcolor{TealBlue!30}{\textbf{154}} & 3.0\\
\texttt{german-credit} & \multicolumn{1}{r}{1000} & \multicolumn{1}{r}{110}  & \cellcolor{TealBlue!30}{\textbf{0.851}} & \cellcolor{TealBlue!30}{\textbf{149}} & 0.0 & 0.789 & 211 & \cellcolor{TealBlue!30}{\textbf{0.0}} & 0.828 & 172 & 3.1\\
\texttt{heart-cleveland} & \multicolumn{1}{r}{296} & \multicolumn{1}{r}{50}  & 0.980 & 6 & 0.0 & 0.919 & 24 & \cellcolor{TealBlue!30}{\textbf{0.0}} & \cellcolor{TealBlue!30}{\textbf{0.983}} & \cellcolor{TealBlue!30}{\textbf{5}} & 3.0\\
\texttt{hepatitis} & \multicolumn{1}{r}{137} & \multicolumn{1}{r}{68}  & \cellcolor{TealBlue!30}{1.000} & \cellcolor{TealBlue!30}{0} & 0.0 & 0.978 & 3 & \cellcolor{TealBlue!30}{\textbf{0.0}} & \cellcolor{TealBlue!30}{1.000} & \cellcolor{TealBlue!30}{0} & 0.0\\
\texttt{hypothyroid} & \multicolumn{1}{r}{3247} & \multicolumn{1}{r}{43}  & \cellcolor{TealBlue!30}{\textbf{0.987}} & \cellcolor{TealBlue!30}{\textbf{42}} & 0.0 & 0.984 & 52 & \cellcolor{TealBlue!30}{\textbf{0.0}} & 0.984 & 52 & 3.1\\
\texttt{ionosphere} & \multicolumn{1}{r}{351} & \multicolumn{1}{r}{444}  & 0.980 & 7 & 0.0 & 0.974 & 9 & \cellcolor{TealBlue!30}{\textbf{0.0}} & \cellcolor{TealBlue!30}{\textbf{1.000}} & \cellcolor{TealBlue!30}{\textbf{0}} & 0.8\\
\texttt{kr-vs-kp} & \multicolumn{1}{r}{3196} & \multicolumn{1}{r}{37}  & 0.968 & 103 & 0.0 & 0.948 & 166 & \cellcolor{TealBlue!30}{\textbf{0.0}} & \cellcolor{TealBlue!30}{\textbf{0.980}} & \cellcolor{TealBlue!30}{\textbf{64}} & 3.1\\
\texttt{letter} & \multicolumn{1}{r}{20000} & \multicolumn{1}{r}{224}  & \cellcolor{TealBlue!30}{\textbf{0.992}} & \cellcolor{TealBlue!30}{\textbf{153}} & 0.6 & 0.973 & 539 & \cellcolor{TealBlue!30}{\textbf{0.0}} & 0.973 & 534 & 3.6\\
\texttt{lymph} & \multicolumn{1}{r}{148} & \multicolumn{1}{r}{41}  & \cellcolor{TealBlue!30}{1.000} & \cellcolor{TealBlue!30}{0} & 0.0 & 0.946 & 8 & \cellcolor{TealBlue!30}{\textbf{0.0}} & \cellcolor{TealBlue!30}{1.000} & \cellcolor{TealBlue!30}{0} & 0.0\\
\texttt{mushroom} & \multicolumn{1}{r}{8124} & \multicolumn{1}{r}{91}  & \cellcolor{TealBlue!30}{1.000} & \cellcolor{TealBlue!30}{0} & 0.0 & 1.000 & 4 & \cellcolor{TealBlue!30}{\textbf{0.0}} & \cellcolor{TealBlue!30}{1.000} & \cellcolor{TealBlue!30}{0} & 0.0\\
\texttt{pendigits} & \multicolumn{1}{r}{7494} & \multicolumn{1}{r}{216}  & \cellcolor{TealBlue!30}{\textbf{1.000}} & \cellcolor{TealBlue!30}{\textbf{1}} & 0.1 & 0.998 & 14 & \cellcolor{TealBlue!30}{\textbf{0.0}} & 0.999 & 5 & 3.1\\
\texttt{primary-tumor} & \multicolumn{1}{r}{336} & \multicolumn{1}{r}{16}  & 0.923 & 26 & 0.0 & 0.923 & 26 & \cellcolor{TealBlue!30}{\textbf{0.0}} & \cellcolor{TealBlue!30}{\textbf{0.946}} & \cellcolor{TealBlue!30}{\textbf{18}} & 3.0\\
\texttt{segment} & \multicolumn{1}{r}{2310} & \multicolumn{1}{r}{234}  & \cellcolor{TealBlue!30}{1.000} & \cellcolor{TealBlue!30}{0} & 0.0 & \cellcolor{TealBlue!30}{1.000} & \cellcolor{TealBlue!30}{0} & \cellcolor{TealBlue!30}{\textbf{0.0}} & \cellcolor{TealBlue!30}{1.000} & \cellcolor{TealBlue!30}{0} & 0.0\\
\texttt{soybean} & \multicolumn{1}{r}{630} & \multicolumn{1}{r}{34}  & 0.983 & 11 & 0.0 & 0.954 & 29 & \cellcolor{TealBlue!30}{\textbf{0.0}} & \cellcolor{TealBlue!30}{\textbf{0.984}} & \cellcolor{TealBlue!30}{\textbf{10}} & 3.0\\
\texttt{splice-1} & \multicolumn{1}{r}{3190} & \multicolumn{1}{r}{227}  & \cellcolor{TealBlue!30}{\textbf{0.982}} & \cellcolor{TealBlue!30}{\textbf{58}} & 0.0 & 0.937 & 202 & \cellcolor{TealBlue!30}{\textbf{0.0}} & 0.943 & 182 & 3.0\\
\texttt{taiwan\_binarised} & \multicolumn{1}{r}{30000} & \multicolumn{1}{r}{198}  & \cellcolor{TealBlue!30}{\textbf{0.828}} & \cellcolor{TealBlue!30}{\textbf{5161}} & 0.7 & 0.826 & 5228 & \cellcolor{TealBlue!30}{\textbf{0.1}} & 0.827 & 5192 & 12.2\\
\texttt{tic-tac-toe} & \multicolumn{1}{r}{958} & \multicolumn{1}{r}{18}  & 0.977 & 22 & 0.0 & 0.916 & 80 & \cellcolor{TealBlue!30}{\textbf{0.0}} & \cellcolor{TealBlue!30}{\textbf{0.990}} & \cellcolor{TealBlue!30}{\textbf{10}} & 3.0\\
\texttt{vehicle} & \multicolumn{1}{r}{846} & \multicolumn{1}{r}{252}  & \cellcolor{TealBlue!30}{\textbf{0.995}} & \cellcolor{TealBlue!30}{\textbf{4}} & 0.0 & 0.949 & 43 & \cellcolor{TealBlue!30}{\textbf{0.0}} & 0.989 & 9 & 3.2\\
\texttt{vote} & \multicolumn{1}{r}{435} & \multicolumn{1}{r}{32}  & 0.995 & 2 & 0.0 & 0.991 & 4 & \cellcolor{TealBlue!30}{\textbf{0.0}} & \cellcolor{TealBlue!30}{\textbf{1.000}} & \cellcolor{TealBlue!30}{\textbf{0}} & 0.0\\
\texttt{wine1-un} & \multicolumn{1}{r}{178} & \multicolumn{1}{r}{1276}  & 0.815 & 33 & 0.0 & \cellcolor{TealBlue!30}{0.831} & \cellcolor{TealBlue!30}{30} & \cellcolor{TealBlue!30}{\textbf{0.0}} & \cellcolor{TealBlue!30}{0.831} & \cellcolor{TealBlue!30}{30} & 3.0\\
\texttt{wine2-un} & \multicolumn{1}{r}{178} & \multicolumn{1}{r}{1276}  & 0.787 & 38 & 0.0 & 0.820 & 32 & \cellcolor{TealBlue!30}{\textbf{0.0}} & \cellcolor{TealBlue!30}{\textbf{0.826}} & \cellcolor{TealBlue!30}{\textbf{31}} & 3.2\\
\texttt{wine3-un} & \multicolumn{1}{r}{178} & \multicolumn{1}{r}{1276}  & 0.854 & 26 & 0.0 & 0.860 & 25 & \cellcolor{TealBlue!30}{\textbf{0.0}} & \cellcolor{TealBlue!30}{\textbf{0.865}} & \cellcolor{TealBlue!30}{\textbf{24}} & 3.3\\
\texttt{yeast} & \multicolumn{1}{r}{1484} & \multicolumn{1}{r}{89}  & \cellcolor{TealBlue!30}{\textbf{0.794}} & \cellcolor{TealBlue!30}{\textbf{306}} & 0.0 & 0.747 & 376 & \cellcolor{TealBlue!30}{\textbf{0.0}} & 0.763 & 351 & 3.0\\
\texttt{zoo-1} & \multicolumn{1}{r}{101} & \multicolumn{1}{r}{20}  & \cellcolor{TealBlue!30}{1.000} & \cellcolor{TealBlue!30}{0} & 0.0 & \cellcolor{TealBlue!30}{1.000} & \cellcolor{TealBlue!30}{0} & \cellcolor{TealBlue!30}{\textbf{0.0}} & \cellcolor{TealBlue!30}{1.000} & \cellcolor{TealBlue!30}{0} & 0.0\\
\bottomrule
\end{tabular}

\end{normalsize}
\end{center}
\caption{\label{tab:f7} Comparison with state of the art heuristics (max depth=7)}
\end{table}

\begin{table}[htbp]
\begin{center}
\begin{normalsize}
\tabcolsep=5pt
\begin{tabular}{lccrrrrrrrrr}
\toprule
& && \multicolumn{3}{c}{\cart} & \multicolumn{3}{c}{\greedy} & \multicolumn{3}{c}{\budalg}\\
\cmidrule(rr){4-6}\cmidrule(rr){7-9}\cmidrule(rr){10-12}
&\multirow{1}{*}{$\#ex.$} & \multirow{1}{*}{\#feat.} &  \multicolumn{1}{c}{acc.} & \multicolumn{1}{c}{error} & \multicolumn{1}{c}{time} & \multicolumn{1}{c}{acc.} & \multicolumn{1}{c}{error} & \multicolumn{1}{c}{time} & \multicolumn{1}{c}{acc.} & \multicolumn{1}{c}{error} & \multicolumn{1}{c}{time} \\
\midrule

\texttt{anneal} & \multicolumn{1}{r}{812} & \multicolumn{1}{r}{47}  & \cellcolor{TealBlue!30}{\textbf{0.927}} & \cellcolor{TealBlue!30}{\textbf{59}} & 0.0 & 0.906 & 76 & \cellcolor{TealBlue!30}{\textbf{0.0}} & 0.906 & 76 & 3.0\\
\texttt{audiology} & \multicolumn{1}{r}{216} & \multicolumn{1}{r}{79}  & \cellcolor{TealBlue!30}{1.000} & \cellcolor{TealBlue!30}{0} & 0.0 & \cellcolor{TealBlue!30}{1.000} & \cellcolor{TealBlue!30}{0} & \cellcolor{TealBlue!30}{\textbf{0.0}} & \cellcolor{TealBlue!30}{1.000} & \cellcolor{TealBlue!30}{0} & 0.0\\
\texttt{australian-credit} & \multicolumn{1}{r}{653} & \multicolumn{1}{r}{73}  & 0.980 & 13 & 0.0 & 0.982 & 12 & \cellcolor{TealBlue!30}{\textbf{0.0}} & \cellcolor{TealBlue!30}{\textbf{1.000}} & \cellcolor{TealBlue!30}{\textbf{0}} & 0.3\\
\texttt{breast-cancer-un} & \multicolumn{1}{r}{683} & \multicolumn{1}{r}{89}  & \cellcolor{TealBlue!30}{1.000} & \cellcolor{TealBlue!30}{0} & 0.0 & 0.991 & 6 & \cellcolor{TealBlue!30}{\textbf{0.0}} & \cellcolor{TealBlue!30}{1.000} & \cellcolor{TealBlue!30}{0} & 0.5\\
\texttt{breast-wisconsin} & \multicolumn{1}{r}{683} & \multicolumn{1}{r}{120}  & \cellcolor{TealBlue!30}{1.000} & \cellcolor{TealBlue!30}{0} & 0.0 & 0.999 & 1 & \cellcolor{TealBlue!30}{\textbf{0.0}} & \cellcolor{TealBlue!30}{1.000} & \cellcolor{TealBlue!30}{0} & 0.0\\
\texttt{car-un} & \multicolumn{1}{r}{1728} & \multicolumn{1}{r}{21}  & 0.994 & 11 & 0.0 & 0.994 & 11 & \cellcolor{TealBlue!30}{\textbf{0.0}} & \cellcolor{TealBlue!30}{\textbf{1.000}} & \cellcolor{TealBlue!30}{\textbf{0}} & 0.3\\
\texttt{diabetes} & \multicolumn{1}{r}{768} & \multicolumn{1}{r}{112}  & 0.954 & 35 & 0.0 & 0.951 & 38 & \cellcolor{TealBlue!30}{\textbf{0.0}} & \cellcolor{TealBlue!30}{\textbf{0.992}} & \cellcolor{TealBlue!30}{\textbf{6}} & 3.0\\
\texttt{forest-fires-un} & \multicolumn{1}{r}{517} & \multicolumn{1}{r}{989}  & \cellcolor{TealBlue!30}{\textbf{0.720}} & \cellcolor{TealBlue!30}{\textbf{145}} & 0.0 & 0.702 & 154 & \cellcolor{TealBlue!30}{\textbf{0.0}} & 0.704 & 153 & 3.0\\
\texttt{german-credit} & \multicolumn{1}{r}{1000} & \multicolumn{1}{r}{110}  & 0.934 & 66 & 0.0 & 0.915 & 85 & \cellcolor{TealBlue!30}{\textbf{0.0}} & \cellcolor{TealBlue!30}{\textbf{0.948}} & \cellcolor{TealBlue!30}{\textbf{52}} & 3.0\\
\texttt{heart-cleveland} & \multicolumn{1}{r}{296} & \multicolumn{1}{r}{50}  & \cellcolor{TealBlue!30}{1.000} & \cellcolor{TealBlue!30}{0} & 0.0 & 0.997 & 1 & \cellcolor{TealBlue!30}{\textbf{0.0}} & \cellcolor{TealBlue!30}{1.000} & \cellcolor{TealBlue!30}{0} & 0.0\\
\texttt{hepatitis} & \multicolumn{1}{r}{137} & \multicolumn{1}{r}{68}  & \cellcolor{TealBlue!30}{1.000} & \cellcolor{TealBlue!30}{0} & 0.0 & \cellcolor{TealBlue!30}{1.000} & \cellcolor{TealBlue!30}{0} & 0.0 & \cellcolor{TealBlue!30}{1.000} & \cellcolor{TealBlue!30}{0} & \cellcolor{TealBlue!30}{\textbf{0.0}}\\
\texttt{hypothyroid} & \multicolumn{1}{r}{3247} & \multicolumn{1}{r}{43}  & \cellcolor{TealBlue!30}{\textbf{0.990}} & \cellcolor{TealBlue!30}{\textbf{31}} & 0.0 & 0.990 & 32 & \cellcolor{TealBlue!30}{\textbf{0.0}} & 0.990 & 32 & 3.0\\
\texttt{ionosphere} & \multicolumn{1}{r}{351} & \multicolumn{1}{r}{444}  & \cellcolor{TealBlue!30}{1.000} & \cellcolor{TealBlue!30}{0} & 0.0 & \cellcolor{TealBlue!30}{1.000} & \cellcolor{TealBlue!30}{0} & \cellcolor{TealBlue!30}{\textbf{0.0}} & \cellcolor{TealBlue!30}{1.000} & \cellcolor{TealBlue!30}{0} & 0.0\\
\texttt{kr-vs-kp} & \multicolumn{1}{r}{3196} & \multicolumn{1}{r}{37}  & 0.996 & 12 & 0.0 & 0.996 & 12 & \cellcolor{TealBlue!30}{\textbf{0.0}} & \cellcolor{TealBlue!30}{\textbf{0.998}} & \cellcolor{TealBlue!30}{\textbf{7}} & 3.0\\
\texttt{letter} & \multicolumn{1}{r}{20000} & \multicolumn{1}{r}{224}  & \cellcolor{TealBlue!30}{\textbf{0.999}} & \cellcolor{TealBlue!30}{\textbf{20}} & 0.4 & 0.994 & 124 & \cellcolor{TealBlue!30}{\textbf{0.0}} & 0.995 & 106 & 3.1\\
\texttt{lymph} & \multicolumn{1}{r}{148} & \multicolumn{1}{r}{41}  & \cellcolor{TealBlue!30}{1.000} & \cellcolor{TealBlue!30}{0} & 0.0 & \cellcolor{TealBlue!30}{1.000} & \cellcolor{TealBlue!30}{0} & \cellcolor{TealBlue!30}{\textbf{0.0}} & \cellcolor{TealBlue!30}{1.000} & \cellcolor{TealBlue!30}{0} & 0.0\\
\texttt{mushroom} & \multicolumn{1}{r}{8124} & \multicolumn{1}{r}{91}  & \cellcolor{TealBlue!30}{1.000} & \cellcolor{TealBlue!30}{0} & 0.0 & \cellcolor{TealBlue!30}{1.000} & \cellcolor{TealBlue!30}{0} & \cellcolor{TealBlue!30}{\textbf{0.0}} & \cellcolor{TealBlue!30}{1.000} & \cellcolor{TealBlue!30}{0} & 0.0\\
\texttt{pendigits} & \multicolumn{1}{r}{7494} & \multicolumn{1}{r}{216}  & \cellcolor{TealBlue!30}{1.000} & \cellcolor{TealBlue!30}{0} & 0.1 & \cellcolor{TealBlue!30}{1.000} & \cellcolor{TealBlue!30}{0} & \cellcolor{TealBlue!30}{\textbf{0.0}} & \cellcolor{TealBlue!30}{1.000} & \cellcolor{TealBlue!30}{0} & 0.1\\
\texttt{primary-tumor} & \multicolumn{1}{r}{336} & \multicolumn{1}{r}{16}  & 0.940 & 20 & 0.0 & 0.943 & 19 & \cellcolor{TealBlue!30}{\textbf{0.0}} & \cellcolor{TealBlue!30}{\textbf{0.946}} & \cellcolor{TealBlue!30}{\textbf{18}} & 3.0\\
\texttt{segment} & \multicolumn{1}{r}{2310} & \multicolumn{1}{r}{234}  & \cellcolor{TealBlue!30}{1.000} & \cellcolor{TealBlue!30}{0} & 0.0 & \cellcolor{TealBlue!30}{1.000} & \cellcolor{TealBlue!30}{0} & \cellcolor{TealBlue!30}{\textbf{0.0}} & \cellcolor{TealBlue!30}{1.000} & \cellcolor{TealBlue!30}{0} & 0.0\\
\texttt{soybean} & \multicolumn{1}{r}{630} & \multicolumn{1}{r}{34}  & \cellcolor{TealBlue!30}{0.997} & \cellcolor{TealBlue!30}{2} & 0.0 & 0.989 & 7 & \cellcolor{TealBlue!30}{\textbf{0.0}} & \cellcolor{TealBlue!30}{0.997} & \cellcolor{TealBlue!30}{2} & 3.0\\
\texttt{splice-1} & \multicolumn{1}{r}{3190} & \multicolumn{1}{r}{227}  & 0.996 & 12 & 0.1 & 0.996 & 13 & \cellcolor{TealBlue!30}{\textbf{0.0}} & \cellcolor{TealBlue!30}{\textbf{0.997}} & \cellcolor{TealBlue!30}{\textbf{10}} & 3.1\\
\texttt{taiwan\_binarised} & \multicolumn{1}{r}{30000} & \multicolumn{1}{r}{198}  & \cellcolor{TealBlue!30}{\textbf{0.843}} & \cellcolor{TealBlue!30}{\textbf{4707}} & 0.8 & 0.841 & 4779 & \cellcolor{TealBlue!30}{\textbf{0.0}} & 0.842 & 4741 & 3.0\\
\texttt{tic-tac-toe} & \multicolumn{1}{r}{958} & \multicolumn{1}{r}{18}  & 0.994 & 6 & 0.0 & 0.994 & 6 & \cellcolor{TealBlue!30}{\textbf{0.0}} & \cellcolor{TealBlue!30}{\textbf{1.000}} & \cellcolor{TealBlue!30}{\textbf{0}} & 0.0\\
\texttt{vehicle} & \multicolumn{1}{r}{846} & \multicolumn{1}{r}{252}  & \cellcolor{TealBlue!30}{1.000} & \cellcolor{TealBlue!30}{0} & 0.0 & \cellcolor{TealBlue!30}{1.000} & \cellcolor{TealBlue!30}{0} & \cellcolor{TealBlue!30}{\textbf{0.0}} & \cellcolor{TealBlue!30}{1.000} & \cellcolor{TealBlue!30}{0} & 0.0\\
\texttt{vote} & \multicolumn{1}{r}{435} & \multicolumn{1}{r}{32}  & \cellcolor{TealBlue!30}{1.000} & \cellcolor{TealBlue!30}{0} & 0.0 & \cellcolor{TealBlue!30}{1.000} & \cellcolor{TealBlue!30}{0} & \cellcolor{TealBlue!30}{\textbf{0.0}} & \cellcolor{TealBlue!30}{1.000} & \cellcolor{TealBlue!30}{0} & 0.0\\
\texttt{wine1-un} & \multicolumn{1}{r}{178} & \multicolumn{1}{r}{1276}  & \cellcolor{TealBlue!30}{\textbf{0.860}} & \cellcolor{TealBlue!30}{\textbf{25}} & 0.0 & 0.831 & 30 & \cellcolor{TealBlue!30}{\textbf{0.0}} & 0.848 & 27 & 3.0\\
\texttt{wine2-un} & \multicolumn{1}{r}{178} & \multicolumn{1}{r}{1276}  & 0.837 & 29 & 0.0 & 0.837 & 29 & \cellcolor{TealBlue!30}{\textbf{0.0}} & \cellcolor{TealBlue!30}{\textbf{0.848}} & \cellcolor{TealBlue!30}{\textbf{27}} & 3.1\\
\texttt{wine3-un} & \multicolumn{1}{r}{178} & \multicolumn{1}{r}{1276}  & 0.916 & 15 & 0.0 & 0.916 & 15 & \cellcolor{TealBlue!30}{\textbf{0.0}} & \cellcolor{TealBlue!30}{\textbf{0.921}} & \cellcolor{TealBlue!30}{\textbf{14}} & 3.1\\
\texttt{yeast} & \multicolumn{1}{r}{1484} & \multicolumn{1}{r}{89}  & \cellcolor{TealBlue!30}{\textbf{0.875}} & \cellcolor{TealBlue!30}{\textbf{185}} & 0.0 & 0.871 & 192 & \cellcolor{TealBlue!30}{\textbf{0.0}} & 0.871 & 192 & 3.0\\
\texttt{zoo-1} & \multicolumn{1}{r}{101} & \multicolumn{1}{r}{20}  & \cellcolor{TealBlue!30}{1.000} & \cellcolor{TealBlue!30}{0} & 0.0 & \cellcolor{TealBlue!30}{1.000} & \cellcolor{TealBlue!30}{0} & \cellcolor{TealBlue!30}{\textbf{0.0}} & \cellcolor{TealBlue!30}{1.000} & \cellcolor{TealBlue!30}{0} & 0.0\\
\bottomrule
\end{tabular}

\end{normalsize}
\end{center}
\caption{\label{tab:f10} Comparison with state of the art heuristics (max depth=10)}
\end{table}


\subsection{Size stats}


\begin{table}[htbp]
\begin{center}
\begin{normalsize}
\tabcolsep=3pt
\begin{tabular}{lccrrrrrrrr}
\toprule
& && \multicolumn{8}{c}{\budalg}\\
\cmidrule(rr){4-11}
&\multirow{1}{*}{$\#ex.$} & \multirow{1}{*}{\#feat.} &  \multicolumn{1}{c}{error (f)} & \multicolumn{1}{c}{error (b)} & \multicolumn{1}{c}{depth (b)} & \multicolumn{1}{c}{size (b)} & \multicolumn{1}{c}{time (b)} & \multicolumn{1}{c}{opt} & \multicolumn{1}{c}{time (a)} & \multicolumn{1}{c}{search (a)} \\
\midrule

\texttt{anneal} & \multicolumn{1}{r}{812} & \multicolumn{1}{r}{47}  & \cellcolor{TealBlue!30}{\textbf{137}} & \cellcolor{TealBlue!30}{\textbf{112}} & \cellcolor{TealBlue!30}{\textbf{3}} & \cellcolor{TealBlue!30}{\textbf{15}} & \cellcolor{TealBlue!30}{\textbf{0.03}} & \cellcolor{TealBlue!30}{\textbf{1}} & \cellcolor{TealBlue!30}{\textbf{0.05}} & \cellcolor{TealBlue!30}{\textbf{6332}}\\
\texttt{audiology} & \multicolumn{1}{r}{216} & \multicolumn{1}{r}{79}  & \cellcolor{TealBlue!30}{\textbf{6}} & \cellcolor{TealBlue!30}{\textbf{5}} & \cellcolor{TealBlue!30}{\textbf{3}} & \cellcolor{TealBlue!30}{\textbf{11}} & \cellcolor{TealBlue!30}{\textbf{0.01}} & \cellcolor{TealBlue!30}{\textbf{1}} & \cellcolor{TealBlue!30}{\textbf{0.08}} & \cellcolor{TealBlue!30}{\textbf{14684}}\\
\texttt{australian-credit} & \multicolumn{1}{r}{653} & \multicolumn{1}{r}{73}  & \cellcolor{TealBlue!30}{\textbf{82}} & \cellcolor{TealBlue!30}{\textbf{73}} & \cellcolor{TealBlue!30}{\textbf{3}} & \cellcolor{TealBlue!30}{\textbf{15}} & \cellcolor{TealBlue!30}{\textbf{0.10}} & \cellcolor{TealBlue!30}{\textbf{1}} & \cellcolor{TealBlue!30}{\textbf{0.15}} & \cellcolor{TealBlue!30}{\textbf{18557}}\\
\texttt{breast-cancer-un} & \multicolumn{1}{r}{683} & \multicolumn{1}{r}{89}  & \cellcolor{TealBlue!30}{\textbf{28}} & \cellcolor{TealBlue!30}{\textbf{24}} & \cellcolor{TealBlue!30}{\textbf{3}} & \cellcolor{TealBlue!30}{\textbf{15}} & \cellcolor{TealBlue!30}{\textbf{0.12}} & \cellcolor{TealBlue!30}{\textbf{1}} & \cellcolor{TealBlue!30}{\textbf{0.13}} & \cellcolor{TealBlue!30}{\textbf{20272}}\\
\texttt{breast-wisconsin} & \multicolumn{1}{r}{683} & \multicolumn{1}{r}{120}  & \cellcolor{TealBlue!30}{\textbf{26}} & \cellcolor{TealBlue!30}{\textbf{15}} & \cellcolor{TealBlue!30}{\textbf{3}} & \cellcolor{TealBlue!30}{\textbf{15}} & \cellcolor{TealBlue!30}{\textbf{0.00}} & \cellcolor{TealBlue!30}{\textbf{1}} & \cellcolor{TealBlue!30}{\textbf{0.07}} & \cellcolor{TealBlue!30}{\textbf{10936}}\\
\texttt{car-un} & \multicolumn{1}{r}{1728} & \multicolumn{1}{r}{21}  & \cellcolor{TealBlue!30}{\textbf{202}} & \cellcolor{TealBlue!30}{\textbf{192}} & \cellcolor{TealBlue!30}{\textbf{3}} & \cellcolor{TealBlue!30}{\textbf{9}} & \cellcolor{TealBlue!30}{\textbf{0.00}} & \cellcolor{TealBlue!30}{\textbf{1}} & \cellcolor{TealBlue!30}{\textbf{0.01}} & \cellcolor{TealBlue!30}{\textbf{1502}}\\
\texttt{diabetes} & \multicolumn{1}{r}{768} & \multicolumn{1}{r}{112}  & \cellcolor{TealBlue!30}{\textbf{169}} & \cellcolor{TealBlue!30}{\textbf{162}} & \cellcolor{TealBlue!30}{\textbf{3}} & \cellcolor{TealBlue!30}{\textbf{15}} & \cellcolor{TealBlue!30}{\textbf{0.01}} & \cellcolor{TealBlue!30}{\textbf{1}} & \cellcolor{TealBlue!30}{\textbf{0.09}} & \cellcolor{TealBlue!30}{\textbf{11564}}\\
\texttt{forest-fires-un} & \multicolumn{1}{r}{517} & \multicolumn{1}{r}{989}  & \cellcolor{TealBlue!30}{\textbf{198}} & \cellcolor{TealBlue!30}{\textbf{193}} & \cellcolor{TealBlue!30}{\textbf{3}} & \cellcolor{TealBlue!30}{\textbf{15}} & \cellcolor{TealBlue!30}{\textbf{20.90}} & \cellcolor{TealBlue!30}{\textbf{1}} & \cellcolor{TealBlue!30}{\textbf{24.10}} & \cellcolor{TealBlue!30}{\textbf{888423}}\\
\texttt{german-credit} & \multicolumn{1}{r}{1000} & \multicolumn{1}{r}{110}  & \cellcolor{TealBlue!30}{\textbf{249}} & \cellcolor{TealBlue!30}{\textbf{236}} & \cellcolor{TealBlue!30}{\textbf{3}} & \cellcolor{TealBlue!30}{\textbf{15}} & \cellcolor{TealBlue!30}{\textbf{0.04}} & \cellcolor{TealBlue!30}{\textbf{1}} & \cellcolor{TealBlue!30}{\textbf{0.28}} & \cellcolor{TealBlue!30}{\textbf{27568}}\\
\texttt{heart-cleveland} & \multicolumn{1}{r}{296} & \multicolumn{1}{r}{50}  & \cellcolor{TealBlue!30}{\textbf{43}} & \cellcolor{TealBlue!30}{\textbf{41}} & \cellcolor{TealBlue!30}{\textbf{3}} & \cellcolor{TealBlue!30}{\textbf{15}} & \cellcolor{TealBlue!30}{\textbf{0.00}} & \cellcolor{TealBlue!30}{\textbf{1}} & \cellcolor{TealBlue!30}{\textbf{0.06}} & \cellcolor{TealBlue!30}{\textbf{10515}}\\
\texttt{hepatitis} & \multicolumn{1}{r}{137} & \multicolumn{1}{r}{68}  & \cellcolor{TealBlue!30}{\textbf{14}} & \cellcolor{TealBlue!30}{\textbf{10}} & \cellcolor{TealBlue!30}{\textbf{3}} & \cellcolor{TealBlue!30}{\textbf{15}} & \cellcolor{TealBlue!30}{\textbf{0.01}} & \cellcolor{TealBlue!30}{\textbf{1}} & \cellcolor{TealBlue!30}{\textbf{0.02}} & \cellcolor{TealBlue!30}{\textbf{3814}}\\
\texttt{hypothyroid} & \multicolumn{1}{r}{3247} & \multicolumn{1}{r}{43}  & \cellcolor{TealBlue!30}{\textbf{62}} & \cellcolor{TealBlue!30}{\textbf{61}} & \cellcolor{TealBlue!30}{\textbf{3}} & \cellcolor{TealBlue!30}{\textbf{13}} & \cellcolor{TealBlue!30}{\textbf{0.04}} & \cellcolor{TealBlue!30}{\textbf{1}} & \cellcolor{TealBlue!30}{\textbf{0.10}} & \cellcolor{TealBlue!30}{\textbf{6339}}\\
\texttt{ionosphere} & \multicolumn{1}{r}{351} & \multicolumn{1}{r}{444}  & \cellcolor{TealBlue!30}{\textbf{29}} & \cellcolor{TealBlue!30}{\textbf{22}} & \cellcolor{TealBlue!30}{\textbf{3}} & \cellcolor{TealBlue!30}{\textbf{15}} & \cellcolor{TealBlue!30}{\textbf{0.57}} & \cellcolor{TealBlue!30}{\textbf{1}} & \cellcolor{TealBlue!30}{\textbf{4.27}} & \cellcolor{TealBlue!30}{\textbf{182760}}\\
\texttt{kr-vs-kp} & \multicolumn{1}{r}{3196} & \multicolumn{1}{r}{37}  & \cellcolor{TealBlue!30}{\textbf{306}} & \cellcolor{TealBlue!30}{\textbf{198}} & \cellcolor{TealBlue!30}{\textbf{3}} & \cellcolor{TealBlue!30}{\textbf{11}} & \cellcolor{TealBlue!30}{\textbf{0.00}} & \cellcolor{TealBlue!30}{\textbf{1}} & \cellcolor{TealBlue!30}{\textbf{0.07}} & \cellcolor{TealBlue!30}{\textbf{4646}}\\
\texttt{letter} & \multicolumn{1}{r}{20000} & \multicolumn{1}{r}{224}  & \cellcolor{TealBlue!30}{\textbf{657}} & \cellcolor{TealBlue!30}{\textbf{369}} & \cellcolor{TealBlue!30}{\textbf{3}} & \cellcolor{TealBlue!30}{\textbf{15}} & \cellcolor{TealBlue!30}{\textbf{6.31}} & \cellcolor{TealBlue!30}{\textbf{1}} & \cellcolor{TealBlue!30}{\textbf{11.00}} & \cellcolor{TealBlue!30}{\textbf{44370}}\\
\texttt{lymph} & \multicolumn{1}{r}{148} & \multicolumn{1}{r}{41}  & \cellcolor{TealBlue!30}{\textbf{16}} & \cellcolor{TealBlue!30}{\textbf{12}} & \cellcolor{TealBlue!30}{\textbf{3}} & \cellcolor{TealBlue!30}{\textbf{15}} & \cellcolor{TealBlue!30}{\textbf{0.00}} & \cellcolor{TealBlue!30}{\textbf{1}} & \cellcolor{TealBlue!30}{\textbf{0.02}} & \cellcolor{TealBlue!30}{\textbf{6294}}\\
\texttt{mushroom} & \multicolumn{1}{r}{8124} & \multicolumn{1}{r}{91}  & \cellcolor{TealBlue!30}{\textbf{280}} & \cellcolor{TealBlue!30}{\textbf{8}} & \cellcolor{TealBlue!30}{\textbf{3}} & \cellcolor{TealBlue!30}{\textbf{13}} & \cellcolor{TealBlue!30}{\textbf{0.00}} & \cellcolor{TealBlue!30}{\textbf{1}} & \cellcolor{TealBlue!30}{\textbf{0.62}} & \cellcolor{TealBlue!30}{\textbf{19409}}\\
\texttt{pendigits} & \multicolumn{1}{r}{7494} & \multicolumn{1}{r}{216}  & \cellcolor{TealBlue!30}{\textbf{51}} & \cellcolor{TealBlue!30}{\textbf{47}} & \cellcolor{TealBlue!30}{\textbf{3}} & \cellcolor{TealBlue!30}{\textbf{13}} & \cellcolor{TealBlue!30}{\textbf{0.68}} & \cellcolor{TealBlue!30}{\textbf{1}} & \cellcolor{TealBlue!30}{\textbf{3.29}} & \cellcolor{TealBlue!30}{\textbf{38424}}\\
\texttt{primary-tumor} & \multicolumn{1}{r}{336} & \multicolumn{1}{r}{16}  & \cellcolor{TealBlue!30}{\textbf{51}} & \cellcolor{TealBlue!30}{\textbf{46}} & \cellcolor{TealBlue!30}{\textbf{3}} & \cellcolor{TealBlue!30}{\textbf{15}} & \cellcolor{TealBlue!30}{\textbf{0.00}} & \cellcolor{TealBlue!30}{\textbf{1}} & \cellcolor{TealBlue!30}{\textbf{0.01}} & \cellcolor{TealBlue!30}{\textbf{1036}}\\
\texttt{segment} & \multicolumn{1}{r}{2310} & \multicolumn{1}{r}{234}  & \cellcolor{TealBlue!30}{\textbf{5}} & \cellcolor{TealBlue!30}{\textbf{0}} & \cellcolor{TealBlue!30}{\textbf{3}} & \cellcolor{TealBlue!30}{\textbf{11}} & \cellcolor{TealBlue!30}{\textbf{0.05}} & \cellcolor{TealBlue!30}{\textbf{1}} & \cellcolor{TealBlue!30}{\textbf{1.12}} & \cellcolor{TealBlue!30}{\textbf{33184}}\\
\texttt{soybean} & \multicolumn{1}{r}{630} & \multicolumn{1}{r}{34}  & \cellcolor{TealBlue!30}{\textbf{47}} & \cellcolor{TealBlue!30}{\textbf{29}} & \cellcolor{TealBlue!30}{\textbf{3}} & \cellcolor{TealBlue!30}{\textbf{15}} & \cellcolor{TealBlue!30}{\textbf{0.01}} & \cellcolor{TealBlue!30}{\textbf{1}} & \cellcolor{TealBlue!30}{\textbf{0.03}} & \cellcolor{TealBlue!30}{\textbf{5227}}\\
\texttt{splice-1} & \multicolumn{1}{r}{3190} & \multicolumn{1}{r}{227}  & \cellcolor{TealBlue!30}{\textbf{279}} & \cellcolor{TealBlue!30}{\textbf{224}} & \cellcolor{TealBlue!30}{\textbf{3}} & \cellcolor{TealBlue!30}{\textbf{15}} & \cellcolor{TealBlue!30}{\textbf{0.12}} & \cellcolor{TealBlue!30}{\textbf{1}} & \cellcolor{TealBlue!30}{\textbf{9.99}} & \cellcolor{TealBlue!30}{\textbf{244223}}\\
\texttt{taiwan\_binarised} & \multicolumn{1}{r}{30000} & \multicolumn{1}{r}{198}  & \cellcolor{TealBlue!30}{\textbf{5333}} & \cellcolor{TealBlue!30}{\textbf{5326}} & \cellcolor{TealBlue!30}{\textbf{3}} & \cellcolor{TealBlue!30}{\textbf{15}} & \cellcolor{TealBlue!30}{\textbf{1.65}} & \cellcolor{TealBlue!30}{\textbf{1}} & \cellcolor{TealBlue!30}{\textbf{29.90}} & \cellcolor{TealBlue!30}{\textbf{143222}}\\
\texttt{tic-tac-toe} & \multicolumn{1}{r}{958} & \multicolumn{1}{r}{18}  & \cellcolor{TealBlue!30}{\textbf{236}} & \cellcolor{TealBlue!30}{\textbf{216}} & \cellcolor{TealBlue!30}{\textbf{3}} & \cellcolor{TealBlue!30}{\textbf{15}} & \cellcolor{TealBlue!30}{\textbf{0.01}} & \cellcolor{TealBlue!30}{\textbf{1}} & \cellcolor{TealBlue!30}{\textbf{0.01}} & \cellcolor{TealBlue!30}{\textbf{2700}}\\
\texttt{vehicle} & \multicolumn{1}{r}{846} & \multicolumn{1}{r}{252}  & \cellcolor{TealBlue!30}{\textbf{55}} & \cellcolor{TealBlue!30}{\textbf{26}} & \cellcolor{TealBlue!30}{\textbf{3}} & \cellcolor{TealBlue!30}{\textbf{15}} & \cellcolor{TealBlue!30}{\textbf{0.03}} & \cellcolor{TealBlue!30}{\textbf{1}} & \cellcolor{TealBlue!30}{\textbf{0.89}} & \cellcolor{TealBlue!30}{\textbf{48196}}\\
\texttt{vote} & \multicolumn{1}{r}{435} & \multicolumn{1}{r}{32}  & \cellcolor{TealBlue!30}{\textbf{14}} & \cellcolor{TealBlue!30}{\textbf{12}} & \cellcolor{TealBlue!30}{\textbf{3}} & \cellcolor{TealBlue!30}{\textbf{13}} & \cellcolor{TealBlue!30}{\textbf{0.01}} & \cellcolor{TealBlue!30}{\textbf{1}} & \cellcolor{TealBlue!30}{\textbf{0.04}} & \cellcolor{TealBlue!30}{\textbf{7608}}\\
\texttt{wine1-un} & \multicolumn{1}{r}{178} & \multicolumn{1}{r}{1276}  & \cellcolor{TealBlue!30}{\textbf{45}} & \cellcolor{TealBlue!30}{\textbf{43}} & \cellcolor{TealBlue!30}{\textbf{3}} & \cellcolor{TealBlue!30}{\textbf{13}} & \cellcolor{TealBlue!30}{\textbf{0.64}} & \cellcolor{TealBlue!30}{\textbf{1}} & \cellcolor{TealBlue!30}{\textbf{20.60}} & \cellcolor{TealBlue!30}{\textbf{832394}}\\
\texttt{wine2-un} & \multicolumn{1}{r}{178} & \multicolumn{1}{r}{1276}  & \cellcolor{TealBlue!30}{\textbf{52}} & \cellcolor{TealBlue!30}{\textbf{49}} & \cellcolor{TealBlue!30}{\textbf{3}} & \cellcolor{TealBlue!30}{\textbf{15}} & \cellcolor{TealBlue!30}{\textbf{0.20}} & \cellcolor{TealBlue!30}{\textbf{1}} & \cellcolor{TealBlue!30}{\textbf{20.00}} & \cellcolor{TealBlue!30}{\textbf{833674}}\\
\texttt{wine3-un} & \multicolumn{1}{r}{178} & \multicolumn{1}{r}{1276}  & \cellcolor{TealBlue!30}{\textbf{35}} & \cellcolor{TealBlue!30}{\textbf{33}} & \cellcolor{TealBlue!30}{\textbf{3}} & \cellcolor{TealBlue!30}{\textbf{13}} & \cellcolor{TealBlue!30}{\textbf{0.10}} & \cellcolor{TealBlue!30}{\textbf{1}} & \cellcolor{TealBlue!30}{\textbf{21.30}} & \cellcolor{TealBlue!30}{\textbf{832097}}\\
\texttt{yeast} & \multicolumn{1}{r}{1484} & \multicolumn{1}{r}{89}  & \cellcolor{TealBlue!30}{\textbf{417}} & \cellcolor{TealBlue!30}{\textbf{403}} & \cellcolor{TealBlue!30}{\textbf{3}} & \cellcolor{TealBlue!30}{\textbf{15}} & \cellcolor{TealBlue!30}{\textbf{0.01}} & \cellcolor{TealBlue!30}{\textbf{1}} & \cellcolor{TealBlue!30}{\textbf{0.08}} & \cellcolor{TealBlue!30}{\textbf{7281}}\\
\texttt{zoo-1} & \multicolumn{1}{r}{101} & \multicolumn{1}{r}{20}  & \cellcolor{TealBlue!30}{\textbf{0}} & \cellcolor{TealBlue!30}{\textbf{0}} & \cellcolor{TealBlue!30}{\textbf{1}} & \cellcolor{TealBlue!30}{\textbf{3}} & \cellcolor{TealBlue!30}{\textbf{0.00}} & \cellcolor{TealBlue!30}{\textbf{1}} & \cellcolor{TealBlue!30}{\textbf{0.00}} & \cellcolor{TealBlue!30}{\textbf{1}}\\
\bottomrule
\end{tabular}

\end{normalsize}
\end{center}
\caption{\label{tab:s3} max depth=3}
\end{table}

\begin{table}[htbp]
\begin{center}
\begin{normalsize}
\tabcolsep=3pt
\begin{tabular}{lccrrrrrrrr}
\toprule
& && \multicolumn{8}{c}{\budalg}\\
\cmidrule(rr){4-11}
&\multirow{1}{*}{$\#ex.$} & \multirow{1}{*}{\#feat.} &  \multicolumn{1}{c}{error (f)} & \multicolumn{1}{c}{error (b)} & \multicolumn{1}{c}{depth (b)} & \multicolumn{1}{c}{size (b)} & \multicolumn{1}{c}{time (b)} & \multicolumn{1}{c}{opt} & \multicolumn{1}{c}{time (a)} & \multicolumn{1}{c}{search (a)} \\
\midrule

\texttt{anneal} & \multicolumn{1}{r}{812} & \multicolumn{1}{r}{47}  & \cellcolor{TealBlue!30}{\textbf{135}} & \cellcolor{TealBlue!30}{\textbf{91}} & \cellcolor{TealBlue!30}{\textbf{4}} & \cellcolor{TealBlue!30}{\textbf{29}} & \cellcolor{TealBlue!30}{\textbf{0.94}} & \cellcolor{TealBlue!30}{\textbf{1}} & \cellcolor{TealBlue!30}{\textbf{1.69}} & \cellcolor{TealBlue!30}{\textbf{341104}}\\
\texttt{audiology} & \multicolumn{1}{r}{216} & \multicolumn{1}{r}{79}  & \cellcolor{TealBlue!30}{\textbf{3}} & \cellcolor{TealBlue!30}{\textbf{1}} & \cellcolor{TealBlue!30}{\textbf{4}} & \cellcolor{TealBlue!30}{\textbf{23}} & \cellcolor{TealBlue!30}{\textbf{0.03}} & \cellcolor{TealBlue!30}{\textbf{1}} & \cellcolor{TealBlue!30}{\textbf{4.62}} & \cellcolor{TealBlue!30}{\textbf{1000235}}\\
\texttt{australian-credit} & \multicolumn{1}{r}{653} & \multicolumn{1}{r}{73}  & \cellcolor{TealBlue!30}{\textbf{73}} & \cellcolor{TealBlue!30}{\textbf{56}} & \cellcolor{TealBlue!30}{\textbf{4}} & \cellcolor{TealBlue!30}{\textbf{31}} & \cellcolor{TealBlue!30}{\textbf{12.20}} & \cellcolor{TealBlue!30}{\textbf{1}} & \cellcolor{TealBlue!30}{\textbf{13.10}} & \cellcolor{TealBlue!30}{\textbf{1891214}}\\
\texttt{breast-cancer-un} & \multicolumn{1}{r}{683} & \multicolumn{1}{r}{89}  & \cellcolor{TealBlue!30}{\textbf{21}} & \cellcolor{TealBlue!30}{\textbf{16}} & \cellcolor{TealBlue!30}{\textbf{4}} & \cellcolor{TealBlue!30}{\textbf{29}} & \cellcolor{TealBlue!30}{\textbf{0.40}} & \cellcolor{TealBlue!30}{\textbf{1}} & \cellcolor{TealBlue!30}{\textbf{10.80}} & \cellcolor{TealBlue!30}{\textbf{1860878}}\\
\texttt{breast-wisconsin} & \multicolumn{1}{r}{683} & \multicolumn{1}{r}{120}  & \cellcolor{TealBlue!30}{\textbf{16}} & \cellcolor{TealBlue!30}{\textbf{7}} & \cellcolor{TealBlue!30}{\textbf{4}} & \cellcolor{TealBlue!30}{\textbf{31}} & \cellcolor{TealBlue!30}{\textbf{1.59}} & \cellcolor{TealBlue!30}{\textbf{1}} & \cellcolor{TealBlue!30}{\textbf{3.82}} & \cellcolor{TealBlue!30}{\textbf{688584}}\\
\texttt{car-un} & \multicolumn{1}{r}{1728} & \multicolumn{1}{r}{21}  & \cellcolor{TealBlue!30}{\textbf{178}} & \cellcolor{TealBlue!30}{\textbf{136}} & \cellcolor{TealBlue!30}{\textbf{4}} & \cellcolor{TealBlue!30}{\textbf{25}} & \cellcolor{TealBlue!30}{\textbf{0.05}} & \cellcolor{TealBlue!30}{\textbf{1}} & \cellcolor{TealBlue!30}{\textbf{0.16}} & \cellcolor{TealBlue!30}{\textbf{46288}}\\
\texttt{diabetes} & \multicolumn{1}{r}{768} & \multicolumn{1}{r}{112}  & \cellcolor{TealBlue!30}{\textbf{159}} & \cellcolor{TealBlue!30}{\textbf{137}} & \cellcolor{TealBlue!30}{\textbf{4}} & \cellcolor{TealBlue!30}{\textbf{31}} & \cellcolor{TealBlue!30}{\textbf{0.07}} & \cellcolor{TealBlue!30}{\textbf{1}} & \cellcolor{TealBlue!30}{\textbf{6.24}} & \cellcolor{TealBlue!30}{\textbf{1052519}}\\
\texttt{forest-fires-un} & \multicolumn{1}{r}{517} & \multicolumn{1}{r}{989}  & \cellcolor{TealBlue!30}{\textbf{186}} & \cellcolor{TealBlue!30}{\textbf{173}} & \cellcolor{TealBlue!30}{\textbf{4}} & \cellcolor{TealBlue!30}{\textbf{29}} & \cellcolor{TealBlue!30}{\textbf{17.90}} & \cellcolor{TealBlue!30}{\textbf{0}} & - & -\\
\texttt{german-credit} & \multicolumn{1}{r}{1000} & \multicolumn{1}{r}{110}  & \cellcolor{TealBlue!30}{\textbf{224}} & \cellcolor{TealBlue!30}{\textbf{204}} & \cellcolor{TealBlue!30}{\textbf{4}} & \cellcolor{TealBlue!30}{\textbf{31}} & \cellcolor{TealBlue!30}{\textbf{0.41}} & \cellcolor{TealBlue!30}{\textbf{1}} & \cellcolor{TealBlue!30}{\textbf{31.30}} & \cellcolor{TealBlue!30}{\textbf{3777150}}\\
\texttt{heart-cleveland} & \multicolumn{1}{r}{296} & \multicolumn{1}{r}{50}  & \cellcolor{TealBlue!30}{\textbf{36}} & \cellcolor{TealBlue!30}{\textbf{25}} & \cellcolor{TealBlue!30}{\textbf{4}} & \cellcolor{TealBlue!30}{\textbf{31}} & \cellcolor{TealBlue!30}{\textbf{1.60}} & \cellcolor{TealBlue!30}{\textbf{1}} & \cellcolor{TealBlue!30}{\textbf{3.82}} & \cellcolor{TealBlue!30}{\textbf{778085}}\\
\texttt{hepatitis} & \multicolumn{1}{r}{137} & \multicolumn{1}{r}{68}  & \cellcolor{TealBlue!30}{\textbf{12}} & \cellcolor{TealBlue!30}{\textbf{3}} & \cellcolor{TealBlue!30}{\textbf{4}} & \cellcolor{TealBlue!30}{\textbf{29}} & \cellcolor{TealBlue!30}{\textbf{0.27}} & \cellcolor{TealBlue!30}{\textbf{1}} & \cellcolor{TealBlue!30}{\textbf{0.34}} & \cellcolor{TealBlue!30}{\textbf{120490}}\\
\texttt{hypothyroid} & \multicolumn{1}{r}{3247} & \multicolumn{1}{r}{43}  & \cellcolor{TealBlue!30}{\textbf{53}} & \cellcolor{TealBlue!30}{\textbf{53}} & \cellcolor{TealBlue!30}{\textbf{4}} & \cellcolor{TealBlue!30}{\textbf{23}} & \cellcolor{TealBlue!30}{\textbf{0.00}} & \cellcolor{TealBlue!30}{\textbf{1}} & \cellcolor{TealBlue!30}{\textbf{3.70}} & \cellcolor{TealBlue!30}{\textbf{329869}}\\
\texttt{ionosphere} & \multicolumn{1}{r}{351} & \multicolumn{1}{r}{444}  & \cellcolor{TealBlue!30}{\textbf{25}} & \cellcolor{TealBlue!30}{\textbf{7}} & \cellcolor{TealBlue!30}{\textbf{4}} & \cellcolor{TealBlue!30}{\textbf{31}} & \cellcolor{TealBlue!30}{\textbf{442.00}} & \cellcolor{TealBlue!30}{\textbf{1}} & \cellcolor{TealBlue!30}{\textbf{966.00}} & \cellcolor{TealBlue!30}{\textbf{44451199}}\\
\texttt{kr-vs-kp} & \multicolumn{1}{r}{3196} & \multicolumn{1}{r}{37}  & \cellcolor{TealBlue!30}{\textbf{188}} & \cellcolor{TealBlue!30}{\textbf{144}} & \cellcolor{TealBlue!30}{\textbf{4}} & \cellcolor{TealBlue!30}{\textbf{27}} & \cellcolor{TealBlue!30}{\textbf{0.51}} & \cellcolor{TealBlue!30}{\textbf{1}} & \cellcolor{TealBlue!30}{\textbf{2.24}} & \cellcolor{TealBlue!30}{\textbf{230292}}\\
\texttt{letter} & \multicolumn{1}{r}{20000} & \multicolumn{1}{r}{224}  & \cellcolor{TealBlue!30}{\textbf{443}} & \cellcolor{TealBlue!30}{\textbf{261}} & \cellcolor{TealBlue!30}{\textbf{4}} & \cellcolor{TealBlue!30}{\textbf{29}} & \cellcolor{TealBlue!30}{\textbf{55.90}} & \cellcolor{TealBlue!30}{\textbf{1}} & \cellcolor{TealBlue!30}{\textbf{979.00}} & \cellcolor{TealBlue!30}{\textbf{7294586}}\\
\texttt{lymph} & \multicolumn{1}{r}{148} & \multicolumn{1}{r}{41}  & \cellcolor{TealBlue!30}{\textbf{9}} & \cellcolor{TealBlue!30}{\textbf{3}} & \cellcolor{TealBlue!30}{\textbf{4}} & \cellcolor{TealBlue!30}{\textbf{31}} & \cellcolor{TealBlue!30}{\textbf{0.00}} & \cellcolor{TealBlue!30}{\textbf{1}} & \cellcolor{TealBlue!30}{\textbf{0.79}} & \cellcolor{TealBlue!30}{\textbf{242060}}\\
\texttt{mushroom} & \multicolumn{1}{r}{8124} & \multicolumn{1}{r}{91}  & \cellcolor{TealBlue!30}{\textbf{4}} & \cellcolor{TealBlue!30}{\textbf{0}} & \cellcolor{TealBlue!30}{\textbf{4}} & \cellcolor{TealBlue!30}{\textbf{15}} & \cellcolor{TealBlue!30}{\textbf{0.27}} & \cellcolor{TealBlue!30}{\textbf{1}} & \cellcolor{TealBlue!30}{\textbf{55.00}} & \cellcolor{TealBlue!30}{\textbf{2017043}}\\
\texttt{pendigits} & \multicolumn{1}{r}{7494} & \multicolumn{1}{r}{216}  & \cellcolor{TealBlue!30}{\textbf{22}} & \cellcolor{TealBlue!30}{\textbf{13}} & \cellcolor{TealBlue!30}{\textbf{4}} & \cellcolor{TealBlue!30}{\textbf{27}} & \cellcolor{TealBlue!30}{\textbf{57.90}} & \cellcolor{TealBlue!30}{\textbf{1}} & \cellcolor{TealBlue!30}{\textbf{246.00}} & \cellcolor{TealBlue!30}{\textbf{4746647}}\\
\texttt{primary-tumor} & \multicolumn{1}{r}{336} & \multicolumn{1}{r}{16}  & \cellcolor{TealBlue!30}{\textbf{43}} & \cellcolor{TealBlue!30}{\textbf{34}} & \cellcolor{TealBlue!30}{\textbf{4}} & \cellcolor{TealBlue!30}{\textbf{31}} & \cellcolor{TealBlue!30}{\textbf{0.00}} & \cellcolor{TealBlue!30}{\textbf{1}} & \cellcolor{TealBlue!30}{\textbf{0.04}} & \cellcolor{TealBlue!30}{\textbf{24241}}\\
\texttt{segment} & \multicolumn{1}{r}{2310} & \multicolumn{1}{r}{234}  & \cellcolor{TealBlue!30}{\textbf{1}} & \cellcolor{TealBlue!30}{\textbf{0}} & \cellcolor{TealBlue!30}{\textbf{4}} & \cellcolor{TealBlue!30}{\textbf{11}} & \cellcolor{TealBlue!30}{\textbf{0.00}} & \cellcolor{TealBlue!30}{\textbf{1}} & \cellcolor{TealBlue!30}{\textbf{69.80}} & \cellcolor{TealBlue!30}{\textbf{3955297}}\\
\texttt{soybean} & \multicolumn{1}{r}{630} & \multicolumn{1}{r}{34}  & \cellcolor{TealBlue!30}{\textbf{32}} & \cellcolor{TealBlue!30}{\textbf{14}} & \cellcolor{TealBlue!30}{\textbf{4}} & \cellcolor{TealBlue!30}{\textbf{27}} & \cellcolor{TealBlue!30}{\textbf{0.13}} & \cellcolor{TealBlue!30}{\textbf{1}} & \cellcolor{TealBlue!30}{\textbf{0.88}} & \cellcolor{TealBlue!30}{\textbf{253314}}\\
\texttt{splice-1} & \multicolumn{1}{r}{3190} & \multicolumn{1}{r}{227}  & \cellcolor{TealBlue!30}{\textbf{141}} & \cellcolor{TealBlue!30}{\textbf{141}} & \cellcolor{TealBlue!30}{\textbf{4}} & \cellcolor{TealBlue!30}{\textbf{29}} & \cellcolor{TealBlue!30}{\textbf{0.00}} & \cellcolor{TealBlue!30}{\textbf{1}} & \cellcolor{TealBlue!30}{\textbf{3520.00}} & \cellcolor{TealBlue!30}{\textbf{108716836}}\\
\texttt{taiwan\_binarised} & \multicolumn{1}{r}{30000} & \multicolumn{1}{r}{198}  & \cellcolor{TealBlue!30}{\textbf{5293}} & \cellcolor{TealBlue!30}{\textbf{5273}} & \cellcolor{TealBlue!30}{\textbf{4}} & \cellcolor{TealBlue!30}{\textbf{31}} & \cellcolor{TealBlue!30}{\textbf{6.53}} & \cellcolor{TealBlue!30}{\textbf{0}} & - & -\\
\texttt{tic-tac-toe} & \multicolumn{1}{r}{958} & \multicolumn{1}{r}{18}  & \cellcolor{TealBlue!30}{\textbf{150}} & \cellcolor{TealBlue!30}{\textbf{137}} & \cellcolor{TealBlue!30}{\textbf{4}} & \cellcolor{TealBlue!30}{\textbf{27}} & \cellcolor{TealBlue!30}{\textbf{0.16}} & \cellcolor{TealBlue!30}{\textbf{1}} & \cellcolor{TealBlue!30}{\textbf{0.41}} & \cellcolor{TealBlue!30}{\textbf{123924}}\\
\texttt{vehicle} & \multicolumn{1}{r}{846} & \multicolumn{1}{r}{252}  & \cellcolor{TealBlue!30}{\textbf{28}} & \cellcolor{TealBlue!30}{\textbf{12}} & \cellcolor{TealBlue!30}{\textbf{4}} & \cellcolor{TealBlue!30}{\textbf{29}} & \cellcolor{TealBlue!30}{\textbf{4.93}} & \cellcolor{TealBlue!30}{\textbf{1}} & \cellcolor{TealBlue!30}{\textbf{86.40}} & \cellcolor{TealBlue!30}{\textbf{6395374}}\\
\texttt{vote} & \multicolumn{1}{r}{435} & \multicolumn{1}{r}{32}  & \cellcolor{TealBlue!30}{\textbf{8}} & \cellcolor{TealBlue!30}{\textbf{5}} & \cellcolor{TealBlue!30}{\textbf{4}} & \cellcolor{TealBlue!30}{\textbf{29}} & \cellcolor{TealBlue!30}{\textbf{0.08}} & \cellcolor{TealBlue!30}{\textbf{1}} & \cellcolor{TealBlue!30}{\textbf{1.55}} & \cellcolor{TealBlue!30}{\textbf{397257}}\\
\texttt{wine1-un} & \multicolumn{1}{r}{178} & \multicolumn{1}{r}{1276}  & \cellcolor{TealBlue!30}{\textbf{42}} & \cellcolor{TealBlue!30}{\textbf{37}} & \cellcolor{TealBlue!30}{\textbf{4}} & \cellcolor{TealBlue!30}{\textbf{23}} & \cellcolor{TealBlue!30}{\textbf{1980.00}} & \cellcolor{TealBlue!30}{\textbf{0}} & - & -\\
\texttt{wine2-un} & \multicolumn{1}{r}{178} & \multicolumn{1}{r}{1276}  & \cellcolor{TealBlue!30}{\textbf{47}} & \cellcolor{TealBlue!30}{\textbf{43}} & \cellcolor{TealBlue!30}{\textbf{4}} & \cellcolor{TealBlue!30}{\textbf{21}} & \cellcolor{TealBlue!30}{\textbf{19.30}} & \cellcolor{TealBlue!30}{\textbf{0}} & - & -\\
\texttt{wine3-un} & \multicolumn{1}{r}{178} & \multicolumn{1}{r}{1276}  & \cellcolor{TealBlue!30}{\textbf{32}} & \cellcolor{TealBlue!30}{\textbf{28}} & \cellcolor{TealBlue!30}{\textbf{4}} & \cellcolor{TealBlue!30}{\textbf{21}} & \cellcolor{TealBlue!30}{\textbf{39.70}} & \cellcolor{TealBlue!30}{\textbf{0}} & - & -\\
\texttt{yeast} & \multicolumn{1}{r}{1484} & \multicolumn{1}{r}{89}  & \cellcolor{TealBlue!30}{\textbf{392}} & \cellcolor{TealBlue!30}{\textbf{366}} & \cellcolor{TealBlue!30}{\textbf{4}} & \cellcolor{TealBlue!30}{\textbf{31}} & \cellcolor{TealBlue!30}{\textbf{2.69}} & \cellcolor{TealBlue!30}{\textbf{1}} & \cellcolor{TealBlue!30}{\textbf{3.56}} & \cellcolor{TealBlue!30}{\textbf{508819}}\\
\texttt{zoo-1} & \multicolumn{1}{r}{101} & \multicolumn{1}{r}{20}  & \cellcolor{TealBlue!30}{\textbf{0}} & \cellcolor{TealBlue!30}{\textbf{0}} & \cellcolor{TealBlue!30}{\textbf{1}} & \cellcolor{TealBlue!30}{\textbf{3}} & \cellcolor{TealBlue!30}{\textbf{0.00}} & \cellcolor{TealBlue!30}{\textbf{1}} & \cellcolor{TealBlue!30}{\textbf{0.00}} & \cellcolor{TealBlue!30}{\textbf{1}}\\
\bottomrule
\end{tabular}

\end{normalsize}
\end{center}
\caption{\label{tab:s4} max depth=4}
\end{table}

\begin{table}[htbp]
\begin{center}
\begin{normalsize}
\tabcolsep=3pt
\begin{tabular}{lccrrrrrrrr}
\toprule
& && \multicolumn{8}{c}{\budalg}\\
\cmidrule(rr){4-11}
&\multirow{1}{*}{$\#ex.$} & \multirow{1}{*}{\#feat.} &  \multicolumn{1}{c}{error (f)} & \multicolumn{1}{c}{error (b)} & \multicolumn{1}{c}{depth (b)} & \multicolumn{1}{c}{size (b)} & \multicolumn{1}{c}{time (b)} & \multicolumn{1}{c}{opt} & \multicolumn{1}{c}{time (a)} & \multicolumn{1}{c}{search (a)} \\
\midrule

\texttt{anneal} & \multicolumn{1}{r}{812} & \multicolumn{1}{r}{47}  & \cellcolor{TealBlue!30}{\textbf{114}} & \cellcolor{TealBlue!30}{\textbf{70}} & \cellcolor{TealBlue!30}{\textbf{5}} & \cellcolor{TealBlue!30}{\textbf{53}} & \cellcolor{TealBlue!30}{\textbf{44.20}} & \cellcolor{TealBlue!30}{\textbf{1}} & \cellcolor{TealBlue!30}{\textbf{69.60}} & \cellcolor{TealBlue!30}{\textbf{15945623}}\\
\texttt{audiology} & \multicolumn{1}{r}{216} & \multicolumn{1}{r}{79}  & \cellcolor{TealBlue!30}{\textbf{2}} & \cellcolor{TealBlue!30}{\textbf{0}} & \cellcolor{TealBlue!30}{\textbf{5}} & \cellcolor{TealBlue!30}{\textbf{21}} & \cellcolor{TealBlue!30}{\textbf{0.53}} & \cellcolor{TealBlue!30}{\textbf{1}} & \cellcolor{TealBlue!30}{\textbf{362.00}} & \cellcolor{TealBlue!30}{\textbf{85985117}}\\
\texttt{australian-credit} & \multicolumn{1}{r}{653} & \multicolumn{1}{r}{73}  & \cellcolor{TealBlue!30}{\textbf{63}} & \cellcolor{TealBlue!30}{\textbf{39}} & \cellcolor{TealBlue!30}{\textbf{5}} & \cellcolor{TealBlue!30}{\textbf{63}} & \cellcolor{TealBlue!30}{\textbf{431.00}} & \cellcolor{TealBlue!30}{\textbf{1}} & \cellcolor{TealBlue!30}{\textbf{923.00}} & \cellcolor{TealBlue!30}{\textbf{150943152}}\\
\texttt{breast-cancer-un} & \multicolumn{1}{r}{683} & \multicolumn{1}{r}{89}  & \cellcolor{TealBlue!30}{\textbf{16}} & \cellcolor{TealBlue!30}{\textbf{6}} & \cellcolor{TealBlue!30}{\textbf{5}} & \cellcolor{TealBlue!30}{\textbf{47}} & \cellcolor{TealBlue!30}{\textbf{38.40}} & \cellcolor{TealBlue!30}{\textbf{1}} & \cellcolor{TealBlue!30}{\textbf{891.00}} & \cellcolor{TealBlue!30}{\textbf{151765942}}\\
\texttt{breast-wisconsin} & \multicolumn{1}{r}{683} & \multicolumn{1}{r}{120}  & \cellcolor{TealBlue!30}{\textbf{13}} & \cellcolor{TealBlue!30}{\textbf{0}} & \cellcolor{TealBlue!30}{\textbf{5}} & \cellcolor{TealBlue!30}{\textbf{49}} & \cellcolor{TealBlue!30}{\textbf{99.60}} & \cellcolor{TealBlue!30}{\textbf{1}} & \cellcolor{TealBlue!30}{\textbf{211.00}} & \cellcolor{TealBlue!30}{\textbf{50820345}}\\
\texttt{car-un} & \multicolumn{1}{r}{1728} & \multicolumn{1}{r}{21}  & \cellcolor{TealBlue!30}{\textbf{106}} & \cellcolor{TealBlue!30}{\textbf{86}} & \cellcolor{TealBlue!30}{\textbf{5}} & \cellcolor{TealBlue!30}{\textbf{37}} & \cellcolor{TealBlue!30}{\textbf{0.82}} & \cellcolor{TealBlue!30}{\textbf{1}} & \cellcolor{TealBlue!30}{\textbf{2.70}} & \cellcolor{TealBlue!30}{\textbf{1100690}}\\
\texttt{diabetes} & \multicolumn{1}{r}{768} & \multicolumn{1}{r}{112}  & \cellcolor{TealBlue!30}{\textbf{141}} & \cellcolor{TealBlue!30}{\textbf{106}} & \cellcolor{TealBlue!30}{\textbf{5}} & \cellcolor{TealBlue!30}{\textbf{63}} & \cellcolor{TealBlue!30}{\textbf{83.70}} & \cellcolor{TealBlue!30}{\textbf{1}} & \cellcolor{TealBlue!30}{\textbf{393.00}} & \cellcolor{TealBlue!30}{\textbf{78408884}}\\
\texttt{forest-fires-un} & \multicolumn{1}{r}{517} & \multicolumn{1}{r}{989}  & \cellcolor{TealBlue!30}{\textbf{176}} & \cellcolor{TealBlue!30}{\textbf{156}} & \cellcolor{TealBlue!30}{\textbf{5}} & \cellcolor{TealBlue!30}{\textbf{49}} & \cellcolor{TealBlue!30}{\textbf{1030.00}} & \cellcolor{TealBlue!30}{\textbf{0}} & - & -\\
\texttt{german-credit} & \multicolumn{1}{r}{1000} & \multicolumn{1}{r}{110}  & \cellcolor{TealBlue!30}{\textbf{201}} & \cellcolor{TealBlue!30}{\textbf{161}} & \cellcolor{TealBlue!30}{\textbf{5}} & \cellcolor{TealBlue!30}{\textbf{63}} & \cellcolor{TealBlue!30}{\textbf{44.50}} & \cellcolor{TealBlue!30}{\textbf{1}} & \cellcolor{TealBlue!30}{\textbf{3330.00}} & \cellcolor{TealBlue!30}{\textbf{431148502}}\\
\texttt{heart-cleveland} & \multicolumn{1}{r}{296} & \multicolumn{1}{r}{50}  & \cellcolor{TealBlue!30}{\textbf{26}} & \cellcolor{TealBlue!30}{\textbf{7}} & \cellcolor{TealBlue!30}{\textbf{5}} & \cellcolor{TealBlue!30}{\textbf{63}} & \cellcolor{TealBlue!30}{\textbf{9.84}} & \cellcolor{TealBlue!30}{\textbf{1}} & \cellcolor{TealBlue!30}{\textbf{127.00}} & \cellcolor{TealBlue!30}{\textbf{30306334}}\\
\texttt{hepatitis} & \multicolumn{1}{r}{137} & \multicolumn{1}{r}{68}  & \cellcolor{TealBlue!30}{\textbf{8}} & \cellcolor{TealBlue!30}{\textbf{0}} & \cellcolor{TealBlue!30}{\textbf{5}} & \cellcolor{TealBlue!30}{\textbf{35}} & \cellcolor{TealBlue!30}{\textbf{0.22}} & \cellcolor{TealBlue!30}{\textbf{1}} & \cellcolor{TealBlue!30}{\textbf{10.10}} & \cellcolor{TealBlue!30}{\textbf{4817264}}\\
\texttt{hypothyroid} & \multicolumn{1}{r}{3247} & \multicolumn{1}{r}{43}  & \cellcolor{TealBlue!30}{\textbf{50}} & \cellcolor{TealBlue!30}{\textbf{44}} & \cellcolor{TealBlue!30}{\textbf{5}} & \cellcolor{TealBlue!30}{\textbf{45}} & \cellcolor{TealBlue!30}{\textbf{40.50}} & \cellcolor{TealBlue!30}{\textbf{1}} & \cellcolor{TealBlue!30}{\textbf{142.00}} & \cellcolor{TealBlue!30}{\textbf{15022732}}\\
\texttt{ionosphere} & \multicolumn{1}{r}{351} & \multicolumn{1}{r}{444}  & \cellcolor{TealBlue!30}{\textbf{16}} & \cellcolor{TealBlue!30}{\textbf{0}} & \cellcolor{TealBlue!30}{\textbf{5}} & \cellcolor{TealBlue!30}{\textbf{43}} & \cellcolor{TealBlue!30}{\textbf{3180.00}} & \cellcolor{TealBlue!30}{\textbf{0}} & - & -\\
\texttt{kr-vs-kp} & \multicolumn{1}{r}{3196} & \multicolumn{1}{r}{37}  & \cellcolor{TealBlue!30}{\textbf{179}} & \cellcolor{TealBlue!30}{\textbf{81}} & \cellcolor{TealBlue!30}{\textbf{5}} & \cellcolor{TealBlue!30}{\textbf{47}} & \cellcolor{TealBlue!30}{\textbf{5.00}} & \cellcolor{TealBlue!30}{\textbf{1}} & \cellcolor{TealBlue!30}{\textbf{65.90}} & \cellcolor{TealBlue!30}{\textbf{8883948}}\\
\texttt{letter} & \multicolumn{1}{r}{20000} & \multicolumn{1}{r}{224}  & \cellcolor{TealBlue!30}{\textbf{335}} & \cellcolor{TealBlue!30}{\textbf{168}} & \cellcolor{TealBlue!30}{\textbf{5}} & \cellcolor{TealBlue!30}{\textbf{55}} & \cellcolor{TealBlue!30}{\textbf{3460.00}} & \cellcolor{TealBlue!30}{\textbf{0}} & - & -\\
\texttt{lymph} & \multicolumn{1}{r}{148} & \multicolumn{1}{r}{41}  & \cellcolor{TealBlue!30}{\textbf{4}} & \cellcolor{TealBlue!30}{\textbf{0}} & \cellcolor{TealBlue!30}{\textbf{5}} & \cellcolor{TealBlue!30}{\textbf{35}} & \cellcolor{TealBlue!30}{\textbf{1.11}} & \cellcolor{TealBlue!30}{\textbf{1}} & \cellcolor{TealBlue!30}{\textbf{33.80}} & \cellcolor{TealBlue!30}{\textbf{12447288}}\\
\texttt{mushroom} & \multicolumn{1}{r}{8124} & \multicolumn{1}{r}{91}  & \cellcolor{TealBlue!30}{\textbf{3}} & \cellcolor{TealBlue!30}{\textbf{0}} & \cellcolor{TealBlue!30}{\textbf{4}} & \cellcolor{TealBlue!30}{\textbf{15}} & \cellcolor{TealBlue!30}{\textbf{0.26}} & \cellcolor{TealBlue!30}{\textbf{1}} & \cellcolor{TealBlue!30}{\textbf{55.70}} & \cellcolor{TealBlue!30}{\textbf{2017055}}\\
\texttt{pendigits} & \multicolumn{1}{r}{7494} & \multicolumn{1}{r}{216}  & \cellcolor{TealBlue!30}{\textbf{11}} & \cellcolor{TealBlue!30}{\textbf{0}} & \cellcolor{TealBlue!30}{\textbf{5}} & \cellcolor{TealBlue!30}{\textbf{43}} & \cellcolor{TealBlue!30}{\textbf{3340.00}} & \cellcolor{TealBlue!30}{\textbf{0}} & - & -\\
\texttt{primary-tumor} & \multicolumn{1}{r}{336} & \multicolumn{1}{r}{16}  & \cellcolor{TealBlue!30}{\textbf{34}} & \cellcolor{TealBlue!30}{\textbf{26}} & \cellcolor{TealBlue!30}{\textbf{5}} & \cellcolor{TealBlue!30}{\textbf{61}} & \cellcolor{TealBlue!30}{\textbf{0.12}} & \cellcolor{TealBlue!30}{\textbf{1}} & \cellcolor{TealBlue!30}{\textbf{0.63}} & \cellcolor{TealBlue!30}{\textbf{434809}}\\
\texttt{segment} & \multicolumn{1}{r}{2310} & \multicolumn{1}{r}{234}  & \cellcolor{TealBlue!30}{\textbf{1}} & \cellcolor{TealBlue!30}{\textbf{0}} & \cellcolor{TealBlue!30}{\textbf{4}} & \cellcolor{TealBlue!30}{\textbf{11}} & \cellcolor{TealBlue!30}{\textbf{0.00}} & \cellcolor{TealBlue!30}{\textbf{1}} & \cellcolor{TealBlue!30}{\textbf{69.30}} & \cellcolor{TealBlue!30}{\textbf{3955315}}\\
\texttt{soybean} & \multicolumn{1}{r}{630} & \multicolumn{1}{r}{34}  & \cellcolor{TealBlue!30}{\textbf{23}} & \cellcolor{TealBlue!30}{\textbf{8}} & \cellcolor{TealBlue!30}{\textbf{5}} & \cellcolor{TealBlue!30}{\textbf{49}} & \cellcolor{TealBlue!30}{\textbf{3.82}} & \cellcolor{TealBlue!30}{\textbf{1}} & \cellcolor{TealBlue!30}{\textbf{30.00}} & \cellcolor{TealBlue!30}{\textbf{9732632}}\\
\texttt{splice-1} & \multicolumn{1}{r}{3190} & \multicolumn{1}{r}{227}  & \cellcolor{TealBlue!30}{\textbf{111}} & \cellcolor{TealBlue!30}{\textbf{101}} & \cellcolor{TealBlue!30}{\textbf{5}} & \cellcolor{TealBlue!30}{\textbf{49}} & \cellcolor{TealBlue!30}{\textbf{1870.00}} & \cellcolor{TealBlue!30}{\textbf{0}} & - & -\\
\texttt{taiwan\_binarised} & \multicolumn{1}{r}{30000} & \multicolumn{1}{r}{198}  & \cellcolor{TealBlue!30}{\textbf{5257}} & \cellcolor{TealBlue!30}{\textbf{5200}} & \cellcolor{TealBlue!30}{\textbf{5}} & \cellcolor{TealBlue!30}{\textbf{63}} & \cellcolor{TealBlue!30}{\textbf{936.00}} & \cellcolor{TealBlue!30}{\textbf{0}} & - & -\\
\texttt{tic-tac-toe} & \multicolumn{1}{r}{958} & \multicolumn{1}{r}{18}  & \cellcolor{TealBlue!30}{\textbf{78}} & \cellcolor{TealBlue!30}{\textbf{63}} & \cellcolor{TealBlue!30}{\textbf{5}} & \cellcolor{TealBlue!30}{\textbf{47}} & \cellcolor{TealBlue!30}{\textbf{0.00}} & \cellcolor{TealBlue!30}{\textbf{1}} & \cellcolor{TealBlue!30}{\textbf{10.90}} & \cellcolor{TealBlue!30}{\textbf{4387432}}\\
\texttt{vehicle} & \multicolumn{1}{r}{846} & \multicolumn{1}{r}{252}  & \cellcolor{TealBlue!30}{\textbf{21}} & \cellcolor{TealBlue!30}{\textbf{1}} & \cellcolor{TealBlue!30}{\textbf{5}} & \cellcolor{TealBlue!30}{\textbf{51}} & \cellcolor{TealBlue!30}{\textbf{763.00}} & \cellcolor{TealBlue!30}{\textbf{0}} & - & -\\
\texttt{vote} & \multicolumn{1}{r}{435} & \multicolumn{1}{r}{32}  & \cellcolor{TealBlue!30}{\textbf{6}} & \cellcolor{TealBlue!30}{\textbf{1}} & \cellcolor{TealBlue!30}{\textbf{5}} & \cellcolor{TealBlue!30}{\textbf{43}} & \cellcolor{TealBlue!30}{\textbf{0.02}} & \cellcolor{TealBlue!30}{\textbf{1}} & \cellcolor{TealBlue!30}{\textbf{30.40}} & \cellcolor{TealBlue!30}{\textbf{8831574}}\\
\texttt{wine1-un} & \multicolumn{1}{r}{178} & \multicolumn{1}{r}{1276}  & \cellcolor{TealBlue!30}{\textbf{39}} & \cellcolor{TealBlue!30}{\textbf{33}} & \cellcolor{TealBlue!30}{\textbf{5}} & \cellcolor{TealBlue!30}{\textbf{25}} & \cellcolor{TealBlue!30}{\textbf{1400.00}} & \cellcolor{TealBlue!30}{\textbf{0}} & - & -\\
\texttt{wine2-un} & \multicolumn{1}{r}{178} & \multicolumn{1}{r}{1276}  & \cellcolor{TealBlue!30}{\textbf{44}} & \cellcolor{TealBlue!30}{\textbf{39}} & \cellcolor{TealBlue!30}{\textbf{5}} & \cellcolor{TealBlue!30}{\textbf{23}} & \cellcolor{TealBlue!30}{\textbf{493.00}} & \cellcolor{TealBlue!30}{\textbf{0}} & - & -\\
\texttt{wine3-un} & \multicolumn{1}{r}{178} & \multicolumn{1}{r}{1276}  & \cellcolor{TealBlue!30}{\textbf{30}} & \cellcolor{TealBlue!30}{\textbf{25}} & \cellcolor{TealBlue!30}{\textbf{5}} & \cellcolor{TealBlue!30}{\textbf{23}} & \cellcolor{TealBlue!30}{\textbf{20.80}} & \cellcolor{TealBlue!30}{\textbf{0}} & - & -\\
\texttt{yeast} & \multicolumn{1}{r}{1484} & \multicolumn{1}{r}{89}  & \cellcolor{TealBlue!30}{\textbf{365}} & \cellcolor{TealBlue!30}{\textbf{313}} & \cellcolor{TealBlue!30}{\textbf{5}} & \cellcolor{TealBlue!30}{\textbf{63}} & \cellcolor{TealBlue!30}{\textbf{36.90}} & \cellcolor{TealBlue!30}{\textbf{1}} & \cellcolor{TealBlue!30}{\textbf{161.00}} & \cellcolor{TealBlue!30}{\textbf{30774291}}\\
\texttt{zoo-1} & \multicolumn{1}{r}{101} & \multicolumn{1}{r}{20}  & \cellcolor{TealBlue!30}{\textbf{0}} & \cellcolor{TealBlue!30}{\textbf{0}} & \cellcolor{TealBlue!30}{\textbf{1}} & \cellcolor{TealBlue!30}{\textbf{3}} & \cellcolor{TealBlue!30}{\textbf{0.00}} & \cellcolor{TealBlue!30}{\textbf{1}} & \cellcolor{TealBlue!30}{\textbf{0.00}} & \cellcolor{TealBlue!30}{\textbf{1}}\\
\bottomrule
\end{tabular}

\end{normalsize}
\end{center}
\caption{\label{tab:s5} max depth=5}
\end{table}

\begin{table}[htbp]
\begin{center}
\begin{normalsize}
\tabcolsep=3pt
\begin{tabular}{lccrrrrrrrr}
\toprule
& && \multicolumn{8}{c}{\budalg}\\
\cmidrule(rr){4-11}
&\multirow{1}{*}{$\#ex.$} & \multirow{1}{*}{\#feat.} &  \multicolumn{1}{c}{error (f)} & \multicolumn{1}{c}{error (b)} & \multicolumn{1}{c}{depth (b)} & \multicolumn{1}{c}{size (b)} & \multicolumn{1}{c}{time (b)} & \multicolumn{1}{c}{opt} & \multicolumn{1}{c}{time (a)} & \multicolumn{1}{c}{search (a)} \\
\midrule

\texttt{anneal} & \multicolumn{1}{r}{812} & \multicolumn{1}{r}{47}  & \cellcolor{TealBlue!30}{\textbf{94}} & \cellcolor{TealBlue!30}{\textbf{50}} & \cellcolor{TealBlue!30}{\textbf{7}} & \cellcolor{TealBlue!30}{\textbf{111}} & \cellcolor{TealBlue!30}{\textbf{383.00}} & \cellcolor{TealBlue!30}{\textbf{0}} & - & -\\
\texttt{audiology} & \multicolumn{1}{r}{216} & \multicolumn{1}{r}{79}  & \cellcolor{TealBlue!30}{\textbf{0}} & \cellcolor{TealBlue!30}{\textbf{0}} & \cellcolor{TealBlue!30}{\textbf{5}} & \cellcolor{TealBlue!30}{\textbf{21}} & \cellcolor{TealBlue!30}{\textbf{0.52}} & \cellcolor{TealBlue!30}{\textbf{1}} & \cellcolor{TealBlue!30}{\textbf{373.00}} & \cellcolor{TealBlue!30}{\textbf{85985144}}\\
\texttt{australian-credit} & \multicolumn{1}{r}{653} & \multicolumn{1}{r}{73}  & \cellcolor{TealBlue!30}{\textbf{43}} & \cellcolor{TealBlue!30}{\textbf{0}} & \cellcolor{TealBlue!30}{\textbf{7}} & \cellcolor{TealBlue!30}{\textbf{173}} & \cellcolor{TealBlue!30}{\textbf{123.00}} & \cellcolor{TealBlue!30}{\textbf{0}} & - & -\\
\texttt{breast-cancer-un} & \multicolumn{1}{r}{683} & \multicolumn{1}{r}{89}  & \cellcolor{TealBlue!30}{\textbf{8}} & \cellcolor{TealBlue!30}{\textbf{0}} & \cellcolor{TealBlue!30}{\textbf{7}} & \cellcolor{TealBlue!30}{\textbf{71}} & \cellcolor{TealBlue!30}{\textbf{1270.00}} & \cellcolor{TealBlue!30}{\textbf{0}} & - & -\\
\texttt{breast-wisconsin} & \multicolumn{1}{r}{683} & \multicolumn{1}{r}{120}  & \cellcolor{TealBlue!30}{\textbf{4}} & \cellcolor{TealBlue!30}{\textbf{0}} & \cellcolor{TealBlue!30}{\textbf{6}} & \cellcolor{TealBlue!30}{\textbf{41}} & \cellcolor{TealBlue!30}{\textbf{3280.00}} & \cellcolor{TealBlue!30}{\textbf{0}} & - & -\\
\texttt{car-un} & \multicolumn{1}{r}{1728} & \multicolumn{1}{r}{21}  & \cellcolor{TealBlue!30}{\textbf{50}} & \cellcolor{TealBlue!30}{\textbf{11}} & \cellcolor{TealBlue!30}{\textbf{7}} & \cellcolor{TealBlue!30}{\textbf{107}} & \cellcolor{TealBlue!30}{\textbf{151.00}} & \cellcolor{TealBlue!30}{\textbf{1}} & \cellcolor{TealBlue!30}{\textbf{275.00}} & \cellcolor{TealBlue!30}{\textbf{182560348}}\\
\texttt{diabetes} & \multicolumn{1}{r}{768} & \multicolumn{1}{r}{112}  & \cellcolor{TealBlue!30}{\textbf{99}} & \cellcolor{TealBlue!30}{\textbf{21}} & \cellcolor{TealBlue!30}{\textbf{7}} & \cellcolor{TealBlue!30}{\textbf{223}} & \cellcolor{TealBlue!30}{\textbf{1150.00}} & \cellcolor{TealBlue!30}{\textbf{0}} & - & -\\
\texttt{forest-fires-un} & \multicolumn{1}{r}{517} & \multicolumn{1}{r}{989}  & \cellcolor{TealBlue!30}{\textbf{162}} & \cellcolor{TealBlue!30}{\textbf{156}} & \cellcolor{TealBlue!30}{\textbf{7}} & \cellcolor{TealBlue!30}{\textbf{73}} & \cellcolor{TealBlue!30}{\textbf{3.15}} & \cellcolor{TealBlue!30}{\textbf{0}} & - & -\\
\texttt{german-credit} & \multicolumn{1}{r}{1000} & \multicolumn{1}{r}{110}  & \cellcolor{TealBlue!30}{\textbf{141}} & \cellcolor{TealBlue!30}{\textbf{56}} & \cellcolor{TealBlue!30}{\textbf{7}} & \cellcolor{TealBlue!30}{\textbf{215}} & \cellcolor{TealBlue!30}{\textbf{1680.00}} & \cellcolor{TealBlue!30}{\textbf{0}} & - & -\\
\texttt{heart-cleveland} & \multicolumn{1}{r}{296} & \multicolumn{1}{r}{50}  & \cellcolor{TealBlue!30}{\textbf{6}} & \cellcolor{TealBlue!30}{\textbf{0}} & \cellcolor{TealBlue!30}{\textbf{6}} & \cellcolor{TealBlue!30}{\textbf{75}} & \cellcolor{TealBlue!30}{\textbf{686.00}} & \cellcolor{TealBlue!30}{\textbf{0}} & - & -\\
\texttt{hepatitis} & \multicolumn{1}{r}{137} & \multicolumn{1}{r}{68}  & \cellcolor{TealBlue!30}{\textbf{0}} & \cellcolor{TealBlue!30}{\textbf{0}} & \cellcolor{TealBlue!30}{\textbf{5}} & \cellcolor{TealBlue!30}{\textbf{35}} & \cellcolor{TealBlue!30}{\textbf{0.24}} & \cellcolor{TealBlue!30}{\textbf{1}} & \cellcolor{TealBlue!30}{\textbf{10.50}} & \cellcolor{TealBlue!30}{\textbf{4823276}}\\
\texttt{hypothyroid} & \multicolumn{1}{r}{3247} & \multicolumn{1}{r}{43}  & \cellcolor{TealBlue!30}{\textbf{42}} & \cellcolor{TealBlue!30}{\textbf{23}} & \cellcolor{TealBlue!30}{\textbf{7}} & \cellcolor{TealBlue!30}{\textbf{111}} & \cellcolor{TealBlue!30}{\textbf{1840.00}} & \cellcolor{TealBlue!30}{\textbf{0}} & - & -\\
\texttt{ionosphere} & \multicolumn{1}{r}{351} & \multicolumn{1}{r}{444}  & \cellcolor{TealBlue!30}{\textbf{7}} & \cellcolor{TealBlue!30}{\textbf{0}} & \cellcolor{TealBlue!30}{\textbf{6}} & \cellcolor{TealBlue!30}{\textbf{55}} & \cellcolor{TealBlue!30}{\textbf{5.42}} & \cellcolor{TealBlue!30}{\textbf{0}} & - & -\\
\texttt{kr-vs-kp} & \multicolumn{1}{r}{3196} & \multicolumn{1}{r}{37}  & \cellcolor{TealBlue!30}{\textbf{102}} & \cellcolor{TealBlue!30}{\textbf{18}} & \cellcolor{TealBlue!30}{\textbf{7}} & \cellcolor{TealBlue!30}{\textbf{131}} & \cellcolor{TealBlue!30}{\textbf{2460.00}} & \cellcolor{TealBlue!30}{\textbf{0}} & - & -\\
\texttt{letter} & \multicolumn{1}{r}{20000} & \multicolumn{1}{r}{224}  & \cellcolor{TealBlue!30}{\textbf{143}} & \cellcolor{TealBlue!30}{\textbf{70}} & \cellcolor{TealBlue!30}{\textbf{7}} & \cellcolor{TealBlue!30}{\textbf{173}} & \cellcolor{TealBlue!30}{\textbf{434.00}} & \cellcolor{TealBlue!30}{\textbf{0}} & - & -\\
\texttt{lymph} & \multicolumn{1}{r}{148} & \multicolumn{1}{r}{41}  & \cellcolor{TealBlue!30}{\textbf{0}} & \cellcolor{TealBlue!30}{\textbf{0}} & \cellcolor{TealBlue!30}{\textbf{5}} & \cellcolor{TealBlue!30}{\textbf{35}} & \cellcolor{TealBlue!30}{\textbf{1.16}} & \cellcolor{TealBlue!30}{\textbf{1}} & \cellcolor{TealBlue!30}{\textbf{35.00}} & \cellcolor{TealBlue!30}{\textbf{12447337}}\\
\texttt{mushroom} & \multicolumn{1}{r}{8124} & \multicolumn{1}{r}{91}  & \cellcolor{TealBlue!30}{\textbf{0}} & \cellcolor{TealBlue!30}{\textbf{0}} & \cellcolor{TealBlue!30}{\textbf{4}} & \cellcolor{TealBlue!30}{\textbf{15}} & \cellcolor{TealBlue!30}{\textbf{0.29}} & \cellcolor{TealBlue!30}{\textbf{1}} & \cellcolor{TealBlue!30}{\textbf{57.60}} & \cellcolor{TealBlue!30}{\textbf{2017076}}\\
\texttt{pendigits} & \multicolumn{1}{r}{7494} & \multicolumn{1}{r}{216}  & \cellcolor{TealBlue!30}{\textbf{1}} & \cellcolor{TealBlue!30}{\textbf{0}} & \cellcolor{TealBlue!30}{\textbf{6}} & \cellcolor{TealBlue!30}{\textbf{41}} & \cellcolor{TealBlue!30}{\textbf{2200.00}} & \cellcolor{TealBlue!30}{\textbf{0}} & - & -\\
\texttt{primary-tumor} & \multicolumn{1}{r}{336} & \multicolumn{1}{r}{16}  & \cellcolor{TealBlue!30}{\textbf{26}} & \cellcolor{TealBlue!30}{\textbf{16}} & \cellcolor{TealBlue!30}{\textbf{7}} & \cellcolor{TealBlue!30}{\textbf{115}} & \cellcolor{TealBlue!30}{\textbf{1.15}} & \cellcolor{TealBlue!30}{\textbf{1}} & \cellcolor{TealBlue!30}{\textbf{88.40}} & \cellcolor{TealBlue!30}{\textbf{75229131}}\\
\texttt{segment} & \multicolumn{1}{r}{2310} & \multicolumn{1}{r}{234}  & \cellcolor{TealBlue!30}{\textbf{0}} & \cellcolor{TealBlue!30}{\textbf{0}} & \cellcolor{TealBlue!30}{\textbf{4}} & \cellcolor{TealBlue!30}{\textbf{11}} & \cellcolor{TealBlue!30}{\textbf{0.00}} & \cellcolor{TealBlue!30}{\textbf{1}} & \cellcolor{TealBlue!30}{\textbf{76.30}} & \cellcolor{TealBlue!30}{\textbf{3955322}}\\
\texttt{soybean} & \multicolumn{1}{r}{630} & \multicolumn{1}{r}{34}  & \cellcolor{TealBlue!30}{\textbf{11}} & \cellcolor{TealBlue!30}{\textbf{2}} & \cellcolor{TealBlue!30}{\textbf{7}} & \cellcolor{TealBlue!30}{\textbf{107}} & \cellcolor{TealBlue!30}{\textbf{482.00}} & \cellcolor{TealBlue!30}{\textbf{0}} & - & -\\
\texttt{splice-1} & \multicolumn{1}{r}{3190} & \multicolumn{1}{r}{227}  & \cellcolor{TealBlue!30}{\textbf{58}} & \cellcolor{TealBlue!30}{\textbf{32}} & \cellcolor{TealBlue!30}{\textbf{7}} & \cellcolor{TealBlue!30}{\textbf{153}} & \cellcolor{TealBlue!30}{\textbf{2430.00}} & \cellcolor{TealBlue!30}{\textbf{0}} & - & -\\
\texttt{taiwan\_binarised} & \multicolumn{1}{r}{30000} & \multicolumn{1}{r}{198}  & \cellcolor{TealBlue!30}{\textbf{5121}} & \cellcolor{TealBlue!30}{\textbf{5065}} & \cellcolor{TealBlue!30}{\textbf{7}} & \cellcolor{TealBlue!30}{\textbf{247}} & \cellcolor{TealBlue!30}{\textbf{85.70}} & \cellcolor{TealBlue!30}{\textbf{0}} & - & -\\
\texttt{tic-tac-toe} & \multicolumn{1}{r}{958} & \multicolumn{1}{r}{18}  & \cellcolor{TealBlue!30}{\textbf{21}} & \cellcolor{TealBlue!30}{\textbf{0}} & \cellcolor{TealBlue!30}{\textbf{7}} & \cellcolor{TealBlue!30}{\textbf{89}} & \cellcolor{TealBlue!30}{\textbf{527.00}} & \cellcolor{TealBlue!30}{\textbf{1}} & \cellcolor{TealBlue!30}{\textbf{1740.00}} & \cellcolor{TealBlue!30}{\textbf{1046729418}}\\
\texttt{vehicle} & \multicolumn{1}{r}{846} & \multicolumn{1}{r}{252}  & \cellcolor{TealBlue!30}{\textbf{4}} & \cellcolor{TealBlue!30}{\textbf{0}} & \cellcolor{TealBlue!30}{\textbf{6}} & \cellcolor{TealBlue!30}{\textbf{73}} & \cellcolor{TealBlue!30}{\textbf{0.19}} & \cellcolor{TealBlue!30}{\textbf{0}} & - & -\\
\texttt{vote} & \multicolumn{1}{r}{435} & \multicolumn{1}{r}{32}  & \cellcolor{TealBlue!30}{\textbf{2}} & \cellcolor{TealBlue!30}{\textbf{0}} & \cellcolor{TealBlue!30}{\textbf{6}} & \cellcolor{TealBlue!30}{\textbf{39}} & \cellcolor{TealBlue!30}{\textbf{10.30}} & \cellcolor{TealBlue!30}{\textbf{1}} & \cellcolor{TealBlue!30}{\textbf{1870.00}} & \cellcolor{TealBlue!30}{\textbf{632757986}}\\
\texttt{wine1-un} & \multicolumn{1}{r}{178} & \multicolumn{1}{r}{1276}  & \cellcolor{TealBlue!30}{\textbf{33}} & \cellcolor{TealBlue!30}{\textbf{28}} & \cellcolor{TealBlue!30}{\textbf{7}} & \cellcolor{TealBlue!30}{\textbf{27}} & \cellcolor{TealBlue!30}{\textbf{1090.00}} & \cellcolor{TealBlue!30}{\textbf{0}} & - & -\\
\texttt{wine2-un} & \multicolumn{1}{r}{178} & \multicolumn{1}{r}{1276}  & \cellcolor{TealBlue!30}{\textbf{38}} & \cellcolor{TealBlue!30}{\textbf{31}} & \cellcolor{TealBlue!30}{\textbf{7}} & \cellcolor{TealBlue!30}{\textbf{31}} & \cellcolor{TealBlue!30}{\textbf{34.90}} & \cellcolor{TealBlue!30}{\textbf{0}} & - & -\\
\texttt{wine3-un} & \multicolumn{1}{r}{178} & \multicolumn{1}{r}{1276}  & \cellcolor{TealBlue!30}{\textbf{26}} & \cellcolor{TealBlue!30}{\textbf{21}} & \cellcolor{TealBlue!30}{\textbf{7}} & \cellcolor{TealBlue!30}{\textbf{31}} & \cellcolor{TealBlue!30}{\textbf{676.00}} & \cellcolor{TealBlue!30}{\textbf{0}} & - & -\\
\texttt{yeast} & \multicolumn{1}{r}{1484} & \multicolumn{1}{r}{89}  & \cellcolor{TealBlue!30}{\textbf{305}} & \cellcolor{TealBlue!30}{\textbf{203}} & \cellcolor{TealBlue!30}{\textbf{7}} & \cellcolor{TealBlue!30}{\textbf{195}} & \cellcolor{TealBlue!30}{\textbf{563.00}} & \cellcolor{TealBlue!30}{\textbf{0}} & - & -\\
\texttt{zoo-1} & \multicolumn{1}{r}{101} & \multicolumn{1}{r}{20}  & \cellcolor{TealBlue!30}{\textbf{0}} & \cellcolor{TealBlue!30}{\textbf{0}} & \cellcolor{TealBlue!30}{\textbf{1}} & \cellcolor{TealBlue!30}{\textbf{3}} & \cellcolor{TealBlue!30}{\textbf{0.00}} & \cellcolor{TealBlue!30}{\textbf{1}} & \cellcolor{TealBlue!30}{\textbf{0.00}} & \cellcolor{TealBlue!30}{\textbf{1}}\\
\bottomrule
\end{tabular}

\end{normalsize}
\end{center}
\caption{\label{tab:s7} max depth=7}
\end{table}

\begin{table}[htbp]
\begin{center}
\begin{normalsize}
\tabcolsep=3pt
\begin{tabular}{lccrrrrrrrr}
\toprule
& && \multicolumn{8}{c}{\budalg}\\
\cmidrule(rr){4-11}
&\multirow{1}{*}{$\#ex.$} & \multirow{1}{*}{\#feat.} &  \multicolumn{1}{c}{error (f)} & \multicolumn{1}{c}{error (b)} & \multicolumn{1}{c}{depth (b)} & \multicolumn{1}{c}{size (b)} & \multicolumn{1}{c}{time (b)} & \multicolumn{1}{c}{opt} & \multicolumn{1}{c}{time (a)} & \multicolumn{1}{c}{search (a)} \\
\midrule

\texttt{anneal} & \multicolumn{1}{r}{812} & \multicolumn{1}{r}{47}  & \cellcolor{TealBlue!30}{\textbf{59}} & \cellcolor{TealBlue!30}{\textbf{53}} & \cellcolor{TealBlue!30}{\textbf{10}} & \cellcolor{TealBlue!30}{\textbf{131}} & \cellcolor{TealBlue!30}{\textbf{610.00}} & \cellcolor{TealBlue!30}{\textbf{0}} & - & -\\
\texttt{audiology} & \multicolumn{1}{r}{216} & \multicolumn{1}{r}{79}  & \cellcolor{TealBlue!30}{\textbf{0}} & \cellcolor{TealBlue!30}{\textbf{0}} & \cellcolor{TealBlue!30}{\textbf{5}} & \cellcolor{TealBlue!30}{\textbf{21}} & \cellcolor{TealBlue!30}{\textbf{0.54}} & \cellcolor{TealBlue!30}{\textbf{1}} & \cellcolor{TealBlue!30}{\textbf{379.00}} & \cellcolor{TealBlue!30}{\textbf{85985144}}\\
\texttt{australian-credit} & \multicolumn{1}{r}{653} & \multicolumn{1}{r}{73}  & \cellcolor{TealBlue!30}{\textbf{12}} & \cellcolor{TealBlue!30}{\textbf{0}} & \cellcolor{TealBlue!30}{\textbf{7}} & \cellcolor{TealBlue!30}{\textbf{173}} & \cellcolor{TealBlue!30}{\textbf{205.00}} & \cellcolor{TealBlue!30}{\textbf{0}} & - & -\\
\texttt{breast-cancer-un} & \multicolumn{1}{r}{683} & \multicolumn{1}{r}{89}  & \cellcolor{TealBlue!30}{\textbf{0}} & \cellcolor{TealBlue!30}{\textbf{0}} & \cellcolor{TealBlue!30}{\textbf{7}} & \cellcolor{TealBlue!30}{\textbf{71}} & \cellcolor{TealBlue!30}{\textbf{1450.00}} & \cellcolor{TealBlue!30}{\textbf{0}} & - & -\\
\texttt{breast-wisconsin} & \multicolumn{1}{r}{683} & \multicolumn{1}{r}{120}  & \cellcolor{TealBlue!30}{\textbf{0}} & \cellcolor{TealBlue!30}{\textbf{0}} & \cellcolor{TealBlue!30}{\textbf{6}} & \cellcolor{TealBlue!30}{\textbf{41}} & \cellcolor{TealBlue!30}{\textbf{3170.00}} & \cellcolor{TealBlue!30}{\textbf{0}} & - & -\\
\texttt{car-un} & \multicolumn{1}{r}{1728} & \multicolumn{1}{r}{21}  & \cellcolor{TealBlue!30}{\textbf{11}} & \cellcolor{TealBlue!30}{\textbf{0}} & \cellcolor{TealBlue!30}{\textbf{9}} & \cellcolor{TealBlue!30}{\textbf{91}} & \cellcolor{TealBlue!30}{\textbf{1910.00}} & \cellcolor{TealBlue!30}{\textbf{0}} & - & -\\
\texttt{diabetes} & \multicolumn{1}{r}{768} & \multicolumn{1}{r}{112}  & \cellcolor{TealBlue!30}{\textbf{38}} & \cellcolor{TealBlue!30}{\textbf{0}} & \cellcolor{TealBlue!30}{\textbf{8}} & \cellcolor{TealBlue!30}{\textbf{323}} & \cellcolor{TealBlue!30}{\textbf{547.00}} & \cellcolor{TealBlue!30}{\textbf{0}} & - & -\\
\texttt{forest-fires-un} & \multicolumn{1}{r}{517} & \multicolumn{1}{r}{989}  & \cellcolor{TealBlue!30}{\textbf{145}} & \cellcolor{TealBlue!30}{\textbf{113}} & \cellcolor{TealBlue!30}{\textbf{10}} & \cellcolor{TealBlue!30}{\textbf{99}} & \cellcolor{TealBlue!30}{\textbf{1390.00}} & \cellcolor{TealBlue!30}{\textbf{0}} & - & -\\
\texttt{german-credit} & \multicolumn{1}{r}{1000} & \multicolumn{1}{r}{110}  & \cellcolor{TealBlue!30}{\textbf{66}} & \cellcolor{TealBlue!30}{\textbf{0}} & \cellcolor{TealBlue!30}{\textbf{10}} & \cellcolor{TealBlue!30}{\textbf{387}} & \cellcolor{TealBlue!30}{\textbf{87.70}} & \cellcolor{TealBlue!30}{\textbf{0}} & - & -\\
\texttt{heart-cleveland} & \multicolumn{1}{r}{296} & \multicolumn{1}{r}{50}  & \cellcolor{TealBlue!30}{\textbf{0}} & \cellcolor{TealBlue!30}{\textbf{0}} & \cellcolor{TealBlue!30}{\textbf{6}} & \cellcolor{TealBlue!30}{\textbf{75}} & \cellcolor{TealBlue!30}{\textbf{671.00}} & \cellcolor{TealBlue!30}{\textbf{0}} & - & -\\
\texttt{hepatitis} & \multicolumn{1}{r}{137} & \multicolumn{1}{r}{68}  & \cellcolor{TealBlue!30}{\textbf{0}} & \cellcolor{TealBlue!30}{\textbf{0}} & \cellcolor{TealBlue!30}{\textbf{5}} & \cellcolor{TealBlue!30}{\textbf{35}} & \cellcolor{TealBlue!30}{\textbf{0.24}} & \cellcolor{TealBlue!30}{\textbf{1}} & \cellcolor{TealBlue!30}{\textbf{10.60}} & \cellcolor{TealBlue!30}{\textbf{4823276}}\\
\texttt{hypothyroid} & \multicolumn{1}{r}{3247} & \multicolumn{1}{r}{43}  & \cellcolor{TealBlue!30}{\textbf{31}} & \cellcolor{TealBlue!30}{\textbf{31}} & \cellcolor{TealBlue!30}{\textbf{10}} & \cellcolor{TealBlue!30}{\textbf{131}} & \cellcolor{TealBlue!30}{\textbf{0.00}} & \cellcolor{TealBlue!30}{\textbf{0}} & - & -\\
\texttt{ionosphere} & \multicolumn{1}{r}{351} & \multicolumn{1}{r}{444}  & \cellcolor{TealBlue!30}{\textbf{0}} & \cellcolor{TealBlue!30}{\textbf{0}} & \cellcolor{TealBlue!30}{\textbf{6}} & \cellcolor{TealBlue!30}{\textbf{55}} & \cellcolor{TealBlue!30}{\textbf{5.47}} & \cellcolor{TealBlue!30}{\textbf{0}} & - & -\\
\texttt{kr-vs-kp} & \multicolumn{1}{r}{3196} & \multicolumn{1}{r}{37}  & \cellcolor{TealBlue!30}{\textbf{12}} & \cellcolor{TealBlue!30}{\textbf{0}} & \cellcolor{TealBlue!30}{\textbf{10}} & \cellcolor{TealBlue!30}{\textbf{151}} & \cellcolor{TealBlue!30}{\textbf{1900.00}} & \cellcolor{TealBlue!30}{\textbf{0}} & - & -\\
\texttt{letter} & \multicolumn{1}{r}{20000} & \multicolumn{1}{r}{224}  & \cellcolor{TealBlue!30}{\textbf{20}} & \cellcolor{TealBlue!30}{\textbf{0}} & \cellcolor{TealBlue!30}{\textbf{10}} & \cellcolor{TealBlue!30}{\textbf{347}} & \cellcolor{TealBlue!30}{\textbf{84.00}} & \cellcolor{TealBlue!30}{\textbf{0}} & - & -\\
\texttt{lymph} & \multicolumn{1}{r}{148} & \multicolumn{1}{r}{41}  & \cellcolor{TealBlue!30}{\textbf{0}} & \cellcolor{TealBlue!30}{\textbf{0}} & \cellcolor{TealBlue!30}{\textbf{5}} & \cellcolor{TealBlue!30}{\textbf{35}} & \cellcolor{TealBlue!30}{\textbf{1.16}} & \cellcolor{TealBlue!30}{\textbf{1}} & \cellcolor{TealBlue!30}{\textbf{34.60}} & \cellcolor{TealBlue!30}{\textbf{12447337}}\\
\texttt{mushroom} & \multicolumn{1}{r}{8124} & \multicolumn{1}{r}{91}  & \cellcolor{TealBlue!30}{\textbf{0}} & \cellcolor{TealBlue!30}{\textbf{0}} & \cellcolor{TealBlue!30}{\textbf{4}} & \cellcolor{TealBlue!30}{\textbf{15}} & \cellcolor{TealBlue!30}{\textbf{0.27}} & \cellcolor{TealBlue!30}{\textbf{1}} & \cellcolor{TealBlue!30}{\textbf{58.30}} & \cellcolor{TealBlue!30}{\textbf{2017076}}\\
\texttt{pendigits} & \multicolumn{1}{r}{7494} & \multicolumn{1}{r}{216}  & \cellcolor{TealBlue!30}{\textbf{0}} & \cellcolor{TealBlue!30}{\textbf{0}} & \cellcolor{TealBlue!30}{\textbf{6}} & \cellcolor{TealBlue!30}{\textbf{41}} & \cellcolor{TealBlue!30}{\textbf{2310.00}} & \cellcolor{TealBlue!30}{\textbf{0}} & - & -\\
\texttt{primary-tumor} & \multicolumn{1}{r}{336} & \multicolumn{1}{r}{16}  & \cellcolor{TealBlue!30}{\textbf{20}} & \cellcolor{TealBlue!30}{\textbf{15}} & \cellcolor{TealBlue!30}{\textbf{10}} & \cellcolor{TealBlue!30}{\textbf{153}} & \cellcolor{TealBlue!30}{\textbf{3.36}} & \cellcolor{TealBlue!30}{\textbf{0}} & - & -\\
\texttt{segment} & \multicolumn{1}{r}{2310} & \multicolumn{1}{r}{234}  & \cellcolor{TealBlue!30}{\textbf{0}} & \cellcolor{TealBlue!30}{\textbf{0}} & \cellcolor{TealBlue!30}{\textbf{4}} & \cellcolor{TealBlue!30}{\textbf{11}} & \cellcolor{TealBlue!30}{\textbf{0.00}} & \cellcolor{TealBlue!30}{\textbf{1}} & \cellcolor{TealBlue!30}{\textbf{72.80}} & \cellcolor{TealBlue!30}{\textbf{3955322}}\\
\texttt{soybean} & \multicolumn{1}{r}{630} & \multicolumn{1}{r}{34}  & \cellcolor{TealBlue!30}{\textbf{2}} & \cellcolor{TealBlue!30}{\textbf{2}} & \cellcolor{TealBlue!30}{\textbf{10}} & \cellcolor{TealBlue!30}{\textbf{87}} & \cellcolor{TealBlue!30}{\textbf{0.00}} & \cellcolor{TealBlue!30}{\textbf{0}} & - & -\\
\texttt{splice-1} & \multicolumn{1}{r}{3190} & \multicolumn{1}{r}{227}  & \cellcolor{TealBlue!30}{\textbf{12}} & \cellcolor{TealBlue!30}{\textbf{5}} & \cellcolor{TealBlue!30}{\textbf{10}} & \cellcolor{TealBlue!30}{\textbf{195}} & \cellcolor{TealBlue!30}{\textbf{1540.00}} & \cellcolor{TealBlue!30}{\textbf{0}} & - & -\\
\texttt{taiwan\_binarised} & \multicolumn{1}{r}{30000} & \multicolumn{1}{r}{198}  & \cellcolor{TealBlue!30}{\textbf{4666}} & \cellcolor{TealBlue!30}{\textbf{4564}} & \cellcolor{TealBlue!30}{\textbf{10}} & \cellcolor{TealBlue!30}{\textbf{999}} & \cellcolor{TealBlue!30}{\textbf{418.00}} & \cellcolor{TealBlue!30}{\textbf{0}} & - & -\\
\texttt{tic-tac-toe} & \multicolumn{1}{r}{958} & \multicolumn{1}{r}{18}  & \cellcolor{TealBlue!30}{\textbf{6}} & \cellcolor{TealBlue!30}{\textbf{0}} & \cellcolor{TealBlue!30}{\textbf{8}} & \cellcolor{TealBlue!30}{\textbf{85}} & \cellcolor{TealBlue!30}{\textbf{799.00}} & \cellcolor{TealBlue!30}{\textbf{0}} & - & -\\
\texttt{vehicle} & \multicolumn{1}{r}{846} & \multicolumn{1}{r}{252}  & \cellcolor{TealBlue!30}{\textbf{0}} & \cellcolor{TealBlue!30}{\textbf{0}} & \cellcolor{TealBlue!30}{\textbf{6}} & \cellcolor{TealBlue!30}{\textbf{73}} & \cellcolor{TealBlue!30}{\textbf{0.22}} & \cellcolor{TealBlue!30}{\textbf{0}} & - & -\\
\texttt{vote} & \multicolumn{1}{r}{435} & \multicolumn{1}{r}{32}  & \cellcolor{TealBlue!30}{\textbf{0}} & \cellcolor{TealBlue!30}{\textbf{0}} & \cellcolor{TealBlue!30}{\textbf{6}} & \cellcolor{TealBlue!30}{\textbf{39}} & \cellcolor{TealBlue!30}{\textbf{9.82}} & \cellcolor{TealBlue!30}{\textbf{1}} & \cellcolor{TealBlue!30}{\textbf{1800.00}} & \cellcolor{TealBlue!30}{\textbf{632823064}}\\
\texttt{wine1-un} & \multicolumn{1}{r}{178} & \multicolumn{1}{r}{1276}  & \cellcolor{TealBlue!30}{\textbf{25}} & \cellcolor{TealBlue!30}{\textbf{22}} & \cellcolor{TealBlue!30}{\textbf{10}} & \cellcolor{TealBlue!30}{\textbf{33}} & \cellcolor{TealBlue!30}{\textbf{682.00}} & \cellcolor{TealBlue!30}{\textbf{0}} & - & -\\
\texttt{wine2-un} & \multicolumn{1}{r}{178} & \multicolumn{1}{r}{1276}  & \cellcolor{TealBlue!30}{\textbf{29}} & \cellcolor{TealBlue!30}{\textbf{24}} & \cellcolor{TealBlue!30}{\textbf{10}} & \cellcolor{TealBlue!30}{\textbf{37}} & \cellcolor{TealBlue!30}{\textbf{526.00}} & \cellcolor{TealBlue!30}{\textbf{0}} & - & -\\
\texttt{wine3-un} & \multicolumn{1}{r}{178} & \multicolumn{1}{r}{1276}  & \cellcolor{TealBlue!30}{\textbf{19}} & \cellcolor{TealBlue!30}{\textbf{16}} & \cellcolor{TealBlue!30}{\textbf{10}} & \cellcolor{TealBlue!30}{\textbf{35}} & \cellcolor{TealBlue!30}{\textbf{342.00}} & \cellcolor{TealBlue!30}{\textbf{0}} & - & -\\
\texttt{yeast} & \multicolumn{1}{r}{1484} & \multicolumn{1}{r}{89}  & \cellcolor{TealBlue!30}{\textbf{180}} & \cellcolor{TealBlue!30}{\textbf{104}} & \cellcolor{TealBlue!30}{\textbf{10}} & \cellcolor{TealBlue!30}{\textbf{497}} & \cellcolor{TealBlue!30}{\textbf{110.00}} & \cellcolor{TealBlue!30}{\textbf{0}} & - & -\\
\texttt{zoo-1} & \multicolumn{1}{r}{101} & \multicolumn{1}{r}{20}  & \cellcolor{TealBlue!30}{\textbf{0}} & \cellcolor{TealBlue!30}{\textbf{0}} & \cellcolor{TealBlue!30}{\textbf{1}} & \cellcolor{TealBlue!30}{\textbf{3}} & \cellcolor{TealBlue!30}{\textbf{0.00}} & \cellcolor{TealBlue!30}{\textbf{1}} & \cellcolor{TealBlue!30}{\textbf{0.00}} & \cellcolor{TealBlue!30}{\textbf{1}}\\
\bottomrule
\end{tabular}

\end{normalsize}
\end{center}
\caption{\label{tab:s10} max depth=10}
\end{table}

%
%
% \begin{table}[htbp]
% \begin{center}
% \begin{normalsize}
% \tabcolsep=5pt
% \begin{tabular}{lccrrrrrrrrr}
\toprule
& && \multicolumn{3}{c}{entropy} & \multicolumn{3}{c}{\budalg} & \multicolumn{3}{c}{error}\\
\cmidrule(rr){4-6}\cmidrule(rr){7-9}\cmidrule(rr){10-12}
&\multirow{1}{*}{$\#ex.$} & \multirow{1}{*}{\#feat.} &  \multicolumn{1}{c}{opt} & \multicolumn{1}{c}{error} & \multicolumn{1}{c}{time} & \multicolumn{1}{c}{opt} & \multicolumn{1}{c}{error} & \multicolumn{1}{c}{time} & \multicolumn{1}{c}{opt} & \multicolumn{1}{c}{error} & \multicolumn{1}{c}{time} \\
\midrule

\texttt{anneal} & \multicolumn{1}{r}{812} & \multicolumn{1}{r}{47}  & \cellcolor{TealBlue!30}{1} & \cellcolor{TealBlue!30}{112} & 0.3 & \cellcolor{TealBlue!30}{1} & \cellcolor{TealBlue!30}{112} & \cellcolor{TealBlue!30}{\textbf{0.2}} & \cellcolor{TealBlue!30}{1} & \cellcolor{TealBlue!30}{112} & 0.2\\
\texttt{audiology} & \multicolumn{1}{r}{216} & \multicolumn{1}{r}{79}  & \cellcolor{TealBlue!30}{1} & \cellcolor{TealBlue!30}{5} & 0.6 & \cellcolor{TealBlue!30}{1} & \cellcolor{TealBlue!30}{5} & 0.3 & \cellcolor{TealBlue!30}{1} & \cellcolor{TealBlue!30}{5} & \cellcolor{TealBlue!30}{\textbf{0.2}}\\
\texttt{australian-credit} & \multicolumn{1}{r}{653} & \multicolumn{1}{r}{73}  & \cellcolor{TealBlue!30}{1} & \cellcolor{TealBlue!30}{73} & 1.1 & \cellcolor{TealBlue!30}{1} & \cellcolor{TealBlue!30}{73} & 0.6 & \cellcolor{TealBlue!30}{1} & \cellcolor{TealBlue!30}{73} & \cellcolor{TealBlue!30}{\textbf{0.6}}\\
\texttt{breast-cancer-un} & \multicolumn{1}{r}{683} & \multicolumn{1}{r}{89}  & \cellcolor{TealBlue!30}{1} & \cellcolor{TealBlue!30}{24} & 0.3 & \cellcolor{TealBlue!30}{1} & \cellcolor{TealBlue!30}{24} & 0.1 & \cellcolor{TealBlue!30}{1} & \cellcolor{TealBlue!30}{24} & \cellcolor{TealBlue!30}{\textbf{0.1}}\\
\texttt{breast-wisconsin} & \multicolumn{1}{r}{683} & \multicolumn{1}{r}{120}  & \cellcolor{TealBlue!30}{1} & \cellcolor{TealBlue!30}{15} & 0.8 & \cellcolor{TealBlue!30}{1} & \cellcolor{TealBlue!30}{15} & 0.4 & \cellcolor{TealBlue!30}{1} & \cellcolor{TealBlue!30}{15} & \cellcolor{TealBlue!30}{\textbf{0.4}}\\
\texttt{car-un} & \multicolumn{1}{r}{1728} & \multicolumn{1}{r}{21}  & \cellcolor{TealBlue!30}{1} & \cellcolor{TealBlue!30}{192} & 0.0 & \cellcolor{TealBlue!30}{1} & \cellcolor{TealBlue!30}{192} & \cellcolor{TealBlue!30}{\textbf{0.0}} & \cellcolor{TealBlue!30}{1} & \cellcolor{TealBlue!30}{192} & 0.0\\
\texttt{diabetes} & \multicolumn{1}{r}{768} & \multicolumn{1}{r}{112}  & \cellcolor{TealBlue!30}{1} & \cellcolor{TealBlue!30}{162} & 1.0 & \cellcolor{TealBlue!30}{1} & \cellcolor{TealBlue!30}{162} & 0.5 & \cellcolor{TealBlue!30}{1} & \cellcolor{TealBlue!30}{162} & \cellcolor{TealBlue!30}{\textbf{0.5}}\\
\texttt{forest-fires-un} & \multicolumn{1}{r}{517} & \multicolumn{1}{r}{989}  & \cellcolor{TealBlue!30}{1} & \cellcolor{TealBlue!30}{193} & 149.0 & \cellcolor{TealBlue!30}{1} & \cellcolor{TealBlue!30}{193} & 69.2 & \cellcolor{TealBlue!30}{1} & \cellcolor{TealBlue!30}{193} & \cellcolor{TealBlue!30}{\textbf{59.9}}\\
\texttt{german-credit} & \multicolumn{1}{r}{1000} & \multicolumn{1}{r}{110}  & \cellcolor{TealBlue!30}{1} & \cellcolor{TealBlue!30}{236} & 1.2 & \cellcolor{TealBlue!30}{1} & \cellcolor{TealBlue!30}{236} & 0.6 & \cellcolor{TealBlue!30}{1} & \cellcolor{TealBlue!30}{236} & \cellcolor{TealBlue!30}{\textbf{0.5}}\\
\texttt{heart-cleveland} & \multicolumn{1}{r}{296} & \multicolumn{1}{r}{50}  & \cellcolor{TealBlue!30}{1} & \cellcolor{TealBlue!30}{41} & 0.5 & \cellcolor{TealBlue!30}{1} & \cellcolor{TealBlue!30}{41} & 0.2 & \cellcolor{TealBlue!30}{1} & \cellcolor{TealBlue!30}{41} & \cellcolor{TealBlue!30}{\textbf{0.2}}\\
\texttt{hepatitis} & \multicolumn{1}{r}{137} & \multicolumn{1}{r}{68}  & \cellcolor{TealBlue!30}{1} & \cellcolor{TealBlue!30}{10} & 0.1 & \cellcolor{TealBlue!30}{1} & \cellcolor{TealBlue!30}{10} & 0.1 & \cellcolor{TealBlue!30}{1} & \cellcolor{TealBlue!30}{10} & \cellcolor{TealBlue!30}{\textbf{0.1}}\\
\texttt{hypothyroid} & \multicolumn{1}{r}{3247} & \multicolumn{1}{r}{43}  & \cellcolor{TealBlue!30}{1} & \cellcolor{TealBlue!30}{61} & 0.9 & \cellcolor{TealBlue!30}{1} & \cellcolor{TealBlue!30}{61} & \cellcolor{TealBlue!30}{\textbf{0.6}} & \cellcolor{TealBlue!30}{1} & \cellcolor{TealBlue!30}{61} & 0.7\\
\texttt{ionosphere} & \multicolumn{1}{r}{351} & \multicolumn{1}{r}{444}  & \cellcolor{TealBlue!30}{1} & \cellcolor{TealBlue!30}{22} & 53.3 & \cellcolor{TealBlue!30}{1} & \cellcolor{TealBlue!30}{22} & 24.6 & \cellcolor{TealBlue!30}{1} & \cellcolor{TealBlue!30}{22} & \cellcolor{TealBlue!30}{\textbf{22.5}}\\
\texttt{kr-vs-kp} & \multicolumn{1}{r}{3196} & \multicolumn{1}{r}{37}  & \cellcolor{TealBlue!30}{1} & \cellcolor{TealBlue!30}{198} & 0.5 & \cellcolor{TealBlue!30}{1} & \cellcolor{TealBlue!30}{198} & \cellcolor{TealBlue!30}{\textbf{0.4}} & \cellcolor{TealBlue!30}{1} & \cellcolor{TealBlue!30}{198} & 0.4\\
\texttt{letter} & \multicolumn{1}{r}{20000} & \multicolumn{1}{r}{224}  & \cellcolor{TealBlue!30}{1} & \cellcolor{TealBlue!30}{369} & 137.0 & \cellcolor{TealBlue!30}{1} & \cellcolor{TealBlue!30}{369} & \cellcolor{TealBlue!30}{\textbf{57.1}} & \cellcolor{TealBlue!30}{1} & \cellcolor{TealBlue!30}{369} & 138.0\\
\texttt{lymph} & \multicolumn{1}{r}{148} & \multicolumn{1}{r}{41}  & \cellcolor{TealBlue!30}{1} & \cellcolor{TealBlue!30}{12} & 0.1 & \cellcolor{TealBlue!30}{1} & \cellcolor{TealBlue!30}{12} & 0.1 & \cellcolor{TealBlue!30}{1} & \cellcolor{TealBlue!30}{12} & \cellcolor{TealBlue!30}{\textbf{0.1}}\\
\texttt{mushroom} & \multicolumn{1}{r}{8124} & \multicolumn{1}{r}{91}  & \cellcolor{TealBlue!30}{1} & \cellcolor{TealBlue!30}{8} & 2.8 & \cellcolor{TealBlue!30}{1} & \cellcolor{TealBlue!30}{8} & \cellcolor{TealBlue!30}{\textbf{1.7}} & \cellcolor{TealBlue!30}{1} & \cellcolor{TealBlue!30}{8} & 2.4\\
\texttt{pendigits} & \multicolumn{1}{r}{7494} & \multicolumn{1}{r}{216}  & \cellcolor{TealBlue!30}{1} & \cellcolor{TealBlue!30}{47} & 32.6 & \cellcolor{TealBlue!30}{1} & \cellcolor{TealBlue!30}{47} & \cellcolor{TealBlue!30}{\textbf{18.7}} & \cellcolor{TealBlue!30}{1} & \cellcolor{TealBlue!30}{47} & 35.5\\
\texttt{primary-tumor} & \multicolumn{1}{r}{336} & \multicolumn{1}{r}{16}  & \cellcolor{TealBlue!30}{1} & \cellcolor{TealBlue!30}{46} & 0.0 & \cellcolor{TealBlue!30}{1} & \cellcolor{TealBlue!30}{46} & \cellcolor{TealBlue!30}{\textbf{0.0}} & \cellcolor{TealBlue!30}{1} & \cellcolor{TealBlue!30}{46} & 0.0\\
\texttt{segment} & \multicolumn{1}{r}{2310} & \multicolumn{1}{r}{234}  & \cellcolor{TealBlue!30}{1} & \cellcolor{TealBlue!30}{0} & 0.4 & \cellcolor{TealBlue!30}{1} & \cellcolor{TealBlue!30}{0} & \cellcolor{TealBlue!30}{\textbf{0.3}} & \cellcolor{TealBlue!30}{1} & \cellcolor{TealBlue!30}{0} & 0.3\\
\texttt{soybean} & \multicolumn{1}{r}{630} & \multicolumn{1}{r}{34}  & \cellcolor{TealBlue!30}{1} & \cellcolor{TealBlue!30}{29} & 0.1 & \cellcolor{TealBlue!30}{1} & \cellcolor{TealBlue!30}{29} & 0.0 & \cellcolor{TealBlue!30}{1} & \cellcolor{TealBlue!30}{29} & \cellcolor{TealBlue!30}{\textbf{0.0}}\\
\texttt{splice-1} & \multicolumn{1}{r}{3190} & \multicolumn{1}{r}{227}  & \cellcolor{TealBlue!30}{1} & \cellcolor{TealBlue!30}{224} & 23.6 & \cellcolor{TealBlue!30}{1} & \cellcolor{TealBlue!30}{224} & 14.4 & \cellcolor{TealBlue!30}{1} & \cellcolor{TealBlue!30}{224} & \cellcolor{TealBlue!30}{\textbf{14.2}}\\
\texttt{taiwan\_binarised} & \multicolumn{1}{r}{30000} & \multicolumn{1}{r}{198}  & \cellcolor{TealBlue!30}{1} & \cellcolor{TealBlue!30}{5326} & 138.0 & \cellcolor{TealBlue!30}{1} & \cellcolor{TealBlue!30}{5326} & \cellcolor{TealBlue!30}{\textbf{45.8}} & \cellcolor{TealBlue!30}{1} & \cellcolor{TealBlue!30}{5326} & 146.0\\
\texttt{tic-tac-toe} & \multicolumn{1}{r}{958} & \multicolumn{1}{r}{18}  & \cellcolor{TealBlue!30}{1} & \cellcolor{TealBlue!30}{216} & 0.0 & \cellcolor{TealBlue!30}{1} & \cellcolor{TealBlue!30}{216} & \cellcolor{TealBlue!30}{\textbf{0.0}} & \cellcolor{TealBlue!30}{1} & \cellcolor{TealBlue!30}{216} & 0.0\\
\texttt{vehicle} & \multicolumn{1}{r}{846} & \multicolumn{1}{r}{252}  & \cellcolor{TealBlue!30}{1} & \cellcolor{TealBlue!30}{26} & 8.2 & \cellcolor{TealBlue!30}{1} & \cellcolor{TealBlue!30}{26} & \cellcolor{TealBlue!30}{\textbf{4.4}} & \cellcolor{TealBlue!30}{1} & \cellcolor{TealBlue!30}{26} & 4.4\\
\texttt{vote} & \multicolumn{1}{r}{435} & \multicolumn{1}{r}{32}  & \cellcolor{TealBlue!30}{1} & \cellcolor{TealBlue!30}{12} & 0.1 & \cellcolor{TealBlue!30}{1} & \cellcolor{TealBlue!30}{12} & 0.0 & \cellcolor{TealBlue!30}{1} & \cellcolor{TealBlue!30}{12} & \cellcolor{TealBlue!30}{\textbf{0.0}}\\
\texttt{wine1-un} & \multicolumn{1}{r}{178} & \multicolumn{1}{r}{1276}  & \cellcolor{TealBlue!30}{1} & \cellcolor{TealBlue!30}{43} & 256.0 & \cellcolor{TealBlue!30}{1} & \cellcolor{TealBlue!30}{43} & 123.0 & \cellcolor{TealBlue!30}{1} & \cellcolor{TealBlue!30}{43} & \cellcolor{TealBlue!30}{\textbf{115.0}}\\
\texttt{wine2-un} & \multicolumn{1}{r}{178} & \multicolumn{1}{r}{1276}  & \cellcolor{TealBlue!30}{1} & \cellcolor{TealBlue!30}{49} & 256.0 & \cellcolor{TealBlue!30}{1} & \cellcolor{TealBlue!30}{49} & 122.0 & \cellcolor{TealBlue!30}{1} & \cellcolor{TealBlue!30}{49} & \cellcolor{TealBlue!30}{\textbf{110.0}}\\
\texttt{wine3-un} & \multicolumn{1}{r}{178} & \multicolumn{1}{r}{1276}  & \cellcolor{TealBlue!30}{1} & \cellcolor{TealBlue!30}{33} & 255.0 & \cellcolor{TealBlue!30}{1} & \cellcolor{TealBlue!30}{33} & 122.0 & \cellcolor{TealBlue!30}{1} & \cellcolor{TealBlue!30}{33} & \cellcolor{TealBlue!30}{\textbf{104.0}}\\
\texttt{yeast} & \multicolumn{1}{r}{1484} & \multicolumn{1}{r}{89}  & \cellcolor{TealBlue!30}{1} & \cellcolor{TealBlue!30}{403} & 0.7 & \cellcolor{TealBlue!30}{1} & \cellcolor{TealBlue!30}{403} & \cellcolor{TealBlue!30}{\textbf{0.4}} & \cellcolor{TealBlue!30}{1} & \cellcolor{TealBlue!30}{403} & 0.5\\
\texttt{zoo-1} & \multicolumn{1}{r}{101} & \multicolumn{1}{r}{20}  & \cellcolor{TealBlue!30}{1} & \cellcolor{TealBlue!30}{0} & 0.0 & \cellcolor{TealBlue!30}{1} & \cellcolor{TealBlue!30}{0} & \cellcolor{TealBlue!30}{\textbf{0.0}} & \cellcolor{TealBlue!30}{1} & \cellcolor{TealBlue!30}{0} & 0.0\\
\bottomrule
\end{tabular}

% \end{normalsize}
% \end{center}
% \caption{\label{tab:ha3} Comparison of heuristics (max depth=3)}
% \end{table}
%
% \begin{table}[htbp]
% \begin{center}
% \begin{normalsize}
% \tabcolsep=5pt
% \begin{tabular}{lccrrrrrrrrr}
\toprule
& && \multicolumn{3}{c}{entropy} & \multicolumn{3}{c}{\budalg} & \multicolumn{3}{c}{error}\\
\cmidrule(rr){4-6}\cmidrule(rr){7-9}\cmidrule(rr){10-12}
&\multirow{1}{*}{$\#ex.$} & \multirow{1}{*}{\#feat.} &  \multicolumn{1}{c}{opt} & \multicolumn{1}{c}{error} & \multicolumn{1}{c}{time} & \multicolumn{1}{c}{opt} & \multicolumn{1}{c}{error} & \multicolumn{1}{c}{time} & \multicolumn{1}{c}{opt} & \multicolumn{1}{c}{error} & \multicolumn{1}{c}{time} \\
\midrule

\texttt{anneal} & \multicolumn{1}{r}{812} & \multicolumn{1}{r}{47}  & \cellcolor{TealBlue!30}{1} & \cellcolor{TealBlue!30}{91} & 23.2 & \cellcolor{TealBlue!30}{1} & \cellcolor{TealBlue!30}{91} & \cellcolor{TealBlue!30}{14.1} & \cellcolor{TealBlue!30}{1} & \cellcolor{TealBlue!30}{91} & \cellcolor{TealBlue!30}{14.1}\\
\texttt{audiology} & \multicolumn{1}{r}{216} & \multicolumn{1}{r}{79}  & \cellcolor{TealBlue!30}{1} & \cellcolor{TealBlue!30}{1} & 63.6 & \cellcolor{TealBlue!30}{1} & \cellcolor{TealBlue!30}{1} & 31.2 & \cellcolor{TealBlue!30}{1} & \cellcolor{TealBlue!30}{1} & \cellcolor{TealBlue!30}{\textbf{26.4}}\\
\texttt{australian-credit} & \multicolumn{1}{r}{653} & \multicolumn{1}{r}{73}  & \cellcolor{TealBlue!30}{1} & \cellcolor{TealBlue!30}{56} & 166.0 & \cellcolor{TealBlue!30}{1} & \cellcolor{TealBlue!30}{56} & 83.4 & \cellcolor{TealBlue!30}{1} & \cellcolor{TealBlue!30}{56} & \cellcolor{TealBlue!30}{\textbf{73.8}}\\
\texttt{breast-cancer-un} & \multicolumn{1}{r}{683} & \multicolumn{1}{r}{89}  & \cellcolor{TealBlue!30}{1} & \cellcolor{TealBlue!30}{16} & 21.6 & \cellcolor{TealBlue!30}{1} & \cellcolor{TealBlue!30}{16} & 12.3 & \cellcolor{TealBlue!30}{1} & \cellcolor{TealBlue!30}{16} & \cellcolor{TealBlue!30}{\textbf{11.2}}\\
\texttt{breast-wisconsin} & \multicolumn{1}{r}{683} & \multicolumn{1}{r}{120}  & \cellcolor{TealBlue!30}{1} & \cellcolor{TealBlue!30}{7} & 80.9 & \cellcolor{TealBlue!30}{1} & \cellcolor{TealBlue!30}{7} & 42.5 & \cellcolor{TealBlue!30}{1} & \cellcolor{TealBlue!30}{7} & \cellcolor{TealBlue!30}{\textbf{37.8}}\\
\texttt{car-un} & \multicolumn{1}{r}{1728} & \multicolumn{1}{r}{21}  & \cellcolor{TealBlue!30}{1} & \cellcolor{TealBlue!30}{136} & 0.4 & \cellcolor{TealBlue!30}{1} & \cellcolor{TealBlue!30}{136} & \cellcolor{TealBlue!30}{\textbf{0.3}} & \cellcolor{TealBlue!30}{1} & \cellcolor{TealBlue!30}{136} & 0.3\\
\texttt{diabetes} & \multicolumn{1}{r}{768} & \multicolumn{1}{r}{112}  & \cellcolor{TealBlue!30}{1} & \cellcolor{TealBlue!30}{137} & 142.0 & \cellcolor{TealBlue!30}{1} & \cellcolor{TealBlue!30}{137} & 71.1 & \cellcolor{TealBlue!30}{1} & \cellcolor{TealBlue!30}{137} & \cellcolor{TealBlue!30}{\textbf{65.8}}\\
\texttt{forest-fires-un} & \multicolumn{1}{r}{517} & \multicolumn{1}{r}{989}  & \cellcolor{TealBlue!30}{0} & \cellcolor{TealBlue!30}{173} & 244.0 & \cellcolor{TealBlue!30}{0} & \cellcolor{TealBlue!30}{173} & 50.9 & \cellcolor{TealBlue!30}{0} & \cellcolor{TealBlue!30}{173} & \cellcolor{TealBlue!30}{\textbf{39.0}}\\
\texttt{german-credit} & \multicolumn{1}{r}{1000} & \multicolumn{1}{r}{110}  & \cellcolor{TealBlue!30}{1} & \cellcolor{TealBlue!30}{204} & 163.0 & \cellcolor{TealBlue!30}{1} & \cellcolor{TealBlue!30}{204} & 81.0 & \cellcolor{TealBlue!30}{1} & \cellcolor{TealBlue!30}{204} & \cellcolor{TealBlue!30}{\textbf{65.1}}\\
\texttt{heart-cleveland} & \multicolumn{1}{r}{296} & \multicolumn{1}{r}{50}  & \cellcolor{TealBlue!30}{1} & \cellcolor{TealBlue!30}{25} & 58.8 & \cellcolor{TealBlue!30}{1} & \cellcolor{TealBlue!30}{25} & 25.4 & \cellcolor{TealBlue!30}{1} & \cellcolor{TealBlue!30}{25} & \cellcolor{TealBlue!30}{\textbf{22.0}}\\
\texttt{hepatitis} & \multicolumn{1}{r}{137} & \multicolumn{1}{r}{68}  & \cellcolor{TealBlue!30}{1} & \cellcolor{TealBlue!30}{3} & 7.6 & \cellcolor{TealBlue!30}{1} & \cellcolor{TealBlue!30}{3} & 3.6 & \cellcolor{TealBlue!30}{1} & \cellcolor{TealBlue!30}{3} & \cellcolor{TealBlue!30}{\textbf{3.5}}\\
\texttt{hypothyroid} & \multicolumn{1}{r}{3247} & \multicolumn{1}{r}{43}  & \cellcolor{TealBlue!30}{1} & \cellcolor{TealBlue!30}{53} & 61.6 & \cellcolor{TealBlue!30}{1} & \cellcolor{TealBlue!30}{53} & \cellcolor{TealBlue!30}{\textbf{45.0}} & \cellcolor{TealBlue!30}{1} & \cellcolor{TealBlue!30}{53} & 48.5\\
\texttt{ionosphere} & \multicolumn{1}{r}{351} & \multicolumn{1}{r}{444}  & \cellcolor{TealBlue!30}{0} & 8 & 123.0 & \cellcolor{TealBlue!30}{0} & 8 & \cellcolor{TealBlue!30}{\textbf{58.4}} & \cellcolor{TealBlue!30}{0} & \cellcolor{TealBlue!30}{\textbf{7}} & 3410.0\\
\texttt{kr-vs-kp} & \multicolumn{1}{r}{3196} & \multicolumn{1}{r}{37}  & \cellcolor{TealBlue!30}{1} & \cellcolor{TealBlue!30}{144} & 38.4 & \cellcolor{TealBlue!30}{1} & \cellcolor{TealBlue!30}{144} & \cellcolor{TealBlue!30}{\textbf{27.7}} & \cellcolor{TealBlue!30}{1} & \cellcolor{TealBlue!30}{144} & 28.8\\
\texttt{letter} & \multicolumn{1}{r}{20000} & \multicolumn{1}{r}{224}  & \cellcolor{TealBlue!30}{0} & \cellcolor{TealBlue!30}{261} & 1190.0 & \cellcolor{TealBlue!30}{0} & \cellcolor{TealBlue!30}{261} & \cellcolor{TealBlue!30}{\textbf{410.0}} & \cellcolor{TealBlue!30}{0} & 263 & 3040.0\\
\texttt{lymph} & \multicolumn{1}{r}{148} & \multicolumn{1}{r}{41}  & \cellcolor{TealBlue!30}{1} & \cellcolor{TealBlue!30}{3} & 5.3 & \cellcolor{TealBlue!30}{1} & \cellcolor{TealBlue!30}{3} & 2.7 & \cellcolor{TealBlue!30}{1} & \cellcolor{TealBlue!30}{3} & \cellcolor{TealBlue!30}{\textbf{2.3}}\\
\texttt{mushroom} & \multicolumn{1}{r}{8124} & \multicolumn{1}{r}{91}  & \cellcolor{TealBlue!30}{1} & \cellcolor{TealBlue!30}{0} & 0.0 & \cellcolor{TealBlue!30}{1} & \cellcolor{TealBlue!30}{0} & \cellcolor{TealBlue!30}{\textbf{0.0}} & \cellcolor{TealBlue!30}{1} & \cellcolor{TealBlue!30}{0} & 0.0\\
\texttt{pendigits} & \multicolumn{1}{r}{7494} & \multicolumn{1}{r}{216}  & 0 & \cellcolor{TealBlue!30}{13} & \cellcolor{TealBlue!30}{\textbf{1250.0}} & \cellcolor{TealBlue!30}{1} & \cellcolor{TealBlue!30}{13} & 3040.0 & \cellcolor{TealBlue!30}{1} & \cellcolor{TealBlue!30}{13} & 3560.0\\
\texttt{primary-tumor} & \multicolumn{1}{r}{336} & \multicolumn{1}{r}{16}  & \cellcolor{TealBlue!30}{1} & \cellcolor{TealBlue!30}{34} & 0.6 & \cellcolor{TealBlue!30}{1} & \cellcolor{TealBlue!30}{34} & \cellcolor{TealBlue!30}{\textbf{0.3}} & \cellcolor{TealBlue!30}{1} & \cellcolor{TealBlue!30}{34} & 0.3\\
\texttt{segment} & \multicolumn{1}{r}{2310} & \multicolumn{1}{r}{234}  & \cellcolor{TealBlue!30}{1} & \cellcolor{TealBlue!30}{0} & 0.0 & \cellcolor{TealBlue!30}{1} & \cellcolor{TealBlue!30}{0} & \cellcolor{TealBlue!30}{\textbf{0.0}} & \cellcolor{TealBlue!30}{1} & \cellcolor{TealBlue!30}{0} & 0.0\\
\texttt{soybean} & \multicolumn{1}{r}{630} & \multicolumn{1}{r}{34}  & \cellcolor{TealBlue!30}{1} & \cellcolor{TealBlue!30}{14} & 2.9 & \cellcolor{TealBlue!30}{1} & \cellcolor{TealBlue!30}{14} & 1.7 & \cellcolor{TealBlue!30}{1} & \cellcolor{TealBlue!30}{14} & \cellcolor{TealBlue!30}{\textbf{1.6}}\\
\texttt{splice-1} & \multicolumn{1}{r}{3190} & \multicolumn{1}{r}{227}  & \cellcolor{TealBlue!30}{0} & \cellcolor{TealBlue!30}{141} & 13.5 & \cellcolor{TealBlue!30}{0} & \cellcolor{TealBlue!30}{141} & 7.7 & \cellcolor{TealBlue!30}{0} & \cellcolor{TealBlue!30}{141} & \cellcolor{TealBlue!30}{\textbf{6.2}}\\
\texttt{taiwan\_binarised} & \multicolumn{1}{r}{30000} & \multicolumn{1}{r}{198}  & \cellcolor{TealBlue!30}{0} & \cellcolor{TealBlue!30}{5273} & 16.3 & \cellcolor{TealBlue!30}{0} & \cellcolor{TealBlue!30}{5273} & \cellcolor{TealBlue!30}{\textbf{7.9}} & \cellcolor{TealBlue!30}{0} & \cellcolor{TealBlue!30}{5273} & 100.0\\
\texttt{tic-tac-toe} & \multicolumn{1}{r}{958} & \multicolumn{1}{r}{18}  & \cellcolor{TealBlue!30}{1} & \cellcolor{TealBlue!30}{137} & 0.8 & \cellcolor{TealBlue!30}{1} & \cellcolor{TealBlue!30}{137} & 0.5 & \cellcolor{TealBlue!30}{1} & \cellcolor{TealBlue!30}{137} & \cellcolor{TealBlue!30}{\textbf{0.5}}\\
\texttt{vehicle} & \multicolumn{1}{r}{846} & \multicolumn{1}{r}{252}  & \cellcolor{TealBlue!30}{1} & \cellcolor{TealBlue!30}{12} & 1800.0 & \cellcolor{TealBlue!30}{1} & \cellcolor{TealBlue!30}{12} & 944.0 & \cellcolor{TealBlue!30}{1} & \cellcolor{TealBlue!30}{12} & \cellcolor{TealBlue!30}{\textbf{902.0}}\\
\texttt{vote} & \multicolumn{1}{r}{435} & \multicolumn{1}{r}{32}  & \cellcolor{TealBlue!30}{1} & \cellcolor{TealBlue!30}{5} & 2.9 & \cellcolor{TealBlue!30}{1} & \cellcolor{TealBlue!30}{5} & 1.6 & \cellcolor{TealBlue!30}{1} & \cellcolor{TealBlue!30}{5} & \cellcolor{TealBlue!30}{\textbf{1.4}}\\
\texttt{wine1-un} & \multicolumn{1}{r}{178} & \multicolumn{1}{r}{1276}  & \cellcolor{TealBlue!30}{0} & 39 & \cellcolor{TealBlue!30}{\textbf{3.8}} & \cellcolor{TealBlue!30}{0} & \cellcolor{TealBlue!30}{38} & 2290.0 & \cellcolor{TealBlue!30}{0} & \cellcolor{TealBlue!30}{38} & 918.0\\
\texttt{wine2-un} & \multicolumn{1}{r}{178} & \multicolumn{1}{r}{1276}  & \cellcolor{TealBlue!30}{0} & \cellcolor{TealBlue!30}{43} & 556.0 & \cellcolor{TealBlue!30}{0} & \cellcolor{TealBlue!30}{43} & 115.0 & \cellcolor{TealBlue!30}{0} & \cellcolor{TealBlue!30}{43} & \cellcolor{TealBlue!30}{\textbf{0.1}}\\
\texttt{wine3-un} & \multicolumn{1}{r}{178} & \multicolumn{1}{r}{1276}  & \cellcolor{TealBlue!30}{0} & \cellcolor{TealBlue!30}{28} & 3600.0 & \cellcolor{TealBlue!30}{0} & \cellcolor{TealBlue!30}{28} & \cellcolor{TealBlue!30}{\textbf{230.0}} & \cellcolor{TealBlue!30}{0} & \cellcolor{TealBlue!30}{28} & 413.0\\
\texttt{yeast} & \multicolumn{1}{r}{1484} & \multicolumn{1}{r}{89}  & \cellcolor{TealBlue!30}{1} & \cellcolor{TealBlue!30}{366} & 70.8 & \cellcolor{TealBlue!30}{1} & \cellcolor{TealBlue!30}{366} & 39.2 & \cellcolor{TealBlue!30}{1} & \cellcolor{TealBlue!30}{366} & \cellcolor{TealBlue!30}{\textbf{38.7}}\\
\texttt{zoo-1} & \multicolumn{1}{r}{101} & \multicolumn{1}{r}{20}  & \cellcolor{TealBlue!30}{1} & \cellcolor{TealBlue!30}{0} & 0.0 & \cellcolor{TealBlue!30}{1} & \cellcolor{TealBlue!30}{0} & \cellcolor{TealBlue!30}{\textbf{0.0}} & \cellcolor{TealBlue!30}{1} & \cellcolor{TealBlue!30}{0} & 0.0\\
\bottomrule
\end{tabular}

% \end{normalsize}
% \end{center}
% \caption{\label{tab:ha4} Comparison of heuristics (max depth=4)}
% \end{table}
%
% \begin{table}[htbp]
% \begin{center}
% \begin{normalsize}
% \tabcolsep=5pt
% \begin{tabular}{lccrrrrrrrrr}
\toprule
& && \multicolumn{3}{c}{entropy} & \multicolumn{3}{c}{\budalg} & \multicolumn{3}{c}{error}\\
\cmidrule(rr){4-6}\cmidrule(rr){7-9}\cmidrule(rr){10-12}
&\multirow{1}{*}{$\#ex.$} & \multirow{1}{*}{\#feat.} &  \multicolumn{1}{c}{opt} & \multicolumn{1}{c}{error} & \multicolumn{1}{c}{time} & \multicolumn{1}{c}{opt} & \multicolumn{1}{c}{error} & \multicolumn{1}{c}{time} & \multicolumn{1}{c}{opt} & \multicolumn{1}{c}{error} & \multicolumn{1}{c}{time} \\
\midrule

\texttt{anneal} & \multicolumn{1}{r}{812} & \multicolumn{1}{r}{47}  & \cellcolor{TealBlue!30}{1} & \cellcolor{TealBlue!30}{70} & 1630.0 & \cellcolor{TealBlue!30}{1} & \cellcolor{TealBlue!30}{70} & 995.0 & \cellcolor{TealBlue!30}{1} & \cellcolor{TealBlue!30}{70} & \cellcolor{TealBlue!30}{\textbf{920.0}}\\
\texttt{audiology} & \multicolumn{1}{r}{216} & \multicolumn{1}{r}{79}  & \cellcolor{TealBlue!30}{1} & \cellcolor{TealBlue!30}{0} & 0.0 & \cellcolor{TealBlue!30}{1} & \cellcolor{TealBlue!30}{0} & 0.0 & \cellcolor{TealBlue!30}{1} & \cellcolor{TealBlue!30}{0} & \cellcolor{TealBlue!30}{\textbf{0.0}}\\
\texttt{australian-credit} & \multicolumn{1}{r}{653} & \multicolumn{1}{r}{73}  & \cellcolor{TealBlue!30}{0} & \cellcolor{TealBlue!30}{40} & 108.0 & \cellcolor{TealBlue!30}{0} & \cellcolor{TealBlue!30}{40} & \cellcolor{TealBlue!30}{\textbf{51.3}} & \cellcolor{TealBlue!30}{0} & \cellcolor{TealBlue!30}{40} & 51.8\\
\texttt{breast-cancer-un} & \multicolumn{1}{r}{683} & \multicolumn{1}{r}{89}  & \cellcolor{TealBlue!30}{1} & \cellcolor{TealBlue!30}{6} & 1800.0 & \cellcolor{TealBlue!30}{1} & \cellcolor{TealBlue!30}{6} & 973.0 & \cellcolor{TealBlue!30}{1} & \cellcolor{TealBlue!30}{6} & \cellcolor{TealBlue!30}{\textbf{892.0}}\\
\texttt{breast-wisconsin} & \multicolumn{1}{r}{683} & \multicolumn{1}{r}{120}  & \cellcolor{TealBlue!30}{1} & \cellcolor{TealBlue!30}{0} & 1180.0 & \cellcolor{TealBlue!30}{1} & \cellcolor{TealBlue!30}{0} & 509.0 & \cellcolor{TealBlue!30}{1} & \cellcolor{TealBlue!30}{0} & \cellcolor{TealBlue!30}{\textbf{381.0}}\\
\texttt{car-un} & \multicolumn{1}{r}{1728} & \multicolumn{1}{r}{21}  & \cellcolor{TealBlue!30}{1} & \cellcolor{TealBlue!30}{86} & 5.4 & \cellcolor{TealBlue!30}{1} & \cellcolor{TealBlue!30}{86} & \cellcolor{TealBlue!30}{\textbf{3.9}} & \cellcolor{TealBlue!30}{1} & \cellcolor{TealBlue!30}{86} & 4.2\\
\texttt{diabetes} & \multicolumn{1}{r}{768} & \multicolumn{1}{r}{112}  & \cellcolor{TealBlue!30}{0} & 107 & \cellcolor{TealBlue!30}{\textbf{131.0}} & \cellcolor{TealBlue!30}{0} & \cellcolor{TealBlue!30}{106} & 1910.0 & \cellcolor{TealBlue!30}{0} & \cellcolor{TealBlue!30}{106} & 3130.0\\
\texttt{forest-fires-un} & \multicolumn{1}{r}{517} & \multicolumn{1}{r}{989}  & \cellcolor{TealBlue!30}{0} & 172 & \cellcolor{TealBlue!30}{\textbf{105.0}} & \cellcolor{TealBlue!30}{0} & \cellcolor{TealBlue!30}{\textbf{156}} & 2980.0 & \cellcolor{TealBlue!30}{0} & 157 & 435.0\\
\texttt{german-credit} & \multicolumn{1}{r}{1000} & \multicolumn{1}{r}{110}  & \cellcolor{TealBlue!30}{0} & \cellcolor{TealBlue!30}{161} & 197.0 & \cellcolor{TealBlue!30}{0} & \cellcolor{TealBlue!30}{161} & \cellcolor{TealBlue!30}{\textbf{103.0}} & \cellcolor{TealBlue!30}{0} & 165 & 2600.0\\
\texttt{heart-cleveland} & \multicolumn{1}{r}{296} & \multicolumn{1}{r}{50}  & \cellcolor{TealBlue!30}{1} & \cellcolor{TealBlue!30}{7} & 3450.0 & \cellcolor{TealBlue!30}{1} & \cellcolor{TealBlue!30}{7} & 1520.0 & \cellcolor{TealBlue!30}{1} & \cellcolor{TealBlue!30}{7} & \cellcolor{TealBlue!30}{\textbf{1370.0}}\\
\texttt{hepatitis} & \multicolumn{1}{r}{137} & \multicolumn{1}{r}{68}  & \cellcolor{TealBlue!30}{1} & \cellcolor{TealBlue!30}{0} & 1.0 & \cellcolor{TealBlue!30}{1} & \cellcolor{TealBlue!30}{0} & \cellcolor{TealBlue!30}{\textbf{0.5}} & \cellcolor{TealBlue!30}{1} & \cellcolor{TealBlue!30}{0} & 1.6\\
\texttt{hypothyroid} & \multicolumn{1}{r}{3247} & \multicolumn{1}{r}{43}  & 0 & \cellcolor{TealBlue!30}{44} & \cellcolor{TealBlue!30}{\textbf{1070.0}} & \cellcolor{TealBlue!30}{1} & \cellcolor{TealBlue!30}{44} & 2850.0 & \cellcolor{TealBlue!30}{1} & \cellcolor{TealBlue!30}{44} & 2910.0\\
\texttt{ionosphere} & \multicolumn{1}{r}{351} & \multicolumn{1}{r}{444}  & \cellcolor{TealBlue!30}{0} & 3 & \cellcolor{TealBlue!30}{\textbf{258.0}} & \cellcolor{TealBlue!30}{0} & \cellcolor{TealBlue!30}{2} & 1980.0 & \cellcolor{TealBlue!30}{0} & \cellcolor{TealBlue!30}{2} & 707.0\\
\texttt{kr-vs-kp} & \multicolumn{1}{r}{3196} & \multicolumn{1}{r}{37}  & \cellcolor{TealBlue!30}{1} & \cellcolor{TealBlue!30}{81} & 1970.0 & \cellcolor{TealBlue!30}{1} & \cellcolor{TealBlue!30}{81} & 1400.0 & \cellcolor{TealBlue!30}{1} & \cellcolor{TealBlue!30}{81} & \cellcolor{TealBlue!30}{\textbf{1390.0}}\\
\texttt{letter} & \multicolumn{1}{r}{20000} & \multicolumn{1}{r}{224}  & \cellcolor{TealBlue!30}{0} & 447 & 2910.0 & \cellcolor{TealBlue!30}{0} & 280 & \cellcolor{TealBlue!30}{\textbf{373.0}} & \cellcolor{TealBlue!30}{0} & \cellcolor{TealBlue!30}{\textbf{251}} & 2970.0\\
\texttt{lymph} & \multicolumn{1}{r}{148} & \multicolumn{1}{r}{41}  & \cellcolor{TealBlue!30}{1} & \cellcolor{TealBlue!30}{0} & 0.0 & \cellcolor{TealBlue!30}{1} & \cellcolor{TealBlue!30}{0} & \cellcolor{TealBlue!30}{\textbf{0.0}} & \cellcolor{TealBlue!30}{1} & \cellcolor{TealBlue!30}{0} & 0.0\\
\texttt{mushroom} & \multicolumn{1}{r}{8124} & \multicolumn{1}{r}{91}  & \cellcolor{TealBlue!30}{1} & \cellcolor{TealBlue!30}{0} & 0.0 & \cellcolor{TealBlue!30}{1} & \cellcolor{TealBlue!30}{0} & \cellcolor{TealBlue!30}{\textbf{0.0}} & \cellcolor{TealBlue!30}{1} & \cellcolor{TealBlue!30}{0} & 0.0\\
\texttt{pendigits} & \multicolumn{1}{r}{7494} & \multicolumn{1}{r}{216}  & \cellcolor{TealBlue!30}{0} & \cellcolor{TealBlue!30}{2} & \cellcolor{TealBlue!30}{\textbf{90.6}} & \cellcolor{TealBlue!30}{0} & \cellcolor{TealBlue!30}{2} & 1780.0 & \cellcolor{TealBlue!30}{0} & \cellcolor{TealBlue!30}{2} & 338.0\\
\texttt{primary-tumor} & \multicolumn{1}{r}{336} & \multicolumn{1}{r}{16}  & \cellcolor{TealBlue!30}{1} & \cellcolor{TealBlue!30}{26} & 16.6 & \cellcolor{TealBlue!30}{1} & \cellcolor{TealBlue!30}{26} & \cellcolor{TealBlue!30}{\textbf{8.9}} & \cellcolor{TealBlue!30}{1} & \cellcolor{TealBlue!30}{26} & 9.1\\
\texttt{segment} & \multicolumn{1}{r}{2310} & \multicolumn{1}{r}{234}  & \cellcolor{TealBlue!30}{1} & \cellcolor{TealBlue!30}{0} & 0.0 & \cellcolor{TealBlue!30}{1} & \cellcolor{TealBlue!30}{0} & \cellcolor{TealBlue!30}{\textbf{0.0}} & \cellcolor{TealBlue!30}{1} & \cellcolor{TealBlue!30}{0} & 0.0\\
\texttt{soybean} & \multicolumn{1}{r}{630} & \multicolumn{1}{r}{34}  & \cellcolor{TealBlue!30}{1} & \cellcolor{TealBlue!30}{8} & 101.0 & \cellcolor{TealBlue!30}{1} & \cellcolor{TealBlue!30}{8} & 62.7 & \cellcolor{TealBlue!30}{1} & \cellcolor{TealBlue!30}{8} & \cellcolor{TealBlue!30}{\textbf{59.0}}\\
\texttt{splice-1} & \multicolumn{1}{r}{3190} & \multicolumn{1}{r}{227}  & \cellcolor{TealBlue!30}{0} & 103 & \cellcolor{TealBlue!30}{\textbf{41.8}} & \cellcolor{TealBlue!30}{0} & \cellcolor{TealBlue!30}{101} & 2260.0 & \cellcolor{TealBlue!30}{0} & \cellcolor{TealBlue!30}{101} & 2010.0\\
\texttt{taiwan\_binarised} & \multicolumn{1}{r}{30000} & \multicolumn{1}{r}{198}  & \cellcolor{TealBlue!30}{0} & \cellcolor{TealBlue!30}{5200} & 1840.0 & \cellcolor{TealBlue!30}{0} & \cellcolor{TealBlue!30}{5200} & \cellcolor{TealBlue!30}{\textbf{1290.0}} & \cellcolor{TealBlue!30}{0} & 5204 & 2210.0\\
\texttt{tic-tac-toe} & \multicolumn{1}{r}{958} & \multicolumn{1}{r}{18}  & \cellcolor{TealBlue!30}{1} & \cellcolor{TealBlue!30}{63} & 21.5 & \cellcolor{TealBlue!30}{1} & \cellcolor{TealBlue!30}{63} & 12.1 & \cellcolor{TealBlue!30}{1} & \cellcolor{TealBlue!30}{63} & \cellcolor{TealBlue!30}{\textbf{11.1}}\\
\texttt{vehicle} & \multicolumn{1}{r}{846} & \multicolumn{1}{r}{252}  & \cellcolor{TealBlue!30}{0} & \cellcolor{TealBlue!30}{3} & 2630.0 & \cellcolor{TealBlue!30}{0} & \cellcolor{TealBlue!30}{3} & \cellcolor{TealBlue!30}{\textbf{88.8}} & \cellcolor{TealBlue!30}{0} & 9 & 3550.0\\
\texttt{vote} & \multicolumn{1}{r}{435} & \multicolumn{1}{r}{32}  & \cellcolor{TealBlue!30}{1} & \cellcolor{TealBlue!30}{1} & 54.8 & \cellcolor{TealBlue!30}{1} & \cellcolor{TealBlue!30}{1} & 30.1 & \cellcolor{TealBlue!30}{1} & \cellcolor{TealBlue!30}{1} & \cellcolor{TealBlue!30}{\textbf{27.7}}\\
\texttt{wine1-un} & \multicolumn{1}{r}{178} & \multicolumn{1}{r}{1276}  & \cellcolor{TealBlue!30}{0} & 35 & 1010.0 & \cellcolor{TealBlue!30}{0} & \cellcolor{TealBlue!30}{34} & 1430.0 & \cellcolor{TealBlue!30}{0} & \cellcolor{TealBlue!30}{34} & \cellcolor{TealBlue!30}{\textbf{804.0}}\\
\texttt{wine2-un} & \multicolumn{1}{r}{178} & \multicolumn{1}{r}{1276}  & \cellcolor{TealBlue!30}{0} & 40 & 226.0 & \cellcolor{TealBlue!30}{0} & 39 & 2820.0 & \cellcolor{TealBlue!30}{0} & \cellcolor{TealBlue!30}{\textbf{37}} & \cellcolor{TealBlue!30}{\textbf{86.9}}\\
\texttt{wine3-un} & \multicolumn{1}{r}{178} & \multicolumn{1}{r}{1276}  & \cellcolor{TealBlue!30}{0} & \cellcolor{TealBlue!30}{25} & 254.0 & \cellcolor{TealBlue!30}{0} & \cellcolor{TealBlue!30}{25} & \cellcolor{TealBlue!30}{\textbf{110.0}} & \cellcolor{TealBlue!30}{0} & 26 & 914.0\\
\texttt{yeast} & \multicolumn{1}{r}{1484} & \multicolumn{1}{r}{89}  & 0 & \cellcolor{TealBlue!30}{313} & \cellcolor{TealBlue!30}{\textbf{1430.0}} & \cellcolor{TealBlue!30}{1} & \cellcolor{TealBlue!30}{313} & 3270.0 & \cellcolor{TealBlue!30}{1} & \cellcolor{TealBlue!30}{313} & 3290.0\\
\texttt{zoo-1} & \multicolumn{1}{r}{101} & \multicolumn{1}{r}{20}  & \cellcolor{TealBlue!30}{1} & \cellcolor{TealBlue!30}{0} & 0.0 & \cellcolor{TealBlue!30}{1} & \cellcolor{TealBlue!30}{0} & \cellcolor{TealBlue!30}{\textbf{0.0}} & \cellcolor{TealBlue!30}{1} & \cellcolor{TealBlue!30}{0} & 0.0\\
\bottomrule
\end{tabular}

% \end{normalsize}
% \end{center}
% \caption{\label{tab:ha5} Comparison of heuristics (max depth=5)}
% \end{table}
%
% \begin{table}[htbp]
% \begin{center}
% \begin{normalsize}
% \tabcolsep=5pt
% \begin{tabular}{lccrrrrrrrrr}
\toprule
& && \multicolumn{3}{c}{entropy} & \multicolumn{3}{c}{\budalg} & \multicolumn{3}{c}{error}\\
\cmidrule(rr){4-6}\cmidrule(rr){7-9}\cmidrule(rr){10-12}
&\multirow{1}{*}{$\#ex.$} & \multirow{1}{*}{\#feat.} &  \multicolumn{1}{c}{opt} & \multicolumn{1}{c}{error} & \multicolumn{1}{c}{time} & \multicolumn{1}{c}{opt} & \multicolumn{1}{c}{error} & \multicolumn{1}{c}{time} & \multicolumn{1}{c}{opt} & \multicolumn{1}{c}{error} & \multicolumn{1}{c}{time} \\
\midrule

\texttt{anneal} & \multicolumn{1}{r}{812} & \multicolumn{1}{r}{47}  & \cellcolor{TealBlue!30}{0} & \cellcolor{TealBlue!30}{58} & 430.0 & \cellcolor{TealBlue!30}{0} & \cellcolor{TealBlue!30}{58} & \cellcolor{TealBlue!30}{\textbf{209.0}} & \cellcolor{TealBlue!30}{0} & 64 & 3090.0\\
\texttt{audiology} & \multicolumn{1}{r}{216} & \multicolumn{1}{r}{79}  & \cellcolor{TealBlue!30}{1} & \cellcolor{TealBlue!30}{0} & 0.0 & \cellcolor{TealBlue!30}{1} & \cellcolor{TealBlue!30}{0} & 0.0 & \cellcolor{TealBlue!30}{1} & \cellcolor{TealBlue!30}{0} & \cellcolor{TealBlue!30}{\textbf{0.0}}\\
\texttt{australian-credit} & \multicolumn{1}{r}{653} & \multicolumn{1}{r}{73}  & 0 & 9 & \cellcolor{TealBlue!30}{\textbf{726.0}} & \cellcolor{TealBlue!30}{\textbf{1}} & \cellcolor{TealBlue!30}{\textbf{0}} & 1040.0 & 0 & 13 & 2020.0\\
\texttt{breast-cancer-un} & \multicolumn{1}{r}{683} & \multicolumn{1}{r}{89}  & \cellcolor{TealBlue!30}{1} & \cellcolor{TealBlue!30}{0} & 1340.0 & \cellcolor{TealBlue!30}{1} & \cellcolor{TealBlue!30}{0} & 1190.0 & \cellcolor{TealBlue!30}{1} & \cellcolor{TealBlue!30}{0} & \cellcolor{TealBlue!30}{\textbf{973.0}}\\
\texttt{breast-wisconsin} & \multicolumn{1}{r}{683} & \multicolumn{1}{r}{120}  & \cellcolor{TealBlue!30}{1} & \cellcolor{TealBlue!30}{0} & 0.3 & \cellcolor{TealBlue!30}{1} & \cellcolor{TealBlue!30}{0} & 0.3 & \cellcolor{TealBlue!30}{1} & \cellcolor{TealBlue!30}{0} & \cellcolor{TealBlue!30}{\textbf{0.1}}\\
\texttt{car-un} & \multicolumn{1}{r}{1728} & \multicolumn{1}{r}{21}  & \cellcolor{TealBlue!30}{1} & \cellcolor{TealBlue!30}{11} & 438.0 & \cellcolor{TealBlue!30}{1} & \cellcolor{TealBlue!30}{11} & 294.0 & \cellcolor{TealBlue!30}{1} & \cellcolor{TealBlue!30}{11} & \cellcolor{TealBlue!30}{\textbf{255.0}}\\
\texttt{diabetes} & \multicolumn{1}{r}{768} & \multicolumn{1}{r}{112}  & \cellcolor{TealBlue!30}{0} & 32 & 2080.0 & \cellcolor{TealBlue!30}{0} & \cellcolor{TealBlue!30}{\textbf{27}} & 3210.0 & \cellcolor{TealBlue!30}{0} & 61 & \cellcolor{TealBlue!30}{\textbf{481.0}}\\
\texttt{forest-fires-un} & \multicolumn{1}{r}{517} & \multicolumn{1}{r}{989}  & \cellcolor{TealBlue!30}{0} & 160 & 669.0 & \cellcolor{TealBlue!30}{0} & 155 & \cellcolor{TealBlue!30}{\textbf{325.0}} & \cellcolor{TealBlue!30}{0} & \cellcolor{TealBlue!30}{\textbf{150}} & 390.0\\
\texttt{german-credit} & \multicolumn{1}{r}{1000} & \multicolumn{1}{r}{110}  & \cellcolor{TealBlue!30}{0} & \cellcolor{TealBlue!30}{57} & 1360.0 & \cellcolor{TealBlue!30}{0} & \cellcolor{TealBlue!30}{57} & \cellcolor{TealBlue!30}{\textbf{706.0}} & \cellcolor{TealBlue!30}{0} & 133 & 3540.0\\
\texttt{heart-cleveland} & \multicolumn{1}{r}{296} & \multicolumn{1}{r}{50}  & \cellcolor{TealBlue!30}{1} & \cellcolor{TealBlue!30}{0} & 1.4 & \cellcolor{TealBlue!30}{1} & \cellcolor{TealBlue!30}{0} & \cellcolor{TealBlue!30}{\textbf{0.0}} & \cellcolor{TealBlue!30}{1} & \cellcolor{TealBlue!30}{0} & 46.6\\
\texttt{hepatitis} & \multicolumn{1}{r}{137} & \multicolumn{1}{r}{68}  & \cellcolor{TealBlue!30}{1} & \cellcolor{TealBlue!30}{0} & 0.0 & \cellcolor{TealBlue!30}{1} & \cellcolor{TealBlue!30}{0} & 0.0 & \cellcolor{TealBlue!30}{1} & \cellcolor{TealBlue!30}{0} & \cellcolor{TealBlue!30}{\textbf{0.0}}\\
\texttt{hypothyroid} & \multicolumn{1}{r}{3247} & \multicolumn{1}{r}{43}  & \cellcolor{TealBlue!30}{0} & 45 & 126.0 & \cellcolor{TealBlue!30}{0} & \cellcolor{TealBlue!30}{\textbf{43}} & \cellcolor{TealBlue!30}{\textbf{78.3}} & \cellcolor{TealBlue!30}{0} & 50 & 338.0\\
\texttt{ionosphere} & \multicolumn{1}{r}{351} & \multicolumn{1}{r}{444}  & \cellcolor{TealBlue!30}{1} & \cellcolor{TealBlue!30}{0} & \cellcolor{TealBlue!30}{\textbf{0.2}} & \cellcolor{TealBlue!30}{1} & \cellcolor{TealBlue!30}{0} & 0.5 & \cellcolor{TealBlue!30}{1} & \cellcolor{TealBlue!30}{0} & 1.1\\
\texttt{kr-vs-kp} & \multicolumn{1}{r}{3196} & \multicolumn{1}{r}{37}  & \cellcolor{TealBlue!30}{0} & 37 & \cellcolor{TealBlue!30}{\textbf{194.0}} & \cellcolor{TealBlue!30}{0} & 37 & 1460.0 & \cellcolor{TealBlue!30}{0} & \cellcolor{TealBlue!30}{\textbf{21}} & 222.0\\
\texttt{letter} & \multicolumn{1}{r}{20000} & \multicolumn{1}{r}{224}  & \cellcolor{TealBlue!30}{0} & \cellcolor{TealBlue!30}{\textbf{112}} & 3260.0 & \cellcolor{TealBlue!30}{0} & 118 & \cellcolor{TealBlue!30}{\textbf{219.0}} & \cellcolor{TealBlue!30}{0} & 184 & 1680.0\\
\texttt{lymph} & \multicolumn{1}{r}{148} & \multicolumn{1}{r}{41}  & \cellcolor{TealBlue!30}{1} & \cellcolor{TealBlue!30}{0} & \cellcolor{TealBlue!30}{\textbf{0.0}} & \cellcolor{TealBlue!30}{1} & \cellcolor{TealBlue!30}{0} & 0.0 & \cellcolor{TealBlue!30}{1} & \cellcolor{TealBlue!30}{0} & 0.0\\
\texttt{mushroom} & \multicolumn{1}{r}{8124} & \multicolumn{1}{r}{91}  & \cellcolor{TealBlue!30}{1} & \cellcolor{TealBlue!30}{0} & 0.0 & \cellcolor{TealBlue!30}{1} & \cellcolor{TealBlue!30}{0} & \cellcolor{TealBlue!30}{\textbf{0.0}} & \cellcolor{TealBlue!30}{1} & \cellcolor{TealBlue!30}{0} & 0.0\\
\texttt{pendigits} & \multicolumn{1}{r}{7494} & \multicolumn{1}{r}{216}  & \cellcolor{TealBlue!30}{1} & \cellcolor{TealBlue!30}{0} & 0.1 & \cellcolor{TealBlue!30}{1} & \cellcolor{TealBlue!30}{0} & \cellcolor{TealBlue!30}{\textbf{0.1}} & \cellcolor{TealBlue!30}{1} & \cellcolor{TealBlue!30}{0} & 55.9\\
\texttt{primary-tumor} & \multicolumn{1}{r}{336} & \multicolumn{1}{r}{16}  & \cellcolor{TealBlue!30}{0} & \cellcolor{TealBlue!30}{16} & 178.0 & \cellcolor{TealBlue!30}{0} & \cellcolor{TealBlue!30}{16} & 93.9 & \cellcolor{TealBlue!30}{0} & \cellcolor{TealBlue!30}{16} & \cellcolor{TealBlue!30}{\textbf{93.7}}\\
\texttt{segment} & \multicolumn{1}{r}{2310} & \multicolumn{1}{r}{234}  & \cellcolor{TealBlue!30}{1} & \cellcolor{TealBlue!30}{0} & 0.0 & \cellcolor{TealBlue!30}{1} & \cellcolor{TealBlue!30}{0} & 0.0 & \cellcolor{TealBlue!30}{1} & \cellcolor{TealBlue!30}{0} & \cellcolor{TealBlue!30}{\textbf{0.0}}\\
\texttt{soybean} & \multicolumn{1}{r}{630} & \multicolumn{1}{r}{34}  & \cellcolor{TealBlue!30}{0} & \cellcolor{TealBlue!30}{2} & 1960.0 & \cellcolor{TealBlue!30}{0} & \cellcolor{TealBlue!30}{2} & 3540.0 & \cellcolor{TealBlue!30}{0} & \cellcolor{TealBlue!30}{2} & \cellcolor{TealBlue!30}{\textbf{1730.0}}\\
\texttt{splice-1} & \multicolumn{1}{r}{3190} & \multicolumn{1}{r}{227}  & \cellcolor{TealBlue!30}{0} & 33 & 2570.0 & \cellcolor{TealBlue!30}{0} & \cellcolor{TealBlue!30}{\textbf{32}} & 2440.0 & \cellcolor{TealBlue!30}{0} & 45 & \cellcolor{TealBlue!30}{\textbf{1730.0}}\\
\texttt{taiwan\_binarised} & \multicolumn{1}{r}{30000} & \multicolumn{1}{r}{198}  & \cellcolor{TealBlue!30}{0} & \cellcolor{TealBlue!30}{\textbf{5017}} & 2250.0 & \cellcolor{TealBlue!30}{0} & 5065 & \cellcolor{TealBlue!30}{\textbf{70.3}} & \cellcolor{TealBlue!30}{0} & 5152 & 177.0\\
\texttt{tic-tac-toe} & \multicolumn{1}{r}{958} & \multicolumn{1}{r}{18}  & \cellcolor{TealBlue!30}{1} & \cellcolor{TealBlue!30}{0} & 48.7 & \cellcolor{TealBlue!30}{1} & \cellcolor{TealBlue!30}{0} & \cellcolor{TealBlue!30}{\textbf{26.4}} & \cellcolor{TealBlue!30}{1} & \cellcolor{TealBlue!30}{0} & 27.5\\
\texttt{vehicle} & \multicolumn{1}{r}{846} & \multicolumn{1}{r}{252}  & \cellcolor{TealBlue!30}{1} & \cellcolor{TealBlue!30}{0} & 1.7 & \cellcolor{TealBlue!30}{1} & \cellcolor{TealBlue!30}{0} & \cellcolor{TealBlue!30}{\textbf{0.6}} & \cellcolor{TealBlue!30}{1} & \cellcolor{TealBlue!30}{0} & 688.0\\
\texttt{vote} & \multicolumn{1}{r}{435} & \multicolumn{1}{r}{32}  & \cellcolor{TealBlue!30}{1} & \cellcolor{TealBlue!30}{0} & 0.0 & \cellcolor{TealBlue!30}{1} & \cellcolor{TealBlue!30}{0} & \cellcolor{TealBlue!30}{\textbf{0.0}} & \cellcolor{TealBlue!30}{1} & \cellcolor{TealBlue!30}{0} & 0.1\\
\texttt{wine1-un} & \multicolumn{1}{r}{178} & \multicolumn{1}{r}{1276}  & \cellcolor{TealBlue!30}{0} & 34 & 2850.0 & \cellcolor{TealBlue!30}{0} & 29 & 509.0 & \cellcolor{TealBlue!30}{0} & \cellcolor{TealBlue!30}{\textbf{28}} & \cellcolor{TealBlue!30}{\textbf{94.4}}\\
\texttt{wine2-un} & \multicolumn{1}{r}{178} & \multicolumn{1}{r}{1276}  & \cellcolor{TealBlue!30}{0} & \cellcolor{TealBlue!30}{31} & 2110.0 & \cellcolor{TealBlue!30}{0} & \cellcolor{TealBlue!30}{31} & 172.0 & \cellcolor{TealBlue!30}{0} & \cellcolor{TealBlue!30}{31} & \cellcolor{TealBlue!30}{\textbf{0.2}}\\
\texttt{wine3-un} & \multicolumn{1}{r}{178} & \multicolumn{1}{r}{1276}  & \cellcolor{TealBlue!30}{0} & \cellcolor{TealBlue!30}{20} & 644.0 & \cellcolor{TealBlue!30}{0} & \cellcolor{TealBlue!30}{20} & 307.0 & \cellcolor{TealBlue!30}{0} & 23 & \cellcolor{TealBlue!30}{\textbf{76.3}}\\
\texttt{yeast} & \multicolumn{1}{r}{1484} & \multicolumn{1}{r}{89}  & \cellcolor{TealBlue!30}{0} & 265 & \cellcolor{TealBlue!30}{\textbf{215.0}} & \cellcolor{TealBlue!30}{0} & 264 & 3130.0 & \cellcolor{TealBlue!30}{0} & \cellcolor{TealBlue!30}{\textbf{252}} & 3190.0\\
\texttt{zoo-1} & \multicolumn{1}{r}{101} & \multicolumn{1}{r}{20}  & \cellcolor{TealBlue!30}{1} & \cellcolor{TealBlue!30}{0} & 0.0 & \cellcolor{TealBlue!30}{1} & \cellcolor{TealBlue!30}{0} & \cellcolor{TealBlue!30}{\textbf{0.0}} & \cellcolor{TealBlue!30}{1} & \cellcolor{TealBlue!30}{0} & 0.0\\
\bottomrule
\end{tabular}

% \end{normalsize}
% \end{center}
% \caption{\label{tab:ha7} Comparison of heuristics (max depth=7)}
% \end{table}
%
% \begin{table}[htbp]
% \begin{center}
% \begin{normalsize}
% \tabcolsep=5pt
% \begin{tabular}{lccrrrrrrrrr}
\toprule
& && \multicolumn{3}{c}{entropy} & \multicolumn{3}{c}{\budalg} & \multicolumn{3}{c}{error}\\
\cmidrule(rr){4-6}\cmidrule(rr){7-9}\cmidrule(rr){10-12}
&\multirow{1}{*}{$\#ex.$} & \multirow{1}{*}{\#feat.} &  \multicolumn{1}{c}{opt} & \multicolumn{1}{c}{error} & \multicolumn{1}{c}{time} & \multicolumn{1}{c}{opt} & \multicolumn{1}{c}{error} & \multicolumn{1}{c}{time} & \multicolumn{1}{c}{opt} & \multicolumn{1}{c}{error} & \multicolumn{1}{c}{time} \\
\midrule

\texttt{anneal} & \multicolumn{1}{r}{812} & \multicolumn{1}{r}{47}  & \cellcolor{TealBlue!30}{0} & 76 & 1260.0 & \cellcolor{TealBlue!30}{0} & 58 & 773.0 & \cellcolor{TealBlue!30}{0} & \cellcolor{TealBlue!30}{\textbf{48}} & \cellcolor{TealBlue!30}{\textbf{678.0}}\\
\texttt{audiology} & \multicolumn{1}{r}{216} & \multicolumn{1}{r}{79}  & \cellcolor{TealBlue!30}{1} & \cellcolor{TealBlue!30}{0} & 0.0 & \cellcolor{TealBlue!30}{1} & \cellcolor{TealBlue!30}{0} & 0.0 & \cellcolor{TealBlue!30}{1} & \cellcolor{TealBlue!30}{0} & \cellcolor{TealBlue!30}{\textbf{0.0}}\\
\texttt{australian-credit} & \multicolumn{1}{r}{653} & \multicolumn{1}{r}{73}  & \cellcolor{TealBlue!30}{1} & \cellcolor{TealBlue!30}{0} & 0.8 & \cellcolor{TealBlue!30}{1} & \cellcolor{TealBlue!30}{0} & \cellcolor{TealBlue!30}{\textbf{0.3}} & 0 & 1 & 2160.0\\
\texttt{breast-cancer-un} & \multicolumn{1}{r}{683} & \multicolumn{1}{r}{89}  & \cellcolor{TealBlue!30}{1} & \cellcolor{TealBlue!30}{0} & 1.9 & \cellcolor{TealBlue!30}{1} & \cellcolor{TealBlue!30}{0} & \cellcolor{TealBlue!30}{\textbf{0.0}} & \cellcolor{TealBlue!30}{1} & \cellcolor{TealBlue!30}{0} & 0.4\\
\texttt{breast-wisconsin} & \multicolumn{1}{r}{683} & \multicolumn{1}{r}{120}  & \cellcolor{TealBlue!30}{1} & \cellcolor{TealBlue!30}{0} & 0.0 & \cellcolor{TealBlue!30}{1} & \cellcolor{TealBlue!30}{0} & \cellcolor{TealBlue!30}{\textbf{0.0}} & \cellcolor{TealBlue!30}{1} & \cellcolor{TealBlue!30}{0} & 0.0\\
\texttt{car-un} & \multicolumn{1}{r}{1728} & \multicolumn{1}{r}{21}  & \cellcolor{TealBlue!30}{1} & \cellcolor{TealBlue!30}{0} & 0.4 & \cellcolor{TealBlue!30}{1} & \cellcolor{TealBlue!30}{0} & \cellcolor{TealBlue!30}{\textbf{0.3}} & \cellcolor{TealBlue!30}{1} & \cellcolor{TealBlue!30}{0} & 25.3\\
\texttt{diabetes} & \multicolumn{1}{r}{768} & \multicolumn{1}{r}{112}  & \cellcolor{TealBlue!30}{1} & \cellcolor{TealBlue!30}{0} & 26.1 & \cellcolor{TealBlue!30}{1} & \cellcolor{TealBlue!30}{0} & \cellcolor{TealBlue!30}{\textbf{12.8}} & 0 & 7 & 3580.0\\
\texttt{forest-fires-un} & \multicolumn{1}{r}{517} & \multicolumn{1}{r}{989}  & \cellcolor{TealBlue!30}{0} & 123 & 1370.0 & \cellcolor{TealBlue!30}{0} & \cellcolor{TealBlue!30}{\textbf{118}} & 3370.0 & \cellcolor{TealBlue!30}{0} & 136 & \cellcolor{TealBlue!30}{\textbf{179.0}}\\
\texttt{german-credit} & \multicolumn{1}{r}{1000} & \multicolumn{1}{r}{110}  & \cellcolor{TealBlue!30}{1} & \cellcolor{TealBlue!30}{0} & 296.0 & \cellcolor{TealBlue!30}{1} & \cellcolor{TealBlue!30}{0} & \cellcolor{TealBlue!30}{\textbf{171.0}} & 0 & 100 & 661.0\\
\texttt{heart-cleveland} & \multicolumn{1}{r}{296} & \multicolumn{1}{r}{50}  & \cellcolor{TealBlue!30}{1} & \cellcolor{TealBlue!30}{0} & \cellcolor{TealBlue!30}{\textbf{0.0}} & \cellcolor{TealBlue!30}{1} & \cellcolor{TealBlue!30}{0} & 0.0 & \cellcolor{TealBlue!30}{1} & \cellcolor{TealBlue!30}{0} & 0.1\\
\texttt{hepatitis} & \multicolumn{1}{r}{137} & \multicolumn{1}{r}{68}  & \cellcolor{TealBlue!30}{1} & \cellcolor{TealBlue!30}{0} & 0.0 & \cellcolor{TealBlue!30}{1} & \cellcolor{TealBlue!30}{0} & \cellcolor{TealBlue!30}{\textbf{0.0}} & \cellcolor{TealBlue!30}{1} & \cellcolor{TealBlue!30}{0} & 0.0\\
\texttt{hypothyroid} & \multicolumn{1}{r}{3247} & \multicolumn{1}{r}{43}  & \cellcolor{TealBlue!30}{0} & \cellcolor{TealBlue!30}{32} & 486.0 & \cellcolor{TealBlue!30}{0} & \cellcolor{TealBlue!30}{32} & \cellcolor{TealBlue!30}{\textbf{34.8}} & \cellcolor{TealBlue!30}{0} & 44 & 126.0\\
\texttt{ionosphere} & \multicolumn{1}{r}{351} & \multicolumn{1}{r}{444}  & \cellcolor{TealBlue!30}{1} & \cellcolor{TealBlue!30}{0} & 0.1 & \cellcolor{TealBlue!30}{1} & \cellcolor{TealBlue!30}{0} & 0.0 & \cellcolor{TealBlue!30}{1} & \cellcolor{TealBlue!30}{0} & \cellcolor{TealBlue!30}{\textbf{0.0}}\\
\texttt{kr-vs-kp} & \multicolumn{1}{r}{3196} & \multicolumn{1}{r}{37}  & \cellcolor{TealBlue!30}{0} & \cellcolor{TealBlue!30}{2} & 1140.0 & \cellcolor{TealBlue!30}{0} & \cellcolor{TealBlue!30}{2} & \cellcolor{TealBlue!30}{\textbf{954.0}} & \cellcolor{TealBlue!30}{0} & 71 & 1520.0\\
\texttt{letter} & \multicolumn{1}{r}{20000} & \multicolumn{1}{r}{224}  & 0 & 20 & 2300.0 & \cellcolor{TealBlue!30}{\textbf{1}} & \cellcolor{TealBlue!30}{\textbf{0}} & \cellcolor{TealBlue!30}{\textbf{1600.0}} & 0 & 204 & 3560.0\\
\texttt{lymph} & \multicolumn{1}{r}{148} & \multicolumn{1}{r}{41}  & \cellcolor{TealBlue!30}{1} & \cellcolor{TealBlue!30}{0} & 0.0 & \cellcolor{TealBlue!30}{1} & \cellcolor{TealBlue!30}{0} & \cellcolor{TealBlue!30}{\textbf{0.0}} & \cellcolor{TealBlue!30}{1} & \cellcolor{TealBlue!30}{0} & 0.0\\
\texttt{mushroom} & \multicolumn{1}{r}{8124} & \multicolumn{1}{r}{91}  & \cellcolor{TealBlue!30}{1} & \cellcolor{TealBlue!30}{0} & 0.0 & \cellcolor{TealBlue!30}{1} & \cellcolor{TealBlue!30}{0} & \cellcolor{TealBlue!30}{\textbf{0.0}} & \cellcolor{TealBlue!30}{1} & \cellcolor{TealBlue!30}{0} & 0.0\\
\texttt{pendigits} & \multicolumn{1}{r}{7494} & \multicolumn{1}{r}{216}  & \cellcolor{TealBlue!30}{1} & \cellcolor{TealBlue!30}{0} & 0.1 & \cellcolor{TealBlue!30}{1} & \cellcolor{TealBlue!30}{0} & \cellcolor{TealBlue!30}{\textbf{0.1}} & \cellcolor{TealBlue!30}{1} & \cellcolor{TealBlue!30}{0} & 0.4\\
\texttt{primary-tumor} & \multicolumn{1}{r}{336} & \multicolumn{1}{r}{16}  & \cellcolor{TealBlue!30}{0} & \cellcolor{TealBlue!30}{15} & 311.0 & \cellcolor{TealBlue!30}{0} & \cellcolor{TealBlue!30}{15} & \cellcolor{TealBlue!30}{\textbf{126.0}} & \cellcolor{TealBlue!30}{0} & \cellcolor{TealBlue!30}{15} & 3180.0\\
\texttt{segment} & \multicolumn{1}{r}{2310} & \multicolumn{1}{r}{234}  & \cellcolor{TealBlue!30}{1} & \cellcolor{TealBlue!30}{0} & 0.0 & \cellcolor{TealBlue!30}{1} & \cellcolor{TealBlue!30}{0} & 0.0 & \cellcolor{TealBlue!30}{1} & \cellcolor{TealBlue!30}{0} & \cellcolor{TealBlue!30}{\textbf{0.0}}\\
\texttt{soybean} & \multicolumn{1}{r}{630} & \multicolumn{1}{r}{34}  & \cellcolor{TealBlue!30}{0} & \cellcolor{TealBlue!30}{2} & \cellcolor{TealBlue!30}{\textbf{11.7}} & \cellcolor{TealBlue!30}{0} & \cellcolor{TealBlue!30}{2} & 229.0 & \cellcolor{TealBlue!30}{0} & \cellcolor{TealBlue!30}{2} & 3470.0\\
\texttt{splice-1} & \multicolumn{1}{r}{3190} & \multicolumn{1}{r}{227}  & \cellcolor{TealBlue!30}{0} & \cellcolor{TealBlue!30}{5} & 359.0 & \cellcolor{TealBlue!30}{0} & \cellcolor{TealBlue!30}{5} & 1420.0 & \cellcolor{TealBlue!30}{0} & 26 & \cellcolor{TealBlue!30}{\textbf{290.0}}\\
\texttt{taiwan\_binarised} & \multicolumn{1}{r}{30000} & \multicolumn{1}{r}{198}  & \cellcolor{TealBlue!30}{0} & 4648 & \cellcolor{TealBlue!30}{\textbf{157.0}} & \cellcolor{TealBlue!30}{0} & \cellcolor{TealBlue!30}{\textbf{4566}} & 207.0 & \cellcolor{TealBlue!30}{0} & 5074 & 176.0\\
\texttt{tic-tac-toe} & \multicolumn{1}{r}{958} & \multicolumn{1}{r}{18}  & \cellcolor{TealBlue!30}{1} & \cellcolor{TealBlue!30}{0} & \cellcolor{TealBlue!30}{\textbf{0.0}} & \cellcolor{TealBlue!30}{1} & \cellcolor{TealBlue!30}{0} & 0.0 & \cellcolor{TealBlue!30}{1} & \cellcolor{TealBlue!30}{0} & 0.0\\
\texttt{vehicle} & \multicolumn{1}{r}{846} & \multicolumn{1}{r}{252}  & \cellcolor{TealBlue!30}{1} & \cellcolor{TealBlue!30}{0} & 0.0 & \cellcolor{TealBlue!30}{1} & \cellcolor{TealBlue!30}{0} & \cellcolor{TealBlue!30}{\textbf{0.0}} & \cellcolor{TealBlue!30}{1} & \cellcolor{TealBlue!30}{0} & 135.0\\
\texttt{vote} & \multicolumn{1}{r}{435} & \multicolumn{1}{r}{32}  & \cellcolor{TealBlue!30}{1} & \cellcolor{TealBlue!30}{0} & 0.0 & \cellcolor{TealBlue!30}{1} & \cellcolor{TealBlue!30}{0} & \cellcolor{TealBlue!30}{\textbf{0.0}} & \cellcolor{TealBlue!30}{1} & \cellcolor{TealBlue!30}{0} & 0.0\\
\texttt{wine1-un} & \multicolumn{1}{r}{178} & \multicolumn{1}{r}{1276}  & \cellcolor{TealBlue!30}{0} & 26 & 1950.0 & \cellcolor{TealBlue!30}{0} & \cellcolor{TealBlue!30}{22} & 3450.0 & \cellcolor{TealBlue!30}{0} & \cellcolor{TealBlue!30}{22} & \cellcolor{TealBlue!30}{\textbf{25.3}}\\
\texttt{wine2-un} & \multicolumn{1}{r}{178} & \multicolumn{1}{r}{1276}  & \cellcolor{TealBlue!30}{0} & 27 & \cellcolor{TealBlue!30}{\textbf{4.1}} & \cellcolor{TealBlue!30}{0} & 24 & 2700.0 & \cellcolor{TealBlue!30}{0} & \cellcolor{TealBlue!30}{\textbf{21}} & 54.8\\
\texttt{wine3-un} & \multicolumn{1}{r}{178} & \multicolumn{1}{r}{1276}  & \cellcolor{TealBlue!30}{0} & \cellcolor{TealBlue!30}{\textbf{10}} & 2590.0 & \cellcolor{TealBlue!30}{0} & 18 & 1900.0 & \cellcolor{TealBlue!30}{0} & 17 & \cellcolor{TealBlue!30}{\textbf{107.0}}\\
\texttt{yeast} & \multicolumn{1}{r}{1484} & \multicolumn{1}{r}{89}  & \cellcolor{TealBlue!30}{0} & 109 & 3100.0 & \cellcolor{TealBlue!30}{0} & \cellcolor{TealBlue!30}{\textbf{104}} & 2680.0 & \cellcolor{TealBlue!30}{0} & 196 & \cellcolor{TealBlue!30}{\textbf{131.0}}\\
\texttt{zoo-1} & \multicolumn{1}{r}{101} & \multicolumn{1}{r}{20}  & \cellcolor{TealBlue!30}{1} & \cellcolor{TealBlue!30}{0} & 0.0 & \cellcolor{TealBlue!30}{1} & \cellcolor{TealBlue!30}{0} & \cellcolor{TealBlue!30}{\textbf{0.0}} & \cellcolor{TealBlue!30}{1} & \cellcolor{TealBlue!30}{0} & 0.0\\
\bottomrule
\end{tabular}

% \end{normalsize}
% \end{center}
% \caption{\label{tab:ha10} Comparison of heuristics (max depth=10)}
% \end{table}
%
%
%
%
%
% \begin{table}[htbp]
% \begin{center}
% \begin{normalsize}
% \tabcolsep=5pt
% \begin{tabular}{lccrrrrrr}
\toprule
& && \multicolumn{2}{c}{entropy} & \multicolumn{2}{c}{\budalg} & \multicolumn{2}{c}{error}\\
\cmidrule(rr){4-5}\cmidrule(rr){6-7}\cmidrule(rr){8-9}
&\multirow{1}{*}{$\#ex.$} & \multirow{1}{*}{\#feat.} &  \multicolumn{1}{c}{error} & \multicolumn{1}{c}{time} & \multicolumn{1}{c}{error} & \multicolumn{1}{c}{time} & \multicolumn{1}{c}{error} & \multicolumn{1}{c}{time} \\
\midrule

\texttt{anneal} & \multicolumn{1}{r}{812} & \multicolumn{1}{r}{47}  & 140 & \cellcolor{TealBlue!30}{\textbf{0.00}} & 137 & 0.00 & \cellcolor{TealBlue!30}{\textbf{130}} & 0.00\\
\texttt{audiology} & \multicolumn{1}{r}{216} & \multicolumn{1}{r}{79}  & \cellcolor{TealBlue!30}{6} & 0.00 & \cellcolor{TealBlue!30}{6} & 0.00 & \cellcolor{TealBlue!30}{6} & \cellcolor{TealBlue!30}{\textbf{0.00}}\\
\texttt{australian-credit} & \multicolumn{1}{r}{653} & \multicolumn{1}{r}{73}  & 84 & 0.00 & \cellcolor{TealBlue!30}{\textbf{82}} & \cellcolor{TealBlue!30}{\textbf{0.00}} & 86 & 0.00\\
\texttt{breast-cancer-un} & \multicolumn{1}{r}{683} & \multicolumn{1}{r}{89}  & 32 & 0.00 & \cellcolor{TealBlue!30}{28} & \cellcolor{TealBlue!30}{\textbf{0.00}} & \cellcolor{TealBlue!30}{28} & 0.00\\
\texttt{breast-wisconsin} & \multicolumn{1}{r}{683} & \multicolumn{1}{r}{120}  & \cellcolor{TealBlue!30}{23} & 0.00 & 26 & \cellcolor{TealBlue!30}{0.00} & \cellcolor{TealBlue!30}{23} & \cellcolor{TealBlue!30}{0.00}\\
\texttt{car-un} & \multicolumn{1}{r}{1728} & \multicolumn{1}{r}{21}  & \cellcolor{TealBlue!30}{202} & 0.00 & \cellcolor{TealBlue!30}{202} & \cellcolor{TealBlue!30}{\textbf{0.00}} & 300 & 0.00\\
\texttt{diabetes} & \multicolumn{1}{r}{768} & \multicolumn{1}{r}{112}  & \cellcolor{TealBlue!30}{169} & 0.00 & \cellcolor{TealBlue!30}{169} & \cellcolor{TealBlue!30}{\textbf{0.00}} & 183 & 0.00\\
\texttt{forest-fires-un} & \multicolumn{1}{r}{517} & \multicolumn{1}{r}{989}  & 209 & 0.00 & \cellcolor{TealBlue!30}{198} & \cellcolor{TealBlue!30}{\textbf{0.00}} & \cellcolor{TealBlue!30}{198} & 0.00\\
\texttt{german-credit} & \multicolumn{1}{r}{1000} & \multicolumn{1}{r}{110}  & \cellcolor{TealBlue!30}{249} & 0.00 & \cellcolor{TealBlue!30}{249} & \cellcolor{TealBlue!30}{\textbf{0.00}} & 271 & 0.00\\
\texttt{heart-cleveland} & \multicolumn{1}{r}{296} & \multicolumn{1}{r}{50}  & \cellcolor{TealBlue!30}{43} & 0.00 & \cellcolor{TealBlue!30}{43} & \cellcolor{TealBlue!30}{\textbf{0.00}} & 54 & 0.00\\
\texttt{hepatitis} & \multicolumn{1}{r}{137} & \multicolumn{1}{r}{68}  & \cellcolor{TealBlue!30}{14} & \cellcolor{TealBlue!30}{\textbf{0.00}} & \cellcolor{TealBlue!30}{14} & 0.00 & 16 & 0.00\\
\texttt{hypothyroid} & \multicolumn{1}{r}{3247} & \multicolumn{1}{r}{43}  & \cellcolor{TealBlue!30}{62} & 0.00 & \cellcolor{TealBlue!30}{62} & 0.00 & \cellcolor{TealBlue!30}{62} & \cellcolor{TealBlue!30}{\textbf{0.00}}\\
\texttt{ionosphere} & \multicolumn{1}{r}{351} & \multicolumn{1}{r}{444}  & \cellcolor{TealBlue!30}{29} & 0.00 & \cellcolor{TealBlue!30}{29} & 0.00 & \cellcolor{TealBlue!30}{29} & \cellcolor{TealBlue!30}{\textbf{0.00}}\\
\texttt{kr-vs-kp} & \multicolumn{1}{r}{3196} & \multicolumn{1}{r}{37}  & 306 & \cellcolor{TealBlue!30}{0.00} & 306 & 0.00 & \cellcolor{TealBlue!30}{\textbf{198}} & \cellcolor{TealBlue!30}{0.00}\\
\texttt{letter} & \multicolumn{1}{r}{20000} & \multicolumn{1}{r}{224}  & 801 & 0.03 & 657 & \cellcolor{TealBlue!30}{\textbf{0.03}} & \cellcolor{TealBlue!30}{\textbf{598}} & 0.03\\
\texttt{lymph} & \multicolumn{1}{r}{148} & \multicolumn{1}{r}{41}  & \cellcolor{TealBlue!30}{16} & 0.00 & \cellcolor{TealBlue!30}{16} & \cellcolor{TealBlue!30}{\textbf{0.00}} & 20 & 0.00\\
\texttt{mushroom} & \multicolumn{1}{r}{8124} & \multicolumn{1}{r}{91}  & 280 & 0.01 & 280 & \cellcolor{TealBlue!30}{\textbf{0.00}} & \cellcolor{TealBlue!30}{\textbf{24}} & 0.01\\
\texttt{pendigits} & \multicolumn{1}{r}{7494} & \multicolumn{1}{r}{216}  & 79 & 0.01 & \cellcolor{TealBlue!30}{\textbf{51}} & \cellcolor{TealBlue!30}{\textbf{0.01}} & 189 & 0.01\\
\texttt{primary-tumor} & \multicolumn{1}{r}{336} & \multicolumn{1}{r}{16}  & \cellcolor{TealBlue!30}{51} & 0.00 & \cellcolor{TealBlue!30}{51} & \cellcolor{TealBlue!30}{\textbf{0.00}} & 52 & 0.00\\
\texttt{segment} & \multicolumn{1}{r}{2310} & \multicolumn{1}{r}{234}  & \cellcolor{TealBlue!30}{5} & 0.00 & \cellcolor{TealBlue!30}{5} & \cellcolor{TealBlue!30}{\textbf{0.00}} & \cellcolor{TealBlue!30}{5} & 0.00\\
\texttt{soybean} & \multicolumn{1}{r}{630} & \multicolumn{1}{r}{34}  & 77 & 0.00 & \cellcolor{TealBlue!30}{\textbf{47}} & \cellcolor{TealBlue!30}{\textbf{0.00}} & 55 & 0.00\\
\texttt{splice-1} & \multicolumn{1}{r}{3190} & \multicolumn{1}{r}{227}  & \cellcolor{TealBlue!30}{279} & \cellcolor{TealBlue!30}{\textbf{0.00}} & \cellcolor{TealBlue!30}{279} & 0.01 & 340 & 0.01\\
\texttt{taiwan\_binarised} & \multicolumn{1}{r}{30000} & \multicolumn{1}{r}{198}  & 5350 & 0.04 & \cellcolor{TealBlue!30}{\textbf{5333}} & \cellcolor{TealBlue!30}{\textbf{0.04}} & 5342 & 0.05\\
\texttt{tic-tac-toe} & \multicolumn{1}{r}{958} & \multicolumn{1}{r}{18}  & \cellcolor{TealBlue!30}{236} & 0.00 & \cellcolor{TealBlue!30}{236} & \cellcolor{TealBlue!30}{\textbf{0.00}} & 241 & 0.00\\
\texttt{vehicle} & \multicolumn{1}{r}{846} & \multicolumn{1}{r}{252}  & 92 & 0.00 & \cellcolor{TealBlue!30}{\textbf{55}} & \cellcolor{TealBlue!30}{\textbf{0.00}} & 86 & 0.00\\
\texttt{vote} & \multicolumn{1}{r}{435} & \multicolumn{1}{r}{32}  & \cellcolor{TealBlue!30}{14} & 0.00 & \cellcolor{TealBlue!30}{14} & \cellcolor{TealBlue!30}{\textbf{0.00}} & 16 & 0.00\\
\texttt{wine1-un} & \multicolumn{1}{r}{178} & \multicolumn{1}{r}{1276}  & 49 & 0.00 & \cellcolor{TealBlue!30}{45} & \cellcolor{TealBlue!30}{\textbf{0.00}} & \cellcolor{TealBlue!30}{45} & 0.00\\
\texttt{wine2-un} & \multicolumn{1}{r}{178} & \multicolumn{1}{r}{1276}  & 53 & 0.00 & 52 & \cellcolor{TealBlue!30}{\textbf{0.00}} & \cellcolor{TealBlue!30}{\textbf{51}} & 0.00\\
\texttt{wine3-un} & \multicolumn{1}{r}{178} & \multicolumn{1}{r}{1276}  & \cellcolor{TealBlue!30}{35} & 0.00 & \cellcolor{TealBlue!30}{35} & \cellcolor{TealBlue!30}{\textbf{0.00}} & \cellcolor{TealBlue!30}{35} & 0.00\\
\texttt{yeast} & \multicolumn{1}{r}{1484} & \multicolumn{1}{r}{89}  & \cellcolor{TealBlue!30}{417} & 0.00 & \cellcolor{TealBlue!30}{417} & \cellcolor{TealBlue!30}{\textbf{0.00}} & 435 & 0.00\\
\texttt{zoo-1} & \multicolumn{1}{r}{101} & \multicolumn{1}{r}{20}  & \cellcolor{TealBlue!30}{0} & 0.00 & \cellcolor{TealBlue!30}{0} & \cellcolor{TealBlue!30}{\textbf{0.00}} & \cellcolor{TealBlue!30}{0} & 0.00\\
\bottomrule
\end{tabular}

% \end{normalsize}
% \end{center}
% \caption{\label{tab:ha3} Comparison of heuristics (max depth=3)}
% \end{table}
%
% \begin{table}[htbp]
% \begin{center}
% \begin{normalsize}
% \tabcolsep=5pt
% \begin{tabular}{lccrrrrrr}
\toprule
& && \multicolumn{2}{c}{entropy} & \multicolumn{2}{c}{\budalg} & \multicolumn{2}{c}{error}\\
\cmidrule(rr){4-5}\cmidrule(rr){6-7}\cmidrule(rr){8-9}
&\multirow{1}{*}{$\#ex.$} & \multirow{1}{*}{\#feat.} &  \multicolumn{1}{c}{error} & \multicolumn{1}{c}{time} & \multicolumn{1}{c}{error} & \multicolumn{1}{c}{time} & \multicolumn{1}{c}{error} & \multicolumn{1}{c}{time} \\
\midrule

\texttt{anneal} & \multicolumn{1}{r}{812} & \multicolumn{1}{r}{47}  & 138 & 0.00 & 135 & \cellcolor{TealBlue!30}{\textbf{0.00}} & \cellcolor{TealBlue!30}{\textbf{122}} & 0.00\\
\texttt{audiology} & \multicolumn{1}{r}{216} & \multicolumn{1}{r}{79}  & \cellcolor{TealBlue!30}{3} & 0.00 & \cellcolor{TealBlue!30}{3} & \cellcolor{TealBlue!30}{\textbf{0.00}} & \cellcolor{TealBlue!30}{3} & 0.00\\
\texttt{australian-credit} & \multicolumn{1}{r}{653} & \multicolumn{1}{r}{73}  & 75 & \cellcolor{TealBlue!30}{\textbf{0.00}} & \cellcolor{TealBlue!30}{\textbf{73}} & 0.00 & 83 & 0.00\\
\texttt{breast-cancer-un} & \multicolumn{1}{r}{683} & \multicolumn{1}{r}{89}  & 29 & 0.00 & \cellcolor{TealBlue!30}{21} & \cellcolor{TealBlue!30}{\textbf{0.00}} & \cellcolor{TealBlue!30}{21} & 0.00\\
\texttt{breast-wisconsin} & \multicolumn{1}{r}{683} & \multicolumn{1}{r}{120}  & 18 & 0.00 & \cellcolor{TealBlue!30}{16} & 0.00 & \cellcolor{TealBlue!30}{16} & \cellcolor{TealBlue!30}{\textbf{0.00}}\\
\texttt{car-un} & \multicolumn{1}{r}{1728} & \multicolumn{1}{r}{21}  & \cellcolor{TealBlue!30}{178} & 0.00 & \cellcolor{TealBlue!30}{178} & \cellcolor{TealBlue!30}{\textbf{0.00}} & 278 & 0.00\\
\texttt{diabetes} & \multicolumn{1}{r}{768} & \multicolumn{1}{r}{112}  & 161 & 0.00 & \cellcolor{TealBlue!30}{\textbf{159}} & 0.00 & 163 & \cellcolor{TealBlue!30}{\textbf{0.00}}\\
\texttt{forest-fires-un} & \multicolumn{1}{r}{517} & \multicolumn{1}{r}{989}  & 201 & 0.00 & \cellcolor{TealBlue!30}{186} & 0.00 & \cellcolor{TealBlue!30}{186} & \cellcolor{TealBlue!30}{\textbf{0.00}}\\
\texttt{german-credit} & \multicolumn{1}{r}{1000} & \multicolumn{1}{r}{110}  & 225 & 0.00 & \cellcolor{TealBlue!30}{\textbf{224}} & 0.00 & 246 & \cellcolor{TealBlue!30}{\textbf{0.00}}\\
\texttt{heart-cleveland} & \multicolumn{1}{r}{296} & \multicolumn{1}{r}{50}  & 39 & 0.00 & \cellcolor{TealBlue!30}{\textbf{36}} & 0.00 & 45 & \cellcolor{TealBlue!30}{\textbf{0.00}}\\
\texttt{hepatitis} & \multicolumn{1}{r}{137} & \multicolumn{1}{r}{68}  & 13 & 0.00 & \cellcolor{TealBlue!30}{\textbf{12}} & 0.00 & 14 & \cellcolor{TealBlue!30}{\textbf{0.00}}\\
\texttt{hypothyroid} & \multicolumn{1}{r}{3247} & \multicolumn{1}{r}{43}  & 54 & 0.00 & \cellcolor{TealBlue!30}{\textbf{53}} & \cellcolor{TealBlue!30}{\textbf{0.00}} & 55 & 0.00\\
\texttt{ionosphere} & \multicolumn{1}{r}{351} & \multicolumn{1}{r}{444}  & \cellcolor{TealBlue!30}{\textbf{22}} & 0.00 & 25 & \cellcolor{TealBlue!30}{\textbf{0.00}} & 26 & 0.00\\
\texttt{kr-vs-kp} & \multicolumn{1}{r}{3196} & \multicolumn{1}{r}{37}  & 189 & 0.00 & 188 & \cellcolor{TealBlue!30}{\textbf{0.00}} & \cellcolor{TealBlue!30}{\textbf{181}} & 0.00\\
\texttt{letter} & \multicolumn{1}{r}{20000} & \multicolumn{1}{r}{224}  & 732 & 0.03 & \cellcolor{TealBlue!30}{\textbf{443}} & \cellcolor{TealBlue!30}{\textbf{0.03}} & 589 & 0.04\\
\texttt{lymph} & \multicolumn{1}{r}{148} & \multicolumn{1}{r}{41}  & \cellcolor{TealBlue!30}{9} & 0.00 & \cellcolor{TealBlue!30}{9} & 0.00 & 17 & \cellcolor{TealBlue!30}{\textbf{0.00}}\\
\texttt{mushroom} & \multicolumn{1}{r}{8124} & \multicolumn{1}{r}{91}  & \cellcolor{TealBlue!30}{4} & 0.01 & \cellcolor{TealBlue!30}{4} & 0.01 & 16 & \cellcolor{TealBlue!30}{\textbf{0.00}}\\
\texttt{pendigits} & \multicolumn{1}{r}{7494} & \multicolumn{1}{r}{216}  & 27 & \cellcolor{TealBlue!30}{\textbf{0.01}} & \cellcolor{TealBlue!30}{\textbf{22}} & 0.01 & 63 & 0.01\\
\texttt{primary-tumor} & \multicolumn{1}{r}{336} & \multicolumn{1}{r}{16}  & 45 & 0.00 & \cellcolor{TealBlue!30}{\textbf{43}} & 0.00 & 46 & \cellcolor{TealBlue!30}{\textbf{0.00}}\\
\texttt{segment} & \multicolumn{1}{r}{2310} & \multicolumn{1}{r}{234}  & \cellcolor{TealBlue!30}{1} & 0.00 & \cellcolor{TealBlue!30}{1} & \cellcolor{TealBlue!30}{\textbf{0.00}} & \cellcolor{TealBlue!30}{1} & 0.00\\
\texttt{soybean} & \multicolumn{1}{r}{630} & \multicolumn{1}{r}{34}  & 71 & 0.00 & \cellcolor{TealBlue!30}{\textbf{32}} & 0.00 & 42 & \cellcolor{TealBlue!30}{\textbf{0.00}}\\
\texttt{splice-1} & \multicolumn{1}{r}{3190} & \multicolumn{1}{r}{227}  & \cellcolor{TealBlue!30}{141} & 0.01 & \cellcolor{TealBlue!30}{141} & \cellcolor{TealBlue!30}{\textbf{0.01}} & 292 & 0.01\\
\texttt{taiwan\_binarised} & \multicolumn{1}{r}{30000} & \multicolumn{1}{r}{198}  & 5305 & 0.05 & \cellcolor{TealBlue!30}{\textbf{5293}} & 0.04 & 5308 & \cellcolor{TealBlue!30}{\textbf{0.04}}\\
\texttt{tic-tac-toe} & \multicolumn{1}{r}{958} & \multicolumn{1}{r}{18}  & \cellcolor{TealBlue!30}{150} & 0.00 & \cellcolor{TealBlue!30}{150} & 0.00 & 179 & \cellcolor{TealBlue!30}{\textbf{0.00}}\\
\texttt{vehicle} & \multicolumn{1}{r}{846} & \multicolumn{1}{r}{252}  & 41 & 0.00 & \cellcolor{TealBlue!30}{\textbf{28}} & \cellcolor{TealBlue!30}{\textbf{0.00}} & 64 & 0.00\\
\texttt{vote} & \multicolumn{1}{r}{435} & \multicolumn{1}{r}{32}  & \cellcolor{TealBlue!30}{8} & 0.00 & \cellcolor{TealBlue!30}{8} & 0.00 & 12 & \cellcolor{TealBlue!30}{\textbf{0.00}}\\
\texttt{wine1-un} & \multicolumn{1}{r}{178} & \multicolumn{1}{r}{1276}  & 49 & 0.00 & 42 & \cellcolor{TealBlue!30}{\textbf{0.00}} & \cellcolor{TealBlue!30}{\textbf{41}} & 0.00\\
\texttt{wine2-un} & \multicolumn{1}{r}{178} & \multicolumn{1}{r}{1276}  & 48 & 0.00 & \cellcolor{TealBlue!30}{47} & \cellcolor{TealBlue!30}{\textbf{0.00}} & \cellcolor{TealBlue!30}{47} & 0.00\\
\texttt{wine3-un} & \multicolumn{1}{r}{178} & \multicolumn{1}{r}{1276}  & \cellcolor{TealBlue!30}{32} & 0.00 & \cellcolor{TealBlue!30}{32} & \cellcolor{TealBlue!30}{\textbf{0.00}} & \cellcolor{TealBlue!30}{32} & 0.00\\
\texttt{yeast} & \multicolumn{1}{r}{1484} & \multicolumn{1}{r}{89}  & \cellcolor{TealBlue!30}{\textbf{391}} & 0.00 & 392 & 0.00 & 430 & \cellcolor{TealBlue!30}{\textbf{0.00}}\\
\texttt{zoo-1} & \multicolumn{1}{r}{101} & \multicolumn{1}{r}{20}  & \cellcolor{TealBlue!30}{0} & 0.00 & \cellcolor{TealBlue!30}{0} & \cellcolor{TealBlue!30}{\textbf{0.00}} & \cellcolor{TealBlue!30}{0} & 0.00\\
\bottomrule
\end{tabular}

% \end{normalsize}
% \end{center}
% \caption{\label{tab:ha4} Comparison of heuristics (max depth=4)}
% \end{table}
%
% \begin{table}[htbp]
% \begin{center}
% \begin{normalsize}
% \tabcolsep=5pt
% \begin{tabular}{lccrrrrrr}
\toprule
& && \multicolumn{2}{c}{entropy} & \multicolumn{2}{c}{\budalg} & \multicolumn{2}{c}{error}\\
\cmidrule(rr){4-5}\cmidrule(rr){6-7}\cmidrule(rr){8-9}
&\multirow{1}{*}{$\#ex.$} & \multirow{1}{*}{\#feat.} &  \multicolumn{1}{c}{error} & \multicolumn{1}{c}{time} & \multicolumn{1}{c}{error} & \multicolumn{1}{c}{time} & \multicolumn{1}{c}{error} & \multicolumn{1}{c}{time} \\
\midrule

\texttt{anneal} & \multicolumn{1}{r}{812} & \multicolumn{1}{r}{47}  & 125 & 0.00 & \cellcolor{TealBlue!30}{\textbf{114}} & \cellcolor{TealBlue!30}{\textbf{0.00}} & 117 & 0.00\\
\texttt{audiology} & \multicolumn{1}{r}{216} & \multicolumn{1}{r}{79}  & \cellcolor{TealBlue!30}{2} & 0.00 & \cellcolor{TealBlue!30}{2} & \cellcolor{TealBlue!30}{\textbf{0.00}} & \cellcolor{TealBlue!30}{2} & 0.00\\
\texttt{australian-credit} & \multicolumn{1}{r}{653} & \multicolumn{1}{r}{73}  & \cellcolor{TealBlue!30}{\textbf{62}} & 0.00 & 63 & 0.00 & 78 & \cellcolor{TealBlue!30}{\textbf{0.00}}\\
\texttt{breast-cancer-un} & \multicolumn{1}{r}{683} & \multicolumn{1}{r}{89}  & 22 & 0.00 & \cellcolor{TealBlue!30}{\textbf{16}} & 0.00 & 17 & \cellcolor{TealBlue!30}{\textbf{0.00}}\\
\texttt{breast-wisconsin} & \multicolumn{1}{r}{683} & \multicolumn{1}{r}{120}  & \cellcolor{TealBlue!30}{\textbf{10}} & 0.00 & 13 & \cellcolor{TealBlue!30}{\textbf{0.00}} & 15 & 0.00\\
\texttt{car-un} & \multicolumn{1}{r}{1728} & \multicolumn{1}{r}{21}  & \cellcolor{TealBlue!30}{106} & 0.00 & \cellcolor{TealBlue!30}{106} & \cellcolor{TealBlue!30}{\textbf{0.00}} & 202 & 0.00\\
\texttt{diabetes} & \multicolumn{1}{r}{768} & \multicolumn{1}{r}{112}  & 143 & 0.00 & \cellcolor{TealBlue!30}{\textbf{141}} & \cellcolor{TealBlue!30}{\textbf{0.00}} & 145 & 0.00\\
\texttt{forest-fires-un} & \multicolumn{1}{r}{517} & \multicolumn{1}{r}{989}  & 189 & 0.00 & 176 & \cellcolor{TealBlue!30}{\textbf{0.00}} & \cellcolor{TealBlue!30}{\textbf{175}} & 0.00\\
\texttt{german-credit} & \multicolumn{1}{r}{1000} & \multicolumn{1}{r}{110}  & 215 & 0.00 & \cellcolor{TealBlue!30}{\textbf{201}} & 0.00 & 236 & \cellcolor{TealBlue!30}{\textbf{0.00}}\\
\texttt{heart-cleveland} & \multicolumn{1}{r}{296} & \multicolumn{1}{r}{50}  & \cellcolor{TealBlue!30}{\textbf{26}} & 0.00 & 28 & 0.00 & 36 & \cellcolor{TealBlue!30}{\textbf{0.00}}\\
\texttt{hepatitis} & \multicolumn{1}{r}{137} & \multicolumn{1}{r}{68}  & 10 & 0.00 & \cellcolor{TealBlue!30}{8} & 0.00 & \cellcolor{TealBlue!30}{8} & \cellcolor{TealBlue!30}{\textbf{0.00}}\\
\texttt{hypothyroid} & \multicolumn{1}{r}{3247} & \multicolumn{1}{r}{43}  & \cellcolor{TealBlue!30}{51} & 0.00 & \cellcolor{TealBlue!30}{51} & \cellcolor{TealBlue!30}{\textbf{0.00}} & 55 & 0.00\\
\texttt{ionosphere} & \multicolumn{1}{r}{351} & \multicolumn{1}{r}{444}  & \cellcolor{TealBlue!30}{16} & 0.00 & \cellcolor{TealBlue!30}{16} & 0.00 & 24 & \cellcolor{TealBlue!30}{\textbf{0.00}}\\
\texttt{kr-vs-kp} & \multicolumn{1}{r}{3196} & \multicolumn{1}{r}{37}  & \cellcolor{TealBlue!30}{179} & 0.00 & \cellcolor{TealBlue!30}{179} & \cellcolor{TealBlue!30}{\textbf{0.00}} & 180 & 0.00\\
\texttt{letter} & \multicolumn{1}{r}{20000} & \multicolumn{1}{r}{224}  & 478 & 0.04 & \cellcolor{TealBlue!30}{\textbf{335}} & \cellcolor{TealBlue!30}{\textbf{0.03}} & 579 & 0.04\\
\texttt{lymph} & \multicolumn{1}{r}{148} & \multicolumn{1}{r}{41}  & \cellcolor{TealBlue!30}{4} & 0.00 & \cellcolor{TealBlue!30}{4} & \cellcolor{TealBlue!30}{\textbf{0.00}} & 11 & 0.00\\
\texttt{mushroom} & \multicolumn{1}{r}{8124} & \multicolumn{1}{r}{91}  & \cellcolor{TealBlue!30}{3} & 0.01 & \cellcolor{TealBlue!30}{3} & \cellcolor{TealBlue!30}{\textbf{0.01}} & 12 & 0.01\\
\texttt{pendigits} & \multicolumn{1}{r}{7494} & \multicolumn{1}{r}{216}  & 15 & 0.01 & \cellcolor{TealBlue!30}{\textbf{11}} & \cellcolor{TealBlue!30}{\textbf{0.01}} & 49 & 0.01\\
\texttt{primary-tumor} & \multicolumn{1}{r}{336} & \multicolumn{1}{r}{16}  & 37 & \cellcolor{TealBlue!30}{\textbf{0.00}} & \cellcolor{TealBlue!30}{\textbf{34}} & 0.00 & 41 & 0.00\\
\texttt{segment} & \multicolumn{1}{r}{2310} & \multicolumn{1}{r}{234}  & \cellcolor{TealBlue!30}{1} & 0.00 & \cellcolor{TealBlue!30}{1} & \cellcolor{TealBlue!30}{\textbf{0.00}} & \cellcolor{TealBlue!30}{1} & 0.00\\
\texttt{soybean} & \multicolumn{1}{r}{630} & \multicolumn{1}{r}{34}  & 54 & 0.00 & \cellcolor{TealBlue!30}{\textbf{23}} & \cellcolor{TealBlue!30}{\textbf{0.00}} & 31 & 0.00\\
\texttt{splice-1} & \multicolumn{1}{r}{3190} & \multicolumn{1}{r}{227}  & \cellcolor{TealBlue!30}{111} & 0.01 & \cellcolor{TealBlue!30}{111} & \cellcolor{TealBlue!30}{\textbf{0.00}} & 256 & 0.00\\
\texttt{taiwan\_binarised} & \multicolumn{1}{r}{30000} & \multicolumn{1}{r}{198}  & 5274 & 0.05 & \cellcolor{TealBlue!30}{\textbf{5257}} & \cellcolor{TealBlue!30}{\textbf{0.04}} & 5280 & 0.04\\
\texttt{tic-tac-toe} & \multicolumn{1}{r}{958} & \multicolumn{1}{r}{18}  & \cellcolor{TealBlue!30}{78} & \cellcolor{TealBlue!30}{\textbf{0.00}} & \cellcolor{TealBlue!30}{78} & 0.00 & 152 & 0.00\\
\texttt{vehicle} & \multicolumn{1}{r}{846} & \multicolumn{1}{r}{252}  & 22 & 0.00 & \cellcolor{TealBlue!30}{\textbf{21}} & 0.00 & 55 & \cellcolor{TealBlue!30}{\textbf{0.00}}\\
\texttt{vote} & \multicolumn{1}{r}{435} & \multicolumn{1}{r}{32}  & \cellcolor{TealBlue!30}{6} & 0.00 & \cellcolor{TealBlue!30}{6} & 0.00 & 7 & \cellcolor{TealBlue!30}{\textbf{0.00}}\\
\texttt{wine1-un} & \multicolumn{1}{r}{178} & \multicolumn{1}{r}{1276}  & 45 & \cellcolor{TealBlue!30}{\textbf{0.00}} & 39 & 0.00 & \cellcolor{TealBlue!30}{\textbf{38}} & 0.00\\
\texttt{wine2-un} & \multicolumn{1}{r}{178} & \multicolumn{1}{r}{1276}  & 44 & 0.00 & 44 & 0.00 & \cellcolor{TealBlue!30}{\textbf{42}} & \cellcolor{TealBlue!30}{\textbf{0.00}}\\
\texttt{wine3-un} & \multicolumn{1}{r}{178} & \multicolumn{1}{r}{1276}  & 30 & 0.00 & 30 & 0.00 & \cellcolor{TealBlue!30}{\textbf{29}} & \cellcolor{TealBlue!30}{\textbf{0.00}}\\
\texttt{yeast} & \multicolumn{1}{r}{1484} & \multicolumn{1}{r}{89}  & \cellcolor{TealBlue!30}{365} & 0.00 & \cellcolor{TealBlue!30}{365} & 0.00 & 429 & \cellcolor{TealBlue!30}{\textbf{0.00}}\\
\texttt{zoo-1} & \multicolumn{1}{r}{101} & \multicolumn{1}{r}{20}  & \cellcolor{TealBlue!30}{0} & 0.00 & \cellcolor{TealBlue!30}{0} & \cellcolor{TealBlue!30}{\textbf{0.00}} & \cellcolor{TealBlue!30}{0} & 0.00\\
\bottomrule
\end{tabular}

% \end{normalsize}
% \end{center}
% \caption{\label{tab:ha5} Comparison of heuristics (max depth=5)}
% \end{table}
%
% \begin{table}[htbp]
% \begin{center}
% \begin{normalsize}
% \tabcolsep=5pt
% \begin{tabular}{lccrrrrrr}
\toprule
& && \multicolumn{2}{c}{entropy} & \multicolumn{2}{c}{\budalg} & \multicolumn{2}{c}{error}\\
\cmidrule(rr){4-5}\cmidrule(rr){6-7}\cmidrule(rr){8-9}
&\multirow{1}{*}{$\#ex.$} & \multirow{1}{*}{\#feat.} &  \multicolumn{1}{c}{error} & \multicolumn{1}{c}{time} & \multicolumn{1}{c}{error} & \multicolumn{1}{c}{time} & \multicolumn{1}{c}{error} & \multicolumn{1}{c}{time} \\
\midrule

\texttt{anneal} & \multicolumn{1}{r}{812} & \multicolumn{1}{r}{47}  & 104 & 0.00 & \cellcolor{TealBlue!30}{\textbf{94}} & 0.00 & 113 & \cellcolor{TealBlue!30}{\textbf{0.00}}\\
\texttt{audiology} & \multicolumn{1}{r}{216} & \multicolumn{1}{r}{79}  & \cellcolor{TealBlue!30}{0} & 0.00 & \cellcolor{TealBlue!30}{0} & \cellcolor{TealBlue!30}{\textbf{0.00}} & 1 & 0.00\\
\texttt{australian-credit} & \multicolumn{1}{r}{653} & \multicolumn{1}{r}{73}  & \cellcolor{TealBlue!30}{43} & 0.00 & \cellcolor{TealBlue!30}{43} & 0.00 & 71 & \cellcolor{TealBlue!30}{\textbf{0.00}}\\
\texttt{breast-cancer-un} & \multicolumn{1}{r}{683} & \multicolumn{1}{r}{89}  & 12 & 0.00 & \cellcolor{TealBlue!30}{\textbf{8}} & \cellcolor{TealBlue!30}{\textbf{0.00}} & 11 & 0.00\\
\texttt{breast-wisconsin} & \multicolumn{1}{r}{683} & \multicolumn{1}{r}{120}  & \cellcolor{TealBlue!30}{4} & 0.00 & \cellcolor{TealBlue!30}{4} & \cellcolor{TealBlue!30}{\textbf{0.00}} & 13 & 0.00\\
\texttt{car-un} & \multicolumn{1}{r}{1728} & \multicolumn{1}{r}{21}  & \cellcolor{TealBlue!30}{50} & 0.00 & \cellcolor{TealBlue!30}{50} & \cellcolor{TealBlue!30}{\textbf{0.00}} & 87 & 0.00\\
\texttt{diabetes} & \multicolumn{1}{r}{768} & \multicolumn{1}{r}{112}  & 101 & 0.00 & \cellcolor{TealBlue!30}{\textbf{99}} & 0.00 & 115 & \cellcolor{TealBlue!30}{\textbf{0.00}}\\
\texttt{forest-fires-un} & \multicolumn{1}{r}{517} & \multicolumn{1}{r}{989}  & 169 & 0.00 & 161 & \cellcolor{TealBlue!30}{\textbf{0.00}} & \cellcolor{TealBlue!30}{\textbf{159}} & 0.00\\
\texttt{german-credit} & \multicolumn{1}{r}{1000} & \multicolumn{1}{r}{110}  & 152 & 0.00 & \cellcolor{TealBlue!30}{\textbf{141}} & 0.00 & 213 & \cellcolor{TealBlue!30}{\textbf{0.00}}\\
\texttt{heart-cleveland} & \multicolumn{1}{r}{296} & \multicolumn{1}{r}{50}  & 11 & 0.00 & \cellcolor{TealBlue!30}{\textbf{7}} & \cellcolor{TealBlue!30}{\textbf{0.00}} & 24 & 0.00\\
\texttt{hepatitis} & \multicolumn{1}{r}{137} & \multicolumn{1}{r}{68}  & \cellcolor{TealBlue!30}{0} & 0.00 & \cellcolor{TealBlue!30}{0} & 0.00 & 3 & \cellcolor{TealBlue!30}{\textbf{0.00}}\\
\texttt{hypothyroid} & \multicolumn{1}{r}{3247} & \multicolumn{1}{r}{43}  & 45 & 0.00 & \cellcolor{TealBlue!30}{\textbf{43}} & \cellcolor{TealBlue!30}{\textbf{0.00}} & 52 & 0.00\\
\texttt{ionosphere} & \multicolumn{1}{r}{351} & \multicolumn{1}{r}{444}  & \cellcolor{TealBlue!30}{\textbf{6}} & 0.00 & 7 & 0.00 & 9 & \cellcolor{TealBlue!30}{\textbf{0.00}}\\
\texttt{kr-vs-kp} & \multicolumn{1}{r}{3196} & \multicolumn{1}{r}{37}  & \cellcolor{TealBlue!30}{\textbf{101}} & 0.00 & 102 & \cellcolor{TealBlue!30}{\textbf{0.00}} & 166 & 0.00\\
\texttt{letter} & \multicolumn{1}{r}{20000} & \multicolumn{1}{r}{224}  & 211 & 0.03 & \cellcolor{TealBlue!30}{\textbf{143}} & 0.04 & 539 & \cellcolor{TealBlue!30}{\textbf{0.03}}\\
\texttt{lymph} & \multicolumn{1}{r}{148} & \multicolumn{1}{r}{41}  & \cellcolor{TealBlue!30}{0} & \cellcolor{TealBlue!30}{\textbf{0.00}} & \cellcolor{TealBlue!30}{0} & 0.00 & 8 & 0.00\\
\texttt{mushroom} & \multicolumn{1}{r}{8124} & \multicolumn{1}{r}{91}  & \cellcolor{TealBlue!30}{0} & 0.01 & \cellcolor{TealBlue!30}{0} & \cellcolor{TealBlue!30}{\textbf{0.00}} & 11 & 0.01\\
\texttt{pendigits} & \multicolumn{1}{r}{7494} & \multicolumn{1}{r}{216}  & \cellcolor{TealBlue!30}{1} & 0.01 & \cellcolor{TealBlue!30}{1} & \cellcolor{TealBlue!30}{\textbf{0.01}} & 14 & 0.01\\
\texttt{primary-tumor} & \multicolumn{1}{r}{336} & \multicolumn{1}{r}{16}  & 28 & 0.00 & \cellcolor{TealBlue!30}{26} & 0.00 & \cellcolor{TealBlue!30}{26} & \cellcolor{TealBlue!30}{\textbf{0.00}}\\
\texttt{segment} & \multicolumn{1}{r}{2310} & \multicolumn{1}{r}{234}  & \cellcolor{TealBlue!30}{0} & 0.00 & \cellcolor{TealBlue!30}{0} & 0.00 & \cellcolor{TealBlue!30}{0} & \cellcolor{TealBlue!30}{\textbf{0.00}}\\
\texttt{soybean} & \multicolumn{1}{r}{630} & \multicolumn{1}{r}{34}  & 33 & 0.00 & \cellcolor{TealBlue!30}{\textbf{11}} & 0.00 & 29 & \cellcolor{TealBlue!30}{\textbf{0.00}}\\
\texttt{splice-1} & \multicolumn{1}{r}{3190} & \multicolumn{1}{r}{227}  & 60 & 0.01 & \cellcolor{TealBlue!30}{\textbf{58}} & \cellcolor{TealBlue!30}{\textbf{0.00}} & 202 & 0.01\\
\texttt{taiwan\_binarised} & \multicolumn{1}{r}{30000} & \multicolumn{1}{r}{198}  & 5160 & 0.06 & \cellcolor{TealBlue!30}{\textbf{5121}} & \cellcolor{TealBlue!30}{\textbf{0.04}} & 5228 & 0.05\\
\texttt{tic-tac-toe} & \multicolumn{1}{r}{958} & \multicolumn{1}{r}{18}  & \cellcolor{TealBlue!30}{21} & \cellcolor{TealBlue!30}{\textbf{0.00}} & \cellcolor{TealBlue!30}{21} & 0.00 & 80 & 0.00\\
\texttt{vehicle} & \multicolumn{1}{r}{846} & \multicolumn{1}{r}{252}  & 7 & 0.00 & \cellcolor{TealBlue!30}{\textbf{4}} & \cellcolor{TealBlue!30}{0.00} & 44 & \cellcolor{TealBlue!30}{0.00}\\
\texttt{vote} & \multicolumn{1}{r}{435} & \multicolumn{1}{r}{32}  & \cellcolor{TealBlue!30}{2} & 0.00 & \cellcolor{TealBlue!30}{2} & \cellcolor{TealBlue!30}{\textbf{0.00}} & 4 & 0.00\\
\texttt{wine1-un} & \multicolumn{1}{r}{178} & \multicolumn{1}{r}{1276}  & 39 & 0.00 & 33 & \cellcolor{TealBlue!30}{\textbf{0.00}} & \cellcolor{TealBlue!30}{\textbf{30}} & 0.00\\
\texttt{wine2-un} & \multicolumn{1}{r}{178} & \multicolumn{1}{r}{1276}  & 38 & 0.00 & 38 & 0.00 & \cellcolor{TealBlue!30}{\textbf{32}} & \cellcolor{TealBlue!30}{\textbf{0.00}}\\
\texttt{wine3-un} & \multicolumn{1}{r}{178} & \multicolumn{1}{r}{1276}  & \cellcolor{TealBlue!30}{\textbf{24}} & 0.00 & 26 & 0.00 & 25 & \cellcolor{TealBlue!30}{\textbf{0.00}}\\
\texttt{yeast} & \multicolumn{1}{r}{1484} & \multicolumn{1}{r}{89}  & 309 & 0.00 & \cellcolor{TealBlue!30}{\textbf{305}} & 0.00 & 376 & \cellcolor{TealBlue!30}{\textbf{0.00}}\\
\texttt{zoo-1} & \multicolumn{1}{r}{101} & \multicolumn{1}{r}{20}  & \cellcolor{TealBlue!30}{0} & 0.00 & \cellcolor{TealBlue!30}{0} & \cellcolor{TealBlue!30}{\textbf{0.00}} & \cellcolor{TealBlue!30}{0} & 0.00\\
\bottomrule
\end{tabular}

% \end{normalsize}
% \end{center}
% \caption{\label{tab:ha7} Comparison of heuristics (max depth=7)}
% \end{table}
%
% \begin{table}[htbp]
% \begin{center}
% \begin{normalsize}
% \tabcolsep=5pt
% \begin{tabular}{lccrrrrrr}
\toprule
& && \multicolumn{2}{c}{entropy} & \multicolumn{2}{c}{\budalg} & \multicolumn{2}{c}{error}\\
\cmidrule(rr){4-5}\cmidrule(rr){6-7}\cmidrule(rr){8-9}
&\multirow{1}{*}{$\#ex.$} & \multirow{1}{*}{\#feat.} &  \multicolumn{1}{c}{error} & \multicolumn{1}{c}{time} & \multicolumn{1}{c}{error} & \multicolumn{1}{c}{time} & \multicolumn{1}{c}{error} & \multicolumn{1}{c}{time} \\
\midrule

\texttt{anneal} & \multicolumn{1}{r}{812} & \multicolumn{1}{r}{47}  & 76 & 0.00 & \cellcolor{TealBlue!30}{\textbf{58}} & \cellcolor{TealBlue!30}{\textbf{0.00}} & 102 & 0.00\\
\texttt{audiology} & \multicolumn{1}{r}{216} & \multicolumn{1}{r}{79}  & \cellcolor{TealBlue!30}{0} & 0.00 & \cellcolor{TealBlue!30}{0} & 0.00 & \cellcolor{TealBlue!30}{0} & \cellcolor{TealBlue!30}{\textbf{0.00}}\\
\texttt{australian-credit} & \multicolumn{1}{r}{653} & \multicolumn{1}{r}{73}  & \cellcolor{TealBlue!30}{\textbf{12}} & 0.00 & 13 & 0.00 & 47 & \cellcolor{TealBlue!30}{\textbf{0.00}}\\
\texttt{breast-cancer-un} & \multicolumn{1}{r}{683} & \multicolumn{1}{r}{89}  & 6 & 0.00 & \cellcolor{TealBlue!30}{\textbf{0}} & \cellcolor{TealBlue!30}{\textbf{0.00}} & 10 & 0.00\\
\texttt{breast-wisconsin} & \multicolumn{1}{r}{683} & \multicolumn{1}{r}{120}  & 1 & 0.00 & \cellcolor{TealBlue!30}{\textbf{0}} & \cellcolor{TealBlue!30}{\textbf{0.00}} & 8 & 0.00\\
\texttt{car-un} & \multicolumn{1}{r}{1728} & \multicolumn{1}{r}{21}  & \cellcolor{TealBlue!30}{11} & 0.00 & \cellcolor{TealBlue!30}{11} & \cellcolor{TealBlue!30}{\textbf{0.00}} & 43 & 0.00\\
\texttt{diabetes} & \multicolumn{1}{r}{768} & \multicolumn{1}{r}{112}  & \cellcolor{TealBlue!30}{\textbf{38}} & 0.00 & 39 & \cellcolor{TealBlue!30}{\textbf{0.00}} & 79 & 0.00\\
\texttt{forest-fires-un} & \multicolumn{1}{r}{517} & \multicolumn{1}{r}{989}  & 154 & 0.00 & 145 & 0.00 & \cellcolor{TealBlue!30}{\textbf{141}} & \cellcolor{TealBlue!30}{\textbf{0.00}}\\
\texttt{german-credit} & \multicolumn{1}{r}{1000} & \multicolumn{1}{r}{110}  & 85 & 0.00 & \cellcolor{TealBlue!30}{\textbf{64}} & 0.00 & 176 & \cellcolor{TealBlue!30}{\textbf{0.00}}\\
\texttt{heart-cleveland} & \multicolumn{1}{r}{296} & \multicolumn{1}{r}{50}  & 1 & \cellcolor{TealBlue!30}{\textbf{0.00}} & \cellcolor{TealBlue!30}{\textbf{0}} & 0.00 & 13 & 0.00\\
\texttt{hepatitis} & \multicolumn{1}{r}{137} & \multicolumn{1}{r}{68}  & \cellcolor{TealBlue!30}{0} & 0.00 & \cellcolor{TealBlue!30}{0} & \cellcolor{TealBlue!30}{\textbf{0.00}} & 1 & 0.00\\
\texttt{hypothyroid} & \multicolumn{1}{r}{3247} & \multicolumn{1}{r}{43}  & \cellcolor{TealBlue!30}{32} & 0.00 & \cellcolor{TealBlue!30}{32} & \cellcolor{TealBlue!30}{\textbf{0.00}} & 45 & 0.00\\
\texttt{ionosphere} & \multicolumn{1}{r}{351} & \multicolumn{1}{r}{444}  & \cellcolor{TealBlue!30}{0} & \cellcolor{TealBlue!30}{\textbf{0.00}} & \cellcolor{TealBlue!30}{0} & 0.00 & 3 & 0.00\\
\texttt{kr-vs-kp} & \multicolumn{1}{r}{3196} & \multicolumn{1}{r}{37}  & \cellcolor{TealBlue!30}{12} & 0.00 & \cellcolor{TealBlue!30}{12} & \cellcolor{TealBlue!30}{\textbf{0.00}} & 134 & 0.00\\
\texttt{letter} & \multicolumn{1}{r}{20000} & \multicolumn{1}{r}{224}  & 124 & 0.04 & \cellcolor{TealBlue!30}{\textbf{20}} & \cellcolor{TealBlue!30}{\textbf{0.03}} & 305 & 0.03\\
\texttt{lymph} & \multicolumn{1}{r}{148} & \multicolumn{1}{r}{41}  & \cellcolor{TealBlue!30}{0} & 0.00 & \cellcolor{TealBlue!30}{0} & \cellcolor{TealBlue!30}{\textbf{0.00}} & 1 & 0.00\\
\texttt{mushroom} & \multicolumn{1}{r}{8124} & \multicolumn{1}{r}{91}  & \cellcolor{TealBlue!30}{0} & 0.01 & \cellcolor{TealBlue!30}{0} & \cellcolor{TealBlue!30}{\textbf{0.01}} & 3 & 0.01\\
\texttt{pendigits} & \multicolumn{1}{r}{7494} & \multicolumn{1}{r}{216}  & \cellcolor{TealBlue!30}{0} & 0.01 & \cellcolor{TealBlue!30}{0} & \cellcolor{TealBlue!30}{\textbf{0.01}} & 13 & 0.01\\
\texttt{primary-tumor} & \multicolumn{1}{r}{336} & \multicolumn{1}{r}{16}  & \cellcolor{TealBlue!30}{\textbf{19}} & 0.00 & 20 & 0.00 & 20 & \cellcolor{TealBlue!30}{\textbf{0.00}}\\
\texttt{segment} & \multicolumn{1}{r}{2310} & \multicolumn{1}{r}{234}  & \cellcolor{TealBlue!30}{0} & 0.00 & \cellcolor{TealBlue!30}{0} & 0.00 & \cellcolor{TealBlue!30}{0} & \cellcolor{TealBlue!30}{\textbf{0.00}}\\
\texttt{soybean} & \multicolumn{1}{r}{630} & \multicolumn{1}{r}{34}  & 7 & 0.00 & \cellcolor{TealBlue!30}{\textbf{2}} & 0.00 & 13 & \cellcolor{TealBlue!30}{\textbf{0.00}}\\
\texttt{splice-1} & \multicolumn{1}{r}{3190} & \multicolumn{1}{r}{227}  & 13 & \cellcolor{TealBlue!30}{\textbf{0.01}} & \cellcolor{TealBlue!30}{\textbf{12}} & 0.01 & 128 & 0.01\\
\texttt{taiwan\_binarised} & \multicolumn{1}{r}{30000} & \multicolumn{1}{r}{198}  & 4779 & 0.07 & \cellcolor{TealBlue!30}{\textbf{4668}} & 0.05 & 5152 & \cellcolor{TealBlue!30}{\textbf{0.04}}\\
\texttt{tic-tac-toe} & \multicolumn{1}{r}{958} & \multicolumn{1}{r}{18}  & \cellcolor{TealBlue!30}{6} & 0.00 & \cellcolor{TealBlue!30}{6} & \cellcolor{TealBlue!30}{\textbf{0.00}} & 14 & 0.00\\
\texttt{vehicle} & \multicolumn{1}{r}{846} & \multicolumn{1}{r}{252}  & \cellcolor{TealBlue!30}{0} & 0.00 & \cellcolor{TealBlue!30}{0} & \cellcolor{TealBlue!30}{\textbf{0.00}} & 14 & 0.00\\
\texttt{vote} & \multicolumn{1}{r}{435} & \multicolumn{1}{r}{32}  & \cellcolor{TealBlue!30}{0} & 0.00 & \cellcolor{TealBlue!30}{0} & \cellcolor{TealBlue!30}{\textbf{0.00}} & 3 & 0.00\\
\texttt{wine1-un} & \multicolumn{1}{r}{178} & \multicolumn{1}{r}{1276}  & 30 & 0.00 & 25 & 0.00 & \cellcolor{TealBlue!30}{\textbf{24}} & \cellcolor{TealBlue!30}{\textbf{0.00}}\\
\texttt{wine2-un} & \multicolumn{1}{r}{178} & \multicolumn{1}{r}{1276}  & 29 & 0.00 & 29 & 0.00 & \cellcolor{TealBlue!30}{\textbf{23}} & \cellcolor{TealBlue!30}{\textbf{0.00}}\\
\texttt{wine3-un} & \multicolumn{1}{r}{178} & \multicolumn{1}{r}{1276}  & \cellcolor{TealBlue!30}{\textbf{15}} & \cellcolor{TealBlue!30}{\textbf{0.00}} & 20 & 0.00 & 19 & 0.00\\
\texttt{yeast} & \multicolumn{1}{r}{1484} & \multicolumn{1}{r}{89}  & 192 & 0.00 & \cellcolor{TealBlue!30}{\textbf{180}} & 0.00 & 274 & \cellcolor{TealBlue!30}{\textbf{0.00}}\\
\texttt{zoo-1} & \multicolumn{1}{r}{101} & \multicolumn{1}{r}{20}  & \cellcolor{TealBlue!30}{0} & 0.00 & \cellcolor{TealBlue!30}{0} & \cellcolor{TealBlue!30}{\textbf{0.00}} & \cellcolor{TealBlue!30}{0} & 0.00\\
\bottomrule
\end{tabular}

% \end{normalsize}
% \end{center}
% \caption{\label{tab:ha10} Comparison of heuristics (max depth=10)}
% \end{table}


% \begin{table}[htbp]
% \begin{center}
% \begin{footnotesize}
% \tabcolsep=5pt
% \input{src/tables/depth5b.tex}
% \end{footnotesize}
% \end{center}
% \caption{\label{tab:thetable} Restarts (max depth=5)}
% \end{table}

% \clearpage

% \begin{table}[htbp]
% \begin{center}
% \begin{footnotesize}
% \tabcolsep=5pt
% \begin{tabular}{lccrrrrrrrrrrrr}
\toprule
& && \multicolumn{6}{c}{dt no restart} & \multicolumn{6}{c}{dt restarts (1.1)}\\
\cmidrule(rr){4-9}\cmidrule(rr){10-15}
&\multirow{1}{*}{$\#ex.$} & \multirow{1}{*}{\#feat.} &  \multicolumn{1}{c}{opt} & \multicolumn{1}{c}{error} & \multicolumn{1}{c}{acc.} & \multicolumn{1}{c}{size} & \multicolumn{1}{c}{time} & \multicolumn{1}{c}{choices} & \multicolumn{1}{c}{opt} & \multicolumn{1}{c}{error} & \multicolumn{1}{c}{acc.} & \multicolumn{1}{c}{size} & \multicolumn{1}{c}{time} & \multicolumn{1}{c}{choices} \\
\midrule

\texttt{anneal} & \multicolumn{1}{r}{812} & \multicolumn{1}{r}{88}  & \cellcolor{TealBlue!30}{0.0} & \cellcolor{TealBlue!30}{\textbf{64.0}} & \cellcolor{TealBlue!30}{\textbf{0.921}} & 12.9 & 1435.5 & 258{\sc m} & \cellcolor{TealBlue!30}{0.0} & 65.0 & 0.920 & \cellcolor{TealBlue!30}{\textbf{12.7}} & \cellcolor{TealBlue!30}{\textbf{826.5}} & \cellcolor{TealBlue!30}{\textbf{136{\sc m}}}\\
\texttt{audiology} & \multicolumn{1}{r}{216} & \multicolumn{1}{r}{145}  & \cellcolor{TealBlue!30}{0.0} & \cellcolor{TealBlue!30}{0.0} & \cellcolor{TealBlue!30}{1.000} & 9.2 & \cellcolor{TealBlue!30}{\textbf{33.6}} & \cellcolor{TealBlue!30}{\textbf{8518{\sc k}}} & \cellcolor{TealBlue!30}{0.0} & \cellcolor{TealBlue!30}{0.0} & \cellcolor{TealBlue!30}{1.000} & \cellcolor{TealBlue!30}{\textbf{9.0}} & 410.9 & 107{\sc m}\\
\texttt{australian-credit} & \multicolumn{1}{r}{653} & \multicolumn{1}{r}{124}  & \cellcolor{TealBlue!30}{0.0} & \cellcolor{TealBlue!30}{\textbf{0.5}} & \cellcolor{TealBlue!30}{\textbf{0.999}} & \cellcolor{TealBlue!30}{\textbf{12.4}} & 1457.1 & 307{\sc m} & \cellcolor{TealBlue!30}{0.0} & 3.1 & 0.995 & 16.5 & \cellcolor{TealBlue!30}{\textbf{711.8}} & \cellcolor{TealBlue!30}{\textbf{139{\sc m}}}\\
\texttt{breast-cancer} & \multicolumn{1}{r}{683} & \multicolumn{1}{r}{89}  & \cellcolor{TealBlue!30}{0.0} & \cellcolor{TealBlue!30}{0.0} & \cellcolor{TealBlue!30}{1.000} & \cellcolor{TealBlue!30}{\textbf{13.1}} & 1084.6 & 330{\sc m} & \cellcolor{TealBlue!30}{0.0} & \cellcolor{TealBlue!30}{0.0} & \cellcolor{TealBlue!30}{1.000} & 15.5 & \cellcolor{TealBlue!30}{\textbf{489.4}} & \cellcolor{TealBlue!30}{\textbf{161{\sc m}}}\\
\texttt{car} & \multicolumn{1}{r}{1728} & \multicolumn{1}{r}{21}  & \cellcolor{TealBlue!30}{0.0} & \cellcolor{TealBlue!30}{\textbf{0.0}} & \cellcolor{TealBlue!30}{\textbf{1.000}} & \cellcolor{TealBlue!30}{\textbf{11.9}} & 934.2 & 542{\sc m} & \cellcolor{TealBlue!30}{0.0} & 2.5 & 0.999 & 13.9 & \cellcolor{TealBlue!30}{\textbf{343.9}} & \cellcolor{TealBlue!30}{\textbf{212{\sc m}}}\\
\texttt{forest-fires} & \multicolumn{1}{r}{517} & \multicolumn{1}{r}{989}  & \cellcolor{TealBlue!30}{0.0} & 145.7 & 0.718 & 26.3 & 742.3 & 38{\sc m} & \cellcolor{TealBlue!30}{0.0} & \cellcolor{TealBlue!30}{\textbf{127.7}} & \cellcolor{TealBlue!30}{\textbf{0.753}} & \cellcolor{TealBlue!30}{\textbf{25.7}} & \cellcolor{TealBlue!30}{\textbf{596.9}} & \cellcolor{TealBlue!30}{\textbf{30{\sc m}}}\\
\texttt{heart-cleveland} & \multicolumn{1}{r}{296} & \multicolumn{1}{r}{95}  & \cellcolor{TealBlue!30}{0.0} & \cellcolor{TealBlue!30}{0.0} & \cellcolor{TealBlue!30}{1.000} & \cellcolor{TealBlue!30}{\textbf{18.3}} & 1489.8 & 416{\sc m} & \cellcolor{TealBlue!30}{0.0} & \cellcolor{TealBlue!30}{0.0} & \cellcolor{TealBlue!30}{1.000} & 25.9 & \cellcolor{TealBlue!30}{\textbf{622.2}} & \cellcolor{TealBlue!30}{\textbf{211{\sc m}}}\\
\texttt{hypothyroid} & \multicolumn{1}{r}{3247} & \multicolumn{1}{r}{83}  & \cellcolor{TealBlue!30}{0.0} & 49.0 & 0.985 & \cellcolor{TealBlue!30}{\textbf{36.5}} & \cellcolor{TealBlue!30}{\textbf{91.5}} & \cellcolor{TealBlue!30}{\textbf{22{\sc m}}} & \cellcolor{TealBlue!30}{0.0} & \cellcolor{TealBlue!30}{\textbf{38.1}} & \cellcolor{TealBlue!30}{\textbf{0.988}} & 41.8 & 259.4 & 23{\sc m}\\
\texttt{kr-vs-kp} & \multicolumn{1}{r}{3196} & \multicolumn{1}{r}{73}  & \cellcolor{TealBlue!30}{0.0} & \cellcolor{TealBlue!30}{\textbf{29.7}} & \cellcolor{TealBlue!30}{\textbf{0.991}} & 17.8 & \cellcolor{TealBlue!30}{\textbf{237.2}} & \cellcolor{TealBlue!30}{\textbf{47{\sc m}}} & \cellcolor{TealBlue!30}{0.0} & 45.9 & 0.986 & \cellcolor{TealBlue!30}{\textbf{17.2}} & 1017.8 & 218{\sc m}\\
\texttt{lymph} & \multicolumn{1}{r}{148} & \multicolumn{1}{r}{68}  & \cellcolor{TealBlue!30}{0.0} & \cellcolor{TealBlue!30}{0.0} & \cellcolor{TealBlue!30}{1.000} & 11.6 & 1526.6 & 640{\sc m} & \cellcolor{TealBlue!30}{0.0} & \cellcolor{TealBlue!30}{0.0} & \cellcolor{TealBlue!30}{1.000} & \cellcolor{TealBlue!30}{\textbf{11.3}} & \cellcolor{TealBlue!30}{\textbf{517.0}} & \cellcolor{TealBlue!30}{\textbf{225{\sc m}}}\\
\texttt{mushroom} & \multicolumn{1}{r}{8124} & \multicolumn{1}{r}{111}  & \cellcolor{TealBlue!30}{0.0} & \cellcolor{TealBlue!30}{0.0} & \cellcolor{TealBlue!30}{1.000} & 8.1 & \cellcolor{TealBlue!30}{\textbf{293.2}} & \cellcolor{TealBlue!30}{\textbf{9653{\sc k}}} & \cellcolor{TealBlue!30}{0.0} & \cellcolor{TealBlue!30}{0.0} & \cellcolor{TealBlue!30}{1.000} & \cellcolor{TealBlue!30}{\textbf{8.0}} & 394.5 & 25{\sc m}\\
\texttt{primary-tumor} & \multicolumn{1}{r}{336} & \multicolumn{1}{r}{31}  & \cellcolor{TealBlue!30}{0.0} & \cellcolor{TealBlue!30}{15.0} & \cellcolor{TealBlue!30}{0.955} & 17.3 & 929.2 & 599{\sc m} & \cellcolor{TealBlue!30}{0.0} & \cellcolor{TealBlue!30}{15.0} & \cellcolor{TealBlue!30}{0.955} & \cellcolor{TealBlue!30}{\textbf{16.6}} & \cellcolor{TealBlue!30}{\textbf{738.1}} & \cellcolor{TealBlue!30}{\textbf{481{\sc m}}}\\
\texttt{soybean} & \multicolumn{1}{r}{630} & \multicolumn{1}{r}{50}  & \cellcolor{TealBlue!30}{0.0} & \cellcolor{TealBlue!30}{2.0} & \cellcolor{TealBlue!30}{0.997} & \cellcolor{TealBlue!30}{\textbf{11.1}} & 879.0 & \cellcolor{TealBlue!30}{\textbf{329{\sc m}}} & \cellcolor{TealBlue!30}{0.0} & \cellcolor{TealBlue!30}{2.0} & \cellcolor{TealBlue!30}{0.997} & 13.6 & \cellcolor{TealBlue!30}{\textbf{848.1}} & 329{\sc m}\\
\texttt{splice-1} & \multicolumn{1}{r}{3190} & \multicolumn{1}{r}{287}  & \cellcolor{TealBlue!30}{0.0} & 85.6 & 0.973 & \cellcolor{TealBlue!30}{\textbf{42.6}} & 1099.9 & 110{\sc m} & \cellcolor{TealBlue!30}{0.0} & \cellcolor{TealBlue!30}{\textbf{45.6}} & \cellcolor{TealBlue!30}{\textbf{0.986}} & 43.7 & \cellcolor{TealBlue!30}{\textbf{516.2}} & \cellcolor{TealBlue!30}{\textbf{48{\sc m}}}\\
\texttt{tic-tac-toe} & \multicolumn{1}{r}{958} & \multicolumn{1}{r}{27}  & \cellcolor{TealBlue!30}{0.0} & \cellcolor{TealBlue!30}{0.0} & \cellcolor{TealBlue!30}{1.000} & \cellcolor{TealBlue!30}{\textbf{19.1}} & \cellcolor{TealBlue!30}{\textbf{306.1}} & \cellcolor{TealBlue!30}{\textbf{219{\sc m}}} & \cellcolor{TealBlue!30}{0.0} & \cellcolor{TealBlue!30}{0.0} & \cellcolor{TealBlue!30}{1.000} & 25.4 & 329.7 & 253{\sc m}\\
\texttt{vote} & \multicolumn{1}{r}{435} & \multicolumn{1}{r}{48}  & \cellcolor{TealBlue!30}{0.0} & \cellcolor{TealBlue!30}{0.0} & \cellcolor{TealBlue!30}{1.000} & \cellcolor{TealBlue!30}{\textbf{12.9}} & \cellcolor{TealBlue!30}{\textbf{330.8}} & \cellcolor{TealBlue!30}{\textbf{202{\sc m}}} & \cellcolor{TealBlue!30}{0.0} & \cellcolor{TealBlue!30}{0.0} & \cellcolor{TealBlue!30}{1.000} & 14.2 & 574.8 & 376{\sc m}\\
\texttt{wine1} & \multicolumn{1}{r}{178} & \multicolumn{1}{r}{1276}  & \cellcolor{TealBlue!30}{0.0} & \cellcolor{TealBlue!30}{26.0} & \cellcolor{TealBlue!30}{0.854} & \cellcolor{TealBlue!30}{12.0} & \cellcolor{TealBlue!30}{\textbf{16.7}} & \cellcolor{TealBlue!30}{\textbf{695{\sc k}}} & \cellcolor{TealBlue!30}{0.0} & \cellcolor{TealBlue!30}{26.0} & \cellcolor{TealBlue!30}{0.854} & \cellcolor{TealBlue!30}{12.0} & 128.7 & 5508{\sc k}\\
\texttt{wine2} & \multicolumn{1}{r}{178} & \multicolumn{1}{r}{1276}  & \cellcolor{TealBlue!30}{0.0} & 27.3 & 0.847 & 13.6 & \cellcolor{TealBlue!30}{\textbf{259.2}} & \cellcolor{TealBlue!30}{\textbf{11{\sc m}}} & \cellcolor{TealBlue!30}{0.0} & \cellcolor{TealBlue!30}{\textbf{26.7}} & \cellcolor{TealBlue!30}{\textbf{0.850}} & \cellcolor{TealBlue!30}{\textbf{13.3}} & 347.5 & 15{\sc m}\\
\texttt{wine3} & \multicolumn{1}{r}{178} & \multicolumn{1}{r}{1276}  & \cellcolor{TealBlue!30}{0.0} & 20.8 & 0.883 & \cellcolor{TealBlue!30}{11.2} & \cellcolor{TealBlue!30}{\textbf{72.4}} & \cellcolor{TealBlue!30}{\textbf{3180{\sc k}}} & \cellcolor{TealBlue!30}{0.0} & \cellcolor{TealBlue!30}{\textbf{19.3}} & \cellcolor{TealBlue!30}{\textbf{0.892}} & \cellcolor{TealBlue!30}{11.2} & 631.9 & 28{\sc m}\\
\texttt{zoo-1} & \multicolumn{1}{r}{101} & \multicolumn{1}{r}{36}  & \cellcolor{TealBlue!30}{1.0} & \cellcolor{TealBlue!30}{0.0} & \cellcolor{TealBlue!30}{1.000} & \cellcolor{TealBlue!30}{1.0} & 0.0 & \cellcolor{TealBlue!30}{1} & \cellcolor{TealBlue!30}{1.0} & \cellcolor{TealBlue!30}{0.0} & \cellcolor{TealBlue!30}{1.000} & \cellcolor{TealBlue!30}{1.0} & \cellcolor{TealBlue!30}{\textbf{0.0}} & \cellcolor{TealBlue!30}{1}\\
\bottomrule
\end{tabular}

% \end{footnotesize}
% \end{center}
% \caption{\label{tab:thetable} Restarts (max depth=8)}
% \end{table}

% \begin{table}[htbp]
% \begin{center}
% \begin{footnotesize}
% \tabcolsep=5pt
% \begin{tabular}{lccrrrrrrrr}
\toprule
& && \multicolumn{4}{c}{\dleight} & \multicolumn{4}{c}{\budalg}\\
\cmidrule(rr){4-7}\cmidrule(rr){8-11}
&\multirow{1}{*}{$\#ex.$} & \multirow{1}{*}{\#feat.} &  \multicolumn{1}{c}{opt} & \multicolumn{1}{c}{error} & \multicolumn{1}{c}{acc.} & \multicolumn{1}{c}{time} & \multicolumn{1}{c}{opt} & \multicolumn{1}{c}{error} & \multicolumn{1}{c}{acc.} & \multicolumn{1}{c}{time} \\
\midrule

\texttt{anneal} & \multicolumn{1}{r}{812} & \multicolumn{1}{r}{47}  & - & - & - & - & \cellcolor{TealBlue!30}{\textbf{0}} & \cellcolor{TealBlue!30}{\textbf{53}} & \cellcolor{TealBlue!30}{\textbf{0.935}} & \cellcolor{TealBlue!30}{\textbf{310.0}}\\
\texttt{audiology} & \multicolumn{1}{r}{216} & \multicolumn{1}{r}{79}  & - & - & - & - & \cellcolor{TealBlue!30}{\textbf{1}} & \cellcolor{TealBlue!30}{\textbf{0}} & \cellcolor{TealBlue!30}{\textbf{1.000}} & \cellcolor{TealBlue!30}{\textbf{0.0}}\\
\texttt{australian-credit} & \multicolumn{1}{r}{653} & \multicolumn{1}{r}{73}  & - & - & - & - & \cellcolor{TealBlue!30}{\textbf{1}} & \cellcolor{TealBlue!30}{\textbf{0}} & \cellcolor{TealBlue!30}{\textbf{1.000}} & \cellcolor{TealBlue!30}{\textbf{0.0}}\\
\texttt{breast-cancer-un} & \multicolumn{1}{r}{683} & \multicolumn{1}{r}{89}  & - & - & - & - & \cellcolor{TealBlue!30}{\textbf{1}} & \cellcolor{TealBlue!30}{\textbf{0}} & \cellcolor{TealBlue!30}{\textbf{1.000}} & \cellcolor{TealBlue!30}{\textbf{0.0}}\\
\texttt{breast-wisconsin} & \multicolumn{1}{r}{683} & \multicolumn{1}{r}{120}  & - & - & - & - & \cellcolor{TealBlue!30}{\textbf{1}} & \cellcolor{TealBlue!30}{\textbf{0}} & \cellcolor{TealBlue!30}{\textbf{1.000}} & \cellcolor{TealBlue!30}{\textbf{0.0}}\\
\texttt{car-un} & \multicolumn{1}{r}{1728} & \multicolumn{1}{r}{21}  & - & - & - & - & \cellcolor{TealBlue!30}{\textbf{1}} & \cellcolor{TealBlue!30}{\textbf{0}} & \cellcolor{TealBlue!30}{\textbf{1.000}} & \cellcolor{TealBlue!30}{\textbf{0.2}}\\
\texttt{diabetes} & \multicolumn{1}{r}{768} & \multicolumn{1}{r}{112}  & - & - & - & - & \cellcolor{TealBlue!30}{\textbf{1}} & \cellcolor{TealBlue!30}{\textbf{0}} & \cellcolor{TealBlue!30}{\textbf{1.000}} & \cellcolor{TealBlue!30}{\textbf{0.5}}\\
\texttt{forest-fires-un} & \multicolumn{1}{r}{517} & \multicolumn{1}{r}{989}  & - & - & - & - & \cellcolor{TealBlue!30}{\textbf{0}} & \cellcolor{TealBlue!30}{\textbf{118}} & \cellcolor{TealBlue!30}{\textbf{0.772}} & \cellcolor{TealBlue!30}{\textbf{830.0}}\\
\texttt{german-credit} & \multicolumn{1}{r}{1000} & \multicolumn{1}{r}{110}  & - & - & - & - & \cellcolor{TealBlue!30}{\textbf{1}} & \cellcolor{TealBlue!30}{\textbf{0}} & \cellcolor{TealBlue!30}{\textbf{1.000}} & \cellcolor{TealBlue!30}{\textbf{67.0}}\\
\texttt{heart-cleveland} & \multicolumn{1}{r}{296} & \multicolumn{1}{r}{50}  & - & - & - & - & \cellcolor{TealBlue!30}{\textbf{1}} & \cellcolor{TealBlue!30}{\textbf{0}} & \cellcolor{TealBlue!30}{\textbf{1.000}} & \cellcolor{TealBlue!30}{\textbf{0.0}}\\
\texttt{hepatitis} & \multicolumn{1}{r}{137} & \multicolumn{1}{r}{68}  & - & - & - & - & \cellcolor{TealBlue!30}{\textbf{1}} & \cellcolor{TealBlue!30}{\textbf{0}} & \cellcolor{TealBlue!30}{\textbf{1.000}} & \cellcolor{TealBlue!30}{\textbf{0.0}}\\
\texttt{hypothyroid} & \multicolumn{1}{r}{3247} & \multicolumn{1}{r}{43}  & - & - & - & - & \cellcolor{TealBlue!30}{\textbf{0}} & \cellcolor{TealBlue!30}{\textbf{31}} & \cellcolor{TealBlue!30}{\textbf{0.990}} & \cellcolor{TealBlue!30}{\textbf{1.4}}\\
\texttt{ionosphere} & \multicolumn{1}{r}{351} & \multicolumn{1}{r}{444}  & - & - & - & - & \cellcolor{TealBlue!30}{\textbf{1}} & \cellcolor{TealBlue!30}{\textbf{0}} & \cellcolor{TealBlue!30}{\textbf{1.000}} & \cellcolor{TealBlue!30}{\textbf{0.0}}\\
\texttt{kr-vs-kp} & \multicolumn{1}{r}{3196} & \multicolumn{1}{r}{37}  & - & - & - & - & \cellcolor{TealBlue!30}{\textbf{0}} & \cellcolor{TealBlue!30}{\textbf{1}} & \cellcolor{TealBlue!30}{\textbf{1.000}} & \cellcolor{TealBlue!30}{\textbf{224.0}}\\
\texttt{letter} & \multicolumn{1}{r}{20000} & \multicolumn{1}{r}{224}  & - & - & - & - & \cellcolor{TealBlue!30}{\textbf{1}} & \cellcolor{TealBlue!30}{\textbf{0}} & \cellcolor{TealBlue!30}{\textbf{1.000}} & \cellcolor{TealBlue!30}{\textbf{58.4}}\\
\texttt{lymph} & \multicolumn{1}{r}{148} & \multicolumn{1}{r}{41}  & - & - & - & - & \cellcolor{TealBlue!30}{\textbf{1}} & \cellcolor{TealBlue!30}{\textbf{0}} & \cellcolor{TealBlue!30}{\textbf{1.000}} & \cellcolor{TealBlue!30}{\textbf{0.0}}\\
\texttt{mushroom} & \multicolumn{1}{r}{8124} & \multicolumn{1}{r}{91}  & - & - & - & - & \cellcolor{TealBlue!30}{\textbf{1}} & \cellcolor{TealBlue!30}{\textbf{0}} & \cellcolor{TealBlue!30}{\textbf{1.000}} & \cellcolor{TealBlue!30}{\textbf{0.0}}\\
\texttt{pendigits} & \multicolumn{1}{r}{7494} & \multicolumn{1}{r}{216}  & - & - & - & - & \cellcolor{TealBlue!30}{\textbf{1}} & \cellcolor{TealBlue!30}{\textbf{0}} & \cellcolor{TealBlue!30}{\textbf{1.000}} & \cellcolor{TealBlue!30}{\textbf{0.0}}\\
\texttt{primary-tumor} & \multicolumn{1}{r}{336} & \multicolumn{1}{r}{16}  & - & - & - & - & \cellcolor{TealBlue!30}{\textbf{0}} & \cellcolor{TealBlue!30}{\textbf{15}} & \cellcolor{TealBlue!30}{\textbf{0.955}} & \cellcolor{TealBlue!30}{\textbf{144.0}}\\
\texttt{segment} & \multicolumn{1}{r}{2310} & \multicolumn{1}{r}{234}  & - & - & - & - & \cellcolor{TealBlue!30}{\textbf{1}} & \cellcolor{TealBlue!30}{\textbf{0}} & \cellcolor{TealBlue!30}{\textbf{1.000}} & \cellcolor{TealBlue!30}{\textbf{0.0}}\\
\texttt{soybean} & \multicolumn{1}{r}{630} & \multicolumn{1}{r}{34}  & - & - & - & - & \cellcolor{TealBlue!30}{\textbf{0}} & \cellcolor{TealBlue!30}{\textbf{2}} & \cellcolor{TealBlue!30}{\textbf{0.997}} & \cellcolor{TealBlue!30}{\textbf{149.0}}\\
\texttt{splice-1} & \multicolumn{1}{r}{3190} & \multicolumn{1}{r}{227}  & - & - & - & - & \cellcolor{TealBlue!30}{\textbf{0}} & \cellcolor{TealBlue!30}{\textbf{7}} & \cellcolor{TealBlue!30}{\textbf{0.998}} & \cellcolor{TealBlue!30}{\textbf{56.1}}\\
\texttt{taiwan\_binarised} & \multicolumn{1}{r}{30000} & \multicolumn{1}{r}{198}  & - & - & - & - & \cellcolor{TealBlue!30}{\textbf{0}} & \cellcolor{TealBlue!30}{\textbf{4564}} & \cellcolor{TealBlue!30}{\textbf{0.848}} & \cellcolor{TealBlue!30}{\textbf{200.0}}\\
\texttt{tic-tac-toe} & \multicolumn{1}{r}{958} & \multicolumn{1}{r}{18}  & - & - & - & - & \cellcolor{TealBlue!30}{\textbf{1}} & \cellcolor{TealBlue!30}{\textbf{0}} & \cellcolor{TealBlue!30}{\textbf{1.000}} & \cellcolor{TealBlue!30}{\textbf{0.0}}\\
\texttt{vehicle} & \multicolumn{1}{r}{846} & \multicolumn{1}{r}{252}  & - & - & - & - & \cellcolor{TealBlue!30}{\textbf{1}} & \cellcolor{TealBlue!30}{\textbf{0}} & \cellcolor{TealBlue!30}{\textbf{1.000}} & \cellcolor{TealBlue!30}{\textbf{0.0}}\\
\texttt{vote} & \multicolumn{1}{r}{435} & \multicolumn{1}{r}{32}  & - & - & - & - & \cellcolor{TealBlue!30}{\textbf{1}} & \cellcolor{TealBlue!30}{\textbf{0}} & \cellcolor{TealBlue!30}{\textbf{1.000}} & \cellcolor{TealBlue!30}{\textbf{0.0}}\\
\texttt{wine1-un} & \multicolumn{1}{r}{178} & \multicolumn{1}{r}{1276}  & - & - & - & - & \cellcolor{TealBlue!30}{\textbf{0}} & \cellcolor{TealBlue!30}{\textbf{22}} & \cellcolor{TealBlue!30}{\textbf{0.876}} & \cellcolor{TealBlue!30}{\textbf{515.0}}\\
\texttt{wine2-un} & \multicolumn{1}{r}{178} & \multicolumn{1}{r}{1276}  & - & - & - & - & \cellcolor{TealBlue!30}{\textbf{0}} & \cellcolor{TealBlue!30}{\textbf{24}} & \cellcolor{TealBlue!30}{\textbf{0.865}} & \cellcolor{TealBlue!30}{\textbf{340.0}}\\
\texttt{wine3-un} & \multicolumn{1}{r}{178} & \multicolumn{1}{r}{1276}  & - & - & - & - & \cellcolor{TealBlue!30}{\textbf{0}} & \cellcolor{TealBlue!30}{\textbf{16}} & \cellcolor{TealBlue!30}{\textbf{0.910}} & \cellcolor{TealBlue!30}{\textbf{260.0}}\\
\texttt{yeast} & \multicolumn{1}{r}{1484} & \multicolumn{1}{r}{89}  & - & - & - & - & \cellcolor{TealBlue!30}{\textbf{0}} & \cellcolor{TealBlue!30}{\textbf{104}} & \cellcolor{TealBlue!30}{\textbf{0.930}} & \cellcolor{TealBlue!30}{\textbf{72.3}}\\
\texttt{zoo-1} & \multicolumn{1}{r}{101} & \multicolumn{1}{r}{20}  & - & - & - & - & \cellcolor{TealBlue!30}{\textbf{1}} & \cellcolor{TealBlue!30}{\textbf{0}} & \cellcolor{TealBlue!30}{\textbf{1.000}} & \cellcolor{TealBlue!30}{\textbf{0.0}}\\
\bottomrule
\end{tabular}

% \end{footnotesize}
% \end{center}
% \caption{\label{tab:thetable} Restarts (max depth=10)}
% \end{table}



\end{document}

