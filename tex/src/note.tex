\documentclass{article}
\usepackage[usenames,dvipsnames,svgnames,table]{xcolor}%% http://ctan.org/pkg/xcolor
\usepackage[utf8]{inputenc}
\usepackage{xspace}
\usepackage{array}
%\usepackage{amsthm}
\usepackage{amsmath} 
\usepackage{amssymb} 
\usepackage[ruled,vlined]{algorithm2e}
\usepackage{booktabs}
\usepackage{multirow}
\usepackage{url}
\usepackage{tikz}
\usepackage{fp}
\usepackage{subfig}
\usetikzlibrary{arrows,shadows,fit,calc,positioning,decorations.pathreplacing,matrix,shapes,petri,topaths,fadings,mindmap,backgrounds,shapes.geometric}
\usepackage{geometry}



\newcommand{\setex}[1]{\ensuremath{{\mathcal X}^{#1}}\xspace}
\newcommand{\posex}{{\setex{1}}\xspace}
\newcommand{\negex}{{\setex{0}}\xspace}
\newcommand{\setcube}[1]{\ensuremath{{\mathcal C}^{#1}}\xspace}
\newcommand{\poscube}{{\setcube{1}}\xspace}
\newcommand{\negcube}{{\setcube{0}}\xspace}
\newcommand{\features}{\ensuremath{{\mathcal F}}\xspace}
\newcommand{\classifier}{\ensuremath{f}}
\newcommand{\lit}[1]{\ensuremath{l_{#1}}}
\newcommand{\var}{\ensuremath{x}}
\newcommand{\truelit}[1]{\ensuremath{\var_{#1}}}
\newcommand{\falselit}[1]{\ensuremath{\bar{\var_{#1}}}}
\newcommand{\ex}{\ensuremath{\var}}
\newcommand{\cube}{\ensuremath{c}}
\newcommand{\universe}{\ensuremath{{\mathcal U}}}



\title{Reducing Data Sets by Computing Explanations}

% \author{Emmanuel Hebrard\inst{1} \and George Katsirelos\inst{2}}
% \institute{LAAS-CNRS, Universit\'e de Toulouse, CNRS, France, email: hebrard@laas.fr
  % \and MIAT, UR-875, INRA, France, email: gkatsi@gmail.com \footnote{The second author was partially supported by the french ``Agence nationale de
% la Recherche'', project DEMOGRAPH, reference ANR-16-C40-0028.}}

\begin{document}
	\newgeometry{bottom=3cm,top=3cm}

\maketitle

\section*{Problem statement}

We consider a data set composed a set \posex of positive examples and a set \negex of negative examples over a set of binary features $\features$.
%$ = \{1,\ldots,m\}$.
The classic task in machine learning is to compute a function $\classifier : 2^{\features} \mapsto \{0,1\}$ that \emph{generalizes} the data set.

In this note we show how we can reduce the data sets by representing \posex and \negex as unions of \emph{cubes} (or DNFs) that we will denote respectively \poscube and \negcube, respectively. Let $\truelit{i}$ be a literal standing for ``feature $i$ is true'' and its negation $\falselit{i}$ standing for ``feature $i$ is false''. An instance $\ex$ is a conjunction $\bigwedge_{i \in \features}\lit{i}$ where $\lit{i}$ is either $\truelit{i}$ or $\falselit{i}$, an example is an instance $\ex \in \posex \cup \negex$.
A cube $\cube \in \poscube \cup \negcube$ is a conjunction $\bigwedge_{i \in {\mathcal S}}\lit{i}$ where $\lit{i}$ is either $\truelit{i}$ or $\falselit{i}$, for a subset ${\mathcal S}$ of \features.
The cubes \poscube and \negcube have the following properties:
\begin{eqnarray}
	\forall \ex \in \posex \exists \cube \in \poscube & \ex \vdash \cube \label{poscomplete} \\
	\forall \ex \in \negex \exists \cube \in \negcube & \ex \vdash \cube \label{negcomplete} \\
	\forall \cube \in \poscube  \forall \cube' \in \negcube  & \cube \wedge \cube' \vdash \perp \label{consistent}
\end{eqnarray}

Properties (\ref{poscomplete}) and (\ref{negcomplete}) state that \poscube and \negcube are complete, i.e., they entail every positive and negative example, respectively. Property (\ref{consistent}) states that \poscube and \negcube are consistent, i.e., there is no instance $\ex$ such that $\poscube \vdash \ex$ and $\negcube \vdash \ex$.

 The reduced data set $(\poscube,\negcube)$ can be seen as a classifier $\classifier : 2^{\features} \mapsto \{0,1,*\}$ with a ``do not know'' value $*$ since it generalizes the original data set to some extent:
 $$
 \classifier(\ex) = \begin{cases} 1 \textrm{~if~} \poscube \vdash \ex\\ 0 \textrm{~if~} \negcube \vdash \ex\\ * \textrm{~otherwise} \end{cases}
 $$
 
 Moreover, standard machine learning techniques can be applied almost directly on such a reduced data set, thus improving their scalability (without degrading their accuracy?): usually the irrelevant features can be handled, and if not, it is always possible to generate full examples by random sampling within a cube.
 
 
 \section*{Algorithm}
 
 
 We represent instances and cubes as subsets of
 $\universe = \{\truelit{i} \mid i\in \features \} \cup \{\falselit{i} \mid i\in \features \}$.
% $\bar{\features} = \{i+|\features| : i \in \features\}$.
 %That is, the set $\ex \subseteq U$ is interpreted as the conjunction $\biwedge_{i \in \ex}$
 Notice that $\truelit{i} \in \ex$ and $\falselit{i} \in \ex$ are mutually exclusive, but $\truelit{i} \not\in \ex$ and $\falselit{i} \not\in \ex$ are not.
 Given a set $\cube \in \universe$ we denote $\bar{\cube}$ the set $\{\truelit{i} \mid \falselit{i} \in \cube\} \cup \{\falselit{i} \mid \truelit{i} \in \cube\}$.
 
 These sets are implemented as bitsets over $2|\features|$ bits, with the first $|\features|$ bits standing for positive literals and the second $|\features|$ bits standing for negative literals. Notice that computing $\bar{\cube}$ can be done by swaping the two halves of the bitset standing for $\cube$. Taking the complement would work for examples (and instances), but not for cubes.
 
 
 \begin{algorithm}
	\SetKwFunction{explain}{Comprime}
   \TitleOfAlgo{\explain} 
   \KwData{A data set $(\posex,\negex)$}
   \KwResult{An explanation set $(\poscube, \negcube)$}
	 $(\poscube, \negcube) \gets (\emptyset, \emptyset)$\;
	 \While{$\posex \cup \negex \neq \emptyset$}
	 {
			\lnl{l:classselect} choose $c \in \{y \mid \setex{y} \neq \emptyset\}$\;
			\lnl{l:exselect} choose $\ex^y \in \setex{y}$\;
			$implicant \gets \emptyset$\;
			$candidates \gets \ex^y$\;
			\ForEach{$\ex \in \setcube{1-y} \cup \setex{1-y}$}
			{
				\If{$implicant \cap \bar{\ex} = \emptyset$}
				{
					$\delta \gets \ex^y \cap \bar{\ex}$\;
					\If{$candidates \cap \delta = \emptyset$} {
						choose $\lit{i} \in candidates$\;
						$implicant \gets implicant \cup \{\lit{i}\}$\;
						$candidates \gets \ex^y$\;
					}
					$candidates \gets candidates \cap \delta$\;
				}
			}
			choose $\lit{i} \in candidates$\;
			$implicant \gets implicant \cup \{\lit{i}\}$\;
			$\setcube{y} \gets \setcube{y} \cup \{implicant\}$\;
			$\setex{y} \gets \setex{y} \setminus \{\ex \mid \ex \in \setex{y}, implicant \subseteq \ex\}$
	 }
	 \Return{$(\poscube, \negcube)$}\;
	 
\end{algorithm}


\section*{Experimental results}

We ran \explain on some data sets, whose number of examples and (binary) features are given respectively in the second and third columns of the following tables. We report the total \emph{size} of the set of cubes, i.e., the sum of the number of literals in all the cubes in Table~\ref{tab:vol}; the number of cubes in Table~\ref{tab:exa}; and the average number of literals per cube in Table~\ref{tab:fea};

We tested several variants with five strategies for selecting the next class for which we compute a cube (at Line~\ref{l:classselect}):
\begin{itemize}
	\item \texttt{positive} always selects examples from $\posex$ first
		\item \texttt{negative} always selects examples from $\negex$ first
	\item \texttt{altern} always selects from the $\setex{1-c}$ where $\setex{c}$ is the previous choice
\item \texttt{uniform} selects from $\posex$ with probability $1/2$ and from $\negex$ otherwise
\item \texttt{biased} selects from $\posex$ with probability $|\posex|/(|\posex|+|\negex|)$ and from $\negex$ otherwise
\end{itemize}
and two strategies for selecting the example $\ex^y$ to minimize in $\setex{y}$ (at Line~\ref{l:exselect}):
\begin{itemize}
	\item \texttt{first} selects examples in input order
	\item \texttt{random} selects any example randomly with uniform probability
\end{itemize}

\newgeometry{left=.1cm,bottom=.1cm,top=.1cm}

\begin{center}

\begin{table}[h!]
\begin{center}
\begin{scriptsize}
\tabcolsep=2.1pt
\begin{tabular}{lccrrrrrrrrrrrrrrrrrr}
\toprule
& && \multicolumn{2}{c}{max(H)-max(P)} & \multicolumn{2}{c}{max(H)-min(P)} & \multicolumn{2}{c}{max(H)-rand} & \multicolumn{2}{c}{min(H)-max(P)} & \multicolumn{2}{c}{min(H)-min(P)} & \multicolumn{2}{c}{min(H)-rand} & \multicolumn{2}{c}{min-max(P)} & \multicolumn{2}{c}{min-min(P)} & \multicolumn{2}{c}{min-rand}\\
\cmidrule(rr){4-5}\cmidrule(rr){6-7}\cmidrule(rr){8-9}\cmidrule(rr){10-11}\cmidrule(rr){12-13}\cmidrule(rr){14-15}\cmidrule(rr){16-17}\cmidrule(rr){18-19}\cmidrule(rr){20-21}
&\multirow{2}{*}{$\#ex.$} & \multirow{2}{*}{\#feat.} &  \multicolumn{2}{c}{size} & \multicolumn{2}{c}{size} & \multicolumn{2}{c}{size} & \multicolumn{2}{c}{size} & \multicolumn{2}{c}{size} & \multicolumn{2}{c}{size} & \multicolumn{2}{c}{size} & \multicolumn{2}{c}{size} & \multicolumn{2}{c}{size} \\\cmidrule(rl){4-5}\cmidrule(rl){6-7}\cmidrule(rl){8-9}\cmidrule(rl){10-11}\cmidrule(rl){12-13}\cmidrule(rl){14-15}\cmidrule(rl){16-17}\cmidrule(rl){18-19}\cmidrule(rl){20-21}
&& & \multicolumn{1}{c}{min} & \multicolumn{1}{c}{avg} & \multicolumn{1}{c}{min} & \multicolumn{1}{c}{avg} & \multicolumn{1}{c}{min} & \multicolumn{1}{c}{avg} & \multicolumn{1}{c}{min} & \multicolumn{1}{c}{avg} & \multicolumn{1}{c}{min} & \multicolumn{1}{c}{avg} & \multicolumn{1}{c}{min} & \multicolumn{1}{c}{avg} & \multicolumn{1}{c}{min} & \multicolumn{1}{c}{avg} & \multicolumn{1}{c}{min} & \multicolumn{1}{c}{avg} & \multicolumn{1}{c}{min} & \multicolumn{1}{c}{avg} \\
\midrule

\texttt{anneal} & \multicolumn{1}{r}{712} & \multicolumn{1}{r}{88}  & 810 & 974 & 677 & 999 & 625 & 939 & 600 & 722 & 600 & 852 & 527 & 704 & 608 & 768 & 530 & \cellcolor{TealBlue!30}{\textbf{633}} & \cellcolor{TealBlue!30}{\textbf{482}} & 687\\
\texttt{audiology} & \multicolumn{1}{r}{216} & \multicolumn{1}{r}{145}  & 66 & 120 & 98 & 171 & 78 & 190 & \cellcolor{TealBlue!30}{\textbf{57}} & \cellcolor{TealBlue!30}{\textbf{78}} & 75 & 132 & 65 & 133 & 85 & 127 & 75 & 138 & 83 & 154\\
\texttt{australian-credit} & \multicolumn{1}{r}{653} & \multicolumn{1}{r}{124}  & 1509 & 1847 & 1804 & 2512 & 1675 & 2106 & 1604 & \cellcolor{TealBlue!30}{\textbf{1847}} & 1775 & 2347 & 1514 & 1890 & 1674 & 1926 & 1864 & 2712 & \cellcolor{TealBlue!30}{\textbf{1402}} & 2024\\
\texttt{breast-cancer} & \multicolumn{1}{r}{683} & \multicolumn{1}{r}{89}  & 432 & 597 & 523 & 846 & 460 & 668 & 470 & 623 & 482 & 788 & \cellcolor{TealBlue!30}{\textbf{360}} & 604 & 496 & \cellcolor{TealBlue!30}{\textbf{593}} & 624 & 945 & 474 & 656\\
\texttt{car} & \multicolumn{1}{r}{1728} & \multicolumn{1}{r}{21}  & 454 & 589 & 471 & 636 & 446 & 687 & 435 & 534 & 334 & \cellcolor{TealBlue!30}{\textbf{499}} & \cellcolor{TealBlue!30}{\textbf{326}} & 500 & 428 & 532 & 484 & 606 & 413 & 628\\
\texttt{forest-fires} & \multicolumn{1}{r}{509} & \multicolumn{1}{r}{989}  & 6662 & 7710 & 11611 & 18659 & 7558 & 10642 & 7080 & 8306 & 12181 & 14359 & 7655 & 10789 & \cellcolor{TealBlue!30}{\textbf{5064}} & \cellcolor{TealBlue!30}{\textbf{5780}} & 10514 & 12606 & 6166 & 8151\\
\texttt{heart-cleveland} & \multicolumn{1}{r}{296} & \multicolumn{1}{r}{95}  & 741 & 982 & 953 & 1283 & 764 & 1096 & \cellcolor{TealBlue!30}{\textbf{554}} & \cellcolor{TealBlue!30}{\textbf{689}} & 685 & 894 & 578 & 754 & 605 & 856 & 799 & 1114 & 691 & 914\\
\texttt{hypothyroid} & \multicolumn{1}{r}{3199} & \multicolumn{1}{r}{83}  & 810 & 938 & 984 & 1193 & 732 & 957 & \cellcolor{TealBlue!30}{\textbf{670}} & \cellcolor{TealBlue!30}{\textbf{781}} & 909 & 1149 & 680 & 812 & 770 & 879 & 1001 & 1210 & 762 & 917\\
\texttt{kr-vs-kp} & \multicolumn{1}{r}{3196} & \multicolumn{1}{r}{73}  & 640 & 905 & 1015 & 1775 & 622 & 1113 & \cellcolor{TealBlue!30}{\textbf{442}} & \cellcolor{TealBlue!30}{\textbf{655}} & 815 & 1109 & 487 & 760 & 517 & 722 & 1053 & 1605 & 518 & 1009\\
\texttt{lymph} & \multicolumn{1}{r}{148} & \multicolumn{1}{r}{68}  & 187 & 256 & 247 & 336 & 206 & 331 & 160 & \cellcolor{TealBlue!30}{\textbf{232}} & 259 & 378 & \cellcolor{TealBlue!30}{\textbf{154}} & 254 & 196 & 268 & 239 & 386 & 210 & 301\\
\texttt{mushroom} & \multicolumn{1}{r}{8124} & \multicolumn{1}{r}{111}  & 126 & 138 & 253 & 292 & 82 & 203 & 76 & 108 & 189 & 244 & \cellcolor{TealBlue!30}{58} & 188 & 63 & \cellcolor{TealBlue!30}{\textbf{78}} & 218 & 272 & \cellcolor{TealBlue!30}{58} & 208\\
\texttt{primary-tumor} & \multicolumn{1}{r}{283} & \multicolumn{1}{r}{31}  & \cellcolor{TealBlue!30}{\textbf{397}} & \cellcolor{TealBlue!30}{\textbf{451}} & 444 & 587 & 408 & 516 & 428 & 479 & 415 & 520 & 407 & 491 & 414 & 484 & 433 & 554 & 407 & 521\\
\texttt{soybean} & \multicolumn{1}{r}{625} & \multicolumn{1}{r}{50}  & 729 & 867 & 320 & 543 & 367 & 600 & 607 & 706 & \cellcolor{TealBlue!30}{\textbf{287}} & \cellcolor{TealBlue!30}{\textbf{397}} & 312 & 471 & 614 & 800 & 339 & 487 & 358 & 558\\
\texttt{splice-1} & \multicolumn{1}{r}{3188} & \multicolumn{1}{r}{287}  & 11333 & 23520 & 21732 & 34922 & 14865 & 23329 & \cellcolor{TealBlue!30}{\textbf{4637}} & 8383 & 6680 & 11619 & 5179 & \cellcolor{TealBlue!30}{\textbf{8381}} & 8908 & 18860 & 16018 & 23680 & 11022 & 18439\\
\texttt{tic-tac-toe} & \multicolumn{1}{r}{958} & \multicolumn{1}{r}{27}  & 1515 & 3023 & 1432 & 2933 & 979 & 2606 & 994 & 1450 & 630 & 1281 & \cellcolor{TealBlue!30}{\textbf{503}} & \cellcolor{TealBlue!30}{\textbf{1133}} & 1335 & 2537 & 704 & 1701 & 933 & 1716\\
\texttt{vote} & \multicolumn{1}{r}{435} & \multicolumn{1}{r}{48}  & 160 & 201 & 211 & 314 & 175 & 238 & 151 & \cellcolor{TealBlue!30}{\textbf{177}} & 188 & 241 & \cellcolor{TealBlue!30}{\textbf{136}} & 191 & 167 & 203 & 195 & 279 & 157 & 209\\
\texttt{wine1} & \multicolumn{1}{r}{178} & \multicolumn{1}{r}{1276}  & 2399 & \cellcolor{TealBlue!30}{2422} & 2599 & 2701 & \cellcolor{TealBlue!30}{2207} & 2588 & 2399 & \cellcolor{TealBlue!30}{2422} & 2599 & 2701 & \cellcolor{TealBlue!30}{2207} & 2588 & 2399 & \cellcolor{TealBlue!30}{2422} & 2599 & 2701 & \cellcolor{TealBlue!30}{2207} & 2588\\
\texttt{wine2} & \multicolumn{1}{r}{178} & \multicolumn{1}{r}{1276}  & 2548 & 2548 & 2448 & 2552 & \cellcolor{TealBlue!30}{2254} & \cellcolor{TealBlue!30}{2527} & 2548 & 2548 & 2448 & 2552 & \cellcolor{TealBlue!30}{2254} & \cellcolor{TealBlue!30}{2527} & 2548 & 2548 & 2448 & 2552 & \cellcolor{TealBlue!30}{2254} & \cellcolor{TealBlue!30}{2527}\\
\texttt{wine3} & \multicolumn{1}{r}{178} & \multicolumn{1}{r}{1276}  & 2160 & 2164 & 1978 & \cellcolor{TealBlue!30}{1980} & \cellcolor{TealBlue!30}{1890} & 2122 & 2160 & 2164 & 1978 & \cellcolor{TealBlue!30}{1980} & \cellcolor{TealBlue!30}{1890} & 2122 & 2160 & 2164 & 1978 & \cellcolor{TealBlue!30}{1980} & \cellcolor{TealBlue!30}{1890} & 2122\\
\texttt{zoo-1} & \multicolumn{1}{r}{101} & \multicolumn{1}{r}{36}  & \cellcolor{TealBlue!30}{2} & \cellcolor{TealBlue!30}{2} & \cellcolor{TealBlue!30}{2} & \cellcolor{TealBlue!30}{2} & \cellcolor{TealBlue!30}{2} & \cellcolor{TealBlue!30}{2} & \cellcolor{TealBlue!30}{2} & \cellcolor{TealBlue!30}{2} & \cellcolor{TealBlue!30}{2} & \cellcolor{TealBlue!30}{2} & \cellcolor{TealBlue!30}{2} & \cellcolor{TealBlue!30}{2} & \cellcolor{TealBlue!30}{2} & \cellcolor{TealBlue!30}{2} & \cellcolor{TealBlue!30}{2} & \cellcolor{TealBlue!30}{2} & \cellcolor{TealBlue!30}{2} & \cellcolor{TealBlue!30}{2}\\\midrule

\texttt{AVERAGE} & \multicolumn{1}{r}{-} & \multicolumn{1}{r}{-}  & 1684 & 2513 & 2490 & 3762 & 1820 & 2673 & 1304 & \cellcolor{TealBlue!30}{\textbf{1645}} & 1677 & 2202 & \cellcolor{TealBlue!30}{\textbf{1265}} & 1765 & 1453 & 2127 & 2106 & 2808 & 1524 & 2217\\
\bottomrule
\end{tabular}

\end{scriptsize}
\end{center}
\vspace{-.1cm}
\caption{\label{tab:vol} Total size (total number of literals)}
\end{table}

\vspace{-.1cm}

\begin{table}[h!]
\begin{center}
\begin{scriptsize}
\tabcolsep=2pt
\begin{tabular}{lccrrrrrrrrrrrrrrrrrr}
\toprule
& && \multicolumn{2}{c}{h(H)-h(P)} & \multicolumn{2}{c}{h(H)-l(P)} & \multicolumn{2}{c}{h(H)-rand} & \multicolumn{2}{c}{l(H)-h(P)} & \multicolumn{2}{c}{l(H)-l(P)} & \multicolumn{2}{c}{l(H)-rand} & \multicolumn{2}{c}{min-h(P)} & \multicolumn{2}{c}{min-l(P)} & \multicolumn{2}{c}{min-rand}\\
\cmidrule(rr){4-5}\cmidrule(rr){6-7}\cmidrule(rr){8-9}\cmidrule(rr){10-11}\cmidrule(rr){12-13}\cmidrule(rr){14-15}\cmidrule(rr){16-17}\cmidrule(rr){18-19}\cmidrule(rr){20-21}
&\multirow{2}{*}{$\#ex.$} & \multirow{2}{*}{\#feat.} &  \multicolumn{2}{c}{\#examples} & \multicolumn{2}{c}{\#examples} & \multicolumn{2}{c}{\#examples} & \multicolumn{2}{c}{\#examples} & \multicolumn{2}{c}{\#examples} & \multicolumn{2}{c}{\#examples} & \multicolumn{2}{c}{\#examples} & \multicolumn{2}{c}{\#examples} & \multicolumn{2}{c}{\#examples} \\\cmidrule(rl){4-5}\cmidrule(rl){6-7}\cmidrule(rl){8-9}\cmidrule(rl){10-11}\cmidrule(rl){12-13}\cmidrule(rl){14-15}\cmidrule(rl){16-17}\cmidrule(rl){18-19}\cmidrule(rl){20-21}
&& & \multicolumn{1}{c}{min} & \multicolumn{1}{c}{avg} & \multicolumn{1}{c}{min} & \multicolumn{1}{c}{avg} & \multicolumn{1}{c}{min} & \multicolumn{1}{c}{avg} & \multicolumn{1}{c}{min} & \multicolumn{1}{c}{avg} & \multicolumn{1}{c}{min} & \multicolumn{1}{c}{avg} & \multicolumn{1}{c}{min} & \multicolumn{1}{c}{avg} & \multicolumn{1}{c}{min} & \multicolumn{1}{c}{avg} & \multicolumn{1}{c}{min} & \multicolumn{1}{c}{avg} & \multicolumn{1}{c}{min} & \multicolumn{1}{c}{avg} \\
\midrule

\texttt{anneal} & \multicolumn{1}{r}{712} & \multicolumn{1}{r}{88}  & 109 & 128.2 & 118 & 132.9 & 99 & 130.5 & 83 & \cellcolor{TealBlue!30}{\textbf{95.5}} & 85 & 107.7 & 79 & 105.1 & 82 & 107.9 & \cellcolor{TealBlue!30}{\textbf{77}} & 97.1 & 78 & 103.6\\
\texttt{audiology} & \multicolumn{1}{r}{216} & \multicolumn{1}{r}{145}  & 17 & 29.4 & 61 & 71.0 & 22 & 38.1 & 14 & \cellcolor{TealBlue!30}{\textbf{20.5}} & 16 & 21.3 & \cellcolor{TealBlue!30}{\textbf{13}} & 22.2 & 19 & 24.7 & 15 & 25.3 & 18 & 27.1\\
\texttt{australian-credit} & \multicolumn{1}{r}{653} & \multicolumn{1}{r}{124}  & \cellcolor{TealBlue!30}{\textbf{158}} & 180.9 & 197 & 242.5 & 173 & 209.5 & 162 & \cellcolor{TealBlue!30}{\textbf{179.6}} & 173 & 194.7 & 159 & 189.9 & 176 & 195.6 & 198 & 254.4 & 179 & 217.8\\
\texttt{breast-cancer} & \multicolumn{1}{r}{683} & \multicolumn{1}{r}{89}  & 45 & 63.4 & 59 & 101.5 & 44 & 72.3 & 46 & \cellcolor{TealBlue!30}{\textbf{59.2}} & \cellcolor{TealBlue!30}{\textbf{42}} & 60.4 & 43 & 61.1 & 49 & 62.1 & 53 & 68.8 & 47 & 65.0\\
\texttt{car} & \multicolumn{1}{r}{1728} & \multicolumn{1}{r}{21}  & 43 & 85.9 & 89 & 132.3 & \cellcolor{TealBlue!30}{42} & 115.5 & 43 & \cellcolor{TealBlue!30}{\textbf{65.6}} & 48 & 80.2 & \cellcolor{TealBlue!30}{42} & 68.5 & \cellcolor{TealBlue!30}{42} & 82.1 & 80 & 114.7 & \cellcolor{TealBlue!30}{42} & 98.7\\
\texttt{forest-fires} & \multicolumn{1}{r}{509} & \multicolumn{1}{r}{989}  & 286 & 345.1 & 321 & 358.9 & 303 & 347.6 & 286 & 326.4 & \cellcolor{TealBlue!30}{\textbf{255}} & \cellcolor{TealBlue!30}{\textbf{299.1}} & 269 & 314.6 & 310 & 347.5 & 290 & 329.7 & 298 & 336.8\\
\texttt{heart-cleveland} & \multicolumn{1}{r}{296} & \multicolumn{1}{r}{95}  & 99 & 125.3 & 125 & 145.2 & 113 & 134.2 & \cellcolor{TealBlue!30}{\textbf{76}} & \cellcolor{TealBlue!30}{\textbf{94.5}} & 85 & 107.5 & 81 & 96.9 & 95 & 118.5 & 108 & 134.8 & 94 & 122.6\\
\texttt{hypothyroid} & \multicolumn{1}{r}{3199} & \multicolumn{1}{r}{83}  & 88 & 119.0 & 108 & 156.6 & 97 & 129.9 & 87 & \cellcolor{TealBlue!30}{\textbf{104.6}} & \cellcolor{TealBlue!30}{\textbf{82}} & 130.3 & 83 & 109.7 & 95 & 120.2 & 105 & 153.2 & 100 & 128.9\\
\texttt{kr-vs-kp} & \multicolumn{1}{r}{3196} & \multicolumn{1}{r}{73}  & 71 & 112.9 & 110 & 187.0 & 76 & 144.7 & 64 & \cellcolor{TealBlue!30}{\textbf{90.5}} & \cellcolor{TealBlue!30}{62} & 123.6 & 64 & 106.1 & 66 & 105.2 & 80 & 163.8 & \cellcolor{TealBlue!30}{62} & 139.0\\
\texttt{lymph} & \multicolumn{1}{r}{148} & \multicolumn{1}{r}{68}  & 38 & 54.4 & 44 & 54.9 & 45 & 58.8 & 29 & \cellcolor{TealBlue!30}{\textbf{35.9}} & 31 & 40.5 & \cellcolor{TealBlue!30}{\textbf{25}} & 37.6 & 37 & 49.0 & 38 & 51.6 & 37 & 52.3\\
\texttt{mushroom} & \multicolumn{1}{r}{8124} & \multicolumn{1}{r}{111}  & \cellcolor{TealBlue!30}{11} & 19.0 & 20 & 32.3 & 12 & 27.6 & \cellcolor{TealBlue!30}{11} & \cellcolor{TealBlue!30}{\textbf{14.4}} & 13 & 18.2 & \cellcolor{TealBlue!30}{11} & 16.6 & \cellcolor{TealBlue!30}{11} & 16.9 & 22 & 33.6 & 13 & 22.2\\
\texttt{primary-tumor} & \multicolumn{1}{r}{283} & \multicolumn{1}{r}{31}  & 65 & 79.8 & 75 & 90.0 & 67 & 82.2 & \cellcolor{TealBlue!30}{\textbf{60}} & \cellcolor{TealBlue!30}{\textbf{76.1}} & 73 & 86.6 & 62 & 79.5 & 70 & 81.4 & 78 & 91.5 & 65 & 85.1\\
\texttt{soybean} & \multicolumn{1}{r}{625} & \multicolumn{1}{r}{50}  & 58 & 77.1 & 47 & 66.1 & 49 & 68.3 & 60 & 71.5 & \cellcolor{TealBlue!30}{\textbf{43}} & 65.0 & 48 & 68.0 & 60 & 76.3 & 45 & \cellcolor{TealBlue!30}{\textbf{63.2}} & 50 & 69.2\\
\texttt{splice-1} & \multicolumn{1}{r}{3188} & \multicolumn{1}{r}{287}  & 1201 & 1732.2 & 1682 & 2115.4 & 1144 & 1760.0 & 301 & 415.8 & 349 & 435.2 & \cellcolor{TealBlue!30}{\textbf{280}} & \cellcolor{TealBlue!30}{\textbf{398.4}} & 824 & 1451.7 & 1326 & 1815.0 & 916 & 1464.9\\
\texttt{tic-tac-toe} & \multicolumn{1}{r}{958} & \multicolumn{1}{r}{27}  & 198 & 395.4 & 227 & 369.4 & 194 & 365.9 & 116 & 156.7 & 100 & 169.5 & \cellcolor{TealBlue!30}{\textbf{84}} & \cellcolor{TealBlue!30}{\textbf{135.1}} & 138 & 300.6 & 126 & 229.2 & 114 & 211.8\\
\texttt{vote} & \multicolumn{1}{r}{435} & \multicolumn{1}{r}{48}  & 34 & 49.8 & 39 & 68.0 & 35 & 51.6 & \cellcolor{TealBlue!30}{\textbf{29}} & \cellcolor{TealBlue!30}{\textbf{36.1}} & 32 & 41.1 & 30 & 38.3 & 31 & 37.8 & 34 & 48.4 & 30 & 42.0\\
\texttt{wine1} & \multicolumn{1}{r}{178} & \multicolumn{1}{r}{1276}  & 65 & 72.0 & 62 & 68.8 & 65 & 73.1 & 33 & 39.8 & 33 & \cellcolor{TealBlue!30}{\textbf{39.5}} & \cellcolor{TealBlue!30}{\textbf{31}} & 40.4 & 54 & 62.7 & 59 & 66.6 & 55 & 64.4\\
\texttt{wine2} & \multicolumn{1}{r}{178} & \multicolumn{1}{r}{1276}  & 81 & 96.4 & 81 & 95.0 & 81 & 95.5 & 42 & 52.5 & 41 & \cellcolor{TealBlue!30}{\textbf{49.0}} & \cellcolor{TealBlue!30}{\textbf{39}} & 50.1 & 54 & 66.0 & 55 & 67.4 & 53 & 66.0\\
\texttt{wine3} & \multicolumn{1}{r}{178} & \multicolumn{1}{r}{1276}  & 50 & 56.5 & 51 & 57.0 & 51 & 56.9 & 30 & 33.9 & \cellcolor{TealBlue!30}{\textbf{27}} & \cellcolor{TealBlue!30}{\textbf{31.6}} & 28 & 33.1 & 49 & 55.2 & 45 & 51.4 & 46 & 54.3\\
\texttt{zoo-1} & \multicolumn{1}{r}{101} & \multicolumn{1}{r}{36}  & \cellcolor{TealBlue!30}{2} & 2.2 & \cellcolor{TealBlue!30}{2} & 3.0 & \cellcolor{TealBlue!30}{2} & 2.9 & \cellcolor{TealBlue!30}{2} & \cellcolor{TealBlue!30}{2.0} & \cellcolor{TealBlue!30}{2} & \cellcolor{TealBlue!30}{2.0} & \cellcolor{TealBlue!30}{2} & \cellcolor{TealBlue!30}{2.0} & \cellcolor{TealBlue!30}{2} & 2.2 & \cellcolor{TealBlue!30}{2} & \cellcolor{TealBlue!30}{2.0} & \cellcolor{TealBlue!30}{2} & 2.1\\\midrule

\texttt{AVERAGE} & \multicolumn{1}{r}{-} & \multicolumn{1}{r}{-}  & 135.9 & 191.2 & 175.9 & 227.4 & 135.7 & 198.3 & 78.7 & \cellcolor{TealBlue!30}{\textbf{98.6}} & 79.6 & 105.1 & \cellcolor{TealBlue!30}{\textbf{73.7}} & 98.7 & 113.2 & 168.2 & 141.8 & 193.1 & 115.0 & 168.7\\
\bottomrule
\end{tabular}

\end{scriptsize}
\end{center}
\vspace{-.25cm}
\caption{\label{tab:exa} Number of explanations}
\end{table}

\vspace{-.1cm}

\begin{table}[h!]
\begin{center}
\begin{scriptsize}
\tabcolsep=2.3pt
\begin{tabular}{lccrrrrrrrrrrrrrrrrrrrr}
\toprule
& && \multicolumn{2}{c}{altern-first} & \multicolumn{2}{c}{altern-rand} & \multicolumn{2}{c}{biased-first} & \multicolumn{2}{c}{biased-rand} & \multicolumn{2}{c}{negative-first} & \multicolumn{2}{c}{negative-rand} & \multicolumn{2}{c}{positive-first} & \multicolumn{2}{c}{positive-rand} & \multicolumn{2}{c}{uniform-first} & \multicolumn{2}{c}{uniform-rand}\\
\cmidrule(rr){4-5}\cmidrule(rr){6-7}\cmidrule(rr){8-9}\cmidrule(rr){10-11}\cmidrule(rr){12-13}\cmidrule(rr){14-15}\cmidrule(rr){16-17}\cmidrule(rr){18-19}\cmidrule(rr){20-21}\cmidrule(rr){22-23}
&\multirow{2}{*}{$\#ex.$} & \multirow{2}{*}{\#feat.} &  \multicolumn{2}{c}{\#lit/cube} & \multicolumn{2}{c}{\#lit/cube} & \multicolumn{2}{c}{\#lit/cube} & \multicolumn{2}{c}{\#lit/cube} & \multicolumn{2}{c}{\#lit/cube} & \multicolumn{2}{c}{\#lit/cube} & \multicolumn{2}{c}{\#lit/cube} & \multicolumn{2}{c}{\#lit/cube} & \multicolumn{2}{c}{\#lit/cube} & \multicolumn{2}{c}{\#lit/cube} \\\cmidrule(rl){4-5}\cmidrule(rl){6-7}\cmidrule(rl){8-9}\cmidrule(rl){10-11}\cmidrule(rl){12-13}\cmidrule(rl){14-15}\cmidrule(rl){16-17}\cmidrule(rl){18-19}\cmidrule(rl){20-21}\cmidrule(rl){22-23}
&& & \multicolumn{1}{c}{min} & \multicolumn{1}{c}{avg} & \multicolumn{1}{c}{min} & \multicolumn{1}{c}{avg} & \multicolumn{1}{c}{min} & \multicolumn{1}{c}{avg} & \multicolumn{1}{c}{min} & \multicolumn{1}{c}{avg} & \multicolumn{1}{c}{min} & \multicolumn{1}{c}{avg} & \multicolumn{1}{c}{min} & \multicolumn{1}{c}{avg} & \multicolumn{1}{c}{min} & \multicolumn{1}{c}{avg} & \multicolumn{1}{c}{min} & \multicolumn{1}{c}{avg} & \multicolumn{1}{c}{min} & \multicolumn{1}{c}{avg} & \multicolumn{1}{c}{min} & \multicolumn{1}{c}{avg} \\
\midrule
\texttt{anneal} & \multicolumn{1}{r}{712} & \multicolumn{1}{r}{90}  & 7.87 & 7.87 & 6.73 & \cellcolor{TealBlue!30}{\textbf{7.63}} & 7.38 & 8.23 & 6.81 & 7.84 & 10.81 & 10.81 & 10.00 & 10.28 & 9.65 & 9.65 & 8.74 & 9.45 & 7.07 & 8.19 & \cellcolor{TealBlue!30}{\textbf{6.65}} & 7.81\\
\texttt{audiology} & \multicolumn{1}{r}{216} & \multicolumn{1}{r}{147}  & 5.83 & 5.83 & 4.22 & 5.55 & 4.71 & 6.13 & \cellcolor{TealBlue!30}{\textbf{3.80}} & 5.70 & 4.84 & \cellcolor{TealBlue!30}{\textbf{4.84}} & 4.68 & 5.52 & 5.68 & 5.68 & 5.24 & 6.61 & 4.42 & 5.87 & 4.14 & 5.69\\
\texttt{australian-credit} & \multicolumn{1}{r}{653} & \multicolumn{1}{r}{125}  & 9.80 & 9.80 & 8.37 & \cellcolor{TealBlue!30}{\textbf{9.26}} & 8.95 & 9.73 & 8.42 & 9.42 & 15.87 & 15.87 & 13.79 & 15.09 & 20.44 & 20.44 & 15.90 & 17.31 & 8.92 & 9.70 & \cellcolor{TealBlue!30}{\textbf{8.33}} & 9.39\\
\texttt{breast-cancer} & \multicolumn{1}{r}{683} & \multicolumn{1}{r}{90}  & 9.54 & 9.54 & 7.36 & 8.93 & 7.90 & 8.94 & 7.32 & \cellcolor{TealBlue!30}{\textbf{8.80}} & 11.40 & 11.40 & 8.27 & 9.37 & 14.27 & 14.27 & 9.47 & 12.21 & 7.73 & 8.95 & \cellcolor{TealBlue!30}{\textbf{7.25}} & 8.83\\
\texttt{car} & \multicolumn{1}{r}{1728} & \multicolumn{1}{r}{22}  & 7.12 & 7.12 & 5.71 & 6.28 & 6.02 & 6.66 & 5.60 & 6.48 & 5.22 & 5.22 & 5.14 & 5.22 & 4.89 & 4.89 & \cellcolor{TealBlue!30}{\textbf{4.77}} & \cellcolor{TealBlue!30}{\textbf{4.80}} & 5.68 & 6.66 & 5.71 & 6.28\\
\texttt{forest-fires} & \multicolumn{1}{r}{509} & \multicolumn{1}{r}{990}  & 24.85 & 24.85 & \cellcolor{TealBlue!30}{\textbf{18.15}} & \cellcolor{TealBlue!30}{\textbf{23.37}} & 22.76 & 25.34 & 19.15 & 23.49 & 31.19 & 31.19 & 28.94 & 31.21 & 41.45 & 41.45 & 37.75 & 41.43 & 22.44 & 25.30 & 18.46 & 23.69\\
\texttt{heart-cleveland} & \multicolumn{1}{r}{296} & \multicolumn{1}{r}{96}  & 9.09 & 9.09 & 6.71 & \cellcolor{TealBlue!30}{\textbf{7.67}} & 7.77 & 8.67 & \cellcolor{TealBlue!30}{\textbf{6.49}} & 7.83 & 15.75 & 15.75 & 12.64 & 13.59 & 12.54 & 12.54 & 9.79 & 10.56 & 7.88 & 8.82 & 6.53 & 7.90\\
\texttt{hypothyroid} & \multicolumn{1}{r}{3199} & \multicolumn{1}{r}{87}  & 8.46 & 8.46 & 7.48 & \cellcolor{TealBlue!30}{\textbf{8.10}} & 7.83 & 8.75 & 7.86 & 8.96 & 10.77 & 10.77 & 9.63 & 10.37 & 13.37 & 13.37 & 10.23 & 11.64 & \cellcolor{TealBlue!30}{\textbf{7.18}} & 8.21 & 7.35 & 8.22\\
\texttt{kr-vs-kp} & \multicolumn{1}{r}{3196} & \multicolumn{1}{r}{74}  & 7.56 & \cellcolor{TealBlue!30}{\textbf{7.56}} & 6.75 & 8.15 & \cellcolor{TealBlue!30}{\textbf{6.56}} & 7.76 & 7.18 & 8.31 & 9.09 & 9.09 & 8.79 & 9.66 & 10.20 & 10.20 & 9.11 & 9.85 & 6.61 & 7.70 & 6.87 & 8.17\\
\texttt{lymph} & \multicolumn{1}{r}{148} & \multicolumn{1}{r}{69}  & 5.33 & \cellcolor{TealBlue!30}{\textbf{5.33}} & 4.89 & 5.90 & 5.23 & 5.97 & \cellcolor{TealBlue!30}{\textbf{4.66}} & 5.88 & 9.40 & 9.40 & 8.66 & 9.49 & 8.12 & 8.12 & 6.92 & 8.47 & 5.22 & 5.95 & 5.05 & 5.96\\
\texttt{mushroom} & \multicolumn{1}{r}{8124} & \multicolumn{1}{r}{113}  & 6.96 & 6.96 & 3.84 & 6.95 & 6.10 & 7.96 & 4.63 & 8.00 & 3.27 & \cellcolor{TealBlue!30}{\textbf{3.27}} & \cellcolor{TealBlue!30}{\textbf{3.17}} & 3.94 & 11.37 & 11.37 & 7.23 & 10.32 & 3.94 & 6.93 & 4.05 & 6.72\\
\texttt{primary-tumor} & \multicolumn{1}{r}{283} & \multicolumn{1}{r}{32}  & 7.15 & 7.15 & 5.85 & 6.45 & 6.10 & 6.65 & \cellcolor{TealBlue!30}{\textbf{5.60}} & 6.26 & 6.64 & 6.64 & 5.82 & \cellcolor{TealBlue!30}{\textbf{6.20}} & 7.46 & 7.46 & 6.96 & 7.25 & 6.07 & 6.63 & 5.87 & 6.39\\
\texttt{soybean} & \multicolumn{1}{r}{625} & \multicolumn{1}{r}{51}  & 8.01 & 8.01 & 6.13 & 6.74 & 6.71 & 7.96 & 6.59 & 7.58 & 10.06 & 10.06 & 7.49 & 8.58 & 6.92 & 6.92 & 6.42 & \cellcolor{TealBlue!30}{\textbf{6.72}} & 6.11 & 7.18 & \cellcolor{TealBlue!30}{\textbf{5.87}} & 6.87\\
\texttt{splice-1} & \multicolumn{1}{r}{3188} & \multicolumn{1}{r}{288}  & 14.66 & 14.66 & 11.18 & 12.44 & 11.38 & 12.90 & 11.14 & \cellcolor{TealBlue!30}{\textbf{12.43}} & 42.44 & 42.44 & 35.24 & 37.39 & 29.58 & 29.58 & 22.76 & 24.09 & 11.17 & 12.98 & \cellcolor{TealBlue!30}{\textbf{10.71}} & 12.49\\
\texttt{tic-tac-toe} & \multicolumn{1}{r}{958} & \multicolumn{1}{r}{28}  & 7.30 & 7.30 & 6.55 & 7.64 & \cellcolor{TealBlue!30}{\textbf{5.36}} & 7.47 & 6.19 & 7.80 & 7.76 & 7.76 & 6.93 & \cellcolor{TealBlue!30}{\textbf{7.24}} & 8.34 & 8.34 & 8.03 & 8.38 & 5.81 & 7.52 & 6.13 & 7.64\\
\texttt{vote} & \multicolumn{1}{r}{435} & \multicolumn{1}{r}{49}  & 5.58 & 5.58 & \cellcolor{TealBlue!30}{\textbf{4.38}} & \cellcolor{TealBlue!30}{\textbf{5.01}} & 5.03 & 5.50 & 4.41 & 5.04 & 5.49 & 5.49 & 4.67 & 5.15 & 7.57 & 7.57 & 5.98 & 6.62 & 4.93 & 5.50 & 4.49 & 5.07\\
\texttt{wine1} & \multicolumn{1}{r}{178} & \multicolumn{1}{r}{1277}  & 24.99 & 24.99 & 23.99 & 25.93 & 29.53 & 31.11 & 28.86 & 31.98 & 1.98 & 1.98 & 1.98 & 1.98 & 1.96 & 1.96 & \cellcolor{TealBlue!30}{\textbf{1.96}} & \cellcolor{TealBlue!30}{\textbf{1.96}} & 18.52 & 25.03 & 17.87 & 25.90\\
\texttt{wine2} & \multicolumn{1}{r}{178} & \multicolumn{1}{r}{1277}  & 25.49 & 25.49 & 23.99 & 25.49 & 29.98 & 32.50 & 29.55 & 32.06 & 1.98 & 1.98 & 1.98 & 1.98 & 1.96 & 1.96 & \cellcolor{TealBlue!30}{\textbf{1.96}} & \cellcolor{TealBlue!30}{\textbf{1.96}} & 18.39 & 25.51 & 17.43 & 25.46\\
\texttt{wine3} & \multicolumn{1}{r}{178} & \multicolumn{1}{r}{1277}  & 23.99 & 23.99 & 22.49 & 23.37 & 29.93 & 31.30 & 28.78 & 30.76 & 1.98 & 1.98 & 1.98 & 1.98 & 1.96 & 1.96 & \cellcolor{TealBlue!30}{\textbf{1.95}} & \cellcolor{TealBlue!30}{\textbf{1.96}} & 17.60 & 24.04 & 15.30 & 23.28\\
\texttt{zoo-1} & \multicolumn{1}{r}{101} & \multicolumn{1}{r}{37}  & \cellcolor{TealBlue!30}{1.00} & \cellcolor{TealBlue!30}{1.00} & \cellcolor{TealBlue!30}{1.00} & \cellcolor{TealBlue!30}{1.00} & \cellcolor{TealBlue!30}{1.00} & \cellcolor{TealBlue!30}{1.00} & \cellcolor{TealBlue!30}{1.00} & 1.07 & \cellcolor{TealBlue!30}{1.00} & \cellcolor{TealBlue!30}{1.00} & \cellcolor{TealBlue!30}{1.00} & 1.12 & \cellcolor{TealBlue!30}{1.00} & \cellcolor{TealBlue!30}{1.00} & \cellcolor{TealBlue!30}{1.00} & \cellcolor{TealBlue!30}{1.00} & \cellcolor{TealBlue!30}{1.00} & \cellcolor{TealBlue!30}{1.00} & \cellcolor{TealBlue!30}{1.00} & 1.05\\\midrule
\texttt{AVERAGE} & \multicolumn{1}{r}{-} & \multicolumn{1}{r}{-}  & 11.03 & 11.03 & 9.29 & 10.59 & 10.81 & 12.03 & 10.20 & 11.79 & 10.35 & 10.35 & 9.04 & \cellcolor{TealBlue!30}{\textbf{9.77}} & 10.94 & 10.94 & 9.11 & 10.13 & 8.84 & 10.88 & \cellcolor{TealBlue!30}{\textbf{8.25}} & 10.64\\
\bottomrule
\end{tabular}

\end{scriptsize}
\end{center}
\vspace{-.25cm}
\caption{\label{tab:fea} Mean number of features}
\end{table}

\end{center}

\restoregeometry




\end{document}

