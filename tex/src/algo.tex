\documentclass{article}
\usepackage[usenames,dvipsnames,svgnames,table]{xcolor}%% http://ctan.org/pkg/xcolor
\usepackage[utf8]{inputenc}
\usepackage{xspace}
\usepackage{array}
%\usepackage{amsthm}
\usepackage{amsmath} 
\usepackage{amssymb} 
\usepackage[ruled,vlined]{algorithm2e}
\usepackage{booktabs}
\usepackage{multirow}
\usepackage{url}
\usepackage{tikz}
\usepackage{fp}
\usepackage{subfig}
\usetikzlibrary{arrows,shadows,fit,calc,positioning,decorations.pathreplacing,matrix,shapes,petri,topaths,fadings,mindmap,backgrounds,shapes.geometric}
\usepackage{geometry}
\usepackage{xifthen}
\usepackage{rotating}

\newgeometry{bottom=2cm,top=2cm}


\DeclareMathOperator*{\argmax}{arg\,max}
\DeclareMathOperator*{\argmin}{arg\,min}


\newtheorem{theorem}{Theorem}


\newcommand{\tru}[0]{\texttt{true}}
\newcommand{\fal}[0]{\texttt{false}}
% \newcommand{\tru}[0]{1}
% \newcommand{\fal}[0]{0}


\newcommand{\setex}[1]{\ensuremath{{\mathcal X}^{#1}}\xspace}
\newcommand{\posex}{{\setex{\top}}\xspace}
\newcommand{\negex}{{\setex{\bot}}\xspace}
\newcommand{\setcube}[1]{\ensuremath{{\mathcal C}^{#1}}\xspace}
\newcommand{\poscube}{{\setcube{1}}\xspace}
\newcommand{\negcube}{{\setcube{0}}\xspace}
\newcommand{\features}{\ensuremath{{\mathcal F}}\xspace}
\newcommand{\feat}{\ensuremath{f}}
\newcommand{\classifier}{\ensuremath{f}}
\newcommand{\lit}[1]{\ensuremath{l_{#1}}}
\newcommand{\var}{\ensuremath{x}}
\newcommand{\truelit}[1]{\ensuremath{\var_{#1}}}
\newcommand{\falselit}[1]{\ensuremath{\bar{\var_{#1}}}}
\newcommand{\ex}{\ensuremath{\var}}
\newcommand{\cube}{\ensuremath{c}}
\newcommand{\universe}{\ensuremath{{\mathcal U}}}
\newcommand{\entropy}[1]{\ensuremath{{H}({#1})}\xspace}
\newcommand{\probability}[1]{\ensuremath{{Pr}({#1})}\xspace}



\newcommand{\nodes}[0]{\ensuremath{{\cal N}}}
\newcommand{\bud}[0]{\ensuremath{{\cal B}}}
\newcommand{\sequence}[0]{\ensuremath{{\cal S}}}

\newcommand{\maxd}[0]{\ensuremath{k}}
\newcommand{\anode}[0]{\ensuremath{i}}
\newcommand{\bnode}[0]{\ensuremath{b}}
\newcommand{\afeat}[0]{\ensuremath{f}}
\newcommand{\aclass}[0]{\ensuremath{l}}
\newcommand{\ub}[0]{\ensuremath{ub}}
% \newcommand{\error}[0]{\ensuremath{error}}
\newcommand{\depth}[1][]{\ensuremath{\ifthenelse{\equal{#1}{}}{depth}{depth({#1})}}}
\newcommand{\test}[1][]{\ensuremath{\ifthenelse{\equal{#1}{}}{test}{test({#1})}}}
\newcommand{\dom}[1][]{\ensuremath{\ifthenelse{\equal{#1}{}}{dom}{dom({#1})}}}
\newcommand{\best}[1][]{\ensuremath{\ifthenelse{\equal{#1}{}}{{best}}{{best({#1})}}}}
\newcommand{\maxe}[1][]{\ensuremath{\ifthenelse{\equal{#1}{}}{{max}}{{max({#1})}}}}
\newcommand{\opt}[1][]{\ensuremath{\ifthenelse{\equal{#1}{}}{terminal}{terminal({#1})}}}
\newcommand{\child}[1][]{\ensuremath{\ifthenelse{\equal{#1}{}}{{child}}{{child({{#1}})}}}}
\newcommand{\error}[1][]{\ensuremath{\ifthenelse{\equal{#1}{}}{{error}}{{error({{#1}})}}}}
\newcommand{\classlabel}[1][]{\ensuremath{\ifthenelse{\equal{#1}{}}{{y}}{{y({{#1}})}}}}

\newcommand{\abranch}[0]{\ensuremath{b}}



	\SetKwFunction{storebest}{storeBest}
	\SetKwFunction{solution}{solution}
	\SetKwFunction{deadend}{deadEnd}
	\SetKwFunction{expend}{expend}
	\SetKwFunction{backtrack}{backtrack}
	\SetKwFunction{dt}{BTDT}
	\SetKwFunction{pop}{pop}
	\SetKwFunction{push}{push}
	\SetKwFunction{prune}{prune}
	% \SetKwFunction{error}{error}
	\SetKwFunction{grow}{grow}
	\SetKwFunction{branch}{branch}
	\SetKwFunction{select}{select\&remove}
	\SetKwFunction{budalg}{Bud-first-search}
	\SetKwFunction{dynprog}{DynProg}
	\SetKwFunction{dleight}{DL8.5}
	\SetKwFunction{cart}{CART}
	\SetKwFunction{greedy}{BudFS (first sol.)}
	
	

\DontPrintSemicolon

\title{Backtracking DT}

% \author{Emmanuel Hebrard\inst{1} \and George Katsirelos\inst{2}}
% \institute{LAAS-CNRS, Universit\'e de Toulouse, CNRS, France, email: hebrard@laas.fr
  % \and MIAT, UR-875, INRA, France, email: gkatsi@gmail.com \footnote{The second author was partially supported by the french ``Agence nationale de
% la Recherche'', project DEMOGRAPH, reference ANR-16-C40-0028.}}

\begin{document}


\maketitle

\section*{Introduction}

We consider the problem of finding the bounded-depth decision tree of maximum accuracy.
The state of the art includes MIP approaches (BinOCT), a MaxSAT approach based on the SAT encoding proposed by Narodytska et al, and DL8.5.
The latter algorithm is by far the most efficient, however, it is not \emph{anytime}: the left branch must be optimally solved before a solution of the right branch can be found. Moreover the use of a cache structure means that it uses a lot of memory. This algorithm is practical for a maximum depth of 4 (although using gigabytes of memory) but often not much beyond. Therefore, in a number of cases, a greedy heuristic (such as CART) is still the best method in practice.

\medskip

In this note we introduce what is essentially an anytime version of DL8.5, without cache.
This algorithm therefore uses linear (in the size of the tree) memory and anytime, hence in principle strictly better than CART. Moreover, on instance where DL8.5 can find a solution, the algorithm described in this note is significantly faster (by about a factor 10).

\section*{Notations}

A binary data set is a pair $\langle \negex,\posex \rangle$ where $\negex$ (resp. $\posex$) is a subset of $2^{\features}$. It is associated label function $\classlabel : 2^{\features} \mapsto \{\top,\bot\}$ such that:
$$
\forall \aclass \in \{\top,\bot\}, \forall \ex \in \setex{\aclass}, \classlabel[\ex] = \aclass
$$
%$\top$ and $\bot$ are class labels and 
A datapoint $\ex$ can equivalently be seen as a subset of $\features$, or as the conjunction:
$$
\bigwedge_{\afeat \in \ex}\afeat \wedge \bigwedge_{\afeat \in \features \setminus \ex}\bar{\afeat}
$$

A \emph{branch} of a decision tree is a conjunction of (possibly negated) features. 
%Moreover, if we also consider data points as conjunctions of features (where every feature appears either positively or negatively),
%given a branch $\abranch \subseteq \features$ 
Given a data set $\langle \negex,\posex \rangle$, we can associate a data set $\langle \negex(\abranch),\posex(\abranch) \rangle$ to a branch $\abranch$ where:
\begin{eqnarray*}
\negex(\abranch) = \{\ex \mid \ex \in \negex, \ex \models \abranch\}\\
\posex(\abranch) = \{\ex \mid \ex \in \posex, \ex \models \abranch\}
\end{eqnarray*}

We write $\error[\abranch]$ for $\min(|\negex(\abranch)|, |\posex(\abranch)|)$, and $\error[\abranch,\afeat]$ for $\min(\error[\abranch \wedge \afeat], \error[\abranch \wedge \bar{\afeat}])$.
A branch \abranch\ is said \emph{pure} iff $\error[\abranch]=0$.

\medskip

Given a binary dataset $\langle \negex,\posex \rangle$ with label function $\classlabel$,
the \emph{minimum error bounded depth decision tree problem} consists in finding the minimum value $\epsilon$ such
there exists a binary tree of depth at most $\maxd$ such that the sum of the error $\error[\abranch]$ for all branches $\abranch$ from the root to the leaves is equal to $\epsilon$.


\section*{Dynamic Programming Algorithm}

The solver DL8.5 is a dynamic programming algorithm. It relies on the observation that given a feature test, the two resulting branches are independent subproblems. Algorithm~\ref{alg:dynprog} gives a high level view of DL8.5.


	% \begin{algorithm}
	% 	\caption{Dynamic Programming Algorithm\label{alg:dynprog}}
	% 	\TitleOfAlgo{\dynprog}
	% 	  \KwData{$\negex,\posex,\maxd,\abranch[=(\tru)]$}
	% 	  \KwResult{The minimum error on $\negex,\posex$ for decision trees of depth at most $\maxd$}
	% 		\lIf{$\maxd = 0$ or $\error[\abranch] = 0$} {
	% 		\Return $\error[\abranch]$
	% 		}
	% 		$\best \gets \error[\abranch]$\;
	% 		\ForEach{$\afeat \in \features \setminus \abranch$} {
	% 				$\best \gets \min(\best, \dynprog(\negex,\posex, \abranch \wedge \afeat, \maxd-1) + \dynprog(\negex,\posex, \abranch \wedge \bar{\afeat}, \maxd-1))$\;
	% 		}
	% 		\Return $\best$\;
	% \end{algorithm}
	
	\begin{algorithm}
		\caption{Dynamic Programming Algorithm\label{alg:dynprog}}
		\TitleOfAlgo{\dynprog}
		  \KwData{$\negex,\posex,\maxd$}
		  \KwResult{The minimum error on $\negex,\posex$ for decision trees of depth at most $\maxd$}
			$\error \gets \min(|\negex|,|\posex|)$\;
			
			\lIf{$\maxd = 0$ or $\error = 0$} {
			\Return $\error$
			}
			
			\ForEach{$\afeat \in \features$} {
					$\error \gets \min(\error, \dynprog(\negex(\afeat),\posex(\afeat), \maxd-1) + \dynprog(\negex(\bar{\afeat}),\posex(\bar{\afeat}), \maxd-1))$\;
			}
			\Return $\error$\;
	\end{algorithm}



\section*{An Anytime Algorithm}

In a nutshell the algorithm we propose works as follows:

\medskip

We start from a singleton set \bud\ of open branches or \emph{buds}.
% (open branches are branches of length strictly less than $\maxd$ that are not pure). % The set \nodes\ contains all the nodes of the current tree, open or closed, it is initially equal to \bud. Finally,
Moreover, we use an initially empty stack of decisions $\sequence$.

\begin{itemize}
	\item As long as there is a bud ($\bud \neq \emptyset$), we pick any one $\abranch \in \bud$, a feature $\afeat$ marked as \emph{available} for \abranch, add the pair $(\abranch,\afeat)$ to \sequence\ and expend the tree with the two branches $\abranch \wedge \afeat$ and $\abranch \wedge \bar{\afeat}$. 
	%These new nodes are added to $\nodes$. 
	They are added to $\bud$ if their depth is strictly less than $\maxd-1$ (the last feature test is chosen according to minimum error) and if they are not pure, otherwise they are terminal tests and we record the corresponding error.

\item As long as there is no bud ($\bud = \emptyset$), we pop the last assignment $(\abranch,\afeat)$ from \sequence, mark the feature $\afeat$ not available for branch $\abranch$ and update the recorded best error of its subtrees. If there is at least one available feature for branch $\abranch$, we add $\abranch$ to $\bud$. 
Otherwise, we consider it terminal and its error is the sum of the error of its best subtrees.
% and we 
%store the minimum error recorded for any of the possible features.

\end{itemize}

The classification error of a tree is equal to the sum of the error of its terminal banches, the algorithm returns the minimum value encountered when exploring the possible decision trees.

\medskip
		
		Let $n = |\posex| + |\negex|$, $m = |\features|$, with the maximum depth $k \leq m$. 
		The number of recursive calls for Algorithm~\ref{alg:dynprog} is $\Theta(2^km^k)$. At each call, the data set must be split into two subsets. This splitting procedure can be done in time $\Theta(n)$ amortised over a given level of a given decision tree. Since every branch of the tree is independent, Algorithm~\ref{alg:dynprog} explores $O(m^d)$ trees of depth $d$. Therefore Algorithm~\ref{alg:dynprog} runs in $\Theta((n+2^k)m^k)$ time.
		
		\begin{theorem}
			The time complexities of Algorithms~\ref{alg:dynprog} and Algorithm~\ref{alg:bud} are equal.
			\end{theorem}
			
			No proof is given, but independent subtrees are not explored in Algorithm~\ref{alg:bud} hence both algorithms do the same computation, except not in the same order. Notice that Algorithm~\ref{alg:bud} eagerly compute the conditional error $\error[\abranch,\afeat]$ for every feature $\afeat \in \features$, which incurs an extra factor $m$ for the data set partitionning task. However, in return the last test of each branch is chosen in $O(1)$ so we gain the same factor $m$.
			
			The big difference is that whereas the cost of finding a first solution is $\Theta((n + 2^{k-1})m^{k-1})$ for Algorithm~\ref{alg:dynprog}, it is equal to $\Theta(kn + 2^k)$ for Algorithm~\ref{alg:bud}, which in practice is very important, as shown in the experimental section.
		
	%T(m,k) = 2mT(m-1, k-1)	
		
		


% \clearpage





	\begin{algorithm}
		\caption{Anytime Algorithm\label{alg:bud}}
		\TitleOfAlgo{\budalg}
		  \KwData{$\negex,\posex, \maxd$}
		  \KwResult{The minimum error on $\negex,\posex$ for decision trees of depth at most $\maxd$}
		$\sequence \gets []$\;
		$\bud \gets \{\emptyset\}$\;
		$\error \gets \min(|\negex|,|\posex|)$\;
		$\dom \gets (\lambda : {2^{\features}} \mapsto \features)$\;
		$\best \gets (\lambda : {2^{\features}} \mapsto \infty)$\;
		% $\opt \gets (\lambda : {2^{\features}} \mapsto \fal)$\;
		
		\While{$|\sequence| + |\bud| > 0$}{
		\eIf{$\bud \neq \emptyset$}{
			$\abranch \gets \select{\bud}$\;
			$\afeat \gets \argmin_{\afeat \in \dom[\abranch]}(\error[\abranch,\afeat])$\;
			\eIf{$\error[\abranch,\afeat] = 0$ or $|\abranch| = \maxd-1$} {
				$\error \gets \error + \error[\abranch,\afeat]$\;
				$\best[\abranch] \gets \error[\abranch,\afeat]$\;
				% $\opt[\abranch] \gets \tru$\;
			}{
			$\dom[\abranch] \gets \dom[\abranch] \setminus \{\afeat\}$\;
			push $(\abranch,\afeat)$ on $\sequence$\; % $ \gets \sequence \oplus (\abranch,\afeat)$\;
			\ForEach{$v \in \{\afeat, \bar{\afeat}\}$}{
					\lIf{$\error[\abranch,v] > 0$}{
						$\bud \gets \bud \cup \{\abranch \wedge v\}$
					}
			}
			}
		}{
			$\best[\emptyset] \gets \min(\best[\emptyset], \error)$\;
			\Repeat{$\dom[\abranch] = \emptyset$ and $|\sequence|>0$}{
				pop $(\abranch,\afeat)$ from $\sequence$\;
				$\best[\abranch] \gets \min(\best[\abranch], \best[\abranch \wedge \afeat] +  \best[\abranch \wedge \bar{\afeat}])$\;
				$\error \gets \error - \best[\abranch \wedge \afeat] -  \best[\abranch \wedge \bar{\afeat}]$\;
				\lIf{$\dom[\abranch] \neq \emptyset$} {
					$\bud \gets \bud \cup \{\abranch\}$
					% \lIf{$\opt[\abranch]$}{$\error \gets \error - \best[\abranch]$}
				} 
				\lElse {	
					$\error \gets \error + \error[\abranch,\afeat]$
				}
				% {
				% 	$\opt[\abranch] \gets \tru$\;
				% }
			}
		}
		}
		\Return $\best[\emptyset]$\;
	\end{algorithm}
	
	



% For readability, we cut the algorithm into four blocks. The initialisation procedure (Algorithm~\ref{alg:init}) set up the data structures used in all other procedures:
% \begin{itemize}
% 	\item \sequence\ is simply the list of nodes in the current tree, ordered as they are explored.
% 	\item \nodes\ is the set of integers used to index a node of the current tree
% 	\item \bud\ is the set of nodes which do no have an assigned test yet
% 	\item \depth\ stores the depth of a node
% 	\item \test\ stores the feature tested at a node
% 	\item \dom\ stores the set of possible features which have no yet been tried for this node
% 	\item \best\ stores the error of the best subtree rooted at a node
% 	\item \opt\ indicates whether the best subtree of a given node is optimal
% 	\item \child\ stores the children of a node (children can be nodes or $\{\top, \bot\}$)
% 	\item $\error{\anode}$ $\min(|\posex(\anode)|,|\negex(\anode)|)$
% 	\item $\error{\anode,\afeat}$ $\min(|\posex(\anode=\afeat)|,|\negex(\anode=\afeat)|)$
% \end{itemize}
%
% Algorithm~\ref{alg:search} is a bactracking procedure which expends a current decision tree
%
% 	\begin{algorithm}
% 		\caption{Data Structures\label{alg:init}}
% 		\TitleOfAlgo{Initialise}
% 		$\sequence \gets []$\;
% 		$\bud \gets \emptyset$\;
% 		$\nodes \gets \emptyset$\;
% 		$\ub \gets \min(|\negex|,|\posex|)$\;
% 		$\error \gets ub$\;
%
% 		$\child \gets (\lambda : \mathbb{N} \times \{\fal, \tru\} \mapsto \emptyset)$\;
% 		$\depth \gets (\lambda : \mathbb{N} \mapsto 0)$\;
%
% 		$\test \gets (\lambda : \mathbb{N} \mapsto \emptyset)$\;
% 		$\dom \gets (\lambda : \mathbb{N} \mapsto \features)$\;
%
% 		$\best \gets (\lambda : \mathbb{N} \mapsto \infty)$\;
% 		$\opt \gets (\lambda : \mathbb{N} \mapsto \fal)$\;
% 	\end{algorithm}
%
%
%
% 	\begin{algorithm}
% 		\caption{Create a new node after branching\label{alg:alloc}}
% 		\TitleOfAlgo{\grow}
% 	  \KwData{integer \anode}
%
% 		$\nodes \gets \nodes \cup \{\anode\}$\;
% 		$\dom[\anode] \gets \features$ sorted by decreasing conditional error $\min(|\posex(\anode=\afeat)|,|\negex(\anode=\afeat)|)$\;
% 		$\test[\anode] \gets \pop(\dom[\anode])$\;
%
%
% 		\eIf{$\depth[\anode]=k-1$ or $\error{\anode,\test[\anode]}$}
% 		{
% 			$\best[\anode] = \error{\anode,\test[\anode]}$\; %\min(|\posex(\anode=\test[\anode])|,|\negex(\anode=\test[\anode])|)$\;
% 			$\opt[\anode] = \tru$\;
% 			\ForEach{$branch \in \{\tru, \fal\}$}
% 			{
% 				% $\child[\anode,branch] \gets (|\posex(\anode=\test[\anode])| > |\negex(\anode=\test[\anode])|)$\;
% 				\lIf{$|\posex(\anode=\test[\anode])| > |\negex(\anode=\test[\anode])|$}{$\child[\anode,branch] \gets \top$}
% 				\lElse{$\child[\anode,branch] \gets \bot$}
% 			}
% 		}
% 		{
% 			$\bud \gets \bud \cup \{n\}$\;
% 			$\best[\anode] = \min(|\posex(\anode)|, |\negex(\anode)|)$\;
% 			$\opt[\anode] \gets \fal$\;
% 		}
%
%
% 	\end{algorithm}
%
%
% 	\begin{algorithm}
% 		\caption{Suppress a node and all its descendants\label{alg:free}}
% 		\TitleOfAlgo{\prune}
% 	  \KwData{integer \anode}
%
% 		$\bud \gets \bud \setminus \{\anode\}$\;
% 		$\nodes \gets \nodes \setminus \{\anode\}$\;
%
% 		\ForEach{$branch \in \{\tru, \fal\}$}
% 		{
% 		\lIf{$\child[\anode,branch] \not\in \{\top, \bot\}$}
% 		{
% 			$\prune{\child[\anode,branch]}$
% 		}
% 		}
%
% 		\lIf{$\depth[\anode] = k-1$ or $\opt[\anode]$}{$error \gets error - \best[\anode]$}
%
% 	\end{algorithm}
%
%
% \begin{algorithm}
% 	\caption{Search loop\label{alg:search}}
%   \TitleOfAlgo{\dt}
%   \KwData{$\negex,\posex, k$}
%   \KwResult{A decision tree}
%
% 	$\bnode \gets 0$\;
% 	$\posex(1),\negex(1) \gets \posex, \negex$\;
% 	$\grow{\bud, \sequence, 1}$\;
%
% 	\While{\textbf{true}}{
% 		\eIf{$\bud = \emptyset$} {
% 			$\ub \gets \min(\ub,\error)$\;
% 			$deadend \gets \fal$\;
% 			\Repeat{$deadend$}{
% 				\lIf{$\bnode > 0$}{$\opt[\bnode] \gets \tru$}
% 				\lIf{$\bnode = 1$}{\Return}
% 				$\bnode \gets \pop{\sequence}$\;
% 				$\best[\bnode] \gets \min(\best[\bnode], \best(\child[\bnode,\tru]) + \best(\child[\bnode,\fal]))$\;
% 				$\test[\bnode] \gets \pop{\dom[\bnode]}$\;
% 				$\prune(\child[\bnode,\tru])$\;
% 				$\prune(\child[\bnode,\fal])$\;
%
% 				$deadend \gets \best[\bnode] = 0 ~\vee~ \dom[\bnode] = \emptyset$\;
% 				\If{$deadend$}
% 				{
% 				$\opt[\bnode] \gets \tru$\;
% 				$\error \gets \error + \error{\bnode}$\; %$\best[\bnode]$\;
% 				}
% 			}
% 			$\bud \gets \bud \cup \{b\}$\;
% 			$\error \gets \error + \min(|\posex(\bnode)|, |\negex(\bnode)|)$\;
% 		}
% 		{
% 			\If{$b = 0$}{
% 				$b=\select{\bud}$\;
% 				% $\bud \gets \bud \setminus \{b\}$\;
% 				$\push(\bnode,\sequence)$\;
% 			}
% 			$c_{\tru}, c_{\fal} = \argmin_{x,y}(\mathbb{N} \setminus \nodes)$\;
% 			$\posex(c_{\tru}),\negex(c_{\tru}),\posex(c_{\fal}),\negex(c_{\fal}) \gets \branch(\posex(\bnode),\negex(\bnode),\test[\bnode])$\;
% 			\ForEach{$branch \in \{\tru, \fal\}$}{
% 				\eIf{$\min(|\posex(c_{branch})|,|\negex(c_{branch})|) = 0$}
% 				{
% 					\lIf{$|\posex(c_{branch})|>|\negex(c_{branch})|$}{$\child[\bnode,branch] \gets \top$}
% 					\lElse{$\child[\bnode,branch] \gets \bot$}
% 				}{
% 					$\child[\bnode,branch] \gets c_{branch}$\;
% 					$\depth[c_{branch}] \gets \depth[\bnode]+1$\;
% 					$\grow(\bud, \sequence, c_{branch})$\;
% 				}
% 			}
% 			$\bnode \gets 0$\;
% 		}
% 	}
%
% \end{algorithm}

%\clearpage

\section*{Improvements}

There are a number of implementation details (among them how to return the actual tree rather than simply its classification error) and other improvements, which we will list later.


% \clearpage

\section*{Experiments}

\subsection*{Comparison with \dleight}

We first compare our algorithm (\budalg) against \dleight on relatively shallow trees (3,4 and 5) in tables~\ref{tab:d3}, \ref{tab:d4} and \ref{tab:d5}, respectively. 
We give the minimum \emph{error}, maximum \emph{accuracy} (acc.), the cpu \emph{time} and size of the search space (\emph{choices}) required to prove optimality (when a proof is given, as markes by a 1 in the column \emph{opt}) or to find the best solution (otherwise).

\medskip

When the maximum depth and number of feature is not too large, both algorithms are comparable, although \budalg is systematically faster. However, when the depth or the number of features grows, the best solution found by \dleight is often of much lower quality. In fact, in most cases, it reaches the time or memory limit without outputing a solution (the missing entries corresponds to \dleight reaching the 50GB memory limit). Notice that \budalg uses a tiny memory space (much lower than the size of the data set).



\begin{table}[htbp]
\begin{center}
\begin{normalsize}
\tabcolsep=5pt
\begin{tabular}{lccrrrrrrrr}
\toprule
& && \multicolumn{4}{c}{\dleight} & \multicolumn{4}{c}{\budalg}\\
\cmidrule(rr){4-7}\cmidrule(rr){8-11}
&\multirow{1}{*}{$\#ex.$} & \multirow{1}{*}{\#feat.} &  \multicolumn{1}{c}{opt} & \multicolumn{1}{c}{error} & \multicolumn{1}{c}{acc.} & \multicolumn{1}{c}{time} & \multicolumn{1}{c}{opt} & \multicolumn{1}{c}{error} & \multicolumn{1}{c}{acc.} & \multicolumn{1}{c}{time} \\
\midrule

\texttt{anneal} & \multicolumn{1}{r}{812} & \multicolumn{1}{r}{47}  & \cellcolor{TealBlue!30}{1} & \cellcolor{TealBlue!30}{112} & \cellcolor{TealBlue!30}{0.862} & 1.9 & \cellcolor{TealBlue!30}{1} & \cellcolor{TealBlue!30}{112} & \cellcolor{TealBlue!30}{0.862} & \cellcolor{TealBlue!30}{\textbf{0.0}}\\
\texttt{audiology} & \multicolumn{1}{r}{216} & \multicolumn{1}{r}{79}  & \cellcolor{TealBlue!30}{1} & \cellcolor{TealBlue!30}{5} & \cellcolor{TealBlue!30}{0.977} & 3.6 & \cellcolor{TealBlue!30}{1} & \cellcolor{TealBlue!30}{5} & \cellcolor{TealBlue!30}{0.977} & \cellcolor{TealBlue!30}{\textbf{0.1}}\\
\texttt{australian-credit} & \multicolumn{1}{r}{653} & \multicolumn{1}{r}{73}  & \cellcolor{TealBlue!30}{1} & \cellcolor{TealBlue!30}{73} & \cellcolor{TealBlue!30}{0.888} & 8.7 & \cellcolor{TealBlue!30}{1} & \cellcolor{TealBlue!30}{73} & \cellcolor{TealBlue!30}{0.888} & \cellcolor{TealBlue!30}{\textbf{0.1}}\\
\texttt{breast-cancer-un} & \multicolumn{1}{r}{683} & \multicolumn{1}{r}{89}  & \cellcolor{TealBlue!30}{1} & \cellcolor{TealBlue!30}{24} & \cellcolor{TealBlue!30}{0.965} & 0.9 & \cellcolor{TealBlue!30}{1} & \cellcolor{TealBlue!30}{24} & \cellcolor{TealBlue!30}{0.965} & \cellcolor{TealBlue!30}{\textbf{0.1}}\\
\texttt{breast-wisconsin} & \multicolumn{1}{r}{683} & \multicolumn{1}{r}{120}  & \cellcolor{TealBlue!30}{1} & \cellcolor{TealBlue!30}{15} & \cellcolor{TealBlue!30}{0.978} & 5.1 & \cellcolor{TealBlue!30}{1} & \cellcolor{TealBlue!30}{15} & \cellcolor{TealBlue!30}{0.978} & \cellcolor{TealBlue!30}{\textbf{0.1}}\\
\texttt{car-un} & \multicolumn{1}{r}{1728} & \multicolumn{1}{r}{21}  & \cellcolor{TealBlue!30}{1} & \cellcolor{TealBlue!30}{192} & \cellcolor{TealBlue!30}{0.889} & 0.0 & \cellcolor{TealBlue!30}{1} & \cellcolor{TealBlue!30}{192} & \cellcolor{TealBlue!30}{0.889} & \cellcolor{TealBlue!30}{\textbf{0.0}}\\
\texttt{diabetes} & \multicolumn{1}{r}{768} & \multicolumn{1}{r}{112}  & \cellcolor{TealBlue!30}{1} & \cellcolor{TealBlue!30}{162} & \cellcolor{TealBlue!30}{0.789} & 9.1 & \cellcolor{TealBlue!30}{1} & \cellcolor{TealBlue!30}{162} & \cellcolor{TealBlue!30}{0.789} & \cellcolor{TealBlue!30}{\textbf{0.1}}\\
\texttt{forest-fires-un} & \multicolumn{1}{r}{517} & \multicolumn{1}{r}{989}  & - & - & - & - & \cellcolor{TealBlue!30}{\textbf{1}} & \cellcolor{TealBlue!30}{\textbf{193}} & \cellcolor{TealBlue!30}{\textbf{0.627}} & \cellcolor{TealBlue!30}{\textbf{19.3}}\\
\texttt{german-credit} & \multicolumn{1}{r}{1000} & \multicolumn{1}{r}{110}  & \cellcolor{TealBlue!30}{1} & \cellcolor{TealBlue!30}{236} & \cellcolor{TealBlue!30}{0.764} & 7.4 & \cellcolor{TealBlue!30}{1} & \cellcolor{TealBlue!30}{236} & \cellcolor{TealBlue!30}{0.764} & \cellcolor{TealBlue!30}{\textbf{0.2}}\\
\texttt{heart-cleveland} & \multicolumn{1}{r}{296} & \multicolumn{1}{r}{50}  & \cellcolor{TealBlue!30}{1} & \cellcolor{TealBlue!30}{41} & \cellcolor{TealBlue!30}{0.861} & 3.5 & \cellcolor{TealBlue!30}{1} & \cellcolor{TealBlue!30}{41} & \cellcolor{TealBlue!30}{0.861} & \cellcolor{TealBlue!30}{\textbf{0.1}}\\
\texttt{hepatitis} & \multicolumn{1}{r}{137} & \multicolumn{1}{r}{68}  & \cellcolor{TealBlue!30}{1} & \cellcolor{TealBlue!30}{10} & \cellcolor{TealBlue!30}{0.927} & 1.0 & \cellcolor{TealBlue!30}{1} & \cellcolor{TealBlue!30}{10} & \cellcolor{TealBlue!30}{0.927} & \cellcolor{TealBlue!30}{\textbf{0.0}}\\
\texttt{hypothyroid} & \multicolumn{1}{r}{3247} & \multicolumn{1}{r}{43}  & \cellcolor{TealBlue!30}{1} & \cellcolor{TealBlue!30}{61} & \cellcolor{TealBlue!30}{0.981} & 4.0 & \cellcolor{TealBlue!30}{1} & \cellcolor{TealBlue!30}{61} & \cellcolor{TealBlue!30}{0.981} & \cellcolor{TealBlue!30}{\textbf{0.1}}\\
\texttt{ionosphere} & \multicolumn{1}{r}{351} & \multicolumn{1}{r}{444}  & \cellcolor{TealBlue!30}{1} & \cellcolor{TealBlue!30}{22} & \cellcolor{TealBlue!30}{0.937} & 392.3 & \cellcolor{TealBlue!30}{1} & \cellcolor{TealBlue!30}{22} & \cellcolor{TealBlue!30}{0.937} & \cellcolor{TealBlue!30}{\textbf{3.7}}\\
\texttt{kr-vs-kp} & \multicolumn{1}{r}{3196} & \multicolumn{1}{r}{37}  & \cellcolor{TealBlue!30}{1} & \cellcolor{TealBlue!30}{198} & \cellcolor{TealBlue!30}{0.938} & 2.0 & \cellcolor{TealBlue!30}{1} & \cellcolor{TealBlue!30}{198} & \cellcolor{TealBlue!30}{0.938} & \cellcolor{TealBlue!30}{\textbf{0.1}}\\
\texttt{letter} & \multicolumn{1}{r}{20000} & \multicolumn{1}{r}{224}  & \cellcolor{TealBlue!30}{1} & \cellcolor{TealBlue!30}{369} & \cellcolor{TealBlue!30}{0.982} & 429.5 & \cellcolor{TealBlue!30}{1} & \cellcolor{TealBlue!30}{369} & \cellcolor{TealBlue!30}{0.982} & \cellcolor{TealBlue!30}{\textbf{8.6}}\\
\texttt{lymph} & \multicolumn{1}{r}{148} & \multicolumn{1}{r}{41}  & \cellcolor{TealBlue!30}{1} & \cellcolor{TealBlue!30}{12} & \cellcolor{TealBlue!30}{0.919} & 0.6 & \cellcolor{TealBlue!30}{1} & \cellcolor{TealBlue!30}{12} & \cellcolor{TealBlue!30}{0.919} & \cellcolor{TealBlue!30}{\textbf{0.0}}\\
\texttt{mushroom} & \multicolumn{1}{r}{8124} & \multicolumn{1}{r}{91}  & \cellcolor{TealBlue!30}{1} & \cellcolor{TealBlue!30}{8} & \cellcolor{TealBlue!30}{0.999} & 5.7 & \cellcolor{TealBlue!30}{1} & \cellcolor{TealBlue!30}{8} & \cellcolor{TealBlue!30}{0.999} & \cellcolor{TealBlue!30}{\textbf{0.5}}\\
\texttt{pendigits} & \multicolumn{1}{r}{7494} & \multicolumn{1}{r}{216}  & \cellcolor{TealBlue!30}{1} & \cellcolor{TealBlue!30}{47} & \cellcolor{TealBlue!30}{0.994} & 124.3 & \cellcolor{TealBlue!30}{1} & \cellcolor{TealBlue!30}{47} & \cellcolor{TealBlue!30}{0.994} & \cellcolor{TealBlue!30}{\textbf{2.9}}\\
\texttt{primary-tumor} & \multicolumn{1}{r}{336} & \multicolumn{1}{r}{16}  & \cellcolor{TealBlue!30}{1} & \cellcolor{TealBlue!30}{46} & \cellcolor{TealBlue!30}{0.863} & 0.1 & \cellcolor{TealBlue!30}{1} & \cellcolor{TealBlue!30}{46} & \cellcolor{TealBlue!30}{0.863} & \cellcolor{TealBlue!30}{\textbf{0.0}}\\
\texttt{segment} & \multicolumn{1}{r}{2310} & \multicolumn{1}{r}{234}  & \cellcolor{TealBlue!30}{1} & \cellcolor{TealBlue!30}{0} & \cellcolor{TealBlue!30}{1.000} & 2.0 & \cellcolor{TealBlue!30}{1} & \cellcolor{TealBlue!30}{0} & \cellcolor{TealBlue!30}{1.000} & \cellcolor{TealBlue!30}{\textbf{0.0}}\\
\texttt{soybean} & \multicolumn{1}{r}{630} & \multicolumn{1}{r}{34}  & \cellcolor{TealBlue!30}{1} & \cellcolor{TealBlue!30}{29} & \cellcolor{TealBlue!30}{0.954} & 0.2 & \cellcolor{TealBlue!30}{1} & \cellcolor{TealBlue!30}{29} & \cellcolor{TealBlue!30}{0.954} & \cellcolor{TealBlue!30}{\textbf{0.0}}\\
\texttt{splice-1} & \multicolumn{1}{r}{3190} & \multicolumn{1}{r}{227}  & \cellcolor{TealBlue!30}{1} & \cellcolor{TealBlue!30}{224} & \cellcolor{TealBlue!30}{0.930} & 104.7 & \cellcolor{TealBlue!30}{1} & \cellcolor{TealBlue!30}{224} & \cellcolor{TealBlue!30}{0.930} & \cellcolor{TealBlue!30}{\textbf{9.1}}\\
\texttt{taiwan\_binarised} & \multicolumn{1}{r}{30000} & \multicolumn{1}{r}{198}  & \cellcolor{TealBlue!30}{1} & \cellcolor{TealBlue!30}{5326} & \cellcolor{TealBlue!30}{0.822} & 473.1 & \cellcolor{TealBlue!30}{1} & \cellcolor{TealBlue!30}{5326} & \cellcolor{TealBlue!30}{0.822} & \cellcolor{TealBlue!30}{\textbf{26.4}}\\
\texttt{tic-tac-toe} & \multicolumn{1}{r}{958} & \multicolumn{1}{r}{18}  & \cellcolor{TealBlue!30}{1} & \cellcolor{TealBlue!30}{216} & \cellcolor{TealBlue!30}{0.775} & 0.1 & \cellcolor{TealBlue!30}{1} & \cellcolor{TealBlue!30}{216} & \cellcolor{TealBlue!30}{0.775} & \cellcolor{TealBlue!30}{\textbf{0.0}}\\
\texttt{vehicle} & \multicolumn{1}{r}{846} & \multicolumn{1}{r}{252}  & \cellcolor{TealBlue!30}{1} & \cellcolor{TealBlue!30}{26} & \cellcolor{TealBlue!30}{0.969} & 56.5 & \cellcolor{TealBlue!30}{1} & \cellcolor{TealBlue!30}{26} & \cellcolor{TealBlue!30}{0.969} & \cellcolor{TealBlue!30}{\textbf{0.7}}\\
\texttt{vote} & \multicolumn{1}{r}{435} & \multicolumn{1}{r}{32}  & \cellcolor{TealBlue!30}{1} & \cellcolor{TealBlue!30}{12} & \cellcolor{TealBlue!30}{0.972} & 0.3 & \cellcolor{TealBlue!30}{1} & \cellcolor{TealBlue!30}{12} & \cellcolor{TealBlue!30}{0.972} & \cellcolor{TealBlue!30}{\textbf{0.0}}\\
\texttt{wine1-un} & \multicolumn{1}{r}{178} & \multicolumn{1}{r}{1276}  & - & - & - & - & \cellcolor{TealBlue!30}{\textbf{1}} & \cellcolor{TealBlue!30}{\textbf{43}} & \cellcolor{TealBlue!30}{\textbf{0.758}} & \cellcolor{TealBlue!30}{\textbf{15.5}}\\
\texttt{wine2-un} & \multicolumn{1}{r}{178} & \multicolumn{1}{r}{1276}  & - & - & - & - & \cellcolor{TealBlue!30}{\textbf{1}} & \cellcolor{TealBlue!30}{\textbf{49}} & \cellcolor{TealBlue!30}{\textbf{0.725}} & \cellcolor{TealBlue!30}{\textbf{15.5}}\\
\texttt{wine3-un} & \multicolumn{1}{r}{178} & \multicolumn{1}{r}{1276}  & - & - & - & - & \cellcolor{TealBlue!30}{\textbf{1}} & \cellcolor{TealBlue!30}{\textbf{33}} & \cellcolor{TealBlue!30}{\textbf{0.815}} & \cellcolor{TealBlue!30}{\textbf{15.6}}\\
\texttt{yeast} & \multicolumn{1}{r}{1484} & \multicolumn{1}{r}{89}  & \cellcolor{TealBlue!30}{1} & \cellcolor{TealBlue!30}{403} & \cellcolor{TealBlue!30}{0.728} & 5.8 & \cellcolor{TealBlue!30}{1} & \cellcolor{TealBlue!30}{403} & \cellcolor{TealBlue!30}{0.728} & \cellcolor{TealBlue!30}{\textbf{0.1}}\\
\texttt{zoo-1} & \multicolumn{1}{r}{101} & \multicolumn{1}{r}{20}  & \cellcolor{TealBlue!30}{1} & \cellcolor{TealBlue!30}{0} & \cellcolor{TealBlue!30}{1.000} & 0.0 & \cellcolor{TealBlue!30}{1} & \cellcolor{TealBlue!30}{0} & \cellcolor{TealBlue!30}{1.000} & \cellcolor{TealBlue!30}{\textbf{0.0}}\\
\bottomrule
\end{tabular}

\end{normalsize}
\end{center}
\caption{\label{tab:d3} Comparison with state of the art for optimal small trees (max depth=3)}
\end{table}

\begin{table}[htbp]
\begin{center}
\begin{normalsize}
\tabcolsep=5pt
\begin{tabular}{lccrrrrrrrr}
\toprule
& && \multicolumn{4}{c}{\dleight} & \multicolumn{4}{c}{\budalg}\\
\cmidrule(rr){4-7}\cmidrule(rr){8-11}
&\multirow{1}{*}{$\#ex.$} & \multirow{1}{*}{\#feat.} &  \multicolumn{1}{c}{opt} & \multicolumn{1}{c}{error} & \multicolumn{1}{c}{acc.} & \multicolumn{1}{c}{time} & \multicolumn{1}{c}{opt} & \multicolumn{1}{c}{error} & \multicolumn{1}{c}{acc.} & \multicolumn{1}{c}{time} \\
\midrule

\texttt{anneal} & \multicolumn{1}{r}{812} & \multicolumn{1}{r}{47}  & \cellcolor{TealBlue!30}{1} & \cellcolor{TealBlue!30}{91} & \cellcolor{TealBlue!30}{0.888} & 91.6 & \cellcolor{TealBlue!30}{1} & \cellcolor{TealBlue!30}{91} & \cellcolor{TealBlue!30}{0.888} & \cellcolor{TealBlue!30}{\textbf{1.5}}\\
\texttt{audiology} & \multicolumn{1}{r}{216} & \multicolumn{1}{r}{79}  & \cellcolor{TealBlue!30}{1} & \cellcolor{TealBlue!30}{1} & \cellcolor{TealBlue!30}{0.995} & 157.5 & \cellcolor{TealBlue!30}{1} & \cellcolor{TealBlue!30}{1} & \cellcolor{TealBlue!30}{0.995} & \cellcolor{TealBlue!30}{\textbf{3.8}}\\
\texttt{australian-credit} & \multicolumn{1}{r}{653} & \multicolumn{1}{r}{73}  & \cellcolor{TealBlue!30}{1} & \cellcolor{TealBlue!30}{56} & \cellcolor{TealBlue!30}{0.914} & 560.2 & \cellcolor{TealBlue!30}{1} & \cellcolor{TealBlue!30}{56} & \cellcolor{TealBlue!30}{0.914} & \cellcolor{TealBlue!30}{\textbf{10.6}}\\
\texttt{breast-cancer-un} & \multicolumn{1}{r}{683} & \multicolumn{1}{r}{89}  & \cellcolor{TealBlue!30}{1} & \cellcolor{TealBlue!30}{16} & \cellcolor{TealBlue!30}{0.977} & 30.0 & \cellcolor{TealBlue!30}{1} & \cellcolor{TealBlue!30}{16} & \cellcolor{TealBlue!30}{0.977} & \cellcolor{TealBlue!30}{\textbf{9.1}}\\
\texttt{breast-wisconsin} & \multicolumn{1}{r}{683} & \multicolumn{1}{r}{120}  & \cellcolor{TealBlue!30}{1} & \cellcolor{TealBlue!30}{7} & \cellcolor{TealBlue!30}{0.990} & 299.7 & \cellcolor{TealBlue!30}{1} & \cellcolor{TealBlue!30}{7} & \cellcolor{TealBlue!30}{0.990} & \cellcolor{TealBlue!30}{\textbf{3.0}}\\
\texttt{car-un} & \multicolumn{1}{r}{1728} & \multicolumn{1}{r}{21}  & \cellcolor{TealBlue!30}{1} & \cellcolor{TealBlue!30}{136} & \cellcolor{TealBlue!30}{0.921} & 0.4 & \cellcolor{TealBlue!30}{1} & \cellcolor{TealBlue!30}{136} & \cellcolor{TealBlue!30}{0.921} & \cellcolor{TealBlue!30}{\textbf{0.1}}\\
\texttt{diabetes} & \multicolumn{1}{r}{768} & \multicolumn{1}{r}{112}  & \cellcolor{TealBlue!30}{1} & \cellcolor{TealBlue!30}{137} & \cellcolor{TealBlue!30}{0.822} & 610.6 & \cellcolor{TealBlue!30}{1} & \cellcolor{TealBlue!30}{137} & \cellcolor{TealBlue!30}{0.822} & \cellcolor{TealBlue!30}{\textbf{5.4}}\\
\texttt{forest-fires-un} & \multicolumn{1}{r}{517} & \multicolumn{1}{r}{989}  & - & - & - & - & \cellcolor{TealBlue!30}{\textbf{0}} & \cellcolor{TealBlue!30}{\textbf{173}} & \cellcolor{TealBlue!30}{\textbf{0.665}} & \cellcolor{TealBlue!30}{\textbf{14.4}}\\
\texttt{german-credit} & \multicolumn{1}{r}{1000} & \multicolumn{1}{r}{110}  & \cellcolor{TealBlue!30}{1} & \cellcolor{TealBlue!30}{204} & \cellcolor{TealBlue!30}{0.796} & 443.6 & \cellcolor{TealBlue!30}{1} & \cellcolor{TealBlue!30}{204} & \cellcolor{TealBlue!30}{0.796} & \cellcolor{TealBlue!30}{\textbf{27.1}}\\
\texttt{heart-cleveland} & \multicolumn{1}{r}{296} & \multicolumn{1}{r}{50}  & \cellcolor{TealBlue!30}{1} & \cellcolor{TealBlue!30}{25} & \cellcolor{TealBlue!30}{0.916} & 174.4 & \cellcolor{TealBlue!30}{1} & \cellcolor{TealBlue!30}{25} & \cellcolor{TealBlue!30}{0.916} & \cellcolor{TealBlue!30}{\textbf{3.0}}\\
\texttt{hepatitis} & \multicolumn{1}{r}{137} & \multicolumn{1}{r}{68}  & \cellcolor{TealBlue!30}{1} & \cellcolor{TealBlue!30}{3} & \cellcolor{TealBlue!30}{0.978} & 22.8 & \cellcolor{TealBlue!30}{1} & \cellcolor{TealBlue!30}{3} & \cellcolor{TealBlue!30}{0.978} & \cellcolor{TealBlue!30}{\textbf{0.3}}\\
\texttt{hypothyroid} & \multicolumn{1}{r}{3247} & \multicolumn{1}{r}{43}  & \cellcolor{TealBlue!30}{1} & \cellcolor{TealBlue!30}{53} & \cellcolor{TealBlue!30}{0.984} & 170.9 & \cellcolor{TealBlue!30}{1} & \cellcolor{TealBlue!30}{53} & \cellcolor{TealBlue!30}{0.984} & \cellcolor{TealBlue!30}{\textbf{3.7}}\\
\texttt{ionosphere} & \multicolumn{1}{r}{351} & \multicolumn{1}{r}{444}  & - & - & - & - & \cellcolor{TealBlue!30}{\textbf{1}} & \cellcolor{TealBlue!30}{\textbf{7}} & \cellcolor{TealBlue!30}{\textbf{0.980}} & \cellcolor{TealBlue!30}{\textbf{818.0}}\\
\texttt{kr-vs-kp} & \multicolumn{1}{r}{3196} & \multicolumn{1}{r}{37}  & \cellcolor{TealBlue!30}{1} & \cellcolor{TealBlue!30}{144} & \cellcolor{TealBlue!30}{0.955} & 89.7 & \cellcolor{TealBlue!30}{1} & \cellcolor{TealBlue!30}{144} & \cellcolor{TealBlue!30}{0.955} & \cellcolor{TealBlue!30}{\textbf{2.2}}\\
\texttt{letter} & \multicolumn{1}{r}{20000} & \multicolumn{1}{r}{224}  & 0 & 335 & 0.983 & 3600.0 & \cellcolor{TealBlue!30}{\textbf{1}} & \cellcolor{TealBlue!30}{\textbf{261}} & \cellcolor{TealBlue!30}{\textbf{0.987}} & \cellcolor{TealBlue!30}{\textbf{944.0}}\\
\texttt{lymph} & \multicolumn{1}{r}{148} & \multicolumn{1}{r}{41}  & \cellcolor{TealBlue!30}{1} & \cellcolor{TealBlue!30}{3} & \cellcolor{TealBlue!30}{0.980} & 12.8 & \cellcolor{TealBlue!30}{1} & \cellcolor{TealBlue!30}{3} & \cellcolor{TealBlue!30}{0.980} & \cellcolor{TealBlue!30}{\textbf{0.6}}\\
\texttt{mushroom} & \multicolumn{1}{r}{8124} & \multicolumn{1}{r}{91}  & \cellcolor{TealBlue!30}{1} & \cellcolor{TealBlue!30}{0} & \cellcolor{TealBlue!30}{1.000} & 42.3 & \cellcolor{TealBlue!30}{1} & \cellcolor{TealBlue!30}{0} & \cellcolor{TealBlue!30}{1.000} & \cellcolor{TealBlue!30}{\textbf{0.0}}\\
\texttt{pendigits} & \multicolumn{1}{r}{7494} & \multicolumn{1}{r}{216}  & - & - & - & - & \cellcolor{TealBlue!30}{\textbf{1}} & \cellcolor{TealBlue!30}{\textbf{13}} & \cellcolor{TealBlue!30}{\textbf{0.998}} & \cellcolor{TealBlue!30}{\textbf{237.0}}\\
\texttt{primary-tumor} & \multicolumn{1}{r}{336} & \multicolumn{1}{r}{16}  & \cellcolor{TealBlue!30}{1} & \cellcolor{TealBlue!30}{34} & \cellcolor{TealBlue!30}{0.899} & 2.0 & \cellcolor{TealBlue!30}{1} & \cellcolor{TealBlue!30}{34} & \cellcolor{TealBlue!30}{0.899} & \cellcolor{TealBlue!30}{\textbf{0.0}}\\
\texttt{segment} & \multicolumn{1}{r}{2310} & \multicolumn{1}{r}{234}  & \cellcolor{TealBlue!30}{1} & \cellcolor{TealBlue!30}{0} & \cellcolor{TealBlue!30}{1.000} & 1.5 & \cellcolor{TealBlue!30}{1} & \cellcolor{TealBlue!30}{0} & \cellcolor{TealBlue!30}{1.000} & \cellcolor{TealBlue!30}{\textbf{0.0}}\\
\texttt{soybean} & \multicolumn{1}{r}{630} & \multicolumn{1}{r}{34}  & \cellcolor{TealBlue!30}{1} & \cellcolor{TealBlue!30}{14} & \cellcolor{TealBlue!30}{0.978} & 4.2 & \cellcolor{TealBlue!30}{1} & \cellcolor{TealBlue!30}{14} & \cellcolor{TealBlue!30}{0.978} & \cellcolor{TealBlue!30}{\textbf{0.8}}\\
\texttt{splice-1} & \multicolumn{1}{r}{3190} & \multicolumn{1}{r}{227}  & - & - & - & - & \cellcolor{TealBlue!30}{\textbf{1}} & \cellcolor{TealBlue!30}{\textbf{141}} & \cellcolor{TealBlue!30}{\textbf{0.956}} & \cellcolor{TealBlue!30}{\textbf{3180.0}}\\
\texttt{taiwan\_binarised} & \multicolumn{1}{r}{30000} & \multicolumn{1}{r}{198}  & \cellcolor{TealBlue!30}{0} & 5307 & 0.823 & 3600.0 & \cellcolor{TealBlue!30}{0} & \cellcolor{TealBlue!30}{\textbf{5273}} & \cellcolor{TealBlue!30}{\textbf{0.824}} & \cellcolor{TealBlue!30}{\textbf{5.1}}\\
\texttt{tic-tac-toe} & \multicolumn{1}{r}{958} & \multicolumn{1}{r}{18}  & \cellcolor{TealBlue!30}{1} & \cellcolor{TealBlue!30}{137} & \cellcolor{TealBlue!30}{0.857} & 1.4 & \cellcolor{TealBlue!30}{1} & \cellcolor{TealBlue!30}{137} & \cellcolor{TealBlue!30}{0.857} & \cellcolor{TealBlue!30}{\textbf{0.4}}\\
\texttt{vehicle} & \multicolumn{1}{r}{846} & \multicolumn{1}{r}{252}  & - & - & - & - & \cellcolor{TealBlue!30}{\textbf{1}} & \cellcolor{TealBlue!30}{\textbf{12}} & \cellcolor{TealBlue!30}{\textbf{0.986}} & \cellcolor{TealBlue!30}{\textbf{74.5}}\\
\texttt{vote} & \multicolumn{1}{r}{435} & \multicolumn{1}{r}{32}  & \cellcolor{TealBlue!30}{1} & \cellcolor{TealBlue!30}{5} & \cellcolor{TealBlue!30}{0.989} & 6.2 & \cellcolor{TealBlue!30}{1} & \cellcolor{TealBlue!30}{5} & \cellcolor{TealBlue!30}{0.989} & \cellcolor{TealBlue!30}{\textbf{1.3}}\\
\texttt{wine1-un} & \multicolumn{1}{r}{178} & \multicolumn{1}{r}{1276}  & - & - & - & - & \cellcolor{TealBlue!30}{\textbf{0}} & \cellcolor{TealBlue!30}{\textbf{37}} & \cellcolor{TealBlue!30}{\textbf{0.792}} & \cellcolor{TealBlue!30}{\textbf{1570.0}}\\
\texttt{wine2-un} & \multicolumn{1}{r}{178} & \multicolumn{1}{r}{1276}  & - & - & - & - & \cellcolor{TealBlue!30}{\textbf{0}} & \cellcolor{TealBlue!30}{\textbf{43}} & \cellcolor{TealBlue!30}{\textbf{0.758}} & \cellcolor{TealBlue!30}{\textbf{15.5}}\\
\texttt{wine3-un} & \multicolumn{1}{r}{178} & \multicolumn{1}{r}{1276}  & - & - & - & - & \cellcolor{TealBlue!30}{\textbf{0}} & \cellcolor{TealBlue!30}{\textbf{28}} & \cellcolor{TealBlue!30}{\textbf{0.843}} & \cellcolor{TealBlue!30}{\textbf{31.2}}\\
\texttt{yeast} & \multicolumn{1}{r}{1484} & \multicolumn{1}{r}{89}  & \cellcolor{TealBlue!30}{1} & \cellcolor{TealBlue!30}{366} & \cellcolor{TealBlue!30}{0.753} & 264.5 & \cellcolor{TealBlue!30}{1} & \cellcolor{TealBlue!30}{366} & \cellcolor{TealBlue!30}{0.753} & \cellcolor{TealBlue!30}{\textbf{3.3}}\\
\texttt{zoo-1} & \multicolumn{1}{r}{101} & \multicolumn{1}{r}{20}  & \cellcolor{TealBlue!30}{1} & \cellcolor{TealBlue!30}{0} & \cellcolor{TealBlue!30}{1.000} & 0.0 & \cellcolor{TealBlue!30}{1} & \cellcolor{TealBlue!30}{0} & \cellcolor{TealBlue!30}{1.000} & \cellcolor{TealBlue!30}{\textbf{0.0}}\\
\bottomrule
\end{tabular}

\end{normalsize}
\end{center}
\caption{\label{tab:d4} Comparison with state of the art for optimal small trees (max depth=4)}
\end{table}

\begin{table}[htbp]
\begin{center}
\begin{normalsize}
\tabcolsep=5pt
\begin{tabular}{lccrrrrrrrrrr}
\toprule
& && \multicolumn{5}{c}{\budalg} & \multicolumn{5}{c}{\dleight}\\
\cmidrule(rr){4-8}\cmidrule(rr){9-13}
&\multirow{1}{*}{$\#ex.$} & \multirow{1}{*}{\#feat.} &  \multicolumn{1}{c}{opt} & \multicolumn{1}{c}{error} & \multicolumn{1}{c}{acc.} & \multicolumn{1}{c}{time} & \multicolumn{1}{c}{choices} & \multicolumn{1}{c}{opt} & \multicolumn{1}{c}{error} & \multicolumn{1}{c}{acc.} & \multicolumn{1}{c}{time} & \multicolumn{1}{c}{choices} \\
\midrule

\texttt{anneal} & \multicolumn{1}{r}{812} & \multicolumn{1}{r}{47}  & \cellcolor{TealBlue!30}{\textbf{1}} & \cellcolor{TealBlue!30}{\textbf{70}} & \cellcolor{TealBlue!30}{\textbf{0.914}} & \cellcolor{TealBlue!30}{\textbf{774.0}} & \cellcolor{TealBlue!30}{\textbf{164{\sc m}}} & - & - & - & - & -\\
\texttt{audiology} & \multicolumn{1}{r}{216} & \multicolumn{1}{r}{79}  & \cellcolor{TealBlue!30}{1} & \cellcolor{TealBlue!30}{0} & \cellcolor{TealBlue!30}{1.000} & \cellcolor{TealBlue!30}{\textbf{0.0}} & \cellcolor{TealBlue!30}{\textbf{508}} & \cellcolor{TealBlue!30}{1} & \cellcolor{TealBlue!30}{0} & \cellcolor{TealBlue!30}{1.000} & 0.0 & 21{\sc k}\\
\texttt{australian-credit} & \multicolumn{1}{r}{653} & \multicolumn{1}{r}{73}  & \cellcolor{TealBlue!30}{\textbf{0}} & \cellcolor{TealBlue!30}{\textbf{39}} & \cellcolor{TealBlue!30}{\textbf{0.940}} & \cellcolor{TealBlue!30}{\textbf{3320.0}} & \cellcolor{TealBlue!30}{\textbf{652{\sc m}}} & - & - & - & - & -\\
\texttt{breast-cancer-un} & \multicolumn{1}{r}{683} & \multicolumn{1}{r}{89}  & \cellcolor{TealBlue!30}{1} & \cellcolor{TealBlue!30}{6} & \cellcolor{TealBlue!30}{0.991} & 690.0 & 153{\sc m} & \cellcolor{TealBlue!30}{1} & \cellcolor{TealBlue!30}{6} & \cellcolor{TealBlue!30}{0.991} & \cellcolor{TealBlue!30}{\textbf{580.5}} & \cellcolor{TealBlue!30}{\textbf{52{\sc m}}}\\
\texttt{breast-wisconsin} & \multicolumn{1}{r}{683} & \multicolumn{1}{r}{120}  & \cellcolor{TealBlue!30}{\textbf{1}} & \cellcolor{TealBlue!30}{\textbf{0}} & \cellcolor{TealBlue!30}{\textbf{1.000}} & \cellcolor{TealBlue!30}{\textbf{302.0}} & \cellcolor{TealBlue!30}{\textbf{72{\sc m}}} & - & - & - & - & -\\
\texttt{car-un} & \multicolumn{1}{r}{1728} & \multicolumn{1}{r}{21}  & \cellcolor{TealBlue!30}{1} & \cellcolor{TealBlue!30}{86} & \cellcolor{TealBlue!30}{0.950} & 3.3 & 864{\sc k} & \cellcolor{TealBlue!30}{1} & \cellcolor{TealBlue!30}{86} & \cellcolor{TealBlue!30}{0.950} & \cellcolor{TealBlue!30}{\textbf{2.3}} & \cellcolor{TealBlue!30}{\textbf{215{\sc k}}}\\
\texttt{diabetes} & \multicolumn{1}{r}{768} & \multicolumn{1}{r}{112}  & \cellcolor{TealBlue!30}{\textbf{0}} & \cellcolor{TealBlue!30}{\textbf{106}} & \cellcolor{TealBlue!30}{\textbf{0.862}} & \cellcolor{TealBlue!30}{\textbf{2250.0}} & \cellcolor{TealBlue!30}{\textbf{447{\sc m}}} & - & - & - & - & -\\
\texttt{forest-fires-un} & \multicolumn{1}{r}{517} & \multicolumn{1}{r}{989}  & \cellcolor{TealBlue!30}{\textbf{0}} & \cellcolor{TealBlue!30}{\textbf{157}} & \cellcolor{TealBlue!30}{\textbf{0.696}} & \cellcolor{TealBlue!30}{\textbf{405.0}} & \cellcolor{TealBlue!30}{\textbf{48{\sc m}}} & - & - & - & - & -\\
\texttt{german-credit} & \multicolumn{1}{r}{1000} & \multicolumn{1}{r}{110}  & \cellcolor{TealBlue!30}{\textbf{0}} & \cellcolor{TealBlue!30}{\textbf{164}} & \cellcolor{TealBlue!30}{\textbf{0.836}} & \cellcolor{TealBlue!30}{\textbf{3140.0}} & \cellcolor{TealBlue!30}{\textbf{565{\sc m}}} & - & - & - & - & -\\
\texttt{heart-cleveland} & \multicolumn{1}{r}{296} & \multicolumn{1}{r}{50}  & \cellcolor{TealBlue!30}{\textbf{1}} & \cellcolor{TealBlue!30}{\textbf{7}} & \cellcolor{TealBlue!30}{\textbf{0.976}} & \cellcolor{TealBlue!30}{\textbf{946.0}} & \cellcolor{TealBlue!30}{\textbf{267{\sc m}}} & - & - & - & - & -\\
\texttt{hepatitis} & \multicolumn{1}{r}{137} & \multicolumn{1}{r}{68}  & \cellcolor{TealBlue!30}{1} & \cellcolor{TealBlue!30}{0} & \cellcolor{TealBlue!30}{1.000} & \cellcolor{TealBlue!30}{\textbf{1.1}} & \cellcolor{TealBlue!30}{\textbf{472{\sc k}}} & \cellcolor{TealBlue!30}{1} & \cellcolor{TealBlue!30}{0} & \cellcolor{TealBlue!30}{1.000} & 60.7 & 14{\sc m}\\
\texttt{hypothyroid} & \multicolumn{1}{r}{3247} & \multicolumn{1}{r}{43}  & \cellcolor{TealBlue!30}{\textbf{1}} & \cellcolor{TealBlue!30}{\textbf{44}} & \cellcolor{TealBlue!30}{\textbf{0.986}} & \cellcolor{TealBlue!30}{\textbf{2670.0}} & \cellcolor{TealBlue!30}{\textbf{187{\sc m}}} & - & - & - & - & -\\
\texttt{ionosphere} & \multicolumn{1}{r}{351} & \multicolumn{1}{r}{444}  & \cellcolor{TealBlue!30}{\textbf{0}} & \cellcolor{TealBlue!30}{\textbf{2}} & \cellcolor{TealBlue!30}{\textbf{0.994}} & \cellcolor{TealBlue!30}{\textbf{494.0}} & \cellcolor{TealBlue!30}{\textbf{27{\sc m}}} & - & - & - & - & -\\
\texttt{kr-vs-kp} & \multicolumn{1}{r}{3196} & \multicolumn{1}{r}{37}  & \cellcolor{TealBlue!30}{\textbf{1}} & \cellcolor{TealBlue!30}{\textbf{81}} & \cellcolor{TealBlue!30}{\textbf{0.975}} & \cellcolor{TealBlue!30}{\textbf{1280.0}} & \cellcolor{TealBlue!30}{\textbf{113{\sc m}}} & - & - & - & - & -\\
\texttt{letter} & \multicolumn{1}{r}{20000} & \multicolumn{1}{r}{224}  & \cellcolor{TealBlue!30}{0} & \cellcolor{TealBlue!30}{\textbf{251}} & \cellcolor{TealBlue!30}{\textbf{0.987}} & \cellcolor{TealBlue!30}{\textbf{2560.0}} & \cellcolor{TealBlue!30}{\textbf{15{\sc m}}} & \cellcolor{TealBlue!30}{0} & 352 & 0.982 & 3600.0 & 94{\sc m}\\
\texttt{lymph} & \multicolumn{1}{r}{148} & \multicolumn{1}{r}{41}  & \cellcolor{TealBlue!30}{1} & \cellcolor{TealBlue!30}{0} & \cellcolor{TealBlue!30}{1.000} & \cellcolor{TealBlue!30}{\textbf{0.0}} & \cellcolor{TealBlue!30}{\textbf{18{\sc k}}} & \cellcolor{TealBlue!30}{1} & \cellcolor{TealBlue!30}{0} & \cellcolor{TealBlue!30}{1.000} & 12.1 & 2615{\sc k}\\
\texttt{mushroom} & \multicolumn{1}{r}{8124} & \multicolumn{1}{r}{91}  & \cellcolor{TealBlue!30}{1} & \cellcolor{TealBlue!30}{0} & \cellcolor{TealBlue!30}{1.000} & \cellcolor{TealBlue!30}{\textbf{0.0}} & \cellcolor{TealBlue!30}{\textbf{278}} & \cellcolor{TealBlue!30}{1} & \cellcolor{TealBlue!30}{0} & \cellcolor{TealBlue!30}{1.000} & 36.4 & 1900{\sc k}\\
\texttt{pendigits} & \multicolumn{1}{r}{7494} & \multicolumn{1}{r}{216}  & \cellcolor{TealBlue!30}{\textbf{0}} & \cellcolor{TealBlue!30}{\textbf{2}} & \cellcolor{TealBlue!30}{\textbf{1.000}} & \cellcolor{TealBlue!30}{\textbf{1940.0}} & \cellcolor{TealBlue!30}{\textbf{25{\sc m}}} & - & - & - & - & -\\
\texttt{primary-tumor} & \multicolumn{1}{r}{336} & \multicolumn{1}{r}{16}  & \cellcolor{TealBlue!30}{1} & \cellcolor{TealBlue!30}{26} & \cellcolor{TealBlue!30}{0.923} & \cellcolor{TealBlue!30}{\textbf{6.6}} & 4418{\sc k} & \cellcolor{TealBlue!30}{1} & \cellcolor{TealBlue!30}{26} & \cellcolor{TealBlue!30}{0.923} & 20.3 & \cellcolor{TealBlue!30}{\textbf{2023{\sc k}}}\\
\texttt{segment} & \multicolumn{1}{r}{2310} & \multicolumn{1}{r}{234}  & \cellcolor{TealBlue!30}{1} & \cellcolor{TealBlue!30}{0} & \cellcolor{TealBlue!30}{1.000} & \cellcolor{TealBlue!30}{\textbf{0.0}} & \cellcolor{TealBlue!30}{\textbf{502}} & \cellcolor{TealBlue!30}{1} & \cellcolor{TealBlue!30}{0} & \cellcolor{TealBlue!30}{1.000} & 1.0 & 220{\sc k}\\
\texttt{soybean} & \multicolumn{1}{r}{630} & \multicolumn{1}{r}{34}  & \cellcolor{TealBlue!30}{1} & \cellcolor{TealBlue!30}{8} & \cellcolor{TealBlue!30}{0.987} & \cellcolor{TealBlue!30}{\textbf{44.8}} & 15{\sc m} & \cellcolor{TealBlue!30}{1} & \cellcolor{TealBlue!30}{8} & \cellcolor{TealBlue!30}{0.987} & 60.0 & \cellcolor{TealBlue!30}{\textbf{7368{\sc k}}}\\
\texttt{splice-1} & \multicolumn{1}{r}{3190} & \multicolumn{1}{r}{227}  & \cellcolor{TealBlue!30}{\textbf{0}} & \cellcolor{TealBlue!30}{\textbf{101}} & \cellcolor{TealBlue!30}{\textbf{0.968}} & \cellcolor{TealBlue!30}{\textbf{1410.0}} & \cellcolor{TealBlue!30}{\textbf{105{\sc m}}} & - & - & - & - & -\\
\texttt{taiwan\_binarised} & \multicolumn{1}{r}{30000} & \multicolumn{1}{r}{198}  & \cellcolor{TealBlue!30}{0} & \cellcolor{TealBlue!30}{\textbf{5207}} & \cellcolor{TealBlue!30}{\textbf{0.826}} & \cellcolor{TealBlue!30}{\textbf{2790.0}} & \cellcolor{TealBlue!30}{\textbf{16{\sc m}}} & \cellcolor{TealBlue!30}{0} & 5412 & 0.820 & 3600.0 & 54{\sc m}\\
\texttt{tic-tac-toe} & \multicolumn{1}{r}{958} & \multicolumn{1}{r}{18}  & \cellcolor{TealBlue!30}{1} & \cellcolor{TealBlue!30}{63} & \cellcolor{TealBlue!30}{0.934} & \cellcolor{TealBlue!30}{\textbf{8.4}} & 3635{\sc k} & \cellcolor{TealBlue!30}{1} & \cellcolor{TealBlue!30}{63} & \cellcolor{TealBlue!30}{0.934} & 11.7 & \cellcolor{TealBlue!30}{\textbf{1118{\sc k}}}\\
\texttt{vehicle} & \multicolumn{1}{r}{846} & \multicolumn{1}{r}{252}  & \cellcolor{TealBlue!30}{\textbf{0}} & \cellcolor{TealBlue!30}{\textbf{5}} & \cellcolor{TealBlue!30}{\textbf{0.994}} & \cellcolor{TealBlue!30}{\textbf{3370.0}} & \cellcolor{TealBlue!30}{\textbf{266{\sc m}}} & - & - & - & - & -\\
\texttt{vote} & \multicolumn{1}{r}{435} & \multicolumn{1}{r}{32}  & \cellcolor{TealBlue!30}{1} & \cellcolor{TealBlue!30}{1} & \cellcolor{TealBlue!30}{0.998} & \cellcolor{TealBlue!30}{\textbf{21.3}} & 8910{\sc k} & \cellcolor{TealBlue!30}{1} & \cellcolor{TealBlue!30}{1} & \cellcolor{TealBlue!30}{0.998} & 40.7 & \cellcolor{TealBlue!30}{\textbf{7873{\sc k}}}\\
\texttt{wine1-un} & \multicolumn{1}{r}{178} & \multicolumn{1}{r}{1276}  & \cellcolor{TealBlue!30}{\textbf{0}} & \cellcolor{TealBlue!30}{\textbf{34}} & \cellcolor{TealBlue!30}{\textbf{0.809}} & \cellcolor{TealBlue!30}{\textbf{498.0}} & \cellcolor{TealBlue!30}{\textbf{25{\sc m}}} & - & - & - & - & -\\
\texttt{wine2-un} & \multicolumn{1}{r}{178} & \multicolumn{1}{r}{1276}  & \cellcolor{TealBlue!30}{\textbf{0}} & \cellcolor{TealBlue!30}{\textbf{37}} & \cellcolor{TealBlue!30}{\textbf{0.792}} & \cellcolor{TealBlue!30}{\textbf{58.2}} & \cellcolor{TealBlue!30}{\textbf{2891{\sc k}}} & - & - & - & - & -\\
\texttt{wine3-un} & \multicolumn{1}{r}{178} & \multicolumn{1}{r}{1276}  & \cellcolor{TealBlue!30}{\textbf{0}} & \cellcolor{TealBlue!30}{\textbf{25}} & \cellcolor{TealBlue!30}{\textbf{0.860}} & \cellcolor{TealBlue!30}{\textbf{3500.0}} & \cellcolor{TealBlue!30}{\textbf{178{\sc m}}} & - & - & - & - & -\\
\texttt{yeast} & \multicolumn{1}{r}{1484} & \multicolumn{1}{r}{89}  & \cellcolor{TealBlue!30}{\textbf{1}} & \cellcolor{TealBlue!30}{\textbf{313}} & \cellcolor{TealBlue!30}{\textbf{0.789}} & \cellcolor{TealBlue!30}{\textbf{2580.0}} & \cellcolor{TealBlue!30}{\textbf{433{\sc m}}} & - & - & - & - & -\\
\texttt{zoo-1} & \multicolumn{1}{r}{101} & \multicolumn{1}{r}{20}  & \cellcolor{TealBlue!30}{1} & \cellcolor{TealBlue!30}{0} & \cellcolor{TealBlue!30}{1.000} & \cellcolor{TealBlue!30}{\textbf{0.0}} & \cellcolor{TealBlue!30}{\textbf{1}} & \cellcolor{TealBlue!30}{1} & \cellcolor{TealBlue!30}{0} & \cellcolor{TealBlue!30}{1.000} & 0.0 & 13\\
\bottomrule
\end{tabular}

\end{normalsize}
\end{center}
\caption{\label{tab:d5} Comparison with state of the art for optimal small trees (max depth=5)}
\end{table}



\subsection*{Comparison with \cart}

Next, we compare our algorithm (\budalg) against \cart (its implemention in scikit-learn) on deeper trees (4 and 7 and 10). Here we give the accuracy of the first solution found by \budalg (\greedy) and the best solution found given a 3 seconds time limit (\budalg) in tables \ref{tab:d3}, \ref{tab:d4} and \ref{tab:d5}.

\medskip

We can see that the first solution found by our algorithm has comparable accuracy (although slightly worse in average) to the one found by \cart. Moreover, although we would need larger data sets to be confident about that, it seems that our algorithm is faster than \cart to find the first tree. One cn conjecture that \cart uses more sophisticated heuristic choices to explain these two observations.

Then, in most cases, it is possible to improve significantly within a few seconds. Notice that for larger depth, improving the initil solution is harder and the 3s time limit is comparatively tighter than for smaller trees, so the gain of \budalg over \cart is more sensible for small trees.





\begin{table}[htbp]
\begin{center}
\begin{normalsize}
\tabcolsep=5pt
\begin{tabular}{lccrrrrrrrrr}
\toprule
& && \multicolumn{3}{c}{\cart} & \multicolumn{3}{c}{\greedy} & \multicolumn{3}{c}{\budalg}\\
\cmidrule(rr){4-6}\cmidrule(rr){7-9}\cmidrule(rr){10-12}
&\multirow{1}{*}{$\#ex.$} & \multirow{1}{*}{\#feat.} &  \multicolumn{1}{c}{acc.} & \multicolumn{1}{c}{error} & \multicolumn{1}{c}{time} & \multicolumn{1}{c}{acc.} & \multicolumn{1}{c}{error} & \multicolumn{1}{c}{time} & \multicolumn{1}{c}{acc.} & \multicolumn{1}{c}{error} & \multicolumn{1}{c}{time} \\
\midrule

\texttt{anneal} & \multicolumn{1}{r}{812} & \multicolumn{1}{r}{47}  & 0.834 & 135 & 0.0 & 0.830 & 138 & \cellcolor{TealBlue!30}{\textbf{0.0}} & \cellcolor{TealBlue!30}{\textbf{0.879}} & \cellcolor{TealBlue!30}{\textbf{98}} & 3.0\\
\texttt{audiology} & \multicolumn{1}{r}{216} & \multicolumn{1}{r}{79}  & 0.986 & 3 & 0.0 & 0.986 & 3 & \cellcolor{TealBlue!30}{\textbf{0.0}} & \cellcolor{TealBlue!30}{\textbf{0.995}} & \cellcolor{TealBlue!30}{\textbf{1}} & 3.0\\
\texttt{australian-credit} & \multicolumn{1}{r}{653} & \multicolumn{1}{r}{73}  & 0.887 & 74 & 0.0 & 0.885 & 75 & \cellcolor{TealBlue!30}{\textbf{0.0}} & \cellcolor{TealBlue!30}{\textbf{0.908}} & \cellcolor{TealBlue!30}{\textbf{60}} & 3.0\\
\texttt{breast-cancer-un} & \multicolumn{1}{r}{683} & \multicolumn{1}{r}{89}  & 0.969 & 21 & 0.0 & 0.958 & 29 & \cellcolor{TealBlue!30}{\textbf{0.0}} & \cellcolor{TealBlue!30}{\textbf{0.977}} & \cellcolor{TealBlue!30}{\textbf{16}} & 3.0\\
\texttt{breast-wisconsin} & \multicolumn{1}{r}{683} & \multicolumn{1}{r}{120}  & 0.977 & 16 & 0.0 & 0.974 & 18 & \cellcolor{TealBlue!30}{\textbf{0.0}} & \cellcolor{TealBlue!30}{\textbf{0.988}} & \cellcolor{TealBlue!30}{\textbf{8}} & 3.0\\
\texttt{car-un} & \multicolumn{1}{r}{1728} & \multicolumn{1}{r}{21}  & 0.883 & 202 & 0.0 & 0.897 & 178 & \cellcolor{TealBlue!30}{\textbf{0.0}} & \cellcolor{TealBlue!30}{\textbf{0.921}} & \cellcolor{TealBlue!30}{\textbf{136}} & 0.2\\
\texttt{diabetes} & \multicolumn{1}{r}{768} & \multicolumn{1}{r}{112}  & 0.784 & 166 & 0.0 & 0.790 & 161 & \cellcolor{TealBlue!30}{\textbf{0.0}} & \cellcolor{TealBlue!30}{\textbf{0.822}} & \cellcolor{TealBlue!30}{\textbf{137}} & 3.1\\
\texttt{forest-fires-un} & \multicolumn{1}{r}{517} & \multicolumn{1}{r}{989}  & \cellcolor{TealBlue!30}{\textbf{0.640}} & \cellcolor{TealBlue!30}{\textbf{186}} & 0.0 & 0.615 & 199 & \cellcolor{TealBlue!30}{\textbf{0.0}} & 0.627 & 193 & 3.1\\
\texttt{german-credit} & \multicolumn{1}{r}{1000} & \multicolumn{1}{r}{110}  & 0.769 & 231 & 0.0 & 0.775 & 225 & \cellcolor{TealBlue!30}{\textbf{0.0}} & \cellcolor{TealBlue!30}{\textbf{0.796}} & \cellcolor{TealBlue!30}{\textbf{204}} & 3.0\\
\texttt{heart-cleveland} & \multicolumn{1}{r}{296} & \multicolumn{1}{r}{50}  & 0.872 & 38 & 0.0 & 0.868 & 39 & \cellcolor{TealBlue!30}{\textbf{0.0}} & \cellcolor{TealBlue!30}{\textbf{0.912}} & \cellcolor{TealBlue!30}{\textbf{26}} & 3.0\\
\texttt{hepatitis} & \multicolumn{1}{r}{137} & \multicolumn{1}{r}{68}  & 0.920 & 11 & 0.0 & 0.905 & 13 & \cellcolor{TealBlue!30}{\textbf{0.0}} & \cellcolor{TealBlue!30}{\textbf{0.978}} & \cellcolor{TealBlue!30}{\textbf{3}} & 2.3\\
\texttt{hypothyroid} & \multicolumn{1}{r}{3247} & \multicolumn{1}{r}{43}  & \cellcolor{TealBlue!30}{0.984} & \cellcolor{TealBlue!30}{53} & 0.0 & 0.983 & 54 & \cellcolor{TealBlue!30}{\textbf{0.0}} & \cellcolor{TealBlue!30}{0.984} & \cellcolor{TealBlue!30}{53} & 3.2\\
\texttt{ionosphere} & \multicolumn{1}{r}{351} & \multicolumn{1}{r}{444}  & 0.923 & 27 & 0.0 & 0.937 & 22 & \cellcolor{TealBlue!30}{\textbf{0.0}} & \cellcolor{TealBlue!30}{\textbf{0.963}} & \cellcolor{TealBlue!30}{\textbf{13}} & 3.2\\
\texttt{kr-vs-kp} & \multicolumn{1}{r}{3196} & \multicolumn{1}{r}{37}  & 0.941 & 189 & 0.0 & 0.941 & 189 & \cellcolor{TealBlue!30}{\textbf{0.0}} & \cellcolor{TealBlue!30}{\textbf{0.952}} & \cellcolor{TealBlue!30}{\textbf{154}} & 3.1\\
\texttt{letter} & \multicolumn{1}{r}{20000} & \multicolumn{1}{r}{224}  & \cellcolor{TealBlue!30}{\textbf{0.977}} & \cellcolor{TealBlue!30}{\textbf{462}} & 0.3 & 0.963 & 732 & \cellcolor{TealBlue!30}{\textbf{0.0}} & 0.964 & 724 & 3.4\\
\texttt{lymph} & \multicolumn{1}{r}{148} & \multicolumn{1}{r}{41}  & 0.932 & 10 & 0.0 & 0.939 & 9 & \cellcolor{TealBlue!30}{\textbf{0.0}} & \cellcolor{TealBlue!30}{\textbf{0.980}} & \cellcolor{TealBlue!30}{\textbf{3}} & 1.8\\
\texttt{mushroom} & \multicolumn{1}{r}{8124} & \multicolumn{1}{r}{91}  & 1.000 & 4 & 0.0 & 1.000 & 4 & \cellcolor{TealBlue!30}{\textbf{0.0}} & \cellcolor{TealBlue!30}{\textbf{1.000}} & \cellcolor{TealBlue!30}{\textbf{0}} & 0.0\\
\texttt{pendigits} & \multicolumn{1}{r}{7494} & \multicolumn{1}{r}{216}  & \cellcolor{TealBlue!30}{\textbf{0.997}} & \cellcolor{TealBlue!30}{\textbf{25}} & 0.1 & 0.996 & 27 & \cellcolor{TealBlue!30}{\textbf{0.0}} & 0.996 & 27 & 3.5\\
\texttt{primary-tumor} & \multicolumn{1}{r}{336} & \multicolumn{1}{r}{16}  & 0.869 & 44 & 0.0 & 0.866 & 45 & \cellcolor{TealBlue!30}{\textbf{0.0}} & \cellcolor{TealBlue!30}{\textbf{0.899}} & \cellcolor{TealBlue!30}{\textbf{34}} & 0.3\\
\texttt{segment} & \multicolumn{1}{r}{2310} & \multicolumn{1}{r}{234}  & 1.000 & 1 & 0.0 & 1.000 & 1 & \cellcolor{TealBlue!30}{\textbf{0.0}} & \cellcolor{TealBlue!30}{\textbf{1.000}} & \cellcolor{TealBlue!30}{\textbf{0}} & 0.0\\
\texttt{soybean} & \multicolumn{1}{r}{630} & \multicolumn{1}{r}{34}  & 0.949 & 32 & 0.0 & 0.887 & 71 & \cellcolor{TealBlue!30}{\textbf{0.0}} & \cellcolor{TealBlue!30}{\textbf{0.978}} & \cellcolor{TealBlue!30}{\textbf{14}} & 1.2\\
\texttt{splice-1} & \multicolumn{1}{r}{3190} & \multicolumn{1}{r}{227}  & \cellcolor{TealBlue!30}{0.956} & \cellcolor{TealBlue!30}{141} & 0.0 & \cellcolor{TealBlue!30}{0.956} & \cellcolor{TealBlue!30}{141} & \cellcolor{TealBlue!30}{\textbf{0.0}} & \cellcolor{TealBlue!30}{0.956} & \cellcolor{TealBlue!30}{141} & 3.2\\
\texttt{taiwan\_binarised} & \multicolumn{1}{r}{30000} & \multicolumn{1}{r}{198}  & 0.823 & 5306 & 0.4 & 0.823 & 5305 & \cellcolor{TealBlue!30}{\textbf{0.0}} & \cellcolor{TealBlue!30}{\textbf{0.824}} & \cellcolor{TealBlue!30}{\textbf{5284}} & 3.4\\
\texttt{tic-tac-toe} & \multicolumn{1}{r}{958} & \multicolumn{1}{r}{18}  & 0.843 & 150 & 0.0 & 0.843 & 150 & \cellcolor{TealBlue!30}{\textbf{0.0}} & \cellcolor{TealBlue!30}{\textbf{0.857}} & \cellcolor{TealBlue!30}{\textbf{137}} & 0.4\\
\texttt{vehicle} & \multicolumn{1}{r}{846} & \multicolumn{1}{r}{252}  & 0.967 & 28 & 0.0 & 0.952 & 41 & \cellcolor{TealBlue!30}{\textbf{0.0}} & \cellcolor{TealBlue!30}{\textbf{0.978}} & \cellcolor{TealBlue!30}{\textbf{19}} & 3.1\\
\texttt{vote} & \multicolumn{1}{r}{435} & \multicolumn{1}{r}{32}  & 0.982 & 8 & 0.0 & 0.982 & 8 & \cellcolor{TealBlue!30}{\textbf{0.0}} & \cellcolor{TealBlue!30}{\textbf{0.989}} & \cellcolor{TealBlue!30}{\textbf{5}} & 1.1\\
\texttt{wine1-un} & \multicolumn{1}{r}{178} & \multicolumn{1}{r}{1276}  & 0.764 & 42 & 0.0 & 0.747 & 45 & \cellcolor{TealBlue!30}{\textbf{0.0}} & \cellcolor{TealBlue!30}{\textbf{0.781}} & \cellcolor{TealBlue!30}{\textbf{39}} & 3.1\\
\texttt{wine2-un} & \multicolumn{1}{r}{178} & \multicolumn{1}{r}{1276}  & 0.736 & 47 & 0.0 & 0.730 & 48 & \cellcolor{TealBlue!30}{\textbf{0.0}} & \cellcolor{TealBlue!30}{\textbf{0.742}} & \cellcolor{TealBlue!30}{\textbf{46}} & 3.2\\
\texttt{wine3-un} & \multicolumn{1}{r}{178} & \multicolumn{1}{r}{1276}  & 0.820 & 32 & 0.0 & 0.820 & 32 & \cellcolor{TealBlue!30}{\textbf{0.0}} & \cellcolor{TealBlue!30}{\textbf{0.831}} & \cellcolor{TealBlue!30}{\textbf{30}} & 3.1\\
\texttt{yeast} & \multicolumn{1}{r}{1484} & \multicolumn{1}{r}{89}  & 0.735 & 394 & 0.0 & 0.737 & 391 & \cellcolor{TealBlue!30}{\textbf{0.0}} & \cellcolor{TealBlue!30}{\textbf{0.751}} & \cellcolor{TealBlue!30}{\textbf{369}} & 3.1\\
\texttt{zoo-1} & \multicolumn{1}{r}{101} & \multicolumn{1}{r}{20}  & \cellcolor{TealBlue!30}{1.000} & \cellcolor{TealBlue!30}{0} & 0.0 & \cellcolor{TealBlue!30}{1.000} & \cellcolor{TealBlue!30}{0} & \cellcolor{TealBlue!30}{\textbf{0.0}} & \cellcolor{TealBlue!30}{1.000} & \cellcolor{TealBlue!30}{0} & 0.0\\
\bottomrule
\end{tabular}

\end{normalsize}
\end{center}
\caption{\label{tab:f4} Comparison with state of the art heuristics (max depth=4)}
\end{table}

\begin{table}[htbp]
\begin{center}
\begin{normalsize}
\tabcolsep=5pt
\begin{tabular}{lccrrrrrr}
\toprule
& && \multicolumn{2}{c}{\cart} & \multicolumn{2}{c}{first sol.} & \multicolumn{2}{c}{$\leq 3$s}\\
\cmidrule(rr){4-5}\cmidrule(rr){6-7}\cmidrule(rr){8-9}
&\multirow{1}{*}{$\#ex.$} & \multirow{1}{*}{\#feat.} &  \multicolumn{1}{c}{error} & \multicolumn{1}{c}{time} & \multicolumn{1}{c}{error} & \multicolumn{1}{c}{time} & \multicolumn{1}{c}{error} & \multicolumn{1}{c}{time} \\
\midrule

\texttt{anneal} & \multicolumn{1}{r}{812} & \multicolumn{1}{r}{47}  & 96 & 0.00 & 94 & \cellcolor{TealBlue!30}{\textbf{0.00}} & \cellcolor{TealBlue!30}{\textbf{91}} & 2.87\\
\texttt{audiology} & \multicolumn{1}{r}{216} & \multicolumn{1}{r}{79}  & \cellcolor{TealBlue!30}{0} & 0.00 & \cellcolor{TealBlue!30}{0} & \cellcolor{TealBlue!30}{\textbf{0.00}} & \cellcolor{TealBlue!30}{0} & 0.00\\
\texttt{australian-credit} & \multicolumn{1}{r}{653} & \multicolumn{1}{r}{73}  & 43 & 0.01 & 43 & \cellcolor{TealBlue!30}{\textbf{0.00}} & \cellcolor{TealBlue!30}{\textbf{30}} & 0.55\\
\texttt{breast-cancer-un} & \multicolumn{1}{r}{683} & \multicolumn{1}{r}{89}  & 8 & 0.00 & 8 & \cellcolor{TealBlue!30}{\textbf{0.00}} & \cellcolor{TealBlue!30}{\textbf{2}} & 0.39\\
\texttt{breast-wisconsin} & \multicolumn{1}{r}{683} & \multicolumn{1}{r}{120}  & 4 & 0.00 & 4 & \cellcolor{TealBlue!30}{\textbf{0.00}} & \cellcolor{TealBlue!30}{\textbf{0}} & 0.33\\
\texttt{car-un} & \multicolumn{1}{r}{1728} & \multicolumn{1}{r}{21}  & 50 & 0.00 & 50 & \cellcolor{TealBlue!30}{\textbf{0.00}} & \cellcolor{TealBlue!30}{\textbf{34}} & 0.27\\
\texttt{diabetes} & \multicolumn{1}{r}{768} & \multicolumn{1}{r}{112}  & 100 & 0.01 & 99 & \cellcolor{TealBlue!30}{\textbf{0.00}} & \cellcolor{TealBlue!30}{\textbf{77}} & 2.05\\
\texttt{forest-fires-un} & \multicolumn{1}{r}{517} & \multicolumn{1}{r}{989}  & 162 & 0.02 & 161 & \cellcolor{TealBlue!30}{\textbf{0.00}} & \cellcolor{TealBlue!30}{\textbf{159}} & 0.22\\
\texttt{german-credit} & \multicolumn{1}{r}{1000} & \multicolumn{1}{r}{110}  & 149 & 0.03 & 141 & \cellcolor{TealBlue!30}{\textbf{0.00}} & \cellcolor{TealBlue!30}{\textbf{114}} & 0.63\\
\texttt{heart-cleveland} & \multicolumn{1}{r}{296} & \multicolumn{1}{r}{50}  & 6 & 0.00 & 7 & \cellcolor{TealBlue!30}{\textbf{0.00}} & \cellcolor{TealBlue!30}{\textbf{0}} & 0.03\\
\texttt{hepatitis} & \multicolumn{1}{r}{137} & \multicolumn{1}{r}{68}  & \cellcolor{TealBlue!30}{0} & 0.00 & \cellcolor{TealBlue!30}{0} & \cellcolor{TealBlue!30}{\textbf{0.00}} & \cellcolor{TealBlue!30}{0} & 0.00\\
\texttt{hypothyroid} & \multicolumn{1}{r}{3247} & \multicolumn{1}{r}{43}  & \cellcolor{TealBlue!30}{\textbf{42}} & 0.01 & 43 & \cellcolor{TealBlue!30}{\textbf{0.00}} & 43 & 1.33\\
\texttt{ionosphere} & \multicolumn{1}{r}{351} & \multicolumn{1}{r}{444}  & 7 & 0.01 & 7 & \cellcolor{TealBlue!30}{\textbf{0.00}} & \cellcolor{TealBlue!30}{\textbf{0}} & 0.52\\
\texttt{kr-vs-kp} & \multicolumn{1}{r}{3196} & \multicolumn{1}{r}{37}  & 103 & 0.01 & 102 & \cellcolor{TealBlue!30}{\textbf{0.00}} & \cellcolor{TealBlue!30}{\textbf{101}} & 2.17\\
\texttt{letter} & \multicolumn{1}{r}{20000} & \multicolumn{1}{r}{224}  & 153 & 0.62 & 143 & \cellcolor{TealBlue!30}{\textbf{0.04}} & \cellcolor{TealBlue!30}{\textbf{142}} & 0.50\\
\texttt{lymph} & \multicolumn{1}{r}{148} & \multicolumn{1}{r}{41}  & \cellcolor{TealBlue!30}{0} & 0.00 & \cellcolor{TealBlue!30}{0} & \cellcolor{TealBlue!30}{\textbf{0.00}} & \cellcolor{TealBlue!30}{0} & 0.00\\
\texttt{mushroom} & \multicolumn{1}{r}{8124} & \multicolumn{1}{r}{91}  & \cellcolor{TealBlue!30}{0} & 0.03 & \cellcolor{TealBlue!30}{0} & \cellcolor{TealBlue!30}{\textbf{0.00}} & \cellcolor{TealBlue!30}{0} & 0.02\\
\texttt{pendigits} & \multicolumn{1}{r}{7494} & \multicolumn{1}{r}{216}  & 1 & 0.09 & 1 & \cellcolor{TealBlue!30}{\textbf{0.01}} & \cellcolor{TealBlue!30}{\textbf{0}} & 0.11\\
\texttt{primary-tumor} & \multicolumn{1}{r}{336} & \multicolumn{1}{r}{16}  & 26 & 0.00 & 26 & \cellcolor{TealBlue!30}{\textbf{0.00}} & \cellcolor{TealBlue!30}{\textbf{17}} & 1.86\\
\texttt{segment} & \multicolumn{1}{r}{2310} & \multicolumn{1}{r}{234}  & \cellcolor{TealBlue!30}{0} & 0.01 & \cellcolor{TealBlue!30}{0} & \cellcolor{TealBlue!30}{\textbf{0.00}} & \cellcolor{TealBlue!30}{0} & 0.03\\
\texttt{soybean} & \multicolumn{1}{r}{630} & \multicolumn{1}{r}{34}  & 11 & 0.00 & 11 & \cellcolor{TealBlue!30}{\textbf{0.00}} & \cellcolor{TealBlue!30}{\textbf{4}} & 0.70\\
\texttt{splice-1} & \multicolumn{1}{r}{3190} & \multicolumn{1}{r}{227}  & 58 & 0.05 & 58 & \cellcolor{TealBlue!30}{\textbf{0.00}} & \cellcolor{TealBlue!30}{\textbf{55}} & 2.24\\
\texttt{taiwan\_binarised} & \multicolumn{1}{r}{30000} & \multicolumn{1}{r}{198}  & 5161 & 0.66 & 5121 & \cellcolor{TealBlue!30}{\textbf{0.04}} & \cellcolor{TealBlue!30}{\textbf{5087}} & 2.06\\
\texttt{tic-tac-toe} & \multicolumn{1}{r}{958} & \multicolumn{1}{r}{18}  & 22 & 0.00 & 21 & \cellcolor{TealBlue!30}{\textbf{0.00}} & \cellcolor{TealBlue!30}{\textbf{1}} & 1.14\\
\texttt{vehicle} & \multicolumn{1}{r}{846} & \multicolumn{1}{r}{252}  & 4 & 0.01 & 4 & \cellcolor{TealBlue!30}{\textbf{0.00}} & \cellcolor{TealBlue!30}{\textbf{0}} & 0.59\\
\texttt{vote} & \multicolumn{1}{r}{435} & \multicolumn{1}{r}{32}  & 2 & 0.00 & 2 & \cellcolor{TealBlue!30}{\textbf{0.00}} & \cellcolor{TealBlue!30}{\textbf{0}} & 0.00\\
\texttt{wine1-un} & \multicolumn{1}{r}{178} & \multicolumn{1}{r}{1276}  & 33 & 0.01 & 33 & \cellcolor{TealBlue!30}{\textbf{0.00}} & \cellcolor{TealBlue!30}{\textbf{30}} & 0.17\\
\texttt{wine2-un} & \multicolumn{1}{r}{178} & \multicolumn{1}{r}{1276}  & 38 & 0.01 & 38 & \cellcolor{TealBlue!30}{\textbf{0.00}} & \cellcolor{TealBlue!30}{\textbf{35}} & 0.49\\
\texttt{wine3-un} & \multicolumn{1}{r}{178} & \multicolumn{1}{r}{1276}  & 26 & 0.01 & 26 & \cellcolor{TealBlue!30}{\textbf{0.00}} & \cellcolor{TealBlue!30}{\textbf{24}} & 2.12\\
\texttt{yeast} & \multicolumn{1}{r}{1484} & \multicolumn{1}{r}{89}  & 306 & 0.01 & 305 & \cellcolor{TealBlue!30}{\textbf{0.00}} & \cellcolor{TealBlue!30}{\textbf{278}} & 1.30\\
\texttt{zoo-1} & \multicolumn{1}{r}{101} & \multicolumn{1}{r}{20}  & \cellcolor{TealBlue!30}{0} & 0.00 & \cellcolor{TealBlue!30}{0} & \cellcolor{TealBlue!30}{\textbf{0.00}} & \cellcolor{TealBlue!30}{0} & 0.00\\
\bottomrule
\end{tabular}

\end{normalsize}
\end{center}
\caption{\label{tab:f7} Comparison with state of the art heuristics (max depth=7)}
\end{table}

\begin{table}[htbp]
\begin{center}
\begin{normalsize}
\tabcolsep=5pt
\begin{tabular}{lccrrrrrr}
\toprule
& && \multicolumn{2}{c}{\cart} & \multicolumn{2}{c}{first sol.} & \multicolumn{2}{c}{$\leq 3$s}\\
\cmidrule(rr){4-5}\cmidrule(rr){6-7}\cmidrule(rr){8-9}
&\multirow{1}{*}{$\#ex.$} & \multirow{1}{*}{\#feat.} &  \multicolumn{1}{c}{error} & \multicolumn{1}{c}{time} & \multicolumn{1}{c}{error} & \multicolumn{1}{c}{time} & \multicolumn{1}{c}{error} & \multicolumn{1}{c}{time} \\
\midrule

\texttt{anneal} & \multicolumn{1}{r}{812} & \multicolumn{1}{r}{47}  & 59 & 0.00 & \cellcolor{TealBlue!30}{58} & \cellcolor{TealBlue!30}{\textbf{0.00}} & \cellcolor{TealBlue!30}{58} & 0.47\\
\texttt{audiology} & \multicolumn{1}{r}{216} & \multicolumn{1}{r}{79}  & \cellcolor{TealBlue!30}{0} & 0.00 & \cellcolor{TealBlue!30}{0} & \cellcolor{TealBlue!30}{\textbf{0.00}} & \cellcolor{TealBlue!30}{0} & 0.00\\
\texttt{australian-credit} & \multicolumn{1}{r}{653} & \multicolumn{1}{r}{73}  & 13 & 0.01 & 13 & \cellcolor{TealBlue!30}{\textbf{0.00}} & \cellcolor{TealBlue!30}{\textbf{0}} & 0.30\\
\texttt{breast-cancer-un} & \multicolumn{1}{r}{683} & \multicolumn{1}{r}{89}  & \cellcolor{TealBlue!30}{0} & 0.00 & \cellcolor{TealBlue!30}{0} & \cellcolor{TealBlue!30}{\textbf{0.00}} & \cellcolor{TealBlue!30}{0} & 0.00\\
\texttt{breast-wisconsin} & \multicolumn{1}{r}{683} & \multicolumn{1}{r}{120}  & \cellcolor{TealBlue!30}{0} & 0.00 & \cellcolor{TealBlue!30}{0} & \cellcolor{TealBlue!30}{\textbf{0.00}} & \cellcolor{TealBlue!30}{0} & 0.00\\
\texttt{car-un} & \multicolumn{1}{r}{1728} & \multicolumn{1}{r}{21}  & 11 & 0.00 & 11 & \cellcolor{TealBlue!30}{\textbf{0.00}} & \cellcolor{TealBlue!30}{\textbf{0}} & 0.30\\
\texttt{diabetes} & \multicolumn{1}{r}{768} & \multicolumn{1}{r}{112}  & 35 & 0.01 & 39 & \cellcolor{TealBlue!30}{\textbf{0.00}} & \cellcolor{TealBlue!30}{\textbf{4}} & 1.28\\
\texttt{forest-fires-un} & \multicolumn{1}{r}{517} & \multicolumn{1}{r}{989}  & 145 & 0.02 & 145 & \cellcolor{TealBlue!30}{\textbf{0.00}} & \cellcolor{TealBlue!30}{\textbf{131}} & 2.59\\
\texttt{german-credit} & \multicolumn{1}{r}{1000} & \multicolumn{1}{r}{110}  & 66 & 0.01 & 64 & \cellcolor{TealBlue!30}{\textbf{0.00}} & \cellcolor{TealBlue!30}{\textbf{21}} & 0.73\\
\texttt{heart-cleveland} & \multicolumn{1}{r}{296} & \multicolumn{1}{r}{50}  & \cellcolor{TealBlue!30}{0} & 0.00 & \cellcolor{TealBlue!30}{0} & \cellcolor{TealBlue!30}{\textbf{0.00}} & \cellcolor{TealBlue!30}{0} & 0.00\\
\texttt{hepatitis} & \multicolumn{1}{r}{137} & \multicolumn{1}{r}{68}  & \cellcolor{TealBlue!30}{0} & 0.00 & \cellcolor{TealBlue!30}{0} & \cellcolor{TealBlue!30}{\textbf{0.00}} & \cellcolor{TealBlue!30}{0} & 0.00\\
\texttt{hypothyroid} & \multicolumn{1}{r}{3247} & \multicolumn{1}{r}{43}  & \cellcolor{TealBlue!30}{\textbf{31}} & 0.01 & 32 & \cellcolor{TealBlue!30}{0.00} & 32 & \cellcolor{TealBlue!30}{0.00}\\
\texttt{ionosphere} & \multicolumn{1}{r}{351} & \multicolumn{1}{r}{444}  & \cellcolor{TealBlue!30}{0} & 0.01 & \cellcolor{TealBlue!30}{0} & \cellcolor{TealBlue!30}{\textbf{0.00}} & \cellcolor{TealBlue!30}{0} & 0.03\\
\texttt{kr-vs-kp} & \multicolumn{1}{r}{3196} & \multicolumn{1}{r}{37}  & 12 & 0.01 & 12 & \cellcolor{TealBlue!30}{\textbf{0.00}} & \cellcolor{TealBlue!30}{\textbf{10}} & 1.35\\
\texttt{letter} & \multicolumn{1}{r}{20000} & \multicolumn{1}{r}{224}  & 20 & 0.40 & 20 & \cellcolor{TealBlue!30}{\textbf{0.03}} & \cellcolor{TealBlue!30}{\textbf{13}} & 2.05\\
\texttt{lymph} & \multicolumn{1}{r}{148} & \multicolumn{1}{r}{41}  & \cellcolor{TealBlue!30}{0} & 0.00 & \cellcolor{TealBlue!30}{0} & \cellcolor{TealBlue!30}{\textbf{0.00}} & \cellcolor{TealBlue!30}{0} & 0.00\\
\texttt{mushroom} & \multicolumn{1}{r}{8124} & \multicolumn{1}{r}{91}  & \cellcolor{TealBlue!30}{0} & 0.03 & \cellcolor{TealBlue!30}{0} & \cellcolor{TealBlue!30}{\textbf{0.01}} & \cellcolor{TealBlue!30}{0} & 0.03\\
\texttt{pendigits} & \multicolumn{1}{r}{7494} & \multicolumn{1}{r}{216}  & \cellcolor{TealBlue!30}{0} & 0.08 & \cellcolor{TealBlue!30}{0} & \cellcolor{TealBlue!30}{\textbf{0.01}} & \cellcolor{TealBlue!30}{0} & 0.11\\
\texttt{primary-tumor} & \multicolumn{1}{r}{336} & \multicolumn{1}{r}{16}  & 20 & 0.00 & 20 & \cellcolor{TealBlue!30}{\textbf{0.00}} & \cellcolor{TealBlue!30}{\textbf{19}} & 2.88\\
\texttt{segment} & \multicolumn{1}{r}{2310} & \multicolumn{1}{r}{234}  & \cellcolor{TealBlue!30}{0} & 0.01 & \cellcolor{TealBlue!30}{0} & \cellcolor{TealBlue!30}{\textbf{0.00}} & \cellcolor{TealBlue!30}{0} & 0.03\\
\texttt{soybean} & \multicolumn{1}{r}{630} & \multicolumn{1}{r}{34}  & \cellcolor{TealBlue!30}{2} & 0.00 & \cellcolor{TealBlue!30}{2} & \cellcolor{TealBlue!30}{\textbf{0.00}} & \cellcolor{TealBlue!30}{2} & 0.01\\
\texttt{splice-1} & \multicolumn{1}{r}{3190} & \multicolumn{1}{r}{227}  & 12 & 0.05 & 12 & \cellcolor{TealBlue!30}{\textbf{0.01}} & \cellcolor{TealBlue!30}{\textbf{9}} & 0.81\\
\texttt{taiwan\_binarised} & \multicolumn{1}{r}{30000} & \multicolumn{1}{r}{198}  & 4707 & 0.80 & 4668 & \cellcolor{TealBlue!30}{\textbf{0.05}} & \cellcolor{TealBlue!30}{\textbf{4607}} & 1.77\\
\texttt{tic-tac-toe} & \multicolumn{1}{r}{958} & \multicolumn{1}{r}{18}  & 6 & 0.00 & 6 & \cellcolor{TealBlue!30}{\textbf{0.00}} & \cellcolor{TealBlue!30}{\textbf{0}} & 0.00\\
\texttt{vehicle} & \multicolumn{1}{r}{846} & \multicolumn{1}{r}{252}  & \cellcolor{TealBlue!30}{0} & 0.01 & \cellcolor{TealBlue!30}{0} & \cellcolor{TealBlue!30}{\textbf{0.00}} & \cellcolor{TealBlue!30}{0} & 0.02\\
\texttt{vote} & \multicolumn{1}{r}{435} & \multicolumn{1}{r}{32}  & \cellcolor{TealBlue!30}{0} & 0.00 & \cellcolor{TealBlue!30}{0} & \cellcolor{TealBlue!30}{\textbf{0.00}} & \cellcolor{TealBlue!30}{0} & 0.00\\
\texttt{wine1-un} & \multicolumn{1}{r}{178} & \multicolumn{1}{r}{1276}  & 25 & 0.01 & 25 & \cellcolor{TealBlue!30}{\textbf{0.00}} & \cellcolor{TealBlue!30}{\textbf{23}} & 1.70\\
\texttt{wine2-un} & \multicolumn{1}{r}{178} & \multicolumn{1}{r}{1276}  & 29 & 0.01 & 29 & \cellcolor{TealBlue!30}{\textbf{0.00}} & \cellcolor{TealBlue!30}{\textbf{27}} & 0.08\\
\texttt{wine3-un} & \multicolumn{1}{r}{178} & \multicolumn{1}{r}{1276}  & \cellcolor{TealBlue!30}{\textbf{15}} & 0.01 & 20 & \cellcolor{TealBlue!30}{\textbf{0.00}} & 19 & 1.07\\
\texttt{yeast} & \multicolumn{1}{r}{1484} & \multicolumn{1}{r}{89}  & 185 & 0.01 & 180 & \cellcolor{TealBlue!30}{\textbf{0.00}} & \cellcolor{TealBlue!30}{\textbf{115}} & 2.39\\
\texttt{zoo-1} & \multicolumn{1}{r}{101} & \multicolumn{1}{r}{20}  & \cellcolor{TealBlue!30}{0} & 0.01 & \cellcolor{TealBlue!30}{0} & \cellcolor{TealBlue!30}{\textbf{0.00}} & \cellcolor{TealBlue!30}{0} & 0.00\\
\bottomrule
\end{tabular}

\end{normalsize}
\end{center}
\caption{\label{tab:f10} Comparison with state of the art heuristics (max depth=10)}
\end{table}


\subsection*{Comparison of heuristics}


\begin{table}[htbp]
\begin{center}
\begin{normalsize}
\tabcolsep=5pt
\begin{tabular}{lccrrrrrrrrr}
\toprule
& && \multicolumn{3}{c}{entropy} & \multicolumn{3}{c}{\budalg} & \multicolumn{3}{c}{error}\\
\cmidrule(rr){4-6}\cmidrule(rr){7-9}\cmidrule(rr){10-12}
&\multirow{1}{*}{$\#ex.$} & \multirow{1}{*}{\#feat.} &  \multicolumn{1}{c}{opt} & \multicolumn{1}{c}{error} & \multicolumn{1}{c}{time} & \multicolumn{1}{c}{opt} & \multicolumn{1}{c}{error} & \multicolumn{1}{c}{time} & \multicolumn{1}{c}{opt} & \multicolumn{1}{c}{error} & \multicolumn{1}{c}{time} \\
\midrule

\texttt{anneal} & \multicolumn{1}{r}{812} & \multicolumn{1}{r}{47}  & \cellcolor{TealBlue!30}{1} & \cellcolor{TealBlue!30}{112} & 0.3 & \cellcolor{TealBlue!30}{1} & \cellcolor{TealBlue!30}{112} & \cellcolor{TealBlue!30}{\textbf{0.2}} & \cellcolor{TealBlue!30}{1} & \cellcolor{TealBlue!30}{112} & 0.2\\
\texttt{audiology} & \multicolumn{1}{r}{216} & \multicolumn{1}{r}{79}  & \cellcolor{TealBlue!30}{1} & \cellcolor{TealBlue!30}{5} & 0.6 & \cellcolor{TealBlue!30}{1} & \cellcolor{TealBlue!30}{5} & 0.3 & \cellcolor{TealBlue!30}{1} & \cellcolor{TealBlue!30}{5} & \cellcolor{TealBlue!30}{\textbf{0.2}}\\
\texttt{australian-credit} & \multicolumn{1}{r}{653} & \multicolumn{1}{r}{73}  & \cellcolor{TealBlue!30}{1} & \cellcolor{TealBlue!30}{73} & 1.1 & \cellcolor{TealBlue!30}{1} & \cellcolor{TealBlue!30}{73} & 0.6 & \cellcolor{TealBlue!30}{1} & \cellcolor{TealBlue!30}{73} & \cellcolor{TealBlue!30}{\textbf{0.6}}\\
\texttt{breast-cancer-un} & \multicolumn{1}{r}{683} & \multicolumn{1}{r}{89}  & \cellcolor{TealBlue!30}{1} & \cellcolor{TealBlue!30}{24} & 0.3 & \cellcolor{TealBlue!30}{1} & \cellcolor{TealBlue!30}{24} & 0.1 & \cellcolor{TealBlue!30}{1} & \cellcolor{TealBlue!30}{24} & \cellcolor{TealBlue!30}{\textbf{0.1}}\\
\texttt{breast-wisconsin} & \multicolumn{1}{r}{683} & \multicolumn{1}{r}{120}  & \cellcolor{TealBlue!30}{1} & \cellcolor{TealBlue!30}{15} & 0.8 & \cellcolor{TealBlue!30}{1} & \cellcolor{TealBlue!30}{15} & 0.4 & \cellcolor{TealBlue!30}{1} & \cellcolor{TealBlue!30}{15} & \cellcolor{TealBlue!30}{\textbf{0.4}}\\
\texttt{car-un} & \multicolumn{1}{r}{1728} & \multicolumn{1}{r}{21}  & \cellcolor{TealBlue!30}{1} & \cellcolor{TealBlue!30}{192} & 0.0 & \cellcolor{TealBlue!30}{1} & \cellcolor{TealBlue!30}{192} & \cellcolor{TealBlue!30}{\textbf{0.0}} & \cellcolor{TealBlue!30}{1} & \cellcolor{TealBlue!30}{192} & 0.0\\
\texttt{diabetes} & \multicolumn{1}{r}{768} & \multicolumn{1}{r}{112}  & \cellcolor{TealBlue!30}{1} & \cellcolor{TealBlue!30}{162} & 1.0 & \cellcolor{TealBlue!30}{1} & \cellcolor{TealBlue!30}{162} & 0.5 & \cellcolor{TealBlue!30}{1} & \cellcolor{TealBlue!30}{162} & \cellcolor{TealBlue!30}{\textbf{0.5}}\\
\texttt{forest-fires-un} & \multicolumn{1}{r}{517} & \multicolumn{1}{r}{989}  & \cellcolor{TealBlue!30}{1} & \cellcolor{TealBlue!30}{193} & 149.0 & \cellcolor{TealBlue!30}{1} & \cellcolor{TealBlue!30}{193} & 69.2 & \cellcolor{TealBlue!30}{1} & \cellcolor{TealBlue!30}{193} & \cellcolor{TealBlue!30}{\textbf{59.9}}\\
\texttt{german-credit} & \multicolumn{1}{r}{1000} & \multicolumn{1}{r}{110}  & \cellcolor{TealBlue!30}{1} & \cellcolor{TealBlue!30}{236} & 1.2 & \cellcolor{TealBlue!30}{1} & \cellcolor{TealBlue!30}{236} & 0.6 & \cellcolor{TealBlue!30}{1} & \cellcolor{TealBlue!30}{236} & \cellcolor{TealBlue!30}{\textbf{0.5}}\\
\texttt{heart-cleveland} & \multicolumn{1}{r}{296} & \multicolumn{1}{r}{50}  & \cellcolor{TealBlue!30}{1} & \cellcolor{TealBlue!30}{41} & 0.5 & \cellcolor{TealBlue!30}{1} & \cellcolor{TealBlue!30}{41} & 0.2 & \cellcolor{TealBlue!30}{1} & \cellcolor{TealBlue!30}{41} & \cellcolor{TealBlue!30}{\textbf{0.2}}\\
\texttt{hepatitis} & \multicolumn{1}{r}{137} & \multicolumn{1}{r}{68}  & \cellcolor{TealBlue!30}{1} & \cellcolor{TealBlue!30}{10} & 0.1 & \cellcolor{TealBlue!30}{1} & \cellcolor{TealBlue!30}{10} & 0.1 & \cellcolor{TealBlue!30}{1} & \cellcolor{TealBlue!30}{10} & \cellcolor{TealBlue!30}{\textbf{0.1}}\\
\texttt{hypothyroid} & \multicolumn{1}{r}{3247} & \multicolumn{1}{r}{43}  & \cellcolor{TealBlue!30}{1} & \cellcolor{TealBlue!30}{61} & 0.9 & \cellcolor{TealBlue!30}{1} & \cellcolor{TealBlue!30}{61} & \cellcolor{TealBlue!30}{\textbf{0.6}} & \cellcolor{TealBlue!30}{1} & \cellcolor{TealBlue!30}{61} & 0.7\\
\texttt{ionosphere} & \multicolumn{1}{r}{351} & \multicolumn{1}{r}{444}  & \cellcolor{TealBlue!30}{1} & \cellcolor{TealBlue!30}{22} & 53.3 & \cellcolor{TealBlue!30}{1} & \cellcolor{TealBlue!30}{22} & 24.6 & \cellcolor{TealBlue!30}{1} & \cellcolor{TealBlue!30}{22} & \cellcolor{TealBlue!30}{\textbf{22.5}}\\
\texttt{kr-vs-kp} & \multicolumn{1}{r}{3196} & \multicolumn{1}{r}{37}  & \cellcolor{TealBlue!30}{1} & \cellcolor{TealBlue!30}{198} & 0.5 & \cellcolor{TealBlue!30}{1} & \cellcolor{TealBlue!30}{198} & \cellcolor{TealBlue!30}{\textbf{0.4}} & \cellcolor{TealBlue!30}{1} & \cellcolor{TealBlue!30}{198} & 0.4\\
\texttt{letter} & \multicolumn{1}{r}{20000} & \multicolumn{1}{r}{224}  & \cellcolor{TealBlue!30}{1} & \cellcolor{TealBlue!30}{369} & 137.0 & \cellcolor{TealBlue!30}{1} & \cellcolor{TealBlue!30}{369} & \cellcolor{TealBlue!30}{\textbf{57.1}} & \cellcolor{TealBlue!30}{1} & \cellcolor{TealBlue!30}{369} & 138.0\\
\texttt{lymph} & \multicolumn{1}{r}{148} & \multicolumn{1}{r}{41}  & \cellcolor{TealBlue!30}{1} & \cellcolor{TealBlue!30}{12} & 0.1 & \cellcolor{TealBlue!30}{1} & \cellcolor{TealBlue!30}{12} & 0.1 & \cellcolor{TealBlue!30}{1} & \cellcolor{TealBlue!30}{12} & \cellcolor{TealBlue!30}{\textbf{0.1}}\\
\texttt{mushroom} & \multicolumn{1}{r}{8124} & \multicolumn{1}{r}{91}  & \cellcolor{TealBlue!30}{1} & \cellcolor{TealBlue!30}{8} & 2.8 & \cellcolor{TealBlue!30}{1} & \cellcolor{TealBlue!30}{8} & \cellcolor{TealBlue!30}{\textbf{1.7}} & \cellcolor{TealBlue!30}{1} & \cellcolor{TealBlue!30}{8} & 2.4\\
\texttt{pendigits} & \multicolumn{1}{r}{7494} & \multicolumn{1}{r}{216}  & \cellcolor{TealBlue!30}{1} & \cellcolor{TealBlue!30}{47} & 32.6 & \cellcolor{TealBlue!30}{1} & \cellcolor{TealBlue!30}{47} & \cellcolor{TealBlue!30}{\textbf{18.7}} & \cellcolor{TealBlue!30}{1} & \cellcolor{TealBlue!30}{47} & 35.5\\
\texttt{primary-tumor} & \multicolumn{1}{r}{336} & \multicolumn{1}{r}{16}  & \cellcolor{TealBlue!30}{1} & \cellcolor{TealBlue!30}{46} & 0.0 & \cellcolor{TealBlue!30}{1} & \cellcolor{TealBlue!30}{46} & \cellcolor{TealBlue!30}{\textbf{0.0}} & \cellcolor{TealBlue!30}{1} & \cellcolor{TealBlue!30}{46} & 0.0\\
\texttt{segment} & \multicolumn{1}{r}{2310} & \multicolumn{1}{r}{234}  & \cellcolor{TealBlue!30}{1} & \cellcolor{TealBlue!30}{0} & 0.4 & \cellcolor{TealBlue!30}{1} & \cellcolor{TealBlue!30}{0} & \cellcolor{TealBlue!30}{\textbf{0.3}} & \cellcolor{TealBlue!30}{1} & \cellcolor{TealBlue!30}{0} & 0.3\\
\texttt{soybean} & \multicolumn{1}{r}{630} & \multicolumn{1}{r}{34}  & \cellcolor{TealBlue!30}{1} & \cellcolor{TealBlue!30}{29} & 0.1 & \cellcolor{TealBlue!30}{1} & \cellcolor{TealBlue!30}{29} & 0.0 & \cellcolor{TealBlue!30}{1} & \cellcolor{TealBlue!30}{29} & \cellcolor{TealBlue!30}{\textbf{0.0}}\\
\texttt{splice-1} & \multicolumn{1}{r}{3190} & \multicolumn{1}{r}{227}  & \cellcolor{TealBlue!30}{1} & \cellcolor{TealBlue!30}{224} & 23.6 & \cellcolor{TealBlue!30}{1} & \cellcolor{TealBlue!30}{224} & 14.4 & \cellcolor{TealBlue!30}{1} & \cellcolor{TealBlue!30}{224} & \cellcolor{TealBlue!30}{\textbf{14.2}}\\
\texttt{taiwan\_binarised} & \multicolumn{1}{r}{30000} & \multicolumn{1}{r}{198}  & \cellcolor{TealBlue!30}{1} & \cellcolor{TealBlue!30}{5326} & 138.0 & \cellcolor{TealBlue!30}{1} & \cellcolor{TealBlue!30}{5326} & \cellcolor{TealBlue!30}{\textbf{45.8}} & \cellcolor{TealBlue!30}{1} & \cellcolor{TealBlue!30}{5326} & 146.0\\
\texttt{tic-tac-toe} & \multicolumn{1}{r}{958} & \multicolumn{1}{r}{18}  & \cellcolor{TealBlue!30}{1} & \cellcolor{TealBlue!30}{216} & 0.0 & \cellcolor{TealBlue!30}{1} & \cellcolor{TealBlue!30}{216} & \cellcolor{TealBlue!30}{\textbf{0.0}} & \cellcolor{TealBlue!30}{1} & \cellcolor{TealBlue!30}{216} & 0.0\\
\texttt{vehicle} & \multicolumn{1}{r}{846} & \multicolumn{1}{r}{252}  & \cellcolor{TealBlue!30}{1} & \cellcolor{TealBlue!30}{26} & 8.2 & \cellcolor{TealBlue!30}{1} & \cellcolor{TealBlue!30}{26} & \cellcolor{TealBlue!30}{\textbf{4.4}} & \cellcolor{TealBlue!30}{1} & \cellcolor{TealBlue!30}{26} & 4.4\\
\texttt{vote} & \multicolumn{1}{r}{435} & \multicolumn{1}{r}{32}  & \cellcolor{TealBlue!30}{1} & \cellcolor{TealBlue!30}{12} & 0.1 & \cellcolor{TealBlue!30}{1} & \cellcolor{TealBlue!30}{12} & 0.0 & \cellcolor{TealBlue!30}{1} & \cellcolor{TealBlue!30}{12} & \cellcolor{TealBlue!30}{\textbf{0.0}}\\
\texttt{wine1-un} & \multicolumn{1}{r}{178} & \multicolumn{1}{r}{1276}  & \cellcolor{TealBlue!30}{1} & \cellcolor{TealBlue!30}{43} & 256.0 & \cellcolor{TealBlue!30}{1} & \cellcolor{TealBlue!30}{43} & 123.0 & \cellcolor{TealBlue!30}{1} & \cellcolor{TealBlue!30}{43} & \cellcolor{TealBlue!30}{\textbf{115.0}}\\
\texttt{wine2-un} & \multicolumn{1}{r}{178} & \multicolumn{1}{r}{1276}  & \cellcolor{TealBlue!30}{1} & \cellcolor{TealBlue!30}{49} & 256.0 & \cellcolor{TealBlue!30}{1} & \cellcolor{TealBlue!30}{49} & 122.0 & \cellcolor{TealBlue!30}{1} & \cellcolor{TealBlue!30}{49} & \cellcolor{TealBlue!30}{\textbf{110.0}}\\
\texttt{wine3-un} & \multicolumn{1}{r}{178} & \multicolumn{1}{r}{1276}  & \cellcolor{TealBlue!30}{1} & \cellcolor{TealBlue!30}{33} & 255.0 & \cellcolor{TealBlue!30}{1} & \cellcolor{TealBlue!30}{33} & 122.0 & \cellcolor{TealBlue!30}{1} & \cellcolor{TealBlue!30}{33} & \cellcolor{TealBlue!30}{\textbf{104.0}}\\
\texttt{yeast} & \multicolumn{1}{r}{1484} & \multicolumn{1}{r}{89}  & \cellcolor{TealBlue!30}{1} & \cellcolor{TealBlue!30}{403} & 0.7 & \cellcolor{TealBlue!30}{1} & \cellcolor{TealBlue!30}{403} & \cellcolor{TealBlue!30}{\textbf{0.4}} & \cellcolor{TealBlue!30}{1} & \cellcolor{TealBlue!30}{403} & 0.5\\
\texttt{zoo-1} & \multicolumn{1}{r}{101} & \multicolumn{1}{r}{20}  & \cellcolor{TealBlue!30}{1} & \cellcolor{TealBlue!30}{0} & 0.0 & \cellcolor{TealBlue!30}{1} & \cellcolor{TealBlue!30}{0} & \cellcolor{TealBlue!30}{\textbf{0.0}} & \cellcolor{TealBlue!30}{1} & \cellcolor{TealBlue!30}{0} & 0.0\\
\bottomrule
\end{tabular}

\end{normalsize}
\end{center}
\caption{\label{tab:ha3} Comparison of heuristics (max depth=3)}
\end{table}

\begin{table}[htbp]
\begin{center}
\begin{normalsize}
\tabcolsep=5pt
\begin{tabular}{lccrrrrrrrrr}
\toprule
& && \multicolumn{3}{c}{entropy} & \multicolumn{3}{c}{\budalg} & \multicolumn{3}{c}{error}\\
\cmidrule(rr){4-6}\cmidrule(rr){7-9}\cmidrule(rr){10-12}
&\multirow{1}{*}{$\#ex.$} & \multirow{1}{*}{\#feat.} &  \multicolumn{1}{c}{opt} & \multicolumn{1}{c}{error} & \multicolumn{1}{c}{time} & \multicolumn{1}{c}{opt} & \multicolumn{1}{c}{error} & \multicolumn{1}{c}{time} & \multicolumn{1}{c}{opt} & \multicolumn{1}{c}{error} & \multicolumn{1}{c}{time} \\
\midrule

\texttt{anneal} & \multicolumn{1}{r}{812} & \multicolumn{1}{r}{47}  & \cellcolor{TealBlue!30}{1} & \cellcolor{TealBlue!30}{91} & 23.2 & \cellcolor{TealBlue!30}{1} & \cellcolor{TealBlue!30}{91} & \cellcolor{TealBlue!30}{14.1} & \cellcolor{TealBlue!30}{1} & \cellcolor{TealBlue!30}{91} & \cellcolor{TealBlue!30}{14.1}\\
\texttt{audiology} & \multicolumn{1}{r}{216} & \multicolumn{1}{r}{79}  & \cellcolor{TealBlue!30}{1} & \cellcolor{TealBlue!30}{1} & 63.6 & \cellcolor{TealBlue!30}{1} & \cellcolor{TealBlue!30}{1} & 31.2 & \cellcolor{TealBlue!30}{1} & \cellcolor{TealBlue!30}{1} & \cellcolor{TealBlue!30}{\textbf{26.4}}\\
\texttt{australian-credit} & \multicolumn{1}{r}{653} & \multicolumn{1}{r}{73}  & \cellcolor{TealBlue!30}{1} & \cellcolor{TealBlue!30}{56} & 166.0 & \cellcolor{TealBlue!30}{1} & \cellcolor{TealBlue!30}{56} & 83.4 & \cellcolor{TealBlue!30}{1} & \cellcolor{TealBlue!30}{56} & \cellcolor{TealBlue!30}{\textbf{73.8}}\\
\texttt{breast-cancer-un} & \multicolumn{1}{r}{683} & \multicolumn{1}{r}{89}  & \cellcolor{TealBlue!30}{1} & \cellcolor{TealBlue!30}{16} & 21.6 & \cellcolor{TealBlue!30}{1} & \cellcolor{TealBlue!30}{16} & 12.3 & \cellcolor{TealBlue!30}{1} & \cellcolor{TealBlue!30}{16} & \cellcolor{TealBlue!30}{\textbf{11.2}}\\
\texttt{breast-wisconsin} & \multicolumn{1}{r}{683} & \multicolumn{1}{r}{120}  & \cellcolor{TealBlue!30}{1} & \cellcolor{TealBlue!30}{7} & 80.9 & \cellcolor{TealBlue!30}{1} & \cellcolor{TealBlue!30}{7} & 42.5 & \cellcolor{TealBlue!30}{1} & \cellcolor{TealBlue!30}{7} & \cellcolor{TealBlue!30}{\textbf{37.8}}\\
\texttt{car-un} & \multicolumn{1}{r}{1728} & \multicolumn{1}{r}{21}  & \cellcolor{TealBlue!30}{1} & \cellcolor{TealBlue!30}{136} & 0.4 & \cellcolor{TealBlue!30}{1} & \cellcolor{TealBlue!30}{136} & \cellcolor{TealBlue!30}{\textbf{0.3}} & \cellcolor{TealBlue!30}{1} & \cellcolor{TealBlue!30}{136} & 0.3\\
\texttt{diabetes} & \multicolumn{1}{r}{768} & \multicolumn{1}{r}{112}  & \cellcolor{TealBlue!30}{1} & \cellcolor{TealBlue!30}{137} & 142.0 & \cellcolor{TealBlue!30}{1} & \cellcolor{TealBlue!30}{137} & 71.1 & \cellcolor{TealBlue!30}{1} & \cellcolor{TealBlue!30}{137} & \cellcolor{TealBlue!30}{\textbf{65.8}}\\
\texttt{forest-fires-un} & \multicolumn{1}{r}{517} & \multicolumn{1}{r}{989}  & \cellcolor{TealBlue!30}{0} & \cellcolor{TealBlue!30}{173} & 244.0 & \cellcolor{TealBlue!30}{0} & \cellcolor{TealBlue!30}{173} & 50.9 & \cellcolor{TealBlue!30}{0} & \cellcolor{TealBlue!30}{173} & \cellcolor{TealBlue!30}{\textbf{39.0}}\\
\texttt{german-credit} & \multicolumn{1}{r}{1000} & \multicolumn{1}{r}{110}  & \cellcolor{TealBlue!30}{1} & \cellcolor{TealBlue!30}{204} & 163.0 & \cellcolor{TealBlue!30}{1} & \cellcolor{TealBlue!30}{204} & 81.0 & \cellcolor{TealBlue!30}{1} & \cellcolor{TealBlue!30}{204} & \cellcolor{TealBlue!30}{\textbf{65.1}}\\
\texttt{heart-cleveland} & \multicolumn{1}{r}{296} & \multicolumn{1}{r}{50}  & \cellcolor{TealBlue!30}{1} & \cellcolor{TealBlue!30}{25} & 58.8 & \cellcolor{TealBlue!30}{1} & \cellcolor{TealBlue!30}{25} & 25.4 & \cellcolor{TealBlue!30}{1} & \cellcolor{TealBlue!30}{25} & \cellcolor{TealBlue!30}{\textbf{22.0}}\\
\texttt{hepatitis} & \multicolumn{1}{r}{137} & \multicolumn{1}{r}{68}  & \cellcolor{TealBlue!30}{1} & \cellcolor{TealBlue!30}{3} & 7.6 & \cellcolor{TealBlue!30}{1} & \cellcolor{TealBlue!30}{3} & 3.6 & \cellcolor{TealBlue!30}{1} & \cellcolor{TealBlue!30}{3} & \cellcolor{TealBlue!30}{\textbf{3.5}}\\
\texttt{hypothyroid} & \multicolumn{1}{r}{3247} & \multicolumn{1}{r}{43}  & \cellcolor{TealBlue!30}{1} & \cellcolor{TealBlue!30}{53} & 61.6 & \cellcolor{TealBlue!30}{1} & \cellcolor{TealBlue!30}{53} & \cellcolor{TealBlue!30}{\textbf{45.0}} & \cellcolor{TealBlue!30}{1} & \cellcolor{TealBlue!30}{53} & 48.5\\
\texttt{ionosphere} & \multicolumn{1}{r}{351} & \multicolumn{1}{r}{444}  & \cellcolor{TealBlue!30}{0} & 8 & 123.0 & \cellcolor{TealBlue!30}{0} & 8 & \cellcolor{TealBlue!30}{\textbf{58.4}} & \cellcolor{TealBlue!30}{0} & \cellcolor{TealBlue!30}{\textbf{7}} & 3410.0\\
\texttt{kr-vs-kp} & \multicolumn{1}{r}{3196} & \multicolumn{1}{r}{37}  & \cellcolor{TealBlue!30}{1} & \cellcolor{TealBlue!30}{144} & 38.4 & \cellcolor{TealBlue!30}{1} & \cellcolor{TealBlue!30}{144} & \cellcolor{TealBlue!30}{\textbf{27.7}} & \cellcolor{TealBlue!30}{1} & \cellcolor{TealBlue!30}{144} & 28.8\\
\texttt{letter} & \multicolumn{1}{r}{20000} & \multicolumn{1}{r}{224}  & \cellcolor{TealBlue!30}{0} & \cellcolor{TealBlue!30}{261} & 1190.0 & \cellcolor{TealBlue!30}{0} & \cellcolor{TealBlue!30}{261} & \cellcolor{TealBlue!30}{\textbf{410.0}} & \cellcolor{TealBlue!30}{0} & 263 & 3040.0\\
\texttt{lymph} & \multicolumn{1}{r}{148} & \multicolumn{1}{r}{41}  & \cellcolor{TealBlue!30}{1} & \cellcolor{TealBlue!30}{3} & 5.3 & \cellcolor{TealBlue!30}{1} & \cellcolor{TealBlue!30}{3} & 2.7 & \cellcolor{TealBlue!30}{1} & \cellcolor{TealBlue!30}{3} & \cellcolor{TealBlue!30}{\textbf{2.3}}\\
\texttt{mushroom} & \multicolumn{1}{r}{8124} & \multicolumn{1}{r}{91}  & \cellcolor{TealBlue!30}{1} & \cellcolor{TealBlue!30}{0} & 0.0 & \cellcolor{TealBlue!30}{1} & \cellcolor{TealBlue!30}{0} & \cellcolor{TealBlue!30}{\textbf{0.0}} & \cellcolor{TealBlue!30}{1} & \cellcolor{TealBlue!30}{0} & 0.0\\
\texttt{pendigits} & \multicolumn{1}{r}{7494} & \multicolumn{1}{r}{216}  & 0 & \cellcolor{TealBlue!30}{13} & \cellcolor{TealBlue!30}{\textbf{1250.0}} & \cellcolor{TealBlue!30}{1} & \cellcolor{TealBlue!30}{13} & 3040.0 & \cellcolor{TealBlue!30}{1} & \cellcolor{TealBlue!30}{13} & 3560.0\\
\texttt{primary-tumor} & \multicolumn{1}{r}{336} & \multicolumn{1}{r}{16}  & \cellcolor{TealBlue!30}{1} & \cellcolor{TealBlue!30}{34} & 0.6 & \cellcolor{TealBlue!30}{1} & \cellcolor{TealBlue!30}{34} & \cellcolor{TealBlue!30}{\textbf{0.3}} & \cellcolor{TealBlue!30}{1} & \cellcolor{TealBlue!30}{34} & 0.3\\
\texttt{segment} & \multicolumn{1}{r}{2310} & \multicolumn{1}{r}{234}  & \cellcolor{TealBlue!30}{1} & \cellcolor{TealBlue!30}{0} & 0.0 & \cellcolor{TealBlue!30}{1} & \cellcolor{TealBlue!30}{0} & \cellcolor{TealBlue!30}{\textbf{0.0}} & \cellcolor{TealBlue!30}{1} & \cellcolor{TealBlue!30}{0} & 0.0\\
\texttt{soybean} & \multicolumn{1}{r}{630} & \multicolumn{1}{r}{34}  & \cellcolor{TealBlue!30}{1} & \cellcolor{TealBlue!30}{14} & 2.9 & \cellcolor{TealBlue!30}{1} & \cellcolor{TealBlue!30}{14} & 1.7 & \cellcolor{TealBlue!30}{1} & \cellcolor{TealBlue!30}{14} & \cellcolor{TealBlue!30}{\textbf{1.6}}\\
\texttt{splice-1} & \multicolumn{1}{r}{3190} & \multicolumn{1}{r}{227}  & \cellcolor{TealBlue!30}{0} & \cellcolor{TealBlue!30}{141} & 13.5 & \cellcolor{TealBlue!30}{0} & \cellcolor{TealBlue!30}{141} & 7.7 & \cellcolor{TealBlue!30}{0} & \cellcolor{TealBlue!30}{141} & \cellcolor{TealBlue!30}{\textbf{6.2}}\\
\texttt{taiwan\_binarised} & \multicolumn{1}{r}{30000} & \multicolumn{1}{r}{198}  & \cellcolor{TealBlue!30}{0} & \cellcolor{TealBlue!30}{5273} & 16.3 & \cellcolor{TealBlue!30}{0} & \cellcolor{TealBlue!30}{5273} & \cellcolor{TealBlue!30}{\textbf{7.9}} & \cellcolor{TealBlue!30}{0} & \cellcolor{TealBlue!30}{5273} & 100.0\\
\texttt{tic-tac-toe} & \multicolumn{1}{r}{958} & \multicolumn{1}{r}{18}  & \cellcolor{TealBlue!30}{1} & \cellcolor{TealBlue!30}{137} & 0.8 & \cellcolor{TealBlue!30}{1} & \cellcolor{TealBlue!30}{137} & 0.5 & \cellcolor{TealBlue!30}{1} & \cellcolor{TealBlue!30}{137} & \cellcolor{TealBlue!30}{\textbf{0.5}}\\
\texttt{vehicle} & \multicolumn{1}{r}{846} & \multicolumn{1}{r}{252}  & \cellcolor{TealBlue!30}{1} & \cellcolor{TealBlue!30}{12} & 1800.0 & \cellcolor{TealBlue!30}{1} & \cellcolor{TealBlue!30}{12} & 944.0 & \cellcolor{TealBlue!30}{1} & \cellcolor{TealBlue!30}{12} & \cellcolor{TealBlue!30}{\textbf{902.0}}\\
\texttt{vote} & \multicolumn{1}{r}{435} & \multicolumn{1}{r}{32}  & \cellcolor{TealBlue!30}{1} & \cellcolor{TealBlue!30}{5} & 2.9 & \cellcolor{TealBlue!30}{1} & \cellcolor{TealBlue!30}{5} & 1.6 & \cellcolor{TealBlue!30}{1} & \cellcolor{TealBlue!30}{5} & \cellcolor{TealBlue!30}{\textbf{1.4}}\\
\texttt{wine1-un} & \multicolumn{1}{r}{178} & \multicolumn{1}{r}{1276}  & \cellcolor{TealBlue!30}{0} & 39 & \cellcolor{TealBlue!30}{\textbf{3.8}} & \cellcolor{TealBlue!30}{0} & \cellcolor{TealBlue!30}{38} & 2290.0 & \cellcolor{TealBlue!30}{0} & \cellcolor{TealBlue!30}{38} & 918.0\\
\texttt{wine2-un} & \multicolumn{1}{r}{178} & \multicolumn{1}{r}{1276}  & \cellcolor{TealBlue!30}{0} & \cellcolor{TealBlue!30}{43} & 556.0 & \cellcolor{TealBlue!30}{0} & \cellcolor{TealBlue!30}{43} & 115.0 & \cellcolor{TealBlue!30}{0} & \cellcolor{TealBlue!30}{43} & \cellcolor{TealBlue!30}{\textbf{0.1}}\\
\texttt{wine3-un} & \multicolumn{1}{r}{178} & \multicolumn{1}{r}{1276}  & \cellcolor{TealBlue!30}{0} & \cellcolor{TealBlue!30}{28} & 3600.0 & \cellcolor{TealBlue!30}{0} & \cellcolor{TealBlue!30}{28} & \cellcolor{TealBlue!30}{\textbf{230.0}} & \cellcolor{TealBlue!30}{0} & \cellcolor{TealBlue!30}{28} & 413.0\\
\texttt{yeast} & \multicolumn{1}{r}{1484} & \multicolumn{1}{r}{89}  & \cellcolor{TealBlue!30}{1} & \cellcolor{TealBlue!30}{366} & 70.8 & \cellcolor{TealBlue!30}{1} & \cellcolor{TealBlue!30}{366} & 39.2 & \cellcolor{TealBlue!30}{1} & \cellcolor{TealBlue!30}{366} & \cellcolor{TealBlue!30}{\textbf{38.7}}\\
\texttt{zoo-1} & \multicolumn{1}{r}{101} & \multicolumn{1}{r}{20}  & \cellcolor{TealBlue!30}{1} & \cellcolor{TealBlue!30}{0} & 0.0 & \cellcolor{TealBlue!30}{1} & \cellcolor{TealBlue!30}{0} & \cellcolor{TealBlue!30}{\textbf{0.0}} & \cellcolor{TealBlue!30}{1} & \cellcolor{TealBlue!30}{0} & 0.0\\
\bottomrule
\end{tabular}

\end{normalsize}
\end{center}
\caption{\label{tab:ha4} Comparison of heuristics (max depth=4)}
\end{table}

\begin{table}[htbp]
\begin{center}
\begin{normalsize}
\tabcolsep=5pt
\begin{tabular}{lccrrrrrrrrr}
\toprule
& && \multicolumn{3}{c}{entropy} & \multicolumn{3}{c}{\budalg} & \multicolumn{3}{c}{error}\\
\cmidrule(rr){4-6}\cmidrule(rr){7-9}\cmidrule(rr){10-12}
&\multirow{1}{*}{$\#ex.$} & \multirow{1}{*}{\#feat.} &  \multicolumn{1}{c}{opt} & \multicolumn{1}{c}{error} & \multicolumn{1}{c}{time} & \multicolumn{1}{c}{opt} & \multicolumn{1}{c}{error} & \multicolumn{1}{c}{time} & \multicolumn{1}{c}{opt} & \multicolumn{1}{c}{error} & \multicolumn{1}{c}{time} \\
\midrule

\texttt{anneal} & \multicolumn{1}{r}{812} & \multicolumn{1}{r}{47}  & \cellcolor{TealBlue!30}{1} & \cellcolor{TealBlue!30}{70} & 1630.0 & \cellcolor{TealBlue!30}{1} & \cellcolor{TealBlue!30}{70} & 995.0 & \cellcolor{TealBlue!30}{1} & \cellcolor{TealBlue!30}{70} & \cellcolor{TealBlue!30}{\textbf{920.0}}\\
\texttt{audiology} & \multicolumn{1}{r}{216} & \multicolumn{1}{r}{79}  & \cellcolor{TealBlue!30}{1} & \cellcolor{TealBlue!30}{0} & 0.0 & \cellcolor{TealBlue!30}{1} & \cellcolor{TealBlue!30}{0} & 0.0 & \cellcolor{TealBlue!30}{1} & \cellcolor{TealBlue!30}{0} & \cellcolor{TealBlue!30}{\textbf{0.0}}\\
\texttt{australian-credit} & \multicolumn{1}{r}{653} & \multicolumn{1}{r}{73}  & \cellcolor{TealBlue!30}{0} & \cellcolor{TealBlue!30}{40} & 108.0 & \cellcolor{TealBlue!30}{0} & \cellcolor{TealBlue!30}{40} & \cellcolor{TealBlue!30}{\textbf{51.3}} & \cellcolor{TealBlue!30}{0} & \cellcolor{TealBlue!30}{40} & 51.8\\
\texttt{breast-cancer-un} & \multicolumn{1}{r}{683} & \multicolumn{1}{r}{89}  & \cellcolor{TealBlue!30}{1} & \cellcolor{TealBlue!30}{6} & 1800.0 & \cellcolor{TealBlue!30}{1} & \cellcolor{TealBlue!30}{6} & 973.0 & \cellcolor{TealBlue!30}{1} & \cellcolor{TealBlue!30}{6} & \cellcolor{TealBlue!30}{\textbf{892.0}}\\
\texttt{breast-wisconsin} & \multicolumn{1}{r}{683} & \multicolumn{1}{r}{120}  & \cellcolor{TealBlue!30}{1} & \cellcolor{TealBlue!30}{0} & 1180.0 & \cellcolor{TealBlue!30}{1} & \cellcolor{TealBlue!30}{0} & 509.0 & \cellcolor{TealBlue!30}{1} & \cellcolor{TealBlue!30}{0} & \cellcolor{TealBlue!30}{\textbf{381.0}}\\
\texttt{car-un} & \multicolumn{1}{r}{1728} & \multicolumn{1}{r}{21}  & \cellcolor{TealBlue!30}{1} & \cellcolor{TealBlue!30}{86} & 5.4 & \cellcolor{TealBlue!30}{1} & \cellcolor{TealBlue!30}{86} & \cellcolor{TealBlue!30}{\textbf{3.9}} & \cellcolor{TealBlue!30}{1} & \cellcolor{TealBlue!30}{86} & 4.2\\
\texttt{diabetes} & \multicolumn{1}{r}{768} & \multicolumn{1}{r}{112}  & \cellcolor{TealBlue!30}{0} & 107 & \cellcolor{TealBlue!30}{\textbf{131.0}} & \cellcolor{TealBlue!30}{0} & \cellcolor{TealBlue!30}{106} & 1910.0 & \cellcolor{TealBlue!30}{0} & \cellcolor{TealBlue!30}{106} & 3130.0\\
\texttt{forest-fires-un} & \multicolumn{1}{r}{517} & \multicolumn{1}{r}{989}  & \cellcolor{TealBlue!30}{0} & 172 & \cellcolor{TealBlue!30}{\textbf{105.0}} & \cellcolor{TealBlue!30}{0} & \cellcolor{TealBlue!30}{\textbf{156}} & 2980.0 & \cellcolor{TealBlue!30}{0} & 157 & 435.0\\
\texttt{german-credit} & \multicolumn{1}{r}{1000} & \multicolumn{1}{r}{110}  & \cellcolor{TealBlue!30}{0} & \cellcolor{TealBlue!30}{161} & 197.0 & \cellcolor{TealBlue!30}{0} & \cellcolor{TealBlue!30}{161} & \cellcolor{TealBlue!30}{\textbf{103.0}} & \cellcolor{TealBlue!30}{0} & 165 & 2600.0\\
\texttt{heart-cleveland} & \multicolumn{1}{r}{296} & \multicolumn{1}{r}{50}  & \cellcolor{TealBlue!30}{1} & \cellcolor{TealBlue!30}{7} & 3450.0 & \cellcolor{TealBlue!30}{1} & \cellcolor{TealBlue!30}{7} & 1520.0 & \cellcolor{TealBlue!30}{1} & \cellcolor{TealBlue!30}{7} & \cellcolor{TealBlue!30}{\textbf{1370.0}}\\
\texttt{hepatitis} & \multicolumn{1}{r}{137} & \multicolumn{1}{r}{68}  & \cellcolor{TealBlue!30}{1} & \cellcolor{TealBlue!30}{0} & 1.0 & \cellcolor{TealBlue!30}{1} & \cellcolor{TealBlue!30}{0} & \cellcolor{TealBlue!30}{\textbf{0.5}} & \cellcolor{TealBlue!30}{1} & \cellcolor{TealBlue!30}{0} & 1.6\\
\texttt{hypothyroid} & \multicolumn{1}{r}{3247} & \multicolumn{1}{r}{43}  & 0 & \cellcolor{TealBlue!30}{44} & \cellcolor{TealBlue!30}{\textbf{1070.0}} & \cellcolor{TealBlue!30}{1} & \cellcolor{TealBlue!30}{44} & 2850.0 & \cellcolor{TealBlue!30}{1} & \cellcolor{TealBlue!30}{44} & 2910.0\\
\texttt{ionosphere} & \multicolumn{1}{r}{351} & \multicolumn{1}{r}{444}  & \cellcolor{TealBlue!30}{0} & 3 & \cellcolor{TealBlue!30}{\textbf{258.0}} & \cellcolor{TealBlue!30}{0} & \cellcolor{TealBlue!30}{2} & 1980.0 & \cellcolor{TealBlue!30}{0} & \cellcolor{TealBlue!30}{2} & 707.0\\
\texttt{kr-vs-kp} & \multicolumn{1}{r}{3196} & \multicolumn{1}{r}{37}  & \cellcolor{TealBlue!30}{1} & \cellcolor{TealBlue!30}{81} & 1970.0 & \cellcolor{TealBlue!30}{1} & \cellcolor{TealBlue!30}{81} & 1400.0 & \cellcolor{TealBlue!30}{1} & \cellcolor{TealBlue!30}{81} & \cellcolor{TealBlue!30}{\textbf{1390.0}}\\
\texttt{letter} & \multicolumn{1}{r}{20000} & \multicolumn{1}{r}{224}  & \cellcolor{TealBlue!30}{0} & 447 & 2910.0 & \cellcolor{TealBlue!30}{0} & 280 & \cellcolor{TealBlue!30}{\textbf{373.0}} & \cellcolor{TealBlue!30}{0} & \cellcolor{TealBlue!30}{\textbf{251}} & 2970.0\\
\texttt{lymph} & \multicolumn{1}{r}{148} & \multicolumn{1}{r}{41}  & \cellcolor{TealBlue!30}{1} & \cellcolor{TealBlue!30}{0} & 0.0 & \cellcolor{TealBlue!30}{1} & \cellcolor{TealBlue!30}{0} & \cellcolor{TealBlue!30}{\textbf{0.0}} & \cellcolor{TealBlue!30}{1} & \cellcolor{TealBlue!30}{0} & 0.0\\
\texttt{mushroom} & \multicolumn{1}{r}{8124} & \multicolumn{1}{r}{91}  & \cellcolor{TealBlue!30}{1} & \cellcolor{TealBlue!30}{0} & 0.0 & \cellcolor{TealBlue!30}{1} & \cellcolor{TealBlue!30}{0} & \cellcolor{TealBlue!30}{\textbf{0.0}} & \cellcolor{TealBlue!30}{1} & \cellcolor{TealBlue!30}{0} & 0.0\\
\texttt{pendigits} & \multicolumn{1}{r}{7494} & \multicolumn{1}{r}{216}  & \cellcolor{TealBlue!30}{0} & \cellcolor{TealBlue!30}{2} & \cellcolor{TealBlue!30}{\textbf{90.6}} & \cellcolor{TealBlue!30}{0} & \cellcolor{TealBlue!30}{2} & 1780.0 & \cellcolor{TealBlue!30}{0} & \cellcolor{TealBlue!30}{2} & 338.0\\
\texttt{primary-tumor} & \multicolumn{1}{r}{336} & \multicolumn{1}{r}{16}  & \cellcolor{TealBlue!30}{1} & \cellcolor{TealBlue!30}{26} & 16.6 & \cellcolor{TealBlue!30}{1} & \cellcolor{TealBlue!30}{26} & \cellcolor{TealBlue!30}{\textbf{8.9}} & \cellcolor{TealBlue!30}{1} & \cellcolor{TealBlue!30}{26} & 9.1\\
\texttt{segment} & \multicolumn{1}{r}{2310} & \multicolumn{1}{r}{234}  & \cellcolor{TealBlue!30}{1} & \cellcolor{TealBlue!30}{0} & 0.0 & \cellcolor{TealBlue!30}{1} & \cellcolor{TealBlue!30}{0} & \cellcolor{TealBlue!30}{\textbf{0.0}} & \cellcolor{TealBlue!30}{1} & \cellcolor{TealBlue!30}{0} & 0.0\\
\texttt{soybean} & \multicolumn{1}{r}{630} & \multicolumn{1}{r}{34}  & \cellcolor{TealBlue!30}{1} & \cellcolor{TealBlue!30}{8} & 101.0 & \cellcolor{TealBlue!30}{1} & \cellcolor{TealBlue!30}{8} & 62.7 & \cellcolor{TealBlue!30}{1} & \cellcolor{TealBlue!30}{8} & \cellcolor{TealBlue!30}{\textbf{59.0}}\\
\texttt{splice-1} & \multicolumn{1}{r}{3190} & \multicolumn{1}{r}{227}  & \cellcolor{TealBlue!30}{0} & 103 & \cellcolor{TealBlue!30}{\textbf{41.8}} & \cellcolor{TealBlue!30}{0} & \cellcolor{TealBlue!30}{101} & 2260.0 & \cellcolor{TealBlue!30}{0} & \cellcolor{TealBlue!30}{101} & 2010.0\\
\texttt{taiwan\_binarised} & \multicolumn{1}{r}{30000} & \multicolumn{1}{r}{198}  & \cellcolor{TealBlue!30}{0} & \cellcolor{TealBlue!30}{5200} & 1840.0 & \cellcolor{TealBlue!30}{0} & \cellcolor{TealBlue!30}{5200} & \cellcolor{TealBlue!30}{\textbf{1290.0}} & \cellcolor{TealBlue!30}{0} & 5204 & 2210.0\\
\texttt{tic-tac-toe} & \multicolumn{1}{r}{958} & \multicolumn{1}{r}{18}  & \cellcolor{TealBlue!30}{1} & \cellcolor{TealBlue!30}{63} & 21.5 & \cellcolor{TealBlue!30}{1} & \cellcolor{TealBlue!30}{63} & 12.1 & \cellcolor{TealBlue!30}{1} & \cellcolor{TealBlue!30}{63} & \cellcolor{TealBlue!30}{\textbf{11.1}}\\
\texttt{vehicle} & \multicolumn{1}{r}{846} & \multicolumn{1}{r}{252}  & \cellcolor{TealBlue!30}{0} & \cellcolor{TealBlue!30}{3} & 2630.0 & \cellcolor{TealBlue!30}{0} & \cellcolor{TealBlue!30}{3} & \cellcolor{TealBlue!30}{\textbf{88.8}} & \cellcolor{TealBlue!30}{0} & 9 & 3550.0\\
\texttt{vote} & \multicolumn{1}{r}{435} & \multicolumn{1}{r}{32}  & \cellcolor{TealBlue!30}{1} & \cellcolor{TealBlue!30}{1} & 54.8 & \cellcolor{TealBlue!30}{1} & \cellcolor{TealBlue!30}{1} & 30.1 & \cellcolor{TealBlue!30}{1} & \cellcolor{TealBlue!30}{1} & \cellcolor{TealBlue!30}{\textbf{27.7}}\\
\texttt{wine1-un} & \multicolumn{1}{r}{178} & \multicolumn{1}{r}{1276}  & \cellcolor{TealBlue!30}{0} & 35 & 1010.0 & \cellcolor{TealBlue!30}{0} & \cellcolor{TealBlue!30}{34} & 1430.0 & \cellcolor{TealBlue!30}{0} & \cellcolor{TealBlue!30}{34} & \cellcolor{TealBlue!30}{\textbf{804.0}}\\
\texttt{wine2-un} & \multicolumn{1}{r}{178} & \multicolumn{1}{r}{1276}  & \cellcolor{TealBlue!30}{0} & 40 & 226.0 & \cellcolor{TealBlue!30}{0} & 39 & 2820.0 & \cellcolor{TealBlue!30}{0} & \cellcolor{TealBlue!30}{\textbf{37}} & \cellcolor{TealBlue!30}{\textbf{86.9}}\\
\texttt{wine3-un} & \multicolumn{1}{r}{178} & \multicolumn{1}{r}{1276}  & \cellcolor{TealBlue!30}{0} & \cellcolor{TealBlue!30}{25} & 254.0 & \cellcolor{TealBlue!30}{0} & \cellcolor{TealBlue!30}{25} & \cellcolor{TealBlue!30}{\textbf{110.0}} & \cellcolor{TealBlue!30}{0} & 26 & 914.0\\
\texttt{yeast} & \multicolumn{1}{r}{1484} & \multicolumn{1}{r}{89}  & 0 & \cellcolor{TealBlue!30}{313} & \cellcolor{TealBlue!30}{\textbf{1430.0}} & \cellcolor{TealBlue!30}{1} & \cellcolor{TealBlue!30}{313} & 3270.0 & \cellcolor{TealBlue!30}{1} & \cellcolor{TealBlue!30}{313} & 3290.0\\
\texttt{zoo-1} & \multicolumn{1}{r}{101} & \multicolumn{1}{r}{20}  & \cellcolor{TealBlue!30}{1} & \cellcolor{TealBlue!30}{0} & 0.0 & \cellcolor{TealBlue!30}{1} & \cellcolor{TealBlue!30}{0} & \cellcolor{TealBlue!30}{\textbf{0.0}} & \cellcolor{TealBlue!30}{1} & \cellcolor{TealBlue!30}{0} & 0.0\\
\bottomrule
\end{tabular}

\end{normalsize}
\end{center}
\caption{\label{tab:ha5} Comparison of heuristics (max depth=5)}
\end{table}

\begin{table}[htbp]
\begin{center}
\begin{normalsize}
\tabcolsep=5pt
\begin{tabular}{lccrrrrrrrrr}
\toprule
& && \multicolumn{3}{c}{entropy} & \multicolumn{3}{c}{\budalg} & \multicolumn{3}{c}{error}\\
\cmidrule(rr){4-6}\cmidrule(rr){7-9}\cmidrule(rr){10-12}
&\multirow{1}{*}{$\#ex.$} & \multirow{1}{*}{\#feat.} &  \multicolumn{1}{c}{opt} & \multicolumn{1}{c}{error} & \multicolumn{1}{c}{time} & \multicolumn{1}{c}{opt} & \multicolumn{1}{c}{error} & \multicolumn{1}{c}{time} & \multicolumn{1}{c}{opt} & \multicolumn{1}{c}{error} & \multicolumn{1}{c}{time} \\
\midrule

\texttt{anneal} & \multicolumn{1}{r}{812} & \multicolumn{1}{r}{47}  & \cellcolor{TealBlue!30}{0} & \cellcolor{TealBlue!30}{58} & 430.0 & \cellcolor{TealBlue!30}{0} & \cellcolor{TealBlue!30}{58} & \cellcolor{TealBlue!30}{\textbf{209.0}} & \cellcolor{TealBlue!30}{0} & 64 & 3090.0\\
\texttt{audiology} & \multicolumn{1}{r}{216} & \multicolumn{1}{r}{79}  & \cellcolor{TealBlue!30}{1} & \cellcolor{TealBlue!30}{0} & 0.0 & \cellcolor{TealBlue!30}{1} & \cellcolor{TealBlue!30}{0} & 0.0 & \cellcolor{TealBlue!30}{1} & \cellcolor{TealBlue!30}{0} & \cellcolor{TealBlue!30}{\textbf{0.0}}\\
\texttt{australian-credit} & \multicolumn{1}{r}{653} & \multicolumn{1}{r}{73}  & 0 & 9 & \cellcolor{TealBlue!30}{\textbf{726.0}} & \cellcolor{TealBlue!30}{\textbf{1}} & \cellcolor{TealBlue!30}{\textbf{0}} & 1040.0 & 0 & 13 & 2020.0\\
\texttt{breast-cancer-un} & \multicolumn{1}{r}{683} & \multicolumn{1}{r}{89}  & \cellcolor{TealBlue!30}{1} & \cellcolor{TealBlue!30}{0} & 1340.0 & \cellcolor{TealBlue!30}{1} & \cellcolor{TealBlue!30}{0} & 1190.0 & \cellcolor{TealBlue!30}{1} & \cellcolor{TealBlue!30}{0} & \cellcolor{TealBlue!30}{\textbf{973.0}}\\
\texttt{breast-wisconsin} & \multicolumn{1}{r}{683} & \multicolumn{1}{r}{120}  & \cellcolor{TealBlue!30}{1} & \cellcolor{TealBlue!30}{0} & 0.3 & \cellcolor{TealBlue!30}{1} & \cellcolor{TealBlue!30}{0} & 0.3 & \cellcolor{TealBlue!30}{1} & \cellcolor{TealBlue!30}{0} & \cellcolor{TealBlue!30}{\textbf{0.1}}\\
\texttt{car-un} & \multicolumn{1}{r}{1728} & \multicolumn{1}{r}{21}  & \cellcolor{TealBlue!30}{1} & \cellcolor{TealBlue!30}{11} & 438.0 & \cellcolor{TealBlue!30}{1} & \cellcolor{TealBlue!30}{11} & 294.0 & \cellcolor{TealBlue!30}{1} & \cellcolor{TealBlue!30}{11} & \cellcolor{TealBlue!30}{\textbf{255.0}}\\
\texttt{diabetes} & \multicolumn{1}{r}{768} & \multicolumn{1}{r}{112}  & \cellcolor{TealBlue!30}{0} & 32 & 2080.0 & \cellcolor{TealBlue!30}{0} & \cellcolor{TealBlue!30}{\textbf{27}} & 3210.0 & \cellcolor{TealBlue!30}{0} & 61 & \cellcolor{TealBlue!30}{\textbf{481.0}}\\
\texttt{forest-fires-un} & \multicolumn{1}{r}{517} & \multicolumn{1}{r}{989}  & \cellcolor{TealBlue!30}{0} & 160 & 669.0 & \cellcolor{TealBlue!30}{0} & 155 & \cellcolor{TealBlue!30}{\textbf{325.0}} & \cellcolor{TealBlue!30}{0} & \cellcolor{TealBlue!30}{\textbf{150}} & 390.0\\
\texttt{german-credit} & \multicolumn{1}{r}{1000} & \multicolumn{1}{r}{110}  & \cellcolor{TealBlue!30}{0} & \cellcolor{TealBlue!30}{57} & 1360.0 & \cellcolor{TealBlue!30}{0} & \cellcolor{TealBlue!30}{57} & \cellcolor{TealBlue!30}{\textbf{706.0}} & \cellcolor{TealBlue!30}{0} & 133 & 3540.0\\
\texttt{heart-cleveland} & \multicolumn{1}{r}{296} & \multicolumn{1}{r}{50}  & \cellcolor{TealBlue!30}{1} & \cellcolor{TealBlue!30}{0} & 1.4 & \cellcolor{TealBlue!30}{1} & \cellcolor{TealBlue!30}{0} & \cellcolor{TealBlue!30}{\textbf{0.0}} & \cellcolor{TealBlue!30}{1} & \cellcolor{TealBlue!30}{0} & 46.6\\
\texttt{hepatitis} & \multicolumn{1}{r}{137} & \multicolumn{1}{r}{68}  & \cellcolor{TealBlue!30}{1} & \cellcolor{TealBlue!30}{0} & 0.0 & \cellcolor{TealBlue!30}{1} & \cellcolor{TealBlue!30}{0} & 0.0 & \cellcolor{TealBlue!30}{1} & \cellcolor{TealBlue!30}{0} & \cellcolor{TealBlue!30}{\textbf{0.0}}\\
\texttt{hypothyroid} & \multicolumn{1}{r}{3247} & \multicolumn{1}{r}{43}  & \cellcolor{TealBlue!30}{0} & 45 & 126.0 & \cellcolor{TealBlue!30}{0} & \cellcolor{TealBlue!30}{\textbf{43}} & \cellcolor{TealBlue!30}{\textbf{78.3}} & \cellcolor{TealBlue!30}{0} & 50 & 338.0\\
\texttt{ionosphere} & \multicolumn{1}{r}{351} & \multicolumn{1}{r}{444}  & \cellcolor{TealBlue!30}{1} & \cellcolor{TealBlue!30}{0} & \cellcolor{TealBlue!30}{\textbf{0.2}} & \cellcolor{TealBlue!30}{1} & \cellcolor{TealBlue!30}{0} & 0.5 & \cellcolor{TealBlue!30}{1} & \cellcolor{TealBlue!30}{0} & 1.1\\
\texttt{kr-vs-kp} & \multicolumn{1}{r}{3196} & \multicolumn{1}{r}{37}  & \cellcolor{TealBlue!30}{0} & 37 & \cellcolor{TealBlue!30}{\textbf{194.0}} & \cellcolor{TealBlue!30}{0} & 37 & 1460.0 & \cellcolor{TealBlue!30}{0} & \cellcolor{TealBlue!30}{\textbf{21}} & 222.0\\
\texttt{letter} & \multicolumn{1}{r}{20000} & \multicolumn{1}{r}{224}  & \cellcolor{TealBlue!30}{0} & \cellcolor{TealBlue!30}{\textbf{112}} & 3260.0 & \cellcolor{TealBlue!30}{0} & 118 & \cellcolor{TealBlue!30}{\textbf{219.0}} & \cellcolor{TealBlue!30}{0} & 184 & 1680.0\\
\texttt{lymph} & \multicolumn{1}{r}{148} & \multicolumn{1}{r}{41}  & \cellcolor{TealBlue!30}{1} & \cellcolor{TealBlue!30}{0} & \cellcolor{TealBlue!30}{\textbf{0.0}} & \cellcolor{TealBlue!30}{1} & \cellcolor{TealBlue!30}{0} & 0.0 & \cellcolor{TealBlue!30}{1} & \cellcolor{TealBlue!30}{0} & 0.0\\
\texttt{mushroom} & \multicolumn{1}{r}{8124} & \multicolumn{1}{r}{91}  & \cellcolor{TealBlue!30}{1} & \cellcolor{TealBlue!30}{0} & 0.0 & \cellcolor{TealBlue!30}{1} & \cellcolor{TealBlue!30}{0} & \cellcolor{TealBlue!30}{\textbf{0.0}} & \cellcolor{TealBlue!30}{1} & \cellcolor{TealBlue!30}{0} & 0.0\\
\texttt{pendigits} & \multicolumn{1}{r}{7494} & \multicolumn{1}{r}{216}  & \cellcolor{TealBlue!30}{1} & \cellcolor{TealBlue!30}{0} & 0.1 & \cellcolor{TealBlue!30}{1} & \cellcolor{TealBlue!30}{0} & \cellcolor{TealBlue!30}{\textbf{0.1}} & \cellcolor{TealBlue!30}{1} & \cellcolor{TealBlue!30}{0} & 55.9\\
\texttt{primary-tumor} & \multicolumn{1}{r}{336} & \multicolumn{1}{r}{16}  & \cellcolor{TealBlue!30}{0} & \cellcolor{TealBlue!30}{16} & 178.0 & \cellcolor{TealBlue!30}{0} & \cellcolor{TealBlue!30}{16} & 93.9 & \cellcolor{TealBlue!30}{0} & \cellcolor{TealBlue!30}{16} & \cellcolor{TealBlue!30}{\textbf{93.7}}\\
\texttt{segment} & \multicolumn{1}{r}{2310} & \multicolumn{1}{r}{234}  & \cellcolor{TealBlue!30}{1} & \cellcolor{TealBlue!30}{0} & 0.0 & \cellcolor{TealBlue!30}{1} & \cellcolor{TealBlue!30}{0} & 0.0 & \cellcolor{TealBlue!30}{1} & \cellcolor{TealBlue!30}{0} & \cellcolor{TealBlue!30}{\textbf{0.0}}\\
\texttt{soybean} & \multicolumn{1}{r}{630} & \multicolumn{1}{r}{34}  & \cellcolor{TealBlue!30}{0} & \cellcolor{TealBlue!30}{2} & 1960.0 & \cellcolor{TealBlue!30}{0} & \cellcolor{TealBlue!30}{2} & 3540.0 & \cellcolor{TealBlue!30}{0} & \cellcolor{TealBlue!30}{2} & \cellcolor{TealBlue!30}{\textbf{1730.0}}\\
\texttt{splice-1} & \multicolumn{1}{r}{3190} & \multicolumn{1}{r}{227}  & \cellcolor{TealBlue!30}{0} & 33 & 2570.0 & \cellcolor{TealBlue!30}{0} & \cellcolor{TealBlue!30}{\textbf{32}} & 2440.0 & \cellcolor{TealBlue!30}{0} & 45 & \cellcolor{TealBlue!30}{\textbf{1730.0}}\\
\texttt{taiwan\_binarised} & \multicolumn{1}{r}{30000} & \multicolumn{1}{r}{198}  & \cellcolor{TealBlue!30}{0} & \cellcolor{TealBlue!30}{\textbf{5017}} & 2250.0 & \cellcolor{TealBlue!30}{0} & 5065 & \cellcolor{TealBlue!30}{\textbf{70.3}} & \cellcolor{TealBlue!30}{0} & 5152 & 177.0\\
\texttt{tic-tac-toe} & \multicolumn{1}{r}{958} & \multicolumn{1}{r}{18}  & \cellcolor{TealBlue!30}{1} & \cellcolor{TealBlue!30}{0} & 48.7 & \cellcolor{TealBlue!30}{1} & \cellcolor{TealBlue!30}{0} & \cellcolor{TealBlue!30}{\textbf{26.4}} & \cellcolor{TealBlue!30}{1} & \cellcolor{TealBlue!30}{0} & 27.5\\
\texttt{vehicle} & \multicolumn{1}{r}{846} & \multicolumn{1}{r}{252}  & \cellcolor{TealBlue!30}{1} & \cellcolor{TealBlue!30}{0} & 1.7 & \cellcolor{TealBlue!30}{1} & \cellcolor{TealBlue!30}{0} & \cellcolor{TealBlue!30}{\textbf{0.6}} & \cellcolor{TealBlue!30}{1} & \cellcolor{TealBlue!30}{0} & 688.0\\
\texttt{vote} & \multicolumn{1}{r}{435} & \multicolumn{1}{r}{32}  & \cellcolor{TealBlue!30}{1} & \cellcolor{TealBlue!30}{0} & 0.0 & \cellcolor{TealBlue!30}{1} & \cellcolor{TealBlue!30}{0} & \cellcolor{TealBlue!30}{\textbf{0.0}} & \cellcolor{TealBlue!30}{1} & \cellcolor{TealBlue!30}{0} & 0.1\\
\texttt{wine1-un} & \multicolumn{1}{r}{178} & \multicolumn{1}{r}{1276}  & \cellcolor{TealBlue!30}{0} & 34 & 2850.0 & \cellcolor{TealBlue!30}{0} & 29 & 509.0 & \cellcolor{TealBlue!30}{0} & \cellcolor{TealBlue!30}{\textbf{28}} & \cellcolor{TealBlue!30}{\textbf{94.4}}\\
\texttt{wine2-un} & \multicolumn{1}{r}{178} & \multicolumn{1}{r}{1276}  & \cellcolor{TealBlue!30}{0} & \cellcolor{TealBlue!30}{31} & 2110.0 & \cellcolor{TealBlue!30}{0} & \cellcolor{TealBlue!30}{31} & 172.0 & \cellcolor{TealBlue!30}{0} & \cellcolor{TealBlue!30}{31} & \cellcolor{TealBlue!30}{\textbf{0.2}}\\
\texttt{wine3-un} & \multicolumn{1}{r}{178} & \multicolumn{1}{r}{1276}  & \cellcolor{TealBlue!30}{0} & \cellcolor{TealBlue!30}{20} & 644.0 & \cellcolor{TealBlue!30}{0} & \cellcolor{TealBlue!30}{20} & 307.0 & \cellcolor{TealBlue!30}{0} & 23 & \cellcolor{TealBlue!30}{\textbf{76.3}}\\
\texttt{yeast} & \multicolumn{1}{r}{1484} & \multicolumn{1}{r}{89}  & \cellcolor{TealBlue!30}{0} & 265 & \cellcolor{TealBlue!30}{\textbf{215.0}} & \cellcolor{TealBlue!30}{0} & 264 & 3130.0 & \cellcolor{TealBlue!30}{0} & \cellcolor{TealBlue!30}{\textbf{252}} & 3190.0\\
\texttt{zoo-1} & \multicolumn{1}{r}{101} & \multicolumn{1}{r}{20}  & \cellcolor{TealBlue!30}{1} & \cellcolor{TealBlue!30}{0} & 0.0 & \cellcolor{TealBlue!30}{1} & \cellcolor{TealBlue!30}{0} & \cellcolor{TealBlue!30}{\textbf{0.0}} & \cellcolor{TealBlue!30}{1} & \cellcolor{TealBlue!30}{0} & 0.0\\
\bottomrule
\end{tabular}

\end{normalsize}
\end{center}
\caption{\label{tab:ha7} Comparison of heuristics (max depth=7)}
\end{table}

\begin{table}[htbp]
\begin{center}
\begin{normalsize}
\tabcolsep=5pt
\begin{tabular}{lccrrrrrrrrr}
\toprule
& && \multicolumn{3}{c}{entropy} & \multicolumn{3}{c}{\budalg} & \multicolumn{3}{c}{error}\\
\cmidrule(rr){4-6}\cmidrule(rr){7-9}\cmidrule(rr){10-12}
&\multirow{1}{*}{$\#ex.$} & \multirow{1}{*}{\#feat.} &  \multicolumn{1}{c}{opt} & \multicolumn{1}{c}{error} & \multicolumn{1}{c}{time} & \multicolumn{1}{c}{opt} & \multicolumn{1}{c}{error} & \multicolumn{1}{c}{time} & \multicolumn{1}{c}{opt} & \multicolumn{1}{c}{error} & \multicolumn{1}{c}{time} \\
\midrule

\texttt{anneal} & \multicolumn{1}{r}{812} & \multicolumn{1}{r}{47}  & \cellcolor{TealBlue!30}{0} & 76 & 1260.0 & \cellcolor{TealBlue!30}{0} & 58 & 773.0 & \cellcolor{TealBlue!30}{0} & \cellcolor{TealBlue!30}{\textbf{48}} & \cellcolor{TealBlue!30}{\textbf{678.0}}\\
\texttt{audiology} & \multicolumn{1}{r}{216} & \multicolumn{1}{r}{79}  & \cellcolor{TealBlue!30}{1} & \cellcolor{TealBlue!30}{0} & 0.0 & \cellcolor{TealBlue!30}{1} & \cellcolor{TealBlue!30}{0} & 0.0 & \cellcolor{TealBlue!30}{1} & \cellcolor{TealBlue!30}{0} & \cellcolor{TealBlue!30}{\textbf{0.0}}\\
\texttt{australian-credit} & \multicolumn{1}{r}{653} & \multicolumn{1}{r}{73}  & \cellcolor{TealBlue!30}{1} & \cellcolor{TealBlue!30}{0} & 0.8 & \cellcolor{TealBlue!30}{1} & \cellcolor{TealBlue!30}{0} & \cellcolor{TealBlue!30}{\textbf{0.3}} & 0 & 1 & 2160.0\\
\texttt{breast-cancer-un} & \multicolumn{1}{r}{683} & \multicolumn{1}{r}{89}  & \cellcolor{TealBlue!30}{1} & \cellcolor{TealBlue!30}{0} & 1.9 & \cellcolor{TealBlue!30}{1} & \cellcolor{TealBlue!30}{0} & \cellcolor{TealBlue!30}{\textbf{0.0}} & \cellcolor{TealBlue!30}{1} & \cellcolor{TealBlue!30}{0} & 0.4\\
\texttt{breast-wisconsin} & \multicolumn{1}{r}{683} & \multicolumn{1}{r}{120}  & \cellcolor{TealBlue!30}{1} & \cellcolor{TealBlue!30}{0} & 0.0 & \cellcolor{TealBlue!30}{1} & \cellcolor{TealBlue!30}{0} & \cellcolor{TealBlue!30}{\textbf{0.0}} & \cellcolor{TealBlue!30}{1} & \cellcolor{TealBlue!30}{0} & 0.0\\
\texttt{car-un} & \multicolumn{1}{r}{1728} & \multicolumn{1}{r}{21}  & \cellcolor{TealBlue!30}{1} & \cellcolor{TealBlue!30}{0} & 0.4 & \cellcolor{TealBlue!30}{1} & \cellcolor{TealBlue!30}{0} & \cellcolor{TealBlue!30}{\textbf{0.3}} & \cellcolor{TealBlue!30}{1} & \cellcolor{TealBlue!30}{0} & 25.3\\
\texttt{diabetes} & \multicolumn{1}{r}{768} & \multicolumn{1}{r}{112}  & \cellcolor{TealBlue!30}{1} & \cellcolor{TealBlue!30}{0} & 26.1 & \cellcolor{TealBlue!30}{1} & \cellcolor{TealBlue!30}{0} & \cellcolor{TealBlue!30}{\textbf{12.8}} & 0 & 7 & 3580.0\\
\texttt{forest-fires-un} & \multicolumn{1}{r}{517} & \multicolumn{1}{r}{989}  & \cellcolor{TealBlue!30}{0} & 123 & 1370.0 & \cellcolor{TealBlue!30}{0} & \cellcolor{TealBlue!30}{\textbf{118}} & 3370.0 & \cellcolor{TealBlue!30}{0} & 136 & \cellcolor{TealBlue!30}{\textbf{179.0}}\\
\texttt{german-credit} & \multicolumn{1}{r}{1000} & \multicolumn{1}{r}{110}  & \cellcolor{TealBlue!30}{1} & \cellcolor{TealBlue!30}{0} & 296.0 & \cellcolor{TealBlue!30}{1} & \cellcolor{TealBlue!30}{0} & \cellcolor{TealBlue!30}{\textbf{171.0}} & 0 & 100 & 661.0\\
\texttt{heart-cleveland} & \multicolumn{1}{r}{296} & \multicolumn{1}{r}{50}  & \cellcolor{TealBlue!30}{1} & \cellcolor{TealBlue!30}{0} & \cellcolor{TealBlue!30}{\textbf{0.0}} & \cellcolor{TealBlue!30}{1} & \cellcolor{TealBlue!30}{0} & 0.0 & \cellcolor{TealBlue!30}{1} & \cellcolor{TealBlue!30}{0} & 0.1\\
\texttt{hepatitis} & \multicolumn{1}{r}{137} & \multicolumn{1}{r}{68}  & \cellcolor{TealBlue!30}{1} & \cellcolor{TealBlue!30}{0} & 0.0 & \cellcolor{TealBlue!30}{1} & \cellcolor{TealBlue!30}{0} & \cellcolor{TealBlue!30}{\textbf{0.0}} & \cellcolor{TealBlue!30}{1} & \cellcolor{TealBlue!30}{0} & 0.0\\
\texttt{hypothyroid} & \multicolumn{1}{r}{3247} & \multicolumn{1}{r}{43}  & \cellcolor{TealBlue!30}{0} & \cellcolor{TealBlue!30}{32} & 486.0 & \cellcolor{TealBlue!30}{0} & \cellcolor{TealBlue!30}{32} & \cellcolor{TealBlue!30}{\textbf{34.8}} & \cellcolor{TealBlue!30}{0} & 44 & 126.0\\
\texttt{ionosphere} & \multicolumn{1}{r}{351} & \multicolumn{1}{r}{444}  & \cellcolor{TealBlue!30}{1} & \cellcolor{TealBlue!30}{0} & 0.1 & \cellcolor{TealBlue!30}{1} & \cellcolor{TealBlue!30}{0} & 0.0 & \cellcolor{TealBlue!30}{1} & \cellcolor{TealBlue!30}{0} & \cellcolor{TealBlue!30}{\textbf{0.0}}\\
\texttt{kr-vs-kp} & \multicolumn{1}{r}{3196} & \multicolumn{1}{r}{37}  & \cellcolor{TealBlue!30}{0} & \cellcolor{TealBlue!30}{2} & 1140.0 & \cellcolor{TealBlue!30}{0} & \cellcolor{TealBlue!30}{2} & \cellcolor{TealBlue!30}{\textbf{954.0}} & \cellcolor{TealBlue!30}{0} & 71 & 1520.0\\
\texttt{letter} & \multicolumn{1}{r}{20000} & \multicolumn{1}{r}{224}  & 0 & 20 & 2300.0 & \cellcolor{TealBlue!30}{\textbf{1}} & \cellcolor{TealBlue!30}{\textbf{0}} & \cellcolor{TealBlue!30}{\textbf{1600.0}} & 0 & 204 & 3560.0\\
\texttt{lymph} & \multicolumn{1}{r}{148} & \multicolumn{1}{r}{41}  & \cellcolor{TealBlue!30}{1} & \cellcolor{TealBlue!30}{0} & 0.0 & \cellcolor{TealBlue!30}{1} & \cellcolor{TealBlue!30}{0} & \cellcolor{TealBlue!30}{\textbf{0.0}} & \cellcolor{TealBlue!30}{1} & \cellcolor{TealBlue!30}{0} & 0.0\\
\texttt{mushroom} & \multicolumn{1}{r}{8124} & \multicolumn{1}{r}{91}  & \cellcolor{TealBlue!30}{1} & \cellcolor{TealBlue!30}{0} & 0.0 & \cellcolor{TealBlue!30}{1} & \cellcolor{TealBlue!30}{0} & \cellcolor{TealBlue!30}{\textbf{0.0}} & \cellcolor{TealBlue!30}{1} & \cellcolor{TealBlue!30}{0} & 0.0\\
\texttt{pendigits} & \multicolumn{1}{r}{7494} & \multicolumn{1}{r}{216}  & \cellcolor{TealBlue!30}{1} & \cellcolor{TealBlue!30}{0} & 0.1 & \cellcolor{TealBlue!30}{1} & \cellcolor{TealBlue!30}{0} & \cellcolor{TealBlue!30}{\textbf{0.1}} & \cellcolor{TealBlue!30}{1} & \cellcolor{TealBlue!30}{0} & 0.4\\
\texttt{primary-tumor} & \multicolumn{1}{r}{336} & \multicolumn{1}{r}{16}  & \cellcolor{TealBlue!30}{0} & \cellcolor{TealBlue!30}{15} & 311.0 & \cellcolor{TealBlue!30}{0} & \cellcolor{TealBlue!30}{15} & \cellcolor{TealBlue!30}{\textbf{126.0}} & \cellcolor{TealBlue!30}{0} & \cellcolor{TealBlue!30}{15} & 3180.0\\
\texttt{segment} & \multicolumn{1}{r}{2310} & \multicolumn{1}{r}{234}  & \cellcolor{TealBlue!30}{1} & \cellcolor{TealBlue!30}{0} & 0.0 & \cellcolor{TealBlue!30}{1} & \cellcolor{TealBlue!30}{0} & 0.0 & \cellcolor{TealBlue!30}{1} & \cellcolor{TealBlue!30}{0} & \cellcolor{TealBlue!30}{\textbf{0.0}}\\
\texttt{soybean} & \multicolumn{1}{r}{630} & \multicolumn{1}{r}{34}  & \cellcolor{TealBlue!30}{0} & \cellcolor{TealBlue!30}{2} & \cellcolor{TealBlue!30}{\textbf{11.7}} & \cellcolor{TealBlue!30}{0} & \cellcolor{TealBlue!30}{2} & 229.0 & \cellcolor{TealBlue!30}{0} & \cellcolor{TealBlue!30}{2} & 3470.0\\
\texttt{splice-1} & \multicolumn{1}{r}{3190} & \multicolumn{1}{r}{227}  & \cellcolor{TealBlue!30}{0} & \cellcolor{TealBlue!30}{5} & 359.0 & \cellcolor{TealBlue!30}{0} & \cellcolor{TealBlue!30}{5} & 1420.0 & \cellcolor{TealBlue!30}{0} & 26 & \cellcolor{TealBlue!30}{\textbf{290.0}}\\
\texttt{taiwan\_binarised} & \multicolumn{1}{r}{30000} & \multicolumn{1}{r}{198}  & \cellcolor{TealBlue!30}{0} & 4648 & \cellcolor{TealBlue!30}{\textbf{157.0}} & \cellcolor{TealBlue!30}{0} & \cellcolor{TealBlue!30}{\textbf{4566}} & 207.0 & \cellcolor{TealBlue!30}{0} & 5074 & 176.0\\
\texttt{tic-tac-toe} & \multicolumn{1}{r}{958} & \multicolumn{1}{r}{18}  & \cellcolor{TealBlue!30}{1} & \cellcolor{TealBlue!30}{0} & \cellcolor{TealBlue!30}{\textbf{0.0}} & \cellcolor{TealBlue!30}{1} & \cellcolor{TealBlue!30}{0} & 0.0 & \cellcolor{TealBlue!30}{1} & \cellcolor{TealBlue!30}{0} & 0.0\\
\texttt{vehicle} & \multicolumn{1}{r}{846} & \multicolumn{1}{r}{252}  & \cellcolor{TealBlue!30}{1} & \cellcolor{TealBlue!30}{0} & 0.0 & \cellcolor{TealBlue!30}{1} & \cellcolor{TealBlue!30}{0} & \cellcolor{TealBlue!30}{\textbf{0.0}} & \cellcolor{TealBlue!30}{1} & \cellcolor{TealBlue!30}{0} & 135.0\\
\texttt{vote} & \multicolumn{1}{r}{435} & \multicolumn{1}{r}{32}  & \cellcolor{TealBlue!30}{1} & \cellcolor{TealBlue!30}{0} & 0.0 & \cellcolor{TealBlue!30}{1} & \cellcolor{TealBlue!30}{0} & \cellcolor{TealBlue!30}{\textbf{0.0}} & \cellcolor{TealBlue!30}{1} & \cellcolor{TealBlue!30}{0} & 0.0\\
\texttt{wine1-un} & \multicolumn{1}{r}{178} & \multicolumn{1}{r}{1276}  & \cellcolor{TealBlue!30}{0} & 26 & 1950.0 & \cellcolor{TealBlue!30}{0} & \cellcolor{TealBlue!30}{22} & 3450.0 & \cellcolor{TealBlue!30}{0} & \cellcolor{TealBlue!30}{22} & \cellcolor{TealBlue!30}{\textbf{25.3}}\\
\texttt{wine2-un} & \multicolumn{1}{r}{178} & \multicolumn{1}{r}{1276}  & \cellcolor{TealBlue!30}{0} & 27 & \cellcolor{TealBlue!30}{\textbf{4.1}} & \cellcolor{TealBlue!30}{0} & 24 & 2700.0 & \cellcolor{TealBlue!30}{0} & \cellcolor{TealBlue!30}{\textbf{21}} & 54.8\\
\texttt{wine3-un} & \multicolumn{1}{r}{178} & \multicolumn{1}{r}{1276}  & \cellcolor{TealBlue!30}{0} & \cellcolor{TealBlue!30}{\textbf{10}} & 2590.0 & \cellcolor{TealBlue!30}{0} & 18 & 1900.0 & \cellcolor{TealBlue!30}{0} & 17 & \cellcolor{TealBlue!30}{\textbf{107.0}}\\
\texttt{yeast} & \multicolumn{1}{r}{1484} & \multicolumn{1}{r}{89}  & \cellcolor{TealBlue!30}{0} & 109 & 3100.0 & \cellcolor{TealBlue!30}{0} & \cellcolor{TealBlue!30}{\textbf{104}} & 2680.0 & \cellcolor{TealBlue!30}{0} & 196 & \cellcolor{TealBlue!30}{\textbf{131.0}}\\
\texttt{zoo-1} & \multicolumn{1}{r}{101} & \multicolumn{1}{r}{20}  & \cellcolor{TealBlue!30}{1} & \cellcolor{TealBlue!30}{0} & 0.0 & \cellcolor{TealBlue!30}{1} & \cellcolor{TealBlue!30}{0} & \cellcolor{TealBlue!30}{\textbf{0.0}} & \cellcolor{TealBlue!30}{1} & \cellcolor{TealBlue!30}{0} & 0.0\\
\bottomrule
\end{tabular}

\end{normalsize}
\end{center}
\caption{\label{tab:ha10} Comparison of heuristics (max depth=10)}
\end{table}


% \begin{table}[htbp]
% \begin{center}
% \begin{footnotesize}
% \tabcolsep=5pt
% \begin{tabular}{lccrrrrrrrrrrrr}
\toprule
& && \multicolumn{6}{c}{dt no restart} & \multicolumn{6}{c}{dt restarts (1.1)}\\
\cmidrule(rr){4-9}\cmidrule(rr){10-15}
&\multirow{1}{*}{$\#ex.$} & \multirow{1}{*}{\#feat.} &  \multicolumn{1}{c}{opt} & \multicolumn{1}{c}{error} & \multicolumn{1}{c}{acc.} & \multicolumn{1}{c}{size} & \multicolumn{1}{c}{time} & \multicolumn{1}{c}{choices} & \multicolumn{1}{c}{opt} & \multicolumn{1}{c}{error} & \multicolumn{1}{c}{acc.} & \multicolumn{1}{c}{size} & \multicolumn{1}{c}{time} & \multicolumn{1}{c}{choices} \\
\midrule

\texttt{anneal} & \multicolumn{1}{r}{812} & \multicolumn{1}{r}{88}  & \cellcolor{TealBlue!30}{1.0} & \cellcolor{TealBlue!30}{70.0} & \cellcolor{TealBlue!30}{0.914} & \cellcolor{TealBlue!30}{9.0} & \cellcolor{TealBlue!30}{\textbf{1034.7}} & \cellcolor{TealBlue!30}{\textbf{170{\sc m}}} & \cellcolor{TealBlue!30}{1.0} & \cellcolor{TealBlue!30}{70.0} & \cellcolor{TealBlue!30}{0.914} & \cellcolor{TealBlue!30}{9.0} & 1324.0 & 216{\sc m}\\
\texttt{audiology} & \multicolumn{1}{r}{216} & \multicolumn{1}{r}{145}  & \cellcolor{TealBlue!30}{0.0} & \cellcolor{TealBlue!30}{0.0} & \cellcolor{TealBlue!30}{1.000} & \cellcolor{TealBlue!30}{6.0} & \cellcolor{TealBlue!30}{\textbf{7.4}} & \cellcolor{TealBlue!30}{\textbf{1510{\sc k}}} & \cellcolor{TealBlue!30}{0.0} & \cellcolor{TealBlue!30}{0.0} & \cellcolor{TealBlue!30}{1.000} & \cellcolor{TealBlue!30}{6.0} & 139.5 & 29{\sc m}\\
\texttt{australian-credit} & \multicolumn{1}{r}{653} & \multicolumn{1}{r}{124}  & \cellcolor{TealBlue!30}{0.0} & \cellcolor{TealBlue!30}{40.0} & \cellcolor{TealBlue!30}{0.939} & \cellcolor{TealBlue!30}{8.0} & \cellcolor{TealBlue!30}{\textbf{60.2}} & \cellcolor{TealBlue!30}{\textbf{10{\sc m}}} & \cellcolor{TealBlue!30}{0.0} & \cellcolor{TealBlue!30}{40.0} & \cellcolor{TealBlue!30}{0.939} & \cellcolor{TealBlue!30}{8.0} & 683.3 & 117{\sc m}\\
\texttt{breast-cancer} & \multicolumn{1}{r}{683} & \multicolumn{1}{r}{89}  & \cellcolor{TealBlue!30}{1.0} & \cellcolor{TealBlue!30}{6.0} & \cellcolor{TealBlue!30}{0.991} & \cellcolor{TealBlue!30}{9.0} & \cellcolor{TealBlue!30}{\textbf{815.8}} & \cellcolor{TealBlue!30}{\textbf{158{\sc m}}} & \cellcolor{TealBlue!30}{1.0} & \cellcolor{TealBlue!30}{6.0} & \cellcolor{TealBlue!30}{0.991} & \cellcolor{TealBlue!30}{9.0} & 1022.2 & 202{\sc m}\\
\texttt{car} & \multicolumn{1}{r}{1728} & \multicolumn{1}{r}{21}  & \cellcolor{TealBlue!30}{1.0} & \cellcolor{TealBlue!30}{86.0} & \cellcolor{TealBlue!30}{0.950} & \cellcolor{TealBlue!30}{9.0} & \cellcolor{TealBlue!30}{\textbf{4.7}} & \cellcolor{TealBlue!30}{\textbf{1255{\sc k}}} & \cellcolor{TealBlue!30}{1.0} & \cellcolor{TealBlue!30}{86.0} & \cellcolor{TealBlue!30}{0.950} & \cellcolor{TealBlue!30}{9.0} & 8.8 & 2286{\sc k}\\
\texttt{forest-fires} & \multicolumn{1}{r}{517} & \multicolumn{1}{r}{989}  & \cellcolor{TealBlue!30}{0.0} & \cellcolor{TealBlue!30}{\textbf{156.6}} & \cellcolor{TealBlue!30}{\textbf{0.697}} & \cellcolor{TealBlue!30}{\textbf{9.0}} & 933.6 & 45{\sc m} & \cellcolor{TealBlue!30}{0.0} & 162.1 & 0.686 & 13.6 & \cellcolor{TealBlue!30}{\textbf{533.4}} & \cellcolor{TealBlue!30}{\textbf{26{\sc m}}}\\
\texttt{heart-cleveland} & \multicolumn{1}{r}{296} & \multicolumn{1}{r}{95}  & \cellcolor{TealBlue!30}{\textbf{0.1}} & \cellcolor{TealBlue!30}{7.0} & \cellcolor{TealBlue!30}{0.976} & \cellcolor{TealBlue!30}{9.0} & \cellcolor{TealBlue!30}{\textbf{279.9}} & \cellcolor{TealBlue!30}{\textbf{63{\sc m}}} & 0.0 & \cellcolor{TealBlue!30}{7.0} & \cellcolor{TealBlue!30}{0.976} & \cellcolor{TealBlue!30}{9.0} & 389.5 & 89{\sc m}\\
\texttt{hypothyroid} & \multicolumn{1}{r}{3247} & \multicolumn{1}{r}{83}  & \cellcolor{TealBlue!30}{0.0} & \cellcolor{TealBlue!30}{\textbf{44.0}} & \cellcolor{TealBlue!30}{\textbf{0.986}} & \cellcolor{TealBlue!30}{9.0} & 1519.0 & 103{\sc m} & \cellcolor{TealBlue!30}{0.0} & 45.0 & 0.986 & \cellcolor{TealBlue!30}{9.0} & \cellcolor{TealBlue!30}{\textbf{315.6}} & \cellcolor{TealBlue!30}{\textbf{16{\sc m}}}\\
\texttt{kr-vs-kp} & \multicolumn{1}{r}{3196} & \multicolumn{1}{r}{73}  & \cellcolor{TealBlue!30}{\textbf{0.2}} & \cellcolor{TealBlue!30}{81.0} & \cellcolor{TealBlue!30}{0.975} & \cellcolor{TealBlue!30}{7.0} & 402.1 & 31{\sc m} & 0.0 & \cellcolor{TealBlue!30}{81.0} & \cellcolor{TealBlue!30}{0.975} & \cellcolor{TealBlue!30}{7.0} & \cellcolor{TealBlue!30}{\textbf{243.4}} & \cellcolor{TealBlue!30}{\textbf{18{\sc m}}}\\
\texttt{lymph} & \multicolumn{1}{r}{148} & \multicolumn{1}{r}{68}  & \cellcolor{TealBlue!30}{1.0} & \cellcolor{TealBlue!30}{0.0} & \cellcolor{TealBlue!30}{1.000} & \cellcolor{TealBlue!30}{6.0} & \cellcolor{TealBlue!30}{\textbf{237.9}} & \cellcolor{TealBlue!30}{\textbf{79{\sc m}}} & \cellcolor{TealBlue!30}{1.0} & \cellcolor{TealBlue!30}{0.0} & \cellcolor{TealBlue!30}{1.000} & \cellcolor{TealBlue!30}{6.0} & 302.4 & 102{\sc m}\\
\texttt{mushroom} & \multicolumn{1}{r}{8124} & \multicolumn{1}{r}{111}  & \cellcolor{TealBlue!30}{0.0} & \cellcolor{TealBlue!30}{0.0} & \cellcolor{TealBlue!30}{1.000} & \cellcolor{TealBlue!30}{4.0} & \cellcolor{TealBlue!30}{\textbf{76.0}} & \cellcolor{TealBlue!30}{\textbf{2059{\sc k}}} & \cellcolor{TealBlue!30}{0.0} & \cellcolor{TealBlue!30}{0.0} & \cellcolor{TealBlue!30}{1.000} & \cellcolor{TealBlue!30}{4.0} & 1344.4 & 30{\sc m}\\
\texttt{primary-tumor} & \multicolumn{1}{r}{336} & \multicolumn{1}{r}{31}  & \cellcolor{TealBlue!30}{1.0} & \cellcolor{TealBlue!30}{26.0} & \cellcolor{TealBlue!30}{0.923} & \cellcolor{TealBlue!30}{9.0} & \cellcolor{TealBlue!30}{\textbf{9.7}} & \cellcolor{TealBlue!30}{\textbf{4936{\sc k}}} & \cellcolor{TealBlue!30}{1.0} & \cellcolor{TealBlue!30}{26.0} & \cellcolor{TealBlue!30}{0.923} & \cellcolor{TealBlue!30}{9.0} & 15.1 & 7745{\sc k}\\
\texttt{soybean} & \multicolumn{1}{r}{630} & \multicolumn{1}{r}{50}  & \cellcolor{TealBlue!30}{1.0} & \cellcolor{TealBlue!30}{8.0} & \cellcolor{TealBlue!30}{0.987} & \cellcolor{TealBlue!30}{8.0} & \cellcolor{TealBlue!30}{\textbf{65.0}} & \cellcolor{TealBlue!30}{\textbf{19{\sc m}}} & \cellcolor{TealBlue!30}{1.0} & \cellcolor{TealBlue!30}{8.0} & \cellcolor{TealBlue!30}{0.987} & \cellcolor{TealBlue!30}{8.0} & 87.8 & 26{\sc m}\\
\texttt{splice-1} & \multicolumn{1}{r}{3190} & \multicolumn{1}{r}{287}  & \cellcolor{TealBlue!30}{0.0} & 103.8 & 0.967 & 12.5 & 1437.2 & 93{\sc m} & \cellcolor{TealBlue!30}{0.0} & \cellcolor{TealBlue!30}{\textbf{103.4}} & \cellcolor{TealBlue!30}{\textbf{0.968}} & \cellcolor{TealBlue!30}{\textbf{11.8}} & \cellcolor{TealBlue!30}{\textbf{424.3}} & \cellcolor{TealBlue!30}{\textbf{21{\sc m}}}\\
\texttt{tic-tac-toe} & \multicolumn{1}{r}{958} & \multicolumn{1}{r}{27}  & \cellcolor{TealBlue!30}{1.0} & \cellcolor{TealBlue!30}{63.0} & \cellcolor{TealBlue!30}{0.934} & 8.5 & \cellcolor{TealBlue!30}{\textbf{13.2}} & \cellcolor{TealBlue!30}{\textbf{4990{\sc k}}} & \cellcolor{TealBlue!30}{1.0} & \cellcolor{TealBlue!30}{63.0} & \cellcolor{TealBlue!30}{0.934} & \cellcolor{TealBlue!30}{\textbf{8.2}} & 20.5 & 7985{\sc k}\\
\texttt{vote} & \multicolumn{1}{r}{435} & \multicolumn{1}{r}{48}  & \cellcolor{TealBlue!30}{1.0} & \cellcolor{TealBlue!30}{1.0} & \cellcolor{TealBlue!30}{0.998} & \cellcolor{TealBlue!30}{8.0} & \cellcolor{TealBlue!30}{\textbf{44.0}} & \cellcolor{TealBlue!30}{\textbf{15{\sc m}}} & \cellcolor{TealBlue!30}{1.0} & \cellcolor{TealBlue!30}{1.0} & \cellcolor{TealBlue!30}{0.998} & \cellcolor{TealBlue!30}{8.0} & 58.8 & 21{\sc m}\\
\texttt{wine1} & \multicolumn{1}{r}{178} & \multicolumn{1}{r}{1276}  & \cellcolor{TealBlue!30}{0.0} & \cellcolor{TealBlue!30}{\textbf{34.0}} & \cellcolor{TealBlue!30}{\textbf{0.809}} & 9.6 & 375.3 & 16{\sc m} & \cellcolor{TealBlue!30}{0.0} & 35.0 & 0.803 & \cellcolor{TealBlue!30}{\textbf{8.0}} & \cellcolor{TealBlue!30}{\textbf{262.9}} & \cellcolor{TealBlue!30}{\textbf{11{\sc m}}}\\
\texttt{wine2} & \multicolumn{1}{r}{178} & \multicolumn{1}{r}{1276}  & \cellcolor{TealBlue!30}{0.0} & \cellcolor{TealBlue!30}{37.0} & \cellcolor{TealBlue!30}{0.792} & \cellcolor{TealBlue!30}{9.0} & 56.5 & 2314{\sc k} & \cellcolor{TealBlue!30}{0.0} & \cellcolor{TealBlue!30}{37.0} & \cellcolor{TealBlue!30}{0.792} & \cellcolor{TealBlue!30}{9.0} & \cellcolor{TealBlue!30}{\textbf{2.7}} & \cellcolor{TealBlue!30}{\textbf{102{\sc k}}}\\
\texttt{wine3} & \multicolumn{1}{r}{178} & \multicolumn{1}{r}{1276}  & \cellcolor{TealBlue!30}{0.0} & \cellcolor{TealBlue!30}{\textbf{25.9}} & \cellcolor{TealBlue!30}{\textbf{0.854}} & 8.9 & 251.2 & 11{\sc m} & \cellcolor{TealBlue!30}{0.0} & 26.7 & 0.850 & \cellcolor{TealBlue!30}{\textbf{7.1}} & \cellcolor{TealBlue!30}{\textbf{176.2}} & \cellcolor{TealBlue!30}{\textbf{8446{\sc k}}}\\
\texttt{zoo-1} & \multicolumn{1}{r}{101} & \multicolumn{1}{r}{36}  & \cellcolor{TealBlue!30}{1.0} & \cellcolor{TealBlue!30}{0.0} & \cellcolor{TealBlue!30}{1.000} & \cellcolor{TealBlue!30}{1.0} & 0.0 & \cellcolor{TealBlue!30}{1} & \cellcolor{TealBlue!30}{1.0} & \cellcolor{TealBlue!30}{0.0} & \cellcolor{TealBlue!30}{1.000} & \cellcolor{TealBlue!30}{1.0} & \cellcolor{TealBlue!30}{\textbf{0.0}} & \cellcolor{TealBlue!30}{1}\\
\bottomrule
\end{tabular}

% \end{footnotesize}
% \end{center}
% \caption{\label{tab:thetable} Restarts (max depth=5)}
% \end{table}

% \clearpage

% \begin{table}[htbp]
% \begin{center}
% \begin{footnotesize}
% \tabcolsep=5pt
% \begin{tabular}{lccrrrrrrrrrrrr}
\toprule
& && \multicolumn{6}{c}{dt no restart} & \multicolumn{6}{c}{dt restarts (1.1)}\\
\cmidrule(rr){4-9}\cmidrule(rr){10-15}
&\multirow{1}{*}{$\#ex.$} & \multirow{1}{*}{\#feat.} &  \multicolumn{1}{c}{opt} & \multicolumn{1}{c}{error} & \multicolumn{1}{c}{acc.} & \multicolumn{1}{c}{size} & \multicolumn{1}{c}{time} & \multicolumn{1}{c}{choices} & \multicolumn{1}{c}{opt} & \multicolumn{1}{c}{error} & \multicolumn{1}{c}{acc.} & \multicolumn{1}{c}{size} & \multicolumn{1}{c}{time} & \multicolumn{1}{c}{choices} \\
\midrule

\texttt{anneal} & \multicolumn{1}{r}{812} & \multicolumn{1}{r}{88}  & \cellcolor{TealBlue!30}{0.0} & \cellcolor{TealBlue!30}{\textbf{64.0}} & \cellcolor{TealBlue!30}{\textbf{0.921}} & 12.9 & 1435.5 & 258{\sc m} & \cellcolor{TealBlue!30}{0.0} & 65.0 & 0.920 & \cellcolor{TealBlue!30}{\textbf{12.7}} & \cellcolor{TealBlue!30}{\textbf{826.5}} & \cellcolor{TealBlue!30}{\textbf{136{\sc m}}}\\
\texttt{audiology} & \multicolumn{1}{r}{216} & \multicolumn{1}{r}{145}  & \cellcolor{TealBlue!30}{0.0} & \cellcolor{TealBlue!30}{0.0} & \cellcolor{TealBlue!30}{1.000} & 9.2 & \cellcolor{TealBlue!30}{\textbf{33.6}} & \cellcolor{TealBlue!30}{\textbf{8518{\sc k}}} & \cellcolor{TealBlue!30}{0.0} & \cellcolor{TealBlue!30}{0.0} & \cellcolor{TealBlue!30}{1.000} & \cellcolor{TealBlue!30}{\textbf{9.0}} & 410.9 & 107{\sc m}\\
\texttt{australian-credit} & \multicolumn{1}{r}{653} & \multicolumn{1}{r}{124}  & \cellcolor{TealBlue!30}{0.0} & \cellcolor{TealBlue!30}{\textbf{0.5}} & \cellcolor{TealBlue!30}{\textbf{0.999}} & \cellcolor{TealBlue!30}{\textbf{12.4}} & 1457.1 & 307{\sc m} & \cellcolor{TealBlue!30}{0.0} & 3.1 & 0.995 & 16.5 & \cellcolor{TealBlue!30}{\textbf{711.8}} & \cellcolor{TealBlue!30}{\textbf{139{\sc m}}}\\
\texttt{breast-cancer} & \multicolumn{1}{r}{683} & \multicolumn{1}{r}{89}  & \cellcolor{TealBlue!30}{0.0} & \cellcolor{TealBlue!30}{0.0} & \cellcolor{TealBlue!30}{1.000} & \cellcolor{TealBlue!30}{\textbf{13.1}} & 1084.6 & 330{\sc m} & \cellcolor{TealBlue!30}{0.0} & \cellcolor{TealBlue!30}{0.0} & \cellcolor{TealBlue!30}{1.000} & 15.5 & \cellcolor{TealBlue!30}{\textbf{489.4}} & \cellcolor{TealBlue!30}{\textbf{161{\sc m}}}\\
\texttt{car} & \multicolumn{1}{r}{1728} & \multicolumn{1}{r}{21}  & \cellcolor{TealBlue!30}{0.0} & \cellcolor{TealBlue!30}{\textbf{0.0}} & \cellcolor{TealBlue!30}{\textbf{1.000}} & \cellcolor{TealBlue!30}{\textbf{11.9}} & 934.2 & 542{\sc m} & \cellcolor{TealBlue!30}{0.0} & 2.5 & 0.999 & 13.9 & \cellcolor{TealBlue!30}{\textbf{343.9}} & \cellcolor{TealBlue!30}{\textbf{212{\sc m}}}\\
\texttt{forest-fires} & \multicolumn{1}{r}{517} & \multicolumn{1}{r}{989}  & \cellcolor{TealBlue!30}{0.0} & 145.7 & 0.718 & 26.3 & 742.3 & 38{\sc m} & \cellcolor{TealBlue!30}{0.0} & \cellcolor{TealBlue!30}{\textbf{127.7}} & \cellcolor{TealBlue!30}{\textbf{0.753}} & \cellcolor{TealBlue!30}{\textbf{25.7}} & \cellcolor{TealBlue!30}{\textbf{596.9}} & \cellcolor{TealBlue!30}{\textbf{30{\sc m}}}\\
\texttt{heart-cleveland} & \multicolumn{1}{r}{296} & \multicolumn{1}{r}{95}  & \cellcolor{TealBlue!30}{0.0} & \cellcolor{TealBlue!30}{0.0} & \cellcolor{TealBlue!30}{1.000} & \cellcolor{TealBlue!30}{\textbf{18.3}} & 1489.8 & 416{\sc m} & \cellcolor{TealBlue!30}{0.0} & \cellcolor{TealBlue!30}{0.0} & \cellcolor{TealBlue!30}{1.000} & 25.9 & \cellcolor{TealBlue!30}{\textbf{622.2}} & \cellcolor{TealBlue!30}{\textbf{211{\sc m}}}\\
\texttt{hypothyroid} & \multicolumn{1}{r}{3247} & \multicolumn{1}{r}{83}  & \cellcolor{TealBlue!30}{0.0} & 49.0 & 0.985 & \cellcolor{TealBlue!30}{\textbf{36.5}} & \cellcolor{TealBlue!30}{\textbf{91.5}} & \cellcolor{TealBlue!30}{\textbf{22{\sc m}}} & \cellcolor{TealBlue!30}{0.0} & \cellcolor{TealBlue!30}{\textbf{38.1}} & \cellcolor{TealBlue!30}{\textbf{0.988}} & 41.8 & 259.4 & 23{\sc m}\\
\texttt{kr-vs-kp} & \multicolumn{1}{r}{3196} & \multicolumn{1}{r}{73}  & \cellcolor{TealBlue!30}{0.0} & \cellcolor{TealBlue!30}{\textbf{29.7}} & \cellcolor{TealBlue!30}{\textbf{0.991}} & 17.8 & \cellcolor{TealBlue!30}{\textbf{237.2}} & \cellcolor{TealBlue!30}{\textbf{47{\sc m}}} & \cellcolor{TealBlue!30}{0.0} & 45.9 & 0.986 & \cellcolor{TealBlue!30}{\textbf{17.2}} & 1017.8 & 218{\sc m}\\
\texttt{lymph} & \multicolumn{1}{r}{148} & \multicolumn{1}{r}{68}  & \cellcolor{TealBlue!30}{0.0} & \cellcolor{TealBlue!30}{0.0} & \cellcolor{TealBlue!30}{1.000} & 11.6 & 1526.6 & 640{\sc m} & \cellcolor{TealBlue!30}{0.0} & \cellcolor{TealBlue!30}{0.0} & \cellcolor{TealBlue!30}{1.000} & \cellcolor{TealBlue!30}{\textbf{11.3}} & \cellcolor{TealBlue!30}{\textbf{517.0}} & \cellcolor{TealBlue!30}{\textbf{225{\sc m}}}\\
\texttt{mushroom} & \multicolumn{1}{r}{8124} & \multicolumn{1}{r}{111}  & \cellcolor{TealBlue!30}{0.0} & \cellcolor{TealBlue!30}{0.0} & \cellcolor{TealBlue!30}{1.000} & 8.1 & \cellcolor{TealBlue!30}{\textbf{293.2}} & \cellcolor{TealBlue!30}{\textbf{9653{\sc k}}} & \cellcolor{TealBlue!30}{0.0} & \cellcolor{TealBlue!30}{0.0} & \cellcolor{TealBlue!30}{1.000} & \cellcolor{TealBlue!30}{\textbf{8.0}} & 394.5 & 25{\sc m}\\
\texttt{primary-tumor} & \multicolumn{1}{r}{336} & \multicolumn{1}{r}{31}  & \cellcolor{TealBlue!30}{0.0} & \cellcolor{TealBlue!30}{15.0} & \cellcolor{TealBlue!30}{0.955} & 17.3 & 929.2 & 599{\sc m} & \cellcolor{TealBlue!30}{0.0} & \cellcolor{TealBlue!30}{15.0} & \cellcolor{TealBlue!30}{0.955} & \cellcolor{TealBlue!30}{\textbf{16.6}} & \cellcolor{TealBlue!30}{\textbf{738.1}} & \cellcolor{TealBlue!30}{\textbf{481{\sc m}}}\\
\texttt{soybean} & \multicolumn{1}{r}{630} & \multicolumn{1}{r}{50}  & \cellcolor{TealBlue!30}{0.0} & \cellcolor{TealBlue!30}{2.0} & \cellcolor{TealBlue!30}{0.997} & \cellcolor{TealBlue!30}{\textbf{11.1}} & 879.0 & \cellcolor{TealBlue!30}{\textbf{329{\sc m}}} & \cellcolor{TealBlue!30}{0.0} & \cellcolor{TealBlue!30}{2.0} & \cellcolor{TealBlue!30}{0.997} & 13.6 & \cellcolor{TealBlue!30}{\textbf{848.1}} & 329{\sc m}\\
\texttt{splice-1} & \multicolumn{1}{r}{3190} & \multicolumn{1}{r}{287}  & \cellcolor{TealBlue!30}{0.0} & 85.6 & 0.973 & \cellcolor{TealBlue!30}{\textbf{42.6}} & 1099.9 & 110{\sc m} & \cellcolor{TealBlue!30}{0.0} & \cellcolor{TealBlue!30}{\textbf{45.6}} & \cellcolor{TealBlue!30}{\textbf{0.986}} & 43.7 & \cellcolor{TealBlue!30}{\textbf{516.2}} & \cellcolor{TealBlue!30}{\textbf{48{\sc m}}}\\
\texttt{tic-tac-toe} & \multicolumn{1}{r}{958} & \multicolumn{1}{r}{27}  & \cellcolor{TealBlue!30}{0.0} & \cellcolor{TealBlue!30}{0.0} & \cellcolor{TealBlue!30}{1.000} & \cellcolor{TealBlue!30}{\textbf{19.1}} & \cellcolor{TealBlue!30}{\textbf{306.1}} & \cellcolor{TealBlue!30}{\textbf{219{\sc m}}} & \cellcolor{TealBlue!30}{0.0} & \cellcolor{TealBlue!30}{0.0} & \cellcolor{TealBlue!30}{1.000} & 25.4 & 329.7 & 253{\sc m}\\
\texttt{vote} & \multicolumn{1}{r}{435} & \multicolumn{1}{r}{48}  & \cellcolor{TealBlue!30}{0.0} & \cellcolor{TealBlue!30}{0.0} & \cellcolor{TealBlue!30}{1.000} & \cellcolor{TealBlue!30}{\textbf{12.9}} & \cellcolor{TealBlue!30}{\textbf{330.8}} & \cellcolor{TealBlue!30}{\textbf{202{\sc m}}} & \cellcolor{TealBlue!30}{0.0} & \cellcolor{TealBlue!30}{0.0} & \cellcolor{TealBlue!30}{1.000} & 14.2 & 574.8 & 376{\sc m}\\
\texttt{wine1} & \multicolumn{1}{r}{178} & \multicolumn{1}{r}{1276}  & \cellcolor{TealBlue!30}{0.0} & \cellcolor{TealBlue!30}{26.0} & \cellcolor{TealBlue!30}{0.854} & \cellcolor{TealBlue!30}{12.0} & \cellcolor{TealBlue!30}{\textbf{16.7}} & \cellcolor{TealBlue!30}{\textbf{695{\sc k}}} & \cellcolor{TealBlue!30}{0.0} & \cellcolor{TealBlue!30}{26.0} & \cellcolor{TealBlue!30}{0.854} & \cellcolor{TealBlue!30}{12.0} & 128.7 & 5508{\sc k}\\
\texttt{wine2} & \multicolumn{1}{r}{178} & \multicolumn{1}{r}{1276}  & \cellcolor{TealBlue!30}{0.0} & 27.3 & 0.847 & 13.6 & \cellcolor{TealBlue!30}{\textbf{259.2}} & \cellcolor{TealBlue!30}{\textbf{11{\sc m}}} & \cellcolor{TealBlue!30}{0.0} & \cellcolor{TealBlue!30}{\textbf{26.7}} & \cellcolor{TealBlue!30}{\textbf{0.850}} & \cellcolor{TealBlue!30}{\textbf{13.3}} & 347.5 & 15{\sc m}\\
\texttt{wine3} & \multicolumn{1}{r}{178} & \multicolumn{1}{r}{1276}  & \cellcolor{TealBlue!30}{0.0} & 20.8 & 0.883 & \cellcolor{TealBlue!30}{11.2} & \cellcolor{TealBlue!30}{\textbf{72.4}} & \cellcolor{TealBlue!30}{\textbf{3180{\sc k}}} & \cellcolor{TealBlue!30}{0.0} & \cellcolor{TealBlue!30}{\textbf{19.3}} & \cellcolor{TealBlue!30}{\textbf{0.892}} & \cellcolor{TealBlue!30}{11.2} & 631.9 & 28{\sc m}\\
\texttt{zoo-1} & \multicolumn{1}{r}{101} & \multicolumn{1}{r}{36}  & \cellcolor{TealBlue!30}{1.0} & \cellcolor{TealBlue!30}{0.0} & \cellcolor{TealBlue!30}{1.000} & \cellcolor{TealBlue!30}{1.0} & 0.0 & \cellcolor{TealBlue!30}{1} & \cellcolor{TealBlue!30}{1.0} & \cellcolor{TealBlue!30}{0.0} & \cellcolor{TealBlue!30}{1.000} & \cellcolor{TealBlue!30}{1.0} & \cellcolor{TealBlue!30}{\textbf{0.0}} & \cellcolor{TealBlue!30}{1}\\
\bottomrule
\end{tabular}

% \end{footnotesize}
% \end{center}
% \caption{\label{tab:thetable} Restarts (max depth=8)}
% \end{table}

% \begin{table}[htbp]
% \begin{center}
% \begin{footnotesize}
% \tabcolsep=5pt
% \begin{tabular}{lccrrrrrrrr}
\toprule
& && \multicolumn{4}{c}{\dleight} & \multicolumn{4}{c}{\budalg}\\
\cmidrule(rr){4-7}\cmidrule(rr){8-11}
&\multirow{1}{*}{$\#ex.$} & \multirow{1}{*}{\#feat.} &  \multicolumn{1}{c}{opt} & \multicolumn{1}{c}{error} & \multicolumn{1}{c}{acc.} & \multicolumn{1}{c}{time} & \multicolumn{1}{c}{opt} & \multicolumn{1}{c}{error} & \multicolumn{1}{c}{acc.} & \multicolumn{1}{c}{time} \\
\midrule

\texttt{anneal} & \multicolumn{1}{r}{812} & \multicolumn{1}{r}{47}  & - & - & - & - & \cellcolor{TealBlue!30}{\textbf{0}} & \cellcolor{TealBlue!30}{\textbf{53}} & \cellcolor{TealBlue!30}{\textbf{0.935}} & \cellcolor{TealBlue!30}{\textbf{310.0}}\\
\texttt{audiology} & \multicolumn{1}{r}{216} & \multicolumn{1}{r}{79}  & - & - & - & - & \cellcolor{TealBlue!30}{\textbf{1}} & \cellcolor{TealBlue!30}{\textbf{0}} & \cellcolor{TealBlue!30}{\textbf{1.000}} & \cellcolor{TealBlue!30}{\textbf{0.0}}\\
\texttt{australian-credit} & \multicolumn{1}{r}{653} & \multicolumn{1}{r}{73}  & - & - & - & - & \cellcolor{TealBlue!30}{\textbf{1}} & \cellcolor{TealBlue!30}{\textbf{0}} & \cellcolor{TealBlue!30}{\textbf{1.000}} & \cellcolor{TealBlue!30}{\textbf{0.0}}\\
\texttt{breast-cancer-un} & \multicolumn{1}{r}{683} & \multicolumn{1}{r}{89}  & - & - & - & - & \cellcolor{TealBlue!30}{\textbf{1}} & \cellcolor{TealBlue!30}{\textbf{0}} & \cellcolor{TealBlue!30}{\textbf{1.000}} & \cellcolor{TealBlue!30}{\textbf{0.0}}\\
\texttt{breast-wisconsin} & \multicolumn{1}{r}{683} & \multicolumn{1}{r}{120}  & - & - & - & - & \cellcolor{TealBlue!30}{\textbf{1}} & \cellcolor{TealBlue!30}{\textbf{0}} & \cellcolor{TealBlue!30}{\textbf{1.000}} & \cellcolor{TealBlue!30}{\textbf{0.0}}\\
\texttt{car-un} & \multicolumn{1}{r}{1728} & \multicolumn{1}{r}{21}  & - & - & - & - & \cellcolor{TealBlue!30}{\textbf{1}} & \cellcolor{TealBlue!30}{\textbf{0}} & \cellcolor{TealBlue!30}{\textbf{1.000}} & \cellcolor{TealBlue!30}{\textbf{0.2}}\\
\texttt{diabetes} & \multicolumn{1}{r}{768} & \multicolumn{1}{r}{112}  & - & - & - & - & \cellcolor{TealBlue!30}{\textbf{1}} & \cellcolor{TealBlue!30}{\textbf{0}} & \cellcolor{TealBlue!30}{\textbf{1.000}} & \cellcolor{TealBlue!30}{\textbf{0.5}}\\
\texttt{forest-fires-un} & \multicolumn{1}{r}{517} & \multicolumn{1}{r}{989}  & - & - & - & - & \cellcolor{TealBlue!30}{\textbf{0}} & \cellcolor{TealBlue!30}{\textbf{118}} & \cellcolor{TealBlue!30}{\textbf{0.772}} & \cellcolor{TealBlue!30}{\textbf{830.0}}\\
\texttt{german-credit} & \multicolumn{1}{r}{1000} & \multicolumn{1}{r}{110}  & - & - & - & - & \cellcolor{TealBlue!30}{\textbf{1}} & \cellcolor{TealBlue!30}{\textbf{0}} & \cellcolor{TealBlue!30}{\textbf{1.000}} & \cellcolor{TealBlue!30}{\textbf{67.0}}\\
\texttt{heart-cleveland} & \multicolumn{1}{r}{296} & \multicolumn{1}{r}{50}  & - & - & - & - & \cellcolor{TealBlue!30}{\textbf{1}} & \cellcolor{TealBlue!30}{\textbf{0}} & \cellcolor{TealBlue!30}{\textbf{1.000}} & \cellcolor{TealBlue!30}{\textbf{0.0}}\\
\texttt{hepatitis} & \multicolumn{1}{r}{137} & \multicolumn{1}{r}{68}  & - & - & - & - & \cellcolor{TealBlue!30}{\textbf{1}} & \cellcolor{TealBlue!30}{\textbf{0}} & \cellcolor{TealBlue!30}{\textbf{1.000}} & \cellcolor{TealBlue!30}{\textbf{0.0}}\\
\texttt{hypothyroid} & \multicolumn{1}{r}{3247} & \multicolumn{1}{r}{43}  & - & - & - & - & \cellcolor{TealBlue!30}{\textbf{0}} & \cellcolor{TealBlue!30}{\textbf{31}} & \cellcolor{TealBlue!30}{\textbf{0.990}} & \cellcolor{TealBlue!30}{\textbf{1.4}}\\
\texttt{ionosphere} & \multicolumn{1}{r}{351} & \multicolumn{1}{r}{444}  & - & - & - & - & \cellcolor{TealBlue!30}{\textbf{1}} & \cellcolor{TealBlue!30}{\textbf{0}} & \cellcolor{TealBlue!30}{\textbf{1.000}} & \cellcolor{TealBlue!30}{\textbf{0.0}}\\
\texttt{kr-vs-kp} & \multicolumn{1}{r}{3196} & \multicolumn{1}{r}{37}  & - & - & - & - & \cellcolor{TealBlue!30}{\textbf{0}} & \cellcolor{TealBlue!30}{\textbf{1}} & \cellcolor{TealBlue!30}{\textbf{1.000}} & \cellcolor{TealBlue!30}{\textbf{224.0}}\\
\texttt{letter} & \multicolumn{1}{r}{20000} & \multicolumn{1}{r}{224}  & - & - & - & - & \cellcolor{TealBlue!30}{\textbf{1}} & \cellcolor{TealBlue!30}{\textbf{0}} & \cellcolor{TealBlue!30}{\textbf{1.000}} & \cellcolor{TealBlue!30}{\textbf{58.4}}\\
\texttt{lymph} & \multicolumn{1}{r}{148} & \multicolumn{1}{r}{41}  & - & - & - & - & \cellcolor{TealBlue!30}{\textbf{1}} & \cellcolor{TealBlue!30}{\textbf{0}} & \cellcolor{TealBlue!30}{\textbf{1.000}} & \cellcolor{TealBlue!30}{\textbf{0.0}}\\
\texttt{mushroom} & \multicolumn{1}{r}{8124} & \multicolumn{1}{r}{91}  & - & - & - & - & \cellcolor{TealBlue!30}{\textbf{1}} & \cellcolor{TealBlue!30}{\textbf{0}} & \cellcolor{TealBlue!30}{\textbf{1.000}} & \cellcolor{TealBlue!30}{\textbf{0.0}}\\
\texttt{pendigits} & \multicolumn{1}{r}{7494} & \multicolumn{1}{r}{216}  & - & - & - & - & \cellcolor{TealBlue!30}{\textbf{1}} & \cellcolor{TealBlue!30}{\textbf{0}} & \cellcolor{TealBlue!30}{\textbf{1.000}} & \cellcolor{TealBlue!30}{\textbf{0.0}}\\
\texttt{primary-tumor} & \multicolumn{1}{r}{336} & \multicolumn{1}{r}{16}  & - & - & - & - & \cellcolor{TealBlue!30}{\textbf{0}} & \cellcolor{TealBlue!30}{\textbf{15}} & \cellcolor{TealBlue!30}{\textbf{0.955}} & \cellcolor{TealBlue!30}{\textbf{144.0}}\\
\texttt{segment} & \multicolumn{1}{r}{2310} & \multicolumn{1}{r}{234}  & - & - & - & - & \cellcolor{TealBlue!30}{\textbf{1}} & \cellcolor{TealBlue!30}{\textbf{0}} & \cellcolor{TealBlue!30}{\textbf{1.000}} & \cellcolor{TealBlue!30}{\textbf{0.0}}\\
\texttt{soybean} & \multicolumn{1}{r}{630} & \multicolumn{1}{r}{34}  & - & - & - & - & \cellcolor{TealBlue!30}{\textbf{0}} & \cellcolor{TealBlue!30}{\textbf{2}} & \cellcolor{TealBlue!30}{\textbf{0.997}} & \cellcolor{TealBlue!30}{\textbf{149.0}}\\
\texttt{splice-1} & \multicolumn{1}{r}{3190} & \multicolumn{1}{r}{227}  & - & - & - & - & \cellcolor{TealBlue!30}{\textbf{0}} & \cellcolor{TealBlue!30}{\textbf{7}} & \cellcolor{TealBlue!30}{\textbf{0.998}} & \cellcolor{TealBlue!30}{\textbf{56.1}}\\
\texttt{taiwan\_binarised} & \multicolumn{1}{r}{30000} & \multicolumn{1}{r}{198}  & - & - & - & - & \cellcolor{TealBlue!30}{\textbf{0}} & \cellcolor{TealBlue!30}{\textbf{4564}} & \cellcolor{TealBlue!30}{\textbf{0.848}} & \cellcolor{TealBlue!30}{\textbf{200.0}}\\
\texttt{tic-tac-toe} & \multicolumn{1}{r}{958} & \multicolumn{1}{r}{18}  & - & - & - & - & \cellcolor{TealBlue!30}{\textbf{1}} & \cellcolor{TealBlue!30}{\textbf{0}} & \cellcolor{TealBlue!30}{\textbf{1.000}} & \cellcolor{TealBlue!30}{\textbf{0.0}}\\
\texttt{vehicle} & \multicolumn{1}{r}{846} & \multicolumn{1}{r}{252}  & - & - & - & - & \cellcolor{TealBlue!30}{\textbf{1}} & \cellcolor{TealBlue!30}{\textbf{0}} & \cellcolor{TealBlue!30}{\textbf{1.000}} & \cellcolor{TealBlue!30}{\textbf{0.0}}\\
\texttt{vote} & \multicolumn{1}{r}{435} & \multicolumn{1}{r}{32}  & - & - & - & - & \cellcolor{TealBlue!30}{\textbf{1}} & \cellcolor{TealBlue!30}{\textbf{0}} & \cellcolor{TealBlue!30}{\textbf{1.000}} & \cellcolor{TealBlue!30}{\textbf{0.0}}\\
\texttt{wine1-un} & \multicolumn{1}{r}{178} & \multicolumn{1}{r}{1276}  & - & - & - & - & \cellcolor{TealBlue!30}{\textbf{0}} & \cellcolor{TealBlue!30}{\textbf{22}} & \cellcolor{TealBlue!30}{\textbf{0.876}} & \cellcolor{TealBlue!30}{\textbf{515.0}}\\
\texttt{wine2-un} & \multicolumn{1}{r}{178} & \multicolumn{1}{r}{1276}  & - & - & - & - & \cellcolor{TealBlue!30}{\textbf{0}} & \cellcolor{TealBlue!30}{\textbf{24}} & \cellcolor{TealBlue!30}{\textbf{0.865}} & \cellcolor{TealBlue!30}{\textbf{340.0}}\\
\texttt{wine3-un} & \multicolumn{1}{r}{178} & \multicolumn{1}{r}{1276}  & - & - & - & - & \cellcolor{TealBlue!30}{\textbf{0}} & \cellcolor{TealBlue!30}{\textbf{16}} & \cellcolor{TealBlue!30}{\textbf{0.910}} & \cellcolor{TealBlue!30}{\textbf{260.0}}\\
\texttt{yeast} & \multicolumn{1}{r}{1484} & \multicolumn{1}{r}{89}  & - & - & - & - & \cellcolor{TealBlue!30}{\textbf{0}} & \cellcolor{TealBlue!30}{\textbf{104}} & \cellcolor{TealBlue!30}{\textbf{0.930}} & \cellcolor{TealBlue!30}{\textbf{72.3}}\\
\texttt{zoo-1} & \multicolumn{1}{r}{101} & \multicolumn{1}{r}{20}  & - & - & - & - & \cellcolor{TealBlue!30}{\textbf{1}} & \cellcolor{TealBlue!30}{\textbf{0}} & \cellcolor{TealBlue!30}{\textbf{1.000}} & \cellcolor{TealBlue!30}{\textbf{0.0}}\\
\bottomrule
\end{tabular}

% \end{footnotesize}
% \end{center}
% \caption{\label{tab:thetable} Restarts (max depth=10)}
% \end{table}



\end{document}

