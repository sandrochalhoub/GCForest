\documentclass{article}
\usepackage[usenames,dvipsnames,svgnames,table]{xcolor}%% http://ctan.org/pkg/xcolor
\usepackage[utf8]{inputenc}
\usepackage{xspace}
\usepackage{array}
%\usepackage{amsthm}
\usepackage{amsmath} 
\usepackage{amssymb} 
\usepackage[ruled,vlined]{algorithm2e}
\usepackage{booktabs}
\usepackage{multirow}
\usepackage{url}
\usepackage{tikz}
\usepackage{fp}
\usepackage{subfig}
\usetikzlibrary{arrows,shadows,fit,calc,positioning,decorations.pathreplacing,matrix,shapes,petri,topaths,fadings,mindmap,backgrounds,shapes.geometric}
\usepackage{geometry}
\usepackage{xifthen}
\usepackage{rotating}


\DeclareMathOperator*{\argmax}{arg\,max}
\DeclareMathOperator*{\argmin}{arg\,min}


\newcommand{\tru}[0]{\texttt{true}}
\newcommand{\fal}[0]{\texttt{false}}
% \newcommand{\tru}[0]{1}
% \newcommand{\fal}[0]{0}


\newcommand{\setex}[1]{\ensuremath{{\mathcal X}^{#1}}\xspace}
\newcommand{\posex}{{\setex{\top}}\xspace}
\newcommand{\negex}{{\setex{\bot}}\xspace}
\newcommand{\setcube}[1]{\ensuremath{{\mathcal C}^{#1}}\xspace}
\newcommand{\poscube}{{\setcube{1}}\xspace}
\newcommand{\negcube}{{\setcube{0}}\xspace}
\newcommand{\features}{\ensuremath{{\mathcal F}}\xspace}
\newcommand{\feat}{\ensuremath{f}}
\newcommand{\classifier}{\ensuremath{f}}
\newcommand{\lit}[1]{\ensuremath{l_{#1}}}
\newcommand{\var}{\ensuremath{x}}
\newcommand{\truelit}[1]{\ensuremath{\var_{#1}}}
\newcommand{\falselit}[1]{\ensuremath{\bar{\var_{#1}}}}
\newcommand{\ex}{\ensuremath{\var}}
\newcommand{\cube}{\ensuremath{c}}
\newcommand{\universe}{\ensuremath{{\mathcal U}}}
\newcommand{\entropy}[1]{\ensuremath{{H}({#1})}\xspace}
\newcommand{\probability}[1]{\ensuremath{{Pr}({#1})}\xspace}


\newcommand{\nodes}[0]{\ensuremath{{\cal N}}}
\newcommand{\blossom}[0]{\ensuremath{{\cal B}}}
\newcommand{\sequence}[0]{\ensuremath{{\cal S}}}

\newcommand{\anode}[0]{\ensuremath{i}}
\newcommand{\bnode}[0]{\ensuremath{b}}
\newcommand{\afeat}[0]{\ensuremath{f}}
\newcommand{\ub}[0]{\ensuremath{ub}}
\newcommand{\error}[0]{\ensuremath{error}}
\newcommand{\depth}[1][]{\ensuremath{\ifthenelse{\equal{#1}{}}{depth}{depth({#1})}}}
\newcommand{\test}[1][]{\ensuremath{\ifthenelse{\equal{#1}{}}{test}{test({#1})}}}
\newcommand{\dom}[1][]{\ensuremath{\ifthenelse{\equal{#1}{}}{dom}{dom({#1})}}}
\newcommand{\best}[1][]{\ensuremath{\ifthenelse{\equal{#1}{}}{{best}}{{best({#1})}}}}
\newcommand{\opt}[1][]{\ensuremath{\ifthenelse{\equal{#1}{}}{opt}{opt({#1})}}}
\newcommand{\child}[1][]{\ensuremath{\ifthenelse{\equal{#1}{}}{{child}}{{child({{#1}})}}}}


	\SetKwFunction{storebest}{storeBest}
	\SetKwFunction{solution}{solution}
	\SetKwFunction{deadend}{deadEnd}
	\SetKwFunction{expend}{expend}
	\SetKwFunction{backtrack}{backtrack}
	\SetKwFunction{dt}{BTDT}
	\SetKwFunction{pop}{pop}
	\SetKwFunction{push}{push}
	\SetKwFunction{prune}{prune}
	\SetKwFunction{error}{error}
	\SetKwFunction{grow}{grow}
	\SetKwFunction{branch}{branch}
	\SetKwFunction{select}{select\&remove}
	
	

\DontPrintSemicolon

\title{Backtracking DT}

% \author{Emmanuel Hebrard\inst{1} \and George Katsirelos\inst{2}}
% \institute{LAAS-CNRS, Universit\'e de Toulouse, CNRS, France, email: hebrard@laas.fr
  % \and MIAT, UR-875, INRA, France, email: gkatsi@gmail.com \footnote{The second author was partially supported by the french ``Agence nationale de
% la Recherche'', project DEMOGRAPH, reference ANR-16-C40-0028.}}

\begin{document}
	% \newgeometry{bottom=3cm,top=3cm}

\maketitle

\section*{Introduction}

We consider the problem of finding the bounded-depth decision tree of maximum accuracy.
The state of the art includes MIP approaches (BinOCT), a MaxSAT approach based on the SAT encoding proposed by Narodytska et al, and DL8.5.
The latter algorithm is by far the most efficient, however, it is not \emph{anytime}: the left branch must be optimally solved before a solution of the right branch can be found. Moreover the use of a cache structure means that it uses a lot of memory. This algorithm is practical for a maximum depth of 4 (although using gigabytes of memory) but often not much beyond. Therefore, in a number of cases, a greedy heuristic (such as CART) is still the best method in practice.

\medskip

In this note we introduce what is essentially an anytime version of DL8.5, without cache.
This algorithm therefore uses linear (in the size of the tree) memory and anytime, hence in principle strictly better than CART. Moreover, on instance where DL8.5 can find a solution, the algorithm described in this note is significantly faster (by about a factor 10).

\medskip

In a nutshell the algorithm we propose works as follows:



We start from a singleton set \blossom\ of open nodes or \emph{blossoms} (open nodes can be expended, we say that a node is \emph{closed} if it has an assigned feature test or a label). % The set \nodes\ contains all the nodes of the current tree, open or closed, it is initially equal to \blossom. Finally,
Moreover, we use an initially empty stack of decisions $\sequence$.

\begin{itemize}
	\item As long as there is a blossom ($\blossom \neq \emptyset$), we pick any one $\anode$, a feature $\afeat$ marked as \emph{available} for \anode, add the pair $(\anode,\afeat)$ to \sequence\ and expend the tree with two children nodes accordingly. 
	%These new nodes are added to $\nodes$. 
	They are added to $\blossom$ if their depth is strictly less than $k$ and if they are not label nodes.

\item As long as there is no blossom ($\blossom = \emptyset$), we pop the last assignment $(\anode,\afeat)$ from sequence
%undo the last assignment (say, $\anode=\afeat$), 
%prune the two subtrees of node \anode (i.e. remove the corresponding nodes to \nodes) 
and mark the feature $f$ not available for node $\anode$. If there is at least one available feature for node $\anode$, we add $\anode$ to $\blossom$. 
Otherwise, we mark $\anode$ as \emph{optimal} and store the minimum error recorded for any of the possible features.

\end{itemize}

\clearpage

\section*{Algorithm}

For readability, we cut the algorithm into four blocks. The initialisation procedure (Algorithm~\ref{alg:init}) set up the data structures used in all other procedures:
\begin{itemize}
	\item \sequence\ is simply the list of nodes in the current tree, ordered as they are explored.
	\item \nodes\ is the set of integers used to index a node of the current tree
	\item \blossom\ is the set of nodes which do no have an assigned test yet
	\item \depth\ stores the depth of a node
	\item \test\ stores the feature tested at a node
	\item \dom\ stores the set of possible features which have no yet been tried for this node 
	\item \best\ stores the error of the best subtree rooted at a node 
	\item \opt\ indicates whether the best subtree of a given node is optimal
	\item \child\ stores the children of a node (children can be nodes or $\{\top, \bot\}$)
	\item $\error{\anode}$ $\min(|\posex(\anode)|,|\negex(\anode)|)$
	\item $\error{\anode,\afeat}$ $\min(|\posex(\anode=\afeat)|,|\negex(\anode=\afeat)|)$
\end{itemize}

Algorithm~\ref{alg:search} is a bactracking procedure which expends a current decision tree
	
	\begin{algorithm}
		\caption{Data Structures\label{alg:init}}
		\TitleOfAlgo{Initialise}
		$\sequence \gets []$\;
		$\blossom \gets \emptyset$\;
		$\nodes \gets \emptyset$\;
		$\ub \gets \min(|\negex|,|\posex|)$\;
		$\error \gets ub$\;
		
		$\child \gets (\lambda : \mathbb{N} \times \{\fal, \tru\} \mapsto \emptyset)$\;
		$\depth \gets (\lambda : \mathbb{N} \mapsto 0)$\;
		
		$\test \gets (\lambda : \mathbb{N} \mapsto \emptyset)$\;
		$\dom \gets (\lambda : \mathbb{N} \mapsto \features)$\;
		
		$\best \gets (\lambda : \mathbb{N} \mapsto \infty)$\;
		$\opt \gets (\lambda : \mathbb{N} \mapsto \fal)$\;
	\end{algorithm}
	
	
	
	\begin{algorithm}
		\caption{Create a new node after branching\label{alg:alloc}}
		\TitleOfAlgo{\grow} 
	  \KwData{integer \anode}

		$\nodes \gets \nodes \cup \{\anode\}$\;
		$\dom[\anode] \gets \features$ sorted by decreasing conditional error $\min(|\posex(\anode=\afeat)|,|\negex(\anode=\afeat)|)$\;
		$\test[\anode] \gets \pop(\dom[\anode])$\;
		
	
		\eIf{$\depth[\anode]=k-1$ or $\error{\anode,\test[\anode]}$}
		{
			$\best[\anode] = \error{\anode,\test[\anode]}$\; %\min(|\posex(\anode=\test[\anode])|,|\negex(\anode=\test[\anode])|)$\;
			$\opt[\anode] = \tru$\;
			\ForEach{$branch \in \{\tru, \fal\}$}
			{
				% $\child[\anode,branch] \gets (|\posex(\anode=\test[\anode])| > |\negex(\anode=\test[\anode])|)$\;
				\lIf{$|\posex(\anode=\test[\anode])| > |\negex(\anode=\test[\anode])|$}{$\child[\anode,branch] \gets \top$}
				\lElse{$\child[\anode,branch] \gets \bot$}
			}
		}
		{
			$\blossom \gets \blossom \cup \{n\}$\;
			$\best[\anode] = \min(|\posex(\anode)|, |\negex(\anode)|)$\;
			$\opt[\anode] \gets \fal$\;
		}


	\end{algorithm}


	\begin{algorithm}
		\caption{Suppress a node and all its descendants\label{alg:free}}
		\TitleOfAlgo{\prune}
	  \KwData{integer \anode}

		$\blossom \gets \blossom \setminus \{\anode\}$\;
		$\nodes \gets \nodes \setminus \{\anode\}$\;

		\ForEach{$branch \in \{\tru, \fal\}$}
		{
		\lIf{$\child[\anode,branch] \not\in \{\top, \bot\}$}
		{
			$\prune{\child[\anode,branch]}$
		}
		}

		\lIf{$\depth[\anode] = k-1$ or $\opt[\anode]$}{$error \gets error - \best[\anode]$}

	\end{algorithm}


\begin{algorithm}
	\caption{Search loop\label{alg:search}}
  \TitleOfAlgo{\dt}
  \KwData{$\negex,\posex, k$}
  \KwResult{A decision tree}

	$\bnode \gets 0$\;
	$\posex(1),\negex(1) \gets \posex, \negex$\;
	$\grow{\blossom, \sequence, 1}$\;

	\While{\textbf{true}}{
		\eIf{$\blossom = \emptyset$} {
			$\ub \gets \min(\ub,\error)$\;
			$deadend \gets \fal$\;
			\Repeat{$deadend$}{
				\lIf{$\bnode > 0$}{$\opt[\bnode] \gets \tru$}
				\lIf{$\bnode = 1$}{\Return}
				$\bnode \gets \pop{\sequence}$\;
				$\best[\bnode] \gets \min(\best[\bnode], \best(\child[\bnode,\tru]) + \best(\child[\bnode,\fal]))$\;
				$\test[\bnode] \gets \pop{\dom[\bnode]}$\;
				$\prune(\child[\bnode,\tru])$\;
				$\prune(\child[\bnode,\fal])$\;

				$deadend \gets \best[\bnode] = 0 ~\vee~ \dom[\bnode] = \emptyset$\;
				\If{$deadend$}
				{
				$\opt[\bnode] \gets \tru$\;
				$\error \gets \error + \error{\bnode}$\; %$\best[\bnode]$\;
				}
			}
			$\blossom \gets \blossom \cup \{b\}$\;
			$\error \gets \error + \min(|\posex(\bnode)|, |\negex(\bnode)|)$\;
		}
		{
			\If{$b = 0$}{
				$b=\select{\blossom}$\;
				% $\blossom \gets \blossom \setminus \{b\}$\;
				$\push(\bnode,\sequence)$\;
			}
			$c_{\tru}, c_{\fal} = \argmin_{x,y}(\mathbb{N} \setminus \nodes)$\;
			$\posex(c_{\tru}),\negex(c_{\tru}),\posex(c_{\fal}),\negex(c_{\fal}) \gets \branch(\posex(\bnode),\negex(\bnode),\test[\bnode])$\;
			\ForEach{$branch \in \{\tru, \fal\}$}{
				\eIf{$\min(|\posex(c_{branch})|,|\negex(c_{branch})|) = 0$}
				{
					\lIf{$|\posex(c_{branch})|>|\negex(c_{branch})|$}{$\child[\bnode,branch] \gets \top$}
					\lElse{$\child[\bnode,branch] \gets \bot$}
				}{
					$\child[\bnode,branch] \gets c_{branch}$\;
					$\depth[c_{branch}] \gets \depth[\bnode]+1$\;
					$\grow(\blossom, \sequence, c_{branch})$\;
				}
			}
			$\bnode \gets 0$\;
		}
	}

\end{algorithm}

\section*{Experiments}


\begin{table}[htbp]
\begin{center}
\begin{footnotesize}
\tabcolsep=5pt
\begin{tabular}{lccrrrrrrrr}
\toprule
& && \multicolumn{8}{c}{dt}\\
\cmidrule(rr){4-11}
&\multirow{1}{*}{$\#ex.$} & \multirow{1}{*}{\#feat.} &  \multicolumn{1}{c}{optimal} & \multicolumn{1}{c}{error} & \multicolumn{1}{c}{accuracy} & \multicolumn{1}{c}{size} & \multicolumn{1}{c}{sol search} & \multicolumn{1}{c}{sol time} & \multicolumn{1}{c}{proof search} & \multicolumn{1}{c}{proof time} \\
\midrule

\texttt{anneal-un} & \multicolumn{1}{r}{812} & \multicolumn{1}{r}{88}  & 1 & 91 & 0.8879 & 7 & 1648151 & 10.60 & 2020429 & 13.00\\
\texttt{audiology-un} & \multicolumn{1}{r}{216} & \multicolumn{1}{r}{145}  & 1 & 1 & 0.9954 & 6 & 3539850 & 14.70 & 5667873 & 23.40\\
\texttt{australian-credit-un} & \multicolumn{1}{r}{653} & \multicolumn{1}{r}{124}  & 1 & 56 & 0.9142 & 7 & 6892551 & 43.60 & 9645105 & 61.70\\
\texttt{breast-cancer-un} & \multicolumn{1}{r}{683} & \multicolumn{1}{r}{89}  & 1 & 16 & 0.9766 & 6 & 58782 & 0.27 & 1873202 & 9.26\\
\texttt{car-un} & \multicolumn{1}{r}{1728} & \multicolumn{1}{r}{21}  & 1 & 136 & 0.9213 & 7 & 39791 & 0.25 & 47014 & 0.29\\
\texttt{forest-fires-un} & \multicolumn{1}{r}{517} & \multicolumn{1}{r}{989}  & 0 & 173 & 0.6654 & 7 & 1495134 & 32.20 & - & -\\
\texttt{heart-cleveland-un} & \multicolumn{1}{r}{296} & \multicolumn{1}{r}{95}  & 1 & 25 & 0.9155 & 7 & 1469572 & 6.28 & 4296958 & 19.10\\
\texttt{hypothyroid-un} & \multicolumn{1}{r}{3247} & \multicolumn{1}{r}{83}  & 1 & 53 & 0.9837 & 7 & 14405 & 0.65 & 2225349 & 42.40\\
\texttt{kr-vs-kp-un} & \multicolumn{1}{r}{3196} & \multicolumn{1}{r}{73}  & 1 & 144 & 0.9549 & 5 & 237983 & 3.75 & 1685678 & 27.70\\
\texttt{lymph-un} & \multicolumn{1}{r}{148} & \multicolumn{1}{r}{68}  & 1 & 3 & 0.9797 & 7 & 7824 & 0.04 & 931373 & 2.69\\
\texttt{mushroom-un} & \multicolumn{1}{r}{8124} & \multicolumn{1}{r}{111}  & 1 & 0 & 1.0000 & 7 & 111 & 0.70 & 111 & 0.70\\
\texttt{primary-tumor-un} & \multicolumn{1}{r}{336} & \multicolumn{1}{r}{31}  & 1 & 34 & 0.8988 & 7 & 2872 & 0.02 & 147401 & 0.30\\
\texttt{soybean-un} & \multicolumn{1}{r}{630} & \multicolumn{1}{r}{50}  & 1 & 14 & 0.9778 & 6 & 228212 & 0.92 & 403788 & 1.48\\
\texttt{splice-1-un} & \multicolumn{1}{r}{3190} & \multicolumn{1}{r}{287}  & 0 & 141 & 0.9558 & 7 & 268289 & 6.02 & - & -\\
\texttt{tic-tac-toe-un} & \multicolumn{1}{r}{958} & \multicolumn{1}{r}{27}  & 1 & 137 & 0.8570 & 6 & 17180 & 0.08 & 123924 & 0.46\\
\texttt{vote-un} & \multicolumn{1}{r}{435} & \multicolumn{1}{r}{48}  & 1 & 5 & 0.9885 & 6 & 35980 & 0.10 & 461878 & 1.34\\
\texttt{wine1-un} & \multicolumn{1}{r}{178} & \multicolumn{1}{r}{1276}  & 0 & 38 & 0.7865 & 7 & 28688277 & 558.00 & - & -\\
\texttt{wine2-un} & \multicolumn{1}{r}{178} & \multicolumn{1}{r}{1276}  & 0 & 43 & 0.7584 & 7 & 2481 & 0.24 & - & -\\
\texttt{wine3-un} & \multicolumn{1}{r}{178} & \multicolumn{1}{r}{1276}  & 0 & 28 & 0.8427 & 7 & 12840495 & 250.00 & - & -\\
\texttt{zoo-1-un} & \multicolumn{1}{r}{101} & \multicolumn{1}{r}{36}  & 1 & 0 & 1.0000 & 1 & 1 & 0.01 & 1 & 0.01\\
\bottomrule
\end{tabular}

\end{footnotesize}
\end{center}
\caption{\label{tab:thetable} Some results (max depth=4)}
\end{table}

\clearpage

\begin{table}[htbp]
\begin{center}
\begin{footnotesize}
\tabcolsep=5pt
\begin{tabular}{lccrrrrrrrrrr}
\toprule
& && \multicolumn{5}{c}{\budalg} & \multicolumn{5}{c}{\dleight}\\
\cmidrule(rr){4-8}\cmidrule(rr){9-13}
&\multirow{1}{*}{$\#ex.$} & \multirow{1}{*}{\#feat.} &  \multicolumn{1}{c}{opt} & \multicolumn{1}{c}{error} & \multicolumn{1}{c}{acc.} & \multicolumn{1}{c}{time} & \multicolumn{1}{c}{choices} & \multicolumn{1}{c}{opt} & \multicolumn{1}{c}{error} & \multicolumn{1}{c}{acc.} & \multicolumn{1}{c}{time} & \multicolumn{1}{c}{choices} \\
\midrule

\texttt{anneal} & \multicolumn{1}{r}{812} & \multicolumn{1}{r}{47}  & \cellcolor{TealBlue!30}{\textbf{1}} & \cellcolor{TealBlue!30}{\textbf{70}} & \cellcolor{TealBlue!30}{\textbf{0.914}} & \cellcolor{TealBlue!30}{\textbf{774.0}} & \cellcolor{TealBlue!30}{\textbf{164{\sc m}}} & - & - & - & - & -\\
\texttt{audiology} & \multicolumn{1}{r}{216} & \multicolumn{1}{r}{79}  & \cellcolor{TealBlue!30}{1} & \cellcolor{TealBlue!30}{0} & \cellcolor{TealBlue!30}{1.000} & \cellcolor{TealBlue!30}{\textbf{0.0}} & \cellcolor{TealBlue!30}{\textbf{508}} & \cellcolor{TealBlue!30}{1} & \cellcolor{TealBlue!30}{0} & \cellcolor{TealBlue!30}{1.000} & 0.0 & 21{\sc k}\\
\texttt{australian-credit} & \multicolumn{1}{r}{653} & \multicolumn{1}{r}{73}  & \cellcolor{TealBlue!30}{\textbf{0}} & \cellcolor{TealBlue!30}{\textbf{39}} & \cellcolor{TealBlue!30}{\textbf{0.940}} & \cellcolor{TealBlue!30}{\textbf{3320.0}} & \cellcolor{TealBlue!30}{\textbf{652{\sc m}}} & - & - & - & - & -\\
\texttt{breast-cancer-un} & \multicolumn{1}{r}{683} & \multicolumn{1}{r}{89}  & \cellcolor{TealBlue!30}{1} & \cellcolor{TealBlue!30}{6} & \cellcolor{TealBlue!30}{0.991} & 690.0 & 153{\sc m} & \cellcolor{TealBlue!30}{1} & \cellcolor{TealBlue!30}{6} & \cellcolor{TealBlue!30}{0.991} & \cellcolor{TealBlue!30}{\textbf{580.5}} & \cellcolor{TealBlue!30}{\textbf{52{\sc m}}}\\
\texttt{breast-wisconsin} & \multicolumn{1}{r}{683} & \multicolumn{1}{r}{120}  & \cellcolor{TealBlue!30}{\textbf{1}} & \cellcolor{TealBlue!30}{\textbf{0}} & \cellcolor{TealBlue!30}{\textbf{1.000}} & \cellcolor{TealBlue!30}{\textbf{302.0}} & \cellcolor{TealBlue!30}{\textbf{72{\sc m}}} & - & - & - & - & -\\
\texttt{car-un} & \multicolumn{1}{r}{1728} & \multicolumn{1}{r}{21}  & \cellcolor{TealBlue!30}{1} & \cellcolor{TealBlue!30}{86} & \cellcolor{TealBlue!30}{0.950} & 3.3 & 864{\sc k} & \cellcolor{TealBlue!30}{1} & \cellcolor{TealBlue!30}{86} & \cellcolor{TealBlue!30}{0.950} & \cellcolor{TealBlue!30}{\textbf{2.3}} & \cellcolor{TealBlue!30}{\textbf{215{\sc k}}}\\
\texttt{diabetes} & \multicolumn{1}{r}{768} & \multicolumn{1}{r}{112}  & \cellcolor{TealBlue!30}{\textbf{0}} & \cellcolor{TealBlue!30}{\textbf{106}} & \cellcolor{TealBlue!30}{\textbf{0.862}} & \cellcolor{TealBlue!30}{\textbf{2250.0}} & \cellcolor{TealBlue!30}{\textbf{447{\sc m}}} & - & - & - & - & -\\
\texttt{forest-fires-un} & \multicolumn{1}{r}{517} & \multicolumn{1}{r}{989}  & \cellcolor{TealBlue!30}{\textbf{0}} & \cellcolor{TealBlue!30}{\textbf{157}} & \cellcolor{TealBlue!30}{\textbf{0.696}} & \cellcolor{TealBlue!30}{\textbf{405.0}} & \cellcolor{TealBlue!30}{\textbf{48{\sc m}}} & - & - & - & - & -\\
\texttt{german-credit} & \multicolumn{1}{r}{1000} & \multicolumn{1}{r}{110}  & \cellcolor{TealBlue!30}{\textbf{0}} & \cellcolor{TealBlue!30}{\textbf{164}} & \cellcolor{TealBlue!30}{\textbf{0.836}} & \cellcolor{TealBlue!30}{\textbf{3140.0}} & \cellcolor{TealBlue!30}{\textbf{565{\sc m}}} & - & - & - & - & -\\
\texttt{heart-cleveland} & \multicolumn{1}{r}{296} & \multicolumn{1}{r}{50}  & \cellcolor{TealBlue!30}{\textbf{1}} & \cellcolor{TealBlue!30}{\textbf{7}} & \cellcolor{TealBlue!30}{\textbf{0.976}} & \cellcolor{TealBlue!30}{\textbf{946.0}} & \cellcolor{TealBlue!30}{\textbf{267{\sc m}}} & - & - & - & - & -\\
\texttt{hepatitis} & \multicolumn{1}{r}{137} & \multicolumn{1}{r}{68}  & \cellcolor{TealBlue!30}{1} & \cellcolor{TealBlue!30}{0} & \cellcolor{TealBlue!30}{1.000} & \cellcolor{TealBlue!30}{\textbf{1.1}} & \cellcolor{TealBlue!30}{\textbf{472{\sc k}}} & \cellcolor{TealBlue!30}{1} & \cellcolor{TealBlue!30}{0} & \cellcolor{TealBlue!30}{1.000} & 60.7 & 14{\sc m}\\
\texttt{hypothyroid} & \multicolumn{1}{r}{3247} & \multicolumn{1}{r}{43}  & \cellcolor{TealBlue!30}{\textbf{1}} & \cellcolor{TealBlue!30}{\textbf{44}} & \cellcolor{TealBlue!30}{\textbf{0.986}} & \cellcolor{TealBlue!30}{\textbf{2670.0}} & \cellcolor{TealBlue!30}{\textbf{187{\sc m}}} & - & - & - & - & -\\
\texttt{ionosphere} & \multicolumn{1}{r}{351} & \multicolumn{1}{r}{444}  & \cellcolor{TealBlue!30}{\textbf{0}} & \cellcolor{TealBlue!30}{\textbf{2}} & \cellcolor{TealBlue!30}{\textbf{0.994}} & \cellcolor{TealBlue!30}{\textbf{494.0}} & \cellcolor{TealBlue!30}{\textbf{27{\sc m}}} & - & - & - & - & -\\
\texttt{kr-vs-kp} & \multicolumn{1}{r}{3196} & \multicolumn{1}{r}{37}  & \cellcolor{TealBlue!30}{\textbf{1}} & \cellcolor{TealBlue!30}{\textbf{81}} & \cellcolor{TealBlue!30}{\textbf{0.975}} & \cellcolor{TealBlue!30}{\textbf{1280.0}} & \cellcolor{TealBlue!30}{\textbf{113{\sc m}}} & - & - & - & - & -\\
\texttt{letter} & \multicolumn{1}{r}{20000} & \multicolumn{1}{r}{224}  & \cellcolor{TealBlue!30}{0} & \cellcolor{TealBlue!30}{\textbf{251}} & \cellcolor{TealBlue!30}{\textbf{0.987}} & \cellcolor{TealBlue!30}{\textbf{2560.0}} & \cellcolor{TealBlue!30}{\textbf{15{\sc m}}} & \cellcolor{TealBlue!30}{0} & 352 & 0.982 & 3600.0 & 94{\sc m}\\
\texttt{lymph} & \multicolumn{1}{r}{148} & \multicolumn{1}{r}{41}  & \cellcolor{TealBlue!30}{1} & \cellcolor{TealBlue!30}{0} & \cellcolor{TealBlue!30}{1.000} & \cellcolor{TealBlue!30}{\textbf{0.0}} & \cellcolor{TealBlue!30}{\textbf{18{\sc k}}} & \cellcolor{TealBlue!30}{1} & \cellcolor{TealBlue!30}{0} & \cellcolor{TealBlue!30}{1.000} & 12.1 & 2615{\sc k}\\
\texttt{mushroom} & \multicolumn{1}{r}{8124} & \multicolumn{1}{r}{91}  & \cellcolor{TealBlue!30}{1} & \cellcolor{TealBlue!30}{0} & \cellcolor{TealBlue!30}{1.000} & \cellcolor{TealBlue!30}{\textbf{0.0}} & \cellcolor{TealBlue!30}{\textbf{278}} & \cellcolor{TealBlue!30}{1} & \cellcolor{TealBlue!30}{0} & \cellcolor{TealBlue!30}{1.000} & 36.4 & 1900{\sc k}\\
\texttt{pendigits} & \multicolumn{1}{r}{7494} & \multicolumn{1}{r}{216}  & \cellcolor{TealBlue!30}{\textbf{0}} & \cellcolor{TealBlue!30}{\textbf{2}} & \cellcolor{TealBlue!30}{\textbf{1.000}} & \cellcolor{TealBlue!30}{\textbf{1940.0}} & \cellcolor{TealBlue!30}{\textbf{25{\sc m}}} & - & - & - & - & -\\
\texttt{primary-tumor} & \multicolumn{1}{r}{336} & \multicolumn{1}{r}{16}  & \cellcolor{TealBlue!30}{1} & \cellcolor{TealBlue!30}{26} & \cellcolor{TealBlue!30}{0.923} & \cellcolor{TealBlue!30}{\textbf{6.6}} & 4418{\sc k} & \cellcolor{TealBlue!30}{1} & \cellcolor{TealBlue!30}{26} & \cellcolor{TealBlue!30}{0.923} & 20.3 & \cellcolor{TealBlue!30}{\textbf{2023{\sc k}}}\\
\texttt{segment} & \multicolumn{1}{r}{2310} & \multicolumn{1}{r}{234}  & \cellcolor{TealBlue!30}{1} & \cellcolor{TealBlue!30}{0} & \cellcolor{TealBlue!30}{1.000} & \cellcolor{TealBlue!30}{\textbf{0.0}} & \cellcolor{TealBlue!30}{\textbf{502}} & \cellcolor{TealBlue!30}{1} & \cellcolor{TealBlue!30}{0} & \cellcolor{TealBlue!30}{1.000} & 1.0 & 220{\sc k}\\
\texttt{soybean} & \multicolumn{1}{r}{630} & \multicolumn{1}{r}{34}  & \cellcolor{TealBlue!30}{1} & \cellcolor{TealBlue!30}{8} & \cellcolor{TealBlue!30}{0.987} & \cellcolor{TealBlue!30}{\textbf{44.8}} & 15{\sc m} & \cellcolor{TealBlue!30}{1} & \cellcolor{TealBlue!30}{8} & \cellcolor{TealBlue!30}{0.987} & 60.0 & \cellcolor{TealBlue!30}{\textbf{7368{\sc k}}}\\
\texttt{splice-1} & \multicolumn{1}{r}{3190} & \multicolumn{1}{r}{227}  & \cellcolor{TealBlue!30}{\textbf{0}} & \cellcolor{TealBlue!30}{\textbf{101}} & \cellcolor{TealBlue!30}{\textbf{0.968}} & \cellcolor{TealBlue!30}{\textbf{1410.0}} & \cellcolor{TealBlue!30}{\textbf{105{\sc m}}} & - & - & - & - & -\\
\texttt{taiwan\_binarised} & \multicolumn{1}{r}{30000} & \multicolumn{1}{r}{198}  & \cellcolor{TealBlue!30}{0} & \cellcolor{TealBlue!30}{\textbf{5207}} & \cellcolor{TealBlue!30}{\textbf{0.826}} & \cellcolor{TealBlue!30}{\textbf{2790.0}} & \cellcolor{TealBlue!30}{\textbf{16{\sc m}}} & \cellcolor{TealBlue!30}{0} & 5412 & 0.820 & 3600.0 & 54{\sc m}\\
\texttt{tic-tac-toe} & \multicolumn{1}{r}{958} & \multicolumn{1}{r}{18}  & \cellcolor{TealBlue!30}{1} & \cellcolor{TealBlue!30}{63} & \cellcolor{TealBlue!30}{0.934} & \cellcolor{TealBlue!30}{\textbf{8.4}} & 3635{\sc k} & \cellcolor{TealBlue!30}{1} & \cellcolor{TealBlue!30}{63} & \cellcolor{TealBlue!30}{0.934} & 11.7 & \cellcolor{TealBlue!30}{\textbf{1118{\sc k}}}\\
\texttt{vehicle} & \multicolumn{1}{r}{846} & \multicolumn{1}{r}{252}  & \cellcolor{TealBlue!30}{\textbf{0}} & \cellcolor{TealBlue!30}{\textbf{5}} & \cellcolor{TealBlue!30}{\textbf{0.994}} & \cellcolor{TealBlue!30}{\textbf{3370.0}} & \cellcolor{TealBlue!30}{\textbf{266{\sc m}}} & - & - & - & - & -\\
\texttt{vote} & \multicolumn{1}{r}{435} & \multicolumn{1}{r}{32}  & \cellcolor{TealBlue!30}{1} & \cellcolor{TealBlue!30}{1} & \cellcolor{TealBlue!30}{0.998} & \cellcolor{TealBlue!30}{\textbf{21.3}} & 8910{\sc k} & \cellcolor{TealBlue!30}{1} & \cellcolor{TealBlue!30}{1} & \cellcolor{TealBlue!30}{0.998} & 40.7 & \cellcolor{TealBlue!30}{\textbf{7873{\sc k}}}\\
\texttt{wine1-un} & \multicolumn{1}{r}{178} & \multicolumn{1}{r}{1276}  & \cellcolor{TealBlue!30}{\textbf{0}} & \cellcolor{TealBlue!30}{\textbf{34}} & \cellcolor{TealBlue!30}{\textbf{0.809}} & \cellcolor{TealBlue!30}{\textbf{498.0}} & \cellcolor{TealBlue!30}{\textbf{25{\sc m}}} & - & - & - & - & -\\
\texttt{wine2-un} & \multicolumn{1}{r}{178} & \multicolumn{1}{r}{1276}  & \cellcolor{TealBlue!30}{\textbf{0}} & \cellcolor{TealBlue!30}{\textbf{37}} & \cellcolor{TealBlue!30}{\textbf{0.792}} & \cellcolor{TealBlue!30}{\textbf{58.2}} & \cellcolor{TealBlue!30}{\textbf{2891{\sc k}}} & - & - & - & - & -\\
\texttt{wine3-un} & \multicolumn{1}{r}{178} & \multicolumn{1}{r}{1276}  & \cellcolor{TealBlue!30}{\textbf{0}} & \cellcolor{TealBlue!30}{\textbf{25}} & \cellcolor{TealBlue!30}{\textbf{0.860}} & \cellcolor{TealBlue!30}{\textbf{3500.0}} & \cellcolor{TealBlue!30}{\textbf{178{\sc m}}} & - & - & - & - & -\\
\texttt{yeast} & \multicolumn{1}{r}{1484} & \multicolumn{1}{r}{89}  & \cellcolor{TealBlue!30}{\textbf{1}} & \cellcolor{TealBlue!30}{\textbf{313}} & \cellcolor{TealBlue!30}{\textbf{0.789}} & \cellcolor{TealBlue!30}{\textbf{2580.0}} & \cellcolor{TealBlue!30}{\textbf{433{\sc m}}} & - & - & - & - & -\\
\texttt{zoo-1} & \multicolumn{1}{r}{101} & \multicolumn{1}{r}{20}  & \cellcolor{TealBlue!30}{1} & \cellcolor{TealBlue!30}{0} & \cellcolor{TealBlue!30}{1.000} & \cellcolor{TealBlue!30}{\textbf{0.0}} & \cellcolor{TealBlue!30}{\textbf{1}} & \cellcolor{TealBlue!30}{1} & \cellcolor{TealBlue!30}{0} & \cellcolor{TealBlue!30}{1.000} & 0.0 & 13\\
\bottomrule
\end{tabular}

\end{footnotesize}
\end{center}
\caption{\label{tab:thetable} Restarts (max depth=5)}
\end{table}

% \begin{table}[htbp]
% \begin{center}
% \begin{footnotesize}
% \tabcolsep=5pt
% \begin{tabular}{lccrrrrrrrrrrrr}
\toprule
& && \multicolumn{6}{c}{dt no restart} & \multicolumn{6}{c}{dt restarts (1.1)}\\
\cmidrule(rr){4-9}\cmidrule(rr){10-15}
&\multirow{1}{*}{$\#ex.$} & \multirow{1}{*}{\#feat.} &  \multicolumn{1}{c}{opt} & \multicolumn{1}{c}{error} & \multicolumn{1}{c}{acc.} & \multicolumn{1}{c}{size} & \multicolumn{1}{c}{time} & \multicolumn{1}{c}{choices} & \multicolumn{1}{c}{opt} & \multicolumn{1}{c}{error} & \multicolumn{1}{c}{acc.} & \multicolumn{1}{c}{size} & \multicolumn{1}{c}{time} & \multicolumn{1}{c}{choices} \\
\midrule

\texttt{anneal} & \multicolumn{1}{r}{812} & \multicolumn{1}{r}{88}  & \cellcolor{TealBlue!30}{1.0} & \cellcolor{TealBlue!30}{70.0} & \cellcolor{TealBlue!30}{0.914} & \cellcolor{TealBlue!30}{9.0} & \cellcolor{TealBlue!30}{\textbf{1034.7}} & \cellcolor{TealBlue!30}{\textbf{170{\sc m}}} & \cellcolor{TealBlue!30}{1.0} & \cellcolor{TealBlue!30}{70.0} & \cellcolor{TealBlue!30}{0.914} & \cellcolor{TealBlue!30}{9.0} & 1324.0 & 216{\sc m}\\
\texttt{audiology} & \multicolumn{1}{r}{216} & \multicolumn{1}{r}{145}  & \cellcolor{TealBlue!30}{0.0} & \cellcolor{TealBlue!30}{0.0} & \cellcolor{TealBlue!30}{1.000} & \cellcolor{TealBlue!30}{6.0} & \cellcolor{TealBlue!30}{\textbf{7.4}} & \cellcolor{TealBlue!30}{\textbf{1510{\sc k}}} & \cellcolor{TealBlue!30}{0.0} & \cellcolor{TealBlue!30}{0.0} & \cellcolor{TealBlue!30}{1.000} & \cellcolor{TealBlue!30}{6.0} & 139.5 & 29{\sc m}\\
\texttt{australian-credit} & \multicolumn{1}{r}{653} & \multicolumn{1}{r}{124}  & \cellcolor{TealBlue!30}{0.0} & \cellcolor{TealBlue!30}{40.0} & \cellcolor{TealBlue!30}{0.939} & \cellcolor{TealBlue!30}{8.0} & \cellcolor{TealBlue!30}{\textbf{60.2}} & \cellcolor{TealBlue!30}{\textbf{10{\sc m}}} & \cellcolor{TealBlue!30}{0.0} & \cellcolor{TealBlue!30}{40.0} & \cellcolor{TealBlue!30}{0.939} & \cellcolor{TealBlue!30}{8.0} & 683.3 & 117{\sc m}\\
\texttt{breast-cancer} & \multicolumn{1}{r}{683} & \multicolumn{1}{r}{89}  & \cellcolor{TealBlue!30}{1.0} & \cellcolor{TealBlue!30}{6.0} & \cellcolor{TealBlue!30}{0.991} & \cellcolor{TealBlue!30}{9.0} & \cellcolor{TealBlue!30}{\textbf{815.8}} & \cellcolor{TealBlue!30}{\textbf{158{\sc m}}} & \cellcolor{TealBlue!30}{1.0} & \cellcolor{TealBlue!30}{6.0} & \cellcolor{TealBlue!30}{0.991} & \cellcolor{TealBlue!30}{9.0} & 1022.2 & 202{\sc m}\\
\texttt{car} & \multicolumn{1}{r}{1728} & \multicolumn{1}{r}{21}  & \cellcolor{TealBlue!30}{1.0} & \cellcolor{TealBlue!30}{86.0} & \cellcolor{TealBlue!30}{0.950} & \cellcolor{TealBlue!30}{9.0} & \cellcolor{TealBlue!30}{\textbf{4.7}} & \cellcolor{TealBlue!30}{\textbf{1255{\sc k}}} & \cellcolor{TealBlue!30}{1.0} & \cellcolor{TealBlue!30}{86.0} & \cellcolor{TealBlue!30}{0.950} & \cellcolor{TealBlue!30}{9.0} & 8.8 & 2286{\sc k}\\
\texttt{forest-fires} & \multicolumn{1}{r}{517} & \multicolumn{1}{r}{989}  & \cellcolor{TealBlue!30}{0.0} & \cellcolor{TealBlue!30}{\textbf{156.6}} & \cellcolor{TealBlue!30}{\textbf{0.697}} & \cellcolor{TealBlue!30}{\textbf{9.0}} & 933.6 & 45{\sc m} & \cellcolor{TealBlue!30}{0.0} & 162.1 & 0.686 & 13.6 & \cellcolor{TealBlue!30}{\textbf{533.4}} & \cellcolor{TealBlue!30}{\textbf{26{\sc m}}}\\
\texttt{heart-cleveland} & \multicolumn{1}{r}{296} & \multicolumn{1}{r}{95}  & \cellcolor{TealBlue!30}{\textbf{0.1}} & \cellcolor{TealBlue!30}{7.0} & \cellcolor{TealBlue!30}{0.976} & \cellcolor{TealBlue!30}{9.0} & \cellcolor{TealBlue!30}{\textbf{279.9}} & \cellcolor{TealBlue!30}{\textbf{63{\sc m}}} & 0.0 & \cellcolor{TealBlue!30}{7.0} & \cellcolor{TealBlue!30}{0.976} & \cellcolor{TealBlue!30}{9.0} & 389.5 & 89{\sc m}\\
\texttt{hypothyroid} & \multicolumn{1}{r}{3247} & \multicolumn{1}{r}{83}  & \cellcolor{TealBlue!30}{0.0} & \cellcolor{TealBlue!30}{\textbf{44.0}} & \cellcolor{TealBlue!30}{\textbf{0.986}} & \cellcolor{TealBlue!30}{9.0} & 1519.0 & 103{\sc m} & \cellcolor{TealBlue!30}{0.0} & 45.0 & 0.986 & \cellcolor{TealBlue!30}{9.0} & \cellcolor{TealBlue!30}{\textbf{315.6}} & \cellcolor{TealBlue!30}{\textbf{16{\sc m}}}\\
\texttt{kr-vs-kp} & \multicolumn{1}{r}{3196} & \multicolumn{1}{r}{73}  & \cellcolor{TealBlue!30}{\textbf{0.2}} & \cellcolor{TealBlue!30}{81.0} & \cellcolor{TealBlue!30}{0.975} & \cellcolor{TealBlue!30}{7.0} & 402.1 & 31{\sc m} & 0.0 & \cellcolor{TealBlue!30}{81.0} & \cellcolor{TealBlue!30}{0.975} & \cellcolor{TealBlue!30}{7.0} & \cellcolor{TealBlue!30}{\textbf{243.4}} & \cellcolor{TealBlue!30}{\textbf{18{\sc m}}}\\
\texttt{lymph} & \multicolumn{1}{r}{148} & \multicolumn{1}{r}{68}  & \cellcolor{TealBlue!30}{1.0} & \cellcolor{TealBlue!30}{0.0} & \cellcolor{TealBlue!30}{1.000} & \cellcolor{TealBlue!30}{6.0} & \cellcolor{TealBlue!30}{\textbf{237.9}} & \cellcolor{TealBlue!30}{\textbf{79{\sc m}}} & \cellcolor{TealBlue!30}{1.0} & \cellcolor{TealBlue!30}{0.0} & \cellcolor{TealBlue!30}{1.000} & \cellcolor{TealBlue!30}{6.0} & 302.4 & 102{\sc m}\\
\texttt{mushroom} & \multicolumn{1}{r}{8124} & \multicolumn{1}{r}{111}  & \cellcolor{TealBlue!30}{0.0} & \cellcolor{TealBlue!30}{0.0} & \cellcolor{TealBlue!30}{1.000} & \cellcolor{TealBlue!30}{4.0} & \cellcolor{TealBlue!30}{\textbf{76.0}} & \cellcolor{TealBlue!30}{\textbf{2059{\sc k}}} & \cellcolor{TealBlue!30}{0.0} & \cellcolor{TealBlue!30}{0.0} & \cellcolor{TealBlue!30}{1.000} & \cellcolor{TealBlue!30}{4.0} & 1344.4 & 30{\sc m}\\
\texttt{primary-tumor} & \multicolumn{1}{r}{336} & \multicolumn{1}{r}{31}  & \cellcolor{TealBlue!30}{1.0} & \cellcolor{TealBlue!30}{26.0} & \cellcolor{TealBlue!30}{0.923} & \cellcolor{TealBlue!30}{9.0} & \cellcolor{TealBlue!30}{\textbf{9.7}} & \cellcolor{TealBlue!30}{\textbf{4936{\sc k}}} & \cellcolor{TealBlue!30}{1.0} & \cellcolor{TealBlue!30}{26.0} & \cellcolor{TealBlue!30}{0.923} & \cellcolor{TealBlue!30}{9.0} & 15.1 & 7745{\sc k}\\
\texttt{soybean} & \multicolumn{1}{r}{630} & \multicolumn{1}{r}{50}  & \cellcolor{TealBlue!30}{1.0} & \cellcolor{TealBlue!30}{8.0} & \cellcolor{TealBlue!30}{0.987} & \cellcolor{TealBlue!30}{8.0} & \cellcolor{TealBlue!30}{\textbf{65.0}} & \cellcolor{TealBlue!30}{\textbf{19{\sc m}}} & \cellcolor{TealBlue!30}{1.0} & \cellcolor{TealBlue!30}{8.0} & \cellcolor{TealBlue!30}{0.987} & \cellcolor{TealBlue!30}{8.0} & 87.8 & 26{\sc m}\\
\texttt{splice-1} & \multicolumn{1}{r}{3190} & \multicolumn{1}{r}{287}  & \cellcolor{TealBlue!30}{0.0} & 103.8 & 0.967 & 12.5 & 1437.2 & 93{\sc m} & \cellcolor{TealBlue!30}{0.0} & \cellcolor{TealBlue!30}{\textbf{103.4}} & \cellcolor{TealBlue!30}{\textbf{0.968}} & \cellcolor{TealBlue!30}{\textbf{11.8}} & \cellcolor{TealBlue!30}{\textbf{424.3}} & \cellcolor{TealBlue!30}{\textbf{21{\sc m}}}\\
\texttt{tic-tac-toe} & \multicolumn{1}{r}{958} & \multicolumn{1}{r}{27}  & \cellcolor{TealBlue!30}{1.0} & \cellcolor{TealBlue!30}{63.0} & \cellcolor{TealBlue!30}{0.934} & 8.5 & \cellcolor{TealBlue!30}{\textbf{13.2}} & \cellcolor{TealBlue!30}{\textbf{4990{\sc k}}} & \cellcolor{TealBlue!30}{1.0} & \cellcolor{TealBlue!30}{63.0} & \cellcolor{TealBlue!30}{0.934} & \cellcolor{TealBlue!30}{\textbf{8.2}} & 20.5 & 7985{\sc k}\\
\texttt{vote} & \multicolumn{1}{r}{435} & \multicolumn{1}{r}{48}  & \cellcolor{TealBlue!30}{1.0} & \cellcolor{TealBlue!30}{1.0} & \cellcolor{TealBlue!30}{0.998} & \cellcolor{TealBlue!30}{8.0} & \cellcolor{TealBlue!30}{\textbf{44.0}} & \cellcolor{TealBlue!30}{\textbf{15{\sc m}}} & \cellcolor{TealBlue!30}{1.0} & \cellcolor{TealBlue!30}{1.0} & \cellcolor{TealBlue!30}{0.998} & \cellcolor{TealBlue!30}{8.0} & 58.8 & 21{\sc m}\\
\texttt{wine1} & \multicolumn{1}{r}{178} & \multicolumn{1}{r}{1276}  & \cellcolor{TealBlue!30}{0.0} & \cellcolor{TealBlue!30}{\textbf{34.0}} & \cellcolor{TealBlue!30}{\textbf{0.809}} & 9.6 & 375.3 & 16{\sc m} & \cellcolor{TealBlue!30}{0.0} & 35.0 & 0.803 & \cellcolor{TealBlue!30}{\textbf{8.0}} & \cellcolor{TealBlue!30}{\textbf{262.9}} & \cellcolor{TealBlue!30}{\textbf{11{\sc m}}}\\
\texttt{wine2} & \multicolumn{1}{r}{178} & \multicolumn{1}{r}{1276}  & \cellcolor{TealBlue!30}{0.0} & \cellcolor{TealBlue!30}{37.0} & \cellcolor{TealBlue!30}{0.792} & \cellcolor{TealBlue!30}{9.0} & 56.5 & 2314{\sc k} & \cellcolor{TealBlue!30}{0.0} & \cellcolor{TealBlue!30}{37.0} & \cellcolor{TealBlue!30}{0.792} & \cellcolor{TealBlue!30}{9.0} & \cellcolor{TealBlue!30}{\textbf{2.7}} & \cellcolor{TealBlue!30}{\textbf{102{\sc k}}}\\
\texttt{wine3} & \multicolumn{1}{r}{178} & \multicolumn{1}{r}{1276}  & \cellcolor{TealBlue!30}{0.0} & \cellcolor{TealBlue!30}{\textbf{25.9}} & \cellcolor{TealBlue!30}{\textbf{0.854}} & 8.9 & 251.2 & 11{\sc m} & \cellcolor{TealBlue!30}{0.0} & 26.7 & 0.850 & \cellcolor{TealBlue!30}{\textbf{7.1}} & \cellcolor{TealBlue!30}{\textbf{176.2}} & \cellcolor{TealBlue!30}{\textbf{8446{\sc k}}}\\
\texttt{zoo-1} & \multicolumn{1}{r}{101} & \multicolumn{1}{r}{36}  & \cellcolor{TealBlue!30}{1.0} & \cellcolor{TealBlue!30}{0.0} & \cellcolor{TealBlue!30}{1.000} & \cellcolor{TealBlue!30}{1.0} & 0.0 & \cellcolor{TealBlue!30}{1} & \cellcolor{TealBlue!30}{1.0} & \cellcolor{TealBlue!30}{0.0} & \cellcolor{TealBlue!30}{1.000} & \cellcolor{TealBlue!30}{1.0} & \cellcolor{TealBlue!30}{\textbf{0.0}} & \cellcolor{TealBlue!30}{1}\\
\bottomrule
\end{tabular}

% \end{footnotesize}
% \end{center}
% \caption{\label{tab:thetable} Restarts (max depth=5)}
% \end{table}

\clearpage

\begin{table}[htbp]
\begin{center}
\begin{footnotesize}
\tabcolsep=5pt
\begin{tabular}{lccrrrrrrrrrrrr}
\toprule
& && \multicolumn{6}{c}{dt no restart} & \multicolumn{6}{c}{dt restarts (1.1)}\\
\cmidrule(rr){4-9}\cmidrule(rr){10-15}
&\multirow{1}{*}{$\#ex.$} & \multirow{1}{*}{\#feat.} &  \multicolumn{1}{c}{opt} & \multicolumn{1}{c}{error} & \multicolumn{1}{c}{acc.} & \multicolumn{1}{c}{size} & \multicolumn{1}{c}{time} & \multicolumn{1}{c}{choices} & \multicolumn{1}{c}{opt} & \multicolumn{1}{c}{error} & \multicolumn{1}{c}{acc.} & \multicolumn{1}{c}{size} & \multicolumn{1}{c}{time} & \multicolumn{1}{c}{choices} \\
\midrule

\texttt{anneal} & \multicolumn{1}{r}{812} & \multicolumn{1}{r}{88}  & \cellcolor{TealBlue!30}{0.0} & \cellcolor{TealBlue!30}{\textbf{64.0}} & \cellcolor{TealBlue!30}{\textbf{0.921}} & 12.9 & 1435.5 & 258{\sc m} & \cellcolor{TealBlue!30}{0.0} & 65.0 & 0.920 & \cellcolor{TealBlue!30}{\textbf{12.7}} & \cellcolor{TealBlue!30}{\textbf{826.5}} & \cellcolor{TealBlue!30}{\textbf{136{\sc m}}}\\
\texttt{audiology} & \multicolumn{1}{r}{216} & \multicolumn{1}{r}{145}  & \cellcolor{TealBlue!30}{0.0} & \cellcolor{TealBlue!30}{0.0} & \cellcolor{TealBlue!30}{1.000} & 9.2 & \cellcolor{TealBlue!30}{\textbf{33.6}} & \cellcolor{TealBlue!30}{\textbf{8518{\sc k}}} & \cellcolor{TealBlue!30}{0.0} & \cellcolor{TealBlue!30}{0.0} & \cellcolor{TealBlue!30}{1.000} & \cellcolor{TealBlue!30}{\textbf{9.0}} & 410.9 & 107{\sc m}\\
\texttt{australian-credit} & \multicolumn{1}{r}{653} & \multicolumn{1}{r}{124}  & \cellcolor{TealBlue!30}{0.0} & \cellcolor{TealBlue!30}{\textbf{0.5}} & \cellcolor{TealBlue!30}{\textbf{0.999}} & \cellcolor{TealBlue!30}{\textbf{12.4}} & 1457.1 & 307{\sc m} & \cellcolor{TealBlue!30}{0.0} & 3.1 & 0.995 & 16.5 & \cellcolor{TealBlue!30}{\textbf{711.8}} & \cellcolor{TealBlue!30}{\textbf{139{\sc m}}}\\
\texttt{breast-cancer} & \multicolumn{1}{r}{683} & \multicolumn{1}{r}{89}  & \cellcolor{TealBlue!30}{0.0} & \cellcolor{TealBlue!30}{0.0} & \cellcolor{TealBlue!30}{1.000} & \cellcolor{TealBlue!30}{\textbf{13.1}} & 1084.6 & 330{\sc m} & \cellcolor{TealBlue!30}{0.0} & \cellcolor{TealBlue!30}{0.0} & \cellcolor{TealBlue!30}{1.000} & 15.5 & \cellcolor{TealBlue!30}{\textbf{489.4}} & \cellcolor{TealBlue!30}{\textbf{161{\sc m}}}\\
\texttt{car} & \multicolumn{1}{r}{1728} & \multicolumn{1}{r}{21}  & \cellcolor{TealBlue!30}{0.0} & \cellcolor{TealBlue!30}{\textbf{0.0}} & \cellcolor{TealBlue!30}{\textbf{1.000}} & \cellcolor{TealBlue!30}{\textbf{11.9}} & 934.2 & 542{\sc m} & \cellcolor{TealBlue!30}{0.0} & 2.5 & 0.999 & 13.9 & \cellcolor{TealBlue!30}{\textbf{343.9}} & \cellcolor{TealBlue!30}{\textbf{212{\sc m}}}\\
\texttt{forest-fires} & \multicolumn{1}{r}{517} & \multicolumn{1}{r}{989}  & \cellcolor{TealBlue!30}{0.0} & 145.7 & 0.718 & 26.3 & 742.3 & 38{\sc m} & \cellcolor{TealBlue!30}{0.0} & \cellcolor{TealBlue!30}{\textbf{127.7}} & \cellcolor{TealBlue!30}{\textbf{0.753}} & \cellcolor{TealBlue!30}{\textbf{25.7}} & \cellcolor{TealBlue!30}{\textbf{596.9}} & \cellcolor{TealBlue!30}{\textbf{30{\sc m}}}\\
\texttt{heart-cleveland} & \multicolumn{1}{r}{296} & \multicolumn{1}{r}{95}  & \cellcolor{TealBlue!30}{0.0} & \cellcolor{TealBlue!30}{0.0} & \cellcolor{TealBlue!30}{1.000} & \cellcolor{TealBlue!30}{\textbf{18.3}} & 1489.8 & 416{\sc m} & \cellcolor{TealBlue!30}{0.0} & \cellcolor{TealBlue!30}{0.0} & \cellcolor{TealBlue!30}{1.000} & 25.9 & \cellcolor{TealBlue!30}{\textbf{622.2}} & \cellcolor{TealBlue!30}{\textbf{211{\sc m}}}\\
\texttt{hypothyroid} & \multicolumn{1}{r}{3247} & \multicolumn{1}{r}{83}  & \cellcolor{TealBlue!30}{0.0} & 49.0 & 0.985 & \cellcolor{TealBlue!30}{\textbf{36.5}} & \cellcolor{TealBlue!30}{\textbf{91.5}} & \cellcolor{TealBlue!30}{\textbf{22{\sc m}}} & \cellcolor{TealBlue!30}{0.0} & \cellcolor{TealBlue!30}{\textbf{38.1}} & \cellcolor{TealBlue!30}{\textbf{0.988}} & 41.8 & 259.4 & 23{\sc m}\\
\texttt{kr-vs-kp} & \multicolumn{1}{r}{3196} & \multicolumn{1}{r}{73}  & \cellcolor{TealBlue!30}{0.0} & \cellcolor{TealBlue!30}{\textbf{29.7}} & \cellcolor{TealBlue!30}{\textbf{0.991}} & 17.8 & \cellcolor{TealBlue!30}{\textbf{237.2}} & \cellcolor{TealBlue!30}{\textbf{47{\sc m}}} & \cellcolor{TealBlue!30}{0.0} & 45.9 & 0.986 & \cellcolor{TealBlue!30}{\textbf{17.2}} & 1017.8 & 218{\sc m}\\
\texttt{lymph} & \multicolumn{1}{r}{148} & \multicolumn{1}{r}{68}  & \cellcolor{TealBlue!30}{0.0} & \cellcolor{TealBlue!30}{0.0} & \cellcolor{TealBlue!30}{1.000} & 11.6 & 1526.6 & 640{\sc m} & \cellcolor{TealBlue!30}{0.0} & \cellcolor{TealBlue!30}{0.0} & \cellcolor{TealBlue!30}{1.000} & \cellcolor{TealBlue!30}{\textbf{11.3}} & \cellcolor{TealBlue!30}{\textbf{517.0}} & \cellcolor{TealBlue!30}{\textbf{225{\sc m}}}\\
\texttt{mushroom} & \multicolumn{1}{r}{8124} & \multicolumn{1}{r}{111}  & \cellcolor{TealBlue!30}{0.0} & \cellcolor{TealBlue!30}{0.0} & \cellcolor{TealBlue!30}{1.000} & 8.1 & \cellcolor{TealBlue!30}{\textbf{293.2}} & \cellcolor{TealBlue!30}{\textbf{9653{\sc k}}} & \cellcolor{TealBlue!30}{0.0} & \cellcolor{TealBlue!30}{0.0} & \cellcolor{TealBlue!30}{1.000} & \cellcolor{TealBlue!30}{\textbf{8.0}} & 394.5 & 25{\sc m}\\
\texttt{primary-tumor} & \multicolumn{1}{r}{336} & \multicolumn{1}{r}{31}  & \cellcolor{TealBlue!30}{0.0} & \cellcolor{TealBlue!30}{15.0} & \cellcolor{TealBlue!30}{0.955} & 17.3 & 929.2 & 599{\sc m} & \cellcolor{TealBlue!30}{0.0} & \cellcolor{TealBlue!30}{15.0} & \cellcolor{TealBlue!30}{0.955} & \cellcolor{TealBlue!30}{\textbf{16.6}} & \cellcolor{TealBlue!30}{\textbf{738.1}} & \cellcolor{TealBlue!30}{\textbf{481{\sc m}}}\\
\texttt{soybean} & \multicolumn{1}{r}{630} & \multicolumn{1}{r}{50}  & \cellcolor{TealBlue!30}{0.0} & \cellcolor{TealBlue!30}{2.0} & \cellcolor{TealBlue!30}{0.997} & \cellcolor{TealBlue!30}{\textbf{11.1}} & 879.0 & \cellcolor{TealBlue!30}{\textbf{329{\sc m}}} & \cellcolor{TealBlue!30}{0.0} & \cellcolor{TealBlue!30}{2.0} & \cellcolor{TealBlue!30}{0.997} & 13.6 & \cellcolor{TealBlue!30}{\textbf{848.1}} & 329{\sc m}\\
\texttt{splice-1} & \multicolumn{1}{r}{3190} & \multicolumn{1}{r}{287}  & \cellcolor{TealBlue!30}{0.0} & 85.6 & 0.973 & \cellcolor{TealBlue!30}{\textbf{42.6}} & 1099.9 & 110{\sc m} & \cellcolor{TealBlue!30}{0.0} & \cellcolor{TealBlue!30}{\textbf{45.6}} & \cellcolor{TealBlue!30}{\textbf{0.986}} & 43.7 & \cellcolor{TealBlue!30}{\textbf{516.2}} & \cellcolor{TealBlue!30}{\textbf{48{\sc m}}}\\
\texttt{tic-tac-toe} & \multicolumn{1}{r}{958} & \multicolumn{1}{r}{27}  & \cellcolor{TealBlue!30}{0.0} & \cellcolor{TealBlue!30}{0.0} & \cellcolor{TealBlue!30}{1.000} & \cellcolor{TealBlue!30}{\textbf{19.1}} & \cellcolor{TealBlue!30}{\textbf{306.1}} & \cellcolor{TealBlue!30}{\textbf{219{\sc m}}} & \cellcolor{TealBlue!30}{0.0} & \cellcolor{TealBlue!30}{0.0} & \cellcolor{TealBlue!30}{1.000} & 25.4 & 329.7 & 253{\sc m}\\
\texttt{vote} & \multicolumn{1}{r}{435} & \multicolumn{1}{r}{48}  & \cellcolor{TealBlue!30}{0.0} & \cellcolor{TealBlue!30}{0.0} & \cellcolor{TealBlue!30}{1.000} & \cellcolor{TealBlue!30}{\textbf{12.9}} & \cellcolor{TealBlue!30}{\textbf{330.8}} & \cellcolor{TealBlue!30}{\textbf{202{\sc m}}} & \cellcolor{TealBlue!30}{0.0} & \cellcolor{TealBlue!30}{0.0} & \cellcolor{TealBlue!30}{1.000} & 14.2 & 574.8 & 376{\sc m}\\
\texttt{wine1} & \multicolumn{1}{r}{178} & \multicolumn{1}{r}{1276}  & \cellcolor{TealBlue!30}{0.0} & \cellcolor{TealBlue!30}{26.0} & \cellcolor{TealBlue!30}{0.854} & \cellcolor{TealBlue!30}{12.0} & \cellcolor{TealBlue!30}{\textbf{16.7}} & \cellcolor{TealBlue!30}{\textbf{695{\sc k}}} & \cellcolor{TealBlue!30}{0.0} & \cellcolor{TealBlue!30}{26.0} & \cellcolor{TealBlue!30}{0.854} & \cellcolor{TealBlue!30}{12.0} & 128.7 & 5508{\sc k}\\
\texttt{wine2} & \multicolumn{1}{r}{178} & \multicolumn{1}{r}{1276}  & \cellcolor{TealBlue!30}{0.0} & 27.3 & 0.847 & 13.6 & \cellcolor{TealBlue!30}{\textbf{259.2}} & \cellcolor{TealBlue!30}{\textbf{11{\sc m}}} & \cellcolor{TealBlue!30}{0.0} & \cellcolor{TealBlue!30}{\textbf{26.7}} & \cellcolor{TealBlue!30}{\textbf{0.850}} & \cellcolor{TealBlue!30}{\textbf{13.3}} & 347.5 & 15{\sc m}\\
\texttt{wine3} & \multicolumn{1}{r}{178} & \multicolumn{1}{r}{1276}  & \cellcolor{TealBlue!30}{0.0} & 20.8 & 0.883 & \cellcolor{TealBlue!30}{11.2} & \cellcolor{TealBlue!30}{\textbf{72.4}} & \cellcolor{TealBlue!30}{\textbf{3180{\sc k}}} & \cellcolor{TealBlue!30}{0.0} & \cellcolor{TealBlue!30}{\textbf{19.3}} & \cellcolor{TealBlue!30}{\textbf{0.892}} & \cellcolor{TealBlue!30}{11.2} & 631.9 & 28{\sc m}\\
\texttt{zoo-1} & \multicolumn{1}{r}{101} & \multicolumn{1}{r}{36}  & \cellcolor{TealBlue!30}{1.0} & \cellcolor{TealBlue!30}{0.0} & \cellcolor{TealBlue!30}{1.000} & \cellcolor{TealBlue!30}{1.0} & 0.0 & \cellcolor{TealBlue!30}{1} & \cellcolor{TealBlue!30}{1.0} & \cellcolor{TealBlue!30}{0.0} & \cellcolor{TealBlue!30}{1.000} & \cellcolor{TealBlue!30}{1.0} & \cellcolor{TealBlue!30}{\textbf{0.0}} & \cellcolor{TealBlue!30}{1}\\
\bottomrule
\end{tabular}

\end{footnotesize}
\end{center}
\caption{\label{tab:thetable} Restarts (max depth=8)}
\end{table}

\begin{table}[htbp]
\begin{center}
\begin{footnotesize}
\tabcolsep=5pt
\begin{tabular}{lccrrrrrrrr}
\toprule
& && \multicolumn{4}{c}{\dleight} & \multicolumn{4}{c}{\budalg}\\
\cmidrule(rr){4-7}\cmidrule(rr){8-11}
&\multirow{1}{*}{$\#ex.$} & \multirow{1}{*}{\#feat.} &  \multicolumn{1}{c}{opt} & \multicolumn{1}{c}{error} & \multicolumn{1}{c}{acc.} & \multicolumn{1}{c}{time} & \multicolumn{1}{c}{opt} & \multicolumn{1}{c}{error} & \multicolumn{1}{c}{acc.} & \multicolumn{1}{c}{time} \\
\midrule

\texttt{anneal} & \multicolumn{1}{r}{812} & \multicolumn{1}{r}{47}  & - & - & - & - & \cellcolor{TealBlue!30}{\textbf{0}} & \cellcolor{TealBlue!30}{\textbf{53}} & \cellcolor{TealBlue!30}{\textbf{0.935}} & \cellcolor{TealBlue!30}{\textbf{310.0}}\\
\texttt{audiology} & \multicolumn{1}{r}{216} & \multicolumn{1}{r}{79}  & - & - & - & - & \cellcolor{TealBlue!30}{\textbf{1}} & \cellcolor{TealBlue!30}{\textbf{0}} & \cellcolor{TealBlue!30}{\textbf{1.000}} & \cellcolor{TealBlue!30}{\textbf{0.0}}\\
\texttt{australian-credit} & \multicolumn{1}{r}{653} & \multicolumn{1}{r}{73}  & - & - & - & - & \cellcolor{TealBlue!30}{\textbf{1}} & \cellcolor{TealBlue!30}{\textbf{0}} & \cellcolor{TealBlue!30}{\textbf{1.000}} & \cellcolor{TealBlue!30}{\textbf{0.0}}\\
\texttt{breast-cancer-un} & \multicolumn{1}{r}{683} & \multicolumn{1}{r}{89}  & - & - & - & - & \cellcolor{TealBlue!30}{\textbf{1}} & \cellcolor{TealBlue!30}{\textbf{0}} & \cellcolor{TealBlue!30}{\textbf{1.000}} & \cellcolor{TealBlue!30}{\textbf{0.0}}\\
\texttt{breast-wisconsin} & \multicolumn{1}{r}{683} & \multicolumn{1}{r}{120}  & - & - & - & - & \cellcolor{TealBlue!30}{\textbf{1}} & \cellcolor{TealBlue!30}{\textbf{0}} & \cellcolor{TealBlue!30}{\textbf{1.000}} & \cellcolor{TealBlue!30}{\textbf{0.0}}\\
\texttt{car-un} & \multicolumn{1}{r}{1728} & \multicolumn{1}{r}{21}  & - & - & - & - & \cellcolor{TealBlue!30}{\textbf{1}} & \cellcolor{TealBlue!30}{\textbf{0}} & \cellcolor{TealBlue!30}{\textbf{1.000}} & \cellcolor{TealBlue!30}{\textbf{0.2}}\\
\texttt{diabetes} & \multicolumn{1}{r}{768} & \multicolumn{1}{r}{112}  & - & - & - & - & \cellcolor{TealBlue!30}{\textbf{1}} & \cellcolor{TealBlue!30}{\textbf{0}} & \cellcolor{TealBlue!30}{\textbf{1.000}} & \cellcolor{TealBlue!30}{\textbf{0.5}}\\
\texttt{forest-fires-un} & \multicolumn{1}{r}{517} & \multicolumn{1}{r}{989}  & - & - & - & - & \cellcolor{TealBlue!30}{\textbf{0}} & \cellcolor{TealBlue!30}{\textbf{118}} & \cellcolor{TealBlue!30}{\textbf{0.772}} & \cellcolor{TealBlue!30}{\textbf{830.0}}\\
\texttt{german-credit} & \multicolumn{1}{r}{1000} & \multicolumn{1}{r}{110}  & - & - & - & - & \cellcolor{TealBlue!30}{\textbf{1}} & \cellcolor{TealBlue!30}{\textbf{0}} & \cellcolor{TealBlue!30}{\textbf{1.000}} & \cellcolor{TealBlue!30}{\textbf{67.0}}\\
\texttt{heart-cleveland} & \multicolumn{1}{r}{296} & \multicolumn{1}{r}{50}  & - & - & - & - & \cellcolor{TealBlue!30}{\textbf{1}} & \cellcolor{TealBlue!30}{\textbf{0}} & \cellcolor{TealBlue!30}{\textbf{1.000}} & \cellcolor{TealBlue!30}{\textbf{0.0}}\\
\texttt{hepatitis} & \multicolumn{1}{r}{137} & \multicolumn{1}{r}{68}  & - & - & - & - & \cellcolor{TealBlue!30}{\textbf{1}} & \cellcolor{TealBlue!30}{\textbf{0}} & \cellcolor{TealBlue!30}{\textbf{1.000}} & \cellcolor{TealBlue!30}{\textbf{0.0}}\\
\texttt{hypothyroid} & \multicolumn{1}{r}{3247} & \multicolumn{1}{r}{43}  & - & - & - & - & \cellcolor{TealBlue!30}{\textbf{0}} & \cellcolor{TealBlue!30}{\textbf{31}} & \cellcolor{TealBlue!30}{\textbf{0.990}} & \cellcolor{TealBlue!30}{\textbf{1.4}}\\
\texttt{ionosphere} & \multicolumn{1}{r}{351} & \multicolumn{1}{r}{444}  & - & - & - & - & \cellcolor{TealBlue!30}{\textbf{1}} & \cellcolor{TealBlue!30}{\textbf{0}} & \cellcolor{TealBlue!30}{\textbf{1.000}} & \cellcolor{TealBlue!30}{\textbf{0.0}}\\
\texttt{kr-vs-kp} & \multicolumn{1}{r}{3196} & \multicolumn{1}{r}{37}  & - & - & - & - & \cellcolor{TealBlue!30}{\textbf{0}} & \cellcolor{TealBlue!30}{\textbf{1}} & \cellcolor{TealBlue!30}{\textbf{1.000}} & \cellcolor{TealBlue!30}{\textbf{224.0}}\\
\texttt{letter} & \multicolumn{1}{r}{20000} & \multicolumn{1}{r}{224}  & - & - & - & - & \cellcolor{TealBlue!30}{\textbf{1}} & \cellcolor{TealBlue!30}{\textbf{0}} & \cellcolor{TealBlue!30}{\textbf{1.000}} & \cellcolor{TealBlue!30}{\textbf{58.4}}\\
\texttt{lymph} & \multicolumn{1}{r}{148} & \multicolumn{1}{r}{41}  & - & - & - & - & \cellcolor{TealBlue!30}{\textbf{1}} & \cellcolor{TealBlue!30}{\textbf{0}} & \cellcolor{TealBlue!30}{\textbf{1.000}} & \cellcolor{TealBlue!30}{\textbf{0.0}}\\
\texttt{mushroom} & \multicolumn{1}{r}{8124} & \multicolumn{1}{r}{91}  & - & - & - & - & \cellcolor{TealBlue!30}{\textbf{1}} & \cellcolor{TealBlue!30}{\textbf{0}} & \cellcolor{TealBlue!30}{\textbf{1.000}} & \cellcolor{TealBlue!30}{\textbf{0.0}}\\
\texttt{pendigits} & \multicolumn{1}{r}{7494} & \multicolumn{1}{r}{216}  & - & - & - & - & \cellcolor{TealBlue!30}{\textbf{1}} & \cellcolor{TealBlue!30}{\textbf{0}} & \cellcolor{TealBlue!30}{\textbf{1.000}} & \cellcolor{TealBlue!30}{\textbf{0.0}}\\
\texttt{primary-tumor} & \multicolumn{1}{r}{336} & \multicolumn{1}{r}{16}  & - & - & - & - & \cellcolor{TealBlue!30}{\textbf{0}} & \cellcolor{TealBlue!30}{\textbf{15}} & \cellcolor{TealBlue!30}{\textbf{0.955}} & \cellcolor{TealBlue!30}{\textbf{144.0}}\\
\texttt{segment} & \multicolumn{1}{r}{2310} & \multicolumn{1}{r}{234}  & - & - & - & - & \cellcolor{TealBlue!30}{\textbf{1}} & \cellcolor{TealBlue!30}{\textbf{0}} & \cellcolor{TealBlue!30}{\textbf{1.000}} & \cellcolor{TealBlue!30}{\textbf{0.0}}\\
\texttt{soybean} & \multicolumn{1}{r}{630} & \multicolumn{1}{r}{34}  & - & - & - & - & \cellcolor{TealBlue!30}{\textbf{0}} & \cellcolor{TealBlue!30}{\textbf{2}} & \cellcolor{TealBlue!30}{\textbf{0.997}} & \cellcolor{TealBlue!30}{\textbf{149.0}}\\
\texttt{splice-1} & \multicolumn{1}{r}{3190} & \multicolumn{1}{r}{227}  & - & - & - & - & \cellcolor{TealBlue!30}{\textbf{0}} & \cellcolor{TealBlue!30}{\textbf{7}} & \cellcolor{TealBlue!30}{\textbf{0.998}} & \cellcolor{TealBlue!30}{\textbf{56.1}}\\
\texttt{taiwan\_binarised} & \multicolumn{1}{r}{30000} & \multicolumn{1}{r}{198}  & - & - & - & - & \cellcolor{TealBlue!30}{\textbf{0}} & \cellcolor{TealBlue!30}{\textbf{4564}} & \cellcolor{TealBlue!30}{\textbf{0.848}} & \cellcolor{TealBlue!30}{\textbf{200.0}}\\
\texttt{tic-tac-toe} & \multicolumn{1}{r}{958} & \multicolumn{1}{r}{18}  & - & - & - & - & \cellcolor{TealBlue!30}{\textbf{1}} & \cellcolor{TealBlue!30}{\textbf{0}} & \cellcolor{TealBlue!30}{\textbf{1.000}} & \cellcolor{TealBlue!30}{\textbf{0.0}}\\
\texttt{vehicle} & \multicolumn{1}{r}{846} & \multicolumn{1}{r}{252}  & - & - & - & - & \cellcolor{TealBlue!30}{\textbf{1}} & \cellcolor{TealBlue!30}{\textbf{0}} & \cellcolor{TealBlue!30}{\textbf{1.000}} & \cellcolor{TealBlue!30}{\textbf{0.0}}\\
\texttt{vote} & \multicolumn{1}{r}{435} & \multicolumn{1}{r}{32}  & - & - & - & - & \cellcolor{TealBlue!30}{\textbf{1}} & \cellcolor{TealBlue!30}{\textbf{0}} & \cellcolor{TealBlue!30}{\textbf{1.000}} & \cellcolor{TealBlue!30}{\textbf{0.0}}\\
\texttt{wine1-un} & \multicolumn{1}{r}{178} & \multicolumn{1}{r}{1276}  & - & - & - & - & \cellcolor{TealBlue!30}{\textbf{0}} & \cellcolor{TealBlue!30}{\textbf{22}} & \cellcolor{TealBlue!30}{\textbf{0.876}} & \cellcolor{TealBlue!30}{\textbf{515.0}}\\
\texttt{wine2-un} & \multicolumn{1}{r}{178} & \multicolumn{1}{r}{1276}  & - & - & - & - & \cellcolor{TealBlue!30}{\textbf{0}} & \cellcolor{TealBlue!30}{\textbf{24}} & \cellcolor{TealBlue!30}{\textbf{0.865}} & \cellcolor{TealBlue!30}{\textbf{340.0}}\\
\texttt{wine3-un} & \multicolumn{1}{r}{178} & \multicolumn{1}{r}{1276}  & - & - & - & - & \cellcolor{TealBlue!30}{\textbf{0}} & \cellcolor{TealBlue!30}{\textbf{16}} & \cellcolor{TealBlue!30}{\textbf{0.910}} & \cellcolor{TealBlue!30}{\textbf{260.0}}\\
\texttt{yeast} & \multicolumn{1}{r}{1484} & \multicolumn{1}{r}{89}  & - & - & - & - & \cellcolor{TealBlue!30}{\textbf{0}} & \cellcolor{TealBlue!30}{\textbf{104}} & \cellcolor{TealBlue!30}{\textbf{0.930}} & \cellcolor{TealBlue!30}{\textbf{72.3}}\\
\texttt{zoo-1} & \multicolumn{1}{r}{101} & \multicolumn{1}{r}{20}  & - & - & - & - & \cellcolor{TealBlue!30}{\textbf{1}} & \cellcolor{TealBlue!30}{\textbf{0}} & \cellcolor{TealBlue!30}{\textbf{1.000}} & \cellcolor{TealBlue!30}{\textbf{0.0}}\\
\bottomrule
\end{tabular}

\end{footnotesize}
\end{center}
\caption{\label{tab:thetable} Restarts (max depth=10)}
\end{table}



\end{document}

